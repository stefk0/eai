\subsubsection*{Премахване на преименуващи правила}
\index{преименуващи правила}
Преименуващите правила са от вида $A \to B$.
Нека е дадена граматика $G = \CFG$, в която има преименуващи правила.
Ще построим еквивалентна граматика $G'$ без преименуващи правила.
В началото нека в $R'$ да добавим всички правила от $R$, които не са преименуващи.
След това, за всякa променлива $A$, за която $A \to^\star_G B$,
ако $B \to \alpha$ е правило в $R$, което не е преименуващо,
то добавяме към $R'$ правилото $A \to \alpha$.

\begin{example}
  Нека е дадена граматиката $G$ с правила  
  \begin{align*}
    & S \to B\ |\ CC\ |\ b\\
    & A \to B\ |\ S\\
    & B \to C\ |\ BC\\
    & C \to AB\ |\ a\ |\ b.
  \end{align*}
  % \[A\rightarrow B|S,B\rightarrow C|BC,C\rightarrow AB|a|b,S\rightarrow B|CC|b.\]
  Първо добавяме към $R'$ правилата $B \to BC, C \to AB\ |\ a\ |\ b, S \to CC\ |\ b$.
  \begin{itemize}
  \item 
    Лесно се съобразява, че $A \to^\star_G B,S,C$.
    Добавяме правилата 
    \[A \to BC\ |\ AB\ |\ a\ |\ b\ |\ CC.\]
  \item
    Имаме $B \to^\star_G C$.
    Добавяме правилата $B \to AB\ |\ a\ |\ b$.
  \item
    Имаме $S \to^\star_G B,C$.
    Добавяме правилата $S \to BC\ |\ AB\ |\ a\ |\ b$.
  \end{itemize}
  Накрая получаваме, че граматиката $G'$ има правила
  \begin{align*}
    & S \to BC\ |\ AB\ |\ CC\ |\ a\ |\ b\\
    & A \to BC\ |\ AB\ |\ a\ |\ b\ |\ CC\\
    & B \to AB\ |\ a\ |\ b\ |\ BC\\
    & C \to AB\ |\ a\ |\ b.
  \end{align*}
\end{example}

\begin{problem}
  Премахнете преименуващите правила от граматиката $G$, като запазите езика, ако $G$ има следните правила:
    \begin{align*}
      & S \to C\ |\ CC\ |\ b\\
      & A \to B\\
      & B \to S\ |\ C\ |\ BC\\
      & C \to a\ |\ AB;
    \end{align*}
\end{problem}

%%% Local Variables:
%%% mode: latex
%%% TeX-master: "../eai"
%%% End:
