\subsubsection*{Премахване на $\varepsilon$-правила}
\index{$\varepsilon$-правила}
За да премахнем правилата от вида $A \to \varepsilon$, следваме процедурата:
\marginpar{Броят на правилата може да се увеличи експоненциално, защото в най-лошия случай извеждаме всички подмножества на дадено множество от променливи}
\begin{enumerate}[1)]
\item 
  Намираме множеството
  \[E = \{A \in V \mid A \to^\star \varepsilon\}\]
  като започваме да намираме множествата $E_n = \{A \in V \mid A \stackrel{n}{\to}_G \varepsilon \}$, за $n \geq 1$.
  Спираме на първото $n$, за което $E_n = E_{n+1}$. Тогава $E = E_n$.
  \begin{itemize}
  \item
    Започваме с $E_1 = \{A \in V \mid A \to_G \varepsilon\}$.
  \item
    Нека имаме $E_n$. Дефинираме $E_{n+1}$ по следния начин.
    Първо $E_{n+1} = E_n$. След това за всяко правило от вида $B \to_G X_1\cdots X_k$, 
    ако всяко $X_i \in E_n$, то добавяме $B$ към $E_{n+1}$.
  \end{itemize}
\item
  Строим множеството от правила $R'$, в което няма правила $\varepsilon$-правила по следния начин.
  За всяко правило $A \to X_1\cdots X_k$,
  добавяме към $R'$ всички правила от вида $A \to Y_1\cdots Y_k$, където:
  \begin{itemize}[-]
  \item 
    ако $X_i \not\in E$, то $Y_i = X_i$;
  \item
    ако $X_i \in E$, то $Y_i = X_i$ или $Y_i = \varepsilon$;
  \item
    не всички $Y_i$ са $\varepsilon$.
  \end{itemize}
  Лесно се съобразява, че $\L(G') = \L(G) \setminus \{\varepsilon\}$.
\end{enumerate}

\begin{example}
  Нека е дадена граматиката $G$ с правила
  \begin{align*}
    & S \to D\\
    & D \to AD\ |\ b\\
    & A \to AB\ |\ BC\ |\ a\\
    & B \to AA\ |\ UC\\
    & C \to \varepsilon\ |\ CA\ |\ a\\
    & U \to \varepsilon\ |\ aUb.
  \end{align*}
  Тогава
  \[E = \{X \in V \mid X \rightarrow^\star_G \varepsilon\} = \{A,B,C,U\}.\]
  Това означава, че $\varepsilon \not\in \L(G)$.
  Граматиката $G'$ без $\varepsilon$-правила, за която $\L(G') = \L(G)$ има следните правила
  \begin{align*}
    & S \to D & \comment\text{нищо не добавяме}\\
    & D\to AD\ |\ D\ |\ b & \comment\text{добавяме $D\to D$, защото $A\in E$}\\
    & A \to A\ |\ B\ |\ C\ |\ AB\ |\ BC\ |\ a \\
    & B\to A\ |\ E\ |\ C\ |\ AA\ |\ UC\\
    & C \to C\ |\ A\ |\ CA\ |\ a\\
    & U \to aUb\ |\ ab.
  \end{align*}
\end{example}

%%% Local Variables:
%%% mode: latex
%%% TeX-master: "../eai"
%%% End:
