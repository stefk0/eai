\subsubsection*{Премахване на безполезните променливи}

Нека е дадена безконтекстната граматика $G = \CFG$.
\marginpar{\cite[стр. 88]{hopcroft1}}
Една променлива $A$ се нарича {\bf полезна}, ако съществува извод от следния вид:
\[S \to^\star_G \alpha A \beta \to^\star_G \gamma,\]
където $\gamma \in \Sigma^\star$, а $\alpha,\beta \in (V \cup \Sigma)^\star$.
Това означава, че една променлива е полезна, ако участва в извода на някоя дума в езика на граматиката.
Една променлива се нарича {\bf безполезна}, ако не е полезна.
Целта ни е да получим еквивалентна граматика $G'$ без безполезни променливи.
Ще решим задачата като разгледаме две леми.

\begin{lemma}
  \label{lem:useless1}
  Нека е дадена безконтекстната граматика $G$ и $\L(G) \neq \emptyset$.
  Съществува алгоритъм, който намира граматика $G'$, за която
  $\L(G) = \L(G')$ и със свойството, че  за всяка променлива $A' \in V'$, съществува дума $\alpha \in \Sigma^\star$,
  за която $A' \to^\star_{G'} \alpha$.
\end{lemma}
\begin{hint}
  Ще построим множествата
  \[V_n = \{A \in V \mid A \stackrel{\leq n}{\to}_G \alpha\ \&\ \alpha\in\Sigma^\star\}.\]

  \begin{itemize}
  \item
    Ясно е, че $V_0 = \emptyset$;
  \item
    Нека сме намерили $V_n$. Тогава
    \[V_{n+1} = V_n \cup \{A\in V \mid A \to_G \alpha\ \&\ \alpha \in (\Sigma \cup V_n)^\star \}.\]
  \item
    Когато намерим първото $n$, за което $V_n = V_{n+1}$,
    то знаем, че
    \[V_n = \{A \in V \mid A \to^\star_G \alpha\ \&\ \alpha \in \Sigma^\star \}.\]
    Дефинираме $G'$ като $V' = V_n$ и правилата на $G'$ са само тези правила на $G$, в които участват променливи от $V_n$ и букви от $\Sigma$.
  \end{itemize}
\end{hint}

\begin{cor}
  Съществува алгоритъм, който определя дали за всяка безконтекстна граматика $G$ дали $\L(G) = \emptyset$.
\end{cor}
\begin{proof}
  Прилагаме алгоритъма от \Lem{useless1}.
  Тогава $\L(G) = \emptyset$ точно тогава, когато $S \not\in V'$.  
\end{proof}

\begin{lemma}
  \label{lem:useless2}
  Съществува алгоритъм, който по дадена безконтекстна граматика $G = \CFG$, намира $G' = \pair{V',\Sigma',S,R'}$, $\L(G') = \L(G)$,
  със свойството, че за всяко $X \in V' \cup \Sigma'$ съществуват $\alpha, \beta \in (V'\cup\Sigma')^\star$,
  за които $S \to^\star \alpha X \beta$,
  т.е. всяка променлива или буква в $G'$ е достижима от началната променлива $S$.
\end{lemma}
\begin{hint}
  Строим множествата
  \begin{align*}
    & V_n = \{A \in V \mid S \stackrel{\leq n}{\to}_G \alpha A \beta\text{, за някои }\alpha,\beta \in (\Sigma \cup V)^\star\}\\
    & \Sigma_n = \{a \in \Sigma \mid S \stackrel{\leq n}{\to}_G \alpha a \beta\text{, за някои }\alpha,\beta \in (\Sigma\cup V)^\star\}.
  \end{align*}
  \begin{itemize}
  \item
    Ясно е, че $V_0 = \{S\}$ и $\Sigma_0 = \emptyset$.
  \item
    \marginpar{Всяка итерация отнема време $\mathcal{O}(|G|)$. Следователно, можем да намерим $G'$ за време $\mathcal{O}(|G|^2)$.}
    Нека имаме $V_n$ и $\Sigma_n$. Тогава:
    \begin{align*}
      & V_{n+1} = \{A \in V \mid (\exists B \in V_n)[ B \to_G \alpha A \beta\text{, за някои }\alpha,\beta \in (\Sigma\cup V)^\star]\}\\
      & \Sigma_{n+1} = \{a \in \Sigma \mid (\exists B \in V_n)[ B \to_G \alpha a \beta\text{, за някои }\alpha,\beta \in (\Sigma\cup V)^\star]\}.
    \end{align*}
  \item
    Спираме, когато намерим първото $n$, за което $V_n = V_{n+1}$ и $\Sigma_n = \Sigma_{n+1}$.
    Тогава $V' = V_n$ и $\Sigma' = \Sigma_n$.
  \end{itemize}
\end{hint}

\begin{thm}
  За всяка безконтекстна граматика $G$, за която $\L(G) \neq \emptyset$, съществува еквивалетнтна на нея безконтекстна граматика $G'$ без безполезни правила.
  Освен това, граматиката $G'$ може да се намери за полиномиално време.
\end{thm}
\begin{hint}
  \marginpar{Защо е важна последователността на прилагане?}
  Прилагаме върху $G$ първо процедурата от \Lem{useless1} и след това върху резултата прилагаме процедурата от \Lem{useless2}.
\end{hint}

\begin{example}
  Да разгледаме следната граматика $G$:
  \begin{align*}
    & S \to AB\ |\ aA\\
    & A \to a\ |\ aAa\\
    & B \to SB\ |\ BC\\
    & C \to \varepsilon\ |\ cC.
  \end{align*}
  Първо да намерим променливите, от които се извеждат думи.
  \begin{itemize}
  \item 
    $V_0 = \{A, C\}$, защото $A \to a$ и $C \to \varepsilon$;
  \item
    $V_1 = \{A, C, S\}$, защото $S \to aA$;
  \item
    не можем да добавим $B$ към $V_2$, следователно $V_1 = V_2$.
  \end{itemize}
  Получаваме граматиката $G'$:
  \begin{align*}
    & S \to aA\\
    & A \to a\ |\ aAa\\
    & C \to \varepsilon\ |\ cC.
  \end{align*}
  Сега премахваме променливите и буквите, които не са достижими от началната промелива $S$. Така получаваме граматиката $G''$:
  \begin{align*}
    & S \to aA\\
    & A \to a\ |\ aAa.
  \end{align*}
\end{example}

\begin{problem}
  Проверете дали $\L(G) = \emptyset$, където правилата на $G$ са:
  \begin{align*}
    & S \to AS\ |\ BC\\
    & A \to 0\ |\ BA\ |\ SB\\
    & B \to 1\ |\ BC\ |\ AB\\
    & C \to CB\ |\ SC\ |\ AS.
  \end{align*}
\end{problem}



%%% Local Variables:
%%% mode: latex
%%% TeX-master: "../eai"
%%% End:
