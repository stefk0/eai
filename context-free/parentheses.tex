\subsection*{Балансирани скоби}

Тук ще разглеждаме азбука $\Sigma$, която включва буквите $\texttt{[}$ и $\texttt{]}$.
Нека за по-голяма яснота да положим
\begin{align*}
  & \texttt{left}(\alpha) \df N_{\texttt{[}}(\alpha);\\
  & \texttt{right}(\alpha) \df N_{\texttt{]}}(\alpha).
\end{align*}
Нека $\alpha$ е дума над азбука, която включва буквите $\texttt{[}$ и $\texttt{]}$. 
Ще казваме, че че $\alpha$ е {\bf балансирана}, ако са изпълнени свойствата:
\begin{enumerate}[1)]
\item 
  $\texttt{left}(\alpha) = \texttt{right}(\alpha)$;
\item
  За всеки префикс $\gamma$ на $\alpha$,
  $\texttt{left}(\gamma) \geq \texttt{right}(\gamma)$.
\end{enumerate}
  
\begin{problem}
  \marginpar{\cite[стр. 135]{kozen}}
  Докажете, че езикът 
  \[L = \{\alpha \in \{\texttt{[},\texttt{]}\}^\star \mid \alpha\text{ е балансирана дума}\}\]
  е безконтекстен.
\end{problem}
\begin{hint}
  Да разгледаме граматиката $G$ с правила
  \[S \to \texttt{[}S\texttt{]}\ |\ SS\ |\ \varepsilon.\]
  Ще докажем, че $L = \L(G)$.
  
  Да разгледаме $M \df \{\alpha \in \{\texttt{[},\texttt{]}, S\}^\star \mid \alpha\text{ е балансирана}\}$.
  Нека $\alpha \in \L(G)$. Ще докажем, че $\alpha \in M$.
  С индукция по дължината на извода ще докажем, че $\alpha$ е балансирана.
  Нека $S \to^{l}_G \beta \to^1_G \alpha$.
  Лесно се съобразява, че за всички случаи за $\beta$ и $\alpha$ имаме следното:
  \begin{center}
    \begin{tabular}{| c | c | c |}
      \hline
      $\text{от И.П. за }\beta$ & $\text{правило на }G$ & $\alpha$ \\ \hline
      $\in M$ & $S \rightarrow \texttt{[}S\texttt{]}$ & $\in M$ \\ \hline
      $\in M$ & $S \rightarrow SS$ & $\in M$ \\ \hline
      $\in M$ & $S \rightarrow \varepsilon$ & $\in M$ \\ \hline
    \end{tabular}
  \end{center}

  За другата посока, нека $\alpha \in L$.
  Ще докажем с индукция по дължината на думата, че $\alpha \in \L(G)$.
  Ясно е, че за всички нетривиални случаи, можем да запишем думата $\alpha$ като $\alpha = \texttt{[}\beta\texttt{]}$.
  Проблемът е, че в общия случай не е ясно дали можем да приложим индукционното предположение за $\beta$,
  защото е възможно $\beta \not\in L$. Например, $\alpha = \texttt{[][]}$.
  Тогава $\beta = \texttt{][} \not \in L$.
  Поради тази причина, трябва да сме по-внимателни и да разгледаме два случая.
  \begin{itemize}
  \item 
    \marginpar{т.е. $\beta \neq \varepsilon$ и $\beta \neq \alpha$}
    Нека $\alpha$ има {\em същински} префикс $\beta \in L$.
    Понеже $\alpha \in L$, лесно се съобразява, че $\alpha = \beta\gamma$ и $\gamma \in L$
    Сега можем да приложим {\bf И.П.} за $\beta$ и $\gamma$ и да получим, че 
    $\beta \in \L(G)$ и $\gamma \in \L(G)$, т.е.
    $S \to^\star_G \beta$ и $S \to^\star_G \gamma$.
    Понеже имаме правило $S \to_G SS$, то е ясно, че $\alpha \in \L(G)$.
  \item
    Нека $\alpha$ да няма същински префикс $\beta \in L$.
    Ясно е, че винаги $\alpha = \texttt{[}\beta\texttt{]}$, за някое $\beta$
    и $\texttt{left}(\beta) = \texttt{right}(\beta)$. Ако $\beta \in L$, то ще можем да приложим {\bf И.П.}
    за $\beta$ и ще сме готови.
    За всеки префикс $\gamma$ на $\beta$ имаме, че $\texttt{[}\gamma$ е префикс на $\alpha$,
    и понеже $\alpha \in L$, то $\texttt{left}(\gamma) \geq \texttt{right}(\gamma)$.
    Това означава, че $\beta \in L$ и можем да приложим {\bf И.П.}
    Тогава $S \to^\star_G \beta$ и чрез правилото $S \to_G \texttt{[}S\texttt{]}$
    получаваме, че $\alpha \in \L(G)$.    
  \end{itemize}
\end{hint}

%%% Local Variables:
%%% mode: latex
%%% TeX-master: "../eai"
%%% End:
