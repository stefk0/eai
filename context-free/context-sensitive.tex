\section{Контекстни граматики}
\index{граматика!контекстна}
\mynote{На англ. context-sensitive \cite[стр. 223]{hopcroft1}. Евентуално позволяваме и правилото $S \to \varepsilon$,
ако искаме да включим $\varepsilon$ в езика, като $S$ не се среща в дясна страна на правило.}
Казваме, че $G = (V,\Sigma,R,S)$ е {\bf контекстна граматика}, ако правилата на $G$ са от вида
$\rho A \delta \to \rho \alpha \delta$, където $\rho,\delta \in (V\cup\Sigma)^\star$ и $\alpha \in (V\cup\Sigma)^+$.

\begin{extra}
\begin{example}
  Езикът $L = \{a^nb^nc^n \mid n > 0\}$ е контекстен.
\end{example}
\begin{hint}
  Разгледайте контекстната граматика $G$ зададена със следните правила:
  \mynote{Съобразете, че имаме извода $CB \derive{4}_G BC$.}
  \begin{align*}
    & S \to aSBC\ |\ aBC\\
    & CB \to CZ\\
    & CZ \to WZ\\
    & WZ \to WC\\
    & WC \to BC\\
    & aB \to ab\\
    & bB \to bb\\
    & bC \to bc\\
    & cC \to cc.
  \end{align*}

  Докажете, че за всяко $n > 0$ е изпълено следното:
  \begin{itemize}
  \item
    $S \derive{n} a^n(BC)^n$;
  \item
    $(BC)^n \derive{n-1} B^nC^n$;
  \item
    $aB^n \derive{n} ab^n$;
  \item
    $bC^n \derive{n} bc^n$.
  \end{itemize}
  Оттук лесно можем да докажем, че $L \subseteq \L(G)$.
\end{hint}
\end{extra}
%%% Local Variables:
%%% mode: latex
%%% TeX-master: "../eai"
%%% End:
