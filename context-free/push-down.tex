\section{Недетерминирани стекови автомати}

\index{автомат!недетерминиран стеков}
\mynote{На англ. {\em Push-down automaton}. В този курс няма да разглеждаме детерминирани стекови автомати. Когато кажем стеков автомат, ще имаме предвид недетерминиран стеков автомат.
  Означаваме $\Sigma_\varepsilon \df \Sigma \cup \{\varepsilon\}$ и $\Gamma^{\leq 2} \df \{\varepsilon\} \cup \Gamma \cup \Gamma^2$.}
{\bf Недетерминиран стеков автомат} е седморка от вида
\[P = \PDA,\] където:
\begin{itemize}
\item
  $Q$ е крайно множество от състояния;
\item  
  $\Sigma$ е крайна входна азбука;
\item
  $\Gamma$ е крайна стекова азбука;
\item
  $\sharp \in \Gamma$ е символ за дъно на стека;
\item
  \mynote{Обикновено се взима $\Delta$ функцията да има сигнатура
    $\Delta:Q\times\Sigma_\varepsilon\times \Gamma \rightarrow \Ps(Q\times\Gamma^{\star})$.
    Дефиницията на стеков автомат има много вариации, всички еквивалентни помежду си.}
  $\Delta:Q\times\Sigma_\varepsilon\times \Gamma \rightarrow \Ps(Q\times\Gamma^{\leq 2})$ 
  е функция на преходите;    
\item
  $\qstart \in Q$ е начално състояние;
\item
  $\qaccept \in Q$ е заключителното състояние.
\end{itemize}

\mynote{На англ. Instanteneous description}
\emph{Конфигурация} (или моментно описание) на изчислението със стеков автомат представлява тройка от вида $(q,\alpha,\gamma) \in Q\times\Sigma^\star\times\Gamma^\star$,
т.е. автоматът се намира в състояние $q$, думата, която остава да се прочете е $\alpha$,
а съдържанието на стека е думата $\gamma$.
Удобно е да въведем бинарната релация $\vdash_P$ над $Q\times\Sigma^\star\times\Gamma^\star$,
която ще ни казва как моментното описание на автомата $P$ се променя след изпълнение на една стъпка.

% \begin{prooftree}
%   \AxiomC{}
%   \RightLabel{\scriptsize{правило (0)}}
%   \UnaryInfC{$(p,\alpha,\gamma) \vdash^0_P (p,\alpha,\gamma)$}
% \end{prooftree}

% \begin{prooftree}
%   \AxiomC{$\Delta(q,b,A) \ni (p,\beta)$}
%   \AxiomC{$(p,\alpha,\beta\gamma) \vdash^\ell_P (r,\alpha',\delta)$}
%   \RightLabel{\scriptsize{правило (1)}}
%   \BinaryInfC{$(q,b\alpha,A\gamma) \vdash^{\ell+1}_P (r,\alpha',\delta)$}
% \end{prooftree}

% \begin{prooftree}
%   \AxiomC{$\Delta(q,\varepsilon,A) \ni (p,\beta)$}
%   \AxiomC{$(p,\alpha,\beta\gamma) \vdash^\ell_P (r,\alpha',\delta)$}
%   \RightLabel{\scriptsize{правило (2)}}
%   \BinaryInfC{$(q,\alpha,A\gamma) \vdash^{\ell+1}_P (r,\alpha',\delta)$}
% \end{prooftree}


\begin{important}
  \begin{figure}[H]
    \begin{subfigure}[b]{0.5\textwidth}
      \begin{prooftree}
        \AxiomC{$\Delta(q,b,A) \ni (p,\beta)$}
        \AxiomC{$b \in \Sigma$}
        \BinaryInfC{$(q,b\alpha,A\gamma) \vdash_P (p,\alpha,\beta\gamma)$}
      \end{prooftree}
    \end{subfigure}
    ~
    \begin{subfigure}[b]{0.5\textwidth}
      \begin{prooftree}
        \AxiomC{$\Delta(q,\varepsilon,A) \ni (p,\beta)$}
        \UnaryInfC{$(q,\alpha,A\gamma) \vdash_P (p,\alpha,\beta\gamma)$}
      \end{prooftree}
    \end{subfigure}
    \caption{Правила за извършване на една стъпка в изчисление на стеков автомат}
  \end{figure}
\end{important}


Сега можем дефинираме бинарната релация $\vdash^\ell_P$ над $Q\times\Sigma^\star\times\Gamma^\star$, която ни казва, че конфигурацията $\kappa$ се променя до конфигурация $\kappa'$ след изчисление от $\ell$ стъпки на стековия автомат:

\begin{figure}[H]
  \begin{subfigure}[b]{0.5\textwidth}
    \begin{prooftree}
      \AxiomC{}
      \RightLabel{\scriptsize{(рефлексивност)}}
      \UnaryInfC{$\kappa \vdash^0_P \kappa$}
    \end{prooftree}
  \end{subfigure}
  ~
  \begin{subfigure}[b]{0.5\textwidth}
    \begin{prooftree}
      \AxiomC{$\kappa \vdash_P \kappa''$}
      \AxiomC{$\kappa'' \vdash^\ell_P \kappa'$}
      \RightLabel{\scriptsize{(транзитивност)}}
      \BinaryInfC{$\kappa \vdash^{\ell+1}_P \kappa'$}
    \end{prooftree}
  \end{subfigure}
\end{figure}

\mynote{Казано по-дискретно, $\vdash^\star_P$ е рефлексивното и транзитивно затваряне на релацията $\vdash_P$.}
Дефинираме релацията $\vdash^\star_P$ като $\kappa \vdash^\star_P \kappa' \dff (\exists \ell)[\kappa \vdash^\ell_P \kappa']$.
% \[(q,\alpha,\beta) \vdash^\star_P (p,\alpha',\beta') \dff (\exists \ell)[(q,\alpha,\beta) \vdash^\ell_P (p,\alpha',\beta')].\]
Езикът $\L(P)$, който се разпознава от стековия автомат $P$ дефинираме по следния начин:
\[\L(P) \df \{\ \omega \in \Sigma^\star \mid (\qstart,\omega,\sharp) \vdash^\star_P (\qaccept,\varepsilon,\varepsilon)\ \}.\]
\mynote{Възможно е да се дадат и други еквивалентни дефиниции на $\L(P)$.}

\begin{proposition}
  За произволни $\ell_1$ и $\ell_2$ е изпълнено, че:
  \begin{prooftree}
    \AxiomC{$(p,\alpha_1,\gamma_1) \vdash^{\ell_1} (q,\varepsilon,\varepsilon)$}
    \AxiomC{$(q,\alpha_2,\gamma_2) \vdash^{\ell_2} (r,\varepsilon,\varepsilon)$}
    \BinaryInfC{$(p,\alpha_1\alpha_2,\gamma_1\gamma_2) \vdash^{\ell_1+\ell_2} (r,\varepsilon,\varepsilon)$}  \end{prooftree}
\end{proposition}

\begin{proposition}
  Ако $(q,\alpha,AB) \vdash^\ell (p,\varepsilon,\varepsilon)$, то съществуват $\ell_1$, $\ell_2$,
  състояние $r$ и разбиване на $\alpha$ като $\alpha = \alpha_1\alpha_2$, така че:
  \[(q,\alpha_1\alpha_2,AB) \vdash^{\ell_1} (r, \alpha_2, B) \vdash^{\ell_2} (p,\varepsilon,\varepsilon).\]
\end{proposition}


\subsection{Примери}

\begin{extra}
\begin{example}
  \label{ex:anbn}
  За езика $L = \{a^nb^n\mid n\in\Nat\}$, да разгледаме $P = \PDA$, където
  \begin{itemize}
  \item
    $Q \df \{q,p,f\}$;
  \item
    $\qstart \df q$ и $\qaccept \df f$;
  \item
    $\Sigma \df \{a,b\}$ и $\Gamma \df \{\sharp,a\}$;
  \item
    \mynote{Тук получаваме детерминистичен стеков автомат.}
    Релацията на преходите $\Delta$ има следната дефиниция:
    \begin{enumerate}[(1)]
    \item
      $\Delta(q,a,\sharp) \df \{(q, a\sharp)\}$;
    \item
      $\Delta(q,a,a) \df \{(q, aa)\}$; \quad \comment{трупаме $a$-та в стека}
    \item 
      $\Delta(q,\varepsilon,\sharp) \df \{(f,\varepsilon)\}$;\quad \comment{трябва да разпознаем и думата $\varepsilon$}
    \item 
      $\Delta(q, b, a) \df \{(p,\varepsilon)\}$; \quad \comment{Започваме да четем само $b$-та}
    \item 
      $\Delta(p, b, a) \df \{(p,\varepsilon)\}$; \quad \comment{Чистим $a$-тата от стека}
    \item
      $\Delta(p, \varepsilon, \sharp) \df \{(f, \varepsilon)\}$.
    \item
      За всички останали тройки $(r,x,y)$, нека $\Delta(r,x,y) \df \emptyset$.
    \end{enumerate}
  \end{itemize}
  
  Да видим как думата $a^2b^2$ се разпознава от стековия автомат $P$:
  \begin{align*}
    (q, a^2b^2, \sharp) & \vdash_P (q, ab^2, a\sharp) & \comment{\text{правило }(1)}\\
                        & \vdash_P (q, b^2, aa\sharp) & \comment{\text{правило }(2)}\\
                        & \vdash_P (p, b, a\sharp) & \comment{\text{правило }(4)}\\
                        & \vdash_P (p, \varepsilon, \sharp) & \comment{\text{правило }(5)}\\
                        & \vdash_P (f, \varepsilon, \varepsilon) & \comment{\text{правило }(6)}
  \end{align*}
  % \mynote{\writedown Докажете, че $L = \L(P)$!}
  Получихме, че $(\qstart, a^2b^2, \sharp) \vdash^\star_P (\qaccept, \varepsilon, \varepsilon)$, откъдето следва, че $a^2b^2 \in \L(P)$.

  \begin{enumerate}[a)]
  \item
    Докажете с индукция по $n$, че за всяко естествено число $n$ са изпълнени свойствата:
    \begin{align}
      & (q, a^n\beta, \sharp) \vdash^n_P (q, \beta, a^n\sharp) \label{eq:anbn:1}\\
      & (p, b^n, a^n\sharp) \vdash^n_P (p, \varepsilon,\sharp). \label{eq:anbn:2}
    \end{align}
    Заключете, че $L \subseteq \L(P)$.
  \item
    Докажете, че с индукция по $n$, че за всяко естествено число $n$ са изпълнени свойствата:
    \begin{align}
      (q, \alpha\beta, \sharp) \vdash^n_P (q, \beta, \gamma\sharp)  & \implies \alpha = \gamma = a^n \label{eq:anbn:3}\\
      (p, \beta, \gamma\sharp) \vdash^n_P (p, \varepsilon, \sharp) & \implies \beta = b^n\ \&\ \gamma = a^n. \label{eq:anbn:4}
    \end{align}
    Оттук заключете, че $\L(P) \subseteq L$.    
  \end{enumerate}
\end{example}

\begin{example}
  \label{ex:omega-omega-r}
  Езикът $L = \{\ \omega\omega^{\rev} \mid \omega \in \{a,b\}^\star\ \}$ се разпознава от стеков автомат
  \[P = \PDA,\] където:
  \begin{itemize}
  \item 
    $Q \df \{q,p,f\}$ и $\qstart \df q$, $\qaccept \df f$;
  \item
    $\Sigma \df \{a,b\}$, $\Gamma \df \{a, b, \sharp\}$;
  \item
    Функцията на преходите $\Delta$ има следната дефиниция:
    \mynote{За всички липсващи твойки  в дефиницията на $\Delta$ приемаме, че $\Delta$ връща $\emptyset$}
    \begin{enumerate}[(1)]
    \item
     $\Delta(q, x, \sharp) \df \{(q, x\sharp)\}$, където $x \in \{a,b\}$;
    % \item 
    %   $\Delta(q, a, \sharp) \df \{(q, a\sharp)\}$;
    % \item 
    %   $\Delta(q, b, \sharp) \df \{(q, b\sharp)\}$;
    \item
      $\Delta(q, \varepsilon, \sharp) \df \{(q,\varepsilon)\}$;
    \item
      $\Delta(q, x, x) \df \{(q, xx), (p, \varepsilon)\}$, където $x \in \{a,b\}$;
    \item
      $\Delta(q, a, b) \df \{(q, ab)\}$;
    \item
      $\Delta(q, b, a) \df \{(q, ba)\}$;
    % \item
      % $\Delta(q, b, b) \df \{(q, bb), (p, \varepsilon)\}$;
    \item
      $\Delta(p, x, x) \df \{(p,\varepsilon)\}$, където $x \in \{a,b\}$;
    % \item
    %   $\Delta(p, b, b) \df \{(p,\varepsilon)\}$;
    \item
      $\Delta(p, \varepsilon, \sharp) \df \{(f,\varepsilon)\}$;
    \end{enumerate}
  \end{itemize}
  Основното наблюдение, което трябва да направим за да разберем конструкцията на автомата е, че
  всяка дума от вида $\omega\omega^{\texttt{rev}}$ може да се запише като $\omega_1aa\omega^{\texttt{rev}}_1$ или $\omega_1bb\omega^{\texttt{rev}}_1$.
  Да видим защо $P$ разпознава думата $abaaba$ с празен стек.
  Започваме по следния начин:
  \begin{align*}
    (q, abaaba,\sharp) & \vdash_P (q, baaba, a\sharp)   & \comment{\text{правило }(1)}\\
                       & \vdash_P (q, aaba, ba\sharp)   & \comment{\text{правило }(5)}\\
                       & \vdash_P (q, aba,  aba\sharp). & \comment{\text{правило }(4)}
  \end{align*}
  Сега можем да направим два избора как да продължим. Състоянието $p$ служи за маркер, което ни казва, че вече сме започнали 
  да четем $\omega^{\texttt{rev}}$. Поради тази причина, продължаваме така:
  \begin{align*}
    (q, aba, aba\sharp) & \vdash_P (p, ba, ba\sharp) & \comment{\text{правило }(3)}\\
                        & \vdash_P (p, a, a\sharp) & \comment{\text{правило }(6)}\\
                        & \vdash_P (p, \varepsilon, \sharp) & \comment{\text{правило }(6)}\\
                        & \vdash_P (f,\varepsilon,\varepsilon). & \comment{\text{правило }(7)}
  \end{align*}
  Да проиграем още един пример. Да видим защо думата $aba$ не се извежда от автомата.
  \begin{align*}
    (q, aba, \sharp) & \vdash_P (q, ba, a\sharp) & \comment{\text{правило }(1)}\\
                     & \vdash_P (q, a, ba\sharp) & \comment{\text{правило }(5)}\\
                     & \vdash_P (q, \varepsilon, aba\sharp). & \comment{\text{правило }(4)}
  \end{align*}
  От последното моментно описание на автомата нямаме нито един преход, следователно
  думата $aba$ не се разпознава от $P$.
  \mynote{\writedown Докажете, че $L = \L(P)$!}
  \begin{enumerate}[a)]
  \item
    \mynote{За (\ref{eq:omega-omega-r:1}) приложете индукция по дължината на думата $\alpha$. За индукционната стъпка разгледайте $\alpha$ като $\alpha = \alpha' x$.}
    Докажете с индукция пo $n$, за всяко естествено число $n$ са изпълнени свойствата:
    \begin{align}
      & |\alpha| = n\ \implies\ (q, \alpha\beta, \sharp) \vdash^n_P (q, \beta, \alpha^{\texttt{rev}}\sharp) \label{eq:omega-omega-r:1}\\
      & |\beta| = n\ \implies\ (p, \beta, \beta\sharp) \vdash^n_P (p, \varepsilon, \sharp). \label{eq:omega-omega-r:2}
    \end{align}
    Оттук заключете, че $L \subseteq \L(P)$.
  \item
    Докажете с индукця по $n$, че за всяко естествено число $n$ са изпълнени свойствата:
    \begin{align}
      (q, \alpha\beta, \sharp) \vdash^n_P (q, \beta, \gamma\sharp) & \implies \gamma = \alpha^{\rev}\ \&\ |\alpha| = n \label{eq:omega-omega-r:3}\\
      (p, \beta, \gamma\sharp) \vdash^n_P (p, \varepsilon, \sharp) & \implies \gamma = \beta\ \&\ |\beta| = n. \label{eq:omega-omega-r:4}
    \end{align}
    Оттук заключете, че $\L(P) \subseteq L$.
  \end{enumerate}
\end{example}

% \begin{problem}
%   Постройте \emph{детерминистичен} стеков автомат за езика $L = \{\omega \$ \omega^{\rev} \mid \omega \in \{a,b\}^\star \}$.
% \end{problem}

\begin{example}
  \mynote{От \Problem{equal-number-parentheses} знаем, че този език е безконтекстен.}
  Езикът $L = \{\ \omega \in \{a,b\}^\star \mid \abs{\omega}_a = \abs{\omega}_b\}$
  се разпознава от стековия автомат $P = \PDA$, където:
  \begin{itemize}
  \item 
    $Q = \{q,f\}$;
  \item
    $\Sigma = \{a,b\}$;
  \item
    $\Gamma = \{a, b, \sharp\}$;
  \item
    $\qstart = q$ и $\qaccept = f$;
  \item
    Можем да дефинираме релацията на преходите $\Delta$ по следния начин:
    \begin{enumerate}[(1)]
    \item 
      $\Delta(q, \varepsilon, \sharp) = \{(f, \varepsilon)\}$;
    \item
      $\Delta(q, x, \sharp) = \{(q, x\sharp)\}$, където $x \in \{a,b\}$;
    \item
      $\Delta(q, x, x) = \{(q, xx)\}$, където $x \in \{a,b\}$;
    \item
      $\Delta(q, a, b) = \{(q, \varepsilon)\}$;
    \item
      $\Delta(q, b, a) = \{(q, \varepsilon)\}$.
    \end{enumerate}
  \end{itemize}
  Да видим защо думата $abbbaa \in \L(P)$.
  \begin{align*}
    (q, abbbaa, \sharp) & \vdash_P (q, bbbaa,\ a\sharp) & \comment{\text{правило }(2)}\\
                        & \vdash_P (q, bbaa,\ \sharp) & \comment{\text{правило }(5)}\\
                        & \vdash_P (q, baa,\ b\sharp) & \comment{\text{правило }(2)}\\
                        & \vdash_P (q, aa, bb\sharp) & \comment{\text{правило }(3)}\\
                        & \vdash_P (q, a,\ b\sharp) & \comment{\text{правило }(4)}\\
                        & \vdash_P (q, \varepsilon,\ \sharp) & \comment{\text{правило }(4)}\\
                        & \vdash_P (f, \varepsilon,\ \varepsilon). & \comment{\text{правило }(1)}
  \end{align*}

  \begin{enumerate}[a)]
  \item
    Докажете с пълна индукция по дължината на думата $\gamma$, че за произволна дума $\gamma \in \{a, b\}^\star$, е изпълнено, че:
    \begin{align}
      (\forall n)[a^n\gamma \in L & \implies (q, \gamma, a^n\sharp) \vdash^\star_P (q, \varepsilon, \sharp)] \label{eq:omega-ab:1}\\
      (\forall n)[b^n\gamma \in L & \implies (q, \gamma, b^n\sharp) \vdash^\star_P (q, \varepsilon, \sharp)]. \label{eq:omega-ab:2}
    \end{align}
    % Ще докажем едновременно \Property{eq:omega-ab:1} и \Property{eq:omega-ab:2} с пълна индукция по дължината на думата $\gamma$.

    % Да разгледаме произволна дума $\gamma$. Случаят, когато $\abs{\gamma} = 0$ е тривиален. 
    % Нека $\abs{\gamma} > 0$ и да приемем, че $a^n\gamma \in L$.
    % Ясно е, че можем да представим $\gamma$ като $\gamma = a^kb\gamma'$, за някое $k$.
    % \begin{itemize}
    % \item
    %   Ако $n+k = 0$, то $a^n\gamma = b\gamma' \in L$ и прилагаме \IndHyp за \Property{eq:omega-ab:2} с думата $\gamma'$ и получаваме, че:
    %   \begin{align*}
    %     (q, \overbrace{b\gamma'}^{\gamma},\sharp) & \vdash (q,\gamma',b\sharp) & \comment{\text{правило (2)}}\\
    %                          & \vdash^\star (q, \varepsilon,\sharp) & \comment\text{\IndHyp}
    %   \end{align*}
    % \item
    %   Ако $n+k>0$, то $a^{n+k-1}\gamma' \in L$ и прилагаме \IndHyp за \Property{eq:omega-ab:1} с думата $\gamma'$ и получаваме, че:
    %   \begin{align*}
    %     (q, \overbrace{a^{k}b\gamma'}^{\gamma}, a^n\sharp) & \vdash^\star_P (q, b\gamma', a^{n+k}\sharp) & \comment{\text{правило (3)}}\\
    %                                                         & \vdash^\star_P (q, \gamma', a^{n+k-1}\sharp) & \comment\text{правило (5)}\\
    %                                                         & \vdash^\star_P (q, \varepsilon, \sharp). & \comment\text{\IndHyp}
    %   \end{align*}
    % \end{itemize}
    % Аналогично се доказва и \Property{eq:omega-ab:2}.
    
    Оттук заключете, че $L \subseteq \L(P)$.
  \item
    Докажете с пълна индукция по дължината на думата $\gamma$, че за произволна дума $\gamma \in \{a, b\}^\star$ е изпълнено, че:
    \begin{align}
      (\forall n)[(q, \gamma, a^n\sharp) \vdash^\star_P (q, \varepsilon, \sharp) & \implies a^n\gamma \in L] \label{eq:omega-ab:3}\\
      (\forall n)[(q, \gamma, b^n\sharp) \vdash^\star_P (q, \varepsilon, \sharp) & \implies b^n\gamma \in L]. \label{eq:omega-ab:4}
    \end{align}

    % Нека имаме изчислението $(q, \gamma, a^n\sharp) \vdash^\star_P (q, \varepsilon, \sharp)$.
    % Да представим думата $\gamma$ като $\gamma = a^kb\gamma'$.
    
    % \begin{itemize}
    % \item
    %   Ако $n+k = 0$, то можем да разбием изчислението по следния начин:
    %   \begin{align*}
    %     (q, b\gamma', \sharp) & \vdash_P (q, \gamma',b\sharp) \\
    %                           & \vdash^{\star} (q,\varepsilon,\sharp).
    %   \end{align*}
    %   Тогава от \IndHyp за \Property{eq:omega-ab:4} следва, че $a^n\gamma = b\gamma' \in L$.
    % \item
    %   Ако $n+k > 0$, то можем да разбием изчислението по следния начин:
    %   \begin{align*}
    %     (q, a^kb\gamma', a^n\sharp) & \vdash^\star_P (q, b\gamma',a^{n+k}\sharp) \\
    %                                 & \vdash_P (q, \gamma',a^{n+k-1}\sharp) \\
    %                                 & \vdash^{\star} (q,\varepsilon,\sharp).
    %   \end{align*}
    %   Тогава от \IndHyp за \Property{eq:omega-ab:3} следва, че $a^{n+k-1}\gamma' \in L$, но оттук
    %   веднага получаваме, че $a^na^kb\gamma' \in L$.
    % \end{itemize}
    % Аналогично се доказва и \Property{eq:omega-ab:4}.
    
    Оттук заключете, че $\L(P) \subseteq L$.
  \end{enumerate}
\end{example}

\begin{example}
  \mynote{От \Problem{balanced-parentheses} знаем, че този език е безконтекстен.}
  Езикът $L = \{\ \omega \in \{a,b\}^\star \mid \omega\text{ е балансирана дума}\ \}$
  се разпознава от стековия автомат $P = \PDA$, където:
  \begin{itemize}
  \item 
    $Q = \{q,f\}$;
  \item
    $\qstart = q$ и $\qaccept = f$;
  \item
    $\Sigma = \{a,b\}$ и $\Gamma = \{a, \sharp\}$;
  \item
    Можем да дефинираме релацията на преходите $\Delta$ по следния начин:
    \mynote{\writedown Докажете, че $L = \L(P)$!}
    \begin{enumerate}[(1)]
    \item 
      $\Delta(q, \varepsilon, \sharp) = \{(f, \varepsilon)\}$;
    \item
      $\Delta(q, a, \sharp) = \{(q, a\sharp)\}$;
    \item
      $\Delta(q, a, a) = \{(q, aa)\}$;
    \item
      $\Delta(q, b, a) = \{(q, \varepsilon)\}$;
    \end{enumerate}
  \end{itemize}  
  \begin{enumerate}[(a)]
  \item
    \mynote{Индукция по дължината на думата $\beta$.}
    Докажете, че за произволно естествено число $n$ и произволна дума $\beta \in \{a, b\}^\star$, 
    е изпълнено, че:
    \[a^n\beta \in L\ \implies (q, \beta, a^n\sharp) \vdash^\star_P (q, \varepsilon, \sharp).\]
    Оттук заключете, че $L \subseteq \L(P)$.
  \item
    \mynote{Индукция по броя на стъпките в изчислението на стековия автомат.}
    Докажете, че за произволно естествено число $n$ и произволна дума $\beta \in \{a, b\}^\star$, е изпълнено, че:
    \[(q,\beta,a^n\sharp) \vdash^\star_P (q, \varepsilon, \sharp)\ \implies\ a^n\beta \in L.\]
    Оттук заключете, че $\L(P) \subseteq L$.
  \end{enumerate}
\end{example}
\end{extra}


%%% Local Variables:
%%% mode: latex
%%% TeX-master: "../eai"
%%% End:

\input{context-free/equivalence}
\subsection*{Сечение с регулярен език}
От \Proposition{context-free:pumping:non-closure} знаем, че безконтекстните езици не
са затворени относно операцията сечение, т.е. възможно е $L_1$ и $L_2$ да са безконтекстни
езици, но $L_1 \cap L_2$ да не е безконтекстен.
Оказва се обаче, че безконтекстните езици са затворени относно сечение с регулярен език.

\begin{important}
  \begin{theorem}\label{th:intersection-context-reg}
    Нека $L$ e безконтекстен език и $R$ е регулярен език.
    Тогава тяхното сечение $L \cap R$ е безконтекстен език.
  \end{theorem}
\end{important}
\begin{hint}
  \mynote{Тук адаптираме доказателството от \cite[стр. 144]{papadimitriou}.}
  Нека имаме стеков автомат
  \[P = \pair{Q',\Sigma,\Gamma,\sharp, \Delta', \qstart', \qaccept'}, \text{ където } \L(P) = L,\]
  и детерминиран краен автомат 
  \[\A = \pair{Q'', \Sigma, \qstart'', \delta'', F''}, \text{ където } \L(\A) = R.\]
  \mynote{Сравнете с конструкцията от \Proposition{automata-cap}.}
  Ще определим нов стеков автомат $\M = \PDA$, където:
  \begin{itemize}
  \item 
    $Q \df Q' \times Q''$;
  \item
    $\qstart \df \pair{\qstart',\qstart''}$;
  \item
    $F \df \{\qaccept'\} \times F''$;
  \item 
    Функцията на преходите $\Delta$ е дефинирана както следва:
    \begin{itemize}
    \item 
      \mynote{Симулираме едновременно изчислението и на двата автомата.}
      Ако $(r_1,\gamma) \in \Delta'(q_1, a, Y)$, то
      \[ \Delta(\pair{q_1,q_2},a,Y) \ni (\pair{r_1,\delta''(q_2,a)}, \gamma).\]
    \item
      \mynote{Нищо не четем от входната дума, следователно правим празен ход на $\A$}
      Ако $(r_1,\gamma) \in \Delta'(q_1,\varepsilon,Y)$ и всяко $q_2 \in Q''$, то
      \[ \Delta(\pair{q_1,q_2},\varepsilon,Y) \ni (\pair{r_1,q_2}, \gamma).\]
    \item
      \mynote{\writedown Докажете, че $\L(\M) = \L(P) \cap \L(\A)$ !}
      $\Delta$ не съдържа други преходи;
    \end{itemize}
  \end{itemize}

  \begin{itemize}
  \item
    Докажете, че ако $(\pair{q_1,q_2},\alpha,\gamma) \vdash^\star_\M (\pair{p_1,p_2},\varepsilon,\varepsilon)$, то
    \[(q_1,\alpha,\gamma) \vdash^\star_P (p_1,\varepsilon,\varepsilon)\text{ и }(q_2,\alpha) \vdash^\star_\A (p_2,\varepsilon).\]
  \item
    Докажете, че ако $(q_1,\alpha,\gamma) \vdash^\star_P (p_1,\varepsilon,\varepsilon)$ и $(q_2,\alpha) \vdash^\star_\A (p_2,\varepsilon)$, то
    \[(\pair{q_1,q_2},\alpha,\gamma) \vdash^\star_\M (\pair{p_1,p_2},\varepsilon,\varepsilon).\]
  \end{itemize}
  
\end{hint}

\begin{extra}
  \Theorem{intersection-context-reg} е удобна, когато искаме да докажем, че даден език не е безконтекстен.
С нейна помощ можем да сведем езика до друг, за който вече знаем, че не е безконтекстен.

  \begin{example}
    Да разгледаме езика $L = \{\omega \in \{a,b,c\}^\star \mid \card{\omega}{a} = \card{\omega}{b} = \card{\omega}{c}\}$.
    Да допуснем, че $L$ е безконтекстен език. Тогава, според \Theorem{intersection-context-reg}, $L^\prime = L \cap \L(a^\star b^\star c^\star)$ също е безконтекстен език.
    Но $L^\prime = \{a^nb^nc^n \mid n \in \Nat\}$, за който знаем от \Example{context-free:pumping:anbncn}, че {\em не} е безконтекстен.
    Достигнахме до противоречие. Следователно, $L$ не е безконтекстен език.
  \end{example}
\end{extra}




%%% Local Variables:
%%% mode: latex
%%% TeX-master: "../eai"
%%% End:
