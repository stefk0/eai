\subsection{Примерни задачи}

\begin{extra}
  \begin{example}
  Нека да видим защо езикът $L = \{a^mb^nc^k\mid m+n \geq k\}$ е безконтекстен.
  Да разгледаме граматиката $G$ с правила
  \begin{align*}
    S& \rightarrow aSc\ |\ aS\ |\ B\\
    B& \rightarrow bBc\ |\  bB\ |\ \varepsilon.
  \end{align*}

  От \Proposition{grammar:yield-approximation} имаме, че:
  \begin{align*}
    \L^0_G(S) & = \emptyset\\
    \L^{\ell+1}_G(S) & = \{a\} \cdot \L^\ell_G(S) \cdot \{c\} \cup \{a\}\cdot \L^\ell_G(S) \cup \L^\ell_G(B)\\
    \L^0_G(B) & = \emptyset\\
    \L^{\ell+1}_G(B) & = \{b\} \cdot \L^\ell_G(B) \cdot \{c\} \cup \{b\} \cdot \L^\ell_G(B) \cup \{\varepsilon\}.
  \end{align*}
  Да предположим, че за произволно естествено число $\ell$ е изпълнено следното:
  \mynote{Тези две свойства ще бъдат нашето \IndHyp. Очевидно е, че те са изпълнени за $\ell = 0$.}
  \begin{align}
    \L^\ell_G(S) & \subseteq \{a^nb^mc^k \mid n+m \geq k\} \\
    \L^\ell_G(B)  & \subseteq \{ b^mc^k \mid m \geq k\}. 
  \end{align}
  Ще докажем, че
  \begin{align*}
    \L^{\ell+1}_G(S) & \subseteq \{a^nb^mc^k \mid n+m \geq k\}\\
    \L^{\ell+1}_G(B)  & \subseteq \{ b^mc^k \mid m \geq k\}.
  \end{align*}
  За първото включване, да разгледаме произволна дума $\alpha \in \L^{\ell+1}_G(S)$. Имаме три случая:
  \begin{itemize}
  \item
    Ако $\alpha \in \L^\ell_G(B)$, то от \IndHyp имаме, че
    \[\alpha \in \{b^mc^k \mid m \geq k\} \subseteq \{a^nb^mc^k \mid n+m \geq k\}.\]
  \item
    Ако $\alpha \in \{a\} \cdot \L^{\ell}_G(S)$, то от \IndHyp имаме, че
    \[\alpha \in \{a^{n+1}b^mc^k \mid n+m \geq k\} \subseteq \{a^nb^mc^k \mid n+m \geq k\}.\]
  \item
    Ако $\alpha \in \{a\} \cdot \L^{\ell}_G(S) \cdot \{c\}$, то от \IndHyp имаме, че
    \[\alpha \in \{a^{n+1}b^mc^{k+1} \mid n+m \geq k\} \subseteq \{a^nb^mc^k \mid n+m \geq k\}.\]
  \end{itemize}

  За второто включване, нека $\alpha \in \L^{\ell+1}_G(B)$. Имаме три случая за думата $\alpha$.
  \begin{itemize}
  \item
    Нека $\alpha \in \{b\} \cdot \L^\ell_G(B) \cdot \{c\}$. Тогава от \IndHyp имаме, че:
    \[\alpha \in \{b^{m+1}c^{k+1} \mid m \geq k\} \subseteq \{b^mc^k \mid  m \geq k\}.\]
  \item
    Нека $\alpha \in \{b\} \cdot \L^\ell_G(B)$. Тогава от \IndHyp имаме, че:
    \[\alpha \in \{b^{m+1}c^{k} \mid m \geq k\} \subseteq \{b^mc^k \mid m \geq k\}.\]
  \item
    Нека $\alpha \in \{\varepsilon\}$. Тогава е ясно, че имаме $\alpha \in \{b^mc^k \mid m \geq k\}$.
  \end{itemize}  

  \mynote{Така доказахме \emph{коректност} на граматиката спрямо езика $L$.}
  Заключаваме, че
  \begin{align*}
    \L_G(S) & = \bigcup_\ell\L^\ell_G(S) \subseteq \{a^nb^mc^k \mid n+m \geq k\}\\
    \L_G(B) & = \bigcup_\ell\L^\ell_G(B) \subseteq \{a^nb^mc^k \mid n+m \geq k\}.
  \end{align*}

  \mynote{Сега ще докажем \emph{пълнота} на граматиката спрямо езика $L$.}
  Сега трябва да докажем обратните включвания, а именно:
  \begin{align}
    & \{a^nb^mc^k \mid n+m \geq k\} \subseteq \L_G(S) \label{eq:anbmck:S}\\
    & \{b^mc^k \mid m \geq k\} \subseteq \L_G(B). \label{eq:anbmck:B}
  \end{align}

  Трябва да започнем първо със \Property{eq:anbmck:B}.
  Да разгледаме произволна дума $\alpha = b^mc^k$. Трябва да докажем, че $\alpha \in \L_G(B)$.
  Ще направим това с индукция по $m$.
  \begin{itemize}
  \item
    Нека $m = 0$. Това означава, че $\alpha = \varepsilon$. Ясно е, че $\varepsilon \in \L_G(B)$.
  \item
    Нека $m > 0$. Тук имаме два подслучая.
    \begin{itemize}
    \item
      Нека $m = k$. Тогава $\alpha = b \gamma c$ и имаме, че $\gamma = b^{m-1}c^{k-1}$.
      Можем да приложим \IndHyp за $\gamma$ и следователно $\gamma \in \L_G(B)$.
      Получаваме, че $\alpha \in \{b\} \cdot \L_G(B) \cdot \{c\} \subseteq \L_G(B)$.
    \item
      Нека $m > k$. Тогава $\alpha = b \gamma$ и имаме, че $\gamma = b^{m-1}c^k$.
      Можем да приложим \IndHyp за $\gamma$ и следователно $\gamma \in \L_G(B)$.
      Получаваме, че $\alpha \in \{b\} \cdot \L_G(B)\subseteq \L_G(B)$.
    \end{itemize}
  \end{itemize}
  Сега преминаваме към \Property{eq:anbmck:S}.
  Да разгледаме произволна дума $\alpha = a^nb^mc^k$. Трябва да докажем, че $\alpha \in \L_G(S)$.
  Ще направим това с индукция по $n$.
  \begin{itemize}
  \item
    Нека $n = 0$. Тогава $\alpha = b^mc^k$ и $m \geq k$.
    От \Property{eq:anbmck:B} следва, че $\alpha \in \L_G(B) \subseteq \L_G(S)$.
  \item
    Нека $n > 0$. Имаме два подслучая.
    \begin{itemize}
    \item
      Нека $n + m = k$. Тогава $\alpha = a\gamma c$ и $\gamma = a^{n-1}b^mc^{k-1}$.
      Можем да приложим \IndHyp за $\gamma$ и следователно $\gamma \in \L_G(S)$.
      Получаваме, че $\alpha \in \{a\} \cdot \L_G(S) \cdot \{c\} \subseteq \L_G(S)$.
    \item
      Нека $n + m > k$. Тогава $\alpha = a \gamma$ и $\gamma = a^{n-1}b^m c^k$.
      Можем да приложим \IndHyp за $\gamma$ и следователно $\gamma \in \L_G(S)$.
      Получаваме, че $\alpha \in \{a\} \cdot \L_G(S) \subseteq \L_G(S)$.
    \end{itemize}
  \end{itemize}

  
  
  % \begin{itemize}
  % \item
  %   Първо ще докажем, че $\{a^nb^mc^k \mid \ell > n+m \geq k\} \subseteq \L^{\ell+1}_G(S)$.
  %   Да разгледаме думата $\alpha = a^nb^mc^k$ и $\ell > n+m \geq k$.
  %   \begin{itemize}
  %   \item
  %     Ако $\ell - 1 > n+m$, то от \IndHyp имаме, че $\alpha \in \L^{\ell}_G(S) \subseteq \L^{\ell+1}_G(S)$.
  %   \item
  %     Нека сега $\ell-1 = n+m \geq k$. Имаме два случая.
  %     \begin{itemize}
  %     \item
  %       Ако $n = 0$, то $\alpha = b^mc^k$ и $\ell-1 = m \geq k$. Тогава от \IndHyp имаме, че $\alpha \in \L^{\ell}_G(B) \subseteq \L^{\ell+1}_G(S)$.
  %     \item
  %       Нека сега $n > 0$. Тогава имаме следните два случая:
  %       \begin{itemize}
  %       \item 
  %         Ако $n + m > k$, то е ясно, че $(n-1) + m \geq k$. От \IndHyp имаме, че $\alpha \in \{a\} \cdot \L^\ell_G(S) \subseteq \L^{\ell+1}_G(S)$.
  %       \item
  %         \mynote{Понеже $n > 0$, то и $k > 0$.}
  %         Ако $n + m = k$, то е ясно, че $(n-1) + m = k-1$. От \IndHyp имаме, че $\alpha \in \{a\} \cdot \L^\ell_G(S) \cdot \{c\} \subseteq \L^{\ell+1}_G(S)$.
  %       \end{itemize}
  %     \end{itemize}
  %   \end{itemize}
  % \item
  %   Сега да разгледаме включването $\L^{\ell+1}_G(S) \subseteq \{a^nb^mc^k \mid \ell > n+m \geq k\}$.
  %   Да разгледаме произволна дума $\alpha \in \L^{\ell+1}_G(S)$. Имаме три случая:
  %   \begin{itemize}
  %   \item
  %     Ако $\alpha \in \L^\ell_G(B)$, то от \IndHyp имаме, че
  %     \[\alpha \in \{b^mc^k \mid \ell > m \geq k\} \subseteq \{a^nb^mc^k \mid \ell > n+m \geq k\}.\]
  %   \item
  %     Ако $\alpha \in \{a\} \cdot \L^{\ell}_G(S)$, то от \IndHyp имаме, че
  %     \[\alpha \in \{a^{n+1}b^mc^k \mid \ell - 1 > n+m \geq k\} \subseteq \{a^nb^mc^k \mid \ell > n+m \geq k\}.\]
  %   \item
  %     Ако $\alpha \in \{a\} \cdot \L^{\ell}_G(S) \cdot \{c\}$, то от \IndHyp имаме, че
  %     \[\alpha \in \{a^{n+1}b^mc^{k+1} \mid \ell - 1 > n+m \geq k\} \subseteq \{a^nb^mc^k \mid \ell > n+m \geq k\}.\]
  %   \end{itemize}
  % \end{itemize}

  % Сега преминаваме към второто равенство.
  % \begin{itemize}
  % \item
  %   Първо ще докажем включването $\{b^mc^k \mid \ell \geq m \geq k\} \subseteq \L^{\ell+1}_G(B)$.
  %   Да разгледаме произволна дума $\alpha = b^mc^k$, за която $\ell \geq m \geq k$.
  %   \begin{itemize}
  %   \item
  %     Ако $m = 0$, то $\alpha = \varepsilon$ и е ясно, че $\alpha \in \L^{1}_G(B) \subseteq \L^{\ell+1}_G(B)$.
  %   \item
  %     Ако $\ell > m$, то от \IndHyp имаме, че $\alpha \in \L^{\ell}_G(B) \subseteq \L^{\ell+1}_G(B)$.
  %   \item
  %     Нека сега $\ell = m \geq k$.
  %     \begin{itemize}
  %     \item
  %       Ако $\ell = m = k$, то от \IndHyp имаме, че $\alpha \in \{a\} \cdot \L^\ell_G(B) \cdot \{c\} \subseteq \L^{\ell+1}_G(B)$.
  %     \item
  %       Ако $\ell = m > k$, то от \IndHyp имаме, че $\alpha \in \{a\} \cdot \L^\ell_G(B) \subseteq \L^{\ell+1}_G(B)$.
  %     \end{itemize}
  %   \end{itemize}
  % \item
  %   Сега да разгледаме обратното включване $\L^{\ell+1}_G(B) \subseteq \{b^mc^k \mid \ell \geq m \geq k\}$.
  %   Нека $\alpha \in \L^{\ell+1}_G(B)$. Имаме три случая за думата $\alpha$.
  %   \begin{itemize}
  %   \item
  %     Нека $\alpha \in \{b\} \cdot \L^\ell_G(B) \cdot \{c\}$. Тогава от \IndHyp имаме, че:
  %     \[\alpha \in \{b^{m+1}c^{k+1} \mid \ell-1\geq m \geq k\} \subseteq \{b^mc^k \mid \ell \geq m \geq k\}.\]
  %   \item
  %     Нека $\alpha \in \{b\} \cdot \L^\ell_G(B)$. Тогава от \IndHyp имаме, че:
  %     \[\alpha \in \{b^{m+1}c^{k} \mid \ell-1\geq m \geq k\} \subseteq \{b^mc^k \mid \ell \geq m \geq k\}.\]
  %   \item
  %     Нека $\alpha \in \{\varepsilon\}$. Тогава е ясно, че имаме $\alpha \in \{b^mc^l \mid \ell \geq m \geq k\}$.
  %   \end{itemize}
  % \end{itemize}
\end{example}
\end{extra}

\newpage
\begin{example}
  Езикът 
  \[L = \{a^nb^mc^kd^\ell \mid n+k = m + \ell\}\]
  е безконтекстен.
\end{example}
\begin{hint}
  Да разгледаме произволна дума от вида $\omega = a^n b^m c^k d^\ell$.
  Имаме два случая.
  \begin{itemize}
  \item
    Ако $n \leq \ell$, то
    \[\omega \in L \iff k = m + (\ell- n).\]
  \item
    Ако $n > \ell$, то 
    \[\omega \in L \iff (n-\ell) + k = m.\]
  \end{itemize}
  Това наблюдение ни подсказва, че трябва да разгледаме следните езици:
  \begin{align*}
    & L_1 = \{a^nb^mc^k \mid m = n+k\},\\
    & L_2 = \{b^mc^kd^\ell \mid k = m+\ell\}.
  \end{align*}
  Така получаваме, че
  \[L = \{a^n \cdot \omega \cdot d^n \mid n\in\Nat\ \&\ \omega \in L_1 \cup L_2\}.\]
  $L_1$ е безконтекстен език, защото може да се опише с безконтекстната граматика $G_1$ със следните правила:
  \[S_1 \to AC,\quad  A \to aAb\ |\ \varepsilon,\quad C \to bCc\ |\ \varepsilon.\]
  $L_1$ също е безконтекстен език, защото може да се опише с безконтекстната граматика $G_2$ със следните правила:
  \[S_2 \to BD,\quad B \to bBc\ |\ \varepsilon,\quad D \to cCd\ |\ \varepsilon.\]
  Тогава безконтекстната граматика $G$ за $L$ 
  съдържа правилата на граматиките $G_1$ и $G_2$, а също и правилата
  \[S \to aSd\ |\ S_1\ |\ S_2.\]
\end{hint}

\begin{example}
  \label{prob:equal-but-different}
  \mynote{Ние вече знаем, че този език не е регулярен}
  Нека да видим защото езикът
  \[L = \{\alpha\beta \in \{a,b\}^\star \mid\ |\alpha| = |\beta|\ \&\ \alpha \neq \beta\}\]
  е безконтекстен.

  Да разгледаме една произволна дума $\omega$, където $\omega = \alpha\beta$, $|\alpha| = |\beta|$ и $\alpha \neq \beta$.
  Знаем, че същестува индекс $i < |\alpha|$, такъв че думата $\omega$ може да се представи така:
  \[\omega = \alpha\texttt{[:i]} \cdot \alpha\texttt{[i]} \cdot \alpha\texttt{[i+1:]} \cdot \beta\texttt{[:i]} \cdot \beta\texttt{[i]} \cdot \beta\texttt{[i+1:]},\]
  където $\alpha\texttt{[i]} \neq \beta\texttt{[i]}$.

  Нека $n = |\alpha| = |\beta|$ и да представим $n$ като $n = i+1+k$. Имаме два случая.
  \begin{itemize}
  \item
    Ако $k \geq i$, то можем да преставим $\omega$ по следния начин:
    \[\omega = \underbrace{\alpha\texttt{[:i]}}_{\text{дълж. }i} \cdot \alpha\texttt{[i]} \cdot \underbrace{\alpha\texttt{[i+1:2i+1]}}_{\text{дълж. }i} \cdot \underbrace{\alpha\texttt{[2i+1:]} \cdot \beta\texttt{[:i]}}_{\text{дълж. }k} \cdot \beta\texttt{[i]} \cdot \underbrace{\beta\texttt{[i+1:]}}_{\text{дълж. }k}.\]
  \item
    Нека $k < i$, то можем да преставим $\omega$ по следния начин:
    \[\omega = \underbrace{\alpha\texttt{[:i]}}_{\text{дълж. }i} \cdot \alpha\texttt{[i]} \cdot \underbrace{\alpha\texttt{[i+1:]} \cdot \beta\texttt{[:i-k]}}_{\text{дълж. }i} \cdot \underbrace{\beta\texttt{[i-k:i]}}_{\text{дълж. }k} \cdot \beta\texttt{[i]} \cdot \underbrace{\beta\texttt{[i+1:]}}_{\text{дълж. }k}.\]
  \end{itemize}
  От тези представяния на $\omega$ е ясно, че можем да изразим езика $L$ по следния начин:
  \[L = L_a \cdot L_b \cup L_b \cdot L_a,\]
  където:
  \begin{align*}
    & L_a = \{\alpha a \beta \mid \alpha,\beta \in \{a,b\}^\star\ \&\ |\alpha| = |\beta|\}\\
    & L_b = \{\alpha b \beta \mid \alpha,\beta \in \{a,b\}^\star\ \&\ |\alpha| = |\beta|\}.
  \end{align*}
  Сега разгледайте безконтекстната граматика $G$ със следните правила:ю
  \begin{align*}
    & S \to AB\ |\ BA\\
    & A \to XAX\ |\ a\\
    & B \to XBX\ |\ b\\
    & X \to a\ |\ b.
  \end{align*}
  Лесно се съобразява, че:
  \begin{align*}
    & L_a = \L_G(A)\\
    & L_b = \L_G(B)\\
    & L = \L_G(S).
  \end{align*}
\end{example}

\begin{problem}
 Докажете, че езикът
 \[L = \{\alpha \sharp \beta \mid \alpha,\beta \in \{a,b\}^\star\ \&\ \alpha \neq \beta \}\]
 е безконтекстен.
\end{problem}
\begin{hint}
  Разгледайте граматиката:
  \begin{align*}
    & S \to AaR\ |\ BbR\ |\ E\\
    & A \to XAX\ |\ bR\sharp\\
    & B \to XBX\ |\ aR\sharp\\
    & E \to XEX\ |\ XR\sharp\ |\ \sharp XR\\
    & R \to XR\ |\ \varepsilon\\
    & X \to a\ |\ b.
  \end{align*}
  Имаме, че за произволни думи $\alpha,\beta,\gamma,\delta \in \{a,b\}^\star$,
  \begin{align*}
    & S \to^\star \alpha b \gamma \sharp \beta a \delta\ \&\ |\alpha| = |\beta|,\\
    & S \to^\star \alpha a \gamma \sharp \beta b \delta\ \&\ |\alpha| = |\beta|, \text{ или}\\
    & S \to^\star \alpha \sharp \beta\ \&\ |\alpha| \neq |\beta|.
  \end{align*}      
\end{hint}






\begin{problem}
  Докажете, че езикът 
  \[L = \{a^nb^mc^kd^\ell \mid n+k \geq m + \ell\}\]
  е безконтекстен.
\end{problem}

\begin{problem}
  \mynote{
    $S \to aS \mid aSc \mid aB \mid bB$\\
    $B \to bB \mid bBc \mid \varepsilon$
  }
  Докажете, че езикът 
  \[L = \{a^mb^nc^k\mid m+n \geq k + 1\}\]
  е безконтекстен.  
\end{problem}

\begin{problem}
  Докажете, че езикът
  \[L = \{\alpha \sharp \beta \mid \alpha,\beta \in \{a,b\}^\star\ \&\ |\alpha| \neq |\beta| \}\]
  е безконтекстен.
\end{problem}

\begin{problem}
  Да разгледаме граматиката $G$ с правила
  \[S \to AA\ |\ B,\ A \to B\ |\ bb,\ B \to aa\ |\ aB.\]
  Да се намери езика на тази граматика и да се докаже, че граматиката разпознава точно този език.
\end{problem}

%%% Local Variables:
%%% mode: latex
%%% TeX-master: "../eai"
%%% End:
