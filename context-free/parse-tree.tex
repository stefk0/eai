\section{Дървета на извод}
\index{дърво}
\index{дърво на извод}

\begin{itemize}
\item
  За всеки две думи $\alpha$ и $\beta$, с $\alpha \preceq \beta$ ще означаваме, че $\alpha$ е префикс на $\beta$.
\item
  За две думи $\alpha$ и $\beta$ ще казваме, че $\alpha$ е лексикографски по-малка от $\beta$, което ще означаваме като $\alpha <_{\texttt{lex}} \beta$, ако
  \[(\exists i < \min\{|\alpha|,|\beta|\})[\ (\forall j < i)[\ \alpha[j] = \beta[j]\ ]\ \&\ \alpha[i] \neq \beta[i]\ ].\]
\item
  Непразното множество $T \subseteq \{0,1,\dots,b-1\}^\star$ се нарича {\bf дърво},
  ако $T$ е затворено относно префикси, т.е.
  \[(\forall \alpha, \beta)[\ \alpha \in T\ \&\ \beta \prec \alpha\ \implies\ \beta \in T\ ].\]
\item
  Нека да въведем следните означения:
  \begin{align*}
    & \texttt{height}(T) \df \max\{\ \abs{\alpha}\ \mid\ \alpha \in T\ \}\\
    & \texttt{ext}_T(\alpha) \df \{ \alpha i \mid \alpha i \in T\}\\
    & T_\alpha \df \{\beta \mid \alpha\beta \in T\}\\
    & \texttt{front}(T) \df \{ \alpha \in T \mid \texttt{ext}_T(\alpha) = \emptyset \}.
  \end{align*}
\item
  С всяко дърво $T$ ще асоциираме функцията $f: T \to V \cup \Sigma \cup \{\varepsilon\}$.
  Нека положим $X_\alpha \df f(\alpha)$.
\item
  За всяка такава функция $f$ и $\alpha \in T$, дефинираме $f_\alpha:T_\alpha \to V \cup \Sigma \cup \{\varepsilon\}$ като
  \[f_\alpha(\beta) \df f(\alpha\beta).\]
\item
  Двойката $P = (T,f)$ се нарича {\bf дърво на извод} съвместимо с $G$, ако са изпълнени свойствата:
  \begin{itemize}
  \item
    $T$ е крайно.
  \item
    Ако $\alpha i \in T$, то $\alpha j \in T$ за всяко $j < i$.
  \item
    \marginpar{Тук $\alpha_i = \alpha i$.}
    Ако $\alpha \in T$ и $\texttt{ext}_T(\alpha) \neq \emptyset$, то $X_\alpha \in V$.
    Освен това, ако $\alpha_0,\dots,\alpha_k$ са всички думи от множеството $\texttt{ext}_T(\alpha)$
    подредени във възходящ ред относно лексикографската наредба, то имаме, че:
    \[X_\alpha \to_G X_{\alpha_0} X_{\alpha_1} \cdots X_{\alpha_k}.\] 
  \end{itemize}
\item
  Нека $\alpha_0, \alpha_1,\dots,\alpha_k$ са всички думи от множеството $\texttt{front}(T)$
  подредени във възходящ ред относно лексикографската наредба. Тогава 
  \[\texttt{yield}(P) \df X_{\alpha_0} X_{\alpha_1}\cdots X_{\alpha_k}.\]
\item
  Нека $P = (T,f)$ и $\alpha \in T$. Тогава
  \[P_\alpha \df (T_\alpha, f_\alpha).\]
\end{itemize}

\begin{problem}
  Докажете, че $T = \texttt{Pref}(\texttt{front}(T))$.
\end{problem}

\begin{lemma}
  Нека $T \subseteq \{0,\dots,b-1\}^\star$ е крайно дърво. Тогава
  \[ |\texttt{front}(T)| \leq b^{\texttt{height}(T)}.\]
\end{lemma}
\begin{proof}
  Индукция по $\texttt{height}(T)$.
  \begin{itemize}
  \item
    Нека $\texttt{height}(T) = 0$. Тогава е ясно, че $|\texttt{front}(T)| = |\{\varepsilon\}| = 1 \leq b^0$.
  \item
    Нека $\texttt{height}(T) > 0$.
    За всяко $i \in T$ е ясно, че $\texttt{height}(T_i) < \texttt{height}(T)$ и $\texttt{front}(T) = \bigcup_i \texttt{front}(T_i)$.
    Тогава
    \begin{align*}
      |\texttt{front}(T)| & = \sum_{i \in T}|\texttt{front}(T_i)| \\
                          & \leq \sum_{i\in T}b^{\texttt{height}(T_i)} & \comment\text{от И.П.}\\
                          & \leq \sum_{i < b}b^{\texttt{height}(T_i)} \\
                          & \leq \sum_{i < b}b^{\texttt{height}(T)-1} & \comment \texttt{height}(T_i) \leq \texttt{height}(T)-1 \\
                          & = b^{\texttt{height}(T)}.
    \end{align*}
  \end{itemize}
\end{proof}

\begin{framed}
  \begin{cor}
    \label{cor:tree:upper-bound}
    Нека $P = (T,f)$ е дърво на извод съвместимо с $G$. Тогава
    \[|\texttt{yield}(T)| \leq b^{\texttt{height}(T)}.\]
  \end{cor}  
\end{framed}
\begin{proof}
  Следва директно от горното твърдение след като съобразим, че
  \[|\texttt{yield}(P)| \leq |\texttt{front}(T)|.\]
\end{proof}

\begin{framed}
  \begin{lemma}
    Нека $X \to^\star_G \beta$.
    Тогава съществува дърво на извод $P = (T,f)$ с корен $X$ за думата $\beta$ в $G$.
  \end{lemma}  
\end{framed}
\begin{proof}
  Индукция по дължината на извода $X \stackrel{l}{\to}_G \beta$.
  \begin{itemize}
  \item
    $l = 0$, т.е. $X \to^0_G X$.
    Тогава $T = \{\varepsilon\}$ и $f(\varepsilon) = X$.
    Ясно е, че $\texttt{ext}_T(\varepsilon) = \emptyset$ и следователно $\texttt{yield}(P) = X$.
  \item
    Нека $l > 0$ и $X \stackrel{l}{\to}_G \beta$.
    Да разгледаме извода
    \[X \to_G X_0X_1\cdots X_k \stackrel{l-1}{\to}_G \beta.\]
    От \Prop{grammar:divide} знаем, че съществува разбиване на $\beta$ на $k+1$ части, така че:
    \begin{itemize}
    \item
      $\beta = \beta_0 \cdots b_{k}$;
    \item
      $X_i \stackrel{l_i}{\to}_G \beta_i$;
    \item
      $l-1 = \sum^k_{j=1} l_j$.
    \end{itemize}
    От И.П. имаме, че същствуват дървета на извод $P^{i} = (T^i,f^i)$ съвместими с $G$, такива че:
    \begin{itemize}
    \item
      $f^i(\varepsilon) = X_i$;
    \item
      $\texttt{yield}(P^i) = \beta_i$.
    \end{itemize}
    Тогава дефинираме $P = (T,f)$ по следния начин:
    \begin{itemize}
    \item
      $T \df \{ i\alpha \mid \alpha \in T^i\ \&\ i \leq k\}$;
    \item
      $f(\varepsilon) \df X$;
    \item
      $f(i\alpha) \df f^i(\alpha)$.
    \end{itemize}
  \end{itemize}
\end{proof}

\begin{framed}
  \begin{lemma}
    Нека $P = (T,f)$ е дърво на извод за думата $\beta$ в $G$.
    Тогава $X_\varepsilon \to^\star_G \beta$.
  \end{lemma}  
\end{framed}
\begin{proof}
  Индукция по $\texttt{height}(T)$.
  \begin{itemize}
  \item
    Нека $\texttt{height}(T) = 0$. Това означава, че $T = \{\varepsilon\}$ и $\texttt{yield}(P) = X_\varepsilon$.
    Ясно е, че $X_\varepsilon \to^\star_G X_\varepsilon$ и $0 < b^0$.
  \item
    Нека $\texttt{height}(T) > 0$ и $\beta = \texttt{yield}(T)$.
    Нека $|\texttt{ext}_P(\varepsilon)| = k$.
    Понеже $P$ е съвместимо с $G$, то
    \[X_\varepsilon \to_G X_{0}\cdots X_{k-1}.\]
    Лесно се съобразява, че $P_i \df (T_i,f_i)$ са дървета на извод съвместими с $G$ и
    $\texttt{height}(T_i) < \texttt{height}(T)$. Нека $\beta_i = \texttt{yield}(T_i)$.
    Също така е ясно, че $\beta = \beta_0 \cdots \beta_{k-1}$.
    Тогава от И.П. получаваме, че за $i < k$,
    \[X_i \stackrel{l_i}{\to}_G \beta_i.\]
    Сега прилагаме \Prop{grammar:concat} и получаваме, че
    \[X_\varepsilon \to_G X_0\cdots X_k \to^\star_G \beta_0 \cdots \beta_k.\]
    Заключаваме, че
    \[X_\varepsilon \to^\star_G \beta.\]
  \end{itemize}
\end{proof}

\begin{problem}
  Нека $P = (T,f)$ е дърво на извод за думата $\alpha$ в $G$. Докажете, че:
  \marginpar{Ясно е, че $l < b^{\texttt{height}(T)}$.}
  \[X_\varepsilon \stackrel{l}{\to}_G \alpha\text{, където }l \leq \sum_{i < \texttt{height}(T)}b^i.\]
\end{problem}

\begin{problem}
  Нека $P = (T,f)$ е дърво на извод съвместимо с $G$.
  Докажете, че ако $\alpha \preceq \beta$, то $\texttt{yield}(T_\beta)$ е инфикс на $\texttt{yield}(T_\alpha)$.
\end{problem}

\begin{problem}
  Нека $P = (T,f)$ и $P' = (T',f')$ са дървета на извод съвместими с $G$,
  като $X_\beta = X'_\varepsilon$. Дефинираме $P''(T'',f'')$ по следния начин:
  \begin{itemize}
  \item
    $T'' = \{\alpha \mid \alpha \in T\ \&\ \beta\not\prec \alpha\} \cup \beta \cdot T'$;
  \item
    $f''(\alpha) =
    \begin{cases}
      f(\alpha), & \text{ако }\beta\not\prec\alpha\\
      f'(\gamma), & \text{ако }\alpha = \beta\gamma.
    \end{cases}$
  \end{itemize}
  Докажете, че $P''$ е дърво на извод съвместимо с $G$ и ако
  $\texttt{yield}(T) = xyz$ и $y = \texttt{yield}(T_\beta)$, то
  $\texttt{yield}(T'') = xy'z$, където $y' = \texttt{yield}(T')$.  
\end{problem}



%%% Local Variables:
%%% mode: latex
%%% TeX-master: "../eai"
%%% End:
