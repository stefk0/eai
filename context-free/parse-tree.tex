\section{Дървета на извод}
\index{дърво}
\index{дърво на извод}

\begin{itemize}
\item
  Ако $\alpha$ е префикс на $\beta$, то означаваме с $\alpha \preceq \beta$.
\item
  За две думи $\alpha$ и $\beta$ ще казваме, че $\alpha$ е лексикографски по-малка от $\beta$, което ще означаваме като $\alpha <_{\texttt{lex}} \beta$, ако
  \[(\exists i < \min\{|\alpha|,|\beta|\})[\ (\forall j < i)[\ \alpha[j] = \beta[j]\ ]\ \&\ \alpha[i] \neq \beta[i]\ ].\]
\item
  Непразното множество $T \subseteq \{0,1,\dots,b-1\}^\star$ се нарича {\bf дърво},
  ако $T$ е затворено относно префикси, т.е.
  \[\alpha \in T\ \&\ \beta \prec \alpha\ \implies\ \beta \in T.\]
\item
  Нека $\texttt{height}(T) \df \max\{\abs{\alpha}\ \mid\ \alpha \in T\}$.
\item
  Нека $\texttt{ext}_T(\alpha) \df \{ \alpha i \mid \alpha i \in T\}$.
\item
  Нека $\texttt{sub}_T(\alpha) \df \{\beta \mid \alpha\beta \in T\}$.
\item
  Нека $\texttt{front}(T) \df \{ \alpha \in T \mid \texttt{ext}_T(\alpha) = \emptyset \}$.
\item
  С всяко дърво $T$ ще асоциираме функцията $f: T \to V \cup \Sigma \cup \{\varepsilon\}$.
  Нека положим $X_\alpha \df f(\alpha)$.
\item
  Двойката $P = (T,f)$ се нарича {\bf дърво на извод} съвместимо с $G$, ако са изпълнени свойствата:
  \begin{itemize}
  \item
    $T$ е крайно.
  \item
    Ако $\alpha i \in T$, то $\alpha j \in T$ за всяко $j < i$.
  \item
    \marginpar{Тук $\alpha_i = \alpha i$.}
    Ако $\alpha \in T$ и $\texttt{ext}_T(\alpha) \neq \emptyset$, то $X_\alpha \in V$.
    Освен това, ако $\alpha_0,\dots,\alpha_k$ са всички думи от множеството $\texttt{ext}_T(\alpha)$
    подредени във възходящ ред относно лексикографската наредба, то
    \[X_\alpha \to_G X_{\alpha_0} X_{\alpha_1} \cdots X_{\alpha_k}.\] 
  \end{itemize}
\item
  Нека $\alpha_0, \alpha_1,\dots,\alpha_k$ са всички думи от множеството $\texttt{front}(T)$
  подредени във възходящ ред относно лексикографската наредба. Тогава 
  \[\texttt{yield}(P) = X_{\alpha_0} X_{\alpha_1}\cdots X_{\alpha_k}.\]
\end{itemize}

\begin{lemma}
  Нека $T \subseteq \{0,\dots,b-1\}^\star$ е крайно дърво. Тогава
  \[ |\texttt{front}(T)| \leq b^{\texttt{height}(T)}.\]
\end{lemma}
\begin{proof}
  Индукция по $\texttt{height}(T)$.
  \begin{itemize}
  \item
    Нека $\texttt{height}(T) = 0$. Тогава е ясно, че $|\texttt{front}(T)| = 1$ и следователно $1 \leq b^0$.
  \item
    Нека $\texttt{height}(T) > 0$.
    За $i \in T$, нека $T_i \df \texttt{sub}_T(i)$.
    Ясно е, че $\texttt{height}(T_i) < \texttt{height}(T)$ и $\texttt{front}(T) = \bigcup_i \texttt{front}(T_i)$.
    Тогава
    \begin{align*}
      |\texttt{front}(T)| & = \sum_{i \in T}|\texttt{front}(T_i)| \\
                          & \leq \sum_{i\in T}b^{\texttt{height}(T_i)}\\
                          & \leq \sum_{i < b}b^{\texttt{height}(T_i)} \\
                          & \leq \sum_{i < b}b^{\texttt{height}(T)-1} \\
                          & = b^{\texttt{height}(T)}.
    \end{align*}
  \end{itemize}
\end{proof}

\begin{cor}
  \label{cr:tree:upper-bound}
  Нека $P = (T,f)$ е дърво на извод съвместимо с $G$. Тогава
  \[|\texttt{yield}(T)| \leq b^{\texttt{height}(T)}.\]
\end{cor}
\begin{proof}
  Следва директно от горното твърдение след като съобразим, че
  \[|\texttt{yield}(P)| \leq |\texttt{front}(T)|.\]
\end{proof}



\begin{lemma}
  Нека $X \to^\star_G \beta$, където $X \in V \cup \Sigma \cup \{\varepsilon\}$.
  Тогава съществува дърво на извод $P = (T,f)$, за което $X_\varepsilon = X$, съвместимо с $G$,
  за което $\texttt{yield}(P) = \beta$.
\end{lemma}
\begin{proof}
  Индукция по дължината на извода $X \stackrel{l}{\to}_G \beta$.
  \begin{itemize}
  \item
    $l = 0$, т.е. $X \to^0_G X$.
    Тогава $T = \{\varepsilon\}$ и $f(\varepsilon) = X$.
    Ясно е, че $\texttt{ext}_T(\varepsilon) = \emptyset$ и следователно $\texttt{yield}(P) = X$.
  \item
    Нека $l > 0$ и $X \stackrel{l}{\to}_G \beta$.
    Да разгледаме извода
    \[X \to_G X_0X_1\cdots X_k \stackrel{l-1}{\to}_G \beta.\]
    От \Prop{grammar:divide} имаме, че $X_i \stackrel{l_i}{\to}_G \beta_i$, $\beta = \beta_0\cdots b_k$ и $l-1 = \sum^k_{j=1} l_j$.
    От И.П. имаме, че същствуват дървета на извод $P_i = (T_i,f_i)$, такива че $f_i(\varepsilon) = X_i$ и $\texttt{yield}(P_i) = \beta_i$.
    
    Тогава дефинираме $P = (T,f)$ по следния начин:
    \begin{itemize}
    \item
      $T \df \{ i\alpha \mid \alpha \in T_i\ \&\ i \leq k\}$;
    \item
      $f(\varepsilon) \df A$;
    \item
      $f(i\alpha) \df f_i(\alpha)$.
    \end{itemize}
  \end{itemize}
\end{proof}

\begin{lemma}
  Нека $P = (T,f)$ е дърво на извод съвместимо с $G$ и $\texttt{yield}(P) = \beta$.
  Тогава $X_\varepsilon \to^\star_G \beta$.
\end{lemma}
\begin{proof}
  Индукция по $\texttt{height}(T)$.
  \begin{itemize}
  \item
    Нека $\texttt{height}(T) = 0$. Това означава, че $T = \{\varepsilon\}$ и $\texttt{yield}(P) = X_\varepsilon$.
    Ясно е, че $X_\varepsilon \to^0_G X_\varepsilon$.
  \item
    Нека $\texttt{height}(T) > 0$ и $\beta = \texttt{yield}(T)$.
    Нека $|\texttt{ext}_P(\varepsilon)| = k+1$.
    Понеже $P$ е съвместимо с $G$, то
    \[X_\varepsilon \to X_{0}\cdots X_k.\]
    Дефинираме дърветата $T_i \df \texttt{sub}_T(i)$ за $i \leq k$ и
    функциите $f_i(\alpha) \df f(i\alpha)$.
    Лесно се съобразява, че $P_i \df (T_i,f_i)$ са дървета на извод съвместими с $G$ и
    $\texttt{height}(T_i) < \texttt{height}(T)$. Нека $\beta_i = \texttt{yield}(T_i)$.
    Също така е ясно, че $\beta = \beta_0 \cdots \beta_k$.
    Това означава, че можем да приложим И.П. за $P_i$ и прилагаме \Prop{grammar:concat}.
    Получаваме, че
    \[X_\varepsilon \to_G X_0\cdots X_k \to^\star_G \beta_0 \cdots \beta_k.\]
    Заключаваме, че
    \[X_\varepsilon \to^\star_G \beta.\]
  \end{itemize}
\end{proof}



%%% Local Variables:
%%% mode: latex
%%% TeX-master: "../eai"
%%% End:
