\section{Дървета на извод}



\begin{itemize}
\item
  Крайното множество $P \subseteq \{0,1,\dots,b-1\}^\star$ се нарича {\bf дърво},
  ако $P$ е затворено относно префикси, т.е.
  \[\alpha \in P\ \&\ \beta \prec \alpha\ \implies\ \beta \in P.\]
\item
  Дефинираме $\texttt{height}(P) = \max\{\abs{\alpha}\ \mid\ \alpha \in P\}$.
\item
  Да дефинираме
  \[Ext(\alpha) = \{ i < b \mid \alpha i \in P\}.\]
\item
  Дървото $P$, заедно с функцията $f:P \to V \cup \Sigma$ се нарича дърво на извод съвместимо с $G$, ако:
  \begin{itemize}
  \item
    Да означим $f(\alpha) = X_\alpha \in V \cup \Sigma \cup \{\varepsilon\}$.
  \item
    Ако $\alpha i \in P$, то $\alpha j \in P$ за всяко $j < i$.
  \item
    $\alpha \in P$ и $Ext(\alpha) \neq \emptyset$, то $X_\alpha \in V$ и за $i = \max\{ j < b\mid j \in Ext(\alpha)\}$,
    \[X_\alpha \to_G X_{\alpha 0} X_{\alpha 1} \cdots X_{\alpha i}.\] 
  \end{itemize}
\item
  Нека $\alpha_0, \alpha_1,\dots,\alpha_k$ са всички думи от множеството $\{\alpha \in P \mid Ext(\alpha) = \emptyset\}$,
  подредени във възходящ ред относно лексикографската наредба. Тогава 
  $\omega(P,f) = X_{\alpha_0} X_{\alpha_1}\cdots X_{\alpha_k}$.
\end{itemize}

\begin{lemma}
  Нека $A \to^\star_G \beta$. Тогава съществува дърво на извод $(P,f)$ съвместимо с $G$,
  за което $\omega(P) = \beta$.
\end{lemma}
\begin{proof}
  Индукция по дължината на извода $A \to^l_G \beta$.
  \begin{itemize}
  \item
    $l = 0$, т.е. $A \to^0_\G A$.
    Тогава $P = \{\varepsilon\}$ и $f(\varepsilon) = A$.
    Ясно е, че $Ext(\varepsilon) = \emptyset$ и следователно $\omega(P) = A$.
  \item
    Нека $A \to^{l+1}_G \beta$.
    Да разгледаме $A \to_G X_0X_1\cdots X_k \to^l_G \beta$.
    Имаме, че $X_i \to^{l_i}_G \beta_i$, $\beta = \beta_0\cdots b_k$ и $l = l_0+l_1+\cdots + l_k$.
    Тогава $P = \{ i\alpha \mid \alpha \in P_i\ \&\ i \leq k\}$,
    
  \end{itemize}
\end{proof}

%%% Local Variables:
%%% mode: latex
%%% TeX-master: "../eai"
%%% End:
