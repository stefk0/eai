\section{Регулярни граматики}
\index{граматика!регулярна}
\index{граматика!тип 3}

Сега ще разгледаме граматики с такъв вид правила,
които пораждат точно регулярните (или еквивалентно автоматни) езици.
\mynote{Също така се наричат граматики от тип 3 в йерархията на Чомски \cite[стр. 217]{hopcroft1}. Този вид граматики понякога се нарича и дясно-регулярна граматика.}
Граматиката $G = (V, \Sigma, R, S)$ се нарича {\bf регулярна граматика},
ако всички правила са от вида 
\begin{align*}
  & A \to aB,\\
  % & A \to a,\\
  & A \to \varepsilon,
\end{align*}
за произволни $A, B \in V$ и $a \in \Sigma$.

\begin{lemma}
  За всяка регулярна граматика $G$ съществува НКА $\N$, такъв че $\L(G) = \L(\N)$.
\end{lemma}
\begin{hint}
  Нека $G = \CFG$ и $V = \{A_0,\dots,A_k\}$, където $S = A_0$. Тогава дефинираме $\N$ по следния начин:
  \begin{itemize}
  \item
    $Q \df \{q_0,\dots,q_k\}$;
  \item
    $Q_{\texttt{start}} \df \{q_0\}$;
  \item
    $F \df \{q_i \mid A_i \to \varepsilon\}$;
  \item
    Релацията на преходите $\Delta$ е дефинирана по следния начин:
    \[\Delta(q_i,a) \df \{ q_j\ \mid\ A_i \to aA_j \text{ е правило в граматиката}\}.\]
  \end{itemize}
  Докажете, че $\L(\N) = \L(G)$.
\end{hint}

\begin{lemma}
  \mynote{Ясно е, че и двете твърдения може да се формулират и за детерминирани автомати.}
  За всеки ДКА $\A$ съществува регулярна граматика $G$, такава че $\L(\A)~=~\L(G)$.
\end{lemma}
\begin{hint}
  Нека $\A = \FA$ и $Q = \{q_0,\dots,q_k\}$, където $\qstart = q_0$. Тогава дефинираме $G = \CFG$ по следния начин:
  \begin{itemize}
  \item 
    $V \df \{A_0,\dots,A_k\}$;
  \item
    $S \df A_0$;
  \item
    $A_i \to aA_j\ \dff\ \delta(q_i,a) = q_j$;
  \item
    $A_{i} \to \varepsilon\ \dff\ q_{i} \in F$.
  \end{itemize}
  Докажете, че $\L(\A) = \L(G)$.
\end{hint}

\begin{framed}
  \begin{theorem}
    Един език е регулярен точно тогава, когато се поражда от регулярна граматика.
  \end{theorem}
\end{framed}

\subsection{Допълнителни задачи}

\begin{extra}

\begin{problem}
  Граматиката $G = (V, \Sigma, R, S)$ се нарича {\bf обобщено дясно-регулярна},
  ако всички правила са от вида 
  \begin{align*}
    & A \to \omega B,\\
    & A \to \omega
  \end{align*}
  за произволни $A, B \in V$ и $\omega \in \Sigma^\star$.
  Докажете, че един език $L$ е автоматен точно тогава, когато $L$ може да се опише с разширена дясно-регулярна граматика.
\end{problem}

\begin{problem}
  Граматиката $G = (V, \Sigma, R, S)$ се нарича {\bf обобщено ляво-регулярна},
  ако всички правила са от вида 
  \begin{align*}
    & A \to B\omega,\\
    & A \to \omega
  \end{align*}
  за произволни $A, B \in V$ и $\omega \in \Sigma^\star$.
  Докажете, че един език $L$ е автоматен точно тогава, когато $L$ може да се опише с разширена ляво-регулярна граматика.
\end{problem}

\end{extra}

%%% Local Variables:
%%% mode: latex
%%% TeX-master: "../eai"
%%% End:
