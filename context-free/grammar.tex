\section{Безконтекстни граматики}

\index{граматика!безконтекстна}
\mynote{В \cite{papadimitriou} дефиницията е различна. Там $\Sigma \subseteq V$. На англ. {\em context-free grammar}. Други срещани наименования на български са
  {\em контекстно-свободна}, {\em контекстно-независима}. Тук всички правила са от вида $A \to \alpha$, където $\alpha \in (V\cup\Sigma)^\star$.}

В Раздел \ref{sect:regular-grammar} въведохме понятието неограничена граматика. След това видяхме как можем да опишем регулярните езици
със специален вид граматики, които нарекохме регулярни граматики.
Сега ще разгледаме още един вид граматики, които описват по-широк клас от езици.

Една граматика $G = (V, \Sigma, R, S)$ се нарича {\bf безконтекстна}, ако 
имаме ограничението, че $R \subseteq V\times (V\cup\Sigma)^\star$.

Да напомним дефиницията на релацията $\alpha \derive{\ell}_G \beta$ за произволни думи $\alpha,\beta \in (V\cup\Sigma)^\star$.

\begin{prooftree}
  \AxiomC{}
  \UnaryInfC{$\alpha \derive{0}_G \alpha$}
\end{prooftree}

\begin{prooftree}
  \AxiomC{$A \to_G \gamma$}
  \AxiomC{$\gamma \derive{\ell}_G \beta$}
  \BinaryInfC{$A \derive{\ell+1}_G \beta$}
\end{prooftree}

\begin{prooftree}
  \AxiomC{$\alpha_1 \derive{\ell_1}_G \beta_1$}
  \AxiomC{$\alpha_2 \derive{\ell_2}_G \beta_2$}
  \BinaryInfC{$\alpha_1\cdot\alpha_2 \derive{\ell_1+\ell_2}_G \beta_1\cdot \beta_2$}
\end{prooftree}

\mynote{В частност имаме, че:
\begin{prooftree}
  \AxiomC{$A \to_G \beta$}
  \UnaryInfC{$A \derive{1}_G \beta$}
\end{prooftree}
}

Нека $\derive{\star}_G$ е рефлексивното и транзитивно затваряне на релацията $\derive{1}_G$. С други думи,
\[ \alpha \derive{\star}_G \beta\ \iff\ (\exists \ell\in\Nat)[\ \alpha \derive{\ell}_G \beta\ ].\]

\index{език!безконтекстен}
Един език $L$ се нарича {\bf безконтекстен}, ако съществува безконтекстна граматика $G$, за която 
$L = \L(G) = \{\omega \in \Sigma^\star \mid S \derive{\star} \omega\}$.

\begin{remark}
  Очевидно е, че всяка регулярна граматика е безконтекстна. Следователно, 
  {\em всеки регулярен език е безконтекстен.}
\end{remark}

\ExtraMaterial{
  % \begin{multicols}{2}
\begin{example}
  Да разгледаме безконтекстната граматика $G$, която има следните правила:
  \begin{align*}
    S \to AS\ |\ \varepsilon\\
    A \to aAb\ |\ ab.
  \end{align*}
  Да видим защо думата $aabbab \in \L(G)$. Ако следваме формално правилата за извод, получаваме следното:

  \begin{prooftree}
    \AxiomC{$S \to_G AS$}
    \AxiomC{$A \derive{0}_G A$}
    \AxiomC{$S \to_G AS$}
    \AxiomC{$S \to_G \varepsilon$}
    \AxiomC{$\varepsilon \derive{0}_G \varepsilon$}
    \BinaryInfC{$S \derive{1}_G \varepsilon$}
    \BinaryInfC{$S \derive{2}_G A$}
    \BinaryInfC{$AS \derive{2}_G AA$}
    \BinaryInfC{$S \derive{3}_G AA$}
  \end{prooftree}

  Освен това имаме и следния формален извод:
  \begin{prooftree}
    \AxiomC{$A \to_G aAb$}
    \AxiomC{$a \derive{0}_G a$}
    \AxiomC{$A \to_G ab$}
    \AxiomC{$b \derive{0}_G b$}
    \BinaryInfC{$Ab \derive{1}_G abb$}
    \BinaryInfC{$aAb \derive{1}_G aabb$}
    \BinaryInfC{$A \derive{2}_G aabb$}
  \end{prooftree}

  Аналогично,
  \begin{prooftree}
    \AxiomC{$A \to_G ab$}
    \AxiomC{$ab \derive{0}_G ab$}
    \BinaryInfC{$A \derive{1}_G ab$}
  \end{prooftree}

  Обединявайки всичко, получаваме:

  \begin{prooftree}
    \AxiomC{$S \derive{3}_G AA$}
    \AxiomC{$A \derive{2}_G aabb$}
    \BinaryInfC{$S \derive{5}_G aabbA$}
    \AxiomC{$A \derive{1}_G ab$}
    \BinaryInfC{$S \derive{6}_G aabbab$}
  \end{prooftree}

  Обърнете внимание, че можем да приложим правилата за извод в различен ред и пак да получим същия краен резултат.
  Например:

  \begin{prooftree}
    \AxiomC{$S \to_G AS$}
    \AxiomC{$A \derive{2} aabb$}
    \BinaryInfC{$S \derive{3}_G aabbS$}
    \AxiomC{$S \derive{2}_G A$}
    \BinaryInfC{$S \derive{5}_G aabbA$}
    \AxiomC{$A \derive{1}_G ab$}
    \BinaryInfC{$S \derive{6}_G aabbab$}
  \end{prooftree}
\end{example}
% \end{multicols}
}


\begin{proposition}\label{pr:grammar:derive-one-step}
  Нека $G$ е безконтекстна граматика.
  Тогава за произволни думи $\alpha,\beta \in (V\cup\Sigma)^\star$ е изпълнено, че 
  $\alpha \derive{1}_G \beta$ точно тогава, когато $\alpha = \delta A \rho$, $A \to_G \gamma$ и $\beta = \delta\gamma\rho$.
\end{proposition}

% \begin{framed}
\begin{proposition}\label{pr:grammar:divide}
  Нека $G$ е безконтекстна граматика и нека $X_1\cdots X_k \derive{\ell}_G \beta$, където $X_i \in V \cup \Sigma$.
  Тогава съществуват думи $\beta_1,\dots,\beta_k$, такива че за $i = 1,\dots, k$ е изпълнено, че
  $X_i \derive{\ell_i} \beta_i$, където $\beta = \beta_1\cdots \beta_k$ и $\ell = \sum^k_{i = 1}\ell_i$.
\end{proposition}
\begin{hint}
  Индукция по $k$.
\end{hint}
% \end{framed}
% \begin{proof}
%   Индукция по $n$.
%   \begin{itemize}
%   \item
%     Нека $n = 0$. Тогава $\beta = \alpha_1 \cdots \alpha_k$ и е ясно, че в този случай $\beta_i = \alpha_i$ и $n_i = 0$.
%   \item
%     Нека $n > 0$ и $\alpha_1\cdots \alpha_k \derive{n} \beta$. Тогава за някое $i$, $\alpha_i \to_G \alpha'_i$ и
%     като приложим \Proposition{grammar:add} получаваме, че
%     \[\alpha_1\cdots\alpha_i\cdots\alpha_k \to_G \alpha_1\cdots\alpha'_i\cdots\alpha_k.\]
%     Според дефиницията на релацията $\derive{n}$ имаме, че
%     \[\alpha_1\cdots\alpha'_i\cdots\alpha_k \stackrel{n-1}{\to}_G \beta.\]
%     От И.П. получаваме, че съществуват думи $\beta_1,\dots,\beta_k$, такива че $\beta = \beta_1 \cdots \beta_k$
%     и $\alpha_j \derive{n_j} \beta_j$ за $j \neq i$ и $\alpha'_i \derive{n'_i} \beta_i$, като
%     \[n-1 = n'_i + \sum_{j\neq i} n_j.\]
%     Понеже имаме, че $\alpha_i \to_G \alpha'_i \derive{n'_i}\beta_i$,
%     то е ясно, че за $n_i = n'_i + 1$ имаме $\alpha_i \derive{n_i} \beta_i$ и
%     \[n = \sum^{k}_{i=1} n_i.\]
%   \end{itemize}
% \end{proof}



\begin{framed}
  \begin{thm}
    Всеки регулярен език е безконтекстен.
  \end{thm}
\end{framed}
\begin{hint}
  Ще направим индукция по построението на регулярните езици.
  \begin{itemize}
  \item
    Всеки от езиците $\emptyset$, $\{\varepsilon\}$ и $\{a\}$, за всяка буква $a \in \Sigma$ е безконтекстен.
  \item
    Нека $L_1$ и $L_2$ са безконтекстни езици. Тогава:
    \begin{itemize}
    \item
      $L_1 \cup L_2$ е безконтекстен език.
    \item
      $L_1 \cdot L_2$ е безконтекстен език.
    \item
      $L^\star_1$ е безконтекстен език.
    \end{itemize}
  \end{itemize}
\end{hint}


За произволна безконтекстна граматика $G$, дефинираме релацията $X \yield{\ell} \alpha$, където $X \in V \cup \Sigma$ и $\alpha \in (V\cup\Sigma)^\star$, по следния начин:
\mynote{Тук е добре да се нарисува картинка. Да се обясни, че $\yield{\star}$ е аналог на BFS, докато $\derive{\star}$ е аналог на DFS.}
\begin{prooftree}
  \AxiomC{}
  \UnaryInfC{$X \yield{0} X$}
\end{prooftree}


\begin{prooftree}
  \AxiomC{$X \to_G X_1\cdots X_n$}
  \AxiomC{$X_i \yield{\ell_i} \gamma_i\text{ за }i=1,\dots,n$}
  \RightLabel{\scriptsize{($\ell = \max\{\ell_1,\dots,\ell_n\})$}}
  \BinaryInfC{$X \yield{\ell+1} \gamma_1\cdots\gamma_n$}
\end{prooftree}

\mynote{ Съобразете, че имаме:
\begin{prooftree}
  \AxiomC{$X \to_G \gamma$}
  \UnaryInfC{$X \yield{1} \gamma$.}
\end{prooftree}}

Нека $\yield{\star}$ е рефлексивното и транзитивно затваряне на релацията $\yield{\ell}$, т.е.
\[X \yield{\star} \gamma \dff (\exists \ell\in\Nat)[X \yield{\ell} \gamma].\]

\mynote{Ще докажем, че $X \yield{\star}_G \gamma \iff X \derive{\star}_G \gamma$.}

Да напомним, че дефинирахме език $\L_\A(q)$.
Сега ще дефинираме език $\L_G(A)$, за произволна променлива.
\[\L_G(A) = \{\alpha \in \Sigma^\star \mid A \yield{\star}\alpha\}.\]
\mynote{Това е аналогично на $\L(\A) = \L_\A(\qstart)$.}
Ясно е, че $\L(G) = \L_G(S)$.
Също така, да дефинираме следните апроксимации на $\L_G(A)$.
\[\L^n_G(A) = \{\alpha \in \Sigma^\star \mid A \yield{\leq n} \alpha\}.\]
Ясно е, че $\L_G(A) = \bigcup_{n\geq 0}\L^n_G(A)$.

Следващото твърдение е ключово за решаването на задачи.

\begin{proposition}\label{pr:grammar:yield-approximation}
  Нека $G$ е произволна безконтекстна граматика и $A$ е променлива в $G$.
  Тогава имаме следното:
  \begin{align*}
    \L^0_G(A) & = \emptyset\\
    \L^{\ell+1}_G(A) & = \bigcup\{\{\alpha_1\}\cdot \L^\ell_G(A_1) \cdots \{\alpha_n\} \cdot \L^\ell_G(A_n) \cdot \{\alpha_{n+1}\} \mid A \to_G \alpha_1A_1\cdots\alpha_nA_n\alpha_{n+1}\}.
  \end{align*}
  % \[\L^0_G(X) =
  %   \begin{cases}
  %     X, & \text{ ако } X \in \Sigma\\
  %     \emptyset, & \text{ ако } X \in V.
  %   \end{cases}
  % \]
  % \[\L^{\ell+1}_G(X) = \L^\ell_G(X) \cup \bigcup\{\L^\ell_G(X_1)\cdots\L^\ell_G(X_n) \mid X \to_G X_1\cdots X_n\text{ е правило в }G\}.\]
\end{proposition}



Като първи пример нека да видим, че това включване е {\em строго}, т.е. съществува безконтекстен език, който не е регулярен.
Да напомним, че вече видяхме, че езикът $L = \{a^nb^n \mid n\in\Nat\}$ не е регулярен.


\begin{example}
  Да разгледаме безконтекстната граматика $G$ зададена със следните правила:
  \begin{align*}
    & S \to aSb \mid \varepsilon.
  \end{align*}
  Тогава
  \begin{align*}
    & \L^0_G(S) = \emptyset\\
    & \L^1_G(S) = \{a\} \cdot \emptyset \cdot \{b\} \cup \{\varepsilon\} = \{\varepsilon\}\\
    & \L^{2}_G(S) = \{a\} \cdot \{\varepsilon\} \cdot \{b\} \cup \{\varepsilon\} = \{\varepsilon, ab\}\\
    & \L^{3}_G(S) = \{a\} \cdot \{\varepsilon, ab\} \cdot \{b\} \cup \{\varepsilon\} = \{\varepsilon,ab,aabb\} = \{a^nb^n \mid n < \ell\}\\
    & \vdots
  \end{align*}
  Лесно се съобразява, че
  \[\L^\ell_G(S) = \{a^nb^n \mid n < \ell\}.\]
  Заключаваме, че:
  \[\L(G) = \L_G(S) = \bigcup_{\ell}\L^\ell_G(S) = \{a^nb^n \mid n\in\Nat\}.\]
\end{example}

\begin{example}
  Да разгледаме безконтекстната граматика $G$ зададена със следните правила:
  \begin{align*}
    & S \to aSc\ |\  B\\
    & B \to bBc\ |\ \varepsilon.
  \end{align*}
  Лесно се съобразява, че $\L(G) = \{a^nb^kc^{n+k} \mid n,k\in\Nat\}$, защото
  \begin{align*}
    \L^\ell_G(S) & = \{a^nb^kc^{n+k} \mid n,k < \ell-1\}\\
    \L^\ell_G(B) & = \{b^kc^k \mid k < \ell\}.
  \end{align*}
\end{example}

\begin{example}
  Да разгледаме граматика с правила
  \begin{align*}
    & S \to S + S\ |\ S * S\ |\ (S)\ |\ V\\
    & V \to x\ |\ y\ |\ z
  \end{align*}

  Думата $x * y + z$ има две различни дървета на извод.

  \begin{framed}
    \qtreecenterfalse
    \Tree [.$S$ [.$S$ [.$V$ $x$ ] ] $*$ [.$S$ [.$S$ [.$V$ $y$ ] ] $+$ [.$S$ [.$V$ $z$ ] ] ] ]
    \hskip 0.4in
    \Tree [.$S$ [.$S$ [.$S$ [.$V$ $x$ ] ] $*$ [.$S$ [.$V$ $y$ ] ] ]  $+$  [.$S$ [.$V$ $z$ ] ] ]
  \end{framed}
  
  
  Да разгледаме граматика с правила
  \begin{align*}
    & S \to E + S\ |\ E\\
    & E \to V * E\ |\ V\ |\ (S) * E\ |\ (S)\\
    & V \to x\ |\ y\ |\ z
  \end{align*}
  Сега думата $x * y + z$ има само едно дърво на извод.

  \begin{framed}
    \Tree [.$S$ [.$E$ [.$V$ $x$ ] $*$ [.$E$ [.$V$ $y$ ] ] ] $+$ [.$S$ [.$E$ [.$V$ $z$ ] ] ] ]
  \end{framed}
\end{example}

\begin{example}
  \begin{align*}
    & S \to \texttt{if } S \texttt{ then } S \texttt{ else }S\ |\ \texttt{ if }S \texttt{ then }S\ |\ V\\
    & V \to x\ |\ y\ |\ z
  \end{align*}

  Ние искаме следната граматика:
  \begin{align*}
    & S \to M\ |\ U\\
    & M \to \texttt{if } S \texttt{ then } M \texttt{ else }M\ |\ X\\
    & U \to \texttt{if } S \texttt{ then } S\ |\ \texttt{if } S \texttt{ then } M \texttt{ else }U
  \end{align*}
\end{example}

\begin{example}
  Да разгледаме граматика с правила
  \begin{align*}
    & S \to E\\
    & E \to E + P\ |\ P\\
    & P \to P * N\ |\ N\\
    & N \to (E)\ |\ a.
  \end{align*}
\end{example}


% \begin{proposition}
%   \label{pr:grammar:concat}
%   Ако $\alpha_1 \derive{n_1} \beta_1, \dots, \alpha_k \derive{n_k} \beta_k$, тогава
%   \[\alpha_1\cdots\alpha_k \derive{n} \beta_1\cdots\beta_k,\]
%   където $n = \sum^k_{i=1} n_i$.
% \end{proposition}
% \begin{proof}
%   Индукция по $k \geq 1$.
%   \begin{itemize}
%   \item
%     За $k = 1$ е очевидно. В този случай $n_1 = n$.
%   \item
%    Нека $k > 1$. Тогава от И.П. за $k-1$ имаме, че
%    $\alpha_1\cdots\alpha_{k-1} \derive{n'} \beta_1\cdots\beta_{k-1}$ и $n' = \sum^{k-1}_{i=1} n_i$.
%    От \Proposition{grammar:add} имаме, че
%    \[\alpha_1\cdots\alpha_{k-1}\alpha_{k} \derive{n'} \beta_1\cdots\beta_{k-1}\alpha_k.\]
%    Понеже $\alpha_k \derive{n_k} \beta_k$, отново от \Proposition{grammar:add} получаваме, че
%    \[\beta_1\cdots\beta_{k-1}\alpha_k \derive{n_k} \beta_1 \cdots \beta_{k-1}\beta_k.\]
%    Сега обединяваме двата извода и получаваме, че
%    \[\alpha_1\cdots\alpha_{k} \derive{n} \beta_1\cdots\beta_k,\]
%    където $n = \sum^k_{i=1} n_i$.
%   \end{itemize}
% \end{proof}


%%% Local Variables:
%%% mode: latex
%%% TeX-master: "../eai"
%%% End:
