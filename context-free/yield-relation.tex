\section{Извод върху синтактично дърво}

За произволна безконтекстна граматика $G$, дефинираме релацията $X \yield{\ell} \alpha$, където $X \in V \cup \Sigma$ и $\alpha \in (V\cup\Sigma)^\star$, по следния начин:
\begin{prooftree}
  \AxiomC{}
  \RightLabel{\scriptsize{правило (0)}}
  \UnaryInfC{$X \yield{0} X$}
\end{prooftree}

\begin{prooftree}
  \AxiomC{$X \to_G X_1\cdots X_n$}
  \AxiomC{$X_1 \yield{\ell_1} \gamma_1$}
  \AxiomC{$\cdots$}
  \AxiomC{$X_n \yield{\ell_n} \gamma_n$}
  \LeftLabel{\scriptsize{($\ell = \max\{\ell_1,\dots,\ell_n\})$}}
  \RightLabel{\scriptsize{правило (1)}}
  \QuaternaryInfC{$X \yield{\ell+1} \gamma_1\cdots\gamma_n$}
\end{prooftree}

\mynote{ Съобразете, че имаме:
\begin{prooftree}
  \AxiomC{$X \to_G \gamma$}
  \UnaryInfC{$X \yield{1} \gamma$.}
\end{prooftree}
Обърнете внимание също, че тази релация е рефликсивна, но не е транзитивна!}

Да дефинираме $\yield{\star}$ по следния начин:
\[X \yield{\star} \gamma \dff (\exists \ell\in\Nat)[X \yield{\ell} \gamma].\]




\begin{lemma}
  Нека $G$ е безконтекстна граматика, $X \in V \cup \Sigma$ и $\beta \in (V \cup \Sigma)^\star$.
  Тогава ако $X \derive{\star} \beta$, то $X \yield{\star} \beta$.
\end{lemma}  
\begin{proof}
  С пълна индукция по $\ell$ ще докажем, че ако $X \derive{\ell} \beta$, то $X \yield{\star} \beta$.
  \begin{itemize}
  \item
    $\ell = 0$, т.е. $X \derive{0} X$.
    Тогава е ясно, че $X \yield{\star} X$.
  \item
    Нека $\ell > 0$ и $X \derive{\ell} \beta$.
    Според правилата на извод в граматика имаме извода

    \begin{prooftree}
      \AxiomC{$X \to_G X_0X_1\cdots X_k$}
      \AxiomC{$X_0X_1\cdots X_k \derive{\ell-1} \beta$}
      \RightLabel{\scriptsize{правило (1)}}
      \BinaryInfC{$X \derive{\ell} \beta$}
    \end{prooftree}

    От \Proposition{grammar:divide} знаем, че съществува разбиване на $\beta$ на $k+1$ части, така че:
    \begin{itemize}
    \item
      $\beta = \beta_0 \cdots \beta_{k}$;
    \item
      $X_i \derive{\ell_i} \beta_i$, за всяко $i = 0,\dots,k$;
    \item
      $\ell-1 = \sum^k_{i=1} \ell_i$.
    \end{itemize}
    Получаваме следното:
    \begin{prooftree}
      \AxiomC{$X \to_G X_0\cdots X_k$}
      \AxiomC{$X_0 \derive{\ell_0} \beta_0$}
      \RightLabel{\scriptsize{\IndHyp}}
      \UnaryInfC{$X_0 \yield{\star} \beta_0$}
      \AxiomC{$\cdots$}
      \AxiomC{$X_k \derive{\ell_k} \beta_k$}
      \RightLabel{\scriptsize{\IndHyp}}
      \UnaryInfC{$X_k \yield{\star} \beta_k$}
      \RightLabel{\scriptsize{правило (1)}}
      \QuaternaryInfC{$X \yield{\star} \underbrace{\beta_0\cdots\beta_k}_{\beta}$}
    \end{prooftree}
  \end{itemize}
\end{proof}

\begin{lemma}
  Нека $G$ е безконтекстна граматика, $X \in V \cup \Sigma$ и $\gamma \in (V \cup \Sigma)^\star$.
  Тогава ако $X \yield{\star}_G \gamma$, то $X \derive{\star}_G \gamma$.
\end{lemma}
\begin{proof}
  С пълна индукция по $\ell$ ще докажем, че ако $X \yield{\ell} \gamma$, то $X \derive{\star} \gamma$.
  \begin{itemize}
  \item
    Нека $\ell = 0$. Това означава, че $X \yield{0} X$. Ясно е, че $X \derive{\star} X$.
  \item
    Нека $\ell > 0$. Тогава имаме следното:
    \begin{prooftree}
      \AxiomC{$X \to_G X_0\cdots X_n$}
      \AxiomC{$X_0 \yield{\ell_0} \gamma_0$}
      \AxiomC{$\cdots$}
      \AxiomC{$X_n \yield{\ell_n} \gamma_n$}
      \RightLabel{\scriptsize{($\ell = 1 + \max\{\ell_0,\dots,\ell_n\})$}}
      \QuaternaryInfC{$X \yield{\ell} \gamma_0\cdots\gamma_n$}
    \end{prooftree}
    Получаваме, че:
    \begin{prooftree}
      \AxiomC{$X \to_G X_0 \cdots X_{n}$}
      \AxiomC{$X_0 \yield{\ell_0} \gamma_0$}
      \RightLabel{\scriptsize{\IndHyp}}
      \UnaryInfC{$X_0 \derive{\star} \gamma_0$}
      \AxiomC{$\cdots$}
      \AxiomC{$X_n \yield{\ell_n} \gamma_n$}
      \RightLabel{\scriptsize{\IndHyp}}
      \UnaryInfC{$X_n \derive{\star} \gamma_n$}
      \RightLabel{\scriptsize{(\Proposition{unrestricted-grammar:concat})}}
      \TrinaryInfC{$X_0 \cdots X_{n} \derive{\star} \gamma_0\cdots\gamma_{n}$}
      \RightLabel{\scriptsize{правило (1)}}
      \BinaryInfC{$X \derive{\star} \underbrace{\gamma_0\cdots\gamma_{n}}_{\gamma}$}
    \end{prooftree}
  \end{itemize}
\end{proof}

Комбинирайки предишните две леми получаваме следната теорема.
\begin{framed}
  \begin{theorem}\label{th:grammar:yield-derive-equivalent}
    Нека $G$ е безконтекстна граматика, $X \in V \cup \Sigma$ и $\beta \in (V \cup \Sigma)^\star$.
    Тогава $X \yield{\star}_G \gamma$ точно тогава, когато $X \derive{\star}_G \gamma$.
    В частност,
    \[\L(G) = \{\alpha \in \Sigma^\star \mid S \yield{\star}\alpha\}.\]
  \end{theorem}  
\end{framed}

Следващото твърдение ни казва, че има пряка връзка между релацията $\yield{\star}$ и дърветата на извод.

\begin{important}
  \begin{proposition}
    Нека $G$ е безконтекстна граматика. Тогава
    $X \yield{\ell} \gamma$ точно тогава, когато съществува дърво на извод $P$ съвместимо с $G$, за което
    $\texttt{root}(P) = X$, $\texttt{yield}(P) = \alpha$ и $\texttt{height}(P) = \ell$.
  \end{proposition}
\end{important}
\begin{hint}
  Индукция по $\ell$.
\end{hint}

\begin{extra}
  
  Добре е да обърнем внимание, че е възможно, ако $A \yield{\ell} \alpha$, то да има няколко дървета на извод с корен променливата $A$,
  които да извеждат думата $\alpha$ и да имат височина $\ell$.
  \begin{example}
    Да разгледаме граматика с правила
    \begin{align*}
      & S \to S + S\ |\ S * S\ |\ (S)\ |\ V\\
      & V \to x\ |\ y\ |\ z
    \end{align*}
    
    Думата $x * y + z$ има две различни дървета на извод $P_1$ и $P_2$, като и двете имат височина $4$ и
    $\texttt{yield}(P_1) = \texttt{yield}(P_2) = x*y+z$.
    
    \begin{framed}
      \centering
      \qtreecenterfalse
      \Tree [.$S$ [.$S$ [.$V$ $x$ ] ] $*$ [.$S$ [.$S$ [.$V$ $y$ ] ] $+$ [.$S$ [.$V$ $z$ ] ] ] ]
      \hskip 0.6in
      \Tree [.$S$ [.$S$ [.$S$ [.$V$ $x$ ] ] $*$ [.$S$ [.$V$ $y$ ] ] ]  $+$  [.$S$ [.$V$ $z$ ] ] ]
    \end{framed}  
    
    Да разгледаме граматика с правила
    \begin{align*}
      & S \to E + S\ |\ E\\
      & E \to V * E\ |\ V\ |\ (S) * E\ |\ (S)\\
      & V \to x\ |\ y\ |\ z
    \end{align*}
    Сега думата $x * y + z$ има само едно дърво на извод.
    \mynote{Тук операцията $*$ е с по-висок приоритет в сравнение с операцията $+$.}    
    \begin{framed}
      \Tree [.$S$ [.$E$ [.$V$ $x$ ] $*$ [.$E$ [.$V$ $y$ ] ] ] $+$ [.$S$ [.$E$ [.$V$ $z$ ] ] ] ]
    \end{framed}
  \end{example}

\begin{example}
  \begin{align*}
    & S \to \texttt{if } S \texttt{ then } S \texttt{ else }S\ |\ \texttt{ if }S \texttt{ then }S\ |\ V\\
    & V \to x\ |\ y\ |\ z
  \end{align*}

  Ние искаме следната граматика:
  \begin{align*}
    & S \to M\ |\ U\\
    & M \to \texttt{if } S \texttt{ then } M \texttt{ else }M\ |\ X\\
    & U \to \texttt{if } S \texttt{ then } S\ |\ \texttt{if } S \texttt{ then } M \texttt{ else }U
  \end{align*}
\end{example}

\begin{example}
  Да разгледаме граматика с правила
  \begin{align*}
    & S \to E\\
    & E \to E + P\ |\ P\\
    & P \to P * N\ |\ N\\
    & N \to (E)\ |\ a.
  \end{align*}
\end{example}

\end{extra}


%%%Local Variables:
%%% mode: latex
%%% TeX-master: "../eai"
%%% End:
