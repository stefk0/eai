\subsection{Нормална Форма на Чомски}
\index{Чомски}
%[стр. 99 от \cite{sipser}]
\index{нормална форма на Чомски}
Една безконтекстна граматика е в {\bf нормална форма на Чомски}, ако
всяко правило е от вида
\[A \rightarrow BC\mbox{ и }A \rightarrow a,\]
като $B, C$ {\em не могат} да бъдат променливата за начало $S$.
Освен това, позволяваме правилото $S\to\varepsilon$.
\footnote{В \cite[стр. 151]{papadimitriou} дефиницията е малко по-различна.
Там дефинират $G$ да бъде в нормална форма на Чомски ако $R \subseteq V\times(V\cup\Sigma)^2$.
В този случай губим езиците $\{\varepsilon\}$ и $\{a\}$, за $a\in\Sigma$.}

\begin{framed}
  \begin{thm}
    Всеки безконтекстен език $L$ се поражда от безконтекстна граматика в нормална форма на Чомски.
  \end{thm}
\end{framed}
\begin{proof}
%  \marginpar{Броят на правилата може да се увеличи експоненциално.}
  Нека имаме контекстно-свободна граматика $G$, за която $L = \L(G)$.
  Ще построим безконтекстна граматика $G^\prime$ в нормална форма на Чомски, $L = \L(G^\prime)$.
  % [стр. 99 от \cite{sipser}]
  Следваме следната процедура:
  \begin{itemize}
  \item
    Добавяме нов начален символ $S_0$ и правило $S_0 \to S$.
  \item
    \marginpar{Време $\mathcal{O}(|G|)$}
    {\em Премахваме дългите правила.}
    Заменяме правилата от вида $A\to X_1X_2\dots X_n$, $n\geq 3$ с
    правилата \[A\to X_1A_1,\ A_1\to X_2A_2,\ \dots,\ A_{n-2} \to X_{n-1}X_n.\]
    където $A_i$ са нови променливи.
  \item
    \marginpar{Време $\mathcal{O}(|G|^2)$}
    {\em Премахваме $\varepsilon$-правилата.}
    За всяка променлива $A \neq S_0$ премахваме правилата от вида $A\to\varepsilon$.
    Това правим по следния начин.
    
    Ако имаме правило от вида $R \to Au$ или $R\to u A$, $u \in V \cup \Sigma$,
    то добавяме правилото $R\to u$.
    %Правим това за всяко срещане на променливата $A$ в дясната страна на правило.
    Например, 
    \begin{itemize}
    \item 
      ако имаме правило $R\to aA$, то добавяме правилото $R \to a$;
    \item
      ако имаме правило $R\to AA$, то добавяме правилото $R \to A$.
    \end{itemize}
    Ако имаме правило от вида $R\to A$, то добавяме правилото $R\to\varepsilon$
    само ако променливата $R$ още не е преминала през процедурата за премахване на $\varepsilon$.
  \item
    \marginpar{Време $\mathcal{O}(|G|^2)$}
    \marginpar{Памет $\mathcal{O}(|G|^2)$}
    Премахваме преименуващите правила, т.е. правила от вида $A\to B$.
    Заменяме всяко правило от вида $B \to \beta$ с $A\to \beta$,
    освен ако $A \to \beta$ е вече премахнато преименуващо правило.
  \item
    \marginpar{Време $\mathcal{O}(|G|)$}
    За правила от вида $A\to u_1 u_2$, където $u_1, u_2 \in V \cup \Sigma$, 
    заменяме всяка буква $u_i$ с новата променлива $U_i$
    и добавяме правилото $U_i\to u_i$.
    Например, правилото $A \to aB$ се заменя с правилото $A \to XB$ и добавяме правилото $X \to a$,
    където $X$ е нова променлива.
  \end{itemize}
\end{proof}

\begin{thm}
  При дадена безконтекстна граматика $G$, можем да намерим еквивалентна
  на нея граматика $G'$ в нормална форма на Чомски за време $\mathcal{O}(|G|^2)$,
  като получената граматика е с дължина $\mathcal{O}(|G|^2)$.
\end{thm}


% \begin{problem}
%   Нека е дадена граматиката  $G = \pair{\{S,A,B,C,D,E\}, \{a,b\},S, R}$.
%   \begin{enumerate}[a)]
%   \item
%     Намерете множеството $\{X \in V \mid X \rightarrow^\star_G \varepsilon\}$.
%   \item
%     Вярно ли е, че $\varepsilon \in L(G)$?
%   \item
%     Постройте граматика $G_1$ без $\varepsilon$-правила, за която $L(G_1)=L(G)\setminus\{\varepsilon\}$.
%   \end{enumerate}
%   Множеството от правила $R$ на граматиката $G$ е зададено като:
%   \begin{enumerate}[a)]
%   \item
%     $R = \{S\rightarrow D,D\rightarrow AD|b,A\rightarrow ACB|BC|a, B\rightarrow ABCA|CEC,C\rightarrow \varepsilon|CA|a, E\rightarrow \varepsilon|aEb\}$;
%   \item
%     $R = \{S \rightarrow aD, D\rightarrow \varepsilon|ABBA|ADD,A\rightarrow DEB|a,B\rightarrow DDD|DC|b,C\rightarrow CCE|a, E\rightarrow \varepsilon|bEa\}$;
%   \item
%     $R = \{ S\rightarrow D,D\rightarrow AD|b,A\rightarrow AB|BC|a, B\rightarrow AB|CC, C\rightarrow \varepsilon|CA|a, E\rightarrow a|EB\}$;
%   \item
%     $R = \{ S \rightarrow AD|a, D\rightarrow \varepsilon|BB|AD,A\rightarrow DB|a,B\rightarrow DD|DC|b,C\rightarrow CE|a, E\rightarrow AB|b|EA\}$;
%   \item
%     $R =\{S\rightarrow AS|SB|SS,B\rightarrow CA|b, C\rightarrow AA|a|BA,A\rightarrow \varepsilon|BS\}$;
%   % \item
%   %   $R = \{S\rightarrow AB|AC,B\rightarrow \varepsilon |BC|b,A\rightarrow BB|CC|a,C\rightarrow CS|a\}$;
%   % \item
%   %   $R = \{S\rightarrow AS|SB|SS,B\rightarrow AC|b, C\rightarrow A|a|AB,A\rightarrow \varepsilon|BS\}$;
%   \item
%     $R = \{S\rightarrow BA|CA,B\rightarrow \varepsilon |BC|b,A\rightarrow BB|CC|a, C\rightarrow CS|a\}$;
%   \item
%     $R = \{S\rightarrow AS|b,A\rightarrow AC|BC|a, B\rightarrow BC|CC,C\rightarrow \varepsilon|CA|a\}$;
%   \item
%     $R = \{S\rightarrow \varepsilon|BA|AS,A\rightarrow SB|a,B\rightarrow SS|SC|b,
%     C\rightarrow CC|a\}$; 
%   \end{enumerate}
% \end{problem}


% \begin{problem}
%   Намерете безконтекстна граматика в нормална форма на Чомски за езиците от задача 6.
% \end{problem}


%%% Local Variables:
%%% mode: latex
%%% TeX-master: "../eai"
%%% End:
