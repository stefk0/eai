\section{Дървета}\index{дърво}

\mynote{Понятието дърво е едно от най-основните в информатиката и може да се дефинира по много различни начини, в зависимост от това за какви цели се използва. Тук на практика следваме \cite{nerode-shore}, защото по-късно ще е важно да имаме наредба между възлите на дървото. При нас всички дървета са крайно разклонени и всеки възел в дървото еднозначно се определя от пътя от възела до корена.}
Нека $b \in \Nat$.
Непразното множество $T \subseteq \{0,1,\dots,b-1\}^\star$ се нарича {\bf дърво},
ако $T$ е затворено относно префикси, т.е. $\texttt{Pref}(T) = T$.
С други думи,
\[(\forall \alpha)(\forall \beta)[\ \alpha \in T\ \&\ \beta \preceq \alpha\ \implies\ \beta \in T\ ].\]



Ако гледаме на елементите на $T$ като върхове на граф, то можем да дефинираме бинарната релацията $Succ$, върху едно дърво $T$ като
\[Succ(\alpha,\beta) \dff \alpha,\beta \in T\ \&\ (\exists c < b)[\alpha \cdot c = \beta],\]
която казва, че върха $\beta$ е пряк наследник на върха $\alpha$.
Обърнете внимание, че е възможно $T \neq T'$ да са такива, че $(T,Succ) \cong (T',Succ')$.

Поради тази причина е удобно да наречем едно дърво $T$ \emph{оптимално}, ако за него е изпълнено свойството:
\begin{equation}
  \label{eq:10}
  \alpha\cdot c \in T\ \&\ a < c \implies \alpha \cdot a \in T.
\end{equation}
Тогава лесно се вижда, че за две оптимални дървета $T$ и $T'$,
\[T = T' \iff (T,Succ) \cong (T',Succ').\]
Нека да въведем следните означения:
\begin{align*}
  & \high(T) \df \max\{\ \abs{\alpha}\ \mid\ \alpha \in T\ \} & \comment\text{височина}\\
  & \successor_T(\alpha) \df \{ \alpha c \mid \alpha c \in T\ \&\ c < b\} & \comment\text{наследниците на }\alpha\\
  & T_\alpha \df \alpha^{-1}(T) & \comment\text{поддървото на }\alpha\\
  & \leaves(T) \df \{ \alpha \in T \mid \successor_T(\alpha) = \emptyset \}. & \comment\text{листата на }T
\end{align*}

\begin{figure}[H]
  \centering
  \begin{tikzpicture}
    \coordinate (A) at (0,0);
    \coordinate (B) at (-2,-3);
    \coordinate (C) at (2,-3);
    \coordinate (D) at (0,-1.5);
    \coordinate (E) at (-1,-3);
    \coordinate (F) at (1,-3);
    \draw (A) -- node[above left]{$T$} (B) -- node[below]{$\alpha^{-1}(T)$}(C) -- (A);
    \draw (D) -- (E);
    \draw (D) -- (F);
    \draw [photon] (A) -- node[left]{$\alpha$} (D);
  \end{tikzpicture}
  \caption{$\alpha^{-1}(T)$ е поддърво на $T$.}
\end{figure}

  
\begin{problem}
  Докажете, че ако $T$ е дърво, то $\texttt{Pref}(\texttt{leaves}(T)) = T$.
\end{problem}

\begin{problem}\label{prob:tree:leaves-upper-bound}
  Нека $T \subseteq \{0,\dots,b-1\}^\star$ е крайно дърво. Докажете, че
  \[ |\leaves(T)| \leq b^{\high(T)}.\]
\end{problem}
\begin{proof}
  Индукция по $\high(T)$.
  \begin{itemize}
  \item
    Нека $\high(T) = 0$. Тогава е ясно, че $|\leaves(T)| = |\{\varepsilon\}| = 1 \leq b^0$.
  \item
    Нека $\high(T) > 0$.
    За всяко $a \in T$ е ясно, че $\high(T_a) < \high(T)$. Тогава:
    \mynote{В този случай имаме, че $\leaves(T) = \bigcup_{a\in T} (\{a\}\cdot \leaves(T_a))$. Тук гледаме на $a$ и $b$ от една страна като букви в азбука, но и като числа.}
    \begin{align*}
      |\texttt{leaves}(T)| & = \sum_{a \in T}|\texttt{leaves}(T_a)| & \comment{T_a \df a^{-1}(T)}\\
                           & \leq \sum_{a \in T} b^{\high(T_a)} & \comment\text{от \IndHyp}\\
                           & \leq \sum_{a \in T} b^{\high(T) -1} & \comment \high(T_a) \leq \high(T)-1 \\
                           & \leq \sum_{a < b} b^{\high(T)-1}\\
                           & = b^{\high(T)}.
    \end{align*}
  \end{itemize}
\end{proof}


%%% Local Variables:
%%% mode: latex
%%% TeX-master: "../eai"
%%% End:
