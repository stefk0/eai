\section{Доказателства на твърдения}

\subsection*{Допускане на противното}

Да приемем, че искаме да докажем, че свойството $P(x)$
е вярно за всяко естествено число.
Един начин да направим това е следният:
\begin{itemize}
\item 
  Допускаме, че съществува елемент $n$, за който $\neg P(n)$.
\item
  Използвайки, че $\neg P(n)$ правим извод, от който следва факт, за който знаем, че винаги е лъжа.
  Това означава, че доказваме следното твърдение
  \[\exists x \neg P(x) \rightarrow \mathbf{0}.\]
\item
  Тогава можем да заключим, че $\forall x P(x)$, защото имаме следния извод:
  \begin{prooftree}
    \AxiomC{$\exists x \neg P(x) \rightarrow \mathbf{0}$}
    \UnaryInfC{$\mathbf{1} \rightarrow \neg \exists x \neg P(x)$}
    \UnaryInfC{$\neg \exists x \neg P(x)$}
    \UnaryInfC{$\forall x P(x)$}
  \end{prooftree}
\end{itemize}

Ще илюстрираме този метод като решим няколко прости задачи.

\begin{problem}
  \label{prob:even-number-square}
  За всяко $a \in \Int$, ако $a^2$ е четно, то $a$ е четно.
\end{problem}
\begin{proof}
  Ние искаме да докажем твърдението $P$, където:
  \[P \equiv (\forall a\in\Z)[a^2\mbox{ е четно}\ \rightarrow\ a\mbox{ е четно}].\]
  \marginpar{$\neg (\forall x)(A(x) \rightarrow B(x))$ е еквивалентно на $(\exists x)(A(x) \wedge \neg B(x))$}
  Да допуснем противното, т.е. изпълнено е $\neg P$. Лесно се вижда, че
  \[\neg P \iff (\exists a\in\Z)[a^2\mbox{ е четно}\ \wedge\ a\mbox{ не е четно}].\]
  Да вземем едно такова нечетно $a$, за което $a^2$ е четно.
  Това означава, че $a = 2k+1$, за някое $k \in \Z$,
  и \[a^2 = (2k+1)^2 = 4k^2 + 4k + 1,\]
  което очевидно е нечетно число.
  Но ние допуснахме, че $a^2$ е четно.
  Така достигаме до противоречие, следователно нашето допускане е грешно 
  и 
  \[(\forall a\in\Z)[a^2\mbox{ е четно}\ \rightarrow\ a\mbox{ е четно}].\]
\end{proof}

\begin{problem}
  Докажете, $\sqrt{2}$ {\bf не} е рационално число.
\end{problem}
\begin{proof}
  Да допуснем, че $\sqrt{2}$ е рационално число. Тогава  съществуват $a,b \in \Z$, такива че
  \[\sqrt{2} = \frac{a}{b}.\]
  Без ограничение, можем да приемем, че $a$ и $b$ са естествени числа,
  които нямат общи делители, т.е. не можем да съкратим дробта $\frac{a}{b}$.
  Получаваме, че \[2b^2 = a^2.\]
  Тогава $a^2$ е четно число и от Задача \ref{prob:even-number-square}, $a$ е четно число.
  Нека $a = 2k$. Получаваме, че
  \[2b^2 = 4k^2,\]
  от което следва, че
  \[b^2 = 2k^2.\]
  Това означава, че $b$ също е четно число, $b = 2n$, за някое $n \in \Z$.
  Следователно, $a$ и $b$ са четни числа и имат общ делител $2$,
  което е противоречие с нашето допускане, че $a$ и $b$ нямат общи делители.
  Така достигаме до противоречие.
  Накрая заключаваме, че $\sqrt{2}$ не е рационално число.
\end{proof}


\subsection*{Индукция върху естествените числа}

\marginpar{Да напомним, че естествените числа са $\Nat = \{0,1,2,\dots\}$}
Доказателството с индукция по $\Nat$ представлява следната схема:
\begin{prooftree}
  \AxiomC{$P(0)$}
  \AxiomC{$(\forall x\in\Nat)[P(x)\rightarrow P(x+1)]$}
  \BinaryInfC{$(\forall x\in\Nat) P(x)$}
\end{prooftree}

Това означава, че ако искаме да докажем, че свойството $P(x)$ е вярно за всяко естествено число $x$,
то трябва да докажем първо, че е изпълнено $P(0)$ и след това, за произволно естествено число $x$, ако $P(x)$ вярно, то също така е вярно $P(x+1)$.

\begin{problem}
  \label{prob:number-prod-prime}  
  Всяко естествено число $n \geq 2$ може да се запише като произведение на прости числа.
\end{problem}
\begin{proof}
  Доказателството протича с индукция по $n \geq 2$.
  \begin{enumerate}[a)]
  \item 
    За $n = 2$  е ясно.
  \item
    Ако $n+1$ е просто число, то всичко е ясно.
    Ако $n+1$ е съставно, то \[n + 1 = n_1\cdot n_2.\]
    Тогава $n_1 = p^{n_1}_1\cdots p^{n_k}_k$ и $n_2 = q^{m_1}_1\cdots q^{m_r}_r$,
    където $p_1,\dots,p_k$ и $q_1,\dots,q_r$ са прости числа.
    Тогава е ясно, че $n+1$ също е произведение на прости числа.
  \end{enumerate}
\end{proof}

\begin{problem}
  Докажете, че за всяко $n$, 
  \[\sum^n_{i=0} 2^i = 2^{n+1} - 1.\]
\end{problem}
\begin{proof}
  Доказателството протича с индукция по $n$.
  \begin{itemize}
  \item 
    За $n = 0$, $\sum^0_{i=0}2^i = 1 = 2^{1} - 1$.
  \item
    Нека твърдението е вярно за $n$.
    Ще докажем, че твърдението е вярно за $n+1$.
    \begin{align*}
      \sum^{n+1}_{i=0} 2^i & = \sum^{n}_{i=0}2^i + 2^{n+1}\\
      & = 2^{n+1} - 1 + 2^{n+1} & (\text{от И.П.})\\
      & = 2.2^{n+1} - 1 \\
      & = 2^{(n+1)+1} - 1.
    \end{align*}
  \end{itemize}
\end{proof}

%\subsection*{Пълна индукция върху естествените числа}

%%% Local Variables:
%%% mode: latex
%%% TeX-master: "../eai"
%%% End:
