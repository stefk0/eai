\subsection*{Основни видове бинарни релации}
% \mynote{Бинарни релации}

Подмножествата $R$ от вида $R \subseteq A\times A\times\cdots\times A$ се наричат релации.
Релациите от вида $R\ \subseteq\ A\times A$ са важен клас, който ще срещаме често.
Да разгледаме няколко основни видове релации от този клас:
\begin{enumerate}[I)]
\item
  \index{релация!рефлексивна}
  {\bf рефлексивна}, ако
  \[(\forall x\in A)[\pair{x,x}\in R].\]
  Например, бинарната релация $\leq$ над $\Nat$ е рефлексивна, защото
  \[(\forall x\in \Nat)[\ x \leq x\ ].\]
% \item
%   {\bf антирефлексивна}, ако
%   \[(\forall x\in A)[\pair{x,x}\not\in R].\]
%   Например, релацията $<\ \subseteq\ \Nat\times\Nat$ е антирефлексивна, защото
%   \[(\forall x\in \Nat)[x \not< x].\]
\item
  \index{релация!транзитивна}
  {\bf транзитивна}, ако
  \[(\forall x,y,z\in A)[\ \pair{x,y}\in R\ \&\ \pair{y,z}\in R \rightarrow \pair{x,z}\in R\ ].\]
  Например, бинарната релация $\leq$ над $\Nat$ е транзитивна, защото
  \[(\forall x,y,z\in \Nat)[\ x \leq y\ \&\ y \leq z\ \rightarrow\ x\leq z\ ].\]
\item
  \index{релация!симетрична}
  {\bf симетрична}, ако
  \[(\forall x,y\in A)[\ \pair{x,y}\in R \rightarrow \pair{y,x}\in R\ ].\]
  Например, бинарната релация $=$ над $\Nat$ е рефлексивна, защото
  \[(\forall x,y\in \Nat)[\ x = y\ \rightarrow\ y = x\ ].\]
\item
  \index{релация!антисиметрична}
  {\bf антисиметрична}, ако
  \[(\forall x,y\in A)[\ \pair{x,y}\in R\ \&\ \pair{y,x}\in R \rightarrow x = y\ ].\]
  Например, релацията $\leq\ \subseteq\ \Nat\times\Nat$ е антисиметрична, защото
  \[(\forall x,y,z\in A)[\ x \leq y\ \&\ y \leq x\ \rightarrow\ x = y\ ].\]
% \item
%   {\bf асиметрична}, ако
%   \[(\forall x,y)[\pair{x,y}\in R \rightarrow \pair{y,x}\not\in R].\]
%   Например, релацията $\leq\ \subseteq\ \Nat\times\Nat$ е асиметрична, защото
%   \[(\forall x,y\in \Nat)[x < y\ \rightarrow\ y \not< x].\]
\end{enumerate}

% \begin{remark}
%   Добре е да запомните как се наричат тези основни видове релации,
%   защото ще ги използваме често.
% \end{remark}

% \begin{example}
%   Да обобщим примерите от по-горе.
%   \begin{enumerate}[a)]
%   \item
%     Релацията $\leq\ \subseteq\ \Nat\times\Nat$ е рефлексивна, транзитивна и антисиметрична.
%   \item
%     Релацията $<\ \subseteq\ \Nat\times\Nat$ е антирефлексивна, транзитивна и асиметрична.
%   \item
%     Релацията $=\ \subseteq\ \Nat\times\Nat$ е рефлексивна, транзитивна и симетрична.
%   \end{enumerate}
% \end{example}

\begin{itemize}
\item
  Една бинарна релация $R$ над множеството $A$ се нарича {\bf релация на еквивалентност}, 
  ако $R$ е рефлексивна, транзитивна и симетрична.
\item 
  За всеки елемент $a \in A$, определяме неговия 
  {\bf клас на еквивалентност} относно релацията на еквивалентност $R$ по следния начин:
  \[[a]_R \df \{b\in A \mid \pair{a,b} \in R\}.\]
\end{itemize}

\begin{remark}
  Лесно се съобразява, че за всеки два елемента $a, b\in A$,
  \[\pair{a,b} \in R \iff [a]_R = [b]_R.\]
\end{remark}

\begin{example}
  За всяко естествено число $n\geq 2$, дефинираме релацията $R_n$ като
  \[\pair{x,y}\in R_n \iff x \equiv y\ (\bmod\ n).\]
  Ясно е, че $R_n$ са релации на еквивалентност.
\end{example}


\subsection*{Операции върху бинарни релации}

\begin{enumerate}[I)]
\item
  \mynote{Това е малко объркващо}
  {\bf Композиция} на две релации $R \subseteq B\times C$ и $P \subseteq A\times B$ е релацията $R\circ P \subseteq A\times C$,
  определена като:
  \[R\circ P \df \{\pair{a,c} \in A\times C \mid (\exists b \in B)[\pair{a,b}\in P\ \&\ \pair{b,c} \in R]\}.\]
% \item
%   {\bf Обръщане} на релацията $R \subseteq A\times B$ е релацията $R^{-1}\subseteq B\times A$, 
%   определена като:
%   \[R^{-1} \df \{\pair{x,y} \in B\times A \mid \pair{y,x} \in R\}.\]
\item
  \mynote{Очевидно е, че $P$ е рефлексивна релация, дори ако $R$ не е.}
  {\bf Рефлексивно затваряне} на релацията $R \subseteq A\times A$ е релацията
  \[P \df R \cup \{\pair{a,a}\mid a \in A\}.\]
\item
  {\bf Итерация} на релацията $R \subseteq A\times A$ дефинираме като за всяко естествено число $n$,
  дефинираме релацията $R^n$ по следния начин:
  \mynote{Лесно се вижда, че  $R^1 = R$}
  \begin{align*}
    R^0 & \df \{\pair{a,a} \mid a \in A\}\\
    R^{n+1} & \df R^n \circ R.
  \end{align*}
\item
  \mynote{\writedown Проверете, че $R^+$ е транзитивна релация!}
  {\bf Транзитивно затваряне} на $R \subseteq A\times A$ е релацията
  \[R^+ \df \bigcup_{n\geq 1} R^n.\]
\end{enumerate}

\index{$R^\star$}
За дадена релация $R$, с $R^\star$ ще означаваме нейното {\em рефлексивно и транзитивно затваряне}.
От дефинициите е ясно, че \[R^\star = \bigcup_{n\geq 0} R^n.\]

%%% Local Variables:
%%% mode: latex
%%% TeX-master: "../eai"
%%% End:
