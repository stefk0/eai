\section{Системи от регулярни изрази}

\mynote{Тук се описва един по-алгебричен подход за описание на регулярните езици \cite[стр. 100]{sakarovitch-book}.}
\begin{important}
  \begin{lemma}[Ардън \cite{arden}]
    \label{lem:regular:system:arden}
    Нека $\mathbf{r}$  и $\mathbf{p}$ са регулярни изрази.
    Тогава регулярният израз $\mathbf{r}^\star \mathbf{p}$ е най-малкото решение на уравнението $X = \mathbf{r} \cdot X + \mathbf{p}$.
    Ако $\varepsilon  \not \in \L(\mathbf{r})$, то това решение е и единствено.
  \end{lemma}
\end{important}
\begin{hint}
  Първо, да видим защо $\mathbf{r}^\star \mathbf{p}$ е решение. Но това е лесно.
  Просто заместваме $X$ с $\mathbf{r}^\star \mathbf{p}$ и получаваме равенствата:
  \begin{align*}
    \mathbf{r}^\star \mathbf{p} & = \mathbf{r} \cdot \mathbf{r}^\star \mathbf{p} + \mathbf{p}\\
                                & = \mathbf{r}^+ \cdot \mathbf{p} + \mathbf{\varepsilon} \cdot \mathbf{p}\\
                                & = (\mathbf{r}^+ + \mathbf{\varepsilon}) \cdot \mathbf{p}\\
                                & = \mathbf{r}^\star \cdot \mathbf{p}.
  \end{align*}
  Второ, да видим защо $\mathbf{r}^\star \cdot \mathbf{p}$ е най-малкото решение.
  И така, нека приемем, че $\mathbf{y}$ е друго решение на системата, т.е.
  \[\mathbf{y} = \mathbf{r} \cdot \mathbf{y} + \mathbf{p}.\]
  Трябва да проверим, че $\L(\mathbf{r}^\star \cdot \mathbf{p}) \subseteq \L(\mathbf{y})$.
  Тук ще използваме представянето
  \[\L(\mathbf{r}^\star \cdot \mathbf{p}) = \bigcup_n \L(\mathbf{r}^n \cdot \mathbf{r}).\]
  Ще докажем с индукция по $n$, че за всяко $n$ е изпълнено:
  \[\L(\mathbf{r}^n \cdot \mathbf{p}) \subseteq \L(\mathbf{y}).\]           
  \begin{itemize}
  \item
    Нека $n = 0$. Понеже $\mathbf{y} = \mathbf{r} \cdot \mathbf{y} + \mathbf{p}$ е ясно, че $\L(\mathbf{p}) \subseteq \L(\mathbf{y})$.
  \item
    Нека сега $n > 0$. От \IndHyp имаме, че $\L(\mathbf{r}^{n-1} \cdot \mathbf{p}) \subseteq \L(\mathbf{y})$. Тогава получаваме, че
    \begin{align*}
      \L(\mathbf{r} \cdot \mathbf{r}^{n-1} \cdot \mathbf{p}) & \subseteq \L(\mathbf{r} \cdot \mathbf{y})\\
                                                             & \subseteq \L(\mathbf{r}\cdot\mathbf{y}+\mathbf{p})\\
                                                             & = \L(\mathbf{y}).
    \end{align*}
    Така заключаваме, че за всяко $n$, $\L(\mathbf{r}^n \cdot p) \subseteq \L(\mathbf{y})$.
  \end{itemize}

  \mynote{Да отбележим, че ако $\varepsilon \in \L(\mathbf{r})$, то $\Sigma^\star$ е решение на системата.}
  Сега остана да докажем третата част. Нека $\varepsilon \not\in \L(\mathbf{r})$.
  Ще докажем, че ако $\mathbf{y}$ е решение на системата, то $\mathbf{y} = \mathbf{r}^\star \cdot \mathbf{p}$. Понеже вече знаем, че $\mathbf{r}^\star \cdot \mathbf{p}$ е най-малкото решение на системата, то е достатъчно да докажем, че за всяко решение $\mathbf{y}$ на системата е изпълнено, че $\L(\mathbf{y}) \subseteq \L(\mathbf{r}^\star \cdot \mathbf{p})$.

  Ще докажем с индукция по дължината на думата $\alpha$, че
  \[(\forall \alpha \in \Sigma^\star)[\alpha \in \L(\mathbf{y}) \implies \alpha \in \L(\mathbf{r}^\star \cdot \mathbf{p})].\]
  Нека $|\alpha| = 0$, т.е. $\alpha = \varepsilon$.
  Да приемем, че $\varepsilon \in \L(\mathbf{y})$.
  Щом $\mathbf{y}$ е решение, то $\varepsilon \in \L(\mathbf{r} \cdot \mathbf{y} + \mathbf{p})$,
  но понеже $\varepsilon \not \in \L(\mathbf{r})$, то със сигурност $\varepsilon \in \L(\mathbf{p})$.
  Така получаваме, че
  \[\varepsilon \in \L(\mathbf{y}) \implies \varepsilon\in\L(\mathbf{r}^\star \cdot \mathbf{p}).\]
  Нека приемем като индукционно предположение, че 
  \[(\forall \alpha \in \Sigma^{\leq n})[\alpha \in \L(\mathbf{y}) \implies \alpha \in \L(\mathbf{r}^\star \cdot \mathbf{p})].\]
  Да вземем сега дума $\alpha \in \L(\mathbf{y})$ с $|\alpha| = n+1$. Ако такива думи няма, то импликацията е изпълнена по тривиални причини.
  Отново, щом $\alpha \in \L(\mathbf{r} \cdot \mathbf{y} + \mathbf{p})$, то имаме два случая.
  Ако $\alpha \in \L(\mathbf{p})$, то всичко е ясно.
  Нека $\alpha\in\L(\mathbf{r}\cdot \mathbf{y})$. Тогава $\alpha = \alpha_1 \cdot \alpha_2 $ и
  $\alpha_1 \in \L(\mathbf{r})$ и $\alpha_2 \in \L(\mathbf{y})$.
  Понеже $\varepsilon \not\in\L(\mathbf{r})$, то $|\alpha_1| \geq 1$ и следователно $|\alpha_2| < |\alpha|$.
  Това означава, че от \IndHyp за $\alpha_2$ получаваме, че $\alpha_2 \in \L(\mathbf{r}^\star\cdot \mathbf{p})$,
  откъдето заключаваме, че $\alpha = \alpha_1 \cdot \alpha_2 \in \L(\mathbf{r}^\star \cdot \mathbf{p})$.
\end{hint}

Тази лема ни дава един нов начин за намиране на регулярен израз за автоматен език.


\begin{example}
Да разгледаме отново автомата от \Figure{a2}.

\begin{figure}[H]
  \begin{center}
    \begin{tikzpicture}[framed,->,>=stealth,thick,node distance=45pt,initial text=начало]
      \tikzstyle{every state}=[circle,minimum size=15pt,auto]
      
      \node[initial,state]      (1) {$q_0$};
      \node[state]              (2) [right of=1]{$q_1$};
      \node[accepting, state]   (3) [right of=2]{$q_2$};
      
      \path 
      (2) edge [loop above]    node [above] {$a$} (2)
      (1) edge [bend left=15]  node [above] {$a$} (2)
      (2) edge [bend left=15]  node [above] {$b$} (3)
      (1) edge [bend right=45] node [below] {$b$} (3)
      (3) edge [loop above]    node [above] {$a,b$} (3);
    \end{tikzpicture}
  \end{center}
  \caption{$\L(\A) \stackrel{?}{=} \L(\mathbf{a^\star b(a+b)^\star})$.}
\end{figure}

На него съответства следната система:
\begin{align*}
  & X_0 = a \cdot X_1 + b \cdot X_2 + \emptyset\\
  & X_1 = a \cdot X_1 + b \cdot X_2 + \emptyset\\
  & X_2 = a \cdot X_2 + b \cdot X_2 + \varepsilon.
\end{align*}

Чрез прилагане на \hyperref[lem:regular:system:arden]{лемата на Ардън} само за последното уравнение и заместване получаваме следното:

\begin{align*}
  & X_0 = a \cdot X_1 + b \cdot (a+b)^\star\\
  & X_1 = a \cdot X_1 + b \cdot (a+b)^\star\\
  & X_2 = (a+b)^\star.
\end{align*}

След това, прилагаме \hyperref[lem:regular:system:arden]{лемата на Ардън} за второто уравнение. Така получаваме, след заместване, следните уравнения:

\begin{align*}
  & X_0 = a \cdot a^\star \cdot b \cdot (a+b)^\star + b \cdot (a+b)^\star\\
  & X_1 = a^\star \cdot b \cdot (a+b)^\star\\
  & X_2 = (a+b)^\star.
\end{align*}

Заключаваме, че езикът на автомата е
\[(a \cdot a^\star + \varepsilon) \cdot b \cdot (a+b)^\star = a^\star \cdot b \cdot (a+b)^\star.\]
  
\end{example}



%%% Local Variables:
%%% mode: latex
%%% TeX-master: "../eai"
%%% End:
