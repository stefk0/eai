\section{Системи от регулярни изрази}

\mynote{\cite[стр. 100]{sakarovitch-book}}
В този раздел ще опишем един алгебричен поглед към регулярните езици, с чиято помощ можем да построим регулярен израз по даден краен автомат.

\begin{important}
  \begin{lemma}[Ардън \cite{arden}]
    \index{Ардън}
    \label{lem:regular:system:arden}
    Нека $\bm{r}$  и $\bm{p}$ са регулярни изрази.
    Тогава регулярният израз $\bm{r}^\star \cdot \bm{p}$ е най-малкото решение на уравнението
    \begin{equation}
      \label{eq:system:arden}
      X = \bm{r} \cdot X + \bm{p}.
    \end{equation}
    Ако $\varepsilon  \not \in \L(\bm{r})$, то това решение е и единствено.
  \end{lemma}
\end{important}
\begin{hint}
  Първо, да видим защо $\bm{r}^\star \bm{p}$ е решение на уравнението (\ref{eq:system:arden}).
  Това е лесно. Просто заместваме променливата $X$ с $\bm{r}^\star \bm{p}$ и получаваме равенствата:
  \begin{align*}
    \bm{r}^\star \bm{p} & = \bm{r} \cdot \bm{r}^\star \bm{p} + \bm{p}\\
                                & = \bm{r}^+ \cdot \bm{p} + \bm{\varepsilon} \cdot \bm{p}\\
                                & = (\bm{r}^+ + \bm{\varepsilon}) \cdot \bm{p}\\
                                & = \bm{r}^\star \cdot \bm{p}.
  \end{align*}
  Второ, да видим защо $\bm{r}^\star \cdot \bm{p}$ е най-малкото решение на уравнението (\ref{eq:system:arden}).
  За целта, нека приемем, че $\bm{y}$ е друго решение, т.е.
  \[\bm{y} = \bm{r} \cdot \bm{y} + \bm{p}.\]
  Трябва да проверим, че $\L(\bm{r}^\star \cdot \bm{p}) \subseteq \L(\bm{y})$.
  Тук ще използваме представянето
  \[\L(\bm{r}^\star \cdot \bm{p}) = \bigcup_n \L(\bm{r}^n \cdot \bm{r}).\]
  Ще докажем с индукция по $n$, че за всяко $n$ е изпълнено включването:
  \[\L(\bm{r}^n \cdot \bm{p}) \subseteq \L(\bm{y}).\]           
  \begin{itemize}
  \item
    Нека $n = 0$. Понеже $\bm{y} = \bm{r} \cdot \bm{y} + \bm{p}$ е ясно, че $\L(\bm{p}) \subseteq \L(\bm{y})$
    или с други думи, $\L(\bm{r}^0 \cdot \bm{p}) \subseteq \L(\bm{y})$.
  \item
    Нека сега $n > 0$. От \IndHyp имаме, че $\L(\bm{r}^{n-1} \cdot \bm{p}) \subseteq \L(\bm{y})$. Тогава,
    като конкатенираме с $\L(\bm{r})$ от двете страни, получаваме:
    \begin{align*}
      \L(\bm{r} \cdot \bm{r}^{n-1} \cdot \bm{p}) & \subseteq \L(\bm{r} \cdot \bm{y})\\
                                                             & \subseteq \L(\bm{r}\cdot\bm{y}+\bm{p})\\
                                                             & = \L(\bm{y}).
    \end{align*}
    Така заключаваме, че за всяко $n$, е изпълнено включването $\L(\bm{r}^n \cdot p) \subseteq \L(\bm{y})$ и следователно $\L(\bm{r}^\star \cdot \bm{p}) \subseteq \L(\bm{y})$.
  \end{itemize}

  \mynote{Да отбележим, че ако $\varepsilon \in \L(\bm{r})$, то $\Sigma^\star$ е решение на системата.}
  Остана да докажем, че ако $\varepsilon \not\in \L(\bm{r})$, to $\bm{r}^\star \cdot \bm{p}$ е единственото решение на уравнението (\ref{eq:system:arden}).
  Ще направим това като докажем, че всяко
  решение $\bm{y}$ на уравнението (\ref{eq:system:arden}) е такова, че $\bm{y} = \bm{r}^\star \cdot \bm{p}$.
  Понеже вече знаем, че $\bm{r}^\star \cdot \bm{p}$ е най-малкото решение на (\ref{eq:system:arden}), то е достатъчно да докажем, че за всяко решение $\bm{y}$ на (\ref{eq:system:arden}) е изпълнено, че
  $\L(\bm{y}) \subseteq \L(\bm{r}^\star \cdot \bm{p})$, защото включването $\L(\bm{r}^\star \cdot \bm{p}) \subseteq \L(\bm{y})$ следва от факта, че $\bm{r}^\star \cdot \bm{p}$ е най-малкото решение.

  Щом трябва да докажем, че $\L(\bm{y}) \subseteq \L(\bm{r}^\star \cdot \bm{p})$,
  то това означава да докажем
  \[(\forall \alpha \in \Sigma^\star)[\alpha \in \L(\bm{y}) \implies \alpha \in \L(\bm{r}^\star \cdot \bm{p})].\]
  И така, ще докажем с индукция по $n$, че
  \[(\forall n)(\forall \alpha \in \Sigma^{\leq n})[\alpha \in \L(\bm{y}) \implies \alpha \in \L(\bm{r}^\star \cdot \bm{p})].\]
  \begin{itemize}
  \item
    Нека $n = 0$ и да вземем дума $\alpha \in \Sigma^{\leq 0}$, което означава, че $\alpha = \varepsilon$.
    Да приемем, че $\varepsilon \in \L(\bm{y})$, защото в противен случай импликацията е автоматично изпълнена.
    Щом $\bm{y}$ е решение, то $\varepsilon \in \L(\bm{r} \cdot \bm{y} + \bm{p})$,
    но понеже $\varepsilon \not \in \L(\bm{r})$, то със сигурност $\varepsilon \in \L(\bm{p})$.
    Така получаваме, че
    \[\varepsilon \in \L(\bm{y}) \implies \varepsilon\in\L(\bm{r}^\star \cdot \bm{p}),\]
    откъдето веднага следва, че
    \[(\forall \alpha \in \Sigma^{\leq 0})[\alpha \in \L(\bm{y}) \implies \alpha \in \L(\bm{r}^\star \cdot \bm{p})].\]
  \item
    Нека сега $n > 0$, като приемем, че имаме индукционно предположение, т.е.
    \[(\forall \alpha \in \Sigma^{\leq n-1})[\alpha \in \L(\bm{y}) \implies \alpha \in \L(\bm{r}^\star \cdot \bm{p})].\]
    Ще докажем, че
    \[(\forall \alpha \in \Sigma^{\leq n})[\alpha \in \L(\bm{y}) \implies \alpha \in \L(\bm{r}^\star \cdot \bm{p})].\]
    За целта, да вземем произволна дума $\alpha \in \Sigma^{\leq n}$.
    Ако всъщност $|\alpha| < n$, то от \IndHyp веднага следва, че импликацията е изпълнена за $\alpha$.

    Нека $|\alpha| = n$. Ако $\alpha \not\in \L(\bm{y})$, то отново импликацията е изпълнена по
    тривиални причини.
    Интересният случай е когато $|\alpha| = n$ и $\alpha \in \L(\bm{y})$.
    Понеже $\bm{y}$ е решение на (\ref{eq:system:arden}), то $\alpha \in \L(\bm{r} \cdot \bm{y} + \bm{p})$. Сега имаме два случая.
    \begin{itemize}
    \item
      Ако $\alpha \in \L(\bm{p})$, то всичко е ясно, защото тогава $\alpha \in \L(\bm{r}^\star \cdot \bm{p})$.
    \item
      Ако $\alpha\in\L(\bm{r}\cdot \bm{y})$, то $\alpha = \alpha_1 \cdot \alpha_2 $, за някои думи $\alpha_1$ и $\alpha_2$, и $\alpha_1 \in \L(\bm{r})$ и $\alpha_2 \in \L(\bm{y})$.
      Понеже $\varepsilon \not\in\L(\bm{r})$, то $|\alpha_1| \geq 1$ и следователно $|\alpha_2| < |\alpha| = n$. Това означава, че от \IndHyp получаваме, че $\alpha_2 \in \L(\bm{r}^\star\cdot \bm{p})$,
    откъдето заключаваме, че $\alpha = \alpha_1 \cdot \alpha_2 \in \L(\bm{r} \cdot \bm{r}^\star \cdot \bm{p}) \subseteq \L(\bm{r}^\star \cdot \bm{p})$.
    \end{itemize}
  \end{itemize}
\end{hint}

\begin{problem}
  Нека $\bm{r}$  и $\bm{p}$ са регулярни изрази.
  Тогава регулярният израз $\bm{p} \cdot \bm{r}^\star$ е най-малкото решение на уравнението
  \[X = X \cdot \bm{r} + \bm{p}.\]
  Ако $\varepsilon  \not \in \L(\bm{r})$, то това решение е и единствено.
\end{problem}

\hyperref[lem:regular:system:arden]{Лемата на Ардън} ни дава един нов начин за намиране на регулярен израз описващ езика на даден автомат.

\begin{example}
Да разгледаме отново автомата от \Figure{a2}.

\begin{figure}[H]
  \begin{center}
    \begin{tikzpicture}[framed,->,>=stealth,thick,node distance=45pt,initial text=начало]
      \tikzstyle{every state}=[circle,minimum size=15pt,auto]
      
      \node[initial,state]      (1) {$q_0$};
      \node[state]              (2) [right of=1]{$q_1$};
      \node[accepting, state]   (3) [right of=2]{$q_2$};
      
      \path 
      (2) edge [loop above]    node [above] {$a$} (2)
      (1) edge [bend left=15]  node [above] {$a$} (2)
      (2) edge [bend left=15]  node [above] {$b$} (3)
      (1) edge [bend right=45] node [below] {$b$} (3)
      (3) edge [loop above]    node [above] {$a,b$} (3);
    \end{tikzpicture}
  \end{center}
  \caption{$\L(\A) \stackrel{?}{=} \L(\bm{a^\star b(a+b)^\star})$.}
\end{figure}

На него съответства следната система:
\begin{align*}
  & L_0 = a \cdot L_1 + b \cdot L_2 + \emptyset\\
  & L_1 = a \cdot L_1 + b \cdot L_2 + \emptyset\\
  & L_2 = a \cdot L_2 + b \cdot L_2 + \varepsilon.
\end{align*}


Да разгледаме само последния ред на системата:
\[L_2 = (a+b) \cdot L_2 + \varepsilon.\]
Чрез прилагане на \hyperref[lem:regular:system:arden]{лемата на Ардън} получаваме, че
единственото решение на това уравнение е $(a+b)^\star$.
В първите два реда на системата заместваме $L_2$ с $(a+b)^\star$ и получаваме системата:
\begin{align*}
  & L_0 = a \cdot L_1 + b \cdot (a+b)^\star\\
  & L_1 = a \cdot L_1 + b \cdot (a+b)^\star\\
  & L_2 = (a+b)^\star.
\end{align*}

Сега вече би трябвало да е ясно как продължаваме. Разглеждаме само втория ред на системата и 
прилагаме \hyperref[lem:regular:system:arden]{лемата на Ардън} за него и получаваме, че единственото
решение на втория ред от системата е $a^\star \cdot b \cdot (a+b)^\star$.
Така получаваме, след заместване, следните уравнения:

\begin{align*}
  & L_0 = a \cdot a^\star \cdot b \cdot (a+b)^\star + b \cdot (a+b)^\star\\
  & L_1 = a^\star \cdot b \cdot (a+b)^\star\\
  & L_2 = (a+b)^\star.
\end{align*}

Заключаваме, че езикът на автомата $\A$ е
\[\L_\A(q_0) = L_0 = (a \cdot a^\star + \varepsilon) \cdot b \cdot (a+b)^\star = a^\star \cdot b \cdot (a+b)^\star.\]
\end{example}


Този пример ни подсказва какво да правим в общия случай.
Нека имаме ДКА $\A$. Да индексираме състоянията на $\A$ така:
\[Q = \{q_0,q_1,\dots,q_n\}.\]
За произволни $i,j \leq n$, за удобство да положим:
\begin{align*}
  & K_{ij} \df \{a \in \Sigma \mid \delta(q_i,a) = q_j\}\\
  & L_i \df \L_\A(q_i)\\
  & E_i \df
    \begin{cases}
      \{\varepsilon\}, & q_i \in F\\
      \emptyset, & q_i \not\in F.
    \end{cases}
\end{align*}

Лесно е да се убедим в правотата на следващото твърдение.

\begin{proposition}
  За всяко $i \leq n$ е изпълнено равенството:
  \[L_i = K_{i0}\cdot L_{0} \cup K_{i1} \cdot L_{1} \cup \cdots K_{in}\cdot L_{n} \cup E_{i}.\]
\end{proposition}

Понеже $\varepsilon \not\in K_{ij}$, то тази система има едиствено решение, което знаем
как да намерим по лемата на Ардън. Това решение представлява една редица от регулярни езици.
Така ние дадохме второ доказателство на теоремата на Клини.

\begin{important}
  \begin{theorem}[Клини]
    Всеки автоматен език е регулярен.
  \end{theorem}
\end{important}


%%% Local Variables:
%%% mode: latex
%%% TeX-master: "../eai"
%%% End:
