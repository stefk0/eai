\section{Един критерий за нерегулярност}

\begin{lemma}[Лема за покачването]
  \index{лема за покачването!регулярни езици}
  \label{lem:pumping-reg}
  \marginpar{На англ. се нарича \\ Pumping Lemma}
  \marginpar{Има подобна лема и за безконтекстни езици, която ще разгледаме по-нататък.}
  Нека $L$ да бъде {\em безкраен} регулярен език.
  Съществува число $p\geq 1$, зависещо само от $L$, 
  за което за всяка дума $\alpha\in L, \abs{\alpha}\geq p$ може да 
  бъде записана във вида $\alpha = xyz$ и 
  \begin{enumerate}[1)]
  \item
    $|y|\geq 1$;
  \item
    $|xy|\leq p$;
    \marginpar{Обърнете внимание, че $0 \in \Nat$ и $xy^0z =  xz$}
  \item
    $(\forall i\in\Nat)[xy^iz \in L]$.
  \end{enumerate}
\end{lemma}
\begin{hint}
  \marginpar{Тази лема я има във всеки учебник по този предмет. Например, \cite[стр. 88]{papadimitriou}, \cite[стр. 77]{sipser3}.}
  Понеже $L$ е регулярен, то $L$ е и автоматен език. Нека
  \[\A = \FA\]
  е краен детерминиран автомат, за който $L = \L(\A)$.
  Да положим $p = \abs{Q}$ и нека $\alpha = a_1a_2\cdots a_k$ е дума, за която $k \geq p$.
  Да разгледаме първите $p$ стъпки от изпълнението на $\alpha$ върху $\A$:
  \[\qstart\stackrel{a_1}{\rightarrow} q_1 \stackrel{a_2}{\rightarrow}q_2 \dots \stackrel{a_p}{\rightarrow} q_p.\]
  Тъй като $\abs{Q} = p$, а по този път участват $p+1$ състояния $q_0,q_1,\dots,q_p$,
  то съществуват числа $i, j$, за които $0\leq i < j\leq p$ и $q_i = q_j$.
  Нека разделим думата $\alpha$ на три части по следния начин:
  \[\underbrace{a_1\cdots a_i}_{x}\quad \underbrace{a_{i+1}\cdots a_j}_{y}\quad \underbrace{a_{j+1}\cdots a_k}_{z}.\]
  Ясно е, че $\abs{y} \geq 1$ и $\abs{xy} = j \leq p$.
  Да разгледаме случая за $i = 0$.
  Думата $xy^0z = xz \in L$, защото имаме следното изчисление:
  \[\qstart\underbrace{\stackrel{a_1}{\rightarrow}q_1 \cdots \stackrel{a_i}{\rightarrow}}_{x} q_i\underbrace{\stackrel{a_{j+1}}{\rightarrow}q_{j+1}\cdots\stackrel{a_{k}}{\rightarrow}}_{z}q_k\in F,\]
  защото $q_i = q_j$.
  Да разгледаме и случая $i = 2$. Тогава думата $xy^2z \in L$, защото имаме следното изчисление:
  \[\qstart\underbrace{\stackrel{a_1}{\rightarrow}q_1 \cdots \stackrel{a_i}{\rightarrow}}_{x} q_i\underbrace{\stackrel{a_{i+1}}{\rightarrow}q_{i+1}\cdots\stackrel{a_{j}}{\rightarrow}}_{y}q_j\underbrace{\stackrel{a_{i+1}}{\rightarrow}q_{i+1}\cdots\stackrel{a_{j}}{\rightarrow}}_{y}q_j\underbrace{\stackrel{a_{j+1}}{\rightarrow}\cdots\stackrel{a_{k}}{\rightarrow}}_{z}q_k\in F.\]
  \marginpar{\writedown Докажете!}
  Вече лесно можем да съобразим, че за всяко естествено число $i$, е изпълнено $xy^iz \in \L(\A)$.
\end{hint}

Практически е по-полезно да разглеждаме следната еквивалентна формулировка на лемата за покачването.

\begin{cor}[Контрапозиция на лемата за покачването]
  \label{cor:pumping-reg}
  \marginpar{Ясно е, че всеки краен език е регулярен. Нали?}
  Нека $L$ е произволен {\bf безкраен} език. Нека също така е изпълнено, че:
  \begin{description}
  \item[($\forall$)]
    за {\em всяко} естествено число $p \geq 1$,
  \item[($\exists$)]
    можем да намерим дума $\alpha \in L$, $\abs{\alpha}\geq p$, такава че
  \item[($\forall$)]
    за {\em всяко} разбиване на думата на три части, $\alpha = xyz$, със свойствата $\abs{xy} \leq p$ и $\abs{y} \geq 1$,
  \item[($\exists$)]
    можем да посочим $i \in \Nat$, за което е изпълнено, че $xy^iz \not\in L$.
  \end{description}  
  Тогава $L$ {\bf не} е регулярен език.
\end{cor}
\begin{proof}
  Да означим с $P_{\text{reg}}(L)$ следната формула:
  \begin{align*}
    (\exists p \geq 1)(\forall \alpha \in L)[\abs{\alpha} \geq p \Rightarrow (\exists x,y,z\in\Sigma^\star)[\ & \alpha = xyz\ \& \\
                                                                                                              & \abs{y} \geq 1\ \&\\
                                                                                                              & \abs{xy} \leq p\ \&\\
                                                                                                              & (\forall i\in\Nat)[xy^iz \in L]]].
  \end{align*}
  Така условието на \hyperref[lem:pumping-reg]{Лемата за покачването} представлява твърдението:
  \begin{center}
    {\em ,,Aко $L$ е регулярен език, то е изпълнено $P_{\text{reg}}(L)$.''}
  \end{center}
  \marginpar{Контрапозиция на твърдението $P \to Q$ е твърдението $\neg Q \to \neg P$}
  \noindent
  Лемата може да се запише по следния еквивалентен начин:
  
  \begin{center}
    {\em ,,Ако $P_{\text{reg}}(L)$ не е изпълнено, то $L$ не е регулярен език.''}
  \end{center}
  \marginpar{Използваме, че $\neg \exists \forall \exists \forall (\dots) \equiv \forall \exists \forall \exists \neg(\dots)$}
  Отрицанието на $P_{\text{reg}}(L)$ преставлява формулата
  \begin{align*}
    (\forall p \geq 1)(\exists \alpha \in L)[\ \abs{\alpha} \geq p\ \&\ (\forall x,y,z\in\Sigma^\star)[\ & \alpha \neq xyz\ \vee\\
                                                                                                         & \abs{y} \not\geq 1\ \vee\\
                                                                                                         & \abs{xy} \not\leq p\ \vee\\
                                                                                                         & (\exists i\in\Nat)[xy^iz \not\in L]\ ]\ ].
  \end{align*}
  Горната формула е еквивалентна на:
  \marginpar{Използваме, че
    \begin{align*}
      \neg P \vee \neg Q \vee R & \equiv \neg(P\ \&\ Q) \vee R\\
                                & \equiv (P \wedge Q) \to R
    \end{align*}}
  \begin{align*}
    (\forall p \geq 1)(\exists \alpha \in L)[\ \abs{\alpha} \geq p\ \&\ (\forall x,y,z\in\Sigma^\star)[\ & (\alpha = xyz \wedge \abs{y} \geq 1\wedge \abs{xy} \leq p)\\
                                                                                                         & \Rightarrow (\exists i\in\Nat)[xy^iz \not\in L]\ ]\ ].
  \end{align*}

  Това означава, че ако
  \begin{description}
  \item[($\forall$)]
    вземем произволна константа $p \geq 1$,
  \item[($\exists$)]
    за нея намерим дума $\alpha \in L$, такава че $\abs{\alpha} \geq p$ и 
  \item[($\forall$)]
    докажем, че за всяко нейно разбиване на три части $x,y,z$, със свойствата
    $\abs{y} \geq 1$ и $\abs{xy} \leq p$,
  \item[($\exists$)]
    можем да намерим естествено число $i$, за което $xy^iz \not\in L$,
  \end{description}
  то можем да заключим, че езикът $L$ не е регулярен.
\end{proof}

%%% Local Variables:
%%% mode: latex
%%% TeX-master: "../eai"
%%% End:
