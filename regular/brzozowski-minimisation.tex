\subsection{Експоненциален алгоритъм за минимизация}



\begin{itemize}
\item
  Нека имаме ДКА $\A = \FA$. Да приемем, че $\A$ не е минимален автомат за езика $\L(\A)$.
  Това означава, че съществуват различни състояния $q_\alpha$ и $q_\beta$, за които $\L_\A(q_\alpha) = \L_\A(q_\beta)$.
  С други думи,
  \begin{equation}
    \label{eq:brzozowski-minimisation:equiv-states}
    (\forall \omega)[\delta^\star(q_\alpha,\omega) \in F \iff \delta^\star(q_\beta,\omega) \in F].
  \end{equation}
\item
  Нека $\N_1 = (\Sigma, Q, F, \Delta_1, \{\qstart\})$, където
  $\Delta_1(q,x) \df \{p \in Q\mid \delta(p,x) = q\}$
  и следователно
  \[\Delta^\star_1(U,\alpha) \df \bigcup\{p \in Q\mid \delta^\star(p,\alpha) \in U\}.\]
  \begin{equation}
    \label{eq:equiv-states:rev1}
    (\forall \omega)[\Delta_1^\star(F,\omega) \ni q_\alpha \iff \Delta_1^\star(F,\omega) \ni q_\beta].
  \end{equation}
\item
  Нека $\D_2 = (\Sigma, Q_2, \code{F}, \delta_2, F_2)$, където
  \begin{itemize}
  \item
    $Q_2 = \{\code{U} \mid (\exists \omega)[\Delta^\star_1(F,\omega) = U]\}$;
  \item
    $F_2 = \{\code{U} \in Q_2 \mid \qstart \in U\}$;
  \item
    $\delta_2(\code{U},x) = \code{\Delta^\star_1(U,x)}$.
  \end{itemize}
  Получаваме, че за едно състояние $\code{U}$,
  \begin{align*}
    q_\alpha \in U & \iff \delta^\star_2(\code{U},\alpha) \in F_2\\
                   & \iff \delta^\star_2(\code{U},\alpha) \ni \qstart.
  \end{align*}
\item
  Нека $\N_3 = (\Sigma, Q_2, F_2, \Delta_3, \{\code{F}\})$, където
  \[\Delta_3(\code{U},x) = \{\code{V} \mid \delta_2(\code{V},x) = \code{U}\}).\]
  Сега имаме, че
  \begin{align*}
    \Delta^\star_3(F_2,\alpha) & = \{\code{U} \mid \delta^\star_2(\code{U},\alpha) = \code{K}\ \&\ \code{K} \in F_2\}\\
                               & = \{\code{U} \mid \delta^\star_2(\code{U},\alpha) = \code{K}\ \&\ \qstart \in K\}\\
                               & = \{\code{U} \mid q_\alpha \in U\}\\
                               & = \{\code{U} \mid q_\beta \in U\}\\
                               & = \Delta^\star_3(F_2,\beta).
  \end{align*}
\end{itemize}

Дефинираме $\B = \rev(\A)$ да бъде ДКА по следния начин:
\begin{itemize}
\item
  Състоянията на $\B$ ще бъдат подмножества на състоянията на $\A$.

\item
  Да положим $Q_\alpha \df \{q \in Q^\A \mid \delta^\star_\A(q,\alpha) \in F\}$.
  Тогава
  $Q^\B \df \{Q_\alpha \mid \alpha \in \Sigma^{\star}\}$.
\item
  $\qstart^\B \df F^\A = Q_\varepsilon$.
\item
  За произволно $R \in Q^\B$,
  $\delta_\B(R,a) \df \{q \in Q^\A \mid \delta_\A(q,a) \in R\}$.
  С други думи,
  $\delta_\B(Q_\beta,a) = Q_{a\beta}$.
\item
  $F^\B = \{Q_\alpha \mid \qstart^\A \in Q_\alpha\}$.
\item
  Съобразете, че $\delta^\star_\B(Q_\alpha,\gamma) = Q_{\gamma^\rev\alpha}$.
  Тогава
  \begin{align*}
    \L(\B) & = \{\alpha \mid \delta^\star_\B(Q_\varepsilon,\alpha) = Q_{\alpha^\rev} \in F^\B\}\\
           & = \{\alpha \mid  \qstart^\A \in Q_{\alpha^\rev}\}\\
           & = \{\alpha \mid  \delta^\star(\qstart^\A,\alpha^\rev) \in F^\A\}\\
           & = \L(\A)^\rev.
  \end{align*}
\end{itemize}

\begin{problem}
  Докажете, че $\rev(\A)$ е минимален автомат за $\L(\A)^\rev$.
\end{problem}
\begin{hint}
  Достатъчно е да се докаже, че
  \[Q_\alpha = Q_\beta \iff \L_\B(Q_\alpha) = \L_\B(Q_\beta).\]
  Нека $\L_\B(Q_\alpha) = \L_\B(Q_\beta)$. Ще докажем, че $Q_\alpha \subseteq Q_\beta$.
  Нека $q \in Q_\alpha$. Това означава, че $\delta^\star_\A(q,\alpha) \in F^\A$.
  Следователно, съществува $\gamma$, за която $\delta^\star_\A(\qstart^\A,\gamma\alpha) \in F^\A$.
  От друга страна,
  \begin{align*}
    \gamma^\rev \in \L_\B(Q_\alpha) & \iff \delta^\star_\B(Q_\alpha,\gamma^\rev) \in F^\B\\
                                    & \iff Q_{\gamma\alpha} \in F^\B\\
                                    & \iff \delta^\star_\A(\qstart^\A,\gamma\alpha) \in F^\A.
  \end{align*}
\end{hint}


\begin{framed}
  \begin{theorem}[Бжозовски]
    Нека $\A$ е ДКА. Тогава $\rev(\rev(\A))$ е минимален автомат за $\L(\A)$.
  \end{theorem}
\end{framed}


%%% Local Variables:
%%% mode: latex
%%% TeX-master: "../eai"
%%% End:
