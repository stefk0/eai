\subsection{Примери}

\begin{example}
  Да разгледаме езика $L = \L(\mathbf{a\cdot(a+b)^\star\cdot b})$.
  \begin{itemize}
  \item 
    $a^{-1}(L) = \{a,b\}^\star b \df M$.
    Имаме, че $M \neq L$, защото $b \in M$, но $b \not\in L$.
    Тогава $\delta(q_L, a) \df q_M$.
    За по-нататък ще е удобно да представим $M$ по следния начин:
    \[M = a\cdot \{a,b\}^\star \cdot b \cup b\cdot \{a,b\}^\star \cdot b \cup \{b\}.\]
  \item
    $b^{-1}(L) = \emptyset$. Тогава $\delta(q_L,b) \df q_\emptyset$;
  \item
    $a^{-1}(M) = \{a,b\}^\star b = M$.
    Тогава $\delta(q_M,a) = q_M$.
  \item
    $b^{-1}(M) = \{\varepsilon\} \cup \{a,b\}^\star b \df N$.
    Ясно е, че $N \neq L,M$, защото $\varepsilon \in N$, но $\varepsilon \not\in L,M$. Тогава
    \[\delta(q_M,b) = q_N.\]
    Също така, $q_N$ е финално състояние.
    Нека за удобство да представим езика $N$ по следния начин:
    \[N = \{\varepsilon\} \cup a\cdot\{a,b\}^\star\cdot b \cup b\cdot\{a,b\}^\star\cdot b \cup \{b\}.\]
  \item
    $a^{-1}(N) = \{a,b\}^\star b = M$, т.е. $\delta(q_N,a) = q_M$;
  \item
    $b^{-1}(N) = \{a,b\}^\star b = M$, т.е. $\delta(q_N,b) = q_N$;
  \item
    Ясно е, че $a^{-1}(\emptyset) = b^{-1}(\emptyset) = \emptyset$, т.е.
    $\delta(\emptyset,a) = \delta(\emptyset,b) = \emptyset$.
  \end{itemize}
\begin{framed}
  \begin{figure}[H]
    \centering
    \begin{tikzpicture}[->,>=stealth,thick,node distance=55pt]
      \tikzstyle{every state}=[circle,minimum size=20pt,auto]
      
      \node[initial, state]                   (L) {$q_L$};
      \node[state]                            (M) [above right of=L]{$q_M$};
      \node[state]                            (E) [below right of=L]{$q_\emptyset$};
      \node[state,accepting]                  (N) [right of=M]{$q_N$};
      
      \path 
      (L) edge [bend left=15]  node [above] {$a$} (M)
      (L) edge [bend right=15] node [above] {$b$} (E)
      (E) edge [loop right]    node [right] {$a,b$} (E) 
      (M) edge [bend right=15] node [below] {$b$} (N)
      (M) edge [loop above]    node [above] {$a$} (M)
      (N) edge [bend right=30] node [above] {$a$} (M)
      (N) edge [loop above]    node [above] {$b$} (N);
    \end{tikzpicture}
    \caption{Минимален автомат за $\L(\mathbf{a\cdot (a+b)^\star\cdot b})$}
  \end{figure}
\end{framed}
\end{example}

\begin{example}
  Да разгледаме езика 
  \[L = \{\omega \in \{a,b\}^\star \mid \omega \text{ съдържа четен брой $a$ и точно едно $b$}\}.\]
  Нека да видим дали можем да построим автомат за този език.
  \begin{itemize}
  \item 
    $a^{-1}(L) = \{\alpha \in \{a,b\}^\star \mid \alpha \text{ съдържа нечетен брой $a$ и точно едно $b$}\} \df M$;
  \item 
    $b^{-1}(L) = \{\alpha \in \{a,b\}^\star \mid \alpha \text{ съдържа четен брой $a$ и нито едно $b$}\} \df N$;
  \item
    $a^{-1}(M) = L$;
  \item
    $b^{-1}(M) = \{\alpha \in \{a,b\}^\star \mid \alpha \text{ съдържа нечетен брой $a$ и нито едно $b$}\} \df P$;
  \item
    $a^{-1}(N) = P$;
  \item
    $b^{-1}(N) = \emptyset$;
  \item
    $a^{-1}(P) = N$;
  \item
    $b^{-1}(P) = \emptyset$;
  \end{itemize}

\begin{framed}
  \begin{figure}[H]
    \centering
    \begin{tikzpicture}[->,>=stealth,thick,node distance=65pt]
      \tikzstyle{every state}=[circle,minimum size=20pt,auto]
      
      \node[initial, state]        (L) {$q_L$};
      \node[state]                 (M) [above right of=L]{$q_M$};
      \node[state, accepting]      (N) [below right of=M]{$q_N$};
      \node[state]                 (P) [right of=M]{$q_P$};
      \node[state]                 (E) [right of=N]{$q_\emptyset$};
            
      \path 
      (L) edge [bend right=15]  node [below] {$a$} (M)
      (M) edge [bend right=15]  node [above] {$a$} (L)
      (L) edge [bend right=15]  node [above] {$b$} (N)
      (M) edge [bend left=15]   node [above] {$b$} (P)
      (N) edge [bend left=15]   node [left] {$a$} (P)
      (P) edge [bend left=15]   node [right] {$a$} (N)
      (P) edge [bend left=15]   node [right] {$b$} (E)
      (N) edge [bend right=15]  node [below] {$b$} (E);
    \end{tikzpicture}
    \caption{Минимален автомат, който приема думи с четен брой $a$ и точно едно $b$}
  \end{figure}  
\end{framed}
\end{example}


Можем да дадем и следната дефиниция на $\ov{\alpha}_{(2)}$:
\begin{itemize}
\item
  $\ov{\varepsilon}_{(2)} = 0$,
\item
  $\ov{0\alpha}_{(2)} = \ov{\alpha}_{(2)}$,
\item
  $\ov{1\alpha}_{(2)} = 2^{|\alpha|} + \ov{\alpha}_{(2)}$.
\end{itemize}

Тогава:
\begin{align*}
  0^{-1}(L) & = \{\alpha \in \{0,1\}^\star \mid 0\alpha \in L\}\\
            & = \{\alpha \in \{0,1\}^\star \mid \ov{0\alpha}_{(2)} \equiv 2 \bmod 3\}\\
            & = \{\alpha \in \{0,1\}^\star \mid \ov{\alpha}_{(2)} \equiv 2 \bmod 3\}\\
            & = L\\
  \\
  1^{-1}(L) & = \{\alpha \in \{0,1\}^\star \mid 1\alpha \in L\}\\
            & = \{\alpha \in \{0,1\}^\star \mid \ov{1\alpha}_{(2)} \equiv 2 \bmod 3\}\\
            & = \{\alpha \in \{0,1\}^\star \mid 2^{|\alpha|} + \ov{\alpha}_{(2)} \equiv 2 \bmod 3\}\\
            & \df L_1.
\end{align*}

Лесно се съобразява, че $L \neq L_1$, защото например за думата $\alpha = 10$
имаме, че $\alpha \in L$, но $\alpha \not\in L_1$.
Продължаваме нататък:
\begin{align*}
  0^{-1}(L_1) & = \{\alpha \in \{0,1\}^\star \mid 0\alpha \in L_1\}\\
              & = \{\alpha \in \{0,1\}^\star \mid 2^{|0\alpha|} + \ov{0\alpha}_{(2)} \equiv 2 \bmod 3\}\\
              & = \{\alpha \in \{0,1\}^\star \mid 2\cdot 2^{|\alpha|} + \ov{\alpha}_{(2)} \equiv 2 \bmod 3\}\\
              & \df L_2\\
  \\
  1^{-1}(L_1) & = \{\alpha \in \{0,1\}^\star \mid 1\alpha \in L_1\}\\
            & = \{\alpha \in \{0,1\}^\star \mid 2^{|1\alpha|} + \ov{1\alpha}_{(2)} \equiv 2 \bmod 3\}\\
            & = \{\alpha \in \{0,1\}^\star \mid 2\cdot 2^{|\alpha|} + 2^{|\alpha|} + \ov{\alpha}_{(2)} \equiv 2 \bmod 3\}\\
              & = \{\alpha \in \{0,1\}^\star \mid 3\cdot 2^{|\alpha|} + \ov{\alpha}_{(2)} \equiv 2 \bmod 3\}\\
              & = \{\alpha \in \{0,1\}^\star \mid \ov{\alpha}_{(2)} \equiv 2 \bmod 3\}\\
              & = L.
\end{align*}
Да проверим, че $L_2 \neq L$ и $L_2 \neq L_1$.
Това отново е лесно. Нека например да разгледаме $\alpha = 11$.
Непосредствено се проверява, че $\alpha \in L_2$, $\alpha \not\in L$, $\alpha \not\in L_1$.
Продължаваме нататък:
\begin{align*}
  0^{-1}(L_2) & = \{\alpha \in \{0,1\}^\star \mid 0\alpha \in L_2\}\\
              & = \{\alpha \in \{0,1\}^\star \mid 2\cdot 2^{|0\alpha|} + \ov{0\alpha}_{(2)} \equiv 2 \bmod 3\}\\
              & = \{\alpha \in \{0,1\}^\star \mid 4\cdot 2^{|\alpha|} + \ov{\alpha}_{(2)} \equiv 2 \bmod 3\}\\
              & = \{\alpha \in \{0,1\}^\star \mid 3\cdot 2^{|\alpha|} + 2^{|\alpha|} + \ov{\alpha}_{(2)} \equiv 2 \bmod 3\}\\
              & = \{\alpha \in \{0,1\}^\star \mid 2^{|\alpha|} + \ov{\alpha}_{(2)} \equiv 2 \bmod 3\}\\
              & = L_1
  \\
  1^{-1}(L_2) & = \{\alpha \in \{0,1\}^\star \mid 1\alpha \in L_2\}\\
              & = \{\alpha \in \{0,1\}^\star \mid 2\cdot 2^{|1\alpha|} + \ov{1\alpha}_{(2)} \equiv 2 \bmod 3\}\\
              & = \{\alpha \in \{0,1\}^\star \mid 4\cdot 2^{|\alpha|} + 2^{|\alpha|} + \ov{\alpha}_{(2)} \equiv 2 \bmod 3\}\\
              & = \{\alpha \in \{0,1\}^\star \mid 3\cdot 2^{|\alpha|} + 2\cdot 2^{|\alpha|} + \ov{\alpha}_{(2)} \equiv 2 \bmod 3\}\\
              & = \{\alpha \in \{0,1\}^\star \mid 2\cdot 2^{|\alpha|} + \ov{\alpha}_{(2)} \equiv 2 \bmod 3\}\\
              & = L_2.
\end{align*}

Така можем да получим автомат за езика $L$, където всяко състояние е свързано с език.

\begin{framed}
  \begin{figure}[H]
    \begin{center}
      \begin{tikzpicture}[->,>=stealth,thick,node distance=55pt]
        \tikzstyle{every state}=[circle,minimum size=15pt,auto]
        
        \node[initial,state]      (0) {$L$};
        \node[state]              (1) [right of=0]{$L_1$};
        \node[accepting, state]   (2) [right of=1]{$L_2$};
        
        \path 
        (0) edge  [loop above]    node [above]  {$0$} (0)
        (0) edge  [bend left=15]  node [above]  {$1$} (1)
        (2) edge  [bend left=15]  node [below]  {$0$} (1)
        (1) edge  [bend left=15]  node [below]  {$1$} (0)
        (1) edge  [bend left=15]  node [above]  {$0$} (2)
        (2) edge  [loop above]    node [above]  {$1$} (2);
      \end{tikzpicture}
    \end{center}
    \caption{$\L(\A) = \{\alpha\in\{0,1\}^\star \mid \ov{\alpha}_{(2)} \equiv 2\ (\bmod\ 3)\}$}
  \end{figure}
\end{framed}




\begin{example}
  Да разгледаме езика $L = \{a^nb^n\mid n \in \Nat\}$. Да се опитаме да построим автомат, който го разпознава.
  Нека
  \[L_k \df \{a^nb^{n+k}\mid n \in \Nat\}.\]
  Да видим какво се получава като приложим процедурата за строене 
  на минимален автомат.
  \begin{itemize}
  \item 
    $a^{-1}(L) = L_1$;
  \item
    $b^{-1}(L) = \emptyset$;
  \item
    $a^{-1}(L_1) = L_2$;
  \item
    $b^{-1}(L_1) = \{\varepsilon\}$;
  \item
    $a^{-1}(\{\varepsilon\}) = b^{-1}(\{\varepsilon\}) = \emptyset$;
  \item
    Вижда се, че $a^{-1}(L_k) = L_{k+1}$, за всяко $k$.
  \item
    Вижда се, че $b^{-1}(L_{k+1}) = \{b^k\}$, за всяко $k$.
    Освен това е ясно, че $b^{-1}(\{b^{k}\}) = \{b^{k-1}\}$, за всяко $k \geq 1$.
  \end{itemize}
  Получаваме, че езикът $L$ се разпознава от автомат с {\em безкрайно много състояния}.

\begin{framed}  
  \begin{figure}[H]
    \centering
    \begin{tikzpicture}[->,>=stealth,thick,node distance=70pt]
      \tikzstyle{every state}=[circle,minimum size=15pt,auto]
      
      \node[state, initial above]             (0) {$L$};
      \node[state]                            (1) [right of=0]{$L_1$};
      \node[state]                            (2) [right of=1]{$L_2$};
      \node[state]                            (3) [right of=2]{$L_3$};
      \node[state,accepting]                  (A) [below of=1]{$\{\varepsilon\}$};
      \node[state]                            (B) [below right of=1]{$\{b\}$};
      \node[state]                            (BB) [below right of=2]{$\{bb\}$};
      \node[state]                            (E) [below of=A]{$\emptyset$};
      
      \coordinate[right of=3] (4);
      \coordinate[below right of=3] (BBB);
      \coordinate[below of=4] (BBBA);

      \path 
      (0) edge [bend left=15]   node [above] {$a$} (1)
      (1) edge [bend left=15]   node [above] {$a$} (2)
      (2) edge [bend left=15]   node [above] {$a$} (3)
      (0) edge [bend right=30]  node [left] {$b$} (E)
      (E) edge [loop left]      node [left] {$a,b$} (E)
      (1) edge [bend right=30]  node [left] {$b$} (A)
      (2) edge [bend right=15]  node [left] {$b$} (B)
      (3) edge [bend right=15]  node [left] {$b$} (BB)
      (B) edge [bend right=15]  node [above] {$b$} (A)
      (B) edge [bend left=15]  node [right] {$a$} (E)
      (A) edge [bend right=15]   node [right] {$a,b$} (E)
      (BB) edge [bend right=15] node [above] {$b$} (B)
      (BB) edge [bend left=15]  node [below] {$a$} (E);
      
      \draw [dashed,->,shorten >=0pt] (3) to[bend left=15] node[auto] {$a$} (4);
      \draw [dashed,->,shorten >=0pt] (BBB) to[bend right=15] node[above] {$b$} (BB);
      \draw [dashed,->,shorten >=0pt] (BBBA) to[bend left=30] node[below] {$a$} (E);
    \end{tikzpicture}
    \caption{{\em Безкраен} автомат за $\{a^nb^n \mid n \in \Nat\}$}
  \end{figure}
\end{framed}
\end{example}

%%% Local Variables:
%%% mode: latex
%%% TeX-master: "../eai"
%%% End:
