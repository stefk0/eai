\section{Регулярни изрази и езици}

Да фиксираме една непразна азбука $\Sigma$.
\index{регулярен израз}
{\bf Регулярните изрази} $\mathbf{r}$ могат да се опишат със следната абстрактна граматика
\[\mathbf{r} ::= \bm{\emptyset}\ |\ \bm{\varepsilon}\ |\ \mathbf{a}\ |\ (\bm{r}_1 \cdot \bm{r}_2)\ |\ (\bm{r}_1 + \bm{r_2})\ |\ \bm{r}^\star_1.\]
Регулярните изрази могат да се опишат и по следния начин:
\mynote{Това е пример за индуктивна дефиниция}
\begin{itemize}
\item 
  Символите $\bm{\emptyset}$, $\bm{\varepsilon}$ са регулярни изрази;
\item
  за всяка буква $a \in \Sigma$, символът $\bm{a}$ е регулярен израз;
\item
  \mynote{В литературата също се среща записът $(\bm{r}_1\ |\ \bm{r}_2)$ вместо $(\bm{r}_1 + \bm{r}_2)$}
  ако $\mathbf{r_1}$ и $\mathbf{r_2}$ са регулярни изрази, то думите $(\bm{r}_1 \cdot \bm{r}_2)$, $(\bm{r}_1 + \bm{r}_2)$ и $\bm{r}^\star_1$
  също са регулярни изрази;
\item
  Всеки регулярен израз е получен по някое от горните правила.
\end{itemize}

\index{Език!регулярен}
\mynote{Това е друг пример за индуктивна (рекурсивна) дефиниция.}
Сега ще дефинираме езиците, които се описват с регулярни изрази.
Тези езици се наричат {\bf регулярни}.
Това ще направим следвайки индуктивната дефиниция на регулярните изрази,
т.е. за всеки регулярен израз $\mathbf{r}$ ще определим език $\L(\mathbf{r})$.
\begin{itemize}
\item
  $\emptyset$ е регулярен език,
  който се описва от регулярния израз $\bm{\emptyset}$. Означаваме
  \[\L(\bm{\emptyset}) \df \emptyset;\]
\item
  $\{\varepsilon\}$ е регулярен език,
  който се описва от регулярния израз $\bm{\varepsilon}$. Означаваме
  \[\L(\bm{\varepsilon}) \df \{\varepsilon\};\]
\item
  за всяка буква $a \in \Sigma$, $\{a\}$ е регулярен език,
  който се описва от регулярния израз $\mathbf{a}$. Означаваме
  \[\L(\mathbf{a}) \df \{a\};\]
\item
  \mynote{Понякога, когато приоритетът на операциите е ясен, ще изпускаме да пишем скоби.}
  Нека $L_1$ и $L_2$ са регулярни езици, т.е. съществуват регулярни изрази $\mathbf{r}_1$
  и $\mathbf{r}_2$, за които $\L(\mathbf{r}_1) = L_1$ и $\L(\mathbf{r}_2) = L_2$.
  Тогава:
  \begin{itemize}
  \item 
    \index{регулярни операции!обединение}
    $L_1 \cup L_2$ е регулярен език, който се описва с регулярния израз $(\mathbf{r}_1 + \mathbf{r}_2)$. Тогава
    \[\L(\mathbf{r}_1 + \mathbf{r}_2) \df \L(\mathbf{r}_1) \cup \L(\mathbf{r}_2).\]
  \item
    \index{регулярни операции!конкатенация}
    \mynote{Тази операция се нарича конкатенация. Обикновено изпускаме знака $\cdot$}
    $L_1 \cdot L_2$ е регулярен език, който се описва с регулярния израз $(\mathbf{r}_1 \cdot \mathbf{r}_2)$. Тогава
    \[\L(\mathbf{r}_1 \cdot \mathbf{r}_2) \df \L(\mathbf{r}_1) \cdot \L(\mathbf{r}_2).\]
  \item
    \mynote{Звезда на Клини}
    \index{регулярни операции!звезда на Клини}
    $L^\star_1$ е регулярен език, който се описва с регулярния израз $\mathbf{r}^\star_1$. Тогава
    \[\L(\mathbf{r}^\star_1) \df \L(\mathbf{r}_1)^\star.\]
  \end{itemize}
\end{itemize}

\begin{remark}
  Ние знаем, че:
  \begin{itemize}
  \item
    Всеки регулярен израз представлява крайна дума над крайна азбука.
    Това означава, че множеството от всички регулярни изрази е изброимо безкрайно.
    Оттук следва, че всички регулярни езици образуват изброимо безкрайно множество.
  \item 
    Понеже $\Sigma$ е крайна азбука, то $\Sigma^\star$ е изброимо безкрайно множество;
  \item
    Един език над азбуката $\Sigma$ представлява елемент на $\Ps(\Sigma^\star)$.
    Това означава, че всички езици над азбуката $\Sigma$ представляват неизброимо безкрайно множество.
  \end{itemize}
  От всичко това следва, че има езици, които не са регулярни.
  По-нататък ще видим примери за такива езици.
\end{remark}

\begin{extra}
\begin{example}
  \mynote{В \cite[стр. 70]{sipser3} е показан алгоритъм, за който по един автомат може да се получи регулярен израз описващ езика на автомата. Ние няма да разглеждаме този алгоритъм. }
  Нека да построим регулярни изрази за всеки от езиците от \Example{automata-pictures}.
  \begin{enumerate}[a)]
  \item 
    Нека $\mathbf{r} \df \mathbf{(a+b)^\star bab(a+b)^\star}$. Тогава
    \[\L(\mathbf{r}) = \{\omega \in \{a,b\}^\star \mid \omega \text{ съдържа } bab\}.\]
  \item
    Нека $\mathbf{r} \df \mathbf{b^\star ab^\star a(a+b)^\star}$. Тогава
    \[\L(\mathbf{r}) = \{\omega \in \{a,b\}^\star \mid \card{\omega}{a} \geq 2\}.\]
  \item
    \mynote{За момента не е ясно как можем да верифицираме дали регулярният израз $\mathbf{r}$ наистина описва посочения език.}
    Нека $\mathbf{r} \df \mathbf{(b + ab)^\star}$. Тогава
    \[\L(\mathbf{r}) = \{\omega \in \{a,b\}^\star \mid \text{ всяко $a$ в $\omega$ се следва от поне едно $b$}\}.\]
  \item
    Нека $\mathbf{r} \df \mathbf{(b^\star ab^\star ab^\star ab^\star)^\star}$. Тогава
    \[\L(\mathbf{r}) = \{\omega \in \{a,b\}^\star \mid \card{\omega}{a} \equiv 0 \bmod 3\}.\]
  \end{enumerate}
\end{example}
  
\begin{problem}
  За произволни регулярни изрази $\bm{r}$ и $\bm{s}$, ще пишем $\bm{r} \equiv \bm{s}$, ако $\L(\bf{r}) = \L(\bm{s})$.
  Проверете дали са изпълнени следните равенства:
  \begin{enumerate}[a)]
  \item 
    $\bm{r + s} \equiv \bm{s + r}$;
  \item
    $\bm{(\varepsilon + r)^\star} \equiv \bm{r^\star}$;
  \item
    $\bm{\emptyset^\star} \equiv \bm{\varepsilon}$;
  \item
    $\bm{(r^\star s^\star)^\star} \equiv \bm{(r+s)^\star}$;
  \item
    $\bm{(r^\star)^\star} \equiv \bm{r^\star}$;
  \item
    $\bm{(rs + r)^\star r} \equiv \bm{r(sr+r)^\star}$;
  \item
    $\bm{s(rs+s)^\star r} \equiv \bm{rr^\star s(rr^\star s)^\star}$;
  \item
    $\bm{(r+s)^\star} \equiv \bm{r^\star + s^\star}$;
  \item
    $\bm{(r+s)^\star s} \equiv \bm{(r^\star s)^\star}$;
  \item
    $\bm{(rs + r)^\star rs} \equiv \bm{(rr^\star s)^\star}$;
  \item
    $\bm{\emptyset^\star} \equiv \bm{\varepsilon^\star}$.
  \end{enumerate}
\end{problem}
\end{extra}


%%% Local Variables:
%%% mode: latex
%%% TeX-master: "../eai"
%%% End:
