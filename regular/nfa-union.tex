\subsection{Затвореност относно обединение}

\begin{lemma}
  \label{lem:union}
  \mynote{Второ доказателство на това твърдение.}
  Класът от автоматните езици е затворен относно операцията {\em обединение}.
\end{lemma}
\begin{hint}
  Нека са дадени детерминистичните автомати:
  \begin{itemize}
  \item 
    $\A_1 = \pair{\Sigma,Q_1,\delta_1,\qstart',F_1}$, като $L(\A_1) = L_1$;
  \item
    $\A_2=\pair{\Sigma,Q_2,\delta_2,\qstart'',F_2}$, като $L(\A_2) = L_2$.
  \end{itemize}
  Ще дефинираме автомата $\N=\NFA$, така че
  \[L(\N) = L(\A_1) \cup L(\A_2).\]
  \begin{itemize}
  \item 
    $Q \df Q_1 \cup Q_2 \cup \{\qstart\}$, където $\qstart\not\in Q_1\cup Q_2$;
  \item
    $F \df 
    \begin{cases}
      F_1 \cup F_2 \cup \{\qstart\}, & \text{ ако } \qstart' \in F_1 \vee \qstart'' \in F_2\\
      F_1 \cup F_2,            & \text{ иначе } 
    \end{cases}$
  \item
    $\Delta(q,a) \df
    \begin{cases}
      \{\delta_1(q,a)\},                       & \text{ ако } q\in Q_1\ \&\ a\in\Sigma\\
      \{\delta_2(q,a)\},                       & \text{ ако } q\in Q_2\ \&\  a\in\Sigma\\
      \{\delta_1(\qstart',a), \delta_2(\qstart'',a)\}, & \text{ ако } q = \qstart\ \&\ a \in\Sigma.
    \end{cases}$
  \end{itemize}
\end{hint}

\begin{example}
    За да построим автомат, който разпознава обединението на $\L(\A_1)$ и $\L(\A_2)$,
    трябва да добавим ново начално състояние, което да свържем с наследниците на началните състояния на $\A_1$ и $\A_2$.
    
    \begin{figure}[H]
      \center
      % \begin{subfigure}[b]{0.3\textwidth}
      \begin{tikzpicture}[framed,->,>=stealth,thick,node distance=2cm]
        \tikzstyle{every state}=[circle,minimum size=20pt,auto]
        \node[initial,state,accepting]      (0) {$q_0$};
        \node[state,accepting] [above right of=0]        (1) {$q'_0$};
        \node[state]    [right of=1]        (2) {$q_1$};
        \node[state]                        (3) [above right of=2] {$q_2$};
        \node[state,accepting]                        (4) [below right of=2] {$q_3$};
        \node[state]    [below right=2cm of 0] (5) {$q''_0$};
        \node[state]     [above right of=5] (6) {$q_4$};
        \node[state,accepting]     [below right of=5] (7) {$q_5$};
        \path
        (1) edge [loop above] node [above] {$b$} (1)
        (1) edge node [above]                  {$a$} (2)
        (2) edge node [above]                  {$a$} (3)
        (2) edge node [below]                  {$b$} (4)
        (3) edge [bend right=15] node [above]  {$a,b$} (1)
        (4) edge [bend left=15]  node [below]  {$b$} (1)
        (4) edge [bend right=15]  node [right]  {$a$} (3)
        (5) edge [bend left=15] node [below]   {$a$} (6)
        (6) edge [bend left=15] node  [right] {$a,b$} (7)
        (5) edge [bend right=15] node [above]  {$b$} (7)
        (7) edge [loop right] node [right] {$a,b$} (7)
        (0) edge [dashed, bend right=15] node [below]  {$a$} (2)
        (0) edge [dashed, bend left=15] node [above]  {$b$} (1)
        (0) edge [dashed, bend right=15] node [below]  {$a$} (6)
        (0) edge [dashed, bend right=45] node [below]  {$b$} (7);
      \end{tikzpicture}
      \caption{$\L(\N) = \L(\A_1)\cup\L(\A_2)$}
  \end{figure}  
  Обърнете внимание, че $\A_1$ и $\A_2$ са детерминирани автомати, но $\N$ е недетерминиран.
  Освен това, новото състояние $q_0$ трябва да бъде маркирано като финално, защото $q'_0$ е финално.
\end{example}

%%% Local Variables:
%%% mode: latex
%%% TeX-master: "../eai"
%%% End:
