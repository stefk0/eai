\begin{framed}
\begin{thm}[Клини]
  \label{th:regular-kleene}
  \index{Клини}
  Всеки автоматен език се описва с регулярен израз.
\end{thm}
\end{framed}
\begin{proof}
  \marginpar{\cite[стр. 79]{papadimitriou}; \cite[стр. 33]{hopcroft1}}
  Нека  $L = \L(\A)$, за някой краен детерминиран автомат $\A$.
  Да фиксираме едно изброяване на състоянията $Q = \{q_1,\dots,q_n\}$,
  като началното състояние е $q_1$.
  Ще означаваме с $L(i,j,k)$ множеството от тези думи, които
  могат да се разпознаят от автомата по път, който започва от $q_i$,
  завършва в $q_j$, и междинните състояния имат индекси $\leq k$.
  Например, за думата $\alpha = a_1a_2\cdots a_n$ имаме, че $\alpha \in L(i,j,k)$
  точно тогава, когато съществуват състояния $q_{l_1},\dots,q_{l_{n-1}}$, като $l_1,\dots,l_{n-1} \leq k$ и
  \[q_i\stackrel{a_1}{\rightarrow} q_{l_1} \stackrel{a_2}{\rightarrow} q_{l_2} \stackrel{a_3}{\rightarrow} \dots \stackrel{a_{n-1}}{\rightarrow} q_{l_{n-1}}\stackrel{a_n}{\rightarrow} q_j.\]
  Тогава за $n = \abs{Q}$, 
  \[L(i,j,n) = \{\alpha\in\Sigma^\star\mid \delta^\star(q_i,\alpha) = q_j\}.\]
  Така получаваме, че 
  \[\L(\A) = \bigcup\{L(1,j,n)\mid q_j \in F\} = \bigcup_{q_j\in F}L(1,j,n).\]
  Ще докажем с {\em индукция по $k$}, че за всяко $i,j,k$, множествата от думи $L(i,j,k)$
  се описват с регулярен израз $\mathbf{r^k_{i,j}}$
  \begin{enumerate}[a)]
  \item
    Нека $k = 0$. Ще докажем, че за всяко $i,j$, $L(i,j,0)$ се описва с регулярен израз.
    Имаме да разгледаме два случая.
    
    Ако $i = j$, то 
    \begin{equation}
      \label{eq:kleene-equal}
      L(i, j, 0) = \{\varepsilon\}\cup\{a\in\Sigma \mid \delta(q_i,a) = q_j\}.
    \end{equation}
    Ако $i \neq j$, то
    \[L(i, j, 0) = \{a\in\Sigma \mid \delta(q_i, a) = q_j\}.\]
    И в двата случая, понеже $L(i,j,0)$ е краен език, то е ясно, че той се описва с регулярен израз.
  \item
    Да предположим, че $k > 0$, и за всяко $i,j \leq n$, имаме регулярните изрази $\mathbf{r}^{k-1}_{i,j}$, които
    описват езиците $L(i,j,k-1)$, т.е. имаме индукционното предположение, че
    \[(\forall i,j \leq n)[L(i,j,k-1) = \L(\mathbf{r}^{k-1}_{i,j})].\] 
    Ще докажем, че съществуват регулярни изрази $\mathbf{r}^k_{i,j}$, такива че
    \[(\forall i,j \leq n)[L(i,j,k) = \L(\mathbf{r}^{k}_{i,j})].\] 
    Можем да изразим езика $L(i,j,k)$ по следния начин:
    \[L(i,j,k) = \underbrace{L(i,j,k-1)}_{\L(\mathbf{r}^{k-1}_{i,j})}\ \cup\ \underbrace{L(i,k,k-1)}_{\L(\mathbf{r}^{k-1}_{i,k})}\cdot (\underbrace{L(k,k,k-1)}_{\L(\mathbf{r}^{k-1}_{k,k})})^\star \cdot \underbrace{L(k,j,k-1)}_{\L(\mathbf{r}^{k-1}_{k,j})}.\]

    Тогава по {\bf И.П.} следва, че $L(i,j,k)$ може да се опише с регулярния израз
    \begin{equation}
      \label{eq:kleene}
      \mathbf{r}^k_{i,j} = \mathbf{r}^{k-1}_{i,j} + \mathbf{r}^{k-1}_{i,k}\cdot (\mathbf{r}^{k-1}_{k,k})^\star\cdot \mathbf{r}^{k-1}_{k,j}.
    \end{equation}
  \end{enumerate}
  Заключаваме, че за всяко $i,j,k$, $L(i,j,k)$ може да се опише с регулярен израз $\mathbf{r}^{k}_{i,j}$.
  Тогава ако $F = \{q_{i_1},\dots,q_{i_k}\}$, то $\L(\A)$ се описва с регулярния израз
  \[\mathbf{r}^n_{1,i_1} + \mathbf{r}^n_{1,i_2} + \dots + \mathbf{r}^n_{1,i_k}.\]
\end{proof}

\begin{cor}
  Съществува алгоритъм, за който при вход краен детерминиран автомат $\A$,
  извежда като изход регулярен израз $\mathbf{r}$, такъв че $\L(\A) = \L(\mathbf{r})$.
\end{cor}

Доказателството на това следствие използва идеята от доказателството на \Th{regular-kleene},
но излиза извън нашия интерес. За подробно изложение на този въпрос, вижте \cite[стр. 69]{sipser3}.
Възможно е по-късно да използваме този резултат наготово.
Нека поне да разгледаме един пример, чиято цел е да ни убеди, 
че наистина може да се извлече алгоритъм от доказателството на \Th{regular-kleene}.

\begin{example}
  Да разгледаме следния автомат:

  \begin{framed}
    \begin{figure}[H]
      \begin{center}
        \begin{tikzpicture}[->,>=stealth,thick,node distance=45pt]
          \tikzstyle{every state}=[circle,minimum size=15pt,auto]
          
          \node[initial,state]      (1) {$q_1$};
          \node[accepting, state]   (2) [right of=1]{$q_2$};
          
          \path 
          (1) edge [loop above]  node [above] {$1$} (1)
          (1) edge  node [above] {$0$} (2)
          (2) edge [loop above] node [above] {$0,1$} (2);
        \end{tikzpicture}
      \end{center}
      \caption{Автомат разпознаващ $\L(\mathbf{1^\star 0 (0 + 1)^\star)}$}
      \label{fig:a1}
    \end{figure}
\end{framed}

Лесно се съобразява, че езикът на автомата от Фигура \ref{fig:a1} се описва с регулярния израз $\mathbf{1^\star 0 (0 + 1)^\star}$.
Следвайки конструкцията от доказателството на \Th{regular-kleene},
езикът на този автомат се описва с регулярния израз $\mathbf{r^2_{1,2}}$, защото началното състояние е $q_1$, финалното е $q_2$ и 
броят на състоянията в автомата е $2$.
\begin{align*}
  \mathbf{r}^2_{1,2} & = \underbrace{\mathbf{r}^{1}_{1,2}}_{\mathbf{1^\star 0}} + \underbrace{\mathbf{r}^{1}_{1,2}}_{\mathbf{1^\star 0}}\cdot \underbrace{\mathbf{(r^1_{2,2})^\star}}_{\mathbf{(\varepsilon+0+1)^\star}} \cdot \underbrace{\mathbf{r}^1_{2,2}}_{\mathbf{\varepsilon+0+1}} & (\text{според (\ref{eq:kleene}})) \\
                     &  = \mathbf{1^\star0 + 1^\star 0 (\varepsilon + 0 + 1)^\star (\varepsilon + 0 + 1)}\\
                     & =  \mathbf{1^\star0 + 1^\star 0 (\varepsilon + 0 + 1)^+} & (\mathbf{r^+ = r^\star r})\\
                     & =  \mathbf{1^\star0 + 1^\star 0 (0 + 1)^\star} & (\mathbf{r^\star = (\varepsilon + r)^+})\\
                     & = \mathbf{1^\star 0 (\varepsilon + (0 + 1)^\star)} & (\mathbf{r + rq = r(\varepsilon + q)})\\
                     & = \mathbf{1^\star 0 (0 + 1)^\star} & (\mathbf{r^\star = \varepsilon + r^\star})
\end{align*}

Тук използвахме, че:
\begin{align*}
  \mathbf{r^1_{1,2}} & = \underbrace{\mathbf{r^0_{1,2}}}_{\mathbf{0}} + \underbrace{\mathbf{r^0_{1,1}}}_{\mathbf{\varepsilon + 1}}\cdot\underbrace{\mathbf{(r^0_{1,1})^\star}}_{\mathbf{(\varepsilon+1)^\star}} \cdot \underbrace{\mathbf{r^0_{1,2}}}_{\mathbf{0}}\\
                     & = \mathbf{0 + (\varepsilon + 1)(\varepsilon + 1)^\star0} \\
                     & = \mathbf{0 + 1^\star 0}\\
                     & = \mathbf{1^\star0},\\
  \mathbf{r^1_{2,2}} & = \underbrace{\mathbf{r^0_{2,2}}}_{\mathbf{\varepsilon+0+1}} + \underbrace{\mathbf{r^0_{2,1}}}_{\mathbf{\emptyset}} \cdot \underbrace{\mathbf{(r^0_{1,1})^\star}}_{\mathbf{\varepsilon+1}}\cdot \underbrace{\mathbf{r^0_{1,2}}}_{\mathbf{0}}\\
                     & = \mathbf{\varepsilon + 0 + 1 + \emptyset(\varepsilon + 1)^\star0}\\
                     & = \varepsilon + 0 + 1 & (\text{защото }\mathbf{\emptyset \cdot r = \emptyset})
\end{align*}
\end{example}

Следващата ни цел е да видим, че имаме и обратната посока на горната лема.
Ще докажем, че всеки регулярен език е автоматен. За тази цел първо ще 
въведем едно обобщение на понятието краен детерминиран автомат.


%%% Local Variables:
%%% mode: latex
%%% TeX-master: "../eai"
%%% End:
