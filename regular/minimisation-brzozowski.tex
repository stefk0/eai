\subsection{Минимизация на Бжозовски}

\index{минимален автомат}
\index{минимизация}
\index{Бжозовски}

\mynote{\cite[стр. 88]{shallit}}

Нека $\A = (\Sigma,Q,\delta,\qstart,F)$ е детерминиран краен автомат. Тогава полагаме
\[\A_q \df (\Sigma,Q,\delta,q,F).\]
Нека $\A = (\Sigma,Q,\Delta,Q_{\texttt{start}},F)$ е недетерминиран краен автомат. Тогава полагаме
\[\A_q \df (\Sigma,Q,\Delta,\{q\},F).\]
За произволен краен автомат $\A$, за да не пишем много индекси, полагаме
\[\L_\A(q) \df \L(\A_q).\]

Нека да започнем с едно твърдение, което ни дава критерий, макар и доста силен, кога
детерминизацията ни дава автомат с минимален брой състояния.

\begin{proposition}
  \label{pr:regular:brzozowski-minimal:det}
  Нека $\A$ е недетерминиран краен автомат със следните свойства:
  \begin{itemize}
  \item
    За всяко състояние $q \in Q$ е изпълнено, че:
    \begin{equation}
      \label{eq:regular:minimisation-brzozowski:1}      
      \L_\A(q) \neq \emptyset.
    \end{equation}
  \item
    За всеки две различни състояния $p$ и $q$ е изпълнено, че:
    \begin{equation}
      \label{eq:regular:minimisation-brzozowski:2}
      \L_\A(q) \cap \L_\A(p) = \emptyset.
    \end{equation}
  \end{itemize}
  Тогава $\M = \texttt{det}(\A)$, получен по \hyperref[th:regular:nfa:construction]{метода на Рабин-Скот}, е минимален автомат за езика на $\A$.
\end{proposition}
\mynote{$\M$ е получен чрез конструкцията от \Theorem{regular:nfa:construction}.
  \newline
  Критерият за минималност на детерминиран $\A$ е за всеки две различни състояния $p$ и $q$,
  то $\L_\A(p) \neq \L_\A(q)$. Понеже тук $\A$ е недетерминирам, искаме по-силното свойство, а именно
  $\L_\A(p) \cap \L_\A(q) = \emptyset$.}
\begin{hint}
  От контрукцията на $\M$ в \hyperref[th:regular:nfa:construction]{теоремата на Рабин-Скот}, знаем, че $\M$ е свързан автомат.
  Според \hyperref[cr:regular:brzozowski-minimal:criterion]{критерият за минималност}, за да докажем, че $\M$ е минимален е достатъчно да
  покажем, че за произволни състояния $P$ и $U$ на $\M$ е изпълнена импликацията:
  \[\L_\M(P) = \L_\M(U) \implies P = U.\]

  И така, да вземем две такива състояния $P$ и $U$, за които $\L_\M(P) = \L_\M(U)$.
  Достатъчно е да докажем, че $P \subseteq U$, защото доказателството на другата посока е симетрично.

  Да разгледаме произволен $p \in P$. Щом от (\ref{eq:regular:minimisation-brzozowski:1}) имаме, че $\L_\A(p) \neq \emptyset$, то съществува дума $\omega$,
  за която $\Delta^\star_\A(\{p\},\omega) \cap F^\A \neq \emptyset$
  или с други думи,
  $\delta^\star_\M(P,\omega) \in F^\M$, откъдето имаме, че $\omega \in \L_\M(P)$.
  Сега, понеже приехме, че $\L_\M(P) = \L_\M(U)$, то $\delta^\star_\M(U,\omega) \in F^\M$,
  което означава, че за някое състояние $u \in U$, $\Delta^\star_\A(\{u\},\omega) \cap F^\A \neq \emptyset$.
  Така получаваме, че $\omega \in \L_\A(u)$. От (\ref{eq:regular:minimisation-brzozowski:2}) следва, че
  $u = p$, защото $\L_\A(p) \cap \L_\A(q) \neq \emptyset$. Заключаваме, че $P \subseteq U$.
\end{hint}


\begin{proposition}
  \label{pr:regular:brzozowski-minimal:rev}
  Нека $\D$ е ДКА, за който всяко състояние е достижимо от началното.
  Да разгледаме $\A \df \texttt{rev}(\D)$.
  Тогава са изпълнени двете свойства от условието на \Proposition{regular:brzozowski-minimal:det},
  а именно:
  \begin{itemize}
  \item
    За всяко състояние $q \in Q$ е изпълнено, че:
    \[\L_\A(q) \neq \emptyset.\]
  \item
    За всеки две различни състояния $p$ и $q$ е изпълнено, че:
    \[\L_\A(q) \cap \L_\A(p) = \emptyset.\]
  \end{itemize}  
\end{proposition}
\begin{hint}
  Да напомним, че според \Problem{regular:nfa:rev}, $\A = (\Sigma,Q,\Delta,F,\{\qstart\})$, където
  \[\Delta_\A(q,a) \df \{p \in Q \mid \delta_\D(p,a) = q\}.\]
  Щом в $\D$ всяко състояние $q$ е достижимо от началното, то е ясно,
  че $\L_\A(q) \neq \emptyset$, което ни дава първото свойство.
  За второто свойство, нека сега приемем, че $\omega \in \L_\A(p) \cap \L_\A(q)$.
  Ще докажем, че $p = q$. И така, щом $\omega \in \L_\A(p)$, то $\Delta^\star_\A(\{p\},\omega) \ni \qstart$.
  Това означава, че $\delta^\star_\D(\qstart,\omega) = p$.
  От друга страна, имаме също, че $\delta^\star_\D(\qstart,\omega) = q$.
  Ясно е, че $p = q$, защото $\D$ е детерминиран автомат.
\end{hint}

\begin{proposition}
  Нека $\D$ е ДКА, за който всяко състояние е достижимо от началното.
  Тогава детерминираният краен автомат
  \[\M \df \texttt{det}(\texttt{rev}(\D))\]
  е минимален за езика $L^\rev(\D)$.
\end{proposition}
\begin{hint}
  Просто комбинираме горните две твърдения.
  Нека $\A \df \texttt{rev}(\D)$. Знаем, че $\L(\A) = \L^\rev(\D)$.
  От \Proposition{regular:brzozowski-minimal:rev} знаем, че $\A$ притежава свойствата необходими
  за приложението на \Proposition{regular:brzozowski-minimal:det}. Така получаваме, че
  $\M \df \texttt{det}(\A)$ е минимален детерминиран автомат за езика $\L(\A)$.
\end{hint}

По този начин получаваме алгоритъм за минимизация на автомат.

\begin{important}
  \begin{theorem}[Бжозовски]
    Нека $\D$ е ДКА, за който всяко състояние е достижимо от началното.
    Тогава детерминираният краен автомат
    \[\M \df \texttt{det}(\texttt{rev}(\texttt{det}(\texttt{rev}(\D))))\]
    е минимален за езика на $\D$.
  \end{theorem}
\end{important}
