\section{Теорема за съществуване на минимален автомат}
\begin{framed}
  \begin{thm}[Майхил-Нероуд]
    \label{th:myhill-nerode}
    \index{Майхил-Нероуд!теорема}
    Нека $L\subseteq \Sigma^\star$ е регулярен език.
    Тогава съществува ДКА $\M = \FA$, който разпознава $L$,
    с точно толкова състояния, колкото са класовете на еквивалентност на релацията $\approx_L$,
    т.е. $\abs{Q} = \abs{\Sigma^\star/_{\approx_L}}$.
  \end{thm}  
\end{framed}
\begin{proof}
  % \marginpar{стр. 96 от \cite{papadimitriou}}
  \marginpar{на англ. Myhill-Nerode}
  Да фиксираме регулярния език $L$.
  Ще дефинираме детерминистичен краен автомат $\M = \FA$, разпознаващ $L$, като:
  \begin{itemize}
  \item
    $Q \df \{\ [\alpha]_L\mid \alpha\in \Sigma^\star\ \}$;
  \item
    $\qstart \df [\varepsilon]_L$;
  \item
    $F \df \{\ [\alpha]_L\mid \alpha\in L\ \} = \{\ [\alpha]_L \mid [\alpha]_L \subseteq L\ \}$;
  \item
    Определяме изображението $\delta$ като 
    за всяка буква $x \in \Sigma$ и всяко състояние $[\alpha]_L\in Q$, 
    \[\delta([\alpha]_L,x) \df [\alpha x]_L.\]
  \end{itemize}
  
  Първо, трябва да се уверим, че множеството от състояния $Q$ е крайно, т.е.
  релацията $\approx_L$ има крайно много класове на еквивалентност.
  И така, тъй като $L$ е регулярен език, то той се разпознава от някой детерминистичен краен автомат $\A$.
  От \Cor{upper-bound} имаме, че $\abs{Q^{\A}} \geq \abs{\Sigma^\star/_{\approx_L}}$.
  Понеже $Q^{\A}$ е крайно множество, то $\approx_L$ има крайно много класове и 
  следователно $Q$ също е крайно множество.

  Второ, трябва да се уверим, че изображението $\delta$ задава функция, т.е. 
  да проверим, че за всеки две думи $\alpha$, $\beta$ и всяка буква $x$,
  \[[\alpha]_L = [\beta]_L \implies \delta([\alpha]_L,x) = \delta([\beta]_L,x).\]
  Но това се вижда веднага, защото от определението на релацията $\approx_L$ следва, че
  ако $\alpha \approx_L \beta$, то за всяка буква $x$, $\alpha x \approx_L \beta x$,
  т.е. $[\alpha x]_L = [\beta x]_L$ и 
  \marginpar{Функцията на преходите $\delta$ е дефинирана чрез произволен представител на всеки клас на еквивалентност относно $\approx_L$. Трябва да съобразим, че няма значение кой представител сме избрали.}
  \begin{align*}
    [\alpha]_L = [\beta]_L & \implies [\alpha x]_L = [\beta x]_L & \comment{\text{свойство на }\approx_L}\\
                           & \implies \delta([\alpha]_L,x) \df [\alpha x]_L = [\beta x]_L \df \delta([\beta]_L,x).
  \end{align*}
  
  Така вече сме показали, че $\M$ е коректно зададен детерминистичен краен автомат.
  Остава да покажем, че $\M$ разпознава езика $L$, т.е. $\L(\M) = L$.
  За целта, първо ще докажем едно помощно твърдение.
  \begin{prop}
    За всеки две думи $\alpha$ и $\beta$ е изпълнено, че:
    \begin{equation}
      \label{eq:4}
      \delta^\star([\alpha]_L,\beta) = [\alpha\beta]_L.
    \end{equation}
  \end{prop}
  \begin{proof}
    Ще докажем това свойство с индукция по дължината на думата $\beta$.
    \begin{itemize}
    \item
      За $\beta = \varepsilon$ свойството следва директно от дефиницията на $\delta^\star$ като рефлексивно и транзитивно затваряне на $\delta$,
      защото $\delta^\star([\alpha]_L,\varepsilon) = [\alpha]_L$.
    \item
      Нека $\abs{\beta} = n+1$ и да приемем, че сме доказали твърдението за думи с дължина $n$.
      Тогава $\beta = \gamma a$, където $\abs{\gamma} = n$. Свойството следва от следните равенства:
      \begin{align*}
        \delta^\star([\alpha]_L, \gamma a) & = \delta(\delta^\star([\alpha]_L,\gamma),a) & \comment{\text{деф. на }\delta^\star}\\
                                          & = \delta([\alpha\gamma]_L,a) & \comment{\text{от {\bf И.П.} за }\gamma}\\
                                          & = [\alpha\gamma a]_L & \comment{\text{от деф. на }\delta}\\
                                          & = [\alpha\beta]_L & \comment{\beta = \gamma a}.
      \end{align*}
    \end{itemize}
  \end{proof}
  \noindent За да се убедим, че $L = \L(\M)$ е достатъчно да проследим еквивалентностите:
  \begin{align*}
    \alpha\in \L(\M) & \iff \delta^\star(\qstart,\alpha) \in F & \comment{\text{деф. на }\L(\M)}\\
                     & \iff \delta^\star([\varepsilon]_L,\alpha) \in F & \comment{\qstart \df [\varepsilon]_L}\\
                     & \iff \delta^\star([\varepsilon]_L,\alpha) = [\varepsilon\alpha]_L\in F & \comment{\text{от (\ref{eq:4})}}\\
                     & \iff \alpha\in L & \comment{\text{деф. на }F}.
  \end{align*}
\end{proof}


%%% Local Variables:
%%% mode: latex
%%% TeX-master: "../eai"
%%% End:
