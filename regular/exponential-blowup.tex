\subsection{Експоненциална експлозия}\label{sect:regular:nfa:exponential-blowup}
\mynote{Тук следваме \cite[стр. 66]{khoussainov-nerode} и \cite[стр. 164]{compilers2}. В \cite[стр. 80]{shallit} има друг пример за НКА с $n$ състояния вместо $n+1$, но доказателството изглежда по-сложно.}
\begin{proposition}
  Съществува НКА $\N$ с $n+1$ състояния, за който не съществува ДКА $\A$ с по-малко от $2^n$ състояния.
\end{proposition}
\begin{hint}
  Да разгледаме следния недетерминиран краен автомат $\N$.
  \begin{figure}[H]
    \centering
    \begin{tikzpicture}[framed,->,>=stealth,thick,node distance=45pt,initial text=начало]
      \tikzstyle{every state}=[circle,minimum size=20pt,auto]
      \node[initial below,state]              (0) {$q_0$};
      \node[state]                [right of=0] (1) {$q_1$};
      \node[state]                [right of=1] (2) {$q_2$};
      \node[state]                [right of=2] (3) {$q_3$};
      \coordinate[right of=3] (4);
      \coordinate[right of=4] (5);
      \node[state,accepting]      [right of=5] (6) {$q_n$};
      \path
      (0) edge [loop above] node [above] {$a,b$} (0)
      (0) edge [bend left=15] node  [above] {$a$} (1)
      (1) edge [bend left=15] node  [above] {$a,b$} (2)
      (2) edge [bend left=15] node  [above] {$a,b$} (3);

      \draw [dashed,->,shorten >=0pt] (3) to[bend left=15] node[above] {$a,b$} (4);
      \draw [dashed,->,shorten >=0pt] (5) to[bend left=15] node[above] {$a,b$} (6);
    \end{tikzpicture}
    \caption{Недетерминиран автомат $\N$ за езика $\{a,b\}^\star \cdot \{a\}\cdot \{a,b\}^{n-1}$.}
    \label{fig:nfa:exp}
  \end{figure}
  Лесно се съобразява, че за $n > 0$, недетерминирният краен автомат $\N$ на \Figure{nfa:exp} с $n+1$ на брой състояния разпознава езика
  \[L = \{\omega \cdot a \cdot \delta \in \{a,b\}^\star \mid |\delta| = n-1\}.\]
  Нека да съобразим, че не е възможно да съществува краен детерминиран автомат $\A = \FA$ разпознаващ същия език $L$ с по-малко от $2^n$ състояния.

  Да допуснем, че $|Q| < 2^n$. От принципа на Дирихле имаме, че съществуват две различни думи $\alpha$ и $\beta$ с дължина $n$,
  за които съществува $q \in Q$ и
  \mynote{Да напомним, че броят на всички думи с дължина $n$ над азбука с $k$ букви е $k^n$. В нашия случай, $k = 2$.}
  \[\delta^\star(\qstart,\alpha) = q = \delta^\star(\qstart,\beta).\]

  Нека първата разлика в тези две думи е на позиция $i < n$, т.е. думите $\alpha$ и $\beta$ могат да се представят така:
  \[\alpha = \omega \cdot \alpha \slice{i} \cdot \rho\ \text{ и } \beta = \omega \cdot \beta\slice{i} \cdot \delta.\]
  Без ограничение на общността, нека $\alpha\slice{i} = a$ и $\beta\slice{i} = b$.
  \begin{itemize}
  \item
    \mynote{Тук $\omega = \varepsilon$.}
    Ако $i = 0$.
    Това означава, че $\alpha \in L$, но $\beta \not\in L$.
    Следователно състоянието $q$ е едновременно финално и нефинално. Това е противоречие.
  \item
    \mynote{Съобразете, че тук $\abs{\omega} = i$ и $\abs{\delta} = n - i - 1$. Не е важно как разширяваме $\alpha$ и $\beta$ за да получим $\alpha_0$ и $\beta_0$. Тук, за да бъдем конкретни, сме избрали разширението да е просто $a^i$. Важното е това разширение да има дължина $i$.}
    Ако $i > 0$. Да разгледаме следните думи:
    \begin{align*}
      & \alpha_0 = \alpha \cdot a^{i}\\
      & \beta_0 = \beta \cdot a^{i}.
    \end{align*}
    Тогава $\alpha_0 = \omega \cdot a \cdot \delta_0$, където $|\delta_0| = n-1$. Аналогично,
    $\beta_0 = \rho \cdot b \cdot \delta_1$, където $|\delta_1| = n-1$.
    Така отново получаваме, че $\alpha_0 \in L$, но $\beta_0 \not\in L$.
    И в този случай получаваме противоречие, защото
    \[\delta^\star(\qstart,\alpha_0) = p = \delta^\star(\qstart,\beta_0)\]
    и състоянието $p$ трябва да е едновременно финално и нефинално.
  \end{itemize}  
\end{hint}

%%% Local Variables:
%%% mode: latex
%%% TeX-master: "../eai"
%%% End:
