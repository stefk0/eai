\section{Минимизация}
\label{sect:regular:minimisation}
\index{минимизация}

\begin{itemize}
\item
  Нека отново да приемем, че сме фиксирали един ДКА $\A = \FA$ като всяко състояние е достижимо от началното.
  Да напомним, че с $\L_\A(p)$ означаваме езикът, който се разпознава от автомата $\A$,
  ако приемем, че $p$ е началното състояние, т.е.
  \[\L_\A(p) \df \{\omega \in \Sigma^\star \mid \delta^\star(p,\omega) \in F\}.\]
  В частност, $\L(\A) = \L_\A(\qstart)$.
\item
  \index{$\equiv_\A$}
  Сега дефинираме следната релация между състояния на автомата $\A$:
  \[p \equiv_\A q\ \dff\ \L_\A(p) = \L_\A(q).\]
  Това означава, че $p \equiv_\A q$ точно тогава, когато
  \begin{equation}
    \label{eq:1}
    (\forall \omega\in \Sigma^\star)[\ \delta^\star(p,\omega) \in F\ \iff\ \delta^\star(q,\omega) \in F\ ].
  \end{equation}
\item
  \mynote{\writedown Защо?}
  Релацията $\equiv_\A$ между състояния на автомата $\A$ е релация на еквивалентност. 
\item
  Да означим
  \[Q/_{\equiv_\A} \df \{\ [q]_{\equiv_\A} \mid q \in Q\ \}.\]
\end{itemize}

\begin{proposition}
  За произволно състояние $q$ е изпълнено, че:
  \[{[q]}_{\equiv_\A} \cap F \neq \emptyset \iff {[q]}_{\equiv_\A} \subseteq F.\]
\end{proposition}
\begin{proof}
  Посоката $(\Leftarrow)$ е очевидна. За $(\Rightarrow)$, нека $p \in [q]_{\equiv_\A} \cap F$. Достатъчно е да докажем, че $q \in F$.
  Щом $p \in F$, то $\varepsilon \in \L_\A(p)$.
  Щом $p \equiv_\A q$, то $\L_\A(p) = \L_\A(q)$ и следователно $\varepsilon \in \L_\A(q)$.
  С други думи, $\delta^\star_\A(q,\varepsilon) \in F$, откъдето получаваме, че $q \in F$.
\end{proof}

\begin{proposition}\label{pr:equiv-delta}
  За произволни състояния $p$ и $q$ и произволна буква $a$ е изпълнено:
  \[p \equiv_\A q \implies \delta(p,a) \equiv_\A \delta(q,a).\]
\end{proposition}
\begin{proof}
  Да разгледаме произволни състояния $p$, $q$ и буква $a$. Тогава:
  \begin{align*}
    p \equiv_\A q & \iff \L_\A(p) = \L_\A(q) \\
                  & \iff (\forall \beta\in\Sigma^\star)[ \delta^\star(p,\beta) \in F \iff \delta^\star(q,\beta) \in F]\\
                  & \Rightarrow (\forall \beta\in\Sigma^\star)[\delta^\star(p, a\beta) \in F \iff \delta^\star(q,a\beta) \in F]\\
                  & \iff (\forall \beta\in\Sigma^\star)[\delta^\star(\delta(p, a),\beta) \in F \iff \delta^\star(\delta(q,a),\beta) \in F]\\
                  & \iff \L_\A(\delta(p,a)) = \L_\A(\delta(q,a))\\
                  & \iff \delta(p,a) \equiv_\A \delta(q,a).
  \end{align*}
\end{proof}

За автомата $\A = \FA$, дефинираме автомата $\A' = (\Sigma,Q',\qstart',\delta',F')$ по следния начин:
\begin{itemize}
\item
  $Q' \df \{[q]_{\equiv_\A} \mid q\in Q\}$;
\item
  $\qstart' \df [\qstart]_{\equiv_\A}$;
\item
  $\delta'([q]_{\equiv_\A}, a) \df [\delta(q,a)]_{\equiv_\A}$;
\item
  $F' \df \{[q]_{\equiv_\A}\mid [q]_{\equiv_\A} \cap F \neq \emptyset\}$;
\end{itemize}

Да отбележим, че от \Proposition{equiv-delta} имаме, че
\[[p]_{\equiv_\A} = [q]_{\equiv_\A} \implies \delta'([p]_{\equiv_\A},a) = \delta'([q]_{\equiv_\A},a).\]
С други думи, $\delta'$ наистина е функция.

\begin{proposition}
  \label{pr:minimisation-delta-1}
  За всяко състояние $q$ и дума $\alpha$ е изпълнено, че
  \begin{equation}
    \label{eq:15}
    \delta^\star_1([q]_{\equiv_\A}, \alpha) = [\delta^\star(q,\alpha)]_{\equiv_\A}.
  \end{equation}
\end{proposition}
\begin{proof}
  Ще докажем Свойство~(\ref{eq:15}) с индукция по дължината на думата $\alpha$.
  \begin{itemize}
  \item
    Нека $|\alpha| = 0$, т.е. $\alpha = \varepsilon$. Тогава
    \[\delta'^\star([q]_{\equiv_\A},\varepsilon) = [q]_{\equiv_\A} = [\delta^\star(q,\varepsilon)]_{\equiv_\A}.\]
  \item
    Да приемем, че Свойство~(\ref{eq:15}) е вярно за думи с дължина $n$.
  \item
    Нека $|\alpha| = n+1$, т.е. $\alpha = \beta a$ и $|\beta| = n$. Тогава
    \begin{align*}
      \delta'^\star([q]_{\equiv_\A},\beta a) & = \delta'(\delta'^\star([q]_{\equiv_\A},\beta),a) & \comment\text{деф. на }\delta'^\star\\
                                              & = \delta'([\underbrace{\delta^\star(q,\beta)}_{p}]_{\equiv_\A},a) & \comment\text{И.П. за }\beta\\
                                              & = \delta'([p]_{\equiv_\A},a) \\
                                              & = [\delta(p,a)]_{\equiv_\A} & \comment\text{деф. на }\delta'\\
                                             & = [\delta(\delta^\star(q,\beta),a)]_{\equiv_\A} & \comment{p = \delta'^\star(q,\beta)}\\
                                             & = [\delta^\star(q,\beta a)]_{\equiv_\A}. & \comment\text{деф. на }\delta^\star
    \end{align*}
  \end{itemize}
\end{proof}

\begin{proposition}\label{pr:quotient-bijection}
  Нека $A$ и $B$ са множества и $f: A \to B$ е сюрекция.
  Дефинираме релация на еквивалентност $\equiv$ между елементи на $A$ по следния начин:
  \[a_0 \equiv a_1 \dff f(a_0) = f(a_1).\]
  Дефинираме $[a]_\equiv$ да бъде класът на еквивалентност на $a$, т.е.
  \[[a]_\equiv \df \{a_0 \in A \mid f(a) = f(a_0)\}.\]
  Нека $A' \df \{[a]_\equiv \mid a \in A\}$.
  Тогава $f' : A' \to B$, където $f'([a]_\equiv) \df f(a)$ е биекция.
\end{proposition}
\begin{proof}
  Достатъчно е да направим следните проверки:
  \begin{itemize}
  \item 
    Ако $[a_0]_\equiv = [a_1]_\equiv$, то $f(a_0) = f(a_1)$ и следователно
    $f'([a_0]_\equiv) = f'([a_1]_\equiv)$. Това означава, че $f'$ е функция.
  \item 
    За произволен елемент $b \in B$, съществува $a \in A$, за който $f(a) = b$.
    По дефиниция, $f'([a]_\equiv) = b$. Това означава, че $f'$ е сюрекция.
  \item 
    Нека сега $[a_0]_\equiv \neq [a_1]_\equiv$, т.е. $f(a_0) \neq f(a_1)$.
    Тогава $f'([a_0]_\equiv) \neq f'([a_1]_\equiv)$. Това означава, че $f'$ е инкеция.
  \end{itemize}
  От всичко това заключаваме, че $f'$ е биекция.
\end{proof}

\begin{framed}
\begin{theorem}
  Автоматът $\A'$ е минимален ДКА за езика $\L(\A)$.
\end{theorem}  
\end{framed}
\begin{proof}
  Първо да проверим, че $\L(\A') = \L(\A)$. 
  Достатъчно е да проследим следните еквивалентности за произволна дума $\alpha$:
  \begin{align*}
    \alpha \in \L(\A) & \iff \delta^\star(\qstart,\alpha) \in F & \comment\text{деф. на }\L(\A)\\
                      & \iff [\delta^\star(\qstart,\alpha)]_{\equiv_\A} \in F' & \comment\text{деф. на }F'\\
                      & \iff \delta'^\star([\qstart]_{\equiv_\A},\alpha) \in F' & \comment\text{\Proposition{minimisation-delta-1}}\\
                      & \iff \delta'^\star(\qstart',\alpha) \in F' & \comment{\qstart' \df [\qstart]_{\equiv_\A}}\\
                      & \iff \alpha \in \L(\A'). & \comment\text{деф. на }\L(\A')
  \end{align*}
  
  За да докажем, че $\A'$ е минимален ДКА, според \Theorem{brzozowski-minimal:unique}, достатъчно е да докажем, че $|Q'| = |Q^\B|$.
  От доказателството на \Lemma{brzozowski:surjective} знаем, че функцията $f:Q^\A \to Q^\B$ е сюрекция, където
  \[f(q) = \hat{M} \dff M = \L_\A(q).\]
  Нека дефинираме $f':Q' \to Q^\B$ като
  \[f'([q]_{\equiv_\A}) \df f(q).\]
  От \Proposition{quotient-bijection} имаме, че $f'$ е биекция.
  Заключаваме, че $|Q'| = |Q^\B|$.
\end{proof}


%%% Local Variables:
%%% mode: latex
%%% TeX-master: "../eai"
%%% End:
