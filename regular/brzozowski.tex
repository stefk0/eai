\section{Метод на Бжозовски за строене на автомат}
\index{Бжозовски}

\marginpar{Бжозовски \cite{brzozowski-derivatives} описва алгоритъм за строене на автомат по регулярен израз.}

Да видим как можем директно да построим минимален тотален краен детерминиран автомат $\A$
разпознаващ регулярен език $L$, където \[\A = \pair{Q,\Sigma,\qstart,\delta,F}.\]
%\marginpar{За момента няма как да знаем, че автоматът, който ще построим е минимален. Това ще докажем по-късно след като разгледаме теоремата на Майхил-Нероуд}
Състоянията на автомата ще бъдат от вида $q_N$, където $N \subseteq \Sigma^\star$, така че накрая искаме да имаме свойството
\[N = \{\omega \in \Sigma^\star \mid \delta^\star(q_N,\omega) \in F\}.\]
Тогава е ясно, че трябва да имаме $\qstart = q_L$, защото 
\[L = \L(\A) = \{\omega \in \Sigma^\star \mid \delta^\star(q_L,\omega) \in F\}.\]
Нека да разгледаме езика $L = \L(\mathbf{((a+b)^+\cdot a)^\star})$.
Сега едновременно ще строим състоянията на автомата $Q$ и функцията на преходите $\delta$.
Преди да започнем, удобно е да имаме предвид следното представяне
\begin{align*}
  L & = \{\varepsilon\} \cup \{a,b\}^+ a L\\
  & = \{\varepsilon\} \cup a\{a,b\}^\star aL \cup b\{a,b\}^\star aL
\end{align*}

\begin{itemize}
\item 
  $a^{-1}(L) = \{a,b\}^\star aL \df M$.
  Имаме, че $M \neq L$, защото $\varepsilon \in L$, но $\varepsilon \not\in M$.
  Понеже $M \neq L$, имаме ново състояние $M$ в автомата и 
  $\delta_\A(L,a) = M$.
  Удобно е да представим
  \begin{align*}
    M & \df \{a,b\}^\star aL\\
    & = aL \cup \{a,b\}^+aL\\
    & = aL \cup aM \cup bM
  \end{align*}
\item
  $b^{-1}(L) = \{a,b\}^\star aL = M$;
  Следователно, $\delta_\A(q_L,b) = q_M$.
\item
  $a^{-1}(M) = L \cup M \df N$. 
  Имаме, че $N \neq L$, защото $a\in N$, но $a \not\in L$.
  Освен това, $N \neq M$, защото $\varepsilon \in N$, но $\varepsilon \not\in M$.
  Понеже $N \neq L, M$, имаме ново състояние $q_N$ в автомата и 
  $\delta_\A(q_M,a) = q_N$.
  Освен това,
  \begin{align*}
    N & \df L \cup M \\
    & = \{\varepsilon\} \cup aM \cup bM \cup M\\
    & = \{\varepsilon\} \cup aM \cup bM \cup aL.
  \end{align*}
\item
  $b^{-1}(M) = M$. Следователно, $\delta_\A(q_M,b) = q_M$.
\item
  $a^{-1}(N) = L \cup M = N$. Следователно, $\delta_\A(q_N,a) = q_N$.
\item
  $b^{-1}(N) = M$.
  Следователно, $\delta_\A(q_M,b) = q_M$.
\end{itemize}
От горните сметки следва, че 
\[(\forall \alpha \in \Sigma^\star)[\alpha^{-1}(L) \in \{L, N, M\}].\]
Така получихме, че 
\[Q = \{q_L, q_N, q_M\}.\]
Нека сега да съобразим кои са финалните състояния.
Понеже имаме свойството 
\[L = \{\alpha \in \Sigma^\star \mid \varepsilon \in \alpha^{-1}(L)\},\]
то следва, че финалните състояния на автомата $\A$ са $q_L$ и $q_N$,
защото $\varepsilon \in L,N$. 
Сега вече сме готови да нарисуваме картинка на автомата.

\begin{framed}
\begin{figure}[H]
  \begin{subfigure}[b]{.4\textwidth}
    \begin{tikzpicture}[->,>=stealth,thick,node distance=55pt]
      \tikzstyle{every state}=[circle,minimum size=20pt,font=\small]
      \node[initial above, state,accepting]   (L) {$q_L$};
      \node[state]                            (M) [right of=L]{$q_M$};
      \node[state,accepting]                  (N) [right of=M]{$q_N$};
      
      
      \path 
      (L) edge [bend left=15] node [above] {$a,b$}   (M)
      (M) edge [loop above] node [above] {$b$} (M)
      (M) edge [bend left=15] node [above] {$a$} (N)
      (N) edge [bend left=15] node [below] {$b$} (M)
      (N) edge [loop above] node [above] {$a$} (N);
    \end{tikzpicture}
    \caption{Автомат $\A$ за езика $\L(\mathbf{((a+b)^+a)^\star})$}
  \end{subfigure}
  \quad
  ~
  \quad
  \begin{subfigure}[b]{.4\textwidth}
    \begin{tikzpicture}[->,>=stealth,thick,node distance=55pt]
      \tikzstyle{every state}=[circle,minimum size=20pt,auto]
      
      \node[initial above, state,accepting]   (L) {$[\varepsilon]_L$};
      \node[state]                            (M) [right of=L]{$[a]_L$};
      \node[state,accepting]                  (N) [right of=M]{$[aa]_L$};
      
      
      \path 
      (L) edge [bend left=15] node [above] {$a,b$}   (M)
      (M) edge [loop above] node [above] {$b$} (M)
      (M) edge [bend left=15] node [above] {$a$} (N)
      (N) edge [bend left=15] node [below] {$b$} (M)
      (N) edge [loop above] node [above] {$a$} (N);
    \end{tikzpicture}
    \caption{Минимален автомат $\M$ за езика $\L(\mathbf{((a+b)^+a)^\star})$}
  \end{subfigure}
\end{figure}
\end{framed}

Конструкцията на автомата $\A$ е следната:
\begin{itemize}
\item 
  $Q^\A = \{q_M \mid M = \alpha^{-1}(L) \text{, за някоя }\alpha \in \Sigma^\star\}$;
\item
  $\qstart^\A = q_L$;
\item
  $F^\A = \{q_N \in Q^\A \mid \varepsilon \in N\}$;
\item
  $\delta_\A(q_N,a) = q_M$, където $M = a^{-1}(N)$.
\end{itemize}

\begin{example}
  Да разгледаме езика $L = \L(\mathbf{a\cdot(a+b)^\star\cdot b})$.
  \begin{itemize}
  \item 
    $a^{-1}(L) = \{a,b\}^\star b \df M$.
    Имаме, че $M \neq L$, защото $b \in M$, но $b \not\in L$.
    Тогава $\delta(q_L, a) \df q_M$.
    За по-нататък ще е удобно да представим $M$ по следния начин:
    \[M = a\cdot \{a,b\}^\star \cdot b \cup b\cdot \{a,b\}^\star \cdot b \cup \{b\}.\]
  \item
    $b^{-1}(L) = \emptyset$. Тогава $\delta(q_L,b) \df q_\emptyset$;
  \item
    $a^{-1}(M) = \{a,b\}^\star b = M$.
    Тогава $\delta(q_M,a) = q_M$.
  \item
    $b^{-1}(M) = \{\varepsilon\} \cup \{a,b\}^\star b \df N$.
    Ясно е, че $N \neq L,M$, защото $\varepsilon \in N$, но $\varepsilon \not\in L,M$. Тогава
    \[\delta(q_M,b) = q_N.\]
    Също така, $q_N$ е финално състояние.
    Нека за удобство да представим езика $N$ по следния начин:
    \[N = \{\varepsilon\} \cup a\cdot\{a,b\}^\star\cdot b \cup b\cdot\{a,b\}^\star\cdot b \cup \{b\}.\]
  \item
    $a^{-1}(N) = \{a,b\}^\star b = M$, т.е. $\delta(q_N,a) = q_M$;
  \item
    $b^{-1}(N) = \{a,b\}^\star b = M$, т.е. $\delta(q_N,b) = q_N$;
  \item
    Ясно е, че $a^{-1}(\emptyset) = b^{-1}(\emptyset) = \emptyset$, т.е.
    $\delta(\emptyset,a) = \delta(\emptyset,b) = \emptyset$.
  \end{itemize}
\begin{framed}
  \begin{figure}[H]
    \centering
    \begin{tikzpicture}[->,>=stealth,thick,node distance=55pt]
      \tikzstyle{every state}=[circle,minimum size=20pt,auto]
      
      \node[initial, state]                   (L) {$q_L$};
      \node[state]                            (M) [above right of=L]{$q_M$};
      \node[state]                            (E) [below right of=L]{$q_\emptyset$};
      \node[state,accepting]                  (N) [right of=M]{$q_N$};
      
      \path 
      (L) edge [bend left=15]  node [above] {$a$} (M)
      (L) edge [bend right=15] node [above] {$b$} (E)
      (E) edge [loop right]    node [right] {$a,b$} (E) 
      (M) edge [bend right=15] node [below] {$b$} (N)
      (M) edge [loop above]    node [above] {$a$} (M)
      (N) edge [bend right=30] node [above] {$a$} (M)
      (N) edge [loop above]    node [above] {$b$} (N);
    \end{tikzpicture}
    \caption{Минимален автомат за $\L(\mathbf{a\cdot (a+b)^\star\cdot b})$}
  \end{figure}
\end{framed}
\end{example}

\begin{example}
  Да разгледаме езика 
  \[L = \{\omega \in \{a,b\}^\star \mid \omega \text{ съдържа четен брой $a$ и точно едно $b$}\}.\]
  Нека да видим дали можем да построим автомат за този език.
  \begin{itemize}
  \item 
    $a^{-1}(L) = \{\alpha \in \{a,b\}^\star \mid \alpha \text{ съдържа нечетен брой $a$ и точно едно $b$}\} \df M$;
  \item 
    $b^{-1}(L) = \{\alpha \in \{a,b\}^\star \mid \alpha \text{ съдържа четен брой $a$ и нито едно $b$}\} \df N$;
  \item
    $a^{-1}(M) = L$;
  \item
    $b^{-1}(M) = \{\alpha \in \{a,b\}^\star \mid \alpha \text{ съдържа нечетен брой $a$ и нито едно $b$}\} \df P$;
  \item
    $a^{-1}(N) = P$;
  \item
    $b^{-1}(N) = \emptyset$;
  \item
    $a^{-1}(P) = N$;
  \item
    $b^{-1}(P) = \emptyset$;
  \end{itemize}

\begin{framed}
  \begin{figure}[H]
    \centering
    \begin{tikzpicture}[->,>=stealth,thick,node distance=65pt]
      \tikzstyle{every state}=[circle,minimum size=20pt,auto]
      
      \node[initial, state]        (L) {$q_L$};
      \node[state]                 (M) [above right of=L]{$q_M$};
      \node[state, accepting]      (N) [below right of=M]{$q_N$};
      \node[state]                 (P) [right of=M]{$q_P$};
      \node[state]                 (E) [right of=N]{$q_\emptyset$};
            
      \path 
      (L) edge [bend right=15]  node [below] {$a$} (M)
      (M) edge [bend right=15]  node [above] {$a$} (L)
      (L) edge [bend right=15]  node [above] {$b$} (N)
      (M) edge [bend left=15]   node [above] {$b$} (P)
      (N) edge [bend left=15]   node [left] {$a$} (P)
      (P) edge [bend left=15]   node [right] {$a$} (N)
      (P) edge [bend left=15]   node [right] {$b$} (E)
      (N) edge [bend right=15]  node [below] {$b$} (E);
    \end{tikzpicture}
    \caption{Минимален автомат, който приема думи с четен брой $a$ и точно едно $b$}
  \end{figure}  
\end{framed}
\end{example}


\begin{example}
  Да разгледаме езика $L = \{a^nb^n\mid n \in \Nat\}$. Да се опитаме да построим автомат, който го разпознава.
  Нека
  \[L_k \df \{a^nb^{n+k}\mid n \in \Nat\}.\]
  Да видим какво се получава като приложим процедурата за строене 
  на минимален автомат.
  \begin{itemize}
  \item 
    $a^{-1}(L) = L_1$;
  \item
    $b^{-1}(L) = \emptyset$;
  \item
    $a^{-1}(L_1) = L_2$;
  \item
    $b^{-1}(L_1) = \{\varepsilon\}$;
  \item
    $a^{-1}(\{\varepsilon\}) = b^{-1}(\{\varepsilon\}) = \emptyset$;
  \item
    Вижда се, че $a^{-1}(L_k) = L_{k+1}$, за всяко $k$.
  \item
    Вижда се, че $b^{-1}(L_{k+1}) = \{b^k\}$, за всяко $k$.
    Освен това е ясно, че $b^{-1}(\{b^{k}\}) = \{b^{k-1}\}$, за всяко $k \geq 1$.
  \end{itemize}
  Получаваме, че езикът $L$ се разпознава от автомат с {\em безкрайно много състояния}.

\begin{framed}  
  \begin{figure}[H]
    \centering
    \begin{tikzpicture}[->,>=stealth,thick,node distance=70pt]
      \tikzstyle{every state}=[circle,minimum size=15pt,auto]
      
      \node[state, initial above]             (0) {$L$};
      \node[state]                            (1) [right of=0]{$L_1$};
      \node[state]                            (2) [right of=1]{$L_2$};
      \node[state]                            (3) [right of=2]{$L_3$};
      \node[state]                            (A) [below of=1]{$\emptyset$};
      \node[state]                            (B) [below right of=1]{$\{b\}$};
      \node[state]                            (BB) [below right of=2]{$\{bb\}$};
      \node[state, accepting]                 (E) [below of=A]{$\{\varepsilon\}$};
      
      \coordinate[right of=3] (4);
      \coordinate[below right of=3] (BBB);
      \coordinate[below of=4] (BBBA);

      \path 
      (0) edge [bend left=15]   node [above] {$a$} (1)
      (1) edge [bend left=15]   node [above] {$a$} (2)
      (2) edge [bend left=15]   node [above] {$a$} (3)
      (0) edge [bend right=30]  node [left] {$b$} (E)
      (E) edge [loop left]      node [left] {$a,b$} (E)
      (1) edge [bend right=30]  node [left] {$b$} (E)
      (2) edge [bend right=15]  node [left] {$b$} (B)
      (3) edge [bend right=15]  node [left] {$b$} (BB)
      (B) edge [bend right=15]  node [above] {$b$} (A)
      (B) edge [bend left=15]  node [right] {$a$} (E)
      (A) edge [bend right=15]   node [right] {$a,b$} (E)
      (BB) edge [bend right=15] node [above] {$b$} (B)
      (BB) edge [bend left=15]  node [below] {$a$} (E);
      
      \draw [dashed,->,shorten >=0pt] (3) to[bend left=15] node[auto] {$a$} (4);
      \draw [dashed,->,shorten >=0pt] (BBB) to[bend right=15] node[above] {$b$} (BB);
      \draw [dashed,->,shorten >=0pt] (BBBA) to[bend left=30] node[below] {$a$} (E);
    \end{tikzpicture}
    \caption{{\em Безкраен} автомат за $\{a^nb^n \mid n \in \Nat\}$}
  \end{figure}
\end{framed}
\end{example}

%%% Local Variables:
%%% mode: latex
%%% TeX-master: "../eai"
%%% End:
