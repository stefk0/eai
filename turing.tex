\chapter{Машини на Тюринг}

\newcommand{\tape}[1]{\dots\bot\bot\bot{#1}\bot\bot\bot\dots}
$\M = \TM$

\begin{itemize}
\item 
  $Q$ - състояния;
\item
  $\Sigma$ - азбука за входа;
\item
  $\Gamma$ - азбука за лентата, $\Sigma \subseteq \Gamma$;
\item
  $\delta:Q\times\Gamma \to Q\times \Gamma \times \{L,R\}$ - (частична) функция на преходите;
\item
  $s$ - начално състояние, $s \in Q$;
\item
  $\bot$ - празен символ,  $\bot \in \Gamma \setminus \Sigma$;
\item
  $F$ - финални състояния, $F \subseteq Q$.
\end{itemize}

\begin{itemize}
\item 
  Първоначално, лентата съдържа входа, който е обграден от безкрайно много символи $\bot$ и в двете посоки.
\item
  МТ се намира в началното състояние $s$ и главата е върху най-левия символ от входа.
\item
  Работим с конфигурации $(\alpha, q, \beta) \in \Gamma^\star\times Q \times \Gamma^\star$. Това означава, че
  машината се намира в състояние $q$ и лентата има вида
  \[\tape{\alpha\beta}\]
\item
  Ако $\delta(q,Z) = (p,Y,R)$, то пишем
  $(\alpha, q, Z\beta) \vdash (\alpha Y, p, \beta)$.
  При $Z = \bot$, също така можем да запишем 
  $(\alpha, q, \varepsilon) \vdash (\alpha Y, p, \varepsilon)$

  Ако $\delta(q,Z) = (p,Y,L)$, то пишем
  $(\alpha X, q, Z\beta) \vdash (\alpha , p, XY\beta)$.
  При $X = \bot$, $(\varepsilon, q, Z\beta) \vdash (\varepsilon, p, BY\beta)$.
\item
  Езикът, който се {\bf разпознава чрез финални състояния} от машината $M$ е:
  \[L_F(M) = \{\alpha\in\Sigma^\star \mid (\varepsilon, s, \alpha) \vdash^\star (\beta, q, \gamma)\ \&\ q\in F\ \&\ \beta,\gamma\in\Gamma^\star\}.\]
\item
  Езикът, който се {\bf разпознава чрез спиране} от $M$ е:
  \[L_H(M) = \{\alpha \in \Sigma^\star \mid (\varepsilon, s, \alpha) \vdash^\star (\beta, q, X\gamma)\ \&\ \neg !\delta(q,X)\}\]
\end{itemize}

Може да се докаже, че се разпознават едни и същи езици.

\index{език!полуразрешим}
Езиците, които се разпознават от МТ се наричат {\bf полуразрешими езици}.

\subsection*{Канонична подредба на $\Sigma^\star$}

Нека $\Sigma = \{a_0,a_1,\dots,a_{k-1}\}$.
Подреждаме думите по ред на тяхната дължина.
Думите с еднаква дължина подреждаме по техния числов ред, т.е.
гледаме на буквите $a_i$ като числото $i$ в $k$-ична бройна система.
Тогава подреждаме думите с дължина $n$, са числата от $0$ до $k^n-1$,
записани в $k$-ична бройна система.
Да означим с $w_i$ $i$-тата дума в $\Sigma^\star$ при тази подредба.

\begin{example}
  Ако $\Sigma = \{0,1\}$, то наредбата започва така:
  \[\varepsilon, 0, 1, \underbrace{00, 01, 10, 11}_{\text{от $0$ до $3$}}, \underbrace{000, 001, 010, 011, 100, 101, 110, 111}_{\text{от $0$ до $7$}}, 0000, 0001, \dots\]  
\end{example}

\subsection*{Многолентови машини на Тюринг}

\subsection*{Недетерминистични машини на Тюринг}

\begin{thm}
  Ако $L$ се разпознава от НМТ $\M_1$, до $L$
  също се разпознава и от ДМТ $\M_2$.
\end{thm}

\subsection*{Полуразрешими и разрешими езици}

\begin{thm}
  Ако $L$ и $\Sigma^\star \setminus L$ са полуразрешими езици, то $L$ е разрешим език.
\end{thm}

\section{Универсална машина на Тюринг}
За простота, нека $\Sigma = \{0,1\}$ и $\Gamma = \{0,1,\bot\}$.
\begin{itemize}
\item 
  $X_1 = 0$, $X_2 = 1$, $X_3 = \bot$;
\item
  $D_1 = L$, $D_2 = R$
\end{itemize}

\subsection*{Кодиране на преход}
Да разгледаме прехода $\delta(q_i,X_j) = (q_k,X_l,D_m)$.
Кодираме този преход по следния начин:
\[0^i10^j10^k10^l10^m\]
Да Обърнем внимание, че в този двоичен код няма последователни единици и той 
започва и завършва с нула.
\subsection*{Кодиране на машина на Тюринг}
За да кодираме една машина на Тюринг $\M$ е достатъчно да кодираме функцията на преходите $\delta$.
Понеже $\delta$ е крайна функция, нека с числото $r$ да означим броя на всички възможни преходи.
По описания по-горе начин, нека $code_i$ е числото в двоичен запис, получено за $i$-тия преход на $\delta$.
Тогава кодът на $\M$ е следното число в двоичен запис:
\[\pair{\M} = 111\ code_1\ 11\ code_2\ 11\ \cdots\ 11\ code_r\ 111.\]
\begin{itemize}
\item
  Лесно се съобразява, че за две МТ $\M$ и $\M'$ с различни функции на преходите, имаме $\pair{\M} \neq \pair{\M'}$.
\item
  Ще казваме, че числото $r$ е {\bf код на } $\M$, ако числото $r$, записано в двоичен запис представлява думата $\pair{\M}$.
  Оттук нататък, когато пишем $\M_r$, ще имаме предвид машината на Тюринг с код $r$.
\item
  С $\pair{\M,w}$ ще означаваме кода на МТ $\M$ при вход $w$ е числото с двоичен запис описанието на $\M$ и след това прикрепена думата $w$.
  При едно число $r = \pair{M,w}$, лесно се намира кода на $\M$.
  Просто започваме да четем двоичния запис на $r$ докато не срещнем за втори път $111$.
  След това започва думата $w$.
\end{itemize}

\section{Пример за език, който не се разпознава от МТ}

Да разгледаме безкрайната таблица $\{a_{ij} \mid i,j \in \Nat\}$, където:
\begin{align*}
  a_{ij} = 
  \begin{cases}
    1, & \text{ ако } w_i \in L(\M_j), \\
    0, & \text{ ако } w_i \not\in L(\M_j).
  \end{cases}
\end{align*}
Идеята е да вземем диагонала на тази таблица.

\begin{framed}
  Езикът 
  $L_d = \{w_i \mid w_i \not\in L(\M_i)\}$ не се разпознава от MT.
\end{framed}
Да допуснем, че $L_d$ се разпознава от МТ, т.е. $L_d = L(\M_i)$, за някоя МТ с код $i$.
Тогава:
\begin{itemize}
\item 
  Ако
  $w_i \in L_d\ \rightarrow\ w_i \in L(\M_i)\ \rightarrow\ w_i \not\in L_d$;
\item
  Ако 
  $w_i \not\in L_d\ \rightarrow\ w_i \not\in L(\M_i)\ \rightarrow\ w_i \in L_d$.
\end{itemize}

\section{Универсалният език $L_u$}

Да разгледаме езика $L_u = \{\pair{\M,w} \mid w\in L(\M)\}$.

\section{Проблемът за съответствие на Пост (PCP)}

\subsection*{MPCP}

\section{Разрешими и полуразрешими езици}

\section{Проблеми за безконтекстни езици}

\begin{lemma}
  Нека е дадена $\M = \TM$.
  Тогава езикът 
  \[L = \{\alpha\sharp\beta^R \mid \alpha,\beta \in \Gamma^\star Q \Gamma^\star\ \&\  \alpha \vdash_\M \beta\}\]
  е безконтекстен.
\end{lemma}
\begin{proof}
  Ще покажем, че съществува стеков автомат $P$, за който $\L_S(P) = L$.
  Четем буквата $X$. Тогава:
  \begin{itemize}
  \item 
    ако $\delta_\M(q,X) =(p,Y,R)$, то слагаме $Yp$ на върха на стека;
  \item
    ако $\delta_\M(q,X) =(p,Y,L)$, то ако $Z$ е върха на стека, заменяме $Z$ с $pZY$;
  \end{itemize}
\end{proof}

\begin{lemma}
  Нека е дадена $\M = \TM$.
  Тогава езикът 
  \[L = \{\alpha\sharp\beta^R \mid \alpha,\beta \in \Gamma^\star Q \Gamma^\star\ \&\  \alpha \not\vdash_\M \beta\}\]
  е безконтекстен.
\end{lemma}


\begin{thm}
  Неразрешим е проблемът за проверка дали при дадени две произволни безконтекстни граматики $G_1$ и $G_2$,
  $\L(G_1) \cap \L(G_2) = \emptyset$.  
\end{thm}

\begin{thm}
  Неразрешим е проблемът за проверка дали при дадена произволна безконтекстна граматика $G$,
  $\L(G) = \Sigma^\star$.  
\end{thm}


\section{Въпроси}

Вярно ли е, че следният проблем е {\em разрешим}:
\begin{itemize}
\item
  за произволна безконтекстна граматика $G$, проверява дали $\L(G) = \emptyset$?
\item
  за произволна безконтекстна граматика $G$, проверява дали $\L(G) = \Sigma^\star$?
\item
  за произволни безконтекстни граматики $G_1$ и $G_2$, проверява дали $\L(G_1) \cap \L(G_2) = \emptyset$?
\item
  за произволни безконтекстни граматики $G_1$ и $G_2$, проверява дали $\L(G_1) \cap \L(G_2) = \Sigma^\star$?
\item
  за произволни безконтекстни граматики $G_1$ и $G_2$, проверява дали $\L(G_1) = \L(G_2)$?
\item
  за произволни безконтекстни граматики $G_1$ и $G_2$, проверява дали $\L(G_1) \subseteq \L(G_2)$?
\item
  за произволна безконтекстна граматика $G$ и произволен регулярен израз $r$,
  проверява дали $\L(G) = \L(r)$?
\item
  за произволна безконтекстна граматика $G$ и произволен регулярен израз $r$,
  проверява дали $\L(G) \subseteq \L(r)$?
\item
  за произволна безконтекстна граматика $G$ и произволен регулярен израз $r$,
  проверява дали $\L(r) \subseteq \L(G)$?
% \item
%   за произволни безконтекстни граматики $G_1$ и $G_2$, проверява дали $\L(G_1) \subseteq \L(G_2)$ 
%   е безконтекстен език ?
% \item
%   за произволна безконтекстна граматика $G$, проверява дали $\Sigma^\star \setminus \L(G)$
%   е безконтекстен език ?
% \item
%   за произволна безконтекстна граматика $G$, проверява дали $\L(G)$ е регулярен език?
\end{itemize}
%%% Local Variables: 
%%% mode: latex
%%% TeX-master: "discrete-math"
%%% End: 
