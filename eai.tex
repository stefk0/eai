\documentclass[a4paper, 12pt, oneside]{report}

% \usepackage[left=0.0cm, right=0.5cm]{geometry}
% \special{papersize=9cm,12cm}

\usepackage{cmap}
\usepackage{gitinfo2}

%%%%%%%%%%%%%
%% MARGINS %%
%%%%%%%%%%%%%

\setlength{\marginparsep}{1cm}
\setlength{\oddsidemargin}{0.3cm}
\setlength{\hoffset}{-0.75in}
\setlength{\voffset}{-0.75in}
\setlength{\marginparwidth}{110pt}
\setlength{\textwidth}{420pt}
\setlength{\textheight}{670pt}

% \setlength{\leftmargin}{0.2cm}

\let\oldmarginpar\marginpar
\renewcommand\marginpar[1]{\leavevmode\oldmarginpar{\raggedright\scriptsize #1}}
% \renewcommand\marginpar[1]{}

  
% \renewcommand\marginpar[1]{\-\oldmarginpar[\raggedleft\scriptsize #1]%
% {\raggedright\scriptsize #1}}
%\renewcommand\marginpar[1]{\oldmarginpar{\scriptsize #1}}

%%%%%%%%%%%%%%

\usepackage[bulgarian]{babel}
\usepackage[OT2,T1]{fontenc}
\usepackage[utf8]{inputenc}
\usepackage[pdfauthor={Stefan Vatev}, pdftitle={Languages, Automata, Computability}, colorlinks=true, linkcolor=blue, pdfstartview=FitV, citecolor=green, urlcolor=blue]{hyperref}
\usepackage{pifont}
\usepackage{amssymb, amsmath, amsthm, mathrsfs, latexsym, bm}

\usepackage{makeidx}
\usepackage{layout, framed}
\usepackage{bussproofs, algorithm}
\usepackage{minted}
\usepackage[noend]{algpseudocode}
\usepackage{float}

\usepackage{paralist}
\usepackage[shortlabels]{enumitem}
\setlist{leftmargin=*}

%%%%%%%%%%%%%%% TIKZ Package %%%%%%%%%%%%%%%%%%%%%%%
\usepackage{tikz, pgf}
\usetikzlibrary{arrows,automata}
\usetikzlibrary{positioning}
\usetikzlibrary{backgrounds}
%%%%%%%%%%%%%%%%%%%%%%%%%%%%%%%%%%%%%%%%%%%%%%%%%%%%
\usepackage{caption, subcaption}

\theoremstyle{definition}
\newtheorem{thm}{Теорема}[chapter]
\newtheorem{cor}{Следствие}[chapter]
\newtheorem{lemma}{Лема}[chapter]
\newtheorem{prop}{Твърдение}[chapter]
\newtheorem{dfn}{Определение}[chapter]
\newtheorem{problem}{Задача}[chapter]
\newtheorem{example}{Пример}[chapter]
\newtheorem{question}{Въпрос}[chapter]
\newtheorem*{remark}{Забележка}
\renewenvironment{proof}{\noindent{\bf Доказателство.}\hspace*{1em}}{\qed\par}
\newenvironment{hint}{\noindent{\bf Упътване.}\hspace*{1em}}{\qed\par}
\newenvironment{solution}{\noindent{\bf Решение.}\hspace*{1em}}{\qed\par}

\usepackage{mysymbols}

\newcommand{\writedown}{\ding{45}\ }
\newcommand{\qstart}{q_{\texttt{start}}}
\newcommand{\qaccept}{q_{\texttt{accept}}}
\newcommand{\qreject}{q_{\texttt{reject}}}

\newcommand{\FA}{\langle{\Sigma,Q,\qstart,\delta,F}\rangle}
\newcommand{\FAn}[1]{\langle{\Sigma,Q_#1,s_#1,\delta_#1,F_#1}\rangle}
\newcommand{\NFA}{\langle{\Sigma,Q,\qstart,\Delta,F}\rangle}
\newcommand{\NFAn}[1]{\langle{\Sigma,Q_#1,s_#1,\Delta_#1,F_#1}\rangle}
\newcommand{\PDA}{\langle{Q,\Sigma,\Gamma,\qstart,\Delta,F}\rangle}
\newcommand{\PDAn}[1]{\langle{Q_#1,\Sigma,\Gamma,\#,\s_#1,\Delta_#1,F_#1}\rangle}
\newcommand{\CFG}{\langle{V,\Sigma,R,S}\rangle}
\newcommand{\TM}{\langle{Q,\Sigma,\Gamma,\delta,\blank,\qstart,\qaccept, \qreject}\rangle}

\newcommand{\Th}[1]{{\em Теорема~\ref{th:#1}}}
\newcommand{\Lem}[1]{{\em Лема~\ref{lem:#1}}}
\newcommand{\Cor}[1]{{\em Следствие~\ref{cor:#1}}}
\newcommand{\Prob}[1]{{\em Задача~\ref{prob:#1}}}
\newcommand{\Prop}[1]{{\em Твърдение~\ref{pr:#1}}}
\newcommand{\Ex}[1]{{\em Пример~\ref{ex:#1}}}

\newif\ifhints
\newif\ifcode
\hintstrue
\codetrue


% \setsecheadstyle{\large\usefont{T2A}{fag}{b}{r}} %\scshape
% \setsubsecheadstyle{\bfseries\sffamily}

% \renewcommand\familydefault{\sfdefault}

\title{Записки по ,,Езици, автомати, изчислимост''}
\author{Стефан Вътев\thanks{Това е чернова. Възможни са неточности и грешки, а също и несъответсвия с терминологията въведена на лекции. За забележки и коментари: \href{mailto:stefanv@fmi.uni-sofia.bg}{stefanv@fmi.uni-sofia.bg}.
    Конкретната версия на \LaTeX файловете, от които е компилиран този файл, може да бъде намерена \href{https://github.com/stefk0/eai/commit/\gitHash}{тук}.}}

\makeindex
\begin{document}
\maketitle
% \layout


\thispagestyle{empty}
\begin{tikzpicture}
\pgftransformscale{.7}

\draw[very thick] (10,0) -- (-10,0);

\draw (-3,0) parabola bend (0,3.5) (3,0);
\node at (0,2.2) {
	\begin{tabular}{c}
          Крайни \\ езици
	\end{tabular}
};

\draw (-5.5,0) parabola bend (0,6) (5.5,0);
\node at (0,4.8) {
  \begin{tabular}{c}
    Регулярни \\ езици
  \end{tabular}
};

\draw (-7.5,0) parabola bend (0,8) (7.5,0);
\node at (0,6.8) {  
  \begin{tabular}{c}
    Безконтекстни \\ езици
  \end{tabular}
};

\draw (-9.5,0) parabola bend (0,10) (9.5,0);
\node at (0,8.8) {  
  \begin{tabular}{c}
    Разрешими \\ езици
  \end{tabular}
};

\draw[very thick] (-9.5,0) parabola bend (0,12.5) (9.5,0);
\node at (0,11) {
  \begin{tabular}{c}
    Полуразрешими \\ езици
  \end{tabular}
};
\end{tikzpicture}

%%% Local Variables:
%%% mode: latex
%%% TeX-master: "eai"
%%% End:


\tableofcontents

\chapter{Увод}
\label{ch:intro}

\section{Съждително смятане}\label{sect:propositional}
\mynote{На англ. Propositional calculus}

Както при езиците за програмиране, всяка логика има свой синтаксис и семантика.
Тук ще разгледаме класическата съждителна логика, при която те са сравнително прости.

Съждителното смятане наподобява аритметичното смятане, като вместо аритметичните операции $+,-,\cdot,/$, 
имаме съждителни операции като $\neg, \wedge, \vee$.
Например, $(p\vee q) \wedge \neg  r$ е съждителна формула.
Освен това, докато аритметичните променливи приемат стойности числа, то
съждителните променливи приемат само стойности {\bf истина (1)} или {\bf неистина (0)}.

\mynote{Това не е формална дефиниция, но за момента е достатъчно.}
{\bf Съждителна формула} наричаме съвкупността от съждителни променливи $p,q,r,\dots$, свързани със знаците за логически операции
$\neg$, $\vee$, $\wedge$, $\rightarrow$, $\leftrightarrow$ и скоби, определящи реда на операциите.

\subsection*{Съждителни операции}

\begin{itemize}
\item
  Отрицание $\neg$
\item 
  Дизюнкция $\vee$
\item
  Конюнкция $\wedge$
\item
  Импликация $\rightarrow$
\item
  Еквивалентност $\iff$
\end{itemize}

Ще използваме таблица за истинност за да определим стойностите на основните съждителни операции
при всички възможни набори на стойностите на променливите.

\[
\begin{array}{|c|c|c|c|c|c|c|c|c|}
  \hline
  p & q & \neg p & p \vee q & p \wedge q & p \rightarrow q & \neg p \vee q & p \iff q & (p \wedge q)\ \vee\ (\neg p\wedge \neg q) \\
  \hline
  0 & 0 & 1 & 0 & 0 & 1 & 1 & 1 & 1\\
  \hline
  0 & 1 & 1 & 1 & 0 & 1 & 1 & 0 & 0\\
  \hline
  1 & 0 & 0 & 1 & 0 & 0 & 0 & 0 & 0\\
  \hline
  1 & 1 & 0 & 1 & 1 & 1 & 1 & 1 & 1\\
  \hline
\end{array}
\]


{\bf Съждително верен} (валиден) е този логически израз, който има верностна стойност {\bf 1} при всички възможни набори на
стойностите на съждителните променливи в израза, т.е. стълбът на израза в таблицата за истинност трябва да съдържа само 
стойности {\bf 1}. 

Два съждителни израза $\varphi$ и $\psi$ са {\bf еквивалентни}, което означаваме $\varphi \equiv \psi$, ако са съставени от 
едни и същи съждителни променливи и двата израза имат едни и същи верностни стойности при всички комбинации от верностни 
стойности на променливите. С други думи, колоните на двата израза в съответните им таблици за истинност трябва да съвпадат.
Така например, от горната таблица се вижда, че 
$p\to q \equiv \neg p \vee q$ и $p \iff q \equiv (\neg{p}\wedge q)\ \vee\ (p\wedge \neg q)$.

\subsection{Съждителни закони}

\begin{enumerate}[I)]
\item
  {\bf Закон за идемпотентността}
  \[p \land p \equiv p\]
  \[p \lor p \equiv p\]
\item
    {\bf Комутативен закон}
    \[p\vee q \equiv q\vee p\] 
    \[p \wedge q \equiv q \wedge p\]
  \item
    {\bf Асоциативен закон}
    \[(p\vee q)\vee r \equiv p\vee(q\vee r)\]
    \[(p\ \wedge\ q)\ \wedge\ r \equiv p\ \wedge\ (q\ \wedge\ r)\]
  \item
    {\bf Дистрибутивен закон}
    \[p\ \wedge\ (q \vee r) \equiv (p\ \wedge q)\vee (p\ \wedge\ r)\]
    \[p\vee (q\ \wedge\ r) \equiv (p\vee q)\ \wedge\ (p\vee r)\]
  \item
    {\bf Закони на де Морган}
    \[\neg(p \wedge q) \equiv (\neg p \vee \neg q)\]
    \[\neg(p\vee q) \equiv (\neg p \wedge \neg q)\]
  \item
    {\bf Закон за контрапозицията}
    \[p\rightarrow q \equiv \neg q \rightarrow \neg p\]
  \item
    {\bf Обобщен закон за контрапозицията}
    \[(p \wedge q)\rightarrow r \equiv (p \wedge \neg r) \rightarrow \neg q\]
  \item
    {\bf Закон за изключеното трето}
    \[p\vee \neg p \equiv {\mathbf 1}\]
  \item
    {\bf Закон за силогизма (транзитивност)}
    \[ ((p\rightarrow q)\ \wedge\ (q\rightarrow r)) \rightarrow (p\rightarrow r) \equiv {\mathbf 1}\]
\end{enumerate}

Лесно се проверява с таблиците за истинност, че законите са валидни.

\subsection{Нормални форми}

\begin{itemize}
\item
  Конюнктивна нормална форма
\item
  Дизюнктивна нормална форма
\end{itemize}


%%% Local Variables:
%%% mode: latex
%%% TeX-master: "../eai"
%%% End:


\section{Предикати и квантори}

\subsection*{Квантори}

Свойствата или отношенията на елементите в произволно множество се наричат {\bf предикати}.
Нека да разгледаме един едноместен предикат $P(\cdot)$.

\bigskip
\begin{tabular}{|l|p{4.2cm}|p{4.5cm}|}
  \hline
  твърдение & Кога е истина? & Кога е неистина?\\
  \hline
  $\forall x P(x)$ & $P(x)$ е вярно за всяко $x$ & съществува $x$, за което $P(x)$ {\bf не} е вярно \\
  \hline
  $\exists x P(x)$ & съществува $x$, за което $P(x)$ е вярно & $P(x)$ {\bf не} е вярно за всяко $x$\\
  \hline
\end{tabular}  
\bigskip

\begin{enumerate}[(I)]
\item 
  {\bf Квантор за общност} $\forall x$.
  Записът $(\forall x \in A) P(x)$ означава, че за всеки елемент $a$ в $A$, 
  твърдението $P(a)$ има стойност истина.
  Например, $(\forall x\in\Real)[(x+1)^2 = x^2+2x+1]$.
\item
  {\bf Квантор за съществуване} $\exists x$.
  Записът $(\exists x \in A) P(x)$ означава, че съществува елемент $a$ в $A$, 
  за който твърдението $P(a)$ има стойност истина.
  Например, $(\exists x \in\mathbb{C})[x^2 = -1]$, но $(\forall x\in\Real)[x^2 \neq -1]$.
\end{enumerate}

% \begin{example}
%   \begin{itemize}
%   \item
%     За всяко естествено число, съществува по-голямо от него:
%     \[(\forall x\in\Nat)(\exists z\in\Nat)[x < z].\]
%   \item
%     Съществува естествено число, от което няма по-малко:
%     \[(\exists x\in\Nat)(\forall y\in\Nat)[x < y \vee x = y].\]
%     Нека да означим с $Zero(x)$ предиката, който казва, че $x$ е най-малкото число, т.е.
%     \[Zero(x) \equiv (\forall y)[x < y \vee x =y].\]
%   \item
%     Нека $S(x,y)$ да бъде предиката, който казва, че $y = x+1$ в естествените числа:
%     \[S(x,y) \equiv (x < y\ \wedge\ (\forall z\in\Nat)[x < z\ \rightarrow (z = y\ \vee\ y < z)].\]
%   \item
%     $One(x)$ - $x$ е числото $1$:
%     \[One(x) \equiv (\exists y)[Z(y)\ \wedge\ S(y,x)].\]
%   \item
%     $Div(x,y)$ - $x$ се дели на $y$:
%     \[Div(x,y) \equiv (\exists z)[x = y.z].\]
%   \item
%     $Prime(x)$ - $x$ е просто число:
%     \[Prime(x) \equiv x \geq 2\ \wedge\ (\forall y\in\Nat)[\neg (O(y)\ \wedge Z(y))\ \rightarrow\ \neg Div(x,y)].\]
%   \end{itemize}
% \end{example}


\subsection*{Закони на предикатното смятане}

\begin{enumerate}[(I)]
  \item
    $\neg\forall x P(x) \iff \exists x \neg P(x)$
  \item
    $\neg\exists x P(x) \iff \forall x \neg P(x)$
  \item
    $\forall x P(x) \iff \neg\exists x \neg P(x)$
  \item
    $\exists x P(x) \iff \neg\forall x \neg P(x)$
  \item
    $\forall x \forall y P(x) \iff \forall y\forall x P(x)$
  \item
    $\exists x\exists y P(x,y) \iff \exists y \exists x P(x)$  
  \item
    $\exists x\forall y P(x,y) \rightarrow \forall y \exists x P(x,y)$
\end{enumerate}

\bigskip
\begin{tabular}{|l|p{2.5cm}|p{3.2cm}|p{3cm}|}
  \hline
  \multicolumn{4}{|c|}{{\bf Закони на Де Морган за квантори}}\\
  \hline
  твърдение & Еквивалентно твърдение & Кога е истина? & Кога е неистина?\\
  \hline
  $\neg \exists x P(x)$ & $\forall x \neg P(x)$ & за всяко $x$ $P(x)$ {\bf не} е вярно & съществува $x$, за което $P(x)$ е вярно \\
  \hline
  $\neg \forall x P(x)$ & $\exists x \neg P(x)$ & съществува $x$, за което $P(x)$ {\bf не} е вярно & $P(x)$ е вярно за всяко $x$\\
  \hline
\end{tabular}  
\bigskip

\begin{problem}
  Да означим с $K(x,y)$ твърдението ``$x$ познава $y$''.
  Изразете като предикатна формула следните твърдения.
  \begin{enumerate}[1)]
  \item
    \marginpar{$\forall x \exists y K(x,y)$}
    Всеки познава някого.
  \item
    \marginpar{$\exists x \forall y K(x,y)$}
    Някой познава всеки.
  \item
    \marginpar{$\exists x\forall y K(y,x)$}
    Някой е познаван от всички.
  \item
    \marginpar{$\forall x \exists y(K(x,y)\wedge \neg K(y,x)) $}
    Всеки знае някой, който не го познава.
  \item
    \marginpar{$\exists x \forall y(K(y,x)\ \rightarrow K(x,y))$}
    Има такъв, който знае всеки, който го познава.
  \item
    \marginpar{$(\forall x,y)(K(x,y)\ \&\ K(y,x) \to \exists z(K(x,z)\ \&\ K(y,z))$}
    Всеки двама познати имат общ познат.
  \end{enumerate}
\end{problem}

\begin{example}
  Нека $D \subseteq \Real$.
  Казваме, че $f:D \to \Real$ е {\em непрекъсната} в точката $x_0 \in D$, ако 
  \[(\forall \varepsilon > 0)(\exists \delta >0)(\forall x\in D)(\ |x_0 - x| < \delta\ \to\ |f(x_0) - f(x)| < \varepsilon\ ).\]
  Да видим какво означава $f$ да бъде {\em прекъсната} в точката $x_0 \in D$:
  \marginpar{$f$ е прекъсната в $x_0$ точно тогава, когато $f$ не е непрекъсната в $x_0$}
  \begin{align*}
    & \neg (\forall \varepsilon > 0)(\exists \delta >0)(\forall x\in D)(\ |x_0 - x| < \delta\ \to\ |f(x_0) - f(x)| < \varepsilon\ ) \equiv \\
    & (\exists \varepsilon > 0) \neg (\exists \delta >0)(\forall x\in D)(\ |x_0 - x| < \delta\ \to\ |f(x_0) - f(x)| < \varepsilon\ ) \equiv \\
    & (\exists \varepsilon > 0)(\forall \delta >0)\neg(\forall x\in D)(\ |x_0 - x| < \delta\ \to\ |f(x_0) - f(x)| < \varepsilon\ ) \equiv \\
    & (\exists \varepsilon > 0)(\forall \delta >0)(\exists x\in D)\neg(\ |x_0 - x| < \delta\ \to\ |f(x_0) - f(x)| < \varepsilon\ ) \equiv \\
    & (\exists \varepsilon > 0)(\forall \delta >0)(\exists x\in D)\neg(\ \neg (|x_0 - x| <\delta) \vee |f(x_0) - f(x)| < \varepsilon\ ) \equiv \\
    & (\exists \varepsilon > 0)(\forall \delta >0)(\exists x\in D)(\ \neg\neg (|x_0 - x| <\delta) \land \neg (|f(x_0) - f(x)| < \varepsilon)\ ) \equiv \\
    & (\exists \varepsilon > 0)(\forall \delta >0)(\exists x\in D)(\ |x_0 - x| < \delta\ \land\ |f(x_0) - f(x)| \geq \varepsilon\ ).
  \end{align*}
\end{example}


%%% Local Variables:
%%% mode: latex
%%% TeX-master: "../eai"
%%% End:



\section{Множества, релации, функции}

\subsection*{Основни релации между множества}

За произволни множества $A$ и $B$, ще казваме, че:
\begin{itemize}
\item
  $A$ е подмножество на $B$, което ще означаваме като $A \subseteq B$, ако:
  \[(\forall x)[x \in A \implies x \in B].\]
\item
  $A$ е равно на $B$, което ще означаваме като $A = B$, ако:
  \[(\forall x)[x \in A \iff x \in B],\]
  или
  \[A = B\ \iff\ A \subseteq B\ \&\ B \subseteq A.\]
\end{itemize}

\subsection*{Основни операции върху множества}

Ще разгледаме няколко операции върху произволни множества $A$ и $B$.
\begin{itemize}
\item
  \index{множества!сечение}
  \marginpar{На англ. \emph{intersection}}
  {\bf Сечение}
  \[A\cap B = \{x\ \mid\ x\in A\ \wedge\ x\in B\}.\]
  Казано по-формално, $A\cap B$ е множеството, за което е изпълнено, че:
  \[(\forall x)[x \in A\cap B \iff (x\in A\ \land\ x \in B)].\]
  Примери:
  \begin{itemize}
  \item
    $A \cap A = A$, за всяко множество $A$.
  \item
    $A \cap \emptyset = \emptyset$, за всяко множество $A$.
  \item
    \marginpar{Макар и $\emptyset$, $\{\emptyset\}$ и $\{1,2\}$ да са множества, те може да са елементи на други множества.}
    $\{1,\emptyset,\{\emptyset\}\} \cap \{\emptyset\} = \{\emptyset\}$.
  \item
    $\{1,2,\{1,2\}\} \cap \{1,\{1\}\} = \{1\}$.
  \end{itemize}
\item
  \index{множества!обединение}
  \marginpar{На англ. \emph{union}}
  {\bf Обединение}
  \[A\cup B = \{x\ \mid x\in A\ \vee\ x\in B\}.\]
  $A\cup B$ е множеството, за което е изпълнено, че:
  \[(\forall x)[x \in A\cup B \iff (x\in A\ \lor\ x \in B)].\]
  Примери:
  \begin{itemize}
  \item
    $A \cup A = A$, за всяко множество $A$.
  \item 
    $A \cup \emptyset = A$, за всяко множество $A$.
  \item
    $\{1,2,\emptyset\} \cup \{1,2,\{\emptyset\}\} = \{1,2,\emptyset,\{\emptyset\}\}$.
  \item
    $\{1,2,\{1,2\}\} \cup \{1,\{1\}\} = \{1,2,\{1\},\{1,2\}\}$.
  \end{itemize}
\item
  \index{множества!разлика}
  {\bf Разлика}
  \[A\setminus B = \{x\ \mid\ x\in A\ \wedge\ x\not\in B\}.\]
  $A\setminus B$ е множеството, за което е изпълнено, че:
  \[(\forall x)[x \in A\setminus B \iff (x\in A\ \wedge\ x \not\in B)].\]
  Примери:
  \begin{itemize}
  \item
    $A \setminus A = \emptyset$, за всяко множество $A$.
  \item 
    $A \setminus \emptyset = A$, за всяко множество $A$.
  \item 
    $\emptyset \setminus A = \emptyset$, за всяко множество $A$.
  \item
    $\{1,2,\emptyset\} \setminus \{1,2,\{\emptyset\}\} = \{\emptyset\}$.
  \item
    $\{1,2,\{1,2\}\} \setminus \{1,\{1\}\} = \{2,\{1,2\}\}$.
  \end{itemize}
\item
  \index{множества!симетрична разлика}
  {\bf Симетрична разлика}
  \[A\triangle B = (A\backslash B)\cup (B\backslash A).\]
  $A\triangle B$ е множеството, за което е изпълнено, че:
  \[(\forall x)[x \in A\triangle B \iff [(x\in A\ \land\ x \not\in B) \vee (x \in B\ \land\ x\not\in A)]].\]
  Примери:
  \begin{itemize}
  \item 
    $A \triangle \emptyset = A$, за всяко множество $A$.
  \item
    $A \triangle A = \emptyset$, за всяко множество $A$.
  \item
    $A\triangle B = B \triangle A$, за всеки две множества $A$ и $B$.
  \item
    $\{1,2,\emptyset\} \triangle \{1,2,\{\emptyset\}\} = \{\emptyset\} \cup \{\{\emptyset\}\} = \{\emptyset,\{\emptyset\}\}$.
  \item
    $\{1,2,\{1,2\}\} \triangle \{1,\{1\}\} = \{2,\{1,2\}\} \cup \{\{1\}\} = \{2,\{1\},\{1,2\}\}$.
  \end{itemize}
\item
  \index{множества!степенно множество}
  {\bf Степенно множество}
  \[\Ps(A) = \{x\mid x\subseteq A\}.\]
  \marginpar{На англ. \emph{power set} }
  $\Ps(A)$ е множеството, за което е изпълнено, че:
  \[(\forall x)[x \in \Ps(A) \iff (\forall y)[y\in x\rightarrow y \in A]].\]
  \marginpar{В литературата се среща също така и означението $2^A$ за степенното множество на $A$.}
  Примери:
  \begin{itemize}
  \item 
    $\Ps(\emptyset) = \{\emptyset\}$.
  \item
    $\Ps(\{\emptyset\}) = \{\emptyset,\{\emptyset\}\}$.
  \item
    $\Ps(\{\emptyset,\{\emptyset\}\}) = \{\emptyset,\{\emptyset\},\{\{\emptyset\}\},\{\emptyset,\{\emptyset\}\}\}$.
  \item
    $\Ps(\{1,2\}) = \{\emptyset,\{1\},\{2\},\{1,2\}\}$.
  \end{itemize}
\end{itemize}
Нека имаме редица от множества $\{A_1,A_2,\dots,A_n\}$.
Тогава имаме следните операции:
\begin{itemize}
\item
  {\bf Обединение на редица от множества}
  \[\bigcup^{n}_{i=1} A_i = \{x \mid \exists i (1\leq i\leq n\ \&\ x\in A_i)\}.\]
  % \[(\forall x)[x \in \bigcup^n_{i=1}A_i \iff (\exists i)[1 \leq i \leq n\ \wedge\ x \in A_i]].\]
\item
  {\bf Сечение на редица от множества}
  \[\bigcap^{n}_{i=1} A_i = \{x \mid \forall i (1\leq i\leq n \rightarrow x\in A_i)\}.\]
  % \[(\forall x)[x \in \bigcap^n_{i=1}A_i \iff (\forall i)[1 \leq i \leq n\ \rightarrow\ x \in A_i]].\]
\end{itemize}

% \begin{example}
%   Нека $A = \{x\in\Nat\mid x > 1\}$ и $B = \{x\in\Nat\mid x>3\}$. Тогава :
%   \begin{itemize}
%     \item
%       $A\cap B = \{x\in\Nat\mid x > 3\}$,
%     \item
%       $A\cup B = \{x\in\Nat\mid x > 1\}$,
%     \item
%       $A\setminus B = \{x\in\Nat\mid 1<x\leq 3\}$,
%     \item
%       $B\setminus A = \emptyset$,
%     % \item
%     %   $A\triangle B = \{x\in\Nat\mid 1<x\leq 3\}$
%     \end{itemize}
% \end{example}


\begin{problem}
  Проверете верни ли са свойствата:
  \begin{enumerate}[a)]
  \item
    $A\subseteq B \iff A\setminus B = \emptyset \iff A\cup B = B \iff A\cap B = A$;
  \item
    $A\setminus \emptyset = A$, $\emptyset\setminus A=\emptyset$, $A\setminus B = B\setminus A$.
  \item
    $A\cap (B\cup A) = A \cap B$;
  \item
    $A\cup(B\cap C) = (A\cup B)\cap(A\cup C)$ и $A \cap (B \cup C) = (A \cup B) \cap (A \cup C)$;
  % \item
  %   $C\subseteq A\ \&\ C\subseteq B \rightarrow C\subseteq A\cap B$;
  % \item
  %   $A\subseteq C\ \&\ B\subseteq C \rightarrow A\cup B\subseteq C$;
  \item
    $A\backslash B = A \iff A\cap B = \emptyset$;
  \item
    $A\backslash B = A\backslash (A\cap B)$ и $A\backslash B = A\backslash (A\cup B)$;
  \item
    $(A\cup B)\setminus C = (A\setminus C) \cup (B\setminus C)$;
  % \item
  %   \marginpar{Не е вярно!}
  %   $A\setminus (B\setminus C) = (A\setminus B)\setminus C$;
  \item
    \marginpar{Закони на Де Морган}
    $C\setminus (A\cup B) = (C\backslash A)\cap(C\backslash B)$ и $C \backslash (A\cap B) = (C\backslash A)\cup(C\backslash B)$
  \item
    $C\backslash(\bigcup^{n}_{i=1} A_i) = \bigcap^{n}_{i=1} (C\backslash A_i)$ и $C \backslash(\bigcap^{n}_{i=1} A_i) = \bigcup^{n}_{i=1} (C\backslash A_i)$;
  \item
    $(A\backslash B)\backslash C = (A\backslash C)\backslash(B \backslash C)$ и $A\backslash (B\backslash C) = (A\backslash B) \cup (A\cap C)$;
  \item
    $A\subseteq B \Rightarrow \Ps(A) \subseteq \Ps(B)$;
  \item
    \marginpar{$X \subseteq A\cup B \stackrel{?}{\Rightarrow} X\subseteq A \vee X \subseteq B$}
    $\Ps(A\cap B) = \Ps(A) \cap \Ps(B)$ и $\Ps(A\cup B) = \Ps(A) \cup \Ps(B)$;
  \end{enumerate}
\end{problem}

За да дадем определение на понятието релация, трябва първо 
да въведем понятието декартово произведение на множества,
което пък от своя страна се основава на понятието наредена двойка.

\subsection*{Наредена двойка}
\index{наредена двойка}
За два елемента $a$ и $b$ въвеждаме опрецията {\bf наредена двойка} $\pair{a,b}$.
Наредената двойка $\pair{a,b}$ има следното характеристичното свойство:
\[a_1 = a_2\ \wedge\ b_1 = b_2\ \iff\ \pair{a_1,b_1} = \pair{a_2,b_2}.\]
Понятието наредена двойка може да се дефинира по много начини, стига да изпълнява харектеристичното свойство.
Ето примери как това може да стане:
\begin{enumerate}[1)]
\item
  \marginpar{Norbert Wiener (1914)}
  Първото теоретико-множествено определение на понятието наредена двойка е
  дадено от Норберт Винер:
  \[\pair{a,b} \df \{\{\{a\},\emptyset\},\{\{b\}\}\}.\]
\item
  \marginpar{Kazimierz Kuratowski (1921)}
  Определението на Куратовски се приема за ,,стандартно'' в наши дни:
  \[\pair{a,b} \df \{\{a\},\{a,b\}\}.\]
\end{enumerate}

\begin{problem}
  Докажете, че горните дефиниции наистина изпълняват харектеристичното свойство за наредени двойки.
\end{problem}

\begin{dfn}
  \marginpar{Пример за индуктивна (рекурсивна) дефиниция}
  Сега можем, за всяко естествено число $n \geq 1$,
  да въведем понятието наредена $n$-орка $\pair{a_1,\dots,a_n}$:
  \begin{align*}
    & \pair{a_1} \df a_1,\\
    & \pair{a_1,a_2,\dots,a_n} \df \pair{a_1,\pair{a_2,\dots,a_n}}
  \end{align*}
\end{dfn}

Оттук нататък ще считаме, че имаме дадено понятието наредена $n$-орка, без да се интересуваме от нейната формална дефиниция.
 
\subsection*{Декартово произведение}
\marginpar{На англ. cartesian product}
\index{декартово произведение}
\marginpar{Считаме, че $(A\times B)\times C = A\times (B\times C) = A\times B \times C$}

За две множества $A$ и $B$, определяме тяхното декартово произведение като
\[A\times B = \{\pair{a,b}\mid a\in A\ \&\ b\in B\}.\]
За краен брой множества $A_1,A_2,\dots,A_n$, определяме
\[A_1\times A_2\times\cdots\times A_n = \{\pair{a_1,a_2,\dots,a_n}\mid a_1 \in A_1\ \&\ \dots\ \&\ a_n \in A_n\}.\]

\begin{problem}
  Проверете, че:
  \begin{enumerate}[a)]
  \item
    $A\times(B\cup C) = (A\times B) \cup (A\times C)$.
  \item
    $(A\cup B)\times C = (A\times C)\cup (B\times C)$.
  \item 
    $A\times(B\cap C) = (A\times B) \cap (A\times C)$.
  \item
    $(A \cap B)\times C = (A \times C)\cap(B\times C)$.
  \item 
    $A\times(B\setminus C) = (A\times B) \setminus (A\times C)$.
  \item
    $(A\setminus B)\times C = (A\times C)\setminus (B\times C)$.
  \end{enumerate}
\end{problem}


\subsection*{Видове функции}

Функцията $f:A \to B$ е:
\begin{itemize}
\item
  \marginpar{\comment{или $f$ е {\bf обратима}}}
  {\bf инекция}\index{функция!инекция}, ако 
  \[(\forall a_1,a_2\in A)[\ a_1\neq a_2\ \to\ f(a_1)\neq f(a_2)\ ],\]
  или еквивалентно,
  \[(\forall a_1,a_2\in A)[\ f(a_1) = f(a_2)\ \to\ a_1 = a_2\ ].\]
\item
  \marginpar{\comment{или $f$ е {\bf върху} $B$ }}
  {\bf сюрекция}\index{функция!сюрекция}, ако 
  \[(\forall b\in B)(\exists a\in A)[\ f(a) = b\ ].\]
\item
  {\bf биекция}\index{функция!биекция}, ако е инекция и сюрекция.
\end{itemize}

\begin{problem}
  \marginpar{Канторово кодиране. Най-добре се вижда като се нарисува таблица}
  Докажете, че $f: \Nat \times \Nat\rightarrow \Nat$ е биекция, където
  \[f(x, y) = \frac{(x+y)(x+y+1)}{2} + x.\]
\end{problem}


%%% Local Variables:
%%% mode: latex
%%% TeX-master: "../eai"
%%% End:


\section{Доказателства на твърдения}

\subsection*{Допускане на противното}

Да приемем, че искаме да докажем, че свойството $P(x)$
е вярно за всяко естествено число.
Един начин да направим това е следният:
\begin{itemize}
\item 
  Допускаме, че съществува елемент $n$, за който $\neg P(n)$.
\item
  Използвайки, че $\neg P(n)$ правим извод, от който следва факт, за който знаем, че винаги е лъжа.
  Това означава, че доказваме следното твърдение
  \[\exists x \neg P(x) \rightarrow \mathbf{0}.\]
\item
  Тогава можем да заключим, че $\forall x P(x)$, защото имаме следния извод:
  \begin{prooftree}
    \AxiomC{$\exists x \neg P(x) \rightarrow \mathbf{0}$}
    \UnaryInfC{$\mathbf{1} \rightarrow \neg \exists x \neg P(x)$}
    \UnaryInfC{$\neg \exists x \neg P(x)$}
    \UnaryInfC{$\forall x P(x)$}
  \end{prooftree}
\end{itemize}

Ще илюстрираме този метод като решим няколко прости задачи.

\begin{problem}
  \label{prob:even-number-square}
  За всяко $a \in \Int$, ако $a^2$ е четно, то $a$ е четно.
\end{problem}
\begin{proof}
  Ние искаме да докажем твърдението $P$, където:
  \[P \equiv (\forall a\in\Int)[a^2\mbox{ е четно}\ \rightarrow\ a\mbox{ е четно}].\]
  \mynote{$\neg (\forall x)(A(x) \rightarrow B(x))$ е еквивалентно на $(\exists x)(A(x) \wedge \neg B(x))$}
  Да допуснем противното, т.е. изпълнено е $\neg P$. Лесно се вижда, че
  \[\neg P \iff (\exists a\in\Int)[a^2\mbox{ е четно}\ \land\ a\mbox{ не е четно}].\]
  Да вземем едно такова нечетно $a$, за което $a^2$ е четно.
  Това означава, че $a = 2k+1$, за някое $k \in \Int$,
  и \[a^2 = (2k+1)^2 = 4k^2 + 4k + 1,\]
  което очевидно е нечетно число.
  Но ние допуснахме, че $a^2$ е четно.
  Така достигаме до противоречие, следователно нашето допускане е грешно 
  и 
  \[(\forall a\in\Int)[a^2\mbox{ е четно}\ \rightarrow\ a\mbox{ е четно}].\]
\end{proof}

\begin{problem}
  Докажете, $\sqrt{2}$ {\bf не} е рационално число.
\end{problem}
\begin{proof}
  Да допуснем, че $\sqrt{2}$ е рационално число. Тогава  съществуват $a,b \in \Int$, такива че
  \[\sqrt{2} = \frac{a}{b}.\]
  Без ограничение, можем да приемем, че $a$ и $b$ са естествени числа,
  които нямат общи делители, т.е. не можем да съкратим дробта $\frac{a}{b}$.
  Получаваме, че \[2b^2 = a^2.\]
  Тогава $a^2$ е четно число и от Задача \ref{prob:even-number-square}, $a$ е четно число.
  Нека $a = 2k$, за някое естествено число $k$. Получаваме, че
  \[2b^2 = 4k^2,\]
  от което следва, че
  \[b^2 = 2k^2.\]
  Това означава, че $b$ също е четно число, $b = 2n$, за някое естествено число $n$.
  Следователно, $a$ и $b$ са четни числа и имат общ делител $2$,
  което е противоречие с нашето допускане, че $a$ и $b$ нямат общи делители.
  Така достигаме до противоречие.
  Накрая заключаваме, че $\sqrt{2}$ не е рационално число.
\end{proof}


\subsection*{Индукция върху естествените числа}
\index{индукция}

\mynote{Да напомним, че естествените числа са $\Nat = \{0,1,2,\dots\}$}
Доказателството с индукция по $\Nat$ представлява следната схема:
\begin{prooftree}
  \AxiomC{$P(0)$}
  \AxiomC{$(\forall x\in\Nat)[P(x)\rightarrow P(x+1)]$}
  \BinaryInfC{$(\forall x\in\Nat) P(x)$}
\end{prooftree}

Това означава, че ако искаме да докажем, че свойството $P(x)$ е вярно за всяко естествено число $x$,
то трябва да докажем първо, че е изпълнено $P(0)$ и след това, за произволно естествено число $x$, ако $P(x)$ вярно, то също така е вярно $P(x+1)$.

\begin{problem}
  \label{prob:number-prod-prime}  
  Всяко естествено число $n \geq 2$ може да се запише като произведение на прости числа.
\end{problem}
\begin{proof}
  Искаме да докажем, че $(\forall n \geq 2)P(n)$, където $P(n)$ казва, че $n$ може да се запише като произведение на прости числа, т.е.
  \[n = p^{m_1}_1p^{m_2}_2\cdots p^{m_k}_k,\]
  за някои прости числа $p_1,p_2,\dots,p_k$ и естествени числа $m_1,m_2,\dots,m_k$.
  
  Доказателството протича с индукция по $n \geq 2$.
  \begin{enumerate}[a)]
  \item 
    За $n = 2$ е ясно, защото $2$ е просто число. В този случай $n = p^{m_1}_1$ и $p_1 = 2$ и $m_1 = 1$.
  \item
    Да приемем, че $P(n)$ е изпълнено за някое естествено число $n > 2$.
  \item
    Да разгледаме следващото естествено число $n+1$.
    Ако $n+1$ е просто число, то всичко е ясно.
    Ако $n+1$ е съставно, то съществуват естествени числа $n_1,n_2 \geq 2$, за които
    \[n + 1 = n_1\cdot n_2.\]
    Тогава, понеже $n_1,n_2 \leq n$, от от И.П. следва, че $P(n_1)$ и $P(n_2)$, т.е.
    \[n_1 = p^{\ell_1}_1\cdots p^{\ell_k}_k\text{ и }n_2 = q^{m_1}_1\cdots q^{m_r}_r,\]
    където $p_1,\dots,p_k$ и $q_1,\dots,q_r$ са прости числа, а $\ell_1,\dots,\ell_k$ и $m_1,\dots,m_r$ са естествени числа.
    Тогава е ясно, че $n+1$ също е произведение на прости числа.
  \end{enumerate}
\end{proof}

\begin{problem}
  Докажете, че за всяко естествено число $n$, 
  \[\sum^n_{i=0} 2^i = 2^{n+1} - 1.\]
\end{problem}
\begin{proof}
  Да разгледаме свойството
  \[P(n) \df \sum^n_{i=0} 2^i = 2^{n+1} - 1.\]
  Ще докажем с индукция по $n$, че $(\forall n)P(n)$, т.е. ще докажем следния извод:
  \begin{prooftree}
    \AxiomC{$P(0)$}
    \AxiomC{$(\forall n)[P(n) \implies P(n+1)]$}
    \BinaryInfC{$(\forall n)P(n)$}
  \end{prooftree}
  \begin{itemize}
  \item
    \mynote{Това е базата на индукцията.}
    Нека първо $n = 0$. Oчевидно е, че $P(0)$ е изпълнено, защото
    \[\sum^0_{i=0}2^i = 1 = 2^{1} - 1.\]
  \item
    \mynote{$P(n)$ се нарича индукционно предположение, а $P(n+1)$ се нарича индукционна стъпка.}
    Да разгледаме сега произволно естествено число $n$, като
    приемем, че свойството $P(n)$ е изпълнено.
    Ще докажем, че $P(n+1)$ също е изпълнено.
    Но това е лесно защото имаме следната верига от равенства:
    \begin{align*}
      \sum^{n+1}_{i=0} 2^i & = \sum^{n}_{i=0}2^i + 2^{n+1}\\
                           & = 2^{n+1} - 1 + 2^{n+1} & \comment\text{защото $P(n)$ е изпълнено}\\
                           & = 2.2^{n+1} - 1 \\
                           & = 2^{1+(n+1)} - 1\\
                           & = 2^{n+2} - 1.
    \end{align*}
  \end{itemize}
\end{proof}

%\subsection*{Пълна индукция върху естествените числа}


%%% Local Variables:
%%% mode: latex
%%% TeX-master: "../eai"
%%% End:


\section{Азбуки, думи, езици}

\epigraph{Language is an app for converting a web of thoughts into a string of words. [Steven Pinker]}

\subsubsection*{Основни понятия}

\begin{itemize}
\item 
  \index{азбука}
  \index{буква}
  {\bf Азбука} ще наричаме всяко крайно множество,
  като обикновено ще я означаваме със $\Sigma$.
  \mynote{Често ще използваме буквите $a$, $b$, $c$ за да означаваме букви.}
  Елементите на азбуката $\Sigma$ ще наричаме {\bf букви}.
\item
  \index{дума}
  {\bf Дума} над азбуката $\Sigma$ е произволна крайна редица от елементи на $\Sigma$.
  Например, за $\Sigma = \{a,b\}$, $aababba$ е дума над $\Sigma$ с дължина $7$.
  С $\abs{\alpha}$ ще означаваме дължината на думата $\alpha$.
  \mynote{Обикновено ще означаваме думите с $\alpha$, $\beta$, $\gamma$, $\omega$.}
\item
  \index{дума!празна}
  Обърнете внимание, че имаме единствена дума с дължина $0$.
  Тази дума ще означаваме с $\varepsilon$ и ще я наричаме {\bf празната дума},
  т.е. $\abs{\varepsilon} = 0$.
\item
  С $a^n$ ще означаваме думата съставена от $n$ $a$-та.
  Формалната индуктивна дефиниция е следната:
  \begin{align*}
    a^0 & \df \varepsilon,\\
    a^{n+1} & \df a^na.
  \end{align*}
\item
  Множеството от всички думи над азбуката $\Sigma$ ще означаваме със $\Sigma^\star$.
  Например, за $\Sigma = \{a,b\}$,
  \[\Sigma^\star = \{\varepsilon,a,b,aa,ab,ba,bb,aaa,aab,\dots\}.\]
  Обърнете внимание, че $\emptyset^\star = \{\varepsilon\}$.
\item
  \index{език}
  {\bf Език} над азбуката $\Sigma$ ще наричаме всяко подмножество на $\Sigma^\star$.
  Например, за азбуката $\Sigma = \{a, b\}$, множоеството от думи $L = \{\alpha \in \{a, b\}^\star \mid \alpha\mbox{ започва с }a\}$ 
  е пример за език над $\Sigma$.
\end{itemize}

\subsection*{Операции върху думи}

\begin{itemize}
\item
  \index{дума!конкатенация}
  Операцията {\bf конкатенация} взима две думи $\alpha$ и $\beta$ и образува 
  новата дума $\alpha\cdot\beta$ като слепва двете думи.
  Например $aba\cdot bb = ababb$.
  Обърнете внимание, че в общия 
  случай $\alpha\cdot\beta \neq \beta\cdot\alpha$. 
  \mynote{Често ще пишем $\alpha\beta$ вместо $\alpha\cdot\beta$}
  Можем да дадем формална индуктивна дефиниция на операцията конкатенация по
  дължината на думата $\beta$.
  \begin{itemize}
  \item 
    Ако $\abs{\beta} = 0$, то $\beta = \varepsilon$.
    Тогава $\alpha\cdot \varepsilon \df \alpha$.
  \item
    Ако $\abs{\beta} = n+1$, то $\beta = \gamma b$, $\abs{\gamma} = n$.
    Тогава $\alpha\cdot\beta \df (\alpha\cdot\gamma)b$.
  \end{itemize}
\item
  \index{дума!обръщане}
  Друга често срещана операция върху думи е {\bf обръщането} на дума.
  Дефинираме думата $\alpha^{\rev}$ като обръщането на $\alpha$ по следния начин.
  \mynote{Например, $reverse^{\rev} = esrever$}
  \begin{itemize}
  \item 
    Ако $\abs{\alpha} = 0$, то $\alpha = \varepsilon$ и $\alpha^{\rev} \df \varepsilon$.
  \item
    Ако $\abs{\alpha} = n+1$, то $\alpha = a\beta$, където $\abs{\beta} = n$. Тогава
    \[\alpha^{\rev} \df (\beta^{\rev})a.\]
  \end{itemize}

\item
  \mynote{Обърнете внимание, че:
    \begin{align*}
      & \emptyset\cdot A = A\cdot\emptyset = \emptyset\\
      & \{\varepsilon\}\cdot A = A\cdot\{\varepsilon\} = A.
    \end{align*}}
  Дефинираме конкатенацията на езиците $A$ и $B$ като
  \[A\cdot B \df \{\alpha\cdot\beta \mid \alpha\in A\ \&\ \beta \in B\}.\]
\item
  Сега за един език $A$, дефинираме $A^n$ индуктивно:
  \begin{align*}
    A^0 & \df \{\varepsilon\},\\
    A^{n+1} & \df A^n \cdot A.
  \end{align*}
  \begin{itemize}
  \item
    Ако $A = \{ab, ba\}$, то $A^0 = \{\varepsilon\}$, $A^1 = A$, $A^2 = \{abab, abba, baba, baab\}$.
  \item
    Ако $A = \{a,b\}$, то $A^n = \{\alpha \in \{a,b\}^\star \mid \abs{\alpha} = n\}$.
  \end{itemize}
\item
  За един език $A$, дефинираме:
  \mynote{Операцията $\star$ е известна като звезда на Клини.}
  \index{език!звезда на Клини}
  \index{Клини}
  \begin{align*}
    A^{\star} & \df \bigcup_{n\geq 0} A^{n}\\
    & = A^{0} \cup A^{1} \cup A^{2} \cup A^{3} \cup \dots\\
    A^{+} & \df A\cdot A^{\star}.
  \end{align*}
\end{itemize}

\begin{example}
  Нека да разгледаме няколко примера какво точно представлява прилагането
  на операцията звезда на Клини върху един език.
  \begin{itemize}
  \item 
    Нека $L = \{0,11\}$. Тогава:
    \begin{itemize}
    \item 
      $L^0 = \{\varepsilon\}$, $L^1 = L$,
    \item
      $L^2 = L^1\cdot L^1 = \{00,011,110,1111\}$,
    \item
      $L^3 = L^1\cdot L^2 = \{000,0011,0110,01111,1100,11011,11110,111111\}$.
    \end{itemize}
  \item
    Нека $L = \emptyset$. Тогава $L^0 = \{\varepsilon\}$, $L^1 = \emptyset$ и $L^2 = L^1 \cdot L^1 = \emptyset$.
    Получаваме, че $L^\star = \{\varepsilon\}$, т.е. {\em краен} език
  \item
    Нека $L = \{0^i\mid i \in \Nat\} = \{\varepsilon, 0, 00, 000, \dots\}$.
    Лесно се вижда, че $L = L^\star$.
  \end{itemize}
\end{example}

% \subsection*{Премахване на префикс}


\begin{extra}
\begin{problem}
  Проверете:
  \begin{enumerate}[a)]
  \item 
    операцията конкатенация е {\em асоциативна}, т.е. за всеки три думи $\alpha$, $\beta$, $\gamma$,
    \[(\alpha\cdot\beta)\cdot\gamma = \alpha\cdot(\beta\cdot\gamma);\]
  \item
    за произволни езици $A$, $B$ и $C$, 
    \[(A\cdot B)\cdot C = A\cdot (B\cdot C);\]
  \item
    $\{\varepsilon\}^\star = \{\varepsilon\}$;
  \item
    за произволен език $A$,
    \[A^\star = A^\star\cdot A^\star\text{ и }(A^\star)^\star = A^\star;\]
  \item
    за произволни букви $a$ и $b$,
    \[\{a,b\}^\star = \{a\}^\star\cdot(\{b\}\cdot\{a\}^\star)^\star;\]
  \item
    за произволни букви $a$ и $b$,
    \[\{a,b\}^\star = (\{a\}^\star\cdot \{b\}^\star)^\star;\]
  \end{enumerate}
\end{problem}

\begin{problem}
  Докажете, че за всеки две думи $\alpha$ и $\beta$ е изпълено:
  \begin{enumerate}[a)]
  \item 
    $(\alpha\cdot\beta)^{\rev} = \beta^{\rev}\cdot\alpha^{\rev}$;
  \item
    $\alpha$ е префикс на $\beta$ точно тогава, когато $\alpha^{\rev}$ е суфикс на $\beta^{\rev}$;
  \item
    $(\alpha^{\rev})^{\rev} = \alpha$;
  \item
    $(\alpha^n)^{\rev} = (\alpha^{\rev})^n$, за всяко $n \geq 0$.
  \end{enumerate}
\end{problem}
\end{extra}

\subsubsection*{Отрези на дума}\label{sect:intro:slices}
\index{дума!отрез}

% https://stackoverflow.com/questions/509211/understanding-slice-notation
Понякога може да е удобно да вземем назаем от \texttt{python} нотацията за \texttt{slices} на масив и
ако думата $\alpha = a_0 a_1 \cdots a_{n-1}$, то нека 
\begin{align*}
  \alpha\slice{i} & \df a_i\\
  \alpha\slice{i:j} & \df
                      \begin{cases}
                        a_i \cdots a_{m-1}, & \text{ ако }i < j\ \&\ m = \min\{j,|\alpha|\}\\
                        \varepsilon, & \text{ иначе}
                      \end{cases}\\
  \alpha\slice{i:} & \df \alpha\slice{i:|\alpha|}\\
  \alpha\slice{:i} & \df \alpha\slice{0:i}\\
\end{align*}


\subsubsection*{Релации между думи}

\begin{itemize}
\item 
  \index{дума!префикс}
  Казваме, че думата $\alpha$ е {\bf префикс} на думата $\beta$,
  ако съществува дума $\gamma$, такава че $\beta = \alpha\cdot\gamma$.
  Да обърнем внимание, че позволяваме $\gamma = \varepsilon$. Тогава $\beta$ е префикс на самата себе си.
  Ако се ограничим до думи $\gamma \neq \varepsilon$, то ще казваме, че $\alpha$ е \emph{същински} префикс на $\beta$.
\item
  За един език $L$, можем да дефинираме езика от префиксите на думите от $L$, т.е.
  \[\texttt{Pref}(L) = \{\alpha \in \Sigma^\star \mid (\exists \gamma \in \Sigma^\star)[\alpha\cdot \gamma \in L]\}.\]
\item
  \index{дума!суфикс}
  $\alpha$ е {\bf суфикс} на $\beta$, ако $\beta = \gamma\cdot\alpha$, за някоя дума $\gamma$.
\item
  За един език $L$, можем да дефинираме езика от суфиксите на думите от $L$, т.е.
  \[\texttt{Suff}(L) = \{\alpha \in \Sigma^\star \mid (\exists \gamma \in \Sigma^\star)[\gamma\cdot\alpha \in L]\}.\]
\item
  \index{наредба!лексикографска}
  \mynote{Тук не е съществено, че буквите в азбуката $\Sigma$ са числа. Можем да дефинираме наредба между елементите на произволна крайна азбука.}
  Нека имаме азбука $\Sigma = \{0,1,2,\dots,n\}$.
  Казваме, че една дума $\alpha$ е лексикографски по-малка от $\beta$, което ще означаваме като $\alpha <_{\texttt{lex}} \beta$, ако $\alpha$ е префикс на $\beta$ или
  \[(\exists i < \min\{|\alpha|,|\beta|\})[\ \alpha\slice{:i} = \beta\slice{:i}\ \&\ \alpha\slice{i} < \beta\slice{i}\ ].\]
\end{itemize}


%%% Local Variables:
%%% mode: latex
%%% TeX-master: "../eai"
%%% End:


\section*{Бележки}

Повечето книги в тази област започват с уводна глава, в която въвеждат понятията множества, релации и езици.
\begin{itemize}
\item 
  Глава 1 от \cite{rosen}.
\item
  Глава 1 от \cite{papadimitriou}.
\item
  За описанието на думи и азбуки следваме \cite[Глава 2]{kozen}.
\end{itemize}



%%% Local Variables:
%%% mode: latex
%%% TeX-master: "../eai"
%%% End:


% Add Problem section

%%% Local Variables:
%%% mode: latex
%%% TeX-master: "../eai"
%%% End:


\chapter{Регулярни езици и автомати}
\label{ch:regular}

\section{Автоматни езици}

\begin{definition}
  \mynote{Често ще пишем съкратено ДКА вместо детерминиран краен автомат.}
  Детерминиран краен автомат е петорка $\A = \FA$, където
  \begin{itemize}
  \item
    $\Sigma$ е азбука;
  \item
    $Q$ е крайно множество от състояния;
  \item
    $\delta:Q\times\Sigma\to Q$ е тотална функция, която ще наричаме
    \emph{функция на преходите};
  \item
    $\qstart\in Q$ е начално състояние;
  \item
    $F\subseteq Q$ е множеството от финални състояния
  \end{itemize}
\end{definition}
\index{автомат!детерминиран}

Нека имаме една дума $\alpha \in \Sigma^\star$, $\alpha = a_0a_1\cdots a_{n-1}$.
Казваме, че $\alpha$ се {\bf разпознава} от автомата $\A$, ако
съществува редица от състояния $q_0,q_1,q_2,\dots,q_n$, такива че:
\begin{itemize}
\item
  $q_0 = \qstart$, началното състояние на автомата;
\item
  $\delta(q_i,a_{i}) = q_{i+1}$, за всяко $i = 0, \dots, n-1$;
\item
  $q_n \in F$.
\end{itemize}

Казваме, че $\A$ {\bf разпознава} езика $L$, ако $\A$ разпознава точно думите от $L$, т.е.
$L = \{\alpha \in \Sigma^\star \mid \A\mbox{ разпознава }\alpha\}$.
Обикновено означаваме езика, който се разпознава от даден автомат $\A$ с $\L(\A)$.
\index{език!автоматен}
В такъв случай ще казваме, че езикът $L$ е {\bf автоматен}.

При дадена функция на преходите $\delta$,
често е удобно да разглеждаме функция $\delta^\star:Q\times\Sigma^\star \to Q$, която е дефинирана за всяко $q\in Q$ и $\alpha \in \Sigma^\star$ по следния начин:
% \mynote{Това е пример за индуктивна (рекурсивна) дефиниция по дължината на думата $\alpha$}
\begin{itemize}
\item
  Ако $\alpha = \varepsilon$, то $\delta^\star(q,\varepsilon) \df q$;
\item
  Ако $\alpha = \beta $, то $\delta^\star(q,\beta a) \df \delta(\delta^\star(q,\beta), a)$.
\end{itemize}
Лесно се съобразява, че една дума $\alpha$ се {\em разпознава} от автомата $\A$ точно тогава, когато $\delta^\star(\qstart,\alpha) \in F$.
Оттук следва, че
\begin{framed}
\[\L(\A) \df \{\ \alpha\in\Sigma^\star \mid \delta^\star(\qstart,\alpha) \in F\ \}.\]
\end{framed}

\mynote{Обърнете внимание, че $\delta(q,a) = \delta^\star(q,a)$ за $a\in\Sigma$}

\begin{proposition}
  \label{pr:dfa:delta-star}
  Нека $\A$ е ДКА. Тогава за всяко състояние $q$ и произволни думи $\alpha$ и $\beta$ е изпълнено, че
  \[\delta^\star(q,\alpha\beta) = \delta^\star(\delta^\star(q,\alpha),\beta).\]
\end{proposition}
\begin{hint}
  Индукция по дължината на $\beta$.

  \begin{itemize}
  \item
    $|\beta| = 0$, т.е. $\beta = \varepsilon$. Тогава за всяко състояние $q$ и произволна дума $\alpha$ имаме:
    \[\delta^\star(q, \alpha\varepsilon) = \delta^\star( \underbrace{\delta^\star(q, \alpha)}_{q}, \varepsilon).\]
  \item
    Да приемем, че твърдението е изпълнено за думи $\beta$ с дължина $n$.
  \item
    Нека $|\beta| = n+1$, т.е. $\beta = \gamma b$, където $|\gamma| = n$. Тогава за всяко състояние $q$ и произволна дума $\alpha$ имаме:
    \begin{align*}
      \delta^\star(q, \alpha\beta) & = \delta^\star(q, \alpha \gamma b)\\
                                   & = \delta( \delta^\star(q, \alpha\gamma),b) & \comment\text{от деф. на }\delta^\star\\
                                   & = \delta(\delta^\star(\underbrace{\delta^\star(q,\alpha)}_{p}, \gamma), b) & \comment{\text{от И.П. приложено за }\gamma}\\
                                   & = \delta( \delta^\star(p, \gamma), b) \\
                                   & = \delta^\star( p, \gamma b) & \comment\text{от деф. на }\delta^\star\\
                                   & = \delta^\star(\delta^\star(q, \alpha), \beta). & \comment{p = \delta^\star(q,\alpha)}
    \end{align*}
  \end{itemize}
\end{hint}

\begin{remark}
Формално погледнато, доказателството на \Proposition{dfa:delta-star} протича по следния начин:
  \begin{prooftree}
    \AxiomC{$P(0)$}
    \AxiomC{$(\forall n\in\Nat)[P(n) \implies P(n+1)]$}
    \BinaryInfC{$(\forall n \in \Nat)[P(n)]$,}
  \end{prooftree}
  където
  \[P(n) \df (\forall \beta\in\Sigma^n)(\forall \alpha\in\Sigma^\star)(\forall q\in Q)[\delta^\star(q,\alpha\beta) = \delta^\star(\delta^\star(q,\alpha),\beta)].\]
\end{remark}


\index{моментно описание}
\mynote{(На англ. {\em instantaneous description}). В случая въвеждането на това понятие не е напълно необходимо, но по-късно, когато въведем стекови автомати и машини на Тюринг, ще работим с такива моментни описания на изчисления и затова е добре още отначало да свикнем с него.}
{\bf Моментното описание} на изчисление с краен автомат представлява двойка от вида $(q,\alpha) \in Q\times\Sigma^\star$,
т.е. автоматът се намира в състояние $q$, а думата, която остава да се прочете е $\alpha$.
Удобно е да въведем бинарната релация $\vdash_\A$ над $Q\times\Sigma^\star$,
която ще ни казва как моментното описание на автомата $\A$ се променя след изпълнение на една стъпка:
\[(q,b\alpha) \vdash_\A (p,\alpha) \stackrel{\text{деф}}{\iff} \delta(q,b) = p.\]
Рефлексивното и транзитивно затваряне на $\vdash_\A$ ще означаваме с $\vdash^\star_\A$.
За да дадем по-удобна дефиниция на $\vdash^\star_\A$, първо ще дефинираме релацията $\vdash^n_\A$, която
определя работата на автомата $\A$ върху моментни описания за $n$ стъпки.
\mynote{Рефл. и транз. затваряне на една релация е разгледано в Глава \ref{ch:intro}.}

\begin{prooftree}
  \AxiomC{}
  \UnaryInfC{$(q,\alpha) \vdash^0_\A (q,\alpha)$}
\end{prooftree}

\begin{prooftree}
  \AxiomC{$(q,\alpha) \vdash^n_\A (r, b\beta)$}
  \AxiomC{$\delta(r,b) = p$}
  \BinaryInfC{$(q,\alpha) \vdash^{n+1}_\A (p,\beta)$}
\end{prooftree}
% \begin{itemize}
% \item 
%   $(q,\alpha) \vdash^0_\A (q,\alpha)$, защото за $0$ стъпки се случва нищо.
% \item
%   Нека $\delta(q,x) = q'$ и $(q',\alpha) \vdash^n_\A (p, \beta)$. Тогава
%   $(q,x\alpha) \vdash^{n+1}_\A (p,\beta)$, защото за $n+1$ стъпки първо правим една стъпка 
%   и отиваме в моментното описание $(q',\alpha)$ и след това правим още $n$ стъпки.
% \end{itemize}
Сега можем да дефинираме $\vdash^\star_\A$ като:
\[(q,\alpha) \vdash^\star_\A (p,\beta) \dff (\exists n\in\Nat)[(q,\alpha) \vdash^n_\A (p,\beta)].\]

\begin{problem}
  Докажете, че за произволно $\beta \in \Sigma^\star$,
  \[(q,\alpha\beta) \vdash^\star_\A (p, \beta) \iff \delta^\star(q,\alpha) = p.\]  
\end{problem}

Получаваме, че 
\[\L(\A) = \{\alpha\in\Sigma^\star \mid (\exists q \in F)[(\qstart,\alpha) \vdash^\star_\A(q,\varepsilon)]\}.\]


%%% Local Variables:
%%% mode: latex
%%% TeX-master: "../eai"
%%% End:

\section{Регулярни изрази и езици}

Да фиксираме една непразна азбука $\Sigma$.
\index{регулярен израз}
{\bf Регулярните изрази} $\mathbf{r}$ могат да се опишат със следната абстрактна граматика
\[\mathbf{r} ::= \bm{\emptyset}\ |\ \bm{\varepsilon}\ |\ \mathbf{a}\ |\ \bm{r}_1 \cdot \bm{r}_2\ |\ \bm{r}_1 + \bm{r_2}\ |\ \bm{r}^\star_1.\]

Регулярните изрази могат да се опишат и по следния начин:
\marginpar{Това е пример за индуктивна дефиниция}
\begin{itemize}
\item 
  Символите $\bm{\emptyset}$, $\bm{\varepsilon}$ са регулярни изрази;
\item
  за всяка буква $a \in \Sigma$, символът $\bm{a}$ е регулярен израз;
\item
  \marginpar{В литературата също се среща записът $(\bm{r}_1\ |\ \bm{r}_2)$ вместо $(\bm{r}_1 + \bm{r}_2)$}
  ако $\mathbf{r_1}$ и $\mathbf{r_2}$ са регулярни изрази, то думите $(\bm{r}_1 \cdot \bm{r}_2)$, $(\bm{r}_1 + \bm{r}_2)$ и $\bm{r}^\star_1$
  също са регулярни изрази;
\item
  Всеки регулярен израз е получен по някое от горните правила.
\end{itemize}

\index{език!регулярен}
\marginpar{Това е друг пример за индуктивна (рекурсивна) дефиниция.}
Сега ще дефинираме езиците, които се описват с регулярни изрази.
Тези езици се наричат {\bf регулярни}.
Това ще направим следвайки индуктивната дефиниция на регулярните изрази,
т.е. за всеки регулярен израз $\mathbf{r}$ ще определим език $\L(\mathbf{r})$.
\begin{itemize}
\item
  $\emptyset$ е регулярен език,
  който се описва от регулярния израз $\bm{\emptyset}$. Означаваме $\L(\bm{\emptyset}) = \emptyset$;
\item
  $\{\varepsilon\}$ е регулярен език,
  който се описва от регулярния израз $\bm{\varepsilon}$.
  Означаваме $\L(\bm{\varepsilon}) = \{\varepsilon\}$;
\item
  за всяка буква $a \in \Sigma$, $\{a\}$ е регулярен език,
  който се описва от регулярния израз $\mathbf{a}$.
  Означаваме $\L(\mathbf{a}) = \{a\}$;
\item
  Нека $L_1$ и $L_2$ са регулярни езици, т.е. съществуват регулярни изрази $\mathbf{r}_1$
  и $\mathbf{r}_2$, за които $\L(\mathbf{r}_1) = L_1$ и $\L(\mathbf{r}_2) = L_2$.
  Тогава:
  \begin{itemize}
  \item 
    \index{регулярни операции!обединение}
    $L_1 \cup L_2$ е регулярен език, който се описва с регулярния израз $(\mathbf{r}_1 + \mathbf{r}_2)$.
    Това означава, че $\L(\mathbf{r}_1) \cup \L(\mathbf{r}_2) = \L(\mathbf{r}_1 + \mathbf{r}_2)$.
  \item
    \index{регулярни операции!конкатенация}
    \marginpar{Тази операция се нарича конкатенация. Обикновено изпускаме знака $\cdot$}
    $L_1 \cdot L_2$ е регулярен език, който се описва с регулярния израз $(\mathbf{r}_1 \cdot \mathbf{r}_2)$.
    Това означава, че $\L(\mathbf{r}_1) \cdot \L(\mathbf{r}_2) = \L(\mathbf{r}_1 \cdot \mathbf{r}_2)$.
  \item
    \marginpar{Звезда на Клини}
    \index{регулярни операции!звезда на Клини}
    $L^\star_1$ е регулярен език, който се описва с регулярния израз $\mathbf{r}^\star_1$.
    Това означава, че $\L(\mathbf{r}_1)^\star = \L(\mathbf{r}^\star_1)$.
  \end{itemize}
\end{itemize}

\begin{remark}
  Ние знаем, че:
  \begin{itemize}
  \item
    Всеки регулярен израз представлява крайна дума над крайна азбука.
    Това означава, че множеството от всички регулярни изрази е изброимо безкрайно.
    Оттук следва, че всички регулярни езици образуват изброимо безкрайно множество.
  \item 
    Понеже $\Sigma$ е крайна азбука, то $\Sigma^\star$ е изброимо безкрайно множество;
  \item
    Един език над азбуката $\Sigma$ представлява елемент на $\Ps(\Sigma^\star)$.
    Това означава, че всички езици над азбуката $\Sigma$ представляват неизброимо безкрайно множество.
  \end{itemize}
  От всичко това следва, че има езици, които не са регулярни.
  По-нататък ще видим примери за такива езици.
\end{remark}

\begin{example}
  \marginpar{В \cite[стр. 73]{sipser1} е показан алгоритъм, за който по един автомат може да се получи регулярен израз описващ езика на автомата. Ние няма да разглеждаме този алгоритъм. }
  Нека да построим регулярни изрази за всеки от езиците от \Ex{automata-pictures}.
  \begin{enumerate}[a)]
  \item 
    Нека $\mathbf{r} \equiv \mathbf{(a+b)^\star bab(a+b)^\star}$. Тогава
    \[\L(\mathbf{r}) = \{\omega \in \{a,b\}^\star \mid \omega \text{ съдържа } bab\}.\]
  \item
    Нека $\mathbf{r} \equiv \mathbf{b^\star ab^\star a(a+b)^\star}$. Тогава
    \[\L(\mathbf{r}) = \{\omega \in \{a,b\}^\star \mid N_a(\omega) \geq 2\}.\]
  \item
    Нека $\mathbf{r} \equiv \mathbf{(b^\star abb^\star)^\star}$. Тогава
    \[\L(\mathbf{r}) = \{\omega \in \{a,b\}^\star \mid \text{ всяко $a$ в $\omega$ се следва от поне едно $b$}\}.\]
  \item
    Нека $\mathbf{r} \equiv \mathbf{(b^\star ab^\star ab^\star ab^\star)^\star}$. Тогава
    \[\L(\mathbf{r}) = \{\omega \in \{a,b\}^\star \mid N_a(\omega) \equiv 0 \bmod 3\}.\]
  \end{enumerate}
\end{example}

\begin{problem}
  \marginpar{Когато пишем $\bm{r} \equiv \bm{s}$ имаме предвид, че $\L(\bf{r}) = \L(\bm{s})$.}
  За произволни регулярни изрази $\bm{r}$ и $\bm{s}$, проверете дали са изпълнени следните равенства:
  \begin{enumerate}[a)]
  \item 
    $\bm{r + s} \equiv \bm{s + r}$;
  \item
    $\bm{(\varepsilon + r)^\star} \equiv \bm{r^\star}$;
  \item
    $\bm{\emptyset^\star} \equiv \bm{\varepsilon}$;
  \item
    $\bm{(r^\star s^\star)^\star} \equiv \bm{(r+s)^\star}$;
  \item
    $\bm{(r^\star)^\star} \equiv \bm{r^\star}$;
  \item
    $\bm{(rs + r)^\star r} \equiv \bm{r(sr+r)^\star}$;
  \item
    $\bm{s(rs+s)^\star r} \equiv \bm{rr^\star s(rr^\star s)^\star}$;
  \item
    $\bm{(r+s)^\star} \equiv \bm{r^\star + s^\star}$;
  \item
    $\bm{(r+s)^\star s} \equiv \bm{(r^\star s)^\star}$;
  \item
    $\bm{(rs + r)^\star rs} \equiv \bm{(rr^\star s)^\star}$;
  \item
    $\bm{\emptyset^\star} \equiv \bm{\varepsilon^\star}$;
  \end{enumerate}
\end{problem}



%%% Local Variables: 
%%% mode: latex
%%% TeX-master: "../eai"
%%% End: 

\begin{framed}
\begin{thm}[Клини]
  \label{th:regular-kleene}
  \index{Клини}
  Всеки автоматен език се описва с регулярен израз.
\end{thm}
\end{framed}
\begin{proof}
  \marginpar{\cite[стр. 79]{papadimitriou}; \cite[стр. 33]{hopcroft1}}
  Нека  $L = \L(\A)$, за някой краен детерминиран автомат $\A$.
  Да фиксираме едно изброяване на състоянията $Q = \{q_1,\dots,q_n\}$,
  като началното състояние е $q_1$.
  Ще означаваме с $L(i,j,k)$ множеството от тези думи, които
  могат да се разпознаят от автомата по път, който започва от $q_i$,
  завършва в $q_j$, и междинните състояния имат индекси $\leq k$.
  Например, за думата $\alpha = a_1a_2\cdots a_n$ имаме, че $\alpha \in L(i,j,k)$
  точно тогава, когато съществуват състояния $q_{l_1},\dots,q_{l_{n-1}}$, като $l_1,\dots,l_{n-1} \leq k$ и
  \[q_i\stackrel{a_1}{\rightarrow} q_{l_1} \stackrel{a_2}{\rightarrow} q_{l_2} \stackrel{a_3}{\rightarrow} \dots \stackrel{a_{n-1}}{\rightarrow} q_{l_{n-1}}\stackrel{a_n}{\rightarrow} q_j.\]
  Тогава за $n = \abs{Q}$, 
  \[L(i,j,n) = \{\alpha\in\Sigma^\star\mid \delta^\star(q_i,\alpha) = q_j\}.\]
  Така получаваме, че 
  \[\L(\A) = \bigcup\{L(1,j,n)\mid q_j \in F\} = \bigcup_{q_j\in F}L(1,j,n).\]
  Ще докажем с {\em индукция по $k$}, че за всяко $i,j,k$, множествата от думи $L(i,j,k)$
  се описват с регулярен израз $\mathbf{r}^k_{i,j}$
  \begin{enumerate}[a)]
  \item
    Нека $k = 0$. Ще докажем, че за всяко $i,j$, $L(i,j,0)$ се описва с регулярен израз.
    Имаме да разгледаме два случая.
    
    Ако $i = j$, то 
    \begin{equation}
      \label{eq:kleene-equal}
      L(i, j, 0) = \{\varepsilon\}\cup\{a\in\Sigma \mid \delta(q_i,a) = q_j\}.
    \end{equation}
    Ако $i \neq j$, то
    \[L(i, j, 0) = \{a\in\Sigma \mid \delta(q_i, a) = q_j\}.\]
    И в двата случая, понеже $L(i,j,0)$ е краен език, то е ясно, че той се описва с регулярен израз.
  \item
    Да предположим, че $k > 0$, и за всяко $i,j \leq n$, имаме регулярните изрази $\mathbf{r}^{k-1}_{i,j}$, които
    описват езиците $L(i,j,k-1)$, т.е. имаме индукционното предположение, че
    \[(\forall i,j \leq n)[L(i,j,k-1) = \L(\mathbf{r}^{k-1}_{i,j})].\] 
    Ще докажем, че съществуват регулярни изрази $\mathbf{r}^k_{i,j}$, такива че
    \[(\forall i,j \leq n)[L(i,j,k) = \L(\mathbf{r}^{k}_{i,j})].\] 
    Можем да изразим езика $L(i,j,k)$ по следния начин:
    \[L(i,j,k) = \underbrace{L(i,j,k-1)}_{\L(\mathbf{r}^{k-1}_{i,j})}\ \cup\ \underbrace{L(i,k,k-1)}_{\L(\mathbf{r}^{k-1}_{i,k})}\cdot (\underbrace{L(k,k,k-1)}_{\L(\mathbf{r}^{k-1}_{k,k})})^\star \cdot \underbrace{L(k,j,k-1)}_{\L(\mathbf{r}^{k-1}_{k,j})}.\]

    Тогава по {\bf И.П.} следва, че $L(i,j,k)$ може да се опише с регулярния израз
    \begin{equation}
      \label{eq:kleene}
      \mathbf{r}^k_{i,j} = \mathbf{r}^{k-1}_{i,j} + \mathbf{r}^{k-1}_{i,k}\cdot (\mathbf{r}^{k-1}_{k,k})^\star\cdot \mathbf{r}^{k-1}_{k,j}.
    \end{equation}
  \end{enumerate}
  Заключаваме, че за всяко $i,j,k \leq n$, $L(i,j,k)$ може да се опише с регулярен израз $\mathbf{r}^{k}_{i,j}$.
  Тогава ако $F = \{q_{i_1},\dots,q_{i_k}\}$, то $\L(\A)$ се описва с регулярния израз
  \[\mathbf{r}^n_{1,i_1} + \mathbf{r}^n_{1,i_2} + \dots + \mathbf{r}^n_{1,i_k}.\]
\end{proof}

\begin{cor}
  Съществува алгоритъм, за който при вход краен детерминиран автомат $\A$,
  извежда като изход регулярен израз $\mathbf{r}$, такъв че $\L(\A) = \L(\mathbf{r})$.
\end{cor}

Доказателството на това следствие използва идеята от доказателството на \Th{regular-kleene},
но излиза извън нашия интерес. За подробно изложение на този въпрос, вижте \cite[стр. 69]{sipser3}.
Възможно е по-късно да използваме този резултат наготово.
Нека поне да разгледаме един пример, чиято цел е да ни убеди, 
че наистина може да се извлече алгоритъм от доказателството на \Th{regular-kleene}.

\begin{example}
  Да разгледаме следния автомат:

  \begin{framed}
    \begin{figure}[H]
      \begin{center}
        \begin{tikzpicture}[->,>=stealth,thick,node distance=45pt]
          \tikzstyle{every state}=[circle,minimum size=15pt,auto]
          
          \node[initial,state]      (1) {$q_1$};
          \node[accepting, state]   (2) [right of=1]{$q_2$};
          
          \path 
          (1) edge [loop above]  node [above] {$1$} (1)
          (1) edge  node [above] {$0$} (2)
          (2) edge [loop above] node [above] {$0,1$} (2);
        \end{tikzpicture}
      \end{center}
      \caption{Автомат разпознаващ $\L(\mathbf{1^\star 0 (0 + 1)^\star)}$}
      \label{fig:a1}
    \end{figure}
\end{framed}

Лесно се съобразява, че езикът на автомата от Фигура \ref{fig:a1} се описва с регулярния израз $\mathbf{1^\star 0 (0 + 1)^\star}$.
Следвайки конструкцията от доказателството на \Th{regular-kleene},
езикът на този автомат се описва с регулярния израз $\mathbf{r^2_{1,2}}$, защото началното състояние е $q_1$, финалното е $q_2$ и 
броят на състоянията в автомата е $2$.
\begin{align*}
  \mathbf{r}^2_{1,2} & = \underbrace{\mathbf{r}^{1}_{1,2}}_{\mathbf{1^\star 0}} + \underbrace{\mathbf{r}^{1}_{1,2}}_{\mathbf{1^\star 0}}\cdot \underbrace{\mathbf{(r^1_{2,2})^\star}}_{\mathbf{(\varepsilon+0+1)^\star}} \cdot \underbrace{\mathbf{r}^1_{2,2}}_{\mathbf{\varepsilon+0+1}} & \comment \text{според (\ref{eq:kleene}}) \\
                     &  = \mathbf{1^\star0 + 1^\star 0 (\varepsilon + 0 + 1)^\star (\varepsilon + 0 + 1)}\\
                     & = \mathbf{1^\star0 + 1^\star 0 (\varepsilon + 0 + 1)^+} & \comment \mathbf{r^+ \df r^\star r}\\
                     & = \mathbf{1^\star0 + 1^\star 0 (0 + 1)^\star} & \comment \mathbf{r^\star = (\varepsilon + r)^+}\\
                     & = \mathbf{1^\star 0 (\varepsilon + (0 + 1)^\star)} & \comment \mathbf{r + rq = r(\varepsilon + q)}\\
                     & = \mathbf{1^\star 0 (0 + 1)^\star} & \comment \mathbf{r^\star = \varepsilon + r^\star}
\end{align*}

Тук използвахме, че:
\begin{align*}
  \mathbf{r^1_{1,2}} & = \underbrace{\mathbf{r^0_{1,2}}}_{\mathbf{0}} + \underbrace{\mathbf{r^0_{1,1}}}_{\mathbf{\varepsilon + 1}}\cdot\underbrace{\mathbf{(r^0_{1,1})^\star}}_{\mathbf{(\varepsilon+1)^\star}} \cdot \underbrace{\mathbf{r^0_{1,2}}}_{\mathbf{0}}\\
                     & = \mathbf{0 + (\varepsilon + 1)(\varepsilon + 1)^\star0} \\
                     & = \mathbf{0 + 1^\star 0}  & \comment \mathbf{r}^\star = \varepsilon + \mathbf{r}^\star \\
                     & = \mathbf{1^\star0}, \\
  \mathbf{r^1_{2,2}} & = \underbrace{\mathbf{r^0_{2,2}}}_{\mathbf{\varepsilon+0+1}} + \underbrace{\mathbf{r^0_{2,1}}}_{\mathbf{\emptyset}} \cdot \underbrace{\mathbf{(r^0_{1,1})^\star}}_{\mathbf{\varepsilon+1}}\cdot \underbrace{\mathbf{r^0_{1,2}}}_{\mathbf{0}}\\
                     & = \mathbf{\varepsilon + 0 + 1 + \emptyset(\varepsilon + 1)^\star0}\\
                     & = \varepsilon + 0 + 1 & \comment \text{защото }\mathbf{\emptyset \cdot r = \emptyset}
\end{align*}
\end{example}


\begin{example}
  Да разгледаме автомата $\A$:
  
  \begin{framed}
    \begin{figure}[H]
      \begin{center}
        \begin{tikzpicture}[->,>=stealth,thick,node distance=45pt]
          \tikzstyle{every state}=[circle,minimum size=15pt,auto]
          
          \node[initial,state]      (1) {$q_1$};
          \node[state]              (2) [right of=1]{$q_2$};
          \node[accepting, state]   (3) [right of=2]{$q_3$};
          
          \path 
          (2) edge [loop above]  node [above] {$a$} (2)
          (1) edge  node [above] {$a$} (2)
          (2) edge  node [above] {$b$} (3)
          (1) edge [bend right=30] node [below] {$b$} (3)
          (3) edge [loop above] node [above] {$a,b$} (3);
        \end{tikzpicture}
      \end{center}
      \caption{Автомат разпознаващ $\L(\mathbf{a^\star b(a+b)^\star})$.}
      \label{fig:a1}
    \end{figure}
  \end{framed}
\end{example}
\begin{solution}
  $\L(\A) = \L(\mathbf{r}^3_{1,3})$.
  \begin{align*}
    \mathbf{r}^3_{1,3} & = \underbrace{\mathbf{r}^2_{1,3}}_{a^\star b} + \underbrace{\mathbf{r}^2_{1,3}}_{a^\star b} \cdot \underbrace{(\mathbf{r}^2_{3,3})^\star}_{(\varepsilon+a+b)^\star} \cdot \underbrace{\mathbf{r}^2_{3,3}}_{\varepsilon+a+b}\\
    & = a^\star b + a^\star b \cdot (a+b)^\star \cdot (\varepsilon+a+b) \\
    & = a^\star b + a^\star b \cdot (a+b)^\star & \comment \mathbf{r}^\star = \mathbf{r}^\star \cdot (\varepsilon + \mathbf{r})\\
    & = a^\star b (\varepsilon + (a+b)^\star)\\
    & = a^\star b (a+b)^\star & \comment \mathbf{r}^\star = \varepsilon + \mathbf{r}^\star.
  \end{align*}

  Тук използвахме, че:
  \begin{align*}
    \mathbf{r}^2_{1,3} & = \mathbf{r}^1_{1,3} + \mathbf{r}^1_{1,2} \cdot (\mathbf{r}^1_{2,2})^\star \cdot \mathbf{r}^1_{2,3}\\
                       & = b + a \cdot a^\star \cdot b\\
                       & = (\varepsilon + a^+)\cdot b & \comment \mathbf{r}^+ \df \mathbf{r} \cdot \mathbf{r}^\star\\
                       & = a^\star b & \comment \mathbf{r}^\star = \varepsilon + \mathbf{r}^+.г
  \end{align*}
\end{solution}


Следващата ни цел е да видим, че имаме и обратната посока на горната лема.
Ще докажем, че всеки регулярен език е автоматен. За тази цел първо ще 
въведем едно обобщение на понятието краен детерминиран автомат.


%%% Local Variables:
%%% mode: latex
%%% TeX-master: "../eai"
%%% End:

\section{Метод на Бжозовски}\label{sect:regular:brzozowski}
\index{Бжозовски}

Имаме следната операция за произволна буква $a$,
\[a^{-1}(L) \df \{\omega \in \Sigma^\star \mid a\omega \in L\}.\]
Аналогично, за произволна дума $\alpha$,
\[\alpha^{-1}(L) \df \{\omega \in \Sigma^\star \mid \alpha\omega \in L\}.\]

\begin{problem}
  Докажете, че:
  \begin{enumerate}[(1)]
  \item
    $a^{-1}(L_1 \cup L_2) = a^{-1}(L_1) \cup a^{-1}(L_2)$;
  \item
    $a^{-1}(L_1 \cap L_2) = a^{-1}(L_1) \cap a^{-1}(L_2)$;
  \item
    $a^{-1}(L_1 \setminus L_2) = a^{-1}(L_1) \setminus a^{-1}(L_2)$;
  \item
    $a^{-1}(L_1 \cdot L_2) =
    \begin{cases}
      a^{-1}(L_1) \cdot L_2, & \text{ ако }\varepsilon\not\in L_1\\
      a^{-1}(L_1) \cdot L_2 \cup L_2, & \text{ ако }\varepsilon\in L_1
    \end{cases}$
  \item
    $a^{-1}(L^\star) = a^{-1}(L) \cdot L^\star$.
  \end{enumerate}
\end{problem}

\begin{problem}
  Докажете, че
  \[(\alpha\beta)^{-1}(L) = \beta^{-1}(\alpha^{-1}(L)).\]
\end{problem}


\mynote{Бжозовски \cite{brzozowski-derivatives} описва алгоритъм за строене на автомат по регулярен израз.}

Нека е даден езикът $L$. Ще покажем конструкция на детерминиран автомат $\B = \FA$,
който разпознава $L$. Ако $L$ е регулярен, то $\B$ ще бъде детерминиран краен автомат,
но ако $L$ не е регулярен, то $\B$ ще бъде детерминиран \emph{безкраен} автомат.
Конструкцията на автомата $\B$ е следната:
\mynote{Да напомним, че имаме свойството
  \[\alpha \in L \iff \varepsilon \in \alpha^{-1}(L).\]
  Все още не ясно, че ако $L$ е регулярен, то $Q$ е крайно множество. Това ще видим след малко.
  В \Example{regular:brzozowski:an-bn} ще видим един детерминиран безкраен автомат за език, който не е регулярен.}
\begin{itemize}
\item
  Състоянията $Q$ ще бъдат от вида $q_M$, за $M \subseteq \Sigma^\star$, където:
  \[Q \df \{q_M \mid (\exists \alpha\in\Sigma^\star)[M = \alpha^{-1}(L)].\]
\item
  $\qstart \df q_L$.
\item
  За произволни езици $M$ и $N$ и буква $a$,
  \[\delta(q_M,a) \df q_N \stackrel{\text{деф}}{\iff} N = a^{-1}(M).\]
\item
  $F \df \{ q_M \in Q\mid \varepsilon \in M\}$.
\end{itemize}

\begin{proposition}\label{pr:regular:brzozowski:delta}
  За всяка дума $\alpha$ е изпълнено, че:
  \[N = \alpha^{-1}(L) \iff \delta^\star(q_L,\alpha) = q_N.\]
\end{proposition}
\begin{hint}
  Индукция по дължината на думата $\alpha$, като използвате, че
  \[(\alpha b)^{-1}(L) = b^{-1}(\alpha^{-1}(L)).\]
\end{hint}

\begin{proposition}
  За даден език $L$, нека $\B$ е детерменираният автомат построен по метода на Бжозовски.
  Тогава $L = \L(\B)$.
\end{proposition}
\begin{hint}
  Съобразете, че имаме следните еквивалентности:
  \begin{align*}
    \alpha \in L & \iff \varepsilon\in\alpha^{-1}(L) & \comment\text{нека }M \df \alpha^{-1}(L)\\
                 & \iff \varepsilon\in M\ \&\ q_M = \delta^\star(q_L,\alpha) & \comment\text{от \Proposition{regular:brzozowski:delta}}\\
                 & \iff \delta^\star(\qstart,\alpha) \in F. & \comment \qstart \df q_L \text{ и }\varepsilon \in \alpha^{-1}(L)
  \end{align*}
\end{hint}

%%% Local Variables:
%%% mode: latex
%%% TeX-master: "../eai"
%%% End:

\section{Недетерминирани крайни автомати}
\index{автомат!недетерминиран}
\begin{dfn}
  \marginpar{Въведени от Рабин и Скот \cite{rabin-scott}}
  \marginpar{За по-голяма яснота, често ще означаваме с $\N$ недетерминирани автомати}
  Недетерминиран краен автомат представлява
  \[\N = \NFA,\]
  \begin{itemize}
  \item
    $Q$ е крайно множество от състояния;
  \item
    $\Sigma$ е крайна азбука;
  \item
    $\Delta: Q\times\Sigma \to \Ps(Q)$ е функцията на преходите.
    \marginpar{Да напомним, че $\Ps(Q) = \{R\mid R\subseteq Q\}$, $\abs{\Ps(Q)} = 2^{\abs{Q}}$}
    Да обърнем внимание, че е възможно за някоя двойка $(q,a)$ да няма нито един преход в автомата.
    Това е възможно, когато $\Delta(q,a) = \emptyset$;
  \item
    $\qstart \in Q$ е началното състояние;
  \item
    $F\subseteq Q$ е множеството от финални състояния.
  \end{itemize}
\end{dfn}

Удобно е да разширим функцията на преходите $\Delta: Q\times\Sigma \to \Ps(Q)$ 
до функцията $\Delta^\star: \Ps(Q)\times\Sigma^\star \to \Ps(Q)$,
която дефинираме по следния начин:
\marginpar{Обърнете внимание, че $\Delta^\star(R,a) = \bigcup_{r\in R}\Delta(r,a)$.}
\begin{itemize}
\item 
  $\Delta^\star(R, \varepsilon) = R$, за произволно $R \subseteq Q$;
\item
  $\Delta^\star(R, \alpha x) = \bigcup_{p \in \Delta^\star(R,\alpha)} \Delta(p, x)$, за произволни $x \in \Sigma$, $\alpha \in \Sigma^\star$, $q\in Q$.
\end{itemize}

\begin{framed}
  \[\L(\N) \df \{\omega \in \Sigma^\star \mid \Delta^\star(\{\qstart\},\omega) \cap F \neq \emptyset \}.\]
\end{framed}

\begin{prop}
  За всеки две думи $\alpha,\beta \in \Sigma^\star$ и всяко $R \subseteq Q$,
  \[ \Delta^\star(R, \alpha\beta) = \Delta^\star( \Delta^\star(R,\alpha),\beta).\]
\end{prop}
\begin{proof}
  Отново индукция по дължината на $\beta$.
  \begin{itemize}
  \item
    Нека $\beta = \varepsilon$. Тогава:
    \begin{align*}
      \Delta^\star(R,\alpha\varepsilon) & = \Delta^\star(R,\alpha) \\
                                        & = \Delta^\star( \Delta^\star(R,\alpha), \varepsilon). & \comment\text{деф. на }\Delta^\star
    \end{align*}
  \item
    Да приемем, че твърдението е вярно за думи с дължина $n$.
  \item
    Нека $\beta = \gamma b$, където $|\gamma| = n$.
    \begin{align*}
      \Delta^\star(R, \alpha\gamma b) & = \bigcup_{q \in \Delta^\star(R,\alpha\gamma)} \Delta(q, b) & \comment\text{от деф. на }\Delta^\star\\
                                      & = \bigcup_{q \in \Delta^\star(\Delta^\star(R,\alpha),\gamma)} \Delta(q,b) & \comment\text{от И.П.}\\
                                      & = \Delta^\star( \Delta^\star(R,\alpha), \gamma b) & \comment\text{от деф. на }\Delta^\star.
    \end{align*}
    
  \end{itemize}
\end{proof}

\begin{problem}
  Докажете, че за произволни $R_i \subseteq Q$, където $i < k$, е изпълнено, че:
  \[\Delta^\star( \bigcup_{i<k} R_i, \alpha) = \bigcup_{i<k} \Delta^\star( R_i, \alpha).\]
\end{problem}

И тук е удобно да въведем бинарната релация $\vdash_\N$ над $Q\times\Sigma^\star$,
която ще ни казва как моментното описание на автомата $\N$ се променя след изпълнение на една стъпка:
\[(q,x\alpha) \vdash_\N (p,\alpha), \text{ ако } p \in \Delta(q,x).\]
\marginpar{Рефл. и транз. затваряне на една релация е разгледано в Глава \ref{ch:intro}}
Рефлексивното и транзитивно затваряне на $\vdash_\N$ ще означаваме с $\vdash^\star_\N$.
За да дадем точна дефиниция на $\vdash^\star_\N$, първо ще дефинираме релацията $\vdash^n_\N$, която
определя работата на автомата $\N$ за $n$ стъпки.
\begin{itemize}
\item 
  $(q,\alpha) \vdash^0_\N (q,\alpha)$, защото за $0$ стъпки се случва нищо.
\item
  Нека $\Delta(q,x) \ni q'$ и $(q',\alpha) \vdash^n_\N (p, \beta)$. Тогава
  $(q,x\alpha) \vdash^{n+1}_\N (p,\beta)$, защото за $n+1$ стъпки първо правим една стъпка 
  и отиваме в моментното описание $(q',\alpha)$ и след това правим още $n$ стъпки.
\end{itemize}
Сега можем да дефинираме $\vdash^\star_\N$ като:
\[(q,\alpha) \vdash^\star_\N (p,\beta) \dff (\exists n\in\Nat)[(q,\alpha) \vdash^n_\N (p,\beta)].\]
Друг начин да дефинираме релацията $\vdash^\star_\N$ е следния:
\[(q,\alpha\beta) \vdash^\star_\N (p, \beta) \iff p \in \Delta^\star(\{q\},\alpha).\]
Получаваме, че 
\[\L(\N) = \{\alpha\in\Sigma^\star \mid (\exists q \in F)[(\qstart,\alpha) \vdash^\star_\N (q,\varepsilon)]\}.\]


\begin{framed}
\begin{thm}[Рабин-Скот \cite{rabin-scott}]
  За всеки недетерминиран краен автомат $\N$ съществува еквивалентен на него детерминиран краен автомат $\D$, т.е. $\L(\N) = \L(\D)$.
\end{thm}
\end{framed}
\begin{hint}
  Нека $\N = \NFA$. Ще построим детерминиран автомат
  \[\D = (Q',\Sigma,\delta,\qstart',F'),\]
  за който $\L(\N) = \L(\D)$.
  Конструкцията е следната:
  \marginpar{Да отбележим, че детерминираният автомат $\D$ има не повече от $2^{\abs{Q}}$ на брой състояния $Q'$}
  \begin{itemize}
  \item
    $Q' = \Ps(Q)$. Някои от тези състояния може да са недостижими и следователно да са излишни, но в общия случай трябва да имаме
    всички подмножества на $Q$.
  \item
    За произволна буква $a\in\Sigma$ и произволно $R \subseteq Q$,
    \begin{align*}
      \delta(R,a) & = \{q\in Q\mid (\exists r\in R)[q\in\Delta(r,a)]\}\\
                  & = \bigcup_{r\in R}\Delta(r,a)\\
                  & = \Delta^\star(R,a).
    \end{align*}
  \item
    $\qstart' = \{\qstart\}$;
  \item
    $F' = \{R \subseteq Q \mid R\cap F \neq \emptyset\}$.
  \end{itemize}
  Ще докажем с индукция по дължината на думата $\alpha$, че
  \begin{equation}
    \label{eq:6}
    (\forall \alpha\in\Sigma^\star)(\forall R \subseteq Q)[\ \Delta^\star(R,\alpha) = \delta^\star(R,\alpha)\ ].
  \end{equation}

  \begin{itemize}
  \item
    Ако $|\alpha| = 0$, т.е. $\alpha = \varepsilon$, то е ясно от дефиницията на $\Delta^\star$ и $\delta^\star$.
  \item
    Да приемем, че (\ref{eq:6}) е изпълнено за думи $\alpha$ с дължина $n$.
  \item
    Нека сега $\alpha$ има дължина $n+1$, т.е. $\alpha = \beta a$, където $|\beta| =n$ и $a \in \Sigma$.
    Тогава:
    \begin{align*}
      \Delta^\star(R, \beta a) & = \Delta^\star(\Delta^\star(R,\beta), a) & \comment\text{деф. на }\Delta^\star\\
                               & = \Delta^\star(\delta^\star(R,\beta), a) & \comment\text{от И.П.}\\
                               & = \delta( \delta^\star(R, \beta), a) & \comment\text{деф. на }\delta\\
                               & = \delta^\star( R, \beta a) & \comment\text{деф. на }\delta^\star
    \end{align*}
  \end{itemize}
  Сега вече е лесно да съобразим, че
  \begin{align*}
    \omega \in \L(\D) & \iff \delta^\star(\{\qstart\},\omega) \in F' & \comment\text{деф. на }\L(\D)\\
                      & \iff \delta^\star(\{\qstart\},\omega) \cap F \neq \emptyset & \comment\text{деф. на }F'\\
                      & \iff \Delta^\star(\{\qstart\},\omega) \cap F \neq \emptyset & \comment\text{от (\ref{eq:6})}\\
                      & \iff \omega \in \L(\N) & \comment\text{деф. на }\L(\N).
  \end{align*}
\end{hint}

\begin{problem}
  За всеки НКА $\N$ съществува НКА $\N'$ с едно финално състояние, 
  за който $\L(\N) = \L(\N')$.
\end{problem}
\begin{hint}
  Вместо формална конструкция, да разгледаме един пример, който илюстрира идеята.
  \begin{figure}[H]
    \begin{subfigure}[b]{0.3\textwidth}
      \begin{tikzpicture}[framed,->,>=stealth,thick,node distance=45pt]
        \tikzstyle{every state}=[circle,minimum size=20pt,auto]
        \node[initial below,state]      (1) {$q_0$};
        \node[state,accepting]     [above right of=1] (2) {$q_1$};
        \node[state,accepting]     [below right of=1] (3) {$q_2$};
        \path
        (1) edge [bend left=15] node  [above] {$a$} (2)
        (2) edge [bend left=15] node  [right] {$b$} (1)
        % (2) edge [loop above] node  [above] {$a$} (2)
        (2) edge [bend left=15] node  [right] {$a$} (3)
        (3) edge [bend left=15] node  [below] {$a$} (1)
        (3) edge [loop below] node  [right] {$b$} (3);
        % (1) edge [bend right=15] node [below] {$b$} (3);
      \end{tikzpicture}
      \caption{автомат $\N$}
    \end{subfigure}
    \begin{subfigure}[b]{0.4\textwidth}
      \begin{tikzpicture}[framed,->,>=stealth,thick,node distance=45pt]
        \tikzstyle{every state}=[circle,minimum size=20pt,auto]
        \node[initial below,state]                        (1) {$q_0$};
        \node[state]     [above right of=1]         (2) {$q_1$};
        \node[state]     [below right of=1]         (3) {$q_2$};
        \node[state,accepting]     [right=3cm of 1] (4) {$f$};
        \path
        (1) edge [bend left=15] node  [above] {$a$} (2)
        % (2) edge [loop above] node  [above] {$a$} (2)
        (2) edge [bend left=15] node  [right] {$b$} (1)
        (2) edge [bend left=15] node  [right] {$a$} (3)
        (3) edge [loop below] node  [right] {$b$} (3)
        (3) edge [bend left=15] node  [below] {$a$} (1)
        (1) edge [dashed,bend left=15] node  [above] {$a$} (4)
        (2) edge [dashed,bend left=15] node  [above] {$a$} (4)
        (3) edge [dashed,bend right=15] node  [below] {$b$} (4);
        % (1) edge [bend right=15] node [below] {$b$} (3);
      \end{tikzpicture}
    \caption{автомат $\N'$, $\L(\N') = \L(\N)$}
  \end{subfigure}
\end{figure}  
За произволен автомат $\N$, формулирайте точно конструкцията на $\N'$ с едно финално състояние и докажете, че наистина $\L(\N) = \L(\N')$.
Обърнете внимание, че примера показва, че е възможно $\N$ да е детерминиран автомат, но полученият $\N'$ да бъде недетерминиран.
\end{hint}

\begin{problem}
  \marginpar{Това означава, че автоматните езици са затворени относно операцията $\texttt{rev}$. По-късно ще видим, че можем да дадем и друго доказателство на това твърдение, като направим индукция по построението на регулярните езици.}
  Докажете, че ако $L$ е автоматен език, то $L^{rev} = \{\omega^{rev} \mid \omega \in L\}$
  също е автоматен.
\end{problem}
% \begin{hint}
%   Нека $\A = \FA$, $L = \L(\A)$, е само с едно финално състояние, т.е. $F = \{\qaccept\}$.
%   Разгледайте недетерминистичния краен автомат $\N = \pair{\Sigma,Q,\qstart',\Delta,F'}$, където
%   \begin{itemize}
%   \item
%     $\Delta(q,a) = \{p \in Q \mid \delta(p,a) = q\}$;
%   \item
%     $\qstart' = \qaccept$;
%   \item
%     $F' = \{\qstart\}$;
%   \end{itemize}

% \end{hint}


\begin{lemma}
  \label{lem:automata-basic}
  Съществува детерминистичен автомат $\A = \FA$, който разпознава езика $L$, където
  \begin{itemize}
  \item
    $L = \emptyset$,
  \item
    $L = \{\varepsilon\}$, или
  \item
    $L = \{a\}$, за произволна буква $a\in\Sigma$.
  \end{itemize}
\end{lemma}
\begin{hint}
  \begin{figure}[H]
    \begin{subfigure}[b]{0.2\textwidth}
      \label{subf:a1}
      \begin{tikzpicture}[->,>=stealth,thick,node distance=35pt]
        \tikzstyle{every state}=[circle,minimum size=15pt,auto]
        \node[initial below,state]      (1) {$q_0$};
      \end{tikzpicture}
      \caption{$L(\A) = \emptyset$}
    \end{subfigure}
    \qquad
    \begin{subfigure}[b]{0.2\textwidth}
      \begin{tikzpicture}[->,>=stealth,thick,node distance=35pt]
        \tikzstyle{every state}=[circle,minimum size=15pt,auto]
        \node[initial below,state,accepting]      (1) {$q_0$};
      \end{tikzpicture}
      \caption{$\L(\A) = \{\varepsilon\}$}
    \end{subfigure}
    \qquad
    \begin{subfigure}[b]{0.3\textwidth}
      \begin{tikzpicture}[->,>=stealth,thick,node distance=45pt]
        \tikzstyle{every state}=[circle,minimum size=15pt,auto]
        \node[initial below,state]      (1)              {$q_0$};
        \node[accepting,state]    (2) [right of=1] {$q_1$};
        \path 
        (1) edge  node [above] {$a$} (2);
      \end{tikzpicture}
      \caption{$\L(\A) = \{a\}$}
    \end{subfigure}
  \end{figure}
\end{hint}

\begin{lemma}
  \label{lem:concat}
  Класът на автоматните езици е затворен относно операцията {\em конкатенация}.
  Това означава, че ако $L_1$ и $L_2$ са два произволни автоматни езика, то $L_1\cdot L_2$
  също е автоматен език.
\end{lemma}
\begin{proof}
  Нека са дадени детерминистичните автомати:
  \begin{itemize}
  \item
    $\A_1 = \pair{\Sigma,Q_1,\delta_1,\qstart',F_1}$, където $\L(\A_1) = L_1$;
  \item
    $\A_2 = \pair{\Sigma,Q_2,\delta_2,\qstart'', F_2}$, където $\L(\A_2) = L_2$.
  \end{itemize}
  Ще дефинираме автомата $\N = \NFA$ по такъв начин, че
  \[\L(\N) = L_1\cdot L_2 = \L(\A_1)\cdot\L(\A_2).\]
  \begin{itemize}
  \item
    $Q = Q_1 \cup Q_2$;
  \item
    $\qstart = \qstart''$;
  \item
    $F = 
    \begin{cases}
      F_1 \cup F_2, & \text{ ако } \qstart'' \in F_2\\
      F_2,          & \text{ иначе}.
    \end{cases}$
  \item 
    $\Delta(q,a) = 
    \begin{cases}
      \{\delta_1(q,a)\},                      & \text{ ако }q\in Q_1\setminus F_1\ \&\ a\in\Sigma\\
      \{\delta_1(q,a), \delta_2(\qstart'',a)\}, & \text{ ако }q \in F_1\ \&\ a\in\Sigma\\
      \{\delta_2(q,a)\},                      & \text{ ако }q\in Q_2\ \&\ a\in\Sigma.
    \end{cases}$
  \end{itemize}

  % \begin{enumerate}[(1)]
  % \item 
  %   $(\forall q,p \in Q_1)[(q, \alpha\beta) \vdash^\star_{\A_1} (p, \beta) \iff (q, \alpha\beta) \vdash^\star_{\N} (p, \beta)]$;
  % \item
  %   $(\forall q,p \in Q_2)[ (q,\alpha\beta) \vdash^\star_{\A_2} (p, \beta) \implies (q,\alpha\beta) \vdash^\star_\N (p, \beta)]$;
  % % \item
  % %   Нека $q \in Q_1$ и $p \in Q_2$ са произволни състояния, за които $(q,\omega) \vdash^\star_\N (p,\varepsilon)$.
  % %   Тогава съществува $f_1 \in F_1$ и $\omega = \omega_1 b\omega_2$, такива че:
  % %   $(q,\omega_1\omega_2) \vdash^\star_{\A_1} (f_1,b\omega_2) \vdash_\N (r, \omega_2) \vdash^\star_{\A_2} (p,\varepsilon)$.
  % \end{enumerate}
  % \marginpar{\writedown Разгледайте и случая, когато $\beta = \varepsilon$!}
  % Нека $\alpha \in \L(\A_1)$ и $\beta = b\gamma \in \L(\A_2)$.
  % Тогава имаме, че:
  % \begin{enumerate}[(a)]
  % \item
  %   $(\qstart',\alpha) \vdash^\star_{\A_1}(f_1,\varepsilon)$, за някое $f_1 \in F_1$;
  % \item
  %   $(\qstart'',b\gamma) \vdash_{\A_2} (p, \gamma) \vdash^\star_{\A_2}(f_2,\varepsilon)$, за някое $p \in Q_2$ и $f_2 \in F_2$.
  % \end{enumerate}
  
  % \begin{align*}
  %   (q,\alpha b\gamma) & \vdash^\star_\N(f_1,b\gamma) & \comment\text{от (1) и (а) }\\
  %                      & \vdash_\N(p,\gamma) & \comment\text{от деф. на $\N$ и (б)}\\
  %                      & \vdash^\star_\N(f_2,\varepsilon) & \comment\text{от (2) и (б)}.
  % \end{align*}

  % Оттук заключаваме, че $\L(\A_1) \cdot \L(\A_2) \subseteq \L(\N)$.

  % Нека сега $\alpha \in \L(\N)$, т.е.
  % $(\qstart,\alpha) \vdash^\star_\N (p, \varepsilon)$, за някое $p \in F$.
  % Нека $p \in F_2$.
  % За $\alpha = a_0a_1\cdots a_{n-1}$, да разгледаме редицата $q_0,q_1,\dots,q_n$,
  % която отговаря на някое от успешните изчисления на $\N$ върху $\alpha$, завършващи в $p$, т.е.
  % \begin{itemize}
  % \item
  %   $q_0 = \qstart$;
  % \item
  %   $q_{i+1} \in \Delta(q_i,a_i)$;
  % \item
  %   $q_n = p \in F_2$.
  % \end{itemize}
  % От конструкцията на $\N$ следва, че редицата $(q_i)^n_{i=0}$ може да се разбие на две части:
  % $q_i \in Q_1$ за $0 \leq i \leq l$ и $q_i \in Q_2$ за $l < i \leq n$.
  % Тогава от конструкцията на $\N$ следва, че:
  % \[(q_0,\alpha) \vdash^\star_{\A_1} ( q_l, a_la_{l+1}\cdots a_n) \vdash_\N (q_{l+1},a_{l+1}\cdots a_n) \vdash^\star_{\A_2} (q_n,\varepsilon).\]
  % От конструкцията на $\N$ следва, че $q_l \in F_1$ и щом $q_{l+1} \in Q_2$, то $\delta_2(\qstart'',a_l) = q_{l+1}$.
  % Оттук заключаваме, че $a_0a_1\cdots a_l \in \L(\A_1)$ и $a_{l+1}\cdots a_{n-1} \in \L(\A_2)$.

  
  Първо ще докажем, че $\L(\A_1)\cdot\L(\A_2) \subseteq \L(\N)$.
  За целта, нека разгледаме думата $\alpha \in \L(\A_1)$, където $\alpha = a_0a_1\cdots a_{l-1}$, и нека редицата $(q_i)^n_{i=0}$ описва приемащото изчисление на $\A_1$ върху думата $\alpha$.
  Това означава, че:
  \begin{itemize}
  \item
    $q_0 = \qstart'$;
  \item
    $q_{i+1} = \delta_1(q_i,a_i)$ за $i < n$;
  \item
    $q_n \in F_1$.
  \end{itemize}  
  Също така, нека разгледаме думата $\beta \in \L(\A_2)$, където $\beta = b_0b_1\cdots b_{m-1}$, и нека редицата $(p_i)^m_{i=0}$ описва приемащото изчисление на $\A_2$ върху думата $\beta$.
  Това означава, че:
  \begin{itemize}
  \item
    $p_0 = \qstart''$;
  \item
    $p_{i+1} = \delta_2(p_i,b_i)$ за $i < m$;
  \item
    $p_m \in F_2$.
  \end{itemize}  
  От конструкцията на $\N$ се вижда лесно, че $(q_i)^l_{i=0}$ описва изчисление на $\N$ върху $\alpha$ и
  $(p_i)^{m}_{i=0}$ описва изчисление на $\N$ върху $\beta$.
  Това означава, че:
  \begin{itemize}
  \item
    $q_{i+1} \in \Delta(q_i,a_i)$ за $i < n$;
  \item
    $p_{i+1} \in \Delta(p_i,b_i)$ за $i < m$.
  \end{itemize}
  Тогава:
  \[(q_0,\alpha\beta) \vdash^\star_\N (q_l,\beta)\text{ и } (p_0, \underbrace{b_0b_1\cdots b_{m-1}}_{\beta}) \vdash_\N (p_1,b_1\cdots b_{m-1}) \vdash^\star_{\N} (p_m,\varepsilon).\]
  Понеже $q_l \in F_1$, а $p_0 = \qstart''$, то от конструкцията на $\N$ следва, че
  \[(q_l, \underbrace{b_0b_1\cdots b_{m-1}}_{\beta}) \vdash_\N (p_1,b_1\cdots b_{m-1}),\]
  защото $\delta_2(\qstart'',b_0) \in \Delta(q_l,b_0)$.
  
  Обединявайки всичко това, получаваме, че:
  \begin{align*}
    (\qstart, \alpha\beta) & \vdash^\star_\N (q_l,\beta)\\
                           & \vdash_\N (p_1,b_1\cdots b_{m-1})\\
                           & \vdash^\star_\N(p_m,\varepsilon).
  \end{align*}
  Понеже $p_m \in F_2$, то $\alpha\beta \in \L(\N)$.
  

  Сега ще докажем, че $\L(\N) \subseteq \L(\A_1) \cdot \L(\A_2)$.
  За целта, нека разгледаме думата $\omega \in \L(\N)$, където $\omega = a_0a_1\cdots a_{n-1}$.
  Да разгледаме редицата от състояния $(q_i)^{n}_{i=0}$, която описва едно приемащо изчисление на $\N$ върху $\omega$.
  \marginpar{Възможно е да има и други редици от състояния $(p_i)^{n}_{i=0}$, които да описват приемащи изчисления на $\N$ върху $\omega$.}
  Това означава, че:
  \begin{itemize}
  \item
    $q_0 = \qstart$;
  \item
    $q_{i+1} \in \Delta(q_i,a_i)$ за $i < n$;
  \item
    $q_n \in F$.
  \end{itemize}

  \marginpar{Ако $q_n \in F_1$, то според конструкцията на $\N$, $\varepsilon \in \L(\A_2)$ и всяко състояние от $(q_i)^{n}_{i=0}$ принадлежи на $Q_1$ и оттам $\omega \in \L(\A_1)$.}
  Интересният случай е когато $q_n \in F_2$.
  Според конструкцията на $\N$, можем да разбием редицата от състояния $(q_i)^n_{i=0}$ на две непразни подредици:
  \begin{itemize}
  \item
    $(q_{i})^{l}_{i=0}$ - тези които са от $Q_1$,
  \item
    $(q_i)^{n}_{i=l+1}$ - тези, които са от $Q_2$.
  \end{itemize}
  Нека $\alpha = a_0a_1\cdots a_{l-1}$ и $\beta = a_la_{l+1}\cdots a_{n-1}$.
  Ясно е, че:
  \[(q_0,\alpha\beta) \vdash^\star_\N (q_{l}, a_{l}a_{l+1}\cdots a_{n-1}) \vdash_\N (q_{l+1},a_{l+1}\cdots a_{n-1}) \vdash^\star_\N (q_n,\varepsilon).\]
  От конструкцията на $\N$ следва, че редицата от състояния $(q_i)^{l}_{i=0}$ описва изчислението на $\A_1$ върху $\alpha$.
  \marginpar{Това е единственият начин да направим преход от състояние на $Q_1$ към състояние на $Q_2$.}
  Също така от конструкцията следва, че щом $q_{l+1} \in \Delta(q_l,a_l)$, то $q_l \in F_1$ и $\delta_2(\qstart'',a_l) = q_{l+1}$. Заключаваме, че:
  \begin{itemize}
  \item
    $(q_0, \alpha) \vdash^\star_{\A_1} (q_l,\varepsilon)$.
    Понеже $q_0 = \qstart'$ и $q_l \in F_1$, то $\alpha \in \L(\A_1)$.
  \item
    $(\qstart'', \beta) \vdash^\star_{\A_2} (q_n,\varepsilon)$.
    Понеже $q_n \in F_2$, то $\beta \in \L(\A_2)$.
  \end{itemize}
\end{proof}

\begin{figure}[H]
  \center
  \begin{subfigure}[b]{0.3\textwidth}
    \label{subf:a1}
    \begin{tikzpicture}[framed,->,>=stealth,thick,node distance=45pt]
      \tikzstyle{every state}=[circle,minimum size=15pt,auto]
      \node[initial,state,accepting]      (1) {$q'_0$};
      \node[state]                        (2) [right of=1] {$q_1$};
      \node[state]                        (3) [above right of=2] {$q_2$};
      \node[state,accepting]              (4) [below right of=2] {$q_3$};
      \path
      (1) edge [loop above] node [above] {$b$} (1)
      (1) edge node [above] {$a,b$} (2)
      (2) edge node [above] {$a$} (3)
      (2) edge node [below] {$b$} (4)
      (3) edge [bend right=30] node [above] {$a$} (1)
      (4) edge [bend right=15] node [right] {$a$} (3)
      (4) edge [bend left=30] node [below] {$b$} (1);
    \end{tikzpicture}
    \caption{автомат $\A_1$}
  \end{subfigure}
  \qquad
  \qquad
  \qquad
  \begin{subfigure}[b]{0.3\textwidth}
    \begin{tikzpicture}[framed,->,>=stealth,thick,node distance=45pt]
      \tikzstyle{every state}=[circle,minimum size=15pt,auto]
      \node[initial,state]                (1) {$q''_0$};
      \node[state]     [above right of=1] (2) {$q_4$};
      \node[state,accepting]     [below right of=1] (3) {$q_5$};
      \path
      (1) edge [bend left=15] node  [above] {$a$} (2)
      (2) edge [bend left=15] node  [right] {$a$} (3)
      (3) edge [loop right]  node [right] {$a,b$} (3)
      (1) edge [bend right=15] node [below] {$b$} (3);
    \end{tikzpicture}
    \caption{автомат $\A_2$}
  \end{subfigure}
\end{figure}

\begin{example}
    За да построим автомат, който разпознава конкатенацията на $\L(\N_1)$ и $\L(\N_2)$,
    трябва да свържем финалните състояния на $\N_1$ с изходящите от $s_2$ състояния на $\N_2$.
    
    \begin{figure}[H]
      \center
      % \begin{subfigure}[b]{0.3\textwidth}
      \begin{tikzpicture}[framed,->,>=stealth,thick,node distance=2cm]
        \tikzstyle{every state}=[circle,minimum size=15pt,auto]
        \node[initial,state]                      (1) {$s_1$};
        \node[state] [right of=1]                 (2) {$q_1$};
        \node[state] [above right of=2]           (3) {$q_2$};
        \node[state] [below right of=2]           (4) {$q_3$};
        \node[state] [right=4cm of 1]             (5) {$s_2$};
        \node[state] [above right of=5]           (6) {$q_4$};
        \node[state,accepting] [below right of=5] (7) {$q_5$};
        \path
        (1) edge node [above]                         {$a$} (2)
        (2) edge node [above]                         {$a$} (3)
        (2) edge node [below]                         {$b$} (4)
        (3) edge [bend right=15] node [above]         {$a$} (1)
        (4) edge [bend left=15] node [below]          {$b$} (1)
        (5) edge [bend left=15] node [below]          {$a$} (6)
        (6) edge [bend left=15] node [right]          {$a$} (7)
        (5) edge [bend right=15] node [above]         {$b$} (7)
        (1) edge [dashed, bend left=45] node [above]  {$a$} (6)
        (1) edge [dashed, bend right=45] node [below] {$b$} (7)
        (4) edge [dashed, bend left=45] node [above]  {$a$} (6)
        (4) edge [dashed, bend left=10] node [above]  {$b$} (7);
      \end{tikzpicture}
      \caption{$\L(\N) = \L(\A_1)\cdot\L(\A_2)$}
  \end{figure}  
  Обърнете внимание, че $\A_1$ и $\A_2$ са детерминирани автомати, но $\N$ е недетерминиран.
  Също така, в този пример се оказва, че вече $q''_0$ е недостижимо състояние, но в общия случай не можем да 
  го премахнем, защото може да има преходи влизащи в $q''_0$.
\end{example}

\begin{lemma}
  \label{lem:union}
  \marginpar{Второ доказателство на това твърдение.}
  Класът от автоматните езици е затворен относно операцията {\em обединение}.
\end{lemma}
\begin{hint}
  Нека са дадени детерминистичните автомати:
  \begin{itemize}
  \item 
    $\A_1 = \pair{\Sigma,Q_1,\delta_1,\qstart',F_1}$, като $L(\A_1) = L_1$;
  \item
    $\A_2=\pair{\Sigma,Q_2,\delta_2,\qstart'',F_2}$, като $L(\A_2) = L_2$.
  \end{itemize}
  Ще дефинираме автомата $\N=\NFA$, така че
  \[L(\N) = L(\A_1) \cup L(\A_2).\]
  \begin{itemize}
  \item 
    $Q = Q_1 \cup Q_2 \cup \{\qstart\}$, където $\qstart\not\in Q_1\cup Q_2$;
  \item
    $F = 
    \begin{cases}
      F_1 \cup F_2 \cup \{\qstart\}, & \text{ ако } \qstart' \in F_1 \vee \qstart'' \in F_2\\
      F_1 \cup F_2,            & \text{ иначе } 
    \end{cases}$
  \item
    $\Delta(q,a) = 
    \begin{cases}
      \{\delta_1(q,a)\},                       & \text{ ако } q\in Q_1\ \&\ a\in\Sigma\\
      \{\delta_2(q,a)\},                       & \text{ ако } q\in Q_2\ \&\  a\in\Sigma\\
      \{\delta_1(\qstart',a), \delta_2(\qstart'',a)\}, & \text{ ако } q = \qstart\ \&\ a \in\Sigma.
    \end{cases}$
  \end{itemize}
\end{hint}


\begin{example}
    За да построим автомат, който разпознава обединението на $\L(\N_1)$ и $\L(\N_2)$,
    трябва да добавим ново начално състояние, което да свържем с наследниците на началните състояния на $\N_1$ и $\N_2$.
    
    \begin{figure}[H]
      \center
      % \begin{subfigure}[b]{0.3\textwidth}
      \begin{tikzpicture}[framed,->,>=stealth,thick,node distance=2cm]
        \tikzstyle{every state}=[circle,minimum size=20pt,auto]
        \node[initial,state,accepting]      (0) {$q_0$};
        \node[state,accepting] [above right of=0]        (1) {$q'_0$};
        \node[state]    [right of=1]        (2) {$q_1$};
        \node[state]                        (3) [above right of=2] {$q_2$};
        \node[state,accepting]                        (4) [below right of=2] {$q_3$};
        \node[state]    [below right=2cm of 0] (5) {$q''_0$};
        \node[state]     [above right of=5] (6) {$q_4$};
        \node[state,accepting]     [below right of=5] (7) {$q_5$};
        \path
        (1) edge [loop above] node [above] {$b$} (1)
        (1) edge node [above]                  {$a$} (2)
        (2) edge node [above]                  {$a$} (3)
        (2) edge node [below]                  {$b$} (4)
        (3) edge [bend right=15] node [above]  {$a,b$} (1)
        (4) edge [bend left=15]  node [below]  {$b$} (1)
        (4) edge [bend right=15]  node [right]  {$a$} (3)
        (5) edge [bend left=15] node [below]   {$a$} (6)
        (6) edge [bend left=15] node  [right] {$a,b$} (7)
        (5) edge [bend right=15] node [above]  {$b$} (7)
        (7) edge [loop right] node [right] {$a,b$} (7)
        (0) edge [dashed, bend right=15] node [below]  {$a$} (2)
        (0) edge [dashed, bend left=15] node [above]  {$b$} (1)
        (0) edge [dashed, bend right=15] node [below]  {$a$} (6)
        (0) edge [dashed, bend right=45] node [below]  {$b$} (7);
      \end{tikzpicture}
      \caption{$\L(\N) = \L(\A_1)\cup\L(\A_2)$}
  \end{figure}  
  Обърнете внимание, че $\A_1$ и $\A_2$ са детерминирани автомати, но $\N$ е недетерминиран.
  Освен това, новото състояние $q_0$ трябва да бъде маркирано като финално, защото $q'_0$ е финално.
\end{example}

\begin{lemma}
  \label{lem:kleene-star}
  Класът от автоматните езици е затворен относно операцията {\em звезда на Клини}, т.е.
  ако $\L(\A) = L$, то съществува $\N$, за който $\L(\N) = L^\star$.
\end{lemma}
\begin{proof}
  Нека е даден автомата $\A = \pair{\Sigma,Q,\qstart,\delta,F}$, за който е изпънено, че
  $\L(\A) = \L(\mathbf{r})$.
  
  Ще построим $\N = \pair{\Sigma,Q', \qstart', \Delta, F'}$, такъв че
  \[\L(\N) = (\L(\A))^\star.\]


  

  % Първата стъпка е да построим $\N_1 = \NFAn{1}$, такъв че 
  % \[\L(\N_1) = \bigcup_{n\geq 1} (\L(\N))^n = \bigcup_{n\geq 1} (\L(\mathbf{r}))^n = \L(\mathbf{r^+}).\]
  Това можем да направим по следния начин:
  \begin{itemize}
  \item
    $Q' = Q \cup \{\qstart'\}$;
  \item
    $F' = F \cup \{\qstart'\}$;
  \item
    За $q \in Q$ и $a \in \Sigma$, определяме функцията на преходите $\Delta$ по следния начин:
    \begin{align*}
      \Delta(q,a) \df
      \begin{cases}
        \{\delta(q,a)\}, & \text{ако }\delta(q,a) \not\in F\\
        \{\delta(q,a), \qstart\}, & \text{ако }\delta(q,a) \in F.
      \end{cases}
    \end{align*}
    За $a \in \Sigma$,
    \begin{align*}
      \Delta(\qstart',a) \df
      \begin{cases}
        \{\delta(\qstart, a)\}, & \text{ако }\delta(\qstart,a) \not\in F\\
        \{\delta(\qstart, a), \qstart\}, & \text{ако }\delta(\qstart,a) \in F.
      \end{cases}
    \end{align*}
  \end{itemize}


  Нека $\alpha = a_0a_1\cdots a_{n-1} \in \L(\N)$.
  Това означава, че $(\qstart',\alpha) \vdash^\star_\N (f,\varepsilon)$ за някое $f \in F$.
  Нека редицата от състояния $(q_i)^n_{i=0}$ описва едно приемащо изчисление на $\N$ върху $\alpha$, т.е.
  \begin{itemize}
  \item
    $q_0 = \qstart'$;
  \item
    $q_{i+1} \in \Delta(q_i,a_i)$;
  \item
    $q_n \in F$.
  \end{itemize}
  \marginpar{Възможно е $l = 0$ и съответно тази подредица да е празна.}
  Да разгледаме подредицата $(q_{i_j})^{l-1}_{j = 0}$ на $(q_i)^{n}_{i=0}$ съставена от тези състояния, за които
  \[\delta(q_{i_j}, a_{i_j}) \in F\ \&\ q_{i_j+1} = \qstart.\]

  Да положим:
  \begin{align*}
    & \alpha_{i_0} \df a_0\cdots a_{i_0};\\
    & \alpha_{i_{j+1}} \df a_{i_j+1}\cdots a_{i_{j+1}}\text{, за }0\leq j \leq l-2\\
    & \alpha_{i_l} \df a_{i_{l-1}+1}\cdots a_n.
  \end{align*}
  Ясно е, че $\alpha = \alpha_{i_0}\alpha_{i_1}\cdots\alpha_{i_l}$.

  Сега можем да разбием изчислението на $\N$ върху $\alpha$ по следния начин:
  \begin{align*}
    (\qstart',\alpha_{i_0}\alpha_{i_1}\cdots \alpha_{i_l}) & \vdash^\star_\N (q_{i_0+1}, \alpha_{i_1}\cdots \alpha_{i_l})\\
                                                           & \vdash^\star_\N (q_{i_1+1},\alpha_{i_2}\cdots \alpha_{i_l})\\
                                                           & \vdash^\star_\N\\
                                                           & \cdots\\
                                                           & \vdash^\star_\N (q_{i_{l-1}+1}, \alpha_{i_l})\\
                                                           & \vdash^\star_\N(q_n,\varepsilon).
  \end{align*}

  За $j = 0$ имаме, че:
  \[(\qstart',\alpha_{i_0}) \vdash_\N (q_1, a_1\cdots a_{i_0}) \vdash^\star_\N (q_{i_0}, a_{i_0}) \vdash_\N (q_{i_0+1}, \varepsilon).\]

  Понеже, според конструкцията на $\N$, $\delta(\qstart, a_0) = q_1$ и освен това $\delta(q_{i_0}, a_{i_0}) = p$, за някое $p \in F$, то 
  \[(\qstart,\alpha_{i_0}) \vdash_\A (q_1, a_1\cdots a_{i_0}) \vdash^\star_\A (q_{i_0}, a_{i_0}) \vdash_\A (p, \varepsilon).\]
  Оттук следва, че думата $\alpha_{i_0} \in \L(\A)$.
  В случая, когато $\alpha_{i_0} = a_0$, то $\delta(\qstart,a_0) \in F$ и $q_{i_0} = \qstart$
  
  
  За всяко $j$, където $0<j<l$, понеже $\delta(q_{i_j}, a_{i_j}) = p$, за някое $p \in F$, то 
  \[(q_{i_j+1}, \alpha_{i_j}) \vdash^\star_\A (q_{i_{j+1}}, a_{i_{j+1}}) \vdash_\A (p, \varepsilon).\]
  Понеже $q_{i_j+1} = \qstart$, то оттук следва, че думата $\alpha_{i_j} \in \L(\A)$.

  За $j = l$ имаме, че
  \[(q_{i_l+1}, \alpha_{i_l}) \vdash^\star_\A (q_n, \varepsilon).\]
  Понеже, $q_{i_{l+1}} = \qstart$ и $q_n \in F$, то $\alpha_{i_l} \in F$.

  Накрая заключаваме, че $\alpha \in \L(\A)^\star$.
\end{proof}


\begin{example}
  Нека да приложим конструкцията за да намерим автомат разпознаващ $\L(\N)^\star$.
  
  \begin{figure}[H]
    % \center
    \begin{subfigure}[b]{0.3\textwidth}
      \begin{tikzpicture}[framed,->,>=stealth,thick,node distance=45pt]
        \tikzstyle{every state}=[circle,minimum size=20pt,auto]
        \node[initial,state]      (1) {$s$};
        \node[state]              (2) [right of=1] {$q_1$};
        \node[state,accepting]    (3) [right of=2] {$q_2$};
        \node[state,accepting]    (4) [above of=2] {$q_3$};
        \path
        (1) edge node [above] {$a$} (2)
        (1) edge [bend left=15] node [above] {$b$} (4)
        (2) edge node [above] {$b$} (3)
        (3) edge [bend left=45] node [below] {$a$} (1);
      \end{tikzpicture}
      \caption{автомат $\N$}
    \end{subfigure}
    \hspace{2cm}
    \begin{subfigure}[b]{0.5\textwidth}
      \begin{tikzpicture}[framed,->,>=stealth,thick,node distance=45pt]
        \tikzstyle{every state}=[circle,minimum size=20pt,auto]
        % \node[initial above,state,accepting] (0) {$s'$};
        \node[initial, state]                (1) [below right of=0] {$s$};
        \node[state]                         (2) [right of=1] {$q_1$};
        \node[state,accepting]               (3) [right of=2] {$q_2$};
        \node[state,accepting]               (4) [above of=2] {$q_3$};
        \path
        % (0) edge [dashed, bend left=15] node [above] {$a$} (2)
        % (0) edge [dashed, bend left=15] node [above] {$b$} (4)
        (1) edge [bend left=15] node [above] {$b$} (4)
        (1) edge node [above] {$a$} (2)
        (2) edge node [below] {$b$} (3)
        (3) edge [bend left=45] node [below] {$a$} (1)
        (3) edge [dashed, bend right=45] node [above] {$b$} (4)
        (4) edge [dashed] node [left] {$a$} (2)
        (4) edge [dashed, loop above] node {$b$} (4)
        (3) edge [dashed, bend right=45] node [above] {$a$} (2);        
      \end{tikzpicture}
      \caption{$\L(\N_1) = \L(\N)^+$}
    \end{subfigure}
  \end{figure}
    
  \marginpar{Лесно се вижда, че $\L(\N) = \{b\} \cup \{aba\}^\star \cdot ab$}
  След като построим автомат за езика $\L(\N)^+$, трябва да приложим
  конструкцията за обединение на автомата за езика $\L(\N)^+$ с автомата за езика $\{\varepsilon\}$.
  Защо трябва да добавим ново начално състояние $s'$?
  Да допуснем, че вместо това сме направили $s$ финално.
  Тогава има опасност да разпознаем повече думи. Например, думата $aba$ би се разпознала от този автомат,
  но $aba \not\in\L(\N)^\star$.
  
  \begin{figure}[H]
    \centering
    % \begin{subfigure}[b]{0.5\textwidth}
      \begin{tikzpicture}[framed,->,>=stealth,thick,node distance=45pt]
        \tikzstyle{every state}=[circle,minimum size=20pt,auto]
        \node[initial, state, accepting]     (0) {$s'$};
        \node[state]                         (1) [below right of=0] {$s$};
        \node[state]                         (2) [right of=1] {$q_1$};
        \node[state, accepting]              (3) [right of=2] {$q_2$};
        \node[state, accepting]              (4) [above of=2] {$q_3$};
        \path
        (0) edge [dashed, bend left=15] node [above] {$a$} (2)
        (0) edge [dashed, bend left=15] node [above] {$b$} (4)
        (1) edge [bend left=15] node [above] {$b$} (4)
        (1) edge node [above] {$a$} (2)
        (2) edge node [below] {$b$} (3)
        (3) edge [bend left=45] node [below] {$a$} (1)
        (3) edge [dashed, bend right=45] node [above] {$b$} (4)
        (4) edge [dashed] node [left] {$a$} (2)
        (4) edge [dashed, loop above] node {$b$} (4)
        (3) edge [dashed, bend right=45] node [above] {$a$} (2);        
      \end{tikzpicture}
      \caption{$\L(\hat\N) = \L(\N)^\star = \L(\N)^+ \cup \{\varepsilon\}$}    
  \end{figure}

\end{example}



%%% Local Variables:
%%% mode: latex
%%% TeX-master: "../eai"
%%% End:

\newpage
\section{Един критерий за нерегулярност}

\begin{lemma}[Лема за покачването]
  \index{лема за покачването!регулярни езици}
  \label{lem:pumping-reg}
  \marginpar{На англ. се нарича \\ Pumping Lemma}
  \marginpar{Има подобна лема и за безконтекстни езици, която ще разгледаме по-нататък.}
  Нека $L$ да бъде {\em безкраен} регулярен език.
  Съществува число $p\geq 1$, зависещо само от $L$, 
  за което за всяка дума $\alpha\in L, \abs{\alpha}\geq p$ може да 
  бъде записана във вида $\alpha = xyz$ и 
  \begin{enumerate}[1)]
  \item
    $|y|\geq 1$;
  \item
    $|xy|\leq p$;
    \marginpar{Обърнете внимание, че $0 \in \Nat$ и $xy^0z =  xz$}
  \item
    $(\forall i\in\Nat)[xy^iz \in L]$.
  \end{enumerate}
\end{lemma}
\begin{hint}
  \marginpar{Тази лема я има във всеки учебник по този предмет. Например, \cite[стр. 88]{papadimitriou}, \cite[стр. 77]{sipser3}.}
  Понеже $L$ е регулярен, то $L$ е и автоматен език. Нека
  \[\A = \FA\]
  е краен детерминиран автомат, за който $L = \L(\A)$.
  Да положим $p = \abs{Q}$ и нека $\alpha = a_1a_2\cdots a_k$ е дума, за която $k \geq p$.
  Да разгледаме първите $p$ стъпки от изпълнението на $\alpha$ върху $\A$:
  \[\qstart\stackrel{a_1}{\rightarrow} q_1 \stackrel{a_2}{\rightarrow}q_2 \dots \stackrel{a_p}{\rightarrow} q_p.\]
  Тъй като $\abs{Q} = p$, а по този път участват $p+1$ състояния $q_0,q_1,\dots,q_p$,
  то съществуват числа $i, j$, за които $0\leq i < j\leq p$ и $q_i = q_j$.
  Нека разделим думата $\alpha$ на три части по следния начин:
  \[\underbrace{a_1\cdots a_i}_{x}\quad \underbrace{a_{i+1}\cdots a_j}_{y}\quad \underbrace{a_{j+1}\cdots a_k}_{z}.\]
  Ясно е, че $\abs{y} \geq 1$ и $\abs{xy} = j \leq p$.
  Да разгледаме случая за $i = 0$.
  Думата $xy^0z = xz \in L$, защото имаме следното изчисление:
  \[\qstart\underbrace{\stackrel{a_1}{\rightarrow}q_1 \cdots \stackrel{a_i}{\rightarrow}}_{x} q_i\underbrace{\stackrel{a_{j+1}}{\rightarrow}q_{j+1}\cdots\stackrel{a_{k}}{\rightarrow}}_{z}q_k\in F,\]
  защото $q_i = q_j$.
  Да разгледаме и случая $i = 2$. Тогава думата $xy^2z \in L$, защото имаме следното изчисление:
  \[\qstart\underbrace{\stackrel{a_1}{\rightarrow}q_1 \cdots \stackrel{a_i}{\rightarrow}}_{x} q_i\underbrace{\stackrel{a_{i+1}}{\rightarrow}q_{i+1}\cdots\stackrel{a_{j}}{\rightarrow}}_{y}q_j\underbrace{\stackrel{a_{i+1}}{\rightarrow}q_{i+1}\cdots\stackrel{a_{j}}{\rightarrow}}_{y}q_j\underbrace{\stackrel{a_{j+1}}{\rightarrow}\cdots\stackrel{a_{k}}{\rightarrow}}_{z}q_k\in F.\]
  \marginpar{\writedown Докажете!}
  Вече лесно можем да съобразим, че за всяко естествено число $i$, е изпълнено $xy^iz \in \L(\A)$.
\end{hint}

Практически е по-полезно да разглеждаме следната еквивалентна формулировка на лемата за покачването.

\begin{cor}[Контрапозиция на лемата за покачването]
  \label{cor:pumping-reg}
  \marginpar{Ясно е, че всеки краен език е регулярен. Нали?}
  Нека $L$ е произволен {\bf безкраен} език. Нека също така е изпълнено, че:
  \begin{description}
  \item[($\forall$)]
    за {\em всяко} естествено число $p \geq 1$,
  \item[($\exists$)]
    можем да намерим дума $\alpha \in L$, $\abs{\alpha}\geq p$, такава че
  \item[($\forall$)]
    за {\em всяко} разбиване на думата на три части, $\alpha = xyz$, със свойствата $\abs{xy} \leq p$ и $\abs{y} \geq 1$,
  \item[($\exists$)]
    можем да посочим $i \in \Nat$, за което е изпълнено, че $xy^iz \not\in L$.
  \end{description}  
  Тогава $L$ {\bf не} е регулярен език.
\end{cor}
\begin{proof}
  Да означим с $P_{\text{reg}}(L)$ следната формула:
  \begin{align*}
    (\exists p \geq 1)(\forall \alpha \in L)[\abs{\alpha} \geq p \Rightarrow (\exists x,y,z\in\Sigma^\star)[\ & \alpha = xyz\ \& \\
                                                                                                              & \abs{y} \geq 1\ \&\\
                                                                                                              & \abs{xy} \leq p\ \&\\
                                                                                                              & (\forall i\in\Nat)[xy^iz \in L]]].
  \end{align*}
  Така условието на \hyperref[lem:pumping-reg]{Лемата за покачването} представлява твърдението:
  \begin{center}
    {\em ,,Aко $L$ е регулярен език, то е изпълнено $P_{\text{reg}}(L)$.''}
  \end{center}
  \marginpar{Контрапозиция на твърдението $P \to Q$ е твърдението $\neg Q \to \neg P$}
  \noindent
  Лемата може да се запише по следния еквивалентен начин:
  
  \begin{center}
    {\em ,,Ако $P_{\text{reg}}(L)$ не е изпълнено, то $L$ не е регулярен език.''}
  \end{center}
  \marginpar{Използваме, че $\neg \exists \forall \exists \forall (\dots) \equiv \forall \exists \forall \exists \neg(\dots)$}
  Отрицанието на $P_{\text{reg}}(L)$ преставлява формулата
  \begin{align*}
    (\forall p \geq 1)(\exists \alpha \in L)[\ \abs{\alpha} \geq p\ \&\ (\forall x,y,z\in\Sigma^\star)[\ & \alpha \neq xyz\ \vee\\
                                                                                                         & \abs{y} \not\geq 1\ \vee\\
                                                                                                         & \abs{xy} \not\leq p\ \vee\\
                                                                                                         & (\exists i\in\Nat)[xy^iz \not\in L]\ ]\ ].
  \end{align*}
  Горната формула е еквивалентна на:
  \marginpar{Използваме, че
    \begin{align*}
      \neg P \vee \neg Q \vee R & \equiv \neg(P\ \&\ Q) \vee R\\
                                & \equiv (P \wedge Q) \to R
    \end{align*}}
  \begin{align*}
    (\forall p \geq 1)(\exists \alpha \in L)[\ \abs{\alpha} \geq p\ \&\ (\forall x,y,z\in\Sigma^\star)[\ & (\alpha = xyz \wedge \abs{y} \geq 1\wedge \abs{xy} \leq p)\\
                                                                                                         & \Rightarrow (\exists i\in\Nat)[xy^iz \not\in L]\ ]\ ].
  \end{align*}

  Това означава, че ако
  \begin{description}
  \item[($\forall$)]
    вземем произволна константа $p \geq 1$,
  \item[($\exists$)]
    за нея намерим дума $\alpha \in L$, такава че $\abs{\alpha} \geq p$ и 
  \item[($\forall$)]
    докажем, че за всяко нейно разбиване на три части $x,y,z$, със свойствата
    $\abs{y} \geq 1$ и $\abs{xy} \leq p$,
  \item[($\exists$)]
    можем да намерим естествено число $i$, за което $xy^iz \not\in L$,
  \end{description}
  то можем да заключим, че езикът $L$ не е регулярен.
\end{proof}

%%% Local Variables:
%%% mode: latex
%%% TeX-master: "../eai"
%%% End:

\newpage
\section{Релация на Майхил-Нероуд}

\begin{itemize}
\item
  \index{Майхил-Нероуд!релация}
  \index{$\approx_L$}
  \mynote{$\approx_L$ е известна като релация на Майхил-Нероуд}
  Нека $L \subseteq \Sigma^\star$ е език и нека $\alpha,\beta \in \Sigma^\star$.
  Казваме, че $\alpha$ и $\beta$ са {\bf еквивалентни относно} $L$, което записваме 
  като $\alpha \approx_L \beta$, когато:
  \[\alpha \approx_L \beta\ \dff\ \alpha^{-1}(L) = \beta^{-1}(L).\]
  С други думи, 
  \[\alpha \approx_L \beta \iff (\forall \omega \in \Sigma^\star)[\ \alpha\omega \in L \iff \beta\omega \in L\ ].\]
\item
  \mynote{Трябва ли $\A$ да е тотален?}
  Нека $\A = \FA$ е краен детерминиран автомат.
  \index{$\sim_\A$}
  Казваме, че две думи $\alpha,\beta \in \Sigma^\star$ са {\bf еквивалентни относно $\A$},
  което означаваме с $\alpha \sim_\A \beta$, ако 
  \[\delta^\star(\qstart,\alpha) = \delta^\star(\qstart,\beta).\]
\item
  Проверете, че $\approx_L$ и $\sim_\A$ са {\bf релации на еквивалентност}, т.е.
  те са рефлексивни, транзитивни и симетрични.
\item
  Класът на еквивалентност на думата $\alpha$ относно релацията $\approx_L$ означаваме като
  \[[\alpha]_L \df \{\beta \in \Sigma^\star \mid \alpha \approx_L \beta\}.\]
  Означаваме 
  \[\Sigma^\star/_{\approx_L} \df \{[\alpha]_L \mid \alpha \in \Sigma^\star\}.\]
  Тогава с $\abs{\Sigma^\star/_{\approx_L}}$ ще означаваме броя на класовете на еквивалентност на релацията $\approx_L$.
\item
  Класът на еквивалентност на думата $\alpha$ относно релацията $\sim_\A$ означаваме като
  \[[\alpha]_\A \df \{\beta \in \Sigma^\star \mid \alpha \sim_\A \beta\}.\]
  Означаваме 
  \[\Sigma^\star/_{\sim_\A} \df \{[\alpha]_\A \mid \alpha \in \Sigma^\star\}.\]
  С $\abs{\Sigma^\star/_{\sim_\A}}$ ще означаваме броя на класовете на еквивалентност на релацията $\sim_\A$.
\item
  Съобразете, че всяко състояние на $\A$, което е достижимо от началното състояние, определя клас на еквивалентност относно 
  релацията $\sim_\A$. Това означава, че функцията $g:\Sigma^\star/_\A \to Q$, където
  \[g([\alpha]_\A) \df \delta^\star(\qstart,\alpha)\]
  е инекция. Следователно,
  \[|\Sigma^\star/_{\sim_\A}| \leq |Q|.\]
  Ако в автомата $\A$ няма недостижими от $\qstart$ състояния, то $g$ е биекция и съответно
  \[|\Sigma^\star/_{\sim_\A}| = |Q|.\]
\item
  Релациите $\approx_\L$ и $\sim_\A$ са дясно-инвариантни, т.е. за всеки две думи $\alpha$ и $\beta$
  е изпълнено:
  \mynote{\writedown Проверете!}
  \begin{align*}
    \alpha \sim_\A \beta  &\implies (\forall \gamma\in\Sigma^\star)[\alpha\gamma \sim_\A \beta\gamma],\\
    \alpha \approx_\L \beta & \implies (\forall \gamma\in\Sigma^\star)[\alpha\gamma \approx_\L \beta\gamma].
  \end{align*}
\end{itemize}

\begin{problem}
  Докажете, че за всяка дума $\alpha \in \Sigma^\star$ е изпълнено, че:
  \[\alpha \in L \iff [\alpha]_L \subseteq L.\]
\end{problem}

\begin{proposition}
  \label{pr:rel-finer}
  \mynote{С други думи,\\ $[\alpha]_\A \subseteq [\alpha]_L$.}
  Нека $\A = \FA$ е краен детерминиран автомат и $L = \L(\A)$. Тогава е изпълнено, че:
  \[(\forall \alpha,\beta \in \Sigma^\star)[\ \alpha\sim_\A\beta \implies \alpha\approx_{L}\beta\ ].\]
\end{proposition}
\begin{proof}
  Да вземем две произволни думи $\alpha$ и $\beta$, за които $\alpha \sim_\A \beta$, т.е. $\delta^\star(\qstart, \alpha) = \delta^\star(\qstart,\beta)$.
  Ще проверим, че  $\alpha \approx_{L} \beta$, т.е. $\alpha^{-1}(L) = \beta^{-1}(L)$.
  За произволна дума $\gamma$ имаме следното:
  \begin{align*}
    \gamma \in \alpha^{-1}(L) & \iff \alpha\gamma \in L\\
                              % & \iff \alpha\gamma \in \L(\A) & \comment{L = \L(\A)}\\
                              & \iff \delta^\star(\qstart,\alpha\gamma)\in F & \comment{L = \L(\A)}\\
                              & \iff \delta^\star(\delta^\star(\qstart,\alpha),\gamma) \in F & \comment{\text{деф. на }\delta^\star}\\
                              & \iff \delta^\star(\delta^\star(\qstart,\beta),\gamma) \in F & \comment{\text{защото }\alpha \sim_\A \beta}\\
                              & \iff \delta^\star(\qstart,\beta\gamma) \in F & \comment{\text{свойство на }\delta^\star}\\
                              & \iff \beta\gamma \in \L(\A) & \comment\text{деф. на }\L(\A)\\
                              % & \iff \beta\gamma \in L & \comment{L = \L(\A)} \\
                              & \iff \gamma \in \beta^{-1}(L).
  \end{align*}
  Заключаваме, че 
  \[(\forall \alpha,\beta \in \Sigma^\star)[\ \alpha\sim_\A\beta \implies \alpha\approx_{L}\beta\ ].\]
\end{proof}

\begin{problem}
  Докажете, че за ДКА $\A$ и $L = \L(\A)$,
  за всяка дума $\alpha$,
  \[[\alpha]_{L} = \bigcup_{\beta \in [\alpha]_{L}}[\beta]_\A.\]
\end{problem}
% \begin{hint}
%   Първо, нека $\gamma \in \bigcup_{\beta \in [\alpha]_{L}}[\beta]_\A$, т.е. за някое $\beta \in [\alpha]_L$ имаме, че $\gamma \in [\beta]_\A$.
%   Но щом $\beta \in [\alpha]_L$, то тогава $[\beta]_L = [\alpha]_L$, а от $\gamma \in [\beta]_\A$, по \Prop{rel-finer} следва, че $\gamma \in [\beta]_L = [\alpha]_L$.
%   Заключаваме, че $\bigcup_{\beta \in [\alpha]_{L}}[\beta]_\A \subseteq [\alpha]_L$.

%   Нека сега $\gamma \in [\alpha]_L$. Но тогава е ясно, че $\gamma \in [\gamma]_\A$ и следователно директно получаваме, че
%   $[\alpha]_L \subseteq \bigcup_{\beta \in [\alpha]_{L}}[\beta]_\A$.
% \end{hint}

\begin{problem}
  \mynote{Тук говорим за тотални функции.}
  Нека $A$ и $B$ са крайни множества, за които съществува сюрективна функция $f: A \to B$.
  Тогава $|B| \leq |A|$.
\end{problem}

\begin{problem}
  Нека $A$ и $B$ са крайни множества, за които $|A| = |B|$.
  Ако $g:A \to B$ е сюрекция, то $g$ е биекция.
\end{problem}


\begin{proposition}
  \label{pr:approx-less-sim}
  Нека $\A$ е детерминиран краен автомат и $L = \L(\A)$.
  Тогава 
  \[\abs{\Sigma^\star/_{\approx_{L}}} \leq \abs{\Sigma^\star/_{\sim_\A}}.\]
\end{proposition}
\begin{hint}
  Да разгледаме функцията $f: \Sigma^\star/_{\sim_\A} \to \Sigma^\star/_{\approx_L}$, където
  \[f([\alpha]_\A) \df [\alpha]_L.\]
  Директно от \Proposition{rel-finer} се съобразява, че
  \[(\forall \alpha,\beta \in \Sigma^\star)[\ [\alpha]_\A = [\beta]_\A\ \implies f([\alpha]_\A) = f([\beta]_\A)\ ],\]
  откъдето следва, че $f$ е добре дефинирана.
  Ясно е, че $f$ е сюрективна функция.
  Оттук следва, че
  \[\abs{\Sigma^\star/_{\approx_{L}}} \leq \abs{\Sigma^\star/_{\sim_\A}}.\]
\end{hint}

\begin{framed}
  \begin{proposition}
    \label{pr:upper-bound}
    Нека $L$ е произволен регулярен език.
    Всеки краен детерминиран автомат $\A$, за който $L = \L(\A)$ има свойството
    \[\abs{\Sigma^\star/_{\approx_L}} \leq \abs{Q},\]
    т.е. броят на класовете на еквивалентност на релацията $\approx_L$
    не надвишава броя на състоянията на автомата $\A$.
  \end{proposition}  
\end{framed}
\begin{proof}
  Да изберем $\A$, който разпознава $L$. 
  Тъй като всяко достижимо състояние на $\A$ определя клас на еквивалентност относно $\sim_\A$,
  то получаваме, че $\abs{\Sigma^\star/_{\sim_\A}} \leq |Q|$.
  Комбинирайки със \Proposition{approx-less-sim},
  \[\abs{\Sigma^\star/_{\approx_L}} \leq \abs{\Sigma^\star/_{\sim_\A}} \leq \abs{Q}.\]
\end{proof}

Така получаваме {\em долна граница} за броя на състоянията в краен детерминистичен автомат разпознаващ езика $L$.
Този брой е не по-малък от броя на класовете на еквивалентност на $\approx_L$.
В следващия раздел ще видим, че тази долна граница може да бъде достигната.

\begin{framed}
  \begin{proposition}
    Езикът $L$ е регулярен точно тогава, когато релацията $\approx_L$ има {\em крайно много} класове на еквивалентност.
  \end{proposition}  
\end{framed}
\begin{proof}
  Ако $L$ е регулярен, то той се разпознава от някой ДКА $\A$, който има крайно много състояния 
  и следователно крайно много класове на еквивалентност относно $\sim_\A$.
  Релацията $\approx_L$ е по-груба от $\sim_\A$ и има по-малко класове на еквивалентност.
  Следователно, $\approx_L$ има крайно много класове на еквивалентност.
  
  За другата посока, ако $\approx_L$ има крайно много класове на еквивалентност, то можем да 
  построим ДКА $\A$ както в доказателството на \hyperref[th:myhill-nerode]{Теоремата на Майхил-Нероуд}, който разпознава $L$.
\end{proof}

Това следствие ни дава още един начин за проверка дали даден език е регулярен.
За разлика от \Lemma{pumping-reg}, сега имаме {\bf необходимо и достатъчно условие}.
При даден език $L$, ние разглеждаме неговата релация $\approx_L$.
Ако тя има крайно много класове, то езикът $L$ е регулярен.
В противен случай, езикът $L$ не е регулярен.

Ще завършим с едно твърдение, което ще ни бъде полезно по-нататък, когато искаме да докажем, че за всеки регулярен език съществува
единствен минимален краен детерминистичен автомат.
\begin{proposition}
  \label{pr:bijection-classes}
  Нека $L$ е произволен регулярен език и $\A$ е краен детерминиран автомат, за който $L = \L(\A)$.
  Ако $|Q| = |\Sigma^\star/_{\approx_L}|$, то функцията $h:\Sigma^\star/_{\approx_L} \to Q$, където
  \[h([\alpha]_L) \df \delta^\star(\qstart,\alpha),\]
  е биекция.
\end{proposition}
\begin{proof}
  Имаме, че:
  \[|Q| = |\Sigma^\star/_{\approx_L}| \leq |\Sigma^\star/_{\sim_\A}| \leq |Q|,\]
  откъдето следва, че
  \[|\Sigma^\star/_{\approx_L}| = |\Sigma^\star/_{\sim_\A}| = |Q|.\]
  Това означава, че функцията $g:\Sigma^\star/_{\sim_\A} \to Q$, където
  \[g([\alpha]_\A) \df \delta^\star(\qstart,\alpha)\] е биекция,
  защото $|\Sigma^\star/_{\sim_\A}| = |Q|$.
  От \Proposition{approx-less-sim} имаме, че функцията $f:\Sigma^\star/_{\sim_\A} \to \Sigma^\star/_{\approx_L}$, където
  \[f([\alpha]_\A) \df [\alpha]_L\] е биекция,
  защото знаем, че $f$ е сюрекция и $|\Sigma^\star/_{\approx_L}| = |\Sigma^\star/_{\sim_\A}|$.
  Оттук заключаваме, че $h = g \circ f^{-1}$ е биекция, защото е композиция на две биекции.
\end{proof}

%%% Local Variables:
%%% mode: latex
%%% TeX-master: "../eai"
%%% End:

\section{Автомат на Майхил-Нероуд}\label{sect:myhill-nerode-theorem}
\index{Майхил-Нероуд}
\mynote{на англ. Myhill-Nerode}

\index{Майхил-Нероуд!релация}
\index{$\approx_L$}

Нека $L$ е език и нека $\alpha$ и $\beta$ са думи.
Казваме, че $\alpha$ и $\beta$ са {\bf еквивалентни относно} $L$, което записваме 
като $\alpha \approx_L \beta$, когато:
\[\alpha \approx_L \beta\ \dff\ \alpha^{-1}(L) = \beta^{-1}(L).\]
С други думи, 
\[\alpha \approx_L \beta \iff (\forall \omega \in \Sigma^\star)[\ \alpha\omega \in L \iff \beta\omega \in L\ ].\]
\mynote{$\approx_L$ е известна като релация на Майхил-Нероуд}
\begin{problem}
  Докажете, че за всяка дума $\alpha$ е изпълнено, че:
  \[\alpha \in L \iff [\alpha]_L \subseteq L.\]
\end{problem}


\mynote{Ще наричаме $\M$ автомат на Майхил-Нероуд. На практика във всеки учебник се разглежда автомата на Майхил-Нероуд вместо автомата на Бжозовски. Например, \cite[стр. 98]{papadimitriou}, \cite[стр. 65]{hopcroft1}, \cite[стр. 91]{sipser3}, \cite[стр. 89]{kozen}}


Ще дефинираме детерминиран автомат $\M = \FA$ по следния начин:
\begin{itemize}
\item
  $Q \df \{\ [\alpha]_L\mid \alpha\in \Sigma^\star\ \}$;
\item
  $\qstart \df [\varepsilon]_L$;
\item
  $F \df \{\ [\alpha]_L\mid \alpha\in L\ \}$;
\item
  Определяме функцията на преходите $\delta$ като 
  за всяка буква $b$ и всяка дума $\alpha$,
  \[\delta([\alpha]_L,b) \df [\alpha b]_L.\]
\end{itemize}

\begin{problem}
  Докажете, че $\delta:Q^\M \times \Sigma \to Q^\M$ е функция.
\end{problem}
\begin{hint}
  Трябва да докажете, че
  \[[\alpha]_L = [\beta]_L \implies \delta([\alpha]_L,b) = \delta([\beta]_L,b).\]
\end{hint}

\begin{problem}\label{prob:myhill-nerode-theorem:language}
  Докажете, че ако $\M$ е автоматът на Майхил-Нероуд за езика $L$, то $\L(\M) = L$.
\end{problem}
\begin{hint}
  Докажете, че за всеки две думи $\alpha$ и $\beta$ е изпълнено, че:
  \[\delta^\star_\M([\alpha]_L,\beta) = [\alpha\beta]_L.\]
  Оттук заключете директно, че $\L(\M) = L$.
\end{hint}

\begin{problem}\label{prob:myhill-nerode-theorem:bijection}
  \mynote{Обърнете внимание, че тук не изискваме $L$ да е регулярен.}
  Да разгледаме един език $L$.
  Нека $\B$ е автоматът на Бжозовски за $L$ и $\M$ е автоматът на Майхил-Нероуд за $L$.
  Да разгледаме $f:Q^\M \to Q^\B$, където:
  \[f([\alpha]_L) = \hat{K} \dff K = \alpha^{-1}(L).\]
  Докажете, че $f$ е биекция.
\end{problem}

\mynote{Теоремата е доказана независимо от Майхил \cite{myhill-min} и Нероуд \cite{nerode-min}.}
\begin{problem}[Теорема на Майхил-Нероуд]\label{prob:myhill-nerode-theorem}
  Докажете, че $L$ е регулярен език точно тогава, когато автоматът на Майхил-Нероуд $\M$ e краен.
  Освен това, докажете, че $\M$ е минимален ДКА за $L$.
\end{problem}


\begin{example}
  Да разгледаме езика 
  \[L = \L(\mathbf{a\cdot(a+b)^\star\cdot b}).\]
  Да напомним, че в \Example{brzozowski-solved-examples-2} вече видяхме автоматът на Бжозовски за $L$.
  \begin{figure}[H]
    \centering
    \begin{tikzpicture}[framed,->,>=stealth,thick,node distance=55pt]
      \tikzstyle{every state}=[circle]
      
      \node[initial, state]                   (L) {$\hat{L}$};
      \node[state]                            (M) [above right of=L]{$\hat{M}$};
      \node[state]                            (E) [below right of=L]{$\hat{\emptyset}$};
      \node[state,accepting]                  (N) [right of=M]{$\hat{N}$};
      
      \path 
      (L) edge [bend left=15]  node [left] {$a$} (M)
      (L) edge [bend right=15] node [left] {$b$} (E)
      (E) edge [loop right]    node [right] {$a,b$} (E) 
      (M) edge [bend right=15] node [below] {$b$} (N)
      (M) edge [loop above]    node [above] {$a$} (M)
      (N) edge [bend right=30] node [above] {$a$} (M)
      (N) edge [loop above]    node [above] {$b$} (N);
    \end{tikzpicture}
    \caption{Автомат за езика $\L(\mathbf{a\cdot (a+b)^\star\cdot b})$ по метода на Бжозовски.}
  \end{figure}

  Вече знаем, че този автомат е минимален за езика $L$.
  Да видим сега как директно можем да построим автомата на Майхил-Нероуд $\M$ за езика $L$ като използваме, че $\B \cong \M$.


    \begin{figure}[H]
    \centering
    \begin{tikzpicture}[framed,->,>=stealth,thick,node distance=55pt]
      \tikzstyle{every state}=[circle]
      
      \node[initial, state]                   (L) {$[\varepsilon]_L$};
      \node[state]                            (M) [above right of=L]{$[a]_L$};
      \node[state]                            (E) [below right of=L]{$[b]_L$};
      \node[state,accepting]                  (N) [right of=M]{$[ab]_L$};
      
      \path 
      (L) edge [bend left=15]  node [left] {$a$} (M)
      (L) edge [bend right=15] node [left] {$b$} (E)
      (E) edge [loop right]    node [right] {$a,b$} (E) 
      (M) edge [bend right=15] node [below] {$b$} (N)
      (M) edge [loop above]    node [above] {$a$} (M)
      (N) edge [bend right=30] node [above] {$a$} (M)
      (N) edge [loop above]    node [above] {$b$} (N);
    \end{tikzpicture}
    \caption{Автомат за езика $\L(\mathbf{a\cdot (a+b)^\star\cdot b})$ по метода на Майхил-Нероуд.}
  \end{figure}
\end{example}


\begin{example}
  Нека сега да приемем, че не знаем минималния автомат за езика $L$, а имаме следния автомат $\A$ за $L$:
  
      \begin{figure}[H]
    \centering
    \begin{tikzpicture}[framed,->,>=stealth,thick,node distance=55pt]
      \tikzstyle{every state}=[circle]
      
      \node[initial, state]                   (L) {$q_0$};
      \node[state]                            (M) [above right of=L]{$q_1$};
      \node[state]                            (MM) [above right of=M]{$q_2$};
      \node[state]                            (E) [below right of=L]{$q_3$};
      \node[state]                            (EE)[right of=E]{$q_4$};
      \node[state,accepting]                  (N) [below right of=MM]{$q_5$};
      
      \path 
      (L) edge [bend left=15]  node [left] {$a$} (M)
      (L) edge [bend right=15] node [left] {$b$} (E)
      (E) edge [loop below]    node [below] {$a$} (E) 
      (M) edge [bend right=15] node [below] {$b$} (N)
      (M) edge [bend left=15] node [above] {$a$} (MM)
      (MM) edge [loop above]    node [above] {$a$} (MM)
      (MM) edge [bend right=15] node [left] {$b$} (N)
      (N) edge [bend right=30] node [above] {$a$} (MM)
      (N) edge [loop right]    node [right] {$b$} (N)
      (E) edge [bend left=15] node [above] {$b$} (EE)
      (EE) edge [bend left=15] node [below] {$a$} (E)
      (EE) edge [loop right]    node [right] {$b$} (EE); 
    \end{tikzpicture}

  \end{figure}
  

  Веднага се вижда, че $\L_\A(q_3) = \L_\A(q_4)$.

\end{example}




%%% Local Variables:
%%% mode: latex
%%% TeX-master: "../eai"
%%% End:

\subsection{Алгоритъм за строене на минимален автомат}
\begin{itemize}
\item
  Да фиксираме произволен тотален ДКА $\A = \FA$.
\item
  За състояние $p$ в автомата $\A$, да означим с $\L_\A(p)$ езикът, който се разпознава от автомата $\A$,
  ако приемем, че $p$ е началното състояние на автомата, т.е.
  \[\L_\A(p) \df \{\omega \in \Sigma^\star \mid \delta^\star(p,\omega) \in F\}.\]
  В частност, $\L(\A) = \L_\A(\qstart)$.
\item
  Сега дефинираме следната релация между състояния на автомата $\A$:
  \[p \equiv_\A q\ \dff\ \L_\A(p) = \L_\A(q).\]
  Това означава, че $p \equiv_\A q$ точно тогава, когато
  \begin{equation}
    \label{eq:1}
    (\forall \omega\in \Sigma^\star)[\delta^\star(p,\omega) \in F\ \iff\ \delta^\star(q,\omega) \in F].
  \end{equation}
\item
  Релацията $\equiv_\A$ между състояния на автомата $\A$ е релация на еквивалентност. 
\item
  Нека $q_\alpha$ е състоянието, което съответства на думата $\alpha$ в $\A$, т.е.
  $\delta^\star_\A(s,\alpha) = q_\alpha$. Тогава 
  \[\L_\A(q_\alpha) = \alpha^{-1}\L(\A).\]
  Оттук получаваме, че 
  \begin{align*}
    q_\alpha \equiv_\A q_\beta & \iff \L_\A(q_\alpha) = \L_\A(q_\beta)\\
    & \iff \alpha^{-1}\L(\A) = \beta^{-1}\L(\A)\\
    & \iff \alpha \sim_{\L(\A)} \beta.
  \end{align*}
  % \[q_\alpha \equiv_\A q_\beta\ \iff\ \alpha\approx_{\L(\A)} \beta.\]
  Това означава, че ако в $\A$ няма недостижими състояния от началното състояние $s$, то 
  \[\abs{\Sigma^\star/_{\equiv_\A}} = \abs{\Sigma^\star/_{\approx_{\L(\A)}}}.\]
\end{itemize}

При даден език $L$ и тотален ДКА $\A = \FA$, който го разпознава, нашата цел е да построим нов ДКА $\A_0$,
който има толкова състояния колкото са класовете на еквивалентност на релацията $\approx_\L$.
Това ще направим като ``слеем'' състоянията на $\A$, които са еквивалентни относно релацията $\equiv_\A$.
Това означава, че всяко състояние на $\A_0$ ще отговаря на един клас на еквивалентност на релацията $\equiv_\A$.
Проблемът с намирането на класовете на еквивалентност на релацията $\equiv_\A$ е кванторът $\forall \gamma \in \Sigma^\star$
във нейната дефиниция чрез (Формула \ref{eq:1}), защото $\Sigma^\star$ е безкрайно множество от думи.

Да фиксираме автомата $\A$ и $L = \L(\A)$.
Да означим 
\[\L^n_\A(p) \df \{\omega \in \Sigma^\star \mid \abs{\omega} \leq n\ \&\ \delta^\star(p,\omega) \in F\}.\]
Според тази дефиниция, $L = \bigcup_{n\geq 0} \L^n_\A(\qstart)$.

За всяко естествено число $n$, дефинираме бинарните релации $\equiv_n$ върху $Q$ по следния начин:
\[p \equiv_n q \dff \L^n_\A(p) = \L^n_\A(q).\]

% Алгоритъмът представлява намирането на релации $\equiv_n$, където
% \[p\equiv_n q \iff (\forall\gamma\in\Sigma^\star)[\abs{\gamma}\leq n\ \rightarrow\ (\delta^\star(p,\gamma) \in F\ \iff\ \delta^\star(q,\gamma) \in F)].\]
Релациите $\equiv_n$ представляват апроксимации на релацията $\equiv_\A$.
Обърнете внимание, че за всяко $n$, $\equiv_n$ е {\em по-груба} релация от $\equiv_{n+1}$, 
която на свой ред е по-груба от $\equiv_\A$.
Алгоритъмът строи $\equiv_n$ докато не срещнем $n$, за което $\equiv_n\ =\ \equiv_{n+1}$.
Тъй като броят на класовете на еквивалентност на $\equiv_\A$ е краен (той е $\leq \abs{Q}$), то 
със сигурност ще намерим такова $n$, за което $\equiv_n\ =\ \equiv_{n+1}$.
Тогава заключаваме, че $\equiv_\A\ =\ \equiv_n$.

Понеже единствената дума с дължина $0$ e $\varepsilon$ и по определение $\delta^\star(p,\varepsilon) = p$, 
лесно се съобразява, че $\equiv_0$ има два класа на еквивалентност.
Единият е $F$, а другият е $Q\setminus F$.

\begin{prop}
  \label{pr:one-letter-test}
  За всеки две състояния $p,q \in Q$, и всяко $n$, $p \equiv_{n+1} q$ точно тогава, когато
  \begin{enumerate}[a)]
  \item
    $p \equiv_{n} q$ и
  \item
    $(\forall a \in \Sigma)[\delta(q,a) \equiv_{n} \delta(p,a)]$.
  \end{enumerate}
\end{prop}
\begin{hint}
  \marginpar{\cite[стр. 99]{papadimitriou}}
  \begin{align*}
    p \equiv_{n+1} q & \iff \L^{n+1}_\A(p) = \L^{n+1}_\A(q)\\
    & \iff \L^n_\A(p) = \L^n_\A(q)\ \&\ (\forall a \in \Sigma)[\L^n_\A(\delta(p,a)) = \L^n_\A(\delta(q,a))]\\
    & \iff p \equiv_n q\ \&\ (\forall a \in \Sigma)[\delta(p,a) \equiv_n \delta(q,a)].
  \end{align*}
\end{hint}

Нека е даден автомата $A = \FA$.
След като сме намерили релацията $\equiv_\A$ за $\A$, 
строим автомата $\A' = (Q',\Sigma,s',\delta',F')$, където:
\begin{itemize}
\item
  $Q' = \{[q]_{\equiv_\A} \mid q\in Q\}$;
\item
  $s' = [s]_{\equiv_\A}$;
\item
  $\delta'([q]_{\equiv_\A}, a) = [\delta(q,a)]_{\equiv_\A}$;
\item
  $F' = \{[q]_{\equiv_\A}\mid F\cap [q]_{\equiv_\A} \neq \emptyset\}$;
\end{itemize}

От всичко казано дотук знаем, че $\A'$ е минимален автомат разпознаващ езика $\L(\A)$.

\begin{example}
  Да разгледаме следния краен детерминиран автомат $\A$.
  \begin{figure}[H]
    \begin{subfigure}[b]{.4\textwidth}
      \begin{tikzpicture}[->,>=stealth,thick,node distance=55pt]
        \tikzstyle{every state}=[circle,minimum size=20pt,auto]
        
        \node[initial above, state]   (0) {$0$};
        \node[state]            (1) [above right of=0]{$1$};
        \node[state]            (2) [below right of=0]{$2$};
        \node[state,accepting]  (3) [right of=1]{$3$};
        \node[state,accepting]  (4) [right of=2]{$4$};
        \node[state,accepting]  (5) [below right of=3]{$5$};
        
        
        \path 
        (0) edge  node [above] {$a$}   (1)
        (0) edge  node [below] {$b$}   (2)
        (1) edge node [above] {$a$}    (3)
        (1) edge [bend left=15] node [below] {$b$}    (4)
        (2) edge [bend left=15] node [left] {$b$}    (3)
        (2) edge node [below] {$a$}   (4)
        (4) edge  node [below] {$a,b$} (5)
        (3) edge  node [left] {$a,b$}  (5)
        (5) edge [loop above]   node [above] {$a,b$}  (5);
      \end{tikzpicture}
      \caption{Ще построим минимален автомат разпознаващ $\L(\A)$}
    \end{subfigure}
    \qquad
    \qquad
    \begin{subfigure}[b]{0.5\textwidth}
      \begin{tikzpicture}[->,>=stealth,thick,node distance=45pt]
        \tikzstyle{every state}=[circle,minimum size=20pt,auto,scale=.9]
        
        \node[initial above, state]   (0) {$B_0$};
        \node[state]            (1) [right of=0]{$B_1$};
        \node[state,accepting]  (2) [right of=1]{$B_2$};
        
        \path 
        (0) edge [bend left=15] node [above] {$a,b$}   (1)
        (1) edge [bend left=15] node [above] {$a,b$}   (2)
        (2) edge [loop above] node [above] {$a,b$}   (2);
      \end{tikzpicture}
      \caption{Получаваме следния минимален автомат $\A_0$, $\L(\A_0) = \L(\A)$}
      \label{sub:min1}
    \end{subfigure}
  \end{figure}
  \marginpar{Съобразете, че $\L(\A) = \{\alpha \in \{a,b\}^\star \mid \abs{\alpha} \geq 2\}$.}

Ще приложим алгоритъма за минимизация за да получим минималния автомат за езика $L$.
За всяко $n = 0,1,2,\dots$, ще намерим класовете на еквивалентност на $\equiv_n$,
докато не намерим $n$, за което $\equiv_n\ =\ \equiv_{n+1}$.

\begin{itemize}
\item 
  Класовете на еквивалентност на $\equiv_0$ са два.
  Те са $A_0 = Q\setminus F = \{0,1,2\}$ и $A_1 = F = \{3,4,5\}$.
\item
  Сега да видим дали можем да разбием някои от класовете на еквивалентност на $\equiv_0$.
  
  \begin{tabular}{|c|c|c|c|c|c|c|}
    \hline
    $Q$ & $0$ & $1$ & $2$ & $3^\star$ & $4^\star$ & $5^\star$ \\
    \hline
    \hline
    $\equiv_0$ & $A_0$ & $A_0$ & $A_0$ & $A_1$ & $A_1$ & $A_1$\\
    \hline
    $a$ & $A_0$& $A_1$ & $A_1$ & $A_1$ & $A_1$ & $A_1$\\
    \hline
    $b$ & $A_0$& $A_1$ & $A_1$ & $A_1$ & $A_1$ & $A_1$\\
    \hline
  \end{tabular}

  Виждаме, че $0 \not\equiv_1 1$ и $1 \equiv_1 2$.
  Класовете на еквивалентност на $\equiv_1$ са 
  $B_0 = \{0\}$, $B_1 = \{1,2\}$, $B_2 = \{3,4,5\}$.
\item
  Сега да видим дали можем да разбием някои от класовете на еквивалентност на $\equiv_1$.
  
  \begin{tabular}{|c|c|c|c|c|c|c|}
    \hline
    $Q$ & $0$ & $1$ & $2$ & $3^\star$ & $4^\star$ & $5^\star$ \\
    \hline
    \hline
    $\equiv_1$ & $B_0$ & $B_1$ & $B_1$ & $B_2$ & $B_2$ & $B_2$\\
    \hline
    $a$ & $B_1$ & $B_2$ & $B_2$ & $B_2$ & $B_2$ & $B_2$\\
    \hline
    $b$ & $B_1$ & $B_2$ & $B_2$ & $B_2$ & $B_2$ & $B_2$\\
    \hline
  \end{tabular}

  Виждаме, че $\equiv_1\ =\ \equiv_2$.
  \marginpar{Получаваме, че $\equiv_\A\ =\ \equiv_1$}
  Следователно, минималният автомат има три състояния.
  Той е изобразен на Фигура \ref{sub:min1}.  
  Минималният автомат може да се представи и таблично:
  
  \begin{tabular}{|c|c|c|c|c|c|c|}
    % \hline
    % $Q$ & $0$ & $1$ & $2$ & $3^\star$ & $4^\star$ & $5^\star$ \\
    % \hline
    \hline
    $\delta$ & $B_0$ & $B_1$ & $B_2$ \\
    \hline
    $a$ & $B_1$ & $B_2$ & $B_2$ \\
    \hline
    $b$ & $B_1$ & $B_2$ & $B_2$ \\
    \hline
  \end{tabular}
\end{itemize}
\end{example}

\begin{example}
  Да разгледаме следния краен детерминиран автомат $\A$.
  \begin{figure}[H]
    % \begin{center}
    \begin{subfigure}[b]{0.4\textwidth}
      \begin{tikzpicture}[->,>=stealth,thick,node distance=55pt]
        \tikzstyle{every state}=[circle,minimum size=20pt,auto]
        
        \node[initial above, state]   (0) {$0$};
        \node[state,accepting]        (1) [above right of=0]{$1$};
        \node[state,accepting]        (2) [below right of=0]{$2$};
        \node[state]                  (3) [right of=1]{$3$};
        \node[state]                  (4) [right of=2]{$4$};
        \node[state,accepting]        (5) [below right of=3]{$5$};
        
        \path 
        (0) edge node [below] {$a$}   (1)
            edge node [below] {$b$}   (2)
        (1) edge node [above] {$a$}    (3)
            edge [bend left=15] node [below] {$b$}    (4)
        (2) edge [bend left=15] node [left] {$b$}    (3)
            edge node [below] {$a$}   (4)
        (4) edge node [below] {$a,b$} (5)
        (3) edge node [left] {$a,b$}  (5)
        (5) edge [loop above]   node [above] {$a,b$}  (5);
      \end{tikzpicture}
      \caption{Ще построим минимален автомат разпознаващ $\L(\A)$}
    \end{subfigure}
    \qquad
    \qquad
    \begin{subfigure}[b]{0.4\textwidth}
      \begin{tikzpicture}[->,>=stealth,thick,node distance=45pt]
        \tikzstyle{every state}=[circle,minimum size=20pt,auto,scale=.9]
        
        \node[initial above, state]   (0) {$C_0$};
        \node[state,accepting]  (1) [right of=0]{$C_1$};
        \node[state]            (2) [right of=1]{$C_2$};
        \node[state,accepting]  (3) [right of=2]{$C_3$};
                
        \path 
        (0) edge [bend left=15] node [above] {$a,b$}   (1)
        (1) edge [bend left=15] node [above] {$a,b$}   (2)
        (2) edge [bend left=15] node [above] {$a,b$}   (3)
        (3) edge [loop above]   node [above] {$a,b$}   (3);
      \end{tikzpicture}
      \caption{Получаваме следния минимален автомат $\A_0$, $\L(\A_0) = \L(\A)$}
      \label{sub:min2}
    \end{subfigure}
  \end{figure}

  \marginpar{Съобразете, че $\L(\A) = \{a,b\} \cup \{\alpha \in \{a,b\}^\star \mid \abs{\alpha} \geq 3\}$.}
  
  Отново следваме същата процедура за минимизация.
  Ще намерим класовете на еквивалентност на $\equiv_n$,
  докато не намерим $n$, за което $\equiv_n\ =\ \equiv_{n+1}$.
  \begin{itemize}
  \item
    Класовете на екиваленост на $\equiv_0$ са 
    $A_0 = Q\setminus F = \{0,3,4\}$ и $A_1 = F = \{1,2,5\}$.
  \item
    Разбиваме класовете на еквивалентност на $\equiv_0$ като използваме \Prop{one-letter-test}.
    
    \begin{tabular}{|c|c|c|c|c|c|c|}
      \hline
      $Q$ & 0 & $1^\star$ & $2^\star$ & 3 & 4 & $5^\star$ \\
      \hline
      \hline
      $\equiv_0$ & $A_0$ & $A_1$ & $A_1$ & $A_0$ & $A_0$ & $A_1$\\
      \hline
      $a$ & $A_1$& $A_0$ & $A_0$ & $A_1$ & $A_1$ & $A_1$\\
      \hline
      $b$ & $A_1$& $A_0$ & $A_0$ & $A_1$ & $A_1$ & $A_1$\\
      \hline
    \end{tabular}
    
    Виждаме, че $1 \not\equiv_1 5$ и $1 \equiv_0 5$.
    Следователно, $\equiv_0\ \neq\ \equiv_1$.
    Класовете на еквивалентност на $\equiv_1$ са 
    $B_0 = \{0,3,4\}$, $B_1 = \{1,2\}$, $B_2 = \{5\}$.
  \item
    Сега се опитваме да разбием класовете на еквивалентност на $\equiv_1$.

    \begin{tabular}{|c|c|c|c|c|c|c|}
      \hline
      $Q$ & 0 & $1^\star$ & $2^\star$ & 3 & 4 & $5^\star$ \\
      \hline
      \hline
      $\equiv_1$ & $B_0$ & $B_1$ & $B_1$ & $B_0$ & $B_0$ & $B_2$\\
      \hline
      $a$ & $B_1$ & $B_0$ & $B_0$ & $B_2$ & $B_2$ & $B_2$\\
      \hline
      $b$ & $B_1$ & $B_0$ & $B_0$ & $B_2$ & $B_2$ & $B_2$\\
      \hline
    \end{tabular}
    
    Имаме, че $0 \equiv_1 3$, но $0 \not\equiv_2 3$. Следователно $\equiv_1\ \neq\ \equiv_2$.
    Класовете на еквивалентност на $\equiv_2$ са 
    $C_0 = \{0\}$, $C_1 = \{1,2\}$, $C_2 = \{3,4\}$, $C_3 = \{5\}$.
  \item
    Отново опитваме да разбием класовете на $\equiv_2$.

      \begin{tabular}{|c|c|c|c|c|c|c|}
        \hline
        $Q$ & 0 & $1^\star$ & $2^\star$ & 3 & 4 & $5^\star$ \\
        \hline
        \hline
        $\equiv_2$ & $C_0$ & $C_1$ & $C_1$ & $C_2$ & $C_2$ & $C_3$\\
        \hline
        $a$ & $C_1$ & $C_2$ & $C_2$ & $C_3$ & $C_3$ & $C_3$\\
        \hline
        $b$ & $C_1$ & $C_2$ & $C_2$ & $C_3$ & $C_3$ & $C_3$\\
        \hline
      \end{tabular}
      
      Виждаме, че не можем да разбием $C_1$ или $C_2$.
      \marginpar{Получаваме, че $\equiv_\A\ =\ \equiv_2$}
      Следователно, $\equiv_2\ =\ \equiv_3$ и минималният автомат разпознаващ езика $L$
      има четири състояния. Вижте Фигура \ref{sub:min2} за преходите на минималния автомат.
      Минималният автомат може да се представи и таблично:

      \begin{tabular}{|c|c|c|c|c|}
        \hline
        $\delta$ & $C_0$ & $C_1$ & $C_2$ & $C_3$ \\
        \hline
        $a$ & $C_1$ & $C_2$ & $C_3$ & $C_3$ \\
        \hline
        $b$ & $C_1$ & $C_2$ & $C_3$ & $C_3$ \\
        \hline
      \end{tabular}
      
  \end{itemize}
\end{example}


%%% Local Variables:
%%% mode: latex
%%% TeX-master: "../eai"
%%% End:

\section{Изоморфни автомати}
\label{sect:isomorphic}

\index{изоморфизъм}
Нека са дадени автоматите
$\A' = (\Sigma,Q',\qstart',\delta',F')$ и $\A'' = (\Sigma, Q'', \qstart'', \delta'', F'')$.
Казваме, че $\A'$ и $\A''$ са {\bf изоморфни}, което означаваме с $\A' \cong \A''$, ако
съществува функция $f: Q'\to Q''$, за която:
\begin{enumerate}[(1)]
\item
  $f$ е биекция;
\item
  $f(\qstart') = \qstart''$;
\item
  $q \in F' \iff f(q) \in F''$;
\item
  $(\forall a\in\Sigma)(\forall q\in Q')[f(\delta'(q,a)) = \delta''(f(q),a)]$.
\end{enumerate}
Ще казваме, че $f$ задава изоморфизъм на $\A'$ върху $\A''$.

\begin{framed}
  \begin{thm}
    Ако $\A' \cong_f \A''$, то $\L(\A') = \L(\A'')$.
  \end{thm}  
\end{framed}
\begin{hint}
  Нека $\A' \cong_f \A''$. Първо с индукция по дължината на думата $\alpha$ ще докажем, че за произволно състояние $q \in Q'$,
  \begin{equation}
    \label{eq:3}
    f(\delta^\star_{\A'}(q,\alpha)) = \delta^\star_{\A''}(f(q), \alpha).
  \end{equation}
  \begin{itemize}
  \item 
    Нека $|\alpha| = 0$, т.е. $\alpha = \varepsilon$. Тогава:
    \begin{align*}
      f(\delta^\star_{\A'}(q,\varepsilon)) & = f(q) & \comment{\text{от деф. на }\delta^\star_{\A'}}\\
                                           & = \delta^\star_{\A''}(f(q), \varepsilon).
    \end{align*}
  \item
    Да приемем, че (\ref{eq:3}) е изпълнено за думи с дължина $n$.
    Да разгледаме произволна дума $\alpha$ с дължина $n+1$, т.е. $\alpha = \beta x$. Тогава:
    \begin{align*}
      f(\delta^\star_{\A'}(q,\beta x)) & = f(\delta_{\A'}(\delta^\star_{\A'}(q,\beta), x)) & \comment{\text{от деф. на }\delta^\star_{\A'}}\\
                                       & = \delta_{\A''}( f(\delta^\star_{\A'}(q,\beta), x)) & \comment{\text{от деф. на }f}\\
                                       & = \delta_{\A''}( \delta^\star_{\A''}(f(q),\beta), x)) & \comment{\text{от И.П.}}\\
                                       & = \delta^\star_{\A''}(f(q), \beta x) & \comment{\text{от деф. на }\delta^\star_{\A''}}.
    \end{align*}
  \end{itemize}
  Сега вече е лесно:
  \begin{align*}
    \alpha \in \L(\A') & \dff \delta^\star_{\A'}(\qstart',\alpha) \in F' \\
                       & \iff f(\delta^\star_{\A'}(\qstart',\alpha)) \in F'' & \comment{\text{от деф. на изоморфизъм}}\\
                       & \iff \delta^\star_{\A''}(f(\qstart'),\alpha) \in F'' & \comment{\text{от (\ref{eq:3})}}\\
                       & \iff \delta^\star_{\A''}(\qstart'',\alpha) \in F'' & \comment{f(\qstart') \df \qstart''}\\
                       & \dff \alpha \in \L(\A'').
  \end{align*}
\end{hint}

\index{Бжозовски}
Нека е даден регулярен език $L$ над азбуката $\Sigma$.
Конструкцията на автомата $\B$ по метода на Бжозовски е следната:
\begin{itemize}
\item 
  $Q^\B \df \{\alpha^{-1}(L) \mid \alpha \in \Sigma^\star\}$;
\item
  $\qstart^\B \df L$;
\item
  $F^\B \df \{N \in Q^\B \mid \varepsilon \in N\}$;
\item
  $\delta_\B(N,x) \df x^{-1}(N)$, за произволни $x \in \Sigma$ и $N \in Q^\B$.
\end{itemize}

Да припомним конструкцията на минималния автомат $\M$ според \hyperref[th:myhill-nerode]{Теоремата на Майхил-Нероуд}.
\begin{itemize}
\item 
  $Q^\M \df \{[\alpha]_L \mid \alpha \in \Sigma^\star\}$;
\item
  $\qstart^\M \df [\varepsilon]_L$;
\item
  $F^\M \df \{[\alpha]_L \mid [\alpha]_L \subseteq L\}$;
\item
  $\delta_\M([\alpha]_L,x) \df [\alpha\cdot x]_L$, за произволни $x \in \Sigma$ и $\alpha \in \Sigma^\star$.
\end{itemize}

\begin{framed}
  \begin{thm}
    $\B \cong \M$.
  \end{thm}  
\end{framed}
\begin{hint}
  Нека да дефинираме $f:Q^\M \to Q^\B$ по следния начин:
  \[f([\alpha]_L) \df \alpha^{-1}(L).\] 
  Ще докажем, че $f$ изпълнява свойствата за изоморфизъм.

  \begin{enumerate}[(1)]
  \item 
    Трябва първо да проверим, че $f$ е биектвна, т.е. $f$
    е инективна и сюрективна.
    \begin{itemize}
    \item
      Дефиницията на $f$ е зададена спрямо представител $\alpha$ на класа на еквивалентност на релацията $\approx_L$.
      \marginpar{Това е важно да се провери, защото дясната страна е дефинирана спрямо произволен представител на класа $[\alpha]_L$}
      Първо да проверим, че $f$ е функция.
      Нека $[\alpha]_L = [\beta]_L$, т.е. $\alpha^{-1}(L) = \beta^{-1}(L)$. Тогава 
      \begin{align*}
        f([\alpha]_L) & \df \alpha^{-1}(L)\\
                      & = \beta^{-1}(L) & \comment{\text{от } [\alpha]_L = [\beta]_L}\\
                      & \df f([\beta]_L).
      \end{align*}
    \item 
      Нека $[\alpha]_L \neq [\beta]_L$.
      Тогава:
      \begin{align*}
        f([\alpha]_L) & \df \alpha^{-1}(L)\\
                      & \neq \beta^{-1}(L) & \comment{\text{от }[\alpha]_L \neq [\beta]_L}\\
                      & \df f([\beta]_L).
      \end{align*}
      Оттук следва, че $f$ е {\em инективна}.
    \item
      Да разгледаме произволен елемент $N \in Q^\B$, т.е. $N$ е множество от думи и $N = \alpha^{-1}(L)$, за някоя дума $\alpha \in \Sigma^\star$.
      Понеже $f([\alpha]_L) \df \alpha^{-1}(L)$, то това означава, че $f$ е {\em сюрективна}.      
    \end{itemize}
  \item
    Лесно се съобразява, че
    \begin{align*}
      f(\qstart^\M) & = f([\varepsilon]_L) & \comment{\qstart^\M \df [\varepsilon]_L}\\
                    & \df \varepsilon^{-1}(L)\\
                    & = L \\
                    & \df \qstart^\B. & \comment{\qstart^\B \df L}
    \end{align*}
  \item
    Също не е трудно да се съобрази, че
    \begin{align*}
      [\alpha]_L \in F^\M & \dff [\alpha]_L \subseteq L\\
                          & \iff \varepsilon \in \alpha^{-1}(L)\\
                          & \iff f([\alpha]_L) \df \alpha^{-1}(L) \in F^\B.
    \end{align*}
  \item
    Имаме и свойството за произволна дума $\alpha \in \Sigma^\star$ и произволна буква $x \in \Sigma$:
    \begin{align*}
      f(\delta_\M([\alpha]_L,x)) & = f([\alpha\cdot x]_L) & \comment{\text{деф. на }\delta_\M}\\
                                 & = (\alpha\cdot x)^{-1}(L) & \comment{\text{деф. на }f}\\
                                 & = x^{-1}(\alpha^{-1}(L)) & \comment{\text{от \Prob{pullback}}}\\
                                 & = \delta_\B(\alpha^{-1}(L), x) & \comment{\text{деф. на }\delta_\B}\\
                                 & = \delta_\B(f([\alpha]_L), x), & \comment{\text{деф. на }f}
    \end{align*}
    от което следва, че $f$ е {\em изоморфизъм}.
  \end{enumerate}
\end{hint}

\begin{cor}
  Автоматът $\B$, построен по метода на Бжожовски, за регулярния език $L$ е минимален.
\end{cor}


\begin{framed}
  \begin{thm}
    \label{th:regular:isomorphic:minimal}
    Нека е даден регулярния език $L$.
    Нека $\A = \FA$ е произволен детерминистичен краен автомат, за който $\L(\A) = L$ и $\abs{Q} = \abs{\Sigma^\star/_{\approx_L}}$.
    Тогава $\A \cong \M$, където $\M$ е автоматът построен според \hyperref[th:myhill-nerode]{Теоремата на Майхил-Нероуд} за езика $L$.
  \end{thm}  
\end{framed}
\begin{proof}
  Съобразете, че $\A$ е {\em свързан}, т.е. всяко състояние на $\A$ е достижимо от началното.
  Искаме да докажем, че $\A \cong \M$.
  Да дефинираме биекцията $h:Q^\M\to Q^\A$ както в \Prop{bijection-classes}, т.е.
  \[h([\alpha]_L) \df \delta^\star_\A(\qstart,\alpha).\]
  Ще докажем, че $h$ задава изоморфизъм на $\A$ върху $\M$. 
  \begin{enumerate}[(1)]
  \item
    Вече знаем, че $h$ е биекция.
  \item
    Понеже $\delta^\star_\A(\qstart,\varepsilon) = \qstart$,
    то е ясно, че
    \[h([\varepsilon]_L) \df \delta^\star_\A(\qstart,\varepsilon) = \qstart.\]
  \item
    Също лесно се съобразява, че
    \begin{align*}
      [\alpha]_L \in F^\M & \iff [\alpha]_L \subseteq L\\
                          & \iff \alpha \in L & \comment\text{свойство на }\approx_L\\
                          & \iff \alpha \in \L(\A) & \comment{L = \L(\A)}\\
                          & \iff \delta^\star_\A(\qstart,\alpha) \in F^\A & \comment\text{деф. на }\L(\A)\\
                          & \iff h([\alpha]_L) \in F^\A. & \comment\text{деф. на }h
    \end{align*}
  \item
    За последно оставихме проверката, че $h$ наистина е {\bf изоморфизъм}:
    \begin{align*}
      h(\delta_\M([\alpha]_L,x)) & = h([\alpha x]_L) & \comment\text{деф. на }\delta_\M\\
                                 & = \delta^\star_\A(\qstart, \alpha x) & \comment\text{деф. на }h\\
                                 & = \delta_\A(\delta^\star_\A(\qstart,\alpha), x) & \comment\text{деф. на }\delta^\star_\A\\
                                 & = \delta_\A( h([\alpha]_L), x). & \comment\text{деф. на }h
    \end{align*}
  \end{enumerate}
\end{proof}




%%% Local Variables:
%%% mode: latex
%%% TeX-master: "../eai"
%%% End:

\section{Регулярни граматики}
\index{граматика!неограничена}
\label{sect:regular-grammar}
{\bf Неограничена граматика} e наредена четворка от вида
\[G = (V, \Sigma, R, S),\]
където
\begin{itemize}
\item
  $V$ е крайно множество от {\em променливи} (нетерминали);
\item
  $\Sigma$ е крайно множество от {\em букви} (терминали), $\Sigma \cap V = \emptyset$;
\item
  \marginpar{В \cite{hopcroft1} правилата се наричат {\em productions}}
  $R \subseteq (V\cup\Sigma)^+ \times (V \cup \Sigma)^\star$ е крайно множество от {\em правила}.
  Обикновено правилата $(\alpha, \beta) \in R$ ще означаваме като 
  $\alpha \to_G \beta$, където $\alpha \in (V \cup \Sigma)^+, \beta \in (V \cup \Sigma)^\star$;
\item
  $S \in V$ е началната променлива (нетерминал). 
\end{itemize}

Казваме, че имаме {\bf извод} $\omega \to_G \omega'$, ако $\omega = \alpha\beta\gamma \in (V\cup\Sigma)^\star$,
$\omega' = \alpha\beta'\gamma \in (V\cup\Sigma)^\star$ и имаме правило $\beta \to \beta'$ в граматиката $G$.
Нека $\to^\star_G$ е рефлексивното и транзитивно затваряне на релацията $\to_G$, т.е.
\begin{align*}
  & \alpha \to^\star_G \alpha\\
  & \alpha \to^\star_G \alpha'\ \&\ \alpha' \to_G \alpha'' \implies \alpha \to^\star_G \alpha''.
\end{align*}

Езикът, който се поражда от граматиката $G$ е
\[\L(G) = \{\omega \in \Sigma^\star \mid S \to^\star_G \omega\}.\]

Граматиките се разделят на няколко вида в зависимост от това какви {\em ограничения} налагаме върху правилата $R$.
В следващите няколко глави ще разгледаме различни ограничения. Сега ще разгледаме граматики с такъв вид правила,
които пораждат точно регулярните (или еквивалентно автоматни) езици.

\index{граматика!регулярна}
Граматиката $G = (V, \Sigma, R, S)$ се нарича {\bf регулярна граматика},
ако правилата са от вида 
\begin{align*}
  & A \to \varepsilon,\\
  % & A \to a,\\
  & A \to aB,
\end{align*}
където $A, B \in V$ и $a \in \Sigma$.

\begin{prop}
  Нека $G = \CFG$ е регулярна граматика и $L = \L(G)$.
  Съществува краен автомат $\A$, такъв че $L = \L(\A)$.
\end{prop}
\begin{hint}
  Нека $V = \{A_1,\dots,A_k\}$. Тогава:
  \begin{itemize}
  \item 
    $Q = \{q_1,\dots,q_k\}$;
  \item
    $F = \{q_i \mid A_i \to \varepsilon\}$.
  \item
    $\delta(q_i,a) = q_j\ \iff\ A_i \to aA_j$.
  \end{itemize}
\end{hint}

\begin{prop}
  Нека $\A$ е краен автомат и $L = \L(\A)$.
  Съществува регулярна граматика $G$, такава че $L = \L(G)$.
\end{prop}
\begin{hint}
  Нека $Q = \{q_1,\dots,q_k\}$. Тогава:
  \begin{itemize}
  \item 
    $V = \{A_1,\dots,A_k\}$;
  \item
    $A_i \to aA_j\ \iff\ \delta(q_i,a) = q_j$;
  \item
    $A_{i} \to \varepsilon\ \iff\ q_{i} \in F$.
  \end{itemize}
\end{hint}


%%% Local Variables:
%%% mode: latex
%%% TeX-master: "../eai"
%%% End:

\subsection{Не толкова лесни задачи}

{\bf Това вече е добре да се махне}
\begin{problem}
  Докажете, че няма полиномиален алгоритъм за детерминизация на краен недетерминиран автомат.
\end{problem}
\begin{hint}
  За произволно $n$, разгледайте недетерминирания автомат $\A_n$, за който
  \[(\forall \alpha,\beta \in \{0,1\}^\star)[|\alpha| = |\beta| = n \implies (\alpha\beta \in \L(\A_n) \iff \alpha \neq \beta)].\]
  Този автомат ще има $2n+2$ състояния.

  Допуснете, че за него съществува детерминиран автомат $\D_n$ с $< 2^n$ на брой състояния.
  Разгледайте всички думи с дължина $n$, $\omega_1,\omega_2,\dots,\omega_{2^n}$.
  Приложете принципа на Дирихле и достигнете до противоречие.
\end{hint}


\begin{problem}
  \mynote{\cite[стр. 84]{papadimitriou}}
  При дадени езици $L$, $L'$ над азбуката $\Sigma$, да разгледаме:
  \begin{enumerate}[a)]
  \item
    $\texttt{Pref}(L) = \{\alpha \in \Sigma^\star \mid (\exists \beta \in \Sigma^\star)[\alpha\beta \in L]\}$;
  \item
    $\texttt{NoPref}(L) = \{\alpha \in L \mid \text{ не съществува префикс на $\alpha$ в $L$}\}$;
  \item
    $\texttt{NoExtend}(L) = \{\alpha \in L \mid \text{ $\alpha$ не е префикс на никоя дума от $L$}\}$;
  \item
    $\mbox{Suf}(L) = \{\beta \in \Sigma^\star \mid (\exists \alpha \in \Sigma^\star)[\alpha\beta \in L]\}$;
  \item
    $\text{Infix}(L) = \{\alpha \mid (\exists \beta,\gamma \in \Sigma^\star)[\beta\alpha\gamma \in L]\}$;
  \item 
    $\frac{1}{2}(L) = \{\omega \in \Sigma^\star \mid (\exists \alpha \in \Sigma^\star)[\omega\alpha \in L\ \&\ \abs{\omega} = \abs{\alpha}]\}$;
  \item
    \mynote{right quotient of $L$ by $L'$}
    $L/L' = \{\alpha \in \Sigma^\star \mid (\exists \beta \in L')[\alpha\beta \in L ] \}$;
  \item
    $L^{-1}(L') = \{ \beta \mid (\exists \alpha \in L)[ \alpha\beta \in L']\}$;
  \item
    $\mbox{Max}(L) = \{\alpha \in \Sigma^\star \mid (\forall \beta\in\Sigma^\star)[\beta \neq \varepsilon\implies \alpha\beta \not\in L]\}$.
  \end{enumerate}
  За всички тези езици, докажете, че са регулярни при условие, че $L$ и $L'$ са регулярни.
  \mynote{Тази конструкция няма да бъде ефективна}
  Освен това, докажете, че $L/L'$ е регулярен и при условието, че $L$ е регулярен, но $L'$ е произволен език над азбуката $\Sigma$.
\end{problem}
\begin{hint}
  \begin{enumerate}[a)]
  \item 
    Индукция по дефиницията на регулярен израз.
  \item[в)]
    Най-лесно е да се построи автомат за $\text{Infix}(L)$ като се използва автомата за $L$.
  \item[г)]
    Конструкция с автомат за $L$ и автомат за $L^{\texttt{rev}}$.
  \end{enumerate}
\end{hint}

\begin{problem}
  \mynote{\cite[стр. 75]{kozen}; \cite[стр. 89]{papadimitriou}}
  Да фиксираме азбука само с един символ $\Sigma = \{a\}$.
  Да положим за всяко $p,q\in\Nat$, 
  \[L(p,q) = \{a^k \mid (\exists n\in\Nat)[k = p+q\cdot n]\}.\]
  Ако за един език $L$ съществуват константи $p_1,\dots,p_k$ и $q_1,\dots,q_k$, такива че 
  \[L = \bigcup_{1\leq i \leq k} L(p_i,q_i),\]
  то казваме, че $L$ е {\em породен от аритметични прогресии}.
  \begin{enumerate}[a)]
  \item 
    Докажете, че $L \subseteq \{a\}^\star$ е регулярен език точно тогава, когато $L$ е породен от аритметична прогресия.
  \item
    За произволна азбука $\Sigma$, докажете, че ако $L \subseteq \Sigma^\star$ е регулярен език,
    то езикът $\{a^{\abs{\omega}} \mid \omega \in L\}$  е породен от аритметична прогресия.    
  \end{enumerate}
\end{problem}
\begin{hint}
  \begin{enumerate}[a)]
  \item 
    За едната посока, разгледайте ДКА за $L$.
  \item
    За втората част, разгледайте $h:\Sigma\to\{a\}$ деф. като $(\forall b\in\Sigma)[h(b) = a]$.
    Докажете, че $h$ е поражда хомоморфизъм между $\Sigma^\star$ и $\{a\}^\star$.
    Тогава $h(L) = \{a^{\abs{\omega}} \mid \omega \in L\}$, а
    ние знаем, че регулярните езици са затворени относно хомоморфни образи.  
  \end{enumerate}
\end{hint}

\begin{problem}
  Вярно ли е, че:
  \begin{itemize}
  \item 
    $\{a^m \mid a^{m^2} \in L(p,q)\}$ е регулярен език ?
  \item
    $\{a^m \mid a^{2^m} \in L(p,q)\}$ е регулярен език ?
  \end{itemize}
\end{problem}


\begin{problem}
  За даден език $L$ над азбуката $\Sigma$, да разгледаме езиците:
  \begin{enumerate}[a)]
  \item
    $L' = \{\alpha \in \Sigma^\star \mid (\exists \beta\in\Sigma^\star)[\abs{\alpha} = 2\abs{\beta}\ \&\ \alpha\beta \in L]\}$;
  \item 
    $L'' = \{\alpha \in \Sigma^\star \mid (\exists \beta\in\Sigma^\star)[2\abs{\alpha} = \abs{\beta}\ \&\ \alpha\beta \in L]\}$;
  \item 
    $\frac{1}{3}(L) = \{\alpha \in \Sigma^\star \mid (\exists \beta,\gamma \in \Sigma^\star)[\abs{\alpha} = \abs{\beta} = \abs{\gamma}\ \&\ \alpha\beta\gamma \in L]\}$;
  \item
    $\frac{2}{3}(L) = \{\beta \in \Sigma^\star \mid (\exists \beta,\gamma \in \Sigma^\star)[\abs{\alpha} = \abs{\beta} = \abs{\gamma}\ \&\ \alpha\beta\gamma \in L]\}$;
  \item
    $\frac{3}{3}(L) = \{\gamma \in \Sigma^\star \mid (\exists \beta,\gamma \in \Sigma^\star)[\abs{\alpha} = \abs{\beta} = \abs{\gamma}\ \&\ \alpha\beta\gamma \in L]\}$;
  \item
    $\hat{L} = \{\alpha\gamma \in \Sigma^\star \mid (\exists \beta,\gamma \in \Sigma^\star)[\abs{\alpha} = \abs{\beta} = \abs{\gamma}\ \&\ \alpha\beta\gamma \in L]\}$;
  \item
    $\sqrt{L} = \{\alpha \mid (\exists \beta \in \Sigma^\star)[\abs{\beta} = \abs{\alpha}^2\ \&\ \alpha\beta \in L]\}$;
  \item
    $\log(L) = \{\alpha \mid (\exists \beta \in \Sigma^\star)[\abs{\beta} = 2^{\abs{\alpha}}\ \&\ \alpha\beta \in L]\}$;
  \end{enumerate}
  Проверете ако $L$ е регулярен, то кои от горните езици също са регулярни.
\end{problem}

\begin{problem}
  Да разгледаме езика
  \[L = \{\omega \in \{0,1\}^\star \mid \omega\text{ съдържа равен брой поднизове }01\text{ и }10\}.\]
  Например, $101 \in L$, защото съдържа по веднъж $10$ и $01$.
  $1010 \not\in  L$, защото съдържа два пъти $10$ и само веднъж $01$.
  Докажете, че $L$ е регулярен.
\end{problem}

\begin{problem}
  Нека $L$ е регулярен език над азбуката $\{a,b\}$. Докажете, че следните езици са регулярни:
  \begin{enumerate}[a)]
  \item 
    $\texttt{Diff}_1(L) \df \{\alpha \in L \mid (\exists \beta \in L)[|\alpha| = |\beta|\ \&\ \alpha \text{ се различава от $\beta$ в една позиция}]\}$;
  \item
    $\texttt{Diff}_n(L) \df \{\alpha \in L \mid (\exists \beta \in L)[n \leq |\alpha| = |\beta|\ \&\ \alpha \text{ се различава от $\beta$ в $n$ позиции}]\}$;
  \item
    $\texttt{Diff}(L) \df \{\alpha \in L \mid (\exists \beta \in L)[|\alpha| = |\beta|\ \&\ \alpha \text{ се различава от $\beta$ във всяка  позиция}]\}$;
  \end{enumerate}
\end{problem}
\begin{hint}
  Ако $L = \L(\A)$, то правим декартово произведение на $\A$ плюс флаг дали сме направили грешка.

  Не е ли по-лесно с индукция по построението на регулярните езици ?
\end{hint}


% \begin{problem}
%   \mynote{(\cite{kozen}, стр. 75)}
%   Да фиксираме азбука само с един символ $\Sigma = \{a\}$.
%   Множеството $U$ е {\em породен от аритметична прогресия}, ако съществуват числа $q \geq 0$ и $p > 0$,
%   такива че $(\forall n \geq q)[n \in U\ \iff\ n+p \in U]$.
%   Докажете, че $L \subseteq \{a\}^\star$ е регулярен точно тогава, когато множеството $\{m \mid a^m \in L\}$
%   е породено от аритметична прогресия.
% \end{problem}
% \begin{hint}
%   Разгледайте КДА за $L$.
% \end{hint}

% \begin{hint}
%   \begin{itemize}
%   \item 
%     Докажете, че за всяко $p,q \in \Nat$, $L(p,q)$ е регулярен език.
%   \item
%     Докажете, че за крайно много $p_0,\dots,p_k$, $q_0,\dots,q_k$,
%     $\bigcup_{i \leq k}L(p_i,q_i)$ е регулярен език.
%   \item
%     С индукция по построението на регулярните езици, 
%     докажете, че ако $L$ е регулярен, то $L$ може да се представи
%     като крайно обединение на езици породени от аритметични прогресии.
%     Съществената част от доказателството се състои в следното:
%     \begin{itemize}
%     \item 
%       \mynote{$L(p_1,q_1)\cdot L(p_2,q_2) = L(p_1+p_2,\mbox{НОД}(q_1,q_2))\setminus F$, където $F$ е крайно м-во, и ако $q_1 = q_2$, то $F = \emptyset$}
%       езикът $L(p_1,q_1) \cdot L(p_2,q_2)$ може да се представи като крайно обединение 
%       на езици породени от артиметични прогресии.
%     \item
%       езикът $L(p,q)^\star$ може да се представи като крайно обединение 
%       на езици породени от артиметични прогресии.
%     \end{itemize}
%   \end{itemize}
% \end{hint}



\begin{problem}
  \mynote{Да обърнем внимание, че езикът $\{\alpha\sharp \beta \sharp \gamma \mid \bin{\alpha}+\bin{\beta} = \bin{\gamma}\}$ не е регулярен.}
  Да разгледаме азбуката:
  \[\Sigma_3 = \left\{\begin{bmatrix} 0\\0\\0\end{bmatrix},\begin{bmatrix} 0\\0\\1\end{bmatrix},\begin{bmatrix} 0\\1\\0\end{bmatrix},\begin{bmatrix} 0\\1\\1\end{bmatrix},\dots,\begin{bmatrix} 1\\1\\1\end{bmatrix}\right\}.\]
  Докажете, че 
  $L = \left\{\begin{bmatrix} \alpha\\\beta\\\gamma\end{bmatrix} \in \Sigma^\star_3 \mid \bin{\alpha}+\bin{\beta} = \bin{\gamma}\right\}$
  е автоматен език.
\end{problem}
\ifhints
\begin{hint}
  Доста по-удобно е да построим автомат $\A$, такъв че $\L(\A) = L^{\texttt{rev}}$.
  Да започнем с състоянието $q_{\scriptscriptstyle{=}}$, за което искаме да имаме свойството, че за произволно състояние $q$,
  \[\delta^\star(q, \tiny{ \begin{bmatrix} \alpha\\ \beta \\ \gamma\end{bmatrix} }) = q_{\scriptscriptstyle{=}}  \iff \bin{\alpha^{\rev}} + \bin{\beta^{\rev}} = \bin{\gamma^{\rev}}.\]
  Понеже за $\bin{\varepsilon} + \bin{\varepsilon} = \bin{\varepsilon}$, състоянието $q_{\scriptscriptstyle{=}}$ ще бъде начално и финално за $\A$.

  Нека $\bin{\alpha}+\bin{\beta} = \bin{\gamma}$. Тогава:
  \begin{align*}
    & \delta(q_{\scriptscriptstyle{=}},\tiny{ \begin{bmatrix} 0\\ 0 \\ 0\end{bmatrix} }) \df q_{\scriptscriptstyle{=}} & \comment\text{ защото }\bin{0\alpha} + \bin{0\beta} = \bin{0\gamma}\\
    & \delta(q_{\scriptscriptstyle{=}},\tiny{ \begin{bmatrix} 0\\ 1 \\ 1\end{bmatrix} }) \df q_{\scriptscriptstyle{=}} & \comment\text{ защото }\bin{0\alpha} + \bin{1\beta} = \bin{1\gamma}\\
    & \delta(q_{\scriptscriptstyle{=}},\tiny{ \begin{bmatrix} 1\\ 0 \\ 1\end{bmatrix} }) = q_{\scriptscriptstyle{=}} & \comment\text{ защото }\bin{1\alpha} + \bin{0\beta} = \bin{1\gamma}
  \end{align*}
  Остана случая $\bin{1\alpha} + \bin{1\beta} = \bin{10\gamma}$. Този случай е по-специален и трябва да бъде разгледан отделно.
  Трябва да отидем в състояние $q_1$, в което ще помним, че третия ред трябва да започва с $1$-ца. Затова имаме следния преход:
  \[\delta(q_{\scriptscriptstyle{=}},\tiny{ \begin{bmatrix} 1\\ 1 \\ 0\end{bmatrix} }) \df q_1.\]
  
  За останалите $\gamma \in \Sigma_3$ имаме, че
  \[\delta(q_{\scriptscriptstyle{=}},\gamma) \df q_{\texttt{err}},\]
  където $q_{\texttt{err}}$ е състоянието, от което не можем да излезем.
  
  Така трябва да дефинираме функцията на преходите, че за състоянието $q_1$ трябва да е изпълнено, че за произволно $q$,
  \[\delta^\star(q, \tiny{ \begin{bmatrix} \alpha\\ \beta \\ \gamma\end{bmatrix} }) = q_{\scriptscriptstyle{1}}  \iff \bin{\alpha^{\rev}} + \bin{\beta^{\rev}} = \bin{1\gamma^{\rev}}.\]
  Да разгледаме сега случая $\bin{\alpha} + \bin{\beta} = \bin{1\gamma}$. Тогава:
  \begin{align*}
    & \delta(q_1,\tiny{ \begin{bmatrix} 0\\ 0 \\ 1\end{bmatrix} }) \df q_{\scriptscriptstyle{=}} & \comment\text{ защото }\bin{0\alpha} + \bin{0\beta} = \bin{1\gamma}\\
    & \delta(q_1,\tiny{ \begin{bmatrix} 1\\ 1 \\ 1\end{bmatrix} }) \df q_{1} & \comment\text{ защото }\bin{1\alpha} + \bin{1\beta} = \bin{11\gamma}\\
    & \delta(q_1,\tiny{ \begin{bmatrix} 1\\ 0 \\ 0\end{bmatrix} }) \df q_{1} & \comment\text{ защото }\bin{1\alpha} + \bin{0\beta} = \bin{10\gamma}\\
    & \delta(q_1,\tiny{ \begin{bmatrix} 0\\ 1 \\ 0\end{bmatrix} }) \df q_{1} & \comment\text{ защото }\bin{0\alpha} + \bin{1\beta} = \bin{10\gamma}\\
    & \delta(q_1, \gamma) \df q_{\texttt{err}} & \comment\text{ за останалите }\gamma \in \Sigma_3
  \end{align*}
\end{hint}
\fi

\begin{problem}
  Да разгледаме азбуката:
  \[\Sigma_2 = \left\{\begin{bmatrix} 0\\0\end{bmatrix},\begin{bmatrix} 0\\1\end{bmatrix},\begin{bmatrix} 1\\0\end{bmatrix},\begin{bmatrix} 1\\1\end{bmatrix}\right\}.\]
  Една дума над азбуката $\Sigma_2$ ни дава два реда от $0$-ли и $1$-ци, които ще разглеждаме като числа в двоична бройна система.
  Да разгледаме езиците:
  \begin{itemize}
  \item 
    $L_1 = \left\{\begin{bmatrix} \alpha\\ \beta \end{bmatrix} \in \Sigma^\star_2 \mid \ov{\alpha}_{(2)} < \ov{\beta}_{(2)}\right\}$;
  \item
    $L_2 = \left\{\begin{bmatrix} \alpha\\ \beta \end{bmatrix} \in \Sigma^\star_2 \mid 3(\ov{\alpha}_{(2)}) = \ov{\beta}_{(2)}\right\}$;
  \item
    $L_3 = \left\{\begin{bmatrix} \alpha\\ \beta \end{bmatrix} \in \Sigma^\star_2 \mid \alpha = \beta^{\rev}\right\}$;
  \end{itemize}
  Докажете, че  $L_1$ и $L_2$ са автоматни, а $L_3$ не е автоматен.
\end{problem}
\ifhints
\begin{hint}
  Ще построим автомат $\A = \FA$ за езика $L^{\rev}_1$.
  За улеснение, в рамките на тази задача ще пишем:
  \begin{itemize}
  \item 
    $\alpha \equiv \beta$, ако $\ov{\alpha^{\rev}}_{(2)} = \ov{\beta^{\rev}}_{(2)}$,
  \item
    $\alpha \prec \beta$, ако $\ov{\alpha^{\rev}}_{(2)} < \ov{\beta^{\rev}}_{(2)}$,
  \item
    $\alpha \succ \beta$, ако $\ov{\alpha^{\rev}}_{(2)} > \ov{\beta^{\rev}}_{(2)}$.
  \end{itemize}
  Нека състоянията на автомата са $Q = \{q_{\scriptscriptstyle{=}},q_{\scriptscriptstyle{<}},q_{\scriptscriptstyle{>}}\}$.
  Искаме да е изпълнено свойствата:
  \begin{itemize}
  \item 
    $\delta^\star(q_{\scriptscriptstyle{=}}, \scriptsize{\begin{bmatrix} \alpha\\ \beta\end{bmatrix}}) = q_{\scriptscriptstyle{=}}$ точно тогава, когато $\alpha \equiv \beta$;
  \item 
    $\delta^\star(q_{\scriptscriptstyle{=}}, \scriptsize{\begin{bmatrix} \alpha\\ \beta\end{bmatrix}}) = q_{\scriptscriptstyle{<}}$ точно тогава, когато $\alpha \prec \beta$;
  \item 
    $\delta^\star(q_{\scriptscriptstyle{=}}, \scriptsize{\begin{bmatrix} \alpha\\ \beta\end{bmatrix}}) = q_{\scriptscriptstyle{>}}$ точно тогава, когато $\alpha \succ \beta$.
  \end{itemize}
  Множеството от финални състояния ще бъде $F = \{q_{\scriptscriptstyle{<}}\}$, а началното състояние $\qstart = q_{\scriptscriptstyle{=}}$.
  За да дефинираме функцията на преходите, трябва да разгледа няколко случая, в зависимост от това какво е отношението между $\alpha$ и $\beta$.
  \begin{itemize}
  \item
    Нека $\alpha \equiv \beta$. Тогава:  
    \begin{itemize}
    \item 
      $\alpha 0 \equiv \beta 0$ и $\alpha 1 \equiv \beta 1$. Следователно,
      \[\delta(q_{\scriptscriptstyle{=}},\scriptsize{\begin{bmatrix} 0\\0\end{bmatrix}}) = \delta(q_{\scriptscriptstyle{=}},\scriptsize{\begin{bmatrix} 1\\1\end{bmatrix}}) = q_{\scriptscriptstyle{=}}.\]
    \item
      $\alpha 0 \prec \beta 1$. Следователно,
      \[\delta(q_{\scriptscriptstyle{=}},\scriptsize{\begin{bmatrix} 0\\1\end{bmatrix}}) = q_{\scriptscriptstyle{>}}.\]
    \item
      $\alpha 1 \succ \beta 0$. Следователно,
      \[\delta(q_{\scriptscriptstyle{=}},\scriptsize{\begin{bmatrix} 1\\0\end{bmatrix}}) = q_{\scriptscriptstyle{<}}.\]
    \end{itemize}
  \item 
    Нека $\alpha \prec \beta$. Тогава:
    \begin{itemize}
    \item 
      $\alpha 0 \prec \beta 0$, $\alpha 1 \prec \beta 1$, $\alpha 0 \prec \beta 1$. Следователно,
      \[\delta(q_{\scriptscriptstyle{<}},\scriptsize{\begin{bmatrix} 0\\0\end{bmatrix}}) = \delta(q_{\scriptscriptstyle{<}},\scriptsize{\begin{bmatrix} 1\\1\end{bmatrix}}) = \delta(q_{\scriptscriptstyle{<}},\scriptsize{\begin{bmatrix} 0\\1\end{bmatrix}}) = q_{\scriptscriptstyle{<}}.\]
    \item
      $\alpha 1 \succ \beta 0$. Следователно,
      \[\delta(q_{\scriptscriptstyle{<}},\scriptsize{\begin{bmatrix} 1\\0\end{bmatrix}}) = q_{\scriptscriptstyle{>}}.\]
    \end{itemize}
  \item
    Нека $\alpha \succ \beta$. Тогава:
    \begin{itemize}
    \item 
      $\alpha 0 \succ \beta 0$, $\alpha 1 \succ \beta 1$, $\alpha 1 \succ \beta 0$. Следователно,
      \[\delta(q_{\scriptscriptstyle{>}},\scriptsize{\begin{bmatrix} 0\\0\end{bmatrix}}) = \delta(q_{\scriptscriptstyle{>}},\scriptsize{\begin{bmatrix} 1\\1\end{bmatrix}}) = \delta(q_{\scriptscriptstyle{>}},\scriptsize{\begin{bmatrix} 1\\0\end{bmatrix}}) = q_{\scriptscriptstyle{>}}.\]
    \item
      $\alpha 0 \prec \beta 1$. Следователно,
      \[\delta(q_{\scriptscriptstyle{>}},\scriptsize{\begin{bmatrix} 0\\1\end{bmatrix}}) = q_{\scriptscriptstyle{<}}.\]
    \end{itemize}
  \end{itemize}
  Докажете, че за така дефинирания автомат $\A$, $\L(\A) = L^{\texttt{rev}}_1$.
\end{hint}
\fi


%%% Local Variables:
%%% mode: latex
%%% TeX-master: "../eai"
%%% End:


%%% Local Variables:
%%% mode: latex
%%% TeX-master: "../eai"
%%% End:

\chapter{Безконтекстни езици и стекови автомати}


Ще започнем като разгледаме понятието извод в граматика в най-общия му вид.
Граматиките се разделят на няколко вида в зависимост от това какви {\em ограничения} налагаме върху правилата на граматиката. В следващите няколко глави ще разгледаме различни ограничения. 
След това ще разгледаме някои класове от граматики като ще се концентрираме основно върху безконтекстните граматики.

\section{Неограничени граматики}\label{sect:unrestricted-grammar}
\index{граматика!неограничена}

\mynote{На англ. {\em unrestricted grammar}. Това е тип 0 граматики в йерархията на Чомски \cite[стр. 220]{hopcroft1}.}
{\bf Неограничена граматика} e наредена четворка от вида
\[G = (V, \Sigma, R, S),\]
където:
\begin{itemize}
\item
  $V$ е крайно множество от {\em променливи} (нетерминали);
\item
  $\Sigma$ е крайно множество от {\em букви} (терминали), като $\Sigma \cap V = \emptyset$;
\item
  \mynote{В \cite{hopcroft1} правилата се наричат {\em productions} или {\em production rules}.}
  $R \subseteq (V\cup\Sigma)^+ \times (V \cup \Sigma)^\star$ е крайно множество от {\em правила}.
  За по-добра яснота, обикновено правилата $(\alpha, \beta) \in R$ ще означаваме като 
  $\alpha \to_G \beta$. Когато е ясно за коя граматика говорим, ще пишем просто $\alpha \to \beta$.
\item
  $S \in V$ е началната променлива (нетерминал). 
\end{itemize}

\index{граматика!извод}
Удобно е да дефинираме извод на думата $\beta$ от думата $\alpha$ в граматиката $G$ за $\ell$ стъпки, което ще означаваме като $\alpha \derive{\ell}_G \beta$,
с индукция по броя на стъпките $\ell$ по следния начин:

\begin{important}
  \begin{figure}[H]
    \begin{subfigure}[b]{0.5\textwidth}
      \begin{prooftree}
        \AxiomC{$\alpha \in (V\cup\Sigma)^\star$}
        \RightLabel{\scriptsize{правило (0)}}
        \UnaryInfC{$\alpha \derive{0}_G \alpha$}
      \end{prooftree}
    \end{subfigure}
    ~
    \begin{subfigure}[b]{0.5\textwidth}
      \begin{prooftree}
        \AxiomC{$(\alpha,\beta)\in R$}
        \AxiomC{$\lambda\beta\rho \derive{\ell} \gamma$}
        \RightLabel{\scriptsize{правило (1)}}
        \BinaryInfC{$\lambda\alpha\rho \derive{\ell+1}_G \gamma$}
      \end{prooftree}
    \end{subfigure}
    \caption{Правила за извод в неограничена граматика}
  \end{figure}
\end{important}

\mynote{Обърнете внимание, че имаме недетерминизъм в тази дефиниция на извод. Също така, понякога ,за удобство, ще пишем просто $\derive{\ell}$ вместо $\derive{\ell}_G$,
  когато се знае за коя граматика говорим.

  С други думи, $\derive{\star}_G$ е рефлексивното и транзитивно затваряне на релацията $\derive{1}_G$.
}
Сега дефинираме релацията $\derive{\star}_G$ като
\[ \alpha \derive{\star}_G \beta\ \dff\ (\exists \ell\in\Nat)[\ \alpha \derive{\ell}_G \beta\ ].\]
Езикът, който се поражда от граматиката $G$ дефинираме по следния начин:
\[\L(G) \df \{\omega \in \Sigma^\star \mid S \derive{\star}_G \omega\}.\]

За да можем да работим по-удобно с релацията за извод в граматика, ще започнем с няколко свойства.

\begin{proposition}\label{pr:unrestricted-grammar:padding}
  За проиволно естествено число $\ell$ имаме извода:
  \begin{prooftree}
    \AxiomC{$\alpha \derive{\ell} \beta$}
    \AxiomC{$\lambda,\rho \in (V\cup\Sigma)^\star$}
    \BinaryInfC{$\lambda \alpha \rho \derive{\ell} \lambda \beta \rho$}
  \end{prooftree}
\end{proposition}
\begin{proof}
  Индукция по $\ell$.
  \begin{itemize}
  \item
    $\ell = 0$. Директно следва от правило $(0)$.
  \item
    $\ell > 0$. Според правилата за извод имаме следния случай:
    \begin{prooftree}
      \AxiomC{$(\alpha',\alpha'') \in R$}
      \AxiomC{$\gamma\alpha''\delta \derive{\ell-1} \beta$}
      \RightLabel{\scriptsize{правило (1)}}
      \BinaryInfC{$\underbrace{\gamma \alpha' \delta}_{\alpha} \derive{\ell} \beta$}
    \end{prooftree}
    Сега като използваме \IndHyp получаваме следния извод:
    \begin{prooftree}
      \AxiomC{$(\alpha',\alpha'') \in R$}
      \AxiomC{$\gamma\alpha''\delta \derive{\ell-1} \beta$}
      \RightLabel{\scriptsize{\IndHyp}}
      \UnaryInfC{$\lambda\gamma\alpha''\delta\rho \derive{\ell-1} \lambda\beta\rho$}
      \RightLabel{\scriptsize{правило (1)}}
      \BinaryInfC{$\lambda\underbrace{\gamma \alpha' \delta}_{\alpha}\rho \derive{\ell} \lambda\beta\rho$}
    \end{prooftree}
  \end{itemize}
\end{proof}


\begin{proposition}\label{pr:unrestricted-grammar:concat2}
  За произволни $\ell_1$ и $\ell_2$ имаме извода:
  \begin{prooftree}
    \AxiomC{$\alpha_1 \derive{\ell_1} \beta_1$}
    \AxiomC{$\alpha_2 \derive{\ell_2} \beta_2$}
    \BinaryInfC{$\alpha_1\alpha_2 \derive{\ell_1+\ell_2} \beta_1\beta_2$}
  \end{prooftree}
\end{proposition}
\begin{proof}
  Индукция по $\ell_1$.
  \begin{itemize}
  \item
    Ако $\ell_1 = 0$, то $\alpha_1 = \beta_1$ и тогава прилагаме \Proposition{unrestricted-grammar:padding}.
  \item
    Ако $\ell_1 > 0$, то имаме следното:
    \begin{prooftree}
      \AxiomC{$(\alpha'_1,\alpha''_1) \in R$}
      \AxiomC{$\gamma\alpha''_1\delta \derive{\ell_1-1} \beta_1$}
      \RightLabel{\scriptsize{правило (1)}}
      \BinaryInfC{$\underbrace{\gamma \alpha'_1 \delta}_{\alpha_1} \derive{\ell_1} \beta_1$}
    \end{prooftree}
    Сега прилагаме \IndHyp и получаваме следния извод:
    \begin{prooftree}
      \AxiomC{$(\alpha'_1,\alpha''_1) \in R$}
      \AxiomC{$\gamma\alpha''_1\delta \derive{\ell_1-1} \beta_1$}
      \AxiomC{$\alpha_2 \derive{\ell_2} \beta_2$}
      \RightLabel{\scriptsize\IndHyp}
      \BinaryInfC{$\gamma \alpha''_1 \delta \alpha_2 \derive{\ell_1-1+\ell_2} \beta_1\beta_2$}
      \RightLabel{\scriptsize{правило (1)}}
      \BinaryInfC{$\underbrace{\gamma\alpha'_1\delta}_{\alpha_1}\alpha_2 \derive{\ell_1+\ell_2} \beta_1\beta_2$}
    \end{prooftree}
  \end{itemize}
  
\end{proof}



\begin{proposition}\label{pr:unrestricted-grammar:concat}
  За всяко $k$ е изпълнено, че:
  \begin{prooftree}
    \AxiomC{$\alpha_1 \derive{\ell_1} \beta_1$}
    \AxiomC{$\dots$}
    \AxiomC{$\alpha_k \derive{\ell_k} \beta_k$}
    \RightLabel{\scriptsize{$(\ell = \sum^k_{i=1} \ell_i)$}}
    \TrinaryInfC{$\alpha_1\cdots\alpha_k \derive{\ell} \beta_1\cdots\beta_k$}
  \end{prooftree}
\end{proposition}
\begin{hint}
  Индукция по $k$.
\end{hint}

% \begin{proposition}\label{pr:unrestricted-grammar:general-step}
%   За произволни естествени числа $\ell_1$ и $\ell_2$ е изпълнено, че:
%   \begin{prooftree}
%     \AxiomC{$\alpha \derive{\ell_1} \beta$}
%     \AxiomC{$\rho \beta \delta \derive{\ell_2} \gamma$}
%     \BinaryInfC{$\rho \alpha \delta \derive{\ell_1+\ell_2} \gamma$}
%   \end{prooftree}  
% \end{proposition}
% \begin{hint}
%   Пълна индукция по $(\ell_1,\abs{\alpha})$ с лексикографската наредба.
%   \begin{itemize}
%   \item
%     Ако $\ell_1 = 0$, то е тривиално, защото тогава $\alpha = \beta$.
%   \item
%     Ако $\ell_1 > 0$, то имаме два случая в зависимост от това кое е последното правило, което сме приложили за да получим $\alpha \derive{\ell_1} \beta$.
%     \begin{itemize}
%     \item
%       Първият случай е, ако сме приложили правило (1), т.е.
%       \begin{prooftree}
%         \AxiomC{$\alpha \to_G \delta$}
%         \AxiomC{$\delta \derive{\ell_1-1} \beta$}
%         \RightLabel{\scriptsize{(1)}}
%         \BinaryInfC{$\alpha \derive{\ell_1} \beta$}
%       \end{prooftree}
%       Тогава получаваме следния извод:
%       \begin{prooftree}
%         \AxiomC{$\alpha \to_G \delta$}
%         \AxiomC{$\delta \derive{\ell_1-1} \beta$}
%         \AxiomC{$\beta \derive{\ell_2} \gamma$}
%         \RightLabel{\scriptsize{\IndHyp}}
%         \BinaryInfC{$\delta \derive{\ell_1+\ell_2-1} \gamma$}
%         \RightLabel{\scriptsize{(1)}}
%         \BinaryInfC{$\alpha \derive{\ell_1+\ell_2} \gamma$}
%       \end{prooftree}
%     \item
%       Вторият случай е, ако сме приложили правило (2), т.е. 
%       \begin{prooftree}
%         \AxiomC{$\alpha_1 \derive{\ell'_1} \beta_1$}
%         \AxiomC{$\alpha_2 \derive{\ell''_1} \beta_2$}
%         \AxiomC{$\alpha_1,\alpha_2 \in (V\cup\Sigma)^+$}
%         \RightLabel{\scriptsize{(2)}}
%         \TrinaryInfC{$\underbrace{\alpha_1\alpha_2}_{\alpha} \derive{\ell'_1+\ell''_1} \underbrace{\beta_1\beta_2}_{\beta}$}
%       \end{prooftree}
%       % Тук със сигурност имаме, че $\abs{\alpha_1} < \abs{\alpha}$ и $\abs{\alpha_2} < \abs{\alpha}$,
%       % което ни позволява да приложим индукционното предположение, защото със сигурност знаем, че
%       % $(\ell'_1,\abs{\alpha_1}) <_{lex} (\ell_1,\abs{\alpha})$ и 
%       % $(\ell''_1,\abs{\alpha_2}) <_{lex} (\ell_1,\abs{\alpha})$.

%       Ако $\ell'_1 > 0$ и $\ell''_1 > 0$, то това означава, че получаваме следния извод:
%       \begin{prooftree}
%         \AxiomC{$\alpha_1 \derive{\ell'_1} \beta_1$}
%         \LeftLabel{\scriptsize{(2)}}
%         \AxiomC{}
%         \LeftLabel{\scriptsize{(0)}}
%         \UnaryInfC{$\alpha_2 \derive{0} \alpha_2$}
%         \LeftLabel{\scriptsize{(2)}}
%         \BinaryInfC{$\alpha \derive{\ell'_1} \beta_1\alpha_2$}
%         \AxiomC{}
%         \RightLabel{\scriptsize{(0)}}
%         \UnaryInfC{$\beta_1 \derive{0} \beta_1$}
%         \AxiomC{$\alpha_2 \derive{\ell''_1} \beta_2$}
%         \RightLabel{\scriptsize{(2)}}
%         \BinaryInfC{$\beta_1\alpha_2 \derive{\ell''_1} \beta$}
%         \AxiomC{$\beta \derive{\ell_2} \gamma$}
%         \RightLabel{\scriptsize{\IndHyp}}
%         \BinaryInfC{$\beta_1\alpha_2 \derive{\ell''_1+\ell_2} \gamma$}
%         \RightLabel{\scriptsize{\IndHyp}}
%         \BinaryInfC{$\alpha \derive{\ell_1+\ell_2} \gamma$}
%       \end{prooftree}

%       Остава да разгледаме случая, когато $\ell'_1 = 0$ или $\ell''_1 = 0$.
%       Нека, без ограничение на общността, да приемем, че $\ell'_1 = 0$.
%       Тогава $\alpha_1 = \beta_1$ и $\ell''_1 = \ell_1$, но пак можем да използваме индукционното предположение, защото $\abs{\alpha_2} < \abs{\alpha}$ и оттук
%       $(\ell''_1,\abs{\alpha_2}) <_{\text{lex}} (\ell_1,\abs{\alpha})$. Така получаваме, че
      
%     \end{itemize}
%   \end{itemize}
% \end{hint}

\begin{proposition}\label{pr:unrestricted-grammar:context-general-step}
  За произволни естествени числа $\ell_1$ и $\ell_2$ е изпълнено, че:
  \begin{prooftree}
    \AxiomC{$\alpha \derive{\ell_1} \lambda\beta\rho$}
    \AxiomC{$\beta \derive{\ell_2} \gamma$}
    \BinaryInfC{$\alpha \derive{\ell_1+\ell_2} \lambda\gamma\rho$}
  \end{prooftree}  
\end{proposition}
\begin{proof}
  Сега ще докажем твърдението с индукция по $\ell_1$.
  \begin{itemize}
  \item
    $\ell_1 = 0$. Тогава всичко е ясно, защото $\alpha = \lambda\beta\rho$.
  \item
    $\ell_1 > 0$. От правилата за извод следва, че имаме следната ситуация:
    \begin{prooftree}
      \AxiomC{$(\alpha',\alpha'') \in R$}
      \AxiomC{$\lambda'\alpha''\rho' \derive{\ell_1-1} \lambda\beta\rho$}
      \BinaryInfC{$\underbrace{\lambda'\alpha'\rho'}_{\alpha} \derive{\ell_1} \lambda\beta\rho$}
    \end{prooftree}

    Използвайки \IndHyp получаваме следния извод:
    \begin{prooftree}
      \AxiomC{$(\alpha',\alpha'') \in R$}
      \AxiomC{$\lambda'\alpha''\rho' \derive{\ell_1-1} \lambda\beta\rho$}
      \AxiomC{$\beta \derive{\ell_2} \gamma$}
      \RightLabel{\scriptsize{\IndHyp}}
      \BinaryInfC{$\lambda'\alpha''\rho' \derive{\ell_1+\ell_2-1} \lambda\gamma\rho$}
      \RightLabel{\scriptsize{правило (1)}}
      \BinaryInfC{$\underbrace{\lambda'\alpha'\rho'}_{\alpha} \derive{\ell_1+\ell_2} \lambda\beta\rho$}
    \end{prooftree}
    
    
  \end{itemize}
  
\end{proof}

% \begin{extra}
%   \begin{remark}\label{rem:unrestricted-grammar:original-def}
%     В повечето учебници авторите дефинират релацията $\derive{\star}_G$ като
%     рефлексивното и транзитивното затваряне на релацията $\derive{}_G$, където
%     \[\alpha \derive{}_G \beta \dff \alpha = \lambda\alpha'\rho\ \&\ (\alpha',\alpha'') \in R\ \&\ \beta = \lambda\alpha''\rho\text{, за някои $\lambda, \rho \in (V\cup\Sigma)^\star$}.\]
%     % Както се вижда от следващото твърдение, релацията $\derive{1}_G$, която дефинирахме по-горе, е точно релацията $\derive{}_G$.
%   \end{remark}
  
  % Както отбелязахме в \Remark{unrestricted-grammar:original-def}, в повечето учебници дават следната дефиниция на релацията $\derive{1}$, която тук ще формулираме като твърдение, което следва от нашите правила за извод.
  % \begin{problem}\label{prob:grammar:alternative-def}
  %   Докажете, че $\alpha \derive{1}_G \beta$ точно тогава, когато $\alpha \derive{}_G \beta$.
  % \end{problem}

  % \begin{problem}
  %   Докажете, че $\derive{\star}_G$ е транзитивното и рефлексивно затваряне на релацията $\derive{}_G$.
  % \end{problem}

  % \begin{hint}
  %   За посоката $(\leftarrow)$, нека $\alpha \derive{}_G \beta$, т.е. $\alpha = \lambda \gamma \rho$, $(\gamma,\gamma') \in R$ и $\beta = \lambda \gamma' \rho$. Тогава
  %   \begin{prooftree}
  %     \AxiomC{$(\gamma,\gamma') \in R$}
  %     \AxiomC{}
  %     \RightLabel{\scriptsize{правило (0)}}
  %     \UnaryInfC{$\delta \gamma \rho \derive{0} \delta \gamma \rho$}
  %     \RightLabel{\scriptsize{правило (1)}}
  %     \BinaryInfC{$\underbrace{\delta \gamma \rho}_{\alpha} \derive{1} \underbrace{\delta \gamma' \rho}_{\beta}$}
  %   \end{prooftree}
  %   За посоката $(\rightarrow)$, нека $\alpha \derive{1}_G \beta$. Тогава имаме следния извод:
  %   \begin{prooftree}
  %     \AxiomC{$(\alpha',\alpha'') \in R$}
  %     \AxiomC{$\lambda\alpha''\rho \derive{0} \lambda\alpha''\rho$}
  %     \BinaryInfC{$\underbrace{\lambda\alpha'\rho}_{\alpha} \derive{1} \underbrace{\lambda\alpha''\rho}_{\beta}$}
  %   \end{prooftree}
  % \end{hint}
% \end{extra}


%%% Local Variables:
%%% mode: latex
%%% TeX-master: "../eai"
%%% End:

\section{Контекстни граматики}
\index{граматика!контекстна}
\mynote{На англ. context-sensitive \cite[стр. 223]{hopcroft1}. Евентуално позволяваме и правилото $S \to \varepsilon$,
ако искаме да включим $\varepsilon$ в езика, като $S$ не се среща в дясна страна на правило.}
Казваме, че $G = (V,\Sigma,R,S)$ е {\bf контекстна граматика}, ако правилата на $G$ са от вида
$\rho A \delta \to \rho \alpha \delta$, където $\rho,\delta \in (V\cup\Sigma)^\star$ и $\alpha \in (V\cup\Sigma)^+$.

\begin{extra}
\begin{example}
  Езикът $L = \{a^nb^nc^n \mid n > 0\}$ е контекстен.
\end{example}
\begin{hint}
  Разгледайте контекстната граматика $G$ зададена със следните правила:
  \mynote{Съобразете, че имаме извода $CB \derive{4}_G BC$.}
  \begin{align*}
    & S \to aSBC\ |\ aBC\\
    & CB \to CZ\\
    & CZ \to WZ\\
    & WZ \to WC\\
    & WC \to BC\\
    & aB \to ab\\
    & bB \to bb\\
    & bC \to bc\\
    & cC \to cc.
  \end{align*}

  Докажете, че за всяко $n > 0$ е изпълено следното:
  \begin{itemize}
  \item
    $S \derive{n} a^n(BC)^n$;
  \item
    $(BC)^n \derive{n-1} B^nC^n$;
  \item
    $aB^n \derive{n} ab^n$;
  \item
    $bC^n \derive{n} bc^n$.
  \end{itemize}
  Оттук лесно можем да докажем, че $L \subseteq \L(G)$.
\end{hint}
\end{extra}
%%% Local Variables:
%%% mode: latex
%%% TeX-master: "../eai"
%%% End:

\section{Регулярни граматики}
\index{граматика!регулярна}
\index{граматика!тип 3}

Сега ще разгледаме граматики с такъв вид правила,
които пораждат точно регулярните (или еквивалентно автоматни) езици.
\mynote{Също така се наричат граматики от тип 3 в йерархията на Чомски \cite[стр. 217]{hopcroft1}. Този вид граматики понякога се нарича и дясно-регулярна граматика.}
Граматиката $G =\CFG$ се нарича {\bf регулярна граматика},
ако всички правила са от вида 
\begin{align*}
  & A \to aB,\\
  % & A \to a,\\
  & A \to \varepsilon,
\end{align*}
за произволни $A, B \in V$ и $a \in \Sigma$.

\begin{lemma}
  За всяка регулярна граматика $G$ съществува НКА $\N$, такъв че $\L(G) = \L(\N)$.
\end{lemma}
\begin{hint}
  Нека $G = \CFG$ и $V = \{A_0,\dots,A_k\}$, където $S = A_0$. Тогава дефинираме $\N$ по следния начин:
  \begin{itemize}
  \item
    $Q \df \{q_0,\dots,q_k\}$;
  \item
    $Q_{\texttt{start}} \df \{q_0\}$;
  \item
    $F \df \{q_i \mid A_i \to \varepsilon\}$;
  \item
    % Релацията на преходите $\Delta$ е дефинирана по следния начин:
    $\Delta(q_i,a) \df \{ q_j\ \mid\ A_i \to aA_j \text{ е правило в граматиката}\}$.
  \end{itemize}
  Докажете, че $\L(\N) = \L(G)$.
\end{hint}

\begin{lemma}
  \mynote{Ясно е, че и двете твърдения може да се формулират и за детерминирани автомати.}
  За всеки ДКА $\A$ съществува регулярна граматика $G$, такава че $\L(\A)~=~\L(G)$.
\end{lemma}
\begin{hint}
  Нека $\A = \FA$ и $Q = \{q_0,\dots,q_k\}$, където $\qstart = q_0$. Тогава дефинираме $G = \CFG$ по следния начин:
  \begin{itemize}
  \item 
    $V \df \{A_0,\dots,A_k\}$;
  \item
    $S \df A_0$;
  \item
    $A_i \to aA_j\ \dff\ \delta(q_i,a) = q_j$;
  \item
    $A_{i} \to \varepsilon\ \dff\ q_{i} \in F$.
  \end{itemize}
  Докажете, че $\L(\A) = \L(G)$.
\end{hint}

\begin{important}
  \begin{theorem}
    Един език е регулярен точно тогава, когато се поражда от регулярна граматика.
  \end{theorem}
\end{important}

\begin{extra2}
\begin{example}
  Да разгледаме отново автомата от \Figure{a2}.
  \begin{figure}[H]
    \begin{center}
      \begin{tikzpicture}[framed,->,>=stealth,thick,node distance=45pt,initial text=начало,scale=0.8, every node/.style={scale=0.8},]
        \tikzstyle{every state}=[circle,minimum size=15pt,auto]
        
        \node[initial,state]      (1) {$q_0$};
        \node[state]              (2) [right of=1]{$q_1$};
        \node[accepting, state]   (3) [right of=2]{$q_2$};
        
        \path 
        (2) edge [loop above]    node [above] {$a$} (2)
        (1) edge [bend left=15]  node [above] {$a$} (2)
        (2) edge [bend left=15]  node [above] {$b$} (3)
        (1) edge [bend right=45] node [below] {$b$} (3)
        (3) edge [loop above]    node [above] {$a,b$} (3);
      \end{tikzpicture}
    \end{center}
    \caption{\scriptsize{$\L(\A) = \L(\mathbf{a^\star b(a+b)^\star})$.}}
  \end{figure}

  Регулярна граматика $G$ за езика $\L(\A)$ можем да дефинираме така:
  \begin{itemize}
  \item
    $\Sigma = \{a,b\}$;
  \item
    На всяко състояние $q_i$ на автомата ще съотвества променливата $A_i$, т.е.
    $V = \{A_0,A_1,A_2\}$;
  \item
    Началната променлива е $A_0$, защото $q_0$ е началното състояние на автомата;
  \item
    Правилата следват дефиницията на $\delta$ функцията:
    \begin{align*}
      & A_0 \to a A_1\ |\ b A_2\\
      & A_1 \to a A_1\ |\ b A_2\\
      & A_2 \to a A_2\ |\ b A_2\ |\ \varepsilon.
    \end{align*}
  \end{itemize}
\end{example}
\end{extra2}

\subsection*{Допълнителни задачи}

\begin{extra}

\begin{problem}
  Граматиката $G = (V, \Sigma, R, S)$ се нарича {\bf обобщено дясно-регулярна},
  ако всички правила са от вида 
  \begin{align*}
    & A \to \omega B,\\
    & A \to \omega
  \end{align*}
  за произволни $A, B \in V$ и $\omega \in \Sigma^\star$.
  Докажете, че един език $L$ е автоматен точно тогава, когато $L$ може да се опише с обобщена дясно-регулярна граматика.
\end{problem}

\begin{problem}
  Граматиката $G = (V, \Sigma, R, S)$ се нарича {\bf обобщено ляво-регулярна},
  ако всички правила са от вида 
  \begin{align*}
    & A \to B\omega,\\
    & A \to \omega
  \end{align*}
  за произволни $A, B \in V$ и $\omega \in \Sigma^\star$.
  Докажете, че един език $L$ е автоматен точно тогава, когато $L$ може да се опише с обобщена ляво-регулярна граматика.
\end{problem}

\end{extra}

%%% Local Variables:
%%% mode: latex
%%% TeX-master: "../eai"
%%% End:

\section{Безконтекстни граматики}
\index{граматика!безконтекстна}
\marginpar{В \cite{papadimitriou} дефиницията е различна. Там $\Sigma \subseteq V$}
\marginpar{На англ. {\em context-free grammar}}
\marginpar{Други срещани наименования на български са {\em контекстно-свободна}, {\em контекстно-независима}}
В Раздел \ref{sect:regular-grammar} въведохме понятието граматика. След това видяхме как можем да опишем регулярните езици
със специален вид граматики, които нарекохме регулярни граматики.
Сега ще разгледаме още един вид граматики, които описват по-широк клас от езици.

\begin{itemize}
\item 
  Една граматика $G = (V, \Sigma, R, S)$ се нарича {\bf безконтекстна}, ако 
  имаме ограничението, че $R \subseteq V\times (V\cup\Sigma)^\star$.
\item
  \index{език!безконтекстен}
  $L$ се нарича {\bf безконтекстен език}, ако съществува безконтекстна граматика $G$, за която 
  $L = \L(G) = \{\omega \in \Sigma^\star \mid S \to^\star_G \omega\}$.
\end{itemize}

\begin{remark}
  Очевидно е, че всяка регулярна граматика е безконтекстна. Следователно, 
  {\em всеки регулярен език е безконтекстен.}
\end{remark}

Като първи пример нека да видим, че това включване е {\em строго}, т.е. съществува безконтекстен език, който не е регулярен.
Да напомним, че вече видяхме, че езикът $L = \{a^nb^n \mid n\in\Nat\}$ не е регулярен.

\begin{example}
  Да разгледаме безконтекстната граматика $G$ зададена със следните правила:
  \begin{align*}
    & S \to aSb \mid \varepsilon.
  \end{align*}
  Лесно се съобразява, че $\L(G) = \{a^nb^n \mid n\in\Nat\}$.
\end{example}

\begin{example}
  Да разгледаме безконтекстната граматика $G$ зададена със следните правила:
  \begin{align*}
    & S \to aSc\ |\  B\\
    & B \to bBc\ |\ \varepsilon.
  \end{align*}
  Лесно се съобразява, че $\L(G) = \{a^nb^kc^{n+k} \mid n,k\in\Nat\}$.
\end{example}

\begin{example}
  Да разгледаме граматика с правила
  \begin{align*}
    & S \to S + S\ |\ S * S\ |\ (S)\ |\ V\\
    & V \to x\ |\ y\ |\ z
  \end{align*}

  Думата $x * y + z$ има две различни дървета на извод.

  Да разгледаме граматика с правила
  \begin{align*}
    & S \to E + S\ |\ E\\
    & E \to V * E\ |\ V\ |\ (S) * E\ |\ (S)\\
    & V \to x\ |\ y\ |\ z
  \end{align*}
  Сега думата $x * y + z$ има само едно дърво на извод.
\end{example}

\begin{example}
  \begin{align*}
    & S \to \texttt{if } S \texttt{ then } S \texttt{ else }S\ |\ \texttt{ if }S \texttt{ then }S\ |\ V\\
    & V \to x\ |\ y\ |\ z
  \end{align*}

  Ние искаме следната граматика:
  \begin{align*}
    & S \to M\ |\ U\\
    & M \to \texttt{if } S \texttt{ then } M \texttt{ else }M\ |\ X\\
    & U \to \texttt{if } S \texttt{ then } S\ |\ \texttt{if } S \texttt{ then } M \texttt{ else }U
  \end{align*}
\end{example}

\begin{example}
  Да разгледаме граматика с правила
  \begin{align*}
    & S \to E\\
    & E \to E + P\ |\ P\\
    & P \to P * N\ |\ N\\
    & N \to (E)\ |\ a.
  \end{align*}
\end{example}

\begin{problem}
  Докажете, че езикът $L = \{a^mb^nc^k\mid m+k \geq n\}$ е безконтекстен.
\end{problem}


\begin{problem}
  Докажете, че езикът $L = \{a^mb^nc^k\mid m+n \geq k\}$ е безконтекстен.
\end{problem}
\begin{proof}
  Да разгледаме граматиката $G$ с правила
  \begin{align*}
    S& \rightarrow aSc\ |\ aS\ |\ B\\
    B& \rightarrow bBc\ |\  bB\ |\ \varepsilon.
  \end{align*}
  
  Лесно се вижда с индукция по $n$, че за всяко $n$ имаме свойствата:
  \marginpar{\ding{45} Докажете!}
  \begin{itemize}
  \item 
    $S \rightarrow^\star a^nSc^n$,
  \item
    $S \rightarrow^\star a^nS$,
  \item
    $B \rightarrow^\star a^nBc^n$,
  \item
    $B \rightarrow^\star b^nB$.
  \end{itemize}
  Комбинирайки горните свойства, можем да видим, че за всяко $n \geq k$,
  \begin{itemize}
  \item 
    $S \rightarrow^\star a^nSc^k$,
  \item
    $B \rightarrow^\star b^nBc^k$.
  \end{itemize}
  За да докажем, че $L \subseteq L(G)$, 
  да разгледаме една дума $\omega \in L$, т.е. $\omega = a^mb^nc^k$, където $m+n \geq k$.
  Имаме два случая:
  \begin{itemize}
  \item 
    $k \leq m$, т.е. $m = k+l$ и $m+n = k+l+n$.
    Тогава имаме изводите:
    \[S \rightarrow^\star a^kSc^k,\ S \rightarrow^\star a^lS,\ S \rightarrow B,\ B \rightarrow^\star b^nB,\ B \rightarrow \varepsilon.\]
    Обединявайки всичко това, получаваме:
    \[S \rightarrow^\star a^mb^nc^k.\]
  \item
    $k > m$, т.е. $k = m+l$, за някое $l > 0$, и $m+n = k+r = m+l+r$, за някое $r$.
    Тогава имаме изводите:
    \[S \rightarrow^\star a^mSc^m,\ S\rightarrow B,\ B\rightarrow^\star b^lBc^l,\ B\rightarrow b^rB,\ B\rightarrow\varepsilon,\]
    и отново получаваме $S \rightarrow^\star a^mb^nc^k$.
  \end{itemize}
  Така доказахме, че $\omega \in \L(G)$.
  
  Сега ще докажем, че $\L(G) \subseteq L$.
  С индукция по дължината на извода $l$,
  ще докажем, че ако $S \stackrel{l}{\rightarrow}\omega$, то $\omega \in M$, където
  \[M = \{a^nSc^k\mid n\geq k\}\cup\{a^nb^mBc^k\mid n+m\geq k\}\cup\{a^nb^mc^k\mid n+m\geq k\}.\]
  
  Ако $l = 0$, то е ясно, че $S \stackrel{0}{\rightarrow} S$ и $S \in M$.

  Нека $l > 0$ и $S \stackrel{l-1}{\rightarrow} \alpha \rightarrow \omega$.
  От {\bf И.П.} имаме, че $\alpha \in M$. Нека $\omega$ се получава от $\alpha$ с прилагане на правилото $C \rightarrow \gamma$.
  Разглеждаме всички варианти за думата $\alpha \in M$ и за правилото $C\rightarrow \gamma$ в граматиката $G$
  за да докажем, че  $\omega \in M$.
  Удобно е да представим всички случаи в таблица.
  \begin{center}
    \begin{tabular}{| c | c | c |}
      \hline
      $\alpha\in M$ & $C \rightarrow \gamma$ & $\omega \in M?$ \\ \hline
      $a^nSc^k$ & $S \rightarrow aSc$ & $a^{n+1}Sc^{k+1}$ \\ \hline
      $a^nSc^k$ & $S \rightarrow aS$ & $a^{n+1}Sc^{k}$ \\ \hline
      $a^nSc^k$ & $S \rightarrow B$ & $a^{n}Bc^{k}$ \\ \hline
      $a^nb^mBc^k$ & $B \rightarrow bBc$ & $a^nb^{m+1}Bc^{k+1}$\\ \hline
      $a^nb^mBc^k$ & $B \rightarrow bB$ & $a^nb^{m+1}Bc^{k}$\\ \hline
      $a^nb^mBc^k$ & $B \rightarrow \varepsilon$ & $a^nb^{m}c^{k}$\\ \hline
    \end{tabular}
  \end{center}
  Във всички случаи се установява, че $\omega \in M$.
  Сега, за всяка дума $\omega \in L(G)$ следва, че
  \[\omega \in \Sigma^\star \cap M = \{a^mb^nc^k\mid m+n \geq k\}.\]
\end{proof}

\begin{problem}
  Докажете, че езикът 
  \[L = \{a^nb^mc^kd^l \mid n+k = m + l\}\]
  е безконтекстен.
\end{problem}
\begin{hint}
  Нека $G_1$ е безконтекстна граматика за езика
  \[L_1 = \{a^nb^mc^k \mid m = n+k\},\]
  където правилата на $G_1$ са
  \[S_1 \to AC,\quad  A \to aAb\ |\ \varepsilon,\quad C \to bCc\ |\ \varepsilon.\]
  Нека $G_2$ е безконтекстна граматика за езика 
  \[L_2 = \{b^mc^kd^l \mid k = m+l\},\]
  където правилата на $G_2$ са
  \[S_2 \to BD,\quad B \to bBc\ |\ \varepsilon,\quad D \to cCd\ |\ \varepsilon.\]
  Тогава граматиката $G$ за $L$ 
  съдържа правилата на граматиките $G_1$ и $G_2$, а също и правилата
  \[S \to aSd\ |\ S_1\ |\ S_2.\]
\end{hint}

\begin{problem}
  Докажете, че езикът 
  \[L = \{a^nb^mc^kd^l \mid n+k \geq m + l\}\]
  е безконтекстен.
\end{problem}

\begin{problem}
  \marginpar{
    $S \to aS \mid aSc \mid aB \mid bB$\\
    $B \to bB \mid bBc \mid \varepsilon$
}
  Докажете, че езикът 
  \[L = \{a^mb^nc^k\mid m+n \geq k + 1\}\]
  е безконтекстен.  
\end{problem}

\begin{problem}
  \label{prob:equal-but-different}
  Докажете, че езикът
  \[L = \{\alpha\beta \in \{a,b\}^\star \mid\ |\alpha| = |\beta|\ \&\ \alpha \neq \beta\}\]
  е безконтекстен.
\end{problem}
\begin{hint}
  Разгледайте граматиката:
  \begin{align*}
    & S \to AB\ |\ BA\\
    & A \to XAX\ |\ a\\
    & B \to XBX\ |\ b\\
    & X \to a\ |\ b.
  \end{align*}
\end{hint}

\begin{problem}
  Докажете, че езикът
  \[L = \{\alpha \sharp \beta \mid \alpha,\beta \in \{a,b\}^\star\ \&\ |\alpha| \neq |\beta| \}\]
  е безконтекстен.
\end{problem}

\begin{problem}
 Докажете, че езикът
 \[L = \{\alpha \sharp \beta \mid \alpha,\beta \in \{a,b\}^\star\ \&\ \alpha \neq \beta \}\]
 е безконтекстен.
\end{problem}
\begin{hint}
  Разгледайте граматиката:
  \begin{align*}
    & S \to AaR\ |\ BbR\ |\ E\\
    & A \to XAX\ |\ bR\sharp\\
    & B \to XBX\ |\ aR\sharp\\
    & E \to XEX\ |\ XR\sharp\ |\ \sharp XR\\
    & R \to XR\ |\ \varepsilon\\
    & X \to a\ |\ b.
  \end{align*}
  Имаме, че за произволни думи $\alpha,\beta,\gamma,\delta \in \{a,b\}^\star$,
  \begin{align*}
    & S \to^\star \alpha b \gamma \sharp \beta a \delta\ \&\ |\alpha| = |\beta|,\\
    & S \to^\star \alpha a \gamma \sharp \beta b \delta\ \&\ |\alpha| = |\beta|, \text{ или}\\
    & S \to^\star \alpha \sharp \beta\ \&\ |\alpha| \neq |\beta|\\
  \end{align*}      
\end{hint}

\begin{problem}
  Нека $L_1$ и $L_2$ са произволни безконтекстни езици.
  Докажете, че:
  \begin{itemize}
  \item 
    $L_1 \cup L_2$ е безконтекстен език;
  \item
    $L_1 \cdot L_2$ е безконтекстен език;
  \item
    $L^\star_1$ е безконтекстен език.
  \end{itemize}
\end{problem}

\begin{problem}
  Да разгледаме граматиката $G$ с правила
  \[S \to AA\ |\ B,\ A \to B\ |\ bb,\ B \to aa\ |\ aB.\]
  Да се намери езика на тази граматика и да се докаже, че граматиката разпознава точно този език.
\end{problem}


%%% Local Variables: 
%%% mode: latex
%%% TeX-master: "../eai"
%%% End: 

\section{Дървета на извод}

\newcommand{\high}{\texttt{height}}
\newcommand{\leaves}{\texttt{leaves}}
\newcommand{\successor}{\texttt{succ}}

\tikzset{
  photon/.style={decorate, decoration={snake}, draw=black}
}

\mynote{Понятието дърво е едно от най-основните в информатиката и може да се дефинира по много различни начини, в зависимост от това за какви цели се използва. Тук на практика следваме \cite{nerode-shore}, защото е важно да имаме наредба между възлите на дървото.}
\begin{itemize}
\item
  За фиксирано $b \in \Nat$, ще разглеждаме думи $\alpha$ и $\beta$ над азбуката $\{0,1,\dots,b-1\}$.
\item
  С $\alpha \preceq \beta$ ще означаваме, че $\alpha$ е префикс на $\beta$, а с $\alpha \prec \beta$,
  че $\alpha$ е \emph{същински} префикс на $\beta$, т.е. $\alpha \preceq \beta\ \&\ \alpha \neq \beta$.
\item
  \index{наредба!лексикографска}
  Ще казваме, че $\alpha$ е лексикографски по-малка от $\beta$, което ще означаваме като $\alpha <_{\texttt{lex}} \beta$, ако
  \[(\exists i < \min\{|\alpha|,|\beta|\})[\ (\forall j < i)[\ \alpha[j] = \beta[j]\ ]\ \&\ \alpha[i] < \beta[i]\ ].\]
\item
  \index{дърво}
  \mynote{Всеки възел в дървото еднозначно се определя от пътя от възела до корена.}
  Непразното множество $T \subseteq \{0,1,\dots,b-1\}^\star$ се нарича {\bf дърво},
  ако $T$ е затворено относно префикси, т.е. $\texttt{Pref}(T) = T$.
  С други думи,
  \[(\forall \alpha)(\forall \beta)[\ \alpha \in T\ \&\ \beta \preceq \alpha\ \implies\ \beta \in T\ ].\]
  \mynote{При нас всички дървета са крайно разклонени.}
\item
  Нека да въведем следните означения:
  \begin{align*}
    & \high(T) \df \max\{\ \abs{\alpha}\ \mid\ \alpha \in T\ \} & \comment\text{височина}\\
    & \successor_T(\alpha) \df \{ \alpha i \mid \alpha i \in T\ \&\ i < b\} & \comment\text{наследниците на }\alpha\\
    & T_\alpha \df \alpha^{-1}(T) & \comment\text{поддървото на }\alpha\\
    & \leaves(T) \df \{ \alpha \in T \mid \successor_T(\alpha) = \emptyset \}. & \comment\text{листата на }T
  \end{align*}

  \begin{figure}[H]
    \centering
    \begin{tikzpicture}
      \coordinate (A) at (0,0);
      \coordinate (B) at (-2,-3);
      \coordinate (C) at (2,-3);
      \coordinate (D) at (0,-1.5);
      \coordinate (E) at (-1,-3);
      \coordinate (F) at (1,-3);
      \draw (A) -- node[above left]{$T$} (B) -- node[below]{$T_\alpha$}(C) -- (A);
      \draw (D) -- (E);
      \draw (D) -- (F);
      \draw [photon] (A) -- node[left]{$\alpha$} (D);
    \end{tikzpicture}
    \caption{Поддърво $T_\alpha$ на $T$.}
  \end{figure}  
\item
  Нека фиксираме граматиката $G = (\Sigma,V,S,R)$.
  С всяко дърво $T$ ще асоциираме функцията $\lambda: T \to V \cup \Sigma \cup \{\varepsilon\}$.
  Нека положим $X_\alpha \df \lambda(\alpha)$.
\item
  \index{дърво на извод}
  \mynote{Също се нарича синтактично дърво. На англ. \emph{parse tree}.}
  Двойката $P = (T,\lambda)$ се нарича {\bf дърво на извод} съвместимо с $G$, ако са изпълнени свойствата:
  \begin{itemize}
  \item
    $T$ е крайно.
  \item
    Ако $\alpha i \in T$, то $\alpha j \in T$ за всяко $j < i$.
  \item
    Ако $\alpha \in T$ и $|\texttt{ext}_T(\alpha)| = k+1$, за някое $k \in \Nat$, то $X_\alpha \in V$,
    като имаме също така и 
    % Освен това, ако $\alpha_0,\dots,\alpha_k$ са всички думи от множеството $\texttt{ext}_T(\alpha)$
    % подредени във възходящ ред относно лексикографската наредба, то имаме, че:
    \mynote{С други думи,
      \[\lambda(\alpha)\to_G\lambda(\alpha 0)\cdots\lambda(\alpha k).\]}
    \[X_\alpha \to_G X_{\alpha 0} X_{\alpha 1} \cdots X_{\alpha k}.\] 
  \end{itemize}
\item
  За дървото на извод $P$, нека
  \[\texttt{root}(P) \df X_\varepsilon.\]
\item
  Нека $\alpha_0, \alpha_1,\dots,\alpha_k$ са всички думи от множеството $\leaves(T)$
  подредени във възходящ ред относно лексикографската наредба. Тогава 
  \[\texttt{yield}(P) \df X_{\alpha_0} X_{\alpha_1}\cdots X_{\alpha_k}.\]
\item
  \mynote{В \cite[стр. 123]{papadimitriou} дават рекурсивна дефиниция на дърво на извод.}
  Нека $P = (T,\lambda)$ е дърво на извод съвместимо с граматиката $G$.
  За всяко $\alpha \in T$, дефинираме $\lambda_\alpha:T_\alpha \to V \cup \Sigma \cup \{\varepsilon\}$ като
  \[\lambda_\alpha(\beta) \df \lambda(\alpha \cdot \beta).\]
\item
  Нека $P = (T,\lambda)$ и $\alpha \in T$. Тогава
  \[P_\alpha \df (T_\alpha, \lambda_\alpha).\]
\end{itemize}

\begin{example}
  Да разгледаме граматиката $G$, където правилата са следните:
  \begin{align*}
    & S \to aS\ |\ aSc\ |\ B\\
    & B \to bB\ |\ bBc\ |\ \varepsilon.
  \end{align*}
  Вече знаем, че $\L(G) = \{a^nb^kc^\ell \mid n+k\geq \ell\}$.
  Да разгледаме дървото на извод $P = (T, \lambda)$, където:

  \begin{framed}
    \begin{figure}[H]
    \qtreecenterfalse
    \Tree [.$\varepsilon$ $0$ [.$1$ $10$ [.$11$ [.$110$ $1100$ [.$1101$ $11010$ ] $1102$ ] ] $12$ ] ]
    \hskip 0.4in
    $\stackrel{\lambda}{\Rightarrow}$
    \hskip 0.4in
    \Tree [.S a [.S a [.S [.B b [.B $\varepsilon$ ] c ] ] c ] ]
    \caption{Дърво на извод за думата $aabcc$.}      
    \end{figure}
  \end{framed}

  Имаме, че:
  \begin{itemize}
  \item
    $\high(T) = 5$;
  \item
    $\leaves(T) = \{0, 10, 1100, 11010, 1102, 12\}$;
  \item
    $\texttt{yield}(P) = aab\varepsilon cc = aabcc$.
  \item
    $\successor_T(110) = \{1100, 1101, 1102\}$;
  \item
    $\lambda(\varepsilon) = S$;
  \item
    $\lambda(0) = a$, $\lambda(1) = S$;
  \item
    Ясно е, че $\lambda(\varepsilon) \to_G \lambda(0)\lambda(1)$;
  \item
    $\lambda(10) = a$, $\lambda(11) = S$, $\lambda(12) = c$;
  \item
    Ясно е, че $\lambda(1) \to_G \lambda(10)\lambda(11)\lambda(12)$;
  \item
    $\lambda(110) = B$;
  \item
    Ясно е, че $\lambda(11) \to_G \lambda(110)$;
  \item
    $\lambda(1100) = b$, $\lambda(1101) = B$, $\lambda(1102) = c$;
  \item
    Ясно е, че $\lambda(110) \to_G \lambda(1100)\lambda(1101)\lambda(1102)$;
  \item
    $\lambda(11010) = \varepsilon$;
  \item
    Ясно е, че $\lambda(1101) \to_G \lambda(11010)$;
  \end{itemize}
  От всичко по-горе следва, че $P = (T,\lambda)$ е дърво на извод за думата $aabcc$ в граматиката $G$.  
\end{example}

\begin{problem}
  Докажете, че:
  \begin{itemize}
  \item
    $T = \texttt{Pref}(\leaves(T))$;
  \item
    $T = T' \iff \leaves(T) = \leaves(T')$.
  \end{itemize}
\end{problem}

\begin{lemma}
  Нека $T \subseteq \{0,\dots,b-1\}^\star$ е крайно дърво. Тогава
  \[ |\leaves(T)| \leq b^{\high(T)}.\]
\end{lemma}
\begin{proof}
  Индукция по $\high(T)$.
  \begin{itemize}
  \item
    Нека $\high(T) = 0$. Тогава е ясно, че $|\leaves(T)| = |\{\varepsilon\}| = 1 \leq b^0$.
  \item
    Нека $\high(T) > 0$.
    За всяко $a \in T$ е ясно, че $\high(T_a) < \high(T)$. Тогава:
    \mynote{В този случай имаме, че $\leaves(T) = \bigcup_{a\in T} (\{a\}\cdot \leaves(T_a))$. Тук гледаме на $a$ и $b$ от една страна като букви в азбука, но и като числа.}
    \begin{align*}
      |\texttt{leaves}(T)| & = \sum_{a \in T}|\texttt{leaves}(T_a)|\\
                           & \leq \sum_{a \in T} b^{\high(T_a)} & \comment\text{от \IndHyp}\\
                           & \leq \sum_{a < b} b^{\high(T_a)}\\
                           & \leq \sum_{a < b} b^{\high(T) -1} & \comment \high(T_a) \leq \high(T)-1 \\
                           & = b^{\high(T)}.
    \end{align*}
  \end{itemize}
\end{proof}

\begin{framed}
  \begin{corollary}
    \label{cor:tree:upper-bound}
    Нека $P = (T,\lambda)$ е дърво на извод съвместимо с $G$. Тогава
    \[|\texttt{yield}(P)| \leq b^{{\high}(P)}.\]
  \end{corollary}
\end{framed}
\begin{proof}
  Следва директно от горното твърдение след като съобразим, че
  \[|\texttt{yield}(P)| \leq |\leaves(P)|.\]
\end{proof}

\begin{framed}
  \begin{lemma}
    Ако $X \derive{\star}_G \beta$, то $X \yield{\star}_G \beta$.
  \end{lemma}  
\end{framed}
\begin{proof}
  Индукция по дължината на извода $X \derive{\ell} \beta$.
  \begin{itemize}
  \item
    $\ell = 0$, т.е. $X \derive{0} X$.
    Тогава е ясно, че $X \yield{\star} X$.
  \item
    \mynote{Знаем, че $k < b$.}
    Нека $\ell > 0$ и $X \derive{\ell} \beta$.
    Според правилата на извод в граматика имаме извода
    \[X \to_G X_0X_1\cdots X_k \derive{\ell-1} \beta.\]
    От \Proposition{grammar:divide} знаем, че съществува разбиване на $\beta$ на $k+1$ части, така че:
    \begin{itemize}
    \item
      $\beta = \beta_0 \cdots \beta_{k}$;
    \item
      $X_i \derive{\ell_i} \beta_i$, за всяко $i = 0,\dots,k$;
    \item
      $\ell-1 = \sum^k_{i=1} \ell_i$.
    \end{itemize}
    От \IndHyp имаме, че $X_i \yield{\star} \beta_i$ за $i \leq k$.
    Получаваме следното:
    \begin{prooftree}
      \AxiomC{$\vdots$}
      \LeftLabel{\scriptsize{\IndHyp}}
      \UnaryInfC{$X_i \yield{\star} \beta_i \text{ за }i \leq k$}
      \AxiomC{$X \to_G X_0\cdots X_k$}
      \BinaryInfC{$X \yield{\star} \underbrace{\beta_0\cdots\beta_k}_{\beta}$}
    \end{prooftree}
  \end{itemize}
\end{proof}

\begin{framed}
  \begin{lemma}
    Ако $X \yield{\star}_G \gamma$, то $X \derive{\star}_G \gamma$.
  \end{lemma}
\end{framed}
\begin{proof}
  Индукция по височината $\ell$ на дървото на извод за $X \yield{\ell} \gamma$.
  \begin{itemize}
  \item
    Нека $\ell = 0$. Това означава, че $X \yield{0} X$. Ясно е, че $X \derive{\star} X$.
  \item
    Нека $\ell > 0$. Тогава имаме следното:
    \begin{prooftree}
      \AxiomC{$X \to_G X_1\cdots X_n$}
      \AxiomC{$X_i \yield{\ell_i} \gamma_i\text{ за }i=1,\dots,n$}
      \RightLabel{\scriptsize{($\ell = 1 + \max\{\ell_1,\dots,\ell_n\})$}}
      \BinaryInfC{$X \yield{\ell} \gamma_1\cdots\gamma_n$}
    \end{prooftree}
    Тогава от И.П. получаваме, че за всяко $i < k$ е изпълнено $X_i \derive{\star} \gamma_i$.
    Получаваме, че:
    \begin{prooftree}
      \AxiomC{$X \to_G X_0 \cdots X_{k-1}$}
      \AxiomC{$\vdots$}
      \RightLabel{\scriptsize{\IndHyp}}
      \UnaryInfC{$X_i \derive{\star} \gamma_i\text{ за }i<k$}
      \UnaryInfC{$X_0 \cdots X_{k-1} \derive{\star} \gamma_0\cdots\gamma_{k-1}$}
      \BinaryInfC{$X \derive{\star} \underbrace{\gamma_0\cdots\gamma_{k-1}}_{\gamma}$}
    \end{prooftree}
  \end{itemize}
\end{proof}

\begin{framed}
\begin{theorem}
  Нека $G$ е произволна безконтекстна граматика.
  Тогава $\alpha \in \L(G)$ точно тогава, когато съществува дърво на извод $P$, съвместимо с $G$, с корен $S$ за думата $\beta$.
\end{theorem}  
\end{framed}



\begin{problem}
  Нека $P = (T,\lambda)$ е дърво на извод за думата $\alpha \in (V\cup\Sigma)^\star$ в граматиката $G$.
  Докажете, че съществува извод
  \mynote{Ясно е, че $\ell < b^{\high(T)}$.}
  $X_\varepsilon \stackrel{\ell}{\to}_G \alpha$, където
  \[\ell \leq \sum_{i < \high(T)}b^i.\]
\end{problem}

\begin{problem}
  Нека $P = (T,\lambda)$ е дърво на извод съвместимо с $G$.
  Докажете, че ако $\alpha \preceq \beta$, то $\texttt{yield}(T_\beta)$ е инфикс на $\texttt{yield}(T_\alpha)$.
\end{problem}
\begin{hint}
  Нека $\beta = \alpha \cdot \gamma$. Тогава имаме следното дърво:
  
  \begin{figure}[H]
    \centering
    \begin{tikzpicture}
      \coordinate (A) at (0,0);
      \coordinate (B) at (-3,-4);
      \coordinate (C) at (3,-4);
      \coordinate (D) at (0,-1.5);
      \coordinate (E) at (-2,-4);
      \coordinate (F) at (2,-4);
      \coordinate (G) at (0,-2.5);
      \coordinate (H) at (-1,-4);
      \coordinate (I) at (1,-4);
      
      \draw (A) -- node[left]{$T$} (B) -- (C) -- (A);
      
      \draw (D) -- node[left]{$T_\alpha$}(E);
      \draw (D) -- (F);
      
      \draw (G) -- node[left]{$T_\beta$}(H);
      \draw (G) -- (I);
      
      \draw [photon] (A) -- node[left]{$\alpha$} (D);
      \draw [photon] (D) -- node[below left]{$\gamma$} (G);
    \end{tikzpicture}
    \caption{}
  \end{figure}
\end{hint}

\begin{problem}
  Нека $P = (T,\lambda)$ и $P' = (T',\lambda')$ са дървета на извод съвместими с граматиката $G$ и нека
  имаме думи $\omega_1, \omega_2 \in \Sigma^\star$, за които
  \mynote{Това означава, че съществува $\alpha \in T$, за което $\lambda(\alpha) = \texttt{root}(P') = \lambda'(\varepsilon)$.}
  \[\texttt{yield}(P) = \omega_1 \cdot \texttt{root}(P') \cdot \omega_2.\]
  Дефинираме $P'' = (T'',\lambda'')$ по следния начин:
  \begin{itemize}
  \item
    \mynote{Ясно е, че $\alpha \in T$ и $\alpha \in \alpha\cdot T'$, но понеже $X_\alpha = X'_\varepsilon$, то нямаме проблем.}
    $T'' \df T \cup \{\alpha\} \cdot T'$;
  \item
    Сега трябва да дефинираме функцията $\lambda'' : T'' \to V \cup \Sigma \cup \{\varepsilon\}$.
    
    $\lambda''(\gamma) \df
    \begin{cases}
      \lambda(\gamma), & \text{ако }\gamma \in T\\
      \lambda'(\beta), & \text{ако }\gamma = \alpha \cdot \beta\ \&\ \beta \in T'
    \end{cases}$

    \mynote{Тук трябва да сме внимателни, защото двата случая на дефиницията на $\lambda''$  се засичат за $\gamma = \alpha$.
      Понеже $\alpha \in \texttt{leaves}(T)$ и $\lambda(\alpha) = \lambda'(\varepsilon)$, то функцията $\lambda''$ е коректно дефинирана.}
  \end{itemize}
  \index{дърво на извод!конкатенация}
  Тогава $P''$ е дърво на извод съвместимо с граматиката $G$ и
  \[\texttt{yield}(P'') = \omega_1 \cdot \texttt{yield}(P') \cdot \omega_2.\]
  Нека в такъв случай да означаваме $P'' = P \odot P'$ и ще казваме, че $P''$ е конкатенацията на $P$ и $P'$.
  
  \begin{figure}[H]
    \begin{subfigure}[t]{0.5\textwidth}
      \centering
      \begin{tikzpicture}
        \coordinate (A) at (0,0);
        \coordinate (B) at (-2,-2.5);
        \coordinate (C) at (2,-2.5);
        \coordinate (D) at (0,-2.5);
        \coordinate (E) at (-1,-3.5);
        \coordinate (F) at (1,-3.5);
        
        \draw (A) -- node[above left]{$T$} (B) -- (C) -- (A);
        \draw (D) -- node[left]{$T'$} (E) -- (F) -- (D);
        
        \draw [photon] (A) -- node[left]{$\alpha$} (D);
      \end{tikzpicture}
      \caption{Дървото $T''$}
      \end{subfigure}
      $\stackrel{\lambda}{\Rightarrow}$
      \begin{subfigure}[t]{0.5\textwidth}
        \centering
        \begin{tikzpicture}
          \node (A) at (0,0) {${\scriptstyle X_\varepsilon}$};
          \coordinate (B) at (-2,-2.5);
          \coordinate (C) at (2,-2.5);
          \node (D) at (0,-2.5) {${\scriptstyle X_\alpha}$};
          \coordinate (E) at (-1,-3.5);
          \coordinate (F) at (1,-3.5);
          
          \draw (A) -- node[above left]{${\scriptstyle P\ =\ }$} (B) -- node[below]{$\omega_1$}(D) -- node[below]{$\omega_2$}(C) -- (A);
          \draw (D) -- node[left]{${\scriptstyle P'\ =\ }$}(E) -- node[below]{$\texttt{yield}(P')$}(F) -- (D);
          
          % \draw [photon] (A) -- node[left]{$\alpha$} (D);
        \end{tikzpicture}
        \caption{$P'' = P \odot P'$}
      \end{subfigure}
      \caption{Конкатенация на дървета}
    \end{figure}
  \end{problem}
  
  Сега да разгледаме частния случай, когато разгледаме $P$ вместо $P'$ в горните условия. Тогава имаме, че $X_\varepsilon = X_\alpha$.
  Дефинираме $n$-тата степен на дървото $P$ по следния начин:
  \begin{itemize}
\item
  $P^{(0)} \df (T_0,\lambda_0)$, където $T_0 = \{\varepsilon\}$ и $\lambda_0(\varepsilon) = \texttt{root}(P)$;
\item
  $P^{(n+1)} \df P^{(n)} \odot P$.
\end{itemize}

\begin{problem}
  Докажете, че $P \odot (P \odot P) = (P \odot P) \odot P$.
\end{problem}


\begin{problem}
  Докажете, че $P^{(n+k)} = P^{(n)} \odot P^{(k)}$.
\end{problem}

\begin{framed}
  \begin{problem}
    \label{prob:tree:iteration}
    Нека $\texttt{yield}(P) = \omega_1 \cdot \texttt{root}(P) \cdot \omega_2$, за някои думи $\omega_1, \omega_2 \in \Sigma^\star$.
    Докажете, че за всяко естествено число $i$ е изпълнено, че:
    \[\texttt{yield}(P^{(i)}) = \omega^i_1 \cdot \texttt{root}(P) \cdot \omega^i_2.\]
  \end{problem}
\end{framed}
\begin{hint}
  Картинката за $i = 2$ изглежда така:

  \begin{figure}[H]
    \begin{subfigure}[t]{0.5\textwidth}
      \centering
      \begin{tikzpicture}
        \coordinate (A) at (0,0);
        \coordinate (B) at (-2,-2.5);
        \coordinate (C) at (2,-2.5);
        \coordinate (D) at (0,-2.5);
        \coordinate (E) at (-2,-5);
        \coordinate (F) at (2,-5);
        \coordinate (G) at (0,-5);
        
        \draw (A) -- node[above left]{$T$} (B) -- (C) -- (A);
        \draw (D) -- node[left]{$T$} (E) -- (F) -- (D);
        
        \draw [photon] (A) -- node[left]{$\alpha$} (D);
        \draw [photon] (D) -- node[left]{$\alpha$} (G);
      \end{tikzpicture}
      \caption{}
    \end{subfigure}
    $\stackrel{\lambda}{\Rightarrow}$
    \begin{subfigure}[t]{0.5\textwidth}
      \centering
      \begin{tikzpicture}
        \node (A) at (0,0) {${\scriptstyle X_\varepsilon}$};
        \coordinate (B) at (-2,-2.5);
        \coordinate (C) at (2,-2.5);
        \node (D) at (0,-2.5) {${\scriptstyle X_\varepsilon}$};
        \coordinate (E) at (-2,-5);
        \coordinate (F) at (2,-5);
        \node (G) at (0,-5) {${\scriptstyle X_\varepsilon}$};
        
        \draw (A) -- node[above left]{${\scriptstyle P\ =\ }$} (B) -- node[below]{$\omega_1$}(D) -- node[below]{$\omega_2$}(C) -- (A);
        \draw (D) -- node[above left]{${\scriptstyle P\ =\ }$}(E) -- node[below]{$\omega_1$} (G) -- node[below]{$\omega_2$} (F) -- (D);
      \end{tikzpicture}
      \caption{$\texttt{yield}(P^{(2)}) = \omega^2_1 \cdot \texttt{root}(P) \cdot \omega^2_2$}
    \end{subfigure}
    \caption{Степенуване на $P$.}
  \end{figure}
\end{hint}

\begin{problem}
  Нека $P = (T,\lambda)$ е дърво на извод съвместимо с граматиката $G$ и нека разгледаме думата $\alpha \in T$.
  % Дефинираме $P \setminus P_\alpha = (T',\lambda')$ по следния начин:
  Дефинираме $\uppercut{P}{\alpha} = (T',\lambda')$ по следния начин:
  \begin{itemize}
  \item
    \mynote{Съобразете, че $\alpha \in \texttt{front}(T')$.}
    $T' = T \setminus \{ \gamma \in T\mid \alpha \prec \gamma\}$, т.е. взимаме същинските разширения на $\alpha$
  \item
    $\lambda'(\gamma) = \lambda(\gamma)$ за $\gamma \in T'$.
  \end{itemize}
  Докажете, че $\uppercut{P}{\alpha}$ е дърво на извод съвместимо с граматиката $G$ и 
  \[P = (\uppercut{P}{\alpha}) \odot (\lowercut{P}{\alpha}).\]
\end{problem}
\begin{hint}
  Имаме следната картинка:

  \begin{figure}[H]
    \begin{subfigure}[t]{0.5\textwidth}
      \centering
      \begin{tikzpicture}
        \coordinate (A) at (0,0);
        \coordinate (B) at (-2,-2.5);
        \coordinate (C) at (2,-2.5);
        \coordinate (D) at (-0.9,-2.5);
        \coordinate (E) at (0.9,-2.5);
        \coordinate (F) at (0,-1.5);
        
        \coordinate (G) at (0,-2.3);
        \coordinate (H) at (-0.9,-3.3);
        \coordinate (I) at (0.9,-3.3);
        
        \draw (A) -- node[above left]{${\scriptstyle T'\ =\ }$}(B) -- (D) -- (F) -- (E) -- (C) -- (A);
        \draw (G) -- node[left]{${\scriptstyle T_\alpha\ =\ }$} (H) -- (I) -- (G);
        
        \draw [photon] (A) -- node[left]{$\alpha$} (F);
      \end{tikzpicture}
      \caption{$T_\alpha = \alpha^{-1}(T)$}
    \end{subfigure}
    $\stackrel{\lambda}{\Rightarrow}$
    \begin{subfigure}[t]{0.5\textwidth}
      \centering
      \begin{tikzpicture}
        \node (A) at (0,0) {${\scriptstyle X_\varepsilon}$};
        \coordinate (B) at (-2,-2.5);
        \coordinate (C) at (2,-2.5);
        \coordinate (D) at (-0.9,-2.5);
        \coordinate (E) at (0.9,-2.5);
        \node (F) at (0,-1.5) {${\scriptstyle X_\alpha}$};
        
        \node (G) at (0,-2.3) {${\scriptstyle X_\alpha}$};
        \coordinate (H) at (-0.9,-3.3);
        \coordinate (I) at (0.9,-3.3);
        
        \draw (A) -- node[above left]{${\scriptstyle \uppercut{P}{\alpha}}\ =\ $}(B) -- (D) -- (F) -- (E) -- (C) -- (A);
        \draw (G) -- node[left]{${\scriptstyle \lowercut{P}{\alpha}}\ =\ $} (H) -- (I) -- (G);
      \end{tikzpicture}
      \caption{$\texttt{yield}(\uppercut{P}{\alpha}) = \omega_1 \cdot \texttt{root}(\lowercut{P}{\alpha}) \cdot \omega_2$}
    \end{subfigure}
  \end{figure}
\end{hint}


\subsection*{Най-ляв извод в граматика}
\index{граматика!най-ляв извод}
В нашата дефиниция на извод, изборът върху коя променлива да приложим правило от граматиката е недетерминистичен.
В някои случаи, за нас ще бъде важно винаги да правим детерминистичен избор на това върху коя променлива прилагаме правило.

За две думи $\alpha,\beta \in (V\cup\Sigma)^\star$, дефинираме {\bf най-ляв извод} в граматиката $G$, $\alpha \lderive{\ell} \beta$, по следния начин:
\begin{prooftree}
  \AxiomC{}
  \UnaryInfC{$\alpha \lderive{0} \alpha$}
\end{prooftree}

\begin{prooftree}
  \AxiomC{$A \to_G \gamma$}
  \AxiomC{$\alpha \in \Sigma^\star$}
  \BinaryInfC{$\alpha A \beta \lderive{1} \alpha \gamma \beta$}
\end{prooftree}

\begin{prooftree}
  \AxiomC{$\alpha \lderive{1} \gamma$}
  \AxiomC{$\gamma \lderive{\ell} \beta$}
  \BinaryInfC{$\alpha \lderive{\ell+1} \beta$}
\end{prooftree}

\begin{lemma}
  За всяка безконтекстна граматика $G$, променлива $A \in V$  и дума $\alpha \in (V\cup\Sigma)^\star$,
  \[A \lderive{\star} \alpha\text{ точно тогава, когато } A \derive{\star} \alpha.\]
\end{lemma}
\begin{hint}
  Очевидно е, че ако $A \lderive{\star} \alpha$, то $A \derive{\star} \alpha$.
  За другата посока, достатъчно е да се докаже, че ако $P$ е дърво на извод с корен $A$ и $\texttt{yield}(P) = \alpha$,
  то $A \lderive{\star} \alpha$.

  Индукция по $\texttt{high}(P)$.
\end{hint}



%%% Local Variables:
%%% mode: latex
%%% TeX-master: "../eai"
%%% End:

\section{Извод върху синтактично дърво}

\mynote{Не знам да има учебник, в който формално да е въведена тази релация. Тя е удобна най-вече за решаване на задачи, както и за доказателството на лемата за покачването.}
\begin{definition}
  За произволна безконтекстна граматика $G$, дефинираме релацията $X \yield{\ell} \alpha$, където $X \in V \cup \Sigma$ и $\alpha \in (V\cup\Sigma)^\star$, по следния начин:
  \begin{important}
    \begin{prooftree}
      \AxiomC{}
      \RightLabel{\scriptsize{правило (0)}}
      \UnaryInfC{$X \yield{0} X$}
    \end{prooftree}
    
    \begin{prooftree}
      \AxiomC{$X \to_G X_1\cdots X_n$}
      \AxiomC{$X_1 \yield{\ell_1} \gamma_1$}
      \AxiomC{$\cdots$}
      \AxiomC{$X_n \yield{\ell_n} \gamma_n$}
      \LeftLabel{\scriptsize{($\ell = \sup\{\ell_1,\dots,\ell_n\})$}}
      \RightLabel{\scriptsize{правило (1)}}
      \QuaternaryInfC{$X \yield{\ell+1} \gamma_1\cdots\gamma_n$}
    \end{prooftree}
  \end{important}
\end{definition}

Да напомним, че по дефиниция, ако $n = 0$, то $X_1\cdots X_n = \varepsilon$.
Освен това, понеже $\sup(A)$ означава най-малкото естествено число по-голямо от всички елементи на $A$, то $\sup(\emptyset) = 0$.
Това означава, че ако в граматиката имаме правилото $A \to_G \varepsilon$, то според правило (1)
получаваме, че $A \yield{1} \varepsilon$.

\mynote{ Съобразете, че имаме:
\begin{prooftree}
  \AxiomC{$X \to_G \gamma$}
  \UnaryInfC{$X \yield{1} \gamma$.}
\end{prooftree}
Обърнете внимание също, че тази релация е рефликсивна, но не е транзитивна!}

Да дефинираме $\yield{\star}$ по следния начин:
\[X \yield{\star} \gamma \dff (\exists \ell\in\Nat)[X \yield{\ell} \gamma].\]

Релацията $\yield{\star}$ е удобна, когато не се интересуваме от реда, по който прилагаме правилата за извод в една безконтекстна граматика.

\begin{lemma}
  Нека $G$ е безконтекстна граматика, $X \in V \cup \Sigma$ и $\beta \in (V \cup \Sigma)^\star$.
  Тогава ако $X \derive{\star} \beta$, то $X \yield{\star} \beta$.
\end{lemma}  
\begin{proof}
  С пълна индукция по $\ell$ ще докажем, че ако $X \derive{\ell} \beta$, то $X \yield{\star} \beta$.
  \begin{itemize}
  \item
    $\ell = 0$, т.е. $X \derive{0} X$.
    Тогава е ясно, че $X \yield{\star} X$.
  \item
    Нека $\ell > 0$ и $X \derive{\ell} \beta$.
    Според правилата на извод в граматика имаме извода

    \begin{prooftree}
      \AxiomC{$X \to_G X_1\cdots X_k$}
      \AxiomC{$X_1\cdots X_k \derive{\ell-1} \beta$}
      % \RightLabel{\scriptsize{правило (1)}}
      \BinaryInfC{$X \derive{\ell} \beta$}
    \end{prooftree}

    \mynote{Естествено, че е възможно някои $X_i$ да са букви от $\Sigma$. Тогава $\beta_i = X_i$ и $X_i \derive{0}_G \beta_i$.}
    Щом имаме, че $X_1\cdots X_k \derive{\ell-1} \beta$, от \Proposition{grammar:divide} знаем, че съществува разбиване на $\beta$ на $k+1$ части, така че:
    \begin{itemize}
    \item
      $\beta = \beta_1 \cdots \beta_{k}$;
    \item
      $X_i \derive{\ell_i} \beta_i$, за всяко $i = 1,\dots,k$;
    \item
      $\ell-1 = \sum^k_{i=1} \ell_i$.
    \end{itemize}
    Понеже $\ell_i < \ell$ за всяко $i = 1,\dots,k$, получаваме следното:
    \begin{prooftree}
      \AxiomC{$X \to_G X_1\cdots X_k$}
      \AxiomC{$X_1 \derive{\ell_1} \beta_1$}
      \RightLabel{\scriptsize{\IndHyp}}
      \UnaryInfC{$X_1 \yield{\star} \beta_1$}
      \AxiomC{$\cdots$}
      \AxiomC{$X_k \derive{\ell_k} \beta_k$}
      \RightLabel{\scriptsize{\IndHyp}}
      \UnaryInfC{$X_k \yield{\star} \beta_k$}
      \RightLabel{\scriptsize{правило (1)}}
      \QuaternaryInfC{$X \yield{\star} \underbrace{\beta_1\cdots\beta_k}_{\beta}$}
    \end{prooftree}
  \end{itemize}
\end{proof}

\begin{lemma}
  Нека $G$ е безконтекстна граматика, $X \in V \cup \Sigma$ и $\gamma \in (V \cup \Sigma)^\star$.
  Тогава ако $X \yield{\star} \gamma$, то $X \derive{\star} \gamma$.
\end{lemma}
\begin{proof}
  С пълна индукция по $\ell$ ще докажем, че ако $X \yield{\ell} \gamma$, то $X \derive{\star} \gamma$.
  \begin{itemize}
  \item
    Нека $\ell = 0$. Това означава, че $X \yield{0} X$. Ясно е, че $X \derive{\star} X$.
  \item
    Нека $\ell > 0$. Тогава имаме следното:
    \begin{prooftree}
      \AxiomC{$X \to_G X_1\cdots X_n$}
      \AxiomC{$X_1 \yield{\ell_1} \gamma_1$}
      \AxiomC{$\cdots$}
      \AxiomC{$X_n \yield{\ell_n} \gamma_n$}
      \RightLabel{\scriptsize{($\ell = 1 + \sup\{\ell_1,\dots,\ell_n\})$}}
      \QuaternaryInfC{$X \yield{\ell} \underbrace{\gamma_1\cdots\gamma_n}_{\gamma}$}
    \end{prooftree}
    Понеже $\ell_i < \ell$ за всяко $i = 1,\dots,n$, получаваме следното:
    \begin{prooftree}
      \AxiomC{$X \to_G X_1 \cdots X_{n}$}
      \AxiomC{$X_1 \yield{\ell_1} \gamma_1$}
      \RightLabel{\scriptsize{\IndHyp}}
      \UnaryInfC{$X_1 \derive{\star} \gamma_1$}
      \AxiomC{$\cdots$}
      \AxiomC{$X_n \yield{\ell_n} \gamma_n$}
      \RightLabel{\scriptsize{\IndHyp}}
      \UnaryInfC{$X_n \derive{\star} \gamma_n$}
      \RightLabel{\scriptsize{(\ShortProposition{unrestricted-grammar:concat})}}
      \TrinaryInfC{$X_1 \cdots X_{n} \derive{\star} \gamma_1\cdots\gamma_{n}$}
      % \RightLabel{\scriptsize{правило (1)}}
      \BinaryInfC{$X \derive{\star} \underbrace{\gamma_1\cdots\gamma_{n}}_{\gamma}$}
    \end{prooftree}
  \end{itemize}
\end{proof}

Комбинирайки предишните две леми получаваме следната теорема.
\begin{framed}
  \begin{theorem}\label{th:grammar:yield-derive-equivalent}
    Нека $G$ е безконтекстна граматика, $X \in V \cup \Sigma$ и $\gamma \in (V \cup \Sigma)^\star$.
    Тогава $X \yield{\star} \gamma$ точно тогава, когато $X \derive{\star} \gamma$.
    В частност,
    \[\L(G) = \{\alpha \in \Sigma^\star \mid S \yield{\star}\alpha\}.\]
  \end{theorem}  
\end{framed}

Следващото твърдение ни казва, че има пряка връзка между релацията $\yield{\star}$ и синтактичните дървета.

\begin{important}
  \begin{proposition}\label{pr:yield-relation:parse-tree}
    Нека $G$ е безконтекстна граматика. Тогава
    $X \yield{\ell} \gamma$ точно тогава, когато съществува дърво на извод $P$ съвместимо с $G$, за което
    $\texttt{root}(P) = X$, $\texttt{yield}(P) = \gamma$ и $\texttt{height}(P) = \ell$.
  \end{proposition}
\end{important}
\begin{hint}
  Индукция по $\ell$.
\end{hint}


%%% Local Variables:
%%% mode: latex
%%% TeX-master: "../eai"
%%% End:

\newpage
\subsection{Примерни задачи}

\begin{extra}

Да напомним, че в Раздел~\ref{sect:regular:minimisation} дефинирахме език $\L_\A(q)$.
Сега ще дефинираме език $\L_G(A)$, за произволна променлива $A$.
\[\L_G(A) \df \{\alpha \in \Sigma^\star \mid A \yield{\star}\alpha\}.\]
Ясно е, че $\L(G) = \L_G(S)$.
Също така, да дефинираме апроксимациите $\L^\ell_G(A)$ на $\L_G(A)$ по следния начин:
\[\L^\ell_G(A) \df \{\alpha \in \Sigma^\star \mid A \yield{\leq \ell} \alpha\}.\]
Следните свойства са ясни:
\begin{itemize}
\item
  $\L^0_G(A) = \emptyset$;
\item
  $\L^\ell_G(A)$ е краен език за всяко $\ell$;
\item
  $\L^\ell_G(A) \subseteq \L^{\ell+1}_G(A)$ за всяко $\ell$;
\item
  $\L_G(A) = \bigcup_{\ell\geq 0}\L^\ell_G(A)$.  
\end{itemize}

Следната характеризация на езиците $\L^\ell_G(A)$ ще е удобна за нас, когато искаме да докажем,
че една безконтекстна граматика разпознава даден език.

\begin{proposition}\label{pr:grammar:yield-approximation}
  Нека $G$ е произволна безконтекстна граматика и $A$ е променлива в $G$.
  Тогава имаме следното:
  \begin{align*}
    \L^0_G(A) & = \emptyset\\
    \L^{\ell+1}_G(A) & = \bigcup\{\{\alpha_1\}\cdot \L^\ell_G(A_1) \cdots \{\alpha_n\} \cdot \L^\ell_G(A_n) \cdot \{\alpha_{n+1}\} \mid A \to_G \alpha_1A_1\cdots\alpha_nA_n\alpha_{n+1}\}.
  \end{align*}
\end{proposition}
\begin{proof}
  Пълна индукция по $\ell$. Твърдението очевидно е изпълнено за $\ell = 0$.
  Нека $\alpha \in \L^{\ell+1}_G(A)$, т.е. $A \yield{\leq \ell+1} \alpha$. Имаме следния извод:
  \begin{prooftree}
    \AxiomC{$A \to_G \alpha_1B_1\cdots\alpha_n B_n\alpha_{n+1}$}
    \AxiomC{$B_1 \yield{\leq\ell} \beta_1$}
    \AxiomC{$\cdots$}
    \AxiomC{$B_n \yield{\leq \ell} \beta_n$}
    \QuaternaryInfC{$A \yield{\leq\ell+1} \underbrace{\alpha_1\beta_1\cdots\alpha_n\beta_n\alpha_{n+1}}_{\alpha}$}
  \end{prooftree}
  Понеже $B_i \yield{\leq \ell} \beta_i$, то от \IndHyp имаме, че $\beta_i \in \L^\ell_G(B_i)$.
  Тогава 
  \[\alpha \in \bigcup\{\{\alpha_1\}\cdot \L^\ell_G(A_1) \cdots \{\alpha_n\} \cdot \L^\ell_G(A_n) \cdot \{\alpha_{n+1}\} \mid A \to_G \alpha_1A_1\cdots\alpha_nA_n\alpha_{n+1}\}\]

  Обратно, нека сега $\alpha = \alpha_1\beta_1\cdots\alpha_n\beta_n\alpha_{n+1}$, където $A \to_G \alpha_1B_1\cdots\alpha_nB_n\alpha_{n+1}$,
  $\beta_i \in \L^\ell_G(B_i)$ и $\alpha = \alpha_1\beta_1\cdots\alpha_n\beta_n\alpha_{n+1}$.
  Получаваме следния извод:
  \begin{prooftree}
    \AxiomC{$A \to_G \alpha_1B_1\cdots\alpha_nB_n\alpha_{n+1}$}
    \AxiomC{$\beta_1 \in \L^\ell_G(B_1)$}
    \RightLabel{\scriptsize{\IndHyp}}
    \UnaryInfC{$B_1 \yield{\leq \ell }\beta_1 $}
    \AxiomC{$\cdots$}
    \AxiomC{$\beta_n \in \L^\ell_G(B_n)$}
    \RightLabel{\scriptsize{\IndHyp}}
    \UnaryInfC{$B_n \yield{\leq \ell }\beta_n$}
    \QuaternaryInfC{$A \yield{\leq \ell+1} \underbrace{\alpha_1\beta_1\cdots\alpha_n\beta_n\alpha_{n+1}}_{\alpha}$}
  \end{prooftree}
\end{proof}

\begin{corollary}\label{cor:grammar:yield-approximation}
  Нека $G$ е произволна безконтекстна граматика и $A$ е променлива в $G$. Тогава
  \[\L_G(A) = \bigcup\{\{\alpha_1\}\cdot \L_G(A_1) \cdots \{\alpha_n\} \cdot \L_G(A_n) \cdot \{\alpha_{n+1}\} \mid A \to_G \alpha_1A_1\cdots\alpha_nA_n\alpha_{n+1}\}.\]
\end{corollary}


Като първи пример, нека да видим, че съществува безконтекстен език, който не е регулярен.
Да напомним, че вече видяхме, че езикът $L = \{a^nb^n \mid n\in\Nat\}$ не е регулярен.
\begin{example}\label{ex:grammar:anbn}
  Да разгледаме безконтекстната граматика $G$ зададена със следните правила:
  \begin{align*}
    & S \to aSb \mid \varepsilon.
  \end{align*}
  Тогава
  \begin{align*}
    & \L^0_G(S) = \emptyset\\
    & \L^1_G(S) = \{a\} \cdot \emptyset \cdot \{b\} \cup \{\varepsilon\} = \{\varepsilon\}\\
    & \L^{2}_G(S) = \{a\} \cdot \{\varepsilon\} \cdot \{b\} \cup \{\varepsilon\} = \{\varepsilon, ab\}\\
    & \L^{3}_G(S) = \{a\} \cdot \{\varepsilon, ab\} \cdot \{b\} \cup \{\varepsilon\} = \{\varepsilon,ab,aabb\} = \{a^nb^n \mid n < 3\}\\
    & \vdots
  \end{align*}
  Лесно се съобразява, че
  \[\L^\ell_G(S) = \{a^nb^n \mid n < \ell\}.\]
  Заключаваме, че:
  \[\L(G) = \L_G(S) = \bigcup_{\ell}\L^\ell_G(S) = \{a^nb^n \mid n\in\Nat\}.\]
\end{example}

\begin{example}
  Да разгледаме безконтекстната граматика $G$ зададена със следните правила:
  \begin{align*}
    & S \to aSc\ |\  B\\
    & B \to bBc\ |\ \varepsilon.
  \end{align*}
  Да видим защо $\L(G) = \{a^nb^kc^{n+k} \mid n,k\in\Nat\}$.
  Първо ще докажем \emph{коректност} на граматиката. Това означава, че $G$ не генерира думи извън езика, т.е.
  $\L_G(S) \subseteq \{a^nb^kc^{n+k} \mid n,k\in\Nat\}$. За да направим това обаче, трябва да знаем също така и $\L_G(B)$.
  Формално казано, трябва да докажем, че за всяко $\ell$ е изпълнено следното:
  \begin{align}
    \L^\ell_G(S) & \subseteq \{a^nb^kc^{n+k} \mid n,k\in\Nat\} \label{eq:nknk1}\\
    \L^\ell_G(B) & \subseteq \{b^kc^k \mid k \in \Nat\} \label{eq:nknk2}.
  \end{align}
  Това ще направим с индукция по $\ell$.
  Да напомним, че според \Proposition{grammar:yield-approximation} имаме следните връзки:
  \begin{align*}
    & \L^0_G(S) = \emptyset\\
    & \L^{\ell+1}_G(S) = \{a\} \cdot \L^\ell_G(S) \cdot \{c\} \cup \L^\ell_G(B)\\
    & \L^0_G(B) = \emptyset\\
    & \L^{\ell+1}_G(B) = \{b\} \cdot \L^\ell_G(B) \cdot \{c\} \cup \{\varepsilon\}.
  \end{align*}
  Очевидно е, че \Property{eq:nknk1} и \Property{eq:nknk2} са изпълнени за $\ell = 0$.
  Да примем, че имаме \Property{eq:nknk1} и \Property{eq:nknk2} за някое $\ell$.
  \mynote{Обърнете внимание, че не можем да докажем \Property{eq:nknk1} независимо от \Property{eq:nknk2}.}
  Ще докажем свойствата и за $\ell+1$.
  \begin{itemize}
  \item
    Първо, нека $\alpha \in \L^{\ell+1}_G(S)$. Имаме два случая.
    \begin{itemize}
    \item
      Нека $\alpha \in \{a\} \cdot \L^\ell_G(S) \cdot \{c\}$. От \IndHyp следва, че
      $\alpha = a^{n+1}b^kc^{n+k+1}$ за някои естествени числа $n$ и $k$.
      Тогава е ясно, че $\alpha \in \{a^nb^kc^{n+k} \mid n,k\in\Nat\}$.
    \item
      Нека $\alpha \in  \L^\ell_G(B)$. От \IndHyp следва, че
      $\alpha \in \{b^kc^k \mid k \in \Nat\} \subseteq \{a^nb^kc^{n+k} \mid n,k\in\Nat\}$.
    \end{itemize}
  \item
    Второ, нека $\alpha \in \L^{\ell+1}_G(B)$. Имаме два случая.
    \begin{itemize}
    \item
      Нека $\alpha \in \{b\} \cdot \L^\ell_G(B) \cdot \{c\}$. От \IndHyp следва, че
      $\alpha = b^{k+1}c^{k+1}$ за някое естествено число $k$.
      Тогава е ясно, че $\alpha \in \{b^{k}c^{k} \mid k\in\Nat\}$.
    \item
      Нека $\alpha = \varepsilon$. В този случай също е ясно, че $\alpha \in \{b^{k}c^{k} \mid k\in\Nat\}$.
    \end{itemize}
  \end{itemize}
  Оттук заключаваме, че $\L_G(S) \subseteq \{a^nb^kc^{n+k} \mid n,k\in\Nat\}$.
  
  Сега да разгледаме пълнота на граматиката, което означава, че
  всяка дума от езика се генерира от $G$. С други думи, $\{a^nb^kc^{n+k} \mid n,k\in\Nat\} \subseteq \L_G(S)$.
  Първо ще докажем, че
  \begin{equation}
    \label{eq:4}
      \{b^kc^k \mid k \in \Nat\} \subseteq \L_G(B).
    \end{equation}
    Това ще направим с \emph{пълна} индукция по дължината на думата.
    Да разгледаме произволна дума $\alpha \in \{b^kc^k \mid k \in \Nat\}$.
    \begin{itemize}
    \item
      Нека $|\alpha| = 0$, т.е. $\alpha = \varepsilon$.
      Ясно е, че $\alpha \in \L_G(B)$.
    \item
      Нека $|\alpha| > 0$, т.е. $\alpha = b^{k+1}c^{k+1}$.
      От \IndHyp за \Property{eq:4} следва, че $\alpha \in \{b\} \cdot \L_G(B) \cdot \{c\} \subseteq \L_G(B)$.
    \end{itemize}
    Вече сме готови да докажем, че:
    \begin{equation}
      \label{eq:9}
      \{a^nb^kc^{n+k} \mid n,k \in \Nat\} \subseteq \L_G(S).
    \end{equation}
    Това включване пак ще докажем с \emph{пълна} индукция по дължината на думата.
    Да разгледаме произволна дума $\alpha \in L$. 
    \begin{itemize}
    \item
      Нека $|\alpha| = 0$, т.е. $\alpha = \varepsilon$.
      Ясно е, че $\alpha \in \L_G(B) \subseteq \L_G(S)$.
    \item
      Нека сега $|\alpha| > 0$, т.е. $\alpha = a^nb^kc^{n+k}$, където $n > 0$ или $k > 0$. Да разгледаме два случая.
      \begin{itemize}
      \item
        Нека $n = 0$. Тогава $\alpha = b^kc^k$ и $k > 0$. Тогава от \Property{eq:4}
        следва, че $\alpha \in \L_G(B) \subseteq \L_G(S)$.
      \item                   
        Нека $n > 0$. Тогава от \IndHyp за \Property{eq:9} следва, че
        $\alpha \in \{a\} \cdot \L_G(S) \cdot \{c\} \subseteq \L_G(S)$.
      \end{itemize}
    \end{itemize}
    Доказахме коректност и пълнота на граматиката и следователно $\L(G) = \{a^nb^kc^{n+k} \mid n,k\in\Nat\}$.
  \end{example}
  
  
  \begin{example}
    Нека да видим защо езикът $L = \{a^mb^nc^k\mid m+n \geq k\}$ е безконтекстен.
    Да разгледаме граматиката $G$ с правила
    \begin{align*}
      S& \rightarrow aSc\ |\ aS\ |\ B\\
      B& \rightarrow bBc\ |\  bB\ |\ \varepsilon.
    \end{align*}
    От \Proposition{grammar:yield-approximation} имаме, че:
    \begin{align*}
      \L^0_G(S) & = \emptyset\\
      \L^{\ell+1}_G(S) & = \{a\} \cdot \L^\ell_G(S) \cdot \{c\} \cup \{a\}\cdot \L^\ell_G(S) \cup \L^\ell_G(B)\\
      \L^0_G(B) & = \emptyset\\
      \L^{\ell+1}_G(B) & = \{b\} \cdot \L^\ell_G(B) \cdot \{c\} \cup \{b\} \cdot \L^\ell_G(B) \cup \{\varepsilon\}.
    \end{align*}
    Да предположим, че за произволно естествено число $\ell$ е изпълнено следното:
    \mynote{Тези две свойства ще бъдат нашето \IndHyp. Очевидно е, че те са изпълнени за $\ell = 0$.}
    \begin{align}
      \L^\ell_G(S) & \subseteq \{a^nb^mc^k \mid n+m \geq k\} \\
      \L^\ell_G(B)  & \subseteq \{ b^mc^k \mid m \geq k\}. 
    \end{align}
    Ще докажем, че
    \begin{align*}
      \L^{\ell+1}_G(S) & \subseteq \{a^nb^mc^k \mid n+m \geq k\}\\
      \L^{\ell+1}_G(B)  & \subseteq \{ b^mc^k \mid m \geq k\}.
    \end{align*}
    За първото включване, да разгледаме произволна дума $\alpha \in \L^{\ell+1}_G(S)$. Имаме три случая:
    \begin{itemize}
    \item
      Ако $\alpha \in \L^\ell_G(B)$, то от \IndHyp имаме, че
      \[\alpha \in \{b^mc^k \mid m \geq k\} \subseteq \{a^nb^mc^k \mid n+m \geq k\}.\]
    \item
      Ако $\alpha \in \{a\} \cdot \L^{\ell}_G(S)$, то от \IndHyp имаме, че
      \[\alpha \in \{a^{n+1}b^mc^k \mid n+m \geq k\} \subseteq \{a^nb^mc^k \mid n+m \geq k\}.\]
    \item
      Ако $\alpha \in \{a\} \cdot \L^{\ell}_G(S) \cdot \{c\}$, то от \IndHyp имаме, че
      \[\alpha \in \{a^{n+1}b^mc^{k+1} \mid n+m \geq k\} \subseteq \{a^nb^mc^k \mid n+m \geq k\}.\]
    \end{itemize}
    За второто включване, нека $\alpha \in \L^{\ell+1}_G(B)$. Имаме три случая за думата $\alpha$.
    \begin{itemize}
    \item
      Нека $\alpha \in \{b\} \cdot \L^\ell_G(B) \cdot \{c\}$. Тогава от \IndHyp имаме, че:
      \[\alpha \in \{b^{m+1}c^{k+1} \mid m \geq k\} \subseteq \{b^mc^k \mid  m \geq k\}.\]
    \item
      Нека $\alpha \in \{b\} \cdot \L^\ell_G(B)$. Тогава от \IndHyp имаме, че:
      \[\alpha \in \{b^{m+1}c^{k} \mid m \geq k\} \subseteq \{b^mc^k \mid m \geq k\}.\]
    \item
      Нека $\alpha \in \{\varepsilon\}$. Тогава е ясно, че имаме $\alpha \in \{b^mc^k \mid m \geq k\}$.
    \end{itemize}  
    
    \mynote{Така доказахме \emph{коректност} на граматиката.}
    Заключаваме, че
    \begin{align*}
      \L_G(S) & = \bigcup_\ell\L^\ell_G(S) \subseteq \{a^nb^mc^k \mid n+m \geq k\}\\
      \L_G(B) & = \bigcup_\ell\L^\ell_G(B) \subseteq \{a^nb^mc^k \mid n+m \geq k\}.
    \end{align*}
    
    \mynote{Сега ще докажем \emph{пълнота} на граматиката. Тук ще използваме представянията на $\L_G(S)$ и $\L_G(B)$ според \Corollary{grammar:yield-approximation}.}
    Сега трябва да докажем обратните включвания, а именно:
    \begin{align}
      & \{a^nb^mc^k \mid n+m \geq k\} \subseteq \L_G(S) \label{eq:anbmck:S}\\
      & \{b^mc^k \mid m \geq k\} \subseteq \L_G(B). \label{eq:anbmck:B}
    \end{align}
    
    Трябва да започнем първо със \Property{eq:anbmck:B}.
    Да разгледаме произволна дума $\alpha = b^mc^k$. Трябва да докажем, че $\alpha \in \L_G(B)$.
    Ще направим това с индукция по $m$.
    \begin{itemize}
    \item
      Нека $m = 0$. Това означава, че $\alpha = \varepsilon$. Ясно е, че $\varepsilon \in \L_G(B)$.
    \item
      Нека $m > 0$. Тук имаме два подслучая.
      \begin{itemize}
      \item
        Нека $m = k$. Тогава $\alpha = b \gamma c$ и имаме, че $\gamma = b^{m-1}c^{k-1}$.
        Можем да приложим \IndHyp за $\gamma$ и следователно $\gamma \in \L_G(B)$.
        Получаваме, че $\alpha \in \{b\} \cdot \L_G(B) \cdot \{c\} \subseteq \L_G(B)$.
      \item
        Нека $m > k$. Тогава $\alpha = b \gamma$ и имаме, че $\gamma = b^{m-1}c^k$.
        Можем да приложим \IndHyp за $\gamma$ и следователно $\gamma \in \L_G(B)$.
        Получаваме, че $\alpha \in \{b\} \cdot \L_G(B)\subseteq \L_G(B)$.
      \end{itemize}
    \end{itemize}
    Сега преминаваме към \Property{eq:anbmck:S}.
    Да разгледаме произволна дума $\alpha = a^nb^mc^k$. Трябва да докажем, че $\alpha \in \L_G(S)$.
    Ще направим това с индукция по $n$.
    \begin{itemize}
    \item
      Нека $n = 0$. Тогава $\alpha = b^mc^k$ и $m \geq k$.
      От \Property{eq:anbmck:B} следва, че $\alpha \in \L_G(B) \subseteq \L_G(S)$.
    \item
      Нека $n > 0$. Имаме два подслучая.
      \begin{itemize}
      \item
        Нека $n + m = k$. Тогава $\alpha = a\gamma c$ и $\gamma = a^{n-1}b^mc^{k-1}$.
        Можем да приложим \IndHyp за $\gamma$ и следователно $\gamma \in \L_G(S)$.
        Получаваме, че $\alpha \in \{a\} \cdot \L_G(S) \cdot \{c\} \subseteq \L_G(S)$.
      \item
        Нека $n + m > k$. Тогава $\alpha = a \gamma$ и $\gamma = a^{n-1}b^m c^k$.
        Можем да приложим \IndHyp за $\gamma$ и следователно $\gamma \in \L_G(S)$.
        Получаваме, че $\alpha \in \{a\} \cdot \L_G(S) \subseteq \L_G(S)$.
      \end{itemize}
    \end{itemize}
  \end{example}

\begin{problem}
  Докажете, че езикът $L = \{a^nb^mc^kd^\ell \mid n+k = m + \ell\}$ е безконтекстен.
\end{problem}
\begin{hint}
  Да разгледаме произволна дума от вида $\omega = a^n b^m c^k d^\ell$.
  Имаме два случая.
  \begin{itemize}
  \item
    Ако $n > \ell$, тогава имаме еквивалентността: $\omega \in L \iff (n-\ell) + k = m$.
  \item
    Ако $n \leq \ell$, тогава имаме еквивалентността: $\omega \in L \iff k = m + (\ell- n)$.
  \end{itemize}
  Това наблюдение ни подсказва, че трябва да разгледаме следните езици:
  \begin{align*}
    & L_1 = \{a^nb^mc^k \mid m = n+k\},\\
    & L_2 = \{b^mc^kd^\ell \mid k = m+\ell\}.
  \end{align*}
  Така получаваме, че
  \[L = \{a^n \cdot \omega \cdot d^n \mid n\in\Nat\ \&\ \omega \in L_1 \cup L_2\}.\]
  $L_1$ е безконтекстен език, защото може да се опише с безконтекстната граматика $G_1$ със следните правила:
  \[S_1 \to AC,\quad  A \to aAb\ |\ \varepsilon,\quad C \to bCc\ |\ \varepsilon.\]
  $L_1$ също е безконтекстен език, защото може да се опише с безконтекстната граматика $G_2$ със следните правила:
  \[S_2 \to BD,\quad B \to bBc\ |\ \varepsilon,\quad D \to cCd\ |\ \varepsilon.\]
  Тогава безконтекстната граматика $G$ за $L$ 
  съдържа правилата на граматиките $G_1$ и $G_2$, а също и правилата
  \[S \to aSd\ |\ S_1\ |\ S_2.\]
\end{hint}

\begin{problem}
  \label{prob:equal-but-different}
  \mynote{Ние вече знаем, че този език не е регулярен}
  Докажете, че езикът $L = \{\alpha\beta \in \{a,b\}^\star \mid\ |\alpha| = |\beta|\ \&\ \alpha \neq \beta\}$ е безконтекстен.
\end{problem}
\begin{hint}
  Да разгледаме една произволна дума $\omega$, където $\omega = \alpha\beta$, $|\alpha| = |\beta|$ и $\alpha \neq \beta$.
  Знаем, че същестува индекс $i < |\alpha|$, такъв че думата $\omega$ може да се представи така:
  \[\omega = \alpha\slice{:i} \cdot \alpha\slice{i} \cdot \alpha\slice{i+1:} \cdot \beta\slice{:i} \cdot \beta\slice{i} \cdot \beta\slice{i+1:},\]
  където $\alpha\slice{i} \neq \beta\slice{i}$.

  Нека $n = |\alpha| = |\beta|$ и да представим $n$ като $n = i+1+k$. Имаме два случая.
  \begin{itemize}
  \item
    Ако $k \geq i$, то можем да преставим $\omega$ по следния начин:
    \[\omega = \underbrace{\alpha\slice{:i}}_{\text{дълж. }i} \cdot \alpha\slice{i} \cdot \underbrace{\alpha\slice{i+1:2i+1}}_{\text{дълж. }i} \cdot \underbrace{\alpha\slice{2i+1:} \cdot \beta\slice{:i}}_{\text{дълж. }k} \cdot \beta\slice{i} \cdot \underbrace{\beta\slice{i+1:}}_{\text{дълж. }k}.\]
  \item
    Нека $k < i$, то можем да преставим $\omega$ по следния начин:
    \[\omega = \underbrace{\alpha\slice{:i}}_{\text{дълж. }i} \cdot \alpha\slice{i} \cdot \underbrace{\alpha\slice{i+1:} \cdot \beta\slice{:i-k}}_{\text{дълж. }i} \cdot \underbrace{\beta\slice{i-k:i}}_{\text{дълж. }k} \cdot \beta\slice{i} \cdot \underbrace{\beta\slice{i+1:}}_{\text{дълж. }k}.\]
  \end{itemize}
  От тези представяния на $\omega$ е ясно, че можем да изразим езика $L$ по следния начин:
  \[L = L_a \cdot L_b \cup L_b \cdot L_a,\]
  където:
  \begin{align*}
    & L_a = \{\alpha a \beta \mid \alpha,\beta \in \{a,b\}^\star\ \&\ |\alpha| = |\beta|\}\\
    & L_b = \{\alpha b \beta \mid \alpha,\beta \in \{a,b\}^\star\ \&\ |\alpha| = |\beta|\}.
  \end{align*}
  Сега разгледайте безконтекстната граматика $G$ със следните правила:
  \begin{align*}
    & S \to AB\ |\ BA\\
    & A \to XAX\ |\ a\\
    & B \to XBX\ |\ b\\
    & X \to a\ |\ b.
  \end{align*}
  Лесно се съобразява, че:
  \begin{align*}
    & L_a = \L_G(A)\\
    & L_b = \L_G(B)\\
    & L = \L_G(S).
  \end{align*}
\end{hint}

\begin{problem}
 Докажете, че езикът $L = \{\alpha \sharp \beta \mid \alpha,\beta \in \{a,b\}^\star\ \&\ \alpha \neq \beta \}$ е безконтекстен.
\end{problem}
\begin{hint}
  Разгледайте граматиката:
  \begin{align*}
    & S \to AaR\ |\ BbR\ |\ E\\
    & A \to XAX\ |\ bR\sharp\\
    & B \to XBX\ |\ aR\sharp\\
    & E \to XEX\ |\ XR\sharp\ |\ \sharp XR\\
    & R \to XR\ |\ \varepsilon\\
    & X \to a\ |\ b.
  \end{align*}
  Имаме, че за произволни думи $\alpha,\beta,\gamma,\delta \in \{a,b\}^\star$,
  \begin{align*}
    & S \derive{\star} \alpha b \gamma \sharp \beta a \delta\ \&\ |\alpha| = |\beta|,\\
    & S \derive{\star} \alpha a \gamma \sharp \beta b \delta\ \&\ |\alpha| = |\beta|, \text{ или}\\
    & S \derive{\star} \alpha \sharp \beta\ \&\ |\alpha| \neq |\beta|.
  \end{align*}      
\end{hint}

\begin{problem}
  Докажете, че езикът $L = \{a^nb^mc^kd^\ell \mid n+k \geq m + \ell\}$ е безконтекстен.
\end{problem}

\begin{problem}
  \mynote{
    $S \to aS \mid aSc \mid aB \mid bB$\\
    $B \to bB \mid bBc \mid \varepsilon$
  }
  Докажете, че езикът $L = \{a^mb^nc^k\mid m+n \geq k + 1\}$ е безконтекстен.  
\end{problem}

\begin{problem}
  Докажете, че езикът $L = \{\alpha \sharp \beta \mid \alpha,\beta \in \{a,b\}^\star\ \&\ |\alpha| \neq |\beta| \}$ е безконтекстен.
\end{problem}

\begin{problem}
  Да разгледаме граматиката $G$ с правила
  \[S \to AA\ |\ B,\ A \to B\ |\ bb,\ B \to aa\ |\ aB.\]
  Да се намери езика на тази граматика и да се докаже, че граматиката разпознава точно този език.
\end{problem}

\end{extra}

%%% Local Variables:
%%% mode: latex
%%% TeX-master: "../eai"
%%% End:

\newpage
\section{Лема за покачването}

В този раздел ще разгледаме едно твърдение, което ще ни даде критерий за проверка кога един език не е безконтекстен. Ще започнем с няколко помощни твърдения.

\begin{proposition}\label{pr:pumping:ground}
  За произволни променливи $A$, $B$ и думи $\alpha_1,\alpha_2, \alpha_3$ е изпълнено:
  \begin{prooftree}
    \AxiomC{$A \yield{\ell_1}\alpha_1B\alpha_3$}
    \AxiomC{$B \yield{\ell_2} \alpha_2$}
    \BinaryInfC{$A \yield{ \leq \ell_1+\ell_2} \alpha_1\alpha_2\alpha_3$}
  \end{prooftree}
\end{proposition}
\begin{hint}
  Формално доказателството протича с индукция по $\ell_1$.
  
  \begin{figure}[H]
    \begin{subfigure}[t]{0.5\textwidth}
      \centering
      \begin{tikzpicture}[scale=0.8]
        \node (A) at (0,0) {$A$};
        \coordinate (B) at (-2,-2.5);
        \coordinate (C) at (2,-2.5);
        \coordinate (D) at (-0.9,-2.5);
        \coordinate (E) at (0.9,-2.5);
        \node (F) at (0,-1.5) {$B$};
        
        \node (G) at (0,-2.6) {$B$};
        \coordinate (H) at (-0.9,-3.6);
        \coordinate (I) at (0.9,-3.6);
        
        \draw (A) -- (B) -- node[below]{$\alpha_1$} (D) -- (F) -- (E) -- node[below]{$\alpha_3$} (C) -- (A);
        \draw (G) -- (H) -- node[below]{$\alpha_2$}(I) -- (G);
        
        % \draw [photon] (A) -- node[left]{$\alpha$} (F);
      \end{tikzpicture}
      \caption{\scriptsize{Синтактични дървета за изводите $A \yield{\ell_1}\alpha_1B\alpha_3$ и $B \yield{\ell_2} \alpha_2$}}
    \end{subfigure}
    % $\stackrel{\lambda}{\Rightarrow}$
    ~
    \begin{subfigure}[t]{0.5\textwidth}
      \centering
      \begin{tikzpicture}[scale=0.8]
        \node (A) at (0,0) {$A$};
        \coordinate (B) at (-2,-2.5);
        \coordinate (C) at (2,-2.5);
        \coordinate (D) at (-0.9,-2.5);
        \coordinate (E) at (0.9,-2.5);
        \node (F) at (0,-1.5) {$B$};
        
        % \coordinate (G) at (0,-2.3) {${\scriptstyle X_\alpha}$};
        % \coordinate (H) at (-0.9,-3.3);
        % \coordinate (I) at (0.9,-3.3);
        
        \draw (A) -- (B) -- node[below]{$\alpha_1$} (D) -- (F) -- (E) -- node[below]{$\alpha_3$} (C) -- (A);
        \draw (D) -- node[below]{$\alpha_2$} (E);
      \end{tikzpicture}
      \caption{\scriptsize{Съединяваме двете дървета}}
      % \caption{$\texttt{yield}(\uppercut{P}{\alpha}) = \omega_1 \cdot \texttt{root}(\lowercut{P}{\alpha}) \cdot \omega_2$}
    \end{subfigure}
  \end{figure}
\end{hint}

\begin{proposition}\label{pr:pumping:iteration}
  За произволна променлива $A$ и произволни думи $\beta$ и $\gamma$ е изпълнено:
  \begin{prooftree}
    \AxiomC{$A \yield{\ell} \beta A \gamma$}
    \AxiomC{$i \in \Nat$}
    \BinaryInfC{$A \yield{\leq i \cdot \ell} \beta^i A \gamma^i$}
  \end{prooftree}
\end{proposition}
\begin{hint}
  Формално доказателството трябва да следва индукция по $i$.
  Понеже това доказателство не крие изненади, ще се задоволим с един пример. Нека $P$ да бъде синтактично дърво съответстващо на извода $A \yield{\ell} \beta A \gamma$.
  Тогава можем да получим синтактично дърво $P^{(2)}$  съответстващо на $A \yield{\leq 2\ell} \beta^2A\gamma^2$ по следния начин.
  \begin{figure}[H]
    \begin{subfigure}[t]{0.3\textwidth}
      \centering
      \begin{tikzpicture}[scale=0.8]
        \node (A) at (0,0) {\footnotesize{$A$}};
        \coordinate (B) at (-2,-2.2);
        \coordinate (C) at (2,-2.2);
        \coordinate (S) at (-1,-2.2);
        \coordinate (T) at (1,-2.2);
        \node (L) at (0,-1) {$A$};
        
        \draw (A) -- (B) -- node[below]{$\beta$}(S) -- (L) -- (T) -- node[below]{$\gamma$}(C) -- (A);
      \end{tikzpicture}
      \caption{\scriptsize{Синтактично дърво $P$ за извода $A \yield{\ell} \beta A\gamma$}}
    \end{subfigure}
    ~
    \begin{subfigure}[t]{0.3\textwidth}
      \centering
      \begin{tikzpicture}[scale=0.8]
        \node       (A) at (0,0) {\footnotesize{$A$}};
        \coordinate (B) at (-2,-2.2);
        \coordinate (C) at (2,-2.2);
        \coordinate (S) at (-1,-2.2);
        \coordinate (T) at (1,-2.2);
        \node       (L) at (0,-1) {\footnotesize{$A$}};
        
        \coordinate (E) at (-0.7,-2.5);
        \coordinate (F) at (0.7,-2.5);
        
        
        \node       (H) at (0,-2) {\footnotesize{$A$}};
        \node       (X) at (0,-3) {\footnotesize{$A$}};
        \coordinate (D) at (-2,-4.2);
        \coordinate (G) at (2,-4.2);
        \coordinate (E) at (-1,-4.2);
        \coordinate (F) at (1,-4.2);
        
        \draw (A) -- (B) -- node[below]{$\beta$}(S) -- (L) -- (T) -- node[below]{$\gamma$}(C) -- (A);
        \draw (H) -- (D) -- node[below]{$\beta$}(E) -- (X) -- (F) -- node[below]{$\gamma$}(G) -- (H);
      \end{tikzpicture}
      \caption{\scriptsize{Съединяваме две копия на $P$}}
    \end{subfigure}
    ~
    \begin{subfigure}[t]{0.3\textwidth}
      \centering
      \begin{tikzpicture}[scale=0.8]
        \node       (A) at (0,0) {\footnotesize{$A$}};
        \coordinate (B) at (-2,-2.2);
        \coordinate (C) at (2,-2.2);
        \coordinate (S) at (-1,-2.2);
        \coordinate (T) at (1,-2.2);
        \node       (L) at (0,-1) {\footnotesize{$A$}};
        
        \coordinate (E) at (-0.7,-2.2);
        \coordinate (F) at (0.7,-2.2);
        
        \node       (X) at (0,-2) {\footnotesize{$A$}};
        \coordinate (D) at (-1.82,-3.2);
        \coordinate (G) at (1.82,-3.2);
        \coordinate (E) at (-1,-3.2);
        \coordinate (F) at (1,-3.2);
        
        \draw (A) -- (B) -- node[below]{$\beta$}(S) -- (L) -- (T) -- node[below]{$\gamma$}(C) -- (A);
        \draw (L) -- (D) -- node[below]{$\beta$}(E) -- (X) -- (F) -- node[below]{$\gamma$}(G) -- (L);
      \end{tikzpicture}
      \caption{\scriptsize{Получаваме синтактично дърво $P^{(2)}$ за извода $A \yield{\leq 2\ell} \beta^2 A \gamma^2$}}
    \end{subfigure}
  \end{figure}
\end{hint}

\begin{proposition}\label{pr:pumping:lower-bound}
  Нека $G$ е безконтекстна граматика и 
  \mynote{Числото $b$ е максималната разклоненост, която можем да получим за синтактично дърво на извод в граматикта $G$.}
  \[b \df \max\{\ |\gamma| \mid A \to \gamma \text{ е правило в }G\ \}.\]
  Тогава:
  \begin{itemize}
  \item 
    Ако $A \yield{\leq\ell} \alpha$, то $|\alpha| \leq b^\ell$.
  \item
    Ако $\alpha \in \L(G)$ и $|\alpha| > b^\ell$, то $S \yield{\geq \ell+1} \alpha$.
  \end{itemize}
\end{proposition}

\begin{proof}

%   \begin{proposition}
%   \label{cor:tree:upper-bound}
%   \mynote{Това следствие ще бъде важно за нас, когато разгледаме лемата за покачването за безконтекстни езици.}
%   Нека $P = (T,\lambda)$ е дърво на извод съвместимо с $G$. Тогава
%   \[|\texttt{yield}(P)| \leq b^{{\high}(P)}.\]
% \end{proposition}
% \begin{proof}
%   Следва директно от горното твърдение след като съобразим, че
%   \[|\texttt{yield}(P)| \leq |\leaves(P)|.\]
% \end{proof}

  
  Това твърдение е на практика следствие от \Problem{tree:leaves-upper-bound} в комбинация с \Proposition{yield-relation:parse-tree}.
  \begin{itemize}
  \item
    Да разгледаме синтактично дърво $P = (T,\lambda)$ за извода $A \yield{\leq\ell} \alpha$.
    Според \Problem{tree:leaves-upper-bound} имаме, че $\abs{\alpha} = \abs{\leaves(T)}\leq b^{\high(T)} = b^{\ell}$.
  \item
    Нека $\alpha \in \L(G)$ и $\abs{\alpha} > b^\ell$. Това означава, че $S \yield{\star} \alpha$.
    Ако допуснем, че $S \yield{\leq \ell} \alpha$, то бихме имали, че $\abs{\alpha} \leq b^\ell$, което е противоречие.
    Заключаваме, че $S \yield{\geq \ell+1} \alpha$.
  \end{itemize}
\end{proof}

\begin{important}
  \begin{lemma}[за покачването (безконтекстни езици)]
    \index{лема за покачването!безконтекстни езици}
    \label{lem:context-free:pumping} 
    \mynote{\cite[стр. 125]{sipser3}, \cite[стр. 125]{hopcroft1}, \cite[стр. 275]{hopcroft2}, \cite[стр. 148]{kozen}.}
    За всеки безконтекстен език $L$ съществува $p>0$, такова
    че ако $\alpha\in L$, за която $\abs{\alpha} \geq p$, то съществува разбиване на думата на пет части, $\alpha=xyuvw$,
    \mynote{Ще казваме, че $p$ е константа на покачването. Тук отново да напомним, че $0 \in \Nat$ и $xy^0uv^0w = xuw$.}
    за което е изпълнено:
    \begin{enumerate}[1)]
    \item
      $\abs{yv}\geq 1$,
    \item
      $\abs{yuv}\leq p$, и
    \item
      $(\forall i\in\Nat)[xy^iuv^iw\in L]$.
    \end{enumerate}
  \end{lemma}
\end{important}
\begin{proof}
  \mynote{Числото $b$ е максималната разклоненост на всяко дърво на извод за дума изводима от граматиката $G$.}
  Нека $G$ е безконтекстна граматика за езика $L$. Да положим
  \[p \df b^{\abs{V}+1}, \text{ където }b \df \max\{\ |\gamma| \mid A \to \gamma \text{ е правило в }G\ \}.\]
  Ще покажем, че този избор на $p$ е удачен за удовлетворяването на условията на лемата. Ще наричаме $p$ константа на покачването за граматиката $G$.
  \mynote{Важното е да вземем дума $\alpha \in \L(G)$, за която $\abs{\alpha} > b^{\abs{V}}$.}
  Да разгледаме произволна дума $\alpha \in \L(G)$, за която $\abs{\alpha} \geq p$.
  Понеже $\L(G)$ е безкраен език, то със сигурност можем да намерим такава дума.
  Щом $S \yield{\star} \alpha$, според \Proposition{pumping:lower-bound}, винаги имаме $S \yield{\ell}\alpha$ за $\ell \geq \abs{V}+1$.
  Нека измежду всички синтактични дървета $P = (T,\lambda)$ за извода $S \yield{\ell} \alpha$ сме избрали такова с \emph{минимален} брой елементи в $T$.
  Да фиксираме \emph{максимален} път $\pi$ в $T$, т.е. дума $\pi \in T$ и $|\pi| = \ell$.
  Чрез функцията $\lambda$ пътя $\pi$ определя редицата $X_0,X_1,\dots,X_\ell$, като $X_i = \lambda(\rho)$, където $\rho \prec \pi$ и $\abs{\rho} = i$. Ясно е, че $X_i \in V$ за $i < \ell$ и $X_\ell \in \Sigma \cup \{\varepsilon\}$.
  Щом $\ell \geq \abs{V}+1$, то това означава, че по този път $\pi$
  \mynote{Принципът на Дирихле е известен също и като принципа на чекмеджетата.}
  се срещат $\ell \geq \abs{V}+1$ на брой променливи. От принципа на Дирихле следва, че трябва да има повторения на променливи по този път. 
  Във вече фиксираното синтактично дърво $P$ можем да изберем това срещане на двойка повтарящи се променливи по пътя $\pi$, както на \Figure{pumping:partition}, което е възможно най-близо до края на пътя.
  Това означава, че можем да разбием извода $S \yield{\ell} \alpha$ като $S \yield{\star} xAw$ и $A \yield{\leq \abs{V}+1} \gamma$, където $\alpha = x\gamma w$.
  Сега според нашия избор, изводът $A \yield{\leq \abs{V}+1} \gamma$ може да се разбие като $A \yield{\leq \abs{V}+1} yAv$ и $A \yield{\leq \abs{V}} u$,
  където $\gamma = yuv$.
  Да обобщим. Можем да разбием думата $\alpha$ на пет части като $\alpha = xyuvw$ с изводите:
  \begin{enumerate}[(1)]
  \item
    $S \yield{\star} x A w$, 
  \item
    $A \yield{\leq \abs{V}+1} y u v$, защото в дървото $P$ можем да изберем първата двойка повтарящи се променливи, които срещнем отзад напред по пътя $\pi$. Тук сме означили тази повтаряща се променлива с $А$.
  \item
    $A \yield{\leq \abs{V}} u$, като тук вече няма повтарящи се променливи по останалата част от пътя $\pi$.
  \end{enumerate}

  \begin{figure}[H]
    \begin{subfigure}[t]{0.3\textwidth}
    \centering
    \begin{tikzpicture}[scale=0.8]
      \node (A) at (0,0) {\footnotesize{$S$}};
      \coordinate (B) at (-2,-3);
      \coordinate (C) at (2,-3);
      \node (D) at (0,-1.2) {\footnotesize{$A$}};
      \coordinate (D1) at (1.4,-3);
      \coordinate (D2) at (-1.4,-3);
      \node (E) at (0,-2.3) {\footnotesize{$A$}};
      \coordinate (E1) at (0.7,-3);
      \coordinate (E2) at (-0.7,-3);
      \coordinate (F) at (0,-3);

      \draw (A) -- node[above left]{$P = $} (B) -- node[below]{$\pi$}(C) -- (A);
      \draw [photon2] (A) -- (D) -- (E) -- (F);
    \end{tikzpicture}
    \caption{Първата двойка повтарящи се променливи, които намерим като тръгнем отдолу нагоре по пътя $\pi$.}
    \label{fig:pumping:partition}
  \end{subfigure}
  ~
  \begin{subfigure}[t]{0.3\textwidth}
    \centering
    \begin{tikzpicture}[scale=0.8]
      \node (A) at (0,0) {\footnotesize{$S$}};
      \coordinate (B) at (-2,-3);
      \coordinate (C) at (2,-3);
      \node (D) at (0,-1.2) {\footnotesize{$A$}};
      \coordinate (D1) at (1.4,-3);
      \coordinate (D2) at (-1.4,-3);
      \node (E) at (0,-2.3) {\footnotesize{$A$}};
      \coordinate (E1) at (0.5,-3);
      \coordinate (E2) at (-0.5,-3);
      \coordinate (F) at (0,-3);

      \draw (A) -- node[above left]{$P = $} (B) -- node[below] {$x$} (D2) -- (D) -- (D1) -- node[below]{$w$} (C) -- (A);
      \draw (D2) -- node[below]{$y$} (E2) -- (E) -- (E1) -- node[below]{$v$} (D1);
      \draw (E2) -- node[below]{$u$} (E1);

      \draw [photon2] (A) -- (D) -- (E) -- (F);
    \end{tikzpicture}
    \caption{Разбиваме дървото на три части.}
    \end{subfigure}
    ~
    \begin{subfigure}[t]{0.3\textwidth}
      \centering
      \begin{tikzpicture}[scale=0.8]
        \node (A) at (0,0) {\footnotesize{$S$}};
        \coordinate (B) at (-2,-2.2);
        \coordinate (C) at (2,-2.2);
        \coordinate (S) at (-1,-2.2);
        \coordinate (T) at (1,-2.2);
        \node (L) at (0,-1) {\footnotesize{$A$}};
        
        \coordinate (E) at (-0.7,-2.5);
        \coordinate (F) at (0.7,-2.5);
        
        
        \node (H) at (0,-2) {\footnotesize{$A$}};
        \node (X) at (0,-3) {\footnotesize{$A$}};
        \coordinate (D) at (-1.5,-4);
        \coordinate (G) at (1.5,-4);
        \coordinate (E) at (-0.7,-4);
        \coordinate (F) at (0.7,-4);
        
        \node (I) at (0,-4.2) {\footnotesize{$A$}};
        \coordinate (U) at (-0.7,-5.2);
        \coordinate (V) at (0.7,-5.2);
        
        \draw (A) -- node[left]{\scriptsize{(1)}}(B) -- node[below]{$x$}(S) -- (L) -- (T) -- node[below]{$w$}(C) -- (A);
        \draw (H) -- node[left]{\scriptsize{(2)}}(D) -- node[below]{$y$}(E) -- (X) -- (F) -- node[below]{$v$}(G) -- (H);
        \draw (I) -- node[left]{\scriptsize{(3)}}(U) -- node[below]{$u$}(V) -- (I);
      \end{tikzpicture}
      \caption{Получаваме три отделни синтактични дървета.}
    \end{subfigure}
  \end{figure}
  Сега остава да проверим условието на лемата:
  \begin{itemize}
  \item
    Да допуснем, че $\abs{yv} = 0$. Това означава, че $\alpha = xuw$ и имаме извода:
    \begin{prooftree}
      \AxiomC{$S \yield{\star} xAw$}
      \AxiomC{$A \yield{\star} u$}
      \RightLabel{\scriptsize{(\Proposition{pumping:ground})}}
      \BinaryInfC{$S \yield{\star} \underbrace{xuw}_{\alpha}$,}
    \end{prooftree}
    за който съществува дърво на извод с по-малко на брой възли отколкото $P$, защото махаме средната част, която съдържа поне един възел.
    Това е противоречие с избора на $P$ като синтактично дърво за $S \yield{\star}\alpha$ с минимален брой възли.
    Заключаваме, че $\abs{yv} \geq 1$.
  \item
    Понеже имаме извода $A \yield{\leq\abs{V}+1} yuv$, то от \Proposition{pumping:lower-bound} следва, че $\abs{yuv} \leq b^{\abs{V}+1} = p$.
  \item
    За произволно $i\in\Nat$ имаме извода:
    \begin{prooftree}
      \AxiomC{$A \yield{\star}yAv$}
      \LeftLabel{\scriptsize{(\ShortProposition{pumping:iteration})}}
      \UnaryInfC{$A \yield{\star}y^iAv^i$}
      \AxiomC{$A \yield{\star} u$}
      \LeftLabel{\scriptsize{(\ShortProposition{pumping:ground})}}
      \BinaryInfC{$A \yield{\star} y^iuv^i$}
      \AxiomC{$S \yield{\star}xAw$}
      \LeftLabel{\scriptsize{(\ShortProposition{pumping:ground})}}
      \BinaryInfC{$S \yield{\star} xy^iuv^iw$}
    \end{prooftree}
    Оттук заключаваме, че $(\forall i \in \Nat)[xy^iuv^iw \in \L(G)]$.
  \end{itemize}
\end{proof}

\mynote{\cite[стр. 153]{kozen}.}
Както и при \hyperref[lem:regular:pumping]{Лема за покачването за регулярни езици}, ние ще използваме \emph{контрапозицията} на условието на \hyperref[lem:context-free:pumping]{Лема за покачването за безконтекстни езици} при проверка, че даден език не е безконтекстен.

\begin{corollary}
  \label{cor:pumping-context-free}
  \mynote{Ако $L$ е краен език, то е ясно, че $L$ е безконтекстен.}
  Нека $L$ е произволен {\bf безкраен} език. Нека също така е изпълнено, че:
  \begin{description}
  \item[($\forall$)]
    за {\em всяко} естествено число $p \geq 1$,
  \item[($\exists$)]
    можем да намерим дума $\alpha \in L$, $\abs{\alpha}\geq p$, такава че
  \item[($\forall$)]
    за {\em всяко} разбиване на думата на пет части, $\alpha = xyuvw$, със свойствата $\abs{yv} \geq 1$ и $\abs{yuv} \leq p$,
  \item[($\exists$)]
    можем да намерим $i \in \Nat$, за което е изпълнено, че $xy^iuv^iw \not\in L$.
  \end{description}  
  \mynote{\writedown Докажете! Аналогично е на \Corollary{regular:pumping}}
  Тогава $L$ {\bf не} е безконтекстен език.
\end{corollary}

\begin{corollary}
  \mynote{\writedown Докажете!}
  Нека $G$ е безконтекстна граматика и $p$ е константата на покачването за $G$.
  Тогава $\abs{\L(G)} = \infty$ точно тогава, когато съществува $\alpha \in \L(G)$, за която $p \leq \abs{\alpha} < 2p$.
\end{corollary}
% \begin{proof}
%   Ако съществува дума $\alpha \in L$, за която $\abs{\alpha} \geq p$, то от \Lem{pumping-context} следва,
%   че $\abs{L} = \infty$, защото $\alpha = xyuvw$ и $xy^iuv^iw \in L$, за всяко $i\in\Nat$.

%   За другата посока, нека сега $\abs{L} = \infty$.
%   Да изберем най-късата дума $\alpha \in L$, за която $\abs{\alpha} \geq p$.
%   Ще докажем, че $p \leq \abs{\alpha} < 2p$. За целта да допуснем, че $\abs{\alpha} \geq 2p$.
%   Тогава от \Lem{pumping-context} следва, че $\alpha = xyuvw$, $\abs{yv} \geq 1$, $\abs{yuv} \leq p$, $xy^0uv^0w = xuw \in L$.
%   Ако $\abs{xuw} < p$, то $\abs{yv} > p$, защото $\abs{yv} + \abs{xuw} = \abs{\alpha} \geq 2p$, и следователно $\abs{yuv} > p$, което е противоречие.
%   Следва, че $\abs{\alpha} > \abs{xuw} \geq p$.
%   Получихме, че думата $xuw\in L$ и $\abs{xuw} \geq p$. Това е противоречие с минималността на $\alpha$.
% \end{proof}

% \begin{framed}
%   \Lem{pumping-context} е полезна, когато искаме да докажем, че даден език $L$ {\bf не} е безконтекстен.
%   За целта, доказваме отрицанието на свойствата от \Lem{pumping-context} за $L$, т.е.
%   за всяка константа $p$, ние намираме дума $\alpha \in L$, $\abs{\alpha}\geq p$, такава че за всяко разбиване на думата на пет части, $\alpha = xyuvw$,
%   със свойствата $\abs{yv} \geq 1$ и $\abs{yuv} \leq p$, е изпълнено, че $(\exists i)[xy^iuv^iw \not\in L]$.
% \end{framed}

\begin{remark}
  \mynote{\todo Защо?}
  Алгоритъм за проверка дали един безконтекстен език е безкраен следвайки горния критерий би 
  имал експоненциална сложност относно $|G|$.
\end{remark}

\begin{example}
  \label{ex:context-free:pumping:anbncn}
  Да видим защо езикът $L = \{a^nb^nc^n\ \mid\ n\in\Nat\}$ не е безконтекстен.
  \begin{description}
  \item[$(\forall)$]
    Разглеждаме произволна константа $p \geq 1$.
  \item[$(\exists)$]
    Избираме дума $\alpha \in L$, $\abs{\alpha} \geq p$.
    В случая, нека $\alpha = a^pb^pc^p$.
  \item[$(\forall)$]
    Разглеждаме произволно разбиване на $\alpha$ на пет части $\alpha = xyuvw$, за което $\abs{yuv} \leq p$ и $1 \leq \abs{yv}$.
  \item[$(\exists)$]
    \mynote{\writedown Съобразете, че може да изберете и $i = 0$. В този пример може да вземете едно и също $i$ за всяко разбиване на думата $\alpha$ от предишната стъпка. В други примери ще видим, че изборът на $i$ зависи от разглежданото разбиване на думата.}
    За всяко такова разбиване ще посочим $i$, за което $xy^iuv^iw \not\in L$.
    Знаем, че поне едно от $y$ и $v$ не е празната дума.
    Имаме няколко случая за $y$ и $v$.
    \begin{itemize}
    \item
      $y$ и $v$ са думи съставени от една буква.
      В този случай получаваме, че $xy^2uv^2w$ има различен брой букви $a$, $b$ и $c$.
    \item
      $y$ или $v$ е съставена от две букви.
      В този случай получаваме, че в $xy^2uv^2w$ редът на буквите е нарушен.
    \item
      понеже $\abs{yuv} \leq p$, то не е възможно в $y$ или $v$ да се срещат и трите букви.
    \end{itemize}  
    Оказа се, че във всички възможни случаи за $y$ и $v$, 
    $xy^2uv^2w \not\in L$.
  \end{description}
  Така от \Corollary{pumping-context-free} следва, че езикът $L$ не е безконтекстен.
\end{example}

\begin{framed}
  \begin{proposition}\label{pr:context-free:pumping:non-closure}
    Безконтекстните езици {\bf не} са затворени относно операциите сечение и допълнение.
  \end{proposition}
\end{framed}
\begin{hint}
  Да разгледаме езика $L_0 = \{a^nb^nc^n\mid n\in\Nat\}$, за който вече знаем от \Example{context-free:pumping:anbncn}, че не е безконтекстен.
  Да вземем също така и безконтекстните езици 
  \mynote{\writedown Защо $L_1$ и $L_2$ са безконтекстни?}
  \[L_1 = \{a^nb^nc^m\mid n,m\in\Nat\},\ L_2 = \{a^mb^nc^n\mid n,m\in\Nat\}.\]
  \mynote{Да напомним, че с $\ov{L}$ означаваме допълнението на езика $L$, т.е. $\ov{L} \df \Sigma^\star \setminus L$.}
  \begin{itemize}
  \item 
    Понеже $L_0 = L_1\cap L_2$, то заключаваме, че безконтекстните езици не са затворени 
    относно операцията сечение.
  \item
    \mynote{Друг пример, с който може да се види, че безконтекстните езици не са затворени относно допълнение е 
      като се докаже, че езикът
      \[\{a,b\}^\star \setminus \{\omega\omega\mid \omega\in \{a,b\}^\star\}\]
      е безконтекстен.
      Това следва лесно като се използва \Problem{equal-but-different}.}
    Да допуснем, че безконтекстните езици са затворени относно операцията допълнение.
    Тогава  $\ov{L}_1$ и $\ov{L}_2$ са безконтекстни.
    Знаем, че безконтекстните езици са затворени относно обединение. 
    Следователно, езикът $L_3 = \ov{L}_1 \cup \ov{L}_2$ също е безконтекстен.
    Понеже допуснахме, че безконтекстните са затворени относно допълнение, то $\ov{L}_3$ също е безконтекстен.
    Но тогава получаваме, че езикът
    \[L_0 = L_1 \cap L_2 = \ov{\ov{L}_1 \cup \ov{L}_2} = \ov{L}_3\]
    е безконтекстен, което е противоречие.
  \end{itemize}
\end{hint}

%%% Local Variables:
%%% mode: latex
%%% TeX-master: "../eai"
%%% End:

\newpage
\subsection{Примери}

\begin{problem}
  Докажете, че езикът
  \[L = \{a^ib^jc^k\ \mid\ 0 \leq i \leq j \leq k\}\]
  не е безконтекстен.
\end{problem}
\begin{proof}
  \begin{description}
  \item[$(\forall)$]
     Разглеждаме произволна константа $p \geq 1$.
   \item[$(\exists)$]
     Избираме дума $\alpha \in L$, $\abs{\alpha} \geq p$.
     В случая, нека $\alpha = a^pb^pc^p$.
   \item[$(\forall)$]
     Разглеждаме произволно разбиване $xyuvw = \alpha$, за което $\abs{yuv} \leq p$ и $1 \leq \abs{yv}$.
     Знаем, че поне една от $y$ и $v$ не е празната дума.
   \item[$(\exists)$] Ще намерим $i \in \Nat$, за което $xy^iuv^iw \not\in L$.
    \begin{itemize}
    \item
      $y$ и $v$ са съставени от една буква.
      Имаме три случая.
      \begin{enumerate}[i)]
      \item
        $a$ не се среща в $y$ и $v$.
        Тогава $xy^0vu^0w$ съдържа повече $a$ от $b$ или $c$.
      \item
        $b$ не се среща в $y$ и $v$.
        \begin{itemize}
        \item 
          Ако $a$ се среща в $y$ или $v$, тогава $xy^2uv^2w$ съдържа повече $a$ от $b$.
        \item
          Ако $c$ се среща в $y$ или $v$, тогава $xy^0uv^0w$ съдържа по-малко $c$ от $b$.
        \end{itemize}
      \item
        $c$ не се среща в $y$ и $v$.
        Тогава $xy^2uv^2w$ съдържа повече $a$ или $b$ от $c$.
      \end{enumerate}      
     \item
       $y$ или $v$ е съставена от две букви.
       Тук разглеждаме $xy^2uv^2w$ и съобразяваме, че редът на буквите е нарушен.
     \end{itemize}    
   \end{description}
\end{proof}

\begin{problem}
  Докажете, че езикът 
  \[L = \{\ \alpha\alpha\mid \alpha\in \{a,b\}^\star\ \}\]
  не е безконтекстен.
\end{problem}
\begin{hint}
  \begin{itemize}
  \item 
    Защо $\omega = a^pba^pb$ не става ?
  \item
    Защо $\omega = a^pb^{2p}a^p$ не става ?
  \item
    Разгледайте $\omega = a^pb^pa^pb^p$.
  \end{itemize}
\end{hint}


\begin{problem}
  Докажете, че езикът 
  \[L = \{\alpha\beta\alpha^{rev} \mid \alpha,\beta \in \{a,b\}^\star\ \&\ |\alpha| = |\beta|\}\]
  не е безконтекстен.
\end{problem}
\begin{hint}
  \begin{itemize}
  \item
    Защо не става ако разгледаме думата $\alpha = a^pb^pa^p$ ?
  \item 
    Защо не става ако разгледаме думата $\alpha = a^p b^p a^{2p} b^p a^p$ ?
  \item
    Разгледайте $\alpha = a^p b^p a^p b^p b^p a^p$.
    Покачване с повече от $p$ би трябвало да свърши работа.
  \end{itemize}
\end{hint}


\begin{problem}
  Докажете, че езикът 
  \[L = \{\alpha\beta\alpha \mid \alpha,\beta \in \{a,b\}^\star\}\]
  не е безконтекстен.
\end{problem}
\begin{hint}
  \begin{itemize}
  \item 
    Защо не става с $\omega = a^pba^pb$ ?
  \item
    Защо не става с $\omega = ab^pab^p$ ?
  \item
    Пробвайте с $\omega = a^pb^pa^pb^p$.
  \end{itemize}
\end{hint}

\begin{problem}
  Докажете, че езикът
  \[L = \{\alpha\sharp\beta \mid \alpha\text{ е подниз на }\beta\}\]
  не е безконтекстен.
\end{problem}
\begin{hint}
  \begin{itemize}
  \item 
    Защо не става ако вземем $\omega = a^p \sharp a^p$ ?
  \item 
    Защо не става ако вземем $\omega = a^pb \sharp a^pb$ ?
  \item
    Разгледайте $\omega = a^pb^p\sharp a^pb^p$.
  \end{itemize}
\end{hint}


\begin{problem}
  Вярно ли е, че следните езици са безконтекстни:
  \begin{enumerate}[a)]
  \item 
    $L = \{\alpha\sharp\beta \mid \alpha,\beta \in \{0,1\}^\star\ \&\ \ov{\alpha}_{(2)} + 1 = \ov{\beta}_{(2)} \}$;
  \item
    $L = \{\alpha\sharp\beta^{rev} \mid \alpha,\beta \in \{0,1\}^\star\ \&\ \ov{\alpha}_{(2)} + 1 = \ov{\beta}_{(2)} \}$ ?
  \end{enumerate}
\end{problem}

\newpage
\section{Алгоритми}

\subsection{Опростяване на безконтекстни граматики}

\subsubsection*{Премахване на безполезните променливи}

Нека е дадена безконтекстната граматика $G = \CFG$.
\marginpar{\cite[стр. 88]{hopcroft1}}
Една променлива $A$ се нарича {\bf полезна}, ако съществува извод от следния вид:
\[S \to^\star \alpha A \beta \to^\star \gamma,\]
където $\gamma \in \Sigma^\star$, а $\alpha,\beta \in (V \cup \Sigma)^\star$.
Това означава, че една променлива е полезна, ако участва в извода на някоя дума в езика на граматиката.
Една променлива се нарича {\bf безполезна}, ако не е полезна.
Целта ни е да получим еквивалентна граматика $G'$ без безполезни променливи.
Ще решим задачата като разгледаме две леми.

\begin{lemma}
  \label{lem:useless1}
  Нека е дадена безконтекстната граматика $G = \CFG$ и $\L(G) \neq \emptyset$.
  Съществува алгоритъм, който намира граматика $G' = \pair{V',\Sigma,S,R'}$, за която 
  $\L(G) = \L(G')$, и за всяка променлива $A' \in V'$, съществува дума $\alpha \in \Sigma^\star$,
  за която $A' \to^\star \alpha$.
\end{lemma}
\begin{hint}
  Да разгледаме следната проста итеративна процедура.
  \begin{algorithm}[H]
    \caption{Намираме $V' = \{A \in V\mid (\exists \alpha \in \Sigma^\star)[A \to^\star \alpha]\}$}
    \label{alg:useless}
    \begin{algorithmic}[1]
      \State $V' := \emptyset$
      \State $V'' := \{A \in V \mid (\exists \alpha \in \Sigma^\star)[A \to \alpha]\}$
      \While{$V' \neq V''$}
      \State $V' := V''$
      \State $V'' := V' \cup \{A \in V \mid (\exists \alpha \in (\Sigma \cup V')^\star)[A \to \alpha]\}$
      \EndWhile
      \State \Return $V'$
    \end{algorithmic}
  \end{algorithm}
  Трябва да докажем, че във $V'$ са точно полезните променливи за $G$.
  Очевидно е, че ако $A \in V'$, то $A$ е полезна променлива.
  \marginpar{\writedown Докажете!}
  За другата посока, с индукция по дължината на извода се доказва, че ако $A \to^\star_G \omega$,
  то $A \in V'$.
  
  Правилата на $G'$ са всички правила на $G$, в които участват променливи от $V'$ и букви от $\Sigma$.
\end{hint}

\begin{lemma}
  \label{lem:useless2}
  Съществува алгоритъм, който по дадена безконтекстна граматика $G = \CFG$, намира $G' = \pair{V',\Sigma',S,R'}$, $\L(G') = \L(G)$,
  със свойството, че за всяко $x \in V' \cup \Sigma'$ съществуват $\alpha, \beta \in (V'\cup\Sigma')^\star$,
  за които $S \to^\star \alpha x \beta$,
  т.е. всяка променлива или буква в $G'$ е достижима от началната променлива $S$.
\end{lemma}
\begin{hint}
  Намираме $V'$ и $\Sigma'$ итеративно, като в началото $V' = \{S\}$, $\Sigma' = \emptyset$.
  Ако $A \in V'$ и имаме правила в $G$:
  \[A \to \alpha_0\ |\ \alpha_1\ |\ \cdots\ |\ \alpha_n,\]
  то за всяко $i = 0,\dots,n$ добавяме всички променливи на $\alpha_i$ към $V'$ и всички нетерминали на $\alpha_i$ към $\Sigma'$.
\end{hint}

\begin{thm}
  За всяка безконтекстна граматика $G$, $\L(G) = \emptyset$ точно тогава, когато $S$ е безполезно правило в граматиката.
\end{thm}
\begin{proof}
  \marginpar{Защо е важна последователността на прилагане?}
  Нека е дадена безконтекстна граматика $G$ пораждаща $L$.
  Прилагаме върху $G$ първо процедурата от \Lem{useless1} и след това върху резултата прилагаме процедурата от \Lem{useless2}.
\end{proof}

\begin{example}
  Да разгледаме следната граматика $G$:
  \begin{align*}
    & S \to AB\ |\ aA\\
    & A \to a\ |\ aAa\\
    & B \to SB\ |\ BC\\
    & C \to \varepsilon\ |\ cC.
  \end{align*}
  Първо да намерим променливите, от които се извеждат думи.
  \begin{itemize}
  \item 
    $V_0 = \{A, C\}$, защото $A \to a$ и $C \to \varepsilon$;
  \item
    $V_1 = \{A, C, S\}$, защото $S \to aA$;
  \item
    не можем да добавим $B$ към $V_2$, следователно $V_1 = V_2$.
  \end{itemize}
  Получаваме граматиката $G'$:
  \begin{align*}
    & S \to aA\\
    & A \to a\ |\ aAa\\
    & C \to \varepsilon\ |\ cC.
  \end{align*}
  Сега премахваме променливите и буквите, които не са достижими от началната промелива $S$. Така получаваме граматиката $G''$:
  \begin{align*}
    & S \to aA\\
    & A \to a\ |\ aAa.
  \end{align*}
\end{example}

\begin{problem}
  Проверете дали $\L(G) = \emptyset$, където правилата на $G$ са:
  \begin{align*}
    & S \to AS\ |\ BC\\
    & A \to 0\ |\ BA\ |\ SB\\
    & B \to 1\ |\ BC\ |\ AB\\
    & C \to CB\ |\ SC\ |\ AS.
  \end{align*}
\end{problem}

\subsubsection*{Премахване на $\varepsilon$-правила}
\index{$\varepsilon$-правила}
За да премахнем правилата от вида $A \to \varepsilon$, следваме процедурата:
\marginpar{Броят на правилата може да се увеличи експоненциално, защото в най-лошия случай извеждаме всички подмножества на дадено множество от променливи}
\begin{enumerate}[1)]
\item 
  Намираме множеството $E = \{A \in V \mid A \to^\star \varepsilon\}$ по следния начин.
  Първо, $E := \{A \in V \mid A \to \varepsilon\}$.
  След това, за всяко правило от вида $B \to X_1\cdots X_k$, 
  ако всяко $X_i \in E$, то добавяме $B$ към $E$.
\item
  Строим множеството от правила $R'$, в което няма правила $\varepsilon$-правила по следния начин.
  За всяко правило $A \to x_1\cdots x_k$, където $x_i \in V\cup\Sigma$,
  добавяме към $R'$ всички правила от вида $A \to y_1\cdots y_k$, където:
  \begin{itemize}[-]
  \item 
    ако $x_i \not\in E$, то $y_i = x_i$;
  \item
    ако $x_i \in E$, то $y_i = x_i$ или $y_i = \varepsilon$;
  \item
    не всички $y_i$-та са $\varepsilon$.
  \end{itemize}
\end{enumerate}

\begin{example}
  Нека е дадена граматиката $G$ с правила
  \begin{align*}
    & S \to D\\
    & D \to AD\ |\ b\\
    & A \to AB\ |\ BC\ |\ a\\
    & B \to AA\ |\ UC\\
    & C \to \varepsilon\ |\ CA\ |\ a\\
    & U \to \varepsilon\ |\ aUb.
  \end{align*}
  % \[S\rightarrow D,D\rightarrow AD|b,A\rightarrow AB|BC|a, B\rightarrow AA|EC,C\rightarrow \varepsilon|CA|a, E\rightarrow \varepsilon|aEb.\]
  Тогава $E = \{X \in V \mid X \rightarrow^\star_G \varepsilon\} = \{A,B,C,U\}$.
  Това означава, че $\varepsilon \not\in \L(G)$.
  Граматиката $G'$ без $\varepsilon$-правила, за която $\L(G') = \L(G)$ има следните правила
  \begin{align*}
    & S \to D\\
    & D\to AD\ |\ D\ |\ b\\
    & A \to A\ |\ B\ |\ C\ |\ AB\ |\ BC\ |\ a\\
    & B\to A\ |\ E\ |\ C\ |\ AA\ |\ UC\\
    & C \to C\ |\ A\ |\ CA\ |\ a\\
    & U \to aUb\ |\ ab.
  \end{align*}
\end{example}

\subsubsection*{Премахване на преименуващи правила}
\index{преименуващи правила}
Преименуващите правила са от вида $A \to B$.
Нека е дадена граматика $G = \CFG$, в която има преименуващи правила.
Ще построим еквивалентна граматика $G'$ без преименуващи правила.
В началото нека в $R'$ да добавим всички правила от $R$, които не са преименуващи.
След това, за всякa променлива $A$, за която $A \to^\star_G B$,
ако $B \to \alpha$ е правило в $R$, което не е преименуващо,
то добавяме към $R'$ правилото $A \to \alpha$.

\begin{example}
  Нека е дадена граматиката $G$ с правила  
  \begin{align*}
    & S \to B\ |\ CC\ |\ b\\
    & A \to B\ |\ S\\
    & B \to C\ |\ BC\\
    & C \to AB\ |\ a\ |\ b.
  \end{align*}
  % \[A\rightarrow B|S,B\rightarrow C|BC,C\rightarrow AB|a|b,S\rightarrow B|CC|b.\]
  Първо добавяме към $R'$ правилата $B \to BC, C \to AB\ |\ a\ |\ b, S \to CC\ |\ b$.
  \begin{itemize}
  \item 
    Лесно се съобразява, че $A \to^\star_G B,S,C$.
    Добавяме правилата 
    \[A \to BC\ |\ AB\ |\ a\ |\ b\ |\ CC.\]
  \item
    Имаме $B \to^\star_G C$.
    Добавяме правилата $B \to AB\ |\ a\ |\ b$.
  \item
    Имаме $S \to^\star_G B,C$.
    Добавяме правилата $S \to BC\ |\ AB\ |\ a\ |\ b$.
  \end{itemize}
  Накрая получаваме, че граматиката $G'$ има правила
  \begin{align*}
    & S \to BC\ |\ AB\ |\ CC\ |\ a\ |\ b\\
    & A \to BC\ |\ AB\ |\ a\ |\ b\ |\ CC\\
    & B \to AB\ |\ a\ |\ b\ |\ BC\\
    & C \to AB\ |\ a\ |\ b.
  \end{align*}
\end{example}

\begin{problem}
  Премахнете преименуващите правила от граматиката $G$, като запазите езика, ако $G$ има следните правила:
    \begin{align*}
      & S \to C\ |\ CC\ |\ b\\
      & A \to B\\
      & B \to S\ |\ C\ |\ BC\\
      & C \to a\ |\ AB;
    \end{align*}
\end{problem}

\subsubsection*{Премахване на дългите правила}

Едно правило се нарича дълго, ако е от вида $A \to \beta$, където $|\beta| \geq 3$.
Да разгледаме едно дълго правило в граматиката от вида $A \to x_1x_2\cdots x_k$, 
където $k \geq 3$ и $x_i \in V \cup \Sigma$. За да получим еквивалентна граматика без това дълго правило,
добавяме нови променливи $A_1,\dots, A_{k-2}$, и правила
\[A \to x_1A_1,\ A_1 \to x_2A_2, \dots,\ A_{k-2} \to x_{k-1}x_k.\]


\begin{problem}
  Нека е дадена граматиката  $G = \pair{\{S,A,B,C\}, \{a,b\}, S, R}$.
  Използвайте обща конструкция, за да премахнете ,,дългите'' правила от $ G$ като при това получите 
  безконтестна граматика $G_1$ с език $\L(G) = \L(G_1)$, където правилата на граматиката са:
  % \begin{enumerate}[a)]
  % \item
  %   \begin{align*}
  %     & S \to \varepsilon\ |\ ab\ |\ aAba\\
  %     & A\to aBCb\\
  %     & B\to bbb\\
  %     & C\to aC\ |\ aCaC;
  %   \end{align*}
  % \item
    \begin{align*}
      & S\to CC\ |\ b\\
      & A\to BSB\ |\ a\\
      & B\to ba\ |\ BC\\
      & C\to BaSA\ |\ a\ |\ b.
    \end{align*}
  % \end{enumerate}
\end{problem}

\subsection{Нормална Форма на Чомски}
\index{Чомски}
%[стр. 99 от \cite{sipser}]
\index{нормална форма на Чомски}
Една безконтекстна граматика е в {\bf нормална форма на Чомски}, ако
всяко правило е от вида
\[A \rightarrow BC\mbox{ и }A \rightarrow a,\]
като $B, C$ {\em не могат} да бъдат променливата за начало $S$.
Освен това, позволяваме правилото $S\to\varepsilon$.
\footnote{В \cite[стр. 151]{papadimitriou} дефиницията е малко по-различна.
Там дефинират $G$ да бъде в нормална форма на Чомски ако $R \subseteq V\times(V\cup\Sigma)^2$.
В този случай губим езиците $\{\varepsilon\}$ и $\{a\}$, за $a\in\Sigma$.}

\begin{framed}
  \begin{thm}
    Всеки безконтекстен език $L$ се поражда от безконтекстна граматика в нормална форма на Чомски.
  \end{thm}
\end{framed}
\begin{proof}
%  \marginpar{Броят на правилата може да се увеличи експоненциално.}
  Нека имаме контекстно-свободна граматика $G$, за която $L = \L(G)$.
  Ще построим безконтекстна граматика $G^\prime$ в нормална форма на Чомски, $L = \L(G^\prime)$.
  % [стр. 99 от \cite{sipser}]
  Следваме следната процедура:
  \begin{itemize}
  \item
    Добавяме нов начален символ $S_0$ и правило $S_0 \to S$.
  \item
    \marginpar{Време $O(n)$}
    Премахваме дългите правила.
    Заменяме правилата от вида $A\to x_1x_2\dots x_n$, $n\geq 3$, $x_i \in V\cup\Sigma$, с
    правилата \[A\to x_1A_1,\ A_1\to x_2A_2,\ \dots,\ A_{n-2} \to x_{n-1}x_n.\]
    където $A_i$ са нови променливи.
  \item
    \marginpar{Време $O(n^2)$}
    Премахваме $\varepsilon$-правилата.
    За всяка променлива $A \neq S_0$ премахваме правилата от вида $A\to\varepsilon$.
    Това правим по следния начин.
    
    Ако имаме правило от вида $R \to Au$ или $R\to u A$, $u \in V \cup \Sigma$,
    то добавяме правилото $R\to u$.
    %Правим това за всяко срещане на променливата $A$ в дясната страна на правило.
    Например, 
    \begin{itemize}
    \item 
      ако имаме правило $R\to aA$, то добавяме правилото $R \to a$;
    \item
      ако имаме правило $R\to AA$, то добавяме правилото $R \to A$.
    \end{itemize}
    Ако имаме правило от вида $R\to A$, то добавяме правилото $R\to\varepsilon$
    само ако променливата $R$ още не е преминала през процедурата за премахване на $\varepsilon$.
  \item
    \marginpar{Време $O(n^2)$}
    \marginpar{Памет $O(n^2)$}
    Премахваме преименуващите правила, т.е. правила от вида $A\to B$.
    Заменяме всяко правило от вида $B \to \beta$ с $A\to \beta$,
    освен ако $A \to \beta$ е вече премахнато преименуващо правило.
  \item
    \marginpar{Време $O(n)$}
    За правила от вида $A\to u_1 u_2$, където $u_1, u_2 \in V \cup \Sigma$, 
    заменяме всяка буква $u_i$ с новата променлива $U_i$
    и добавяме правилото $U_i\to u_i$.
    Например, правилото $A \to aB$ се заменя с правилото $A \to XB$ и добавяме правилото $X \to a$,
    където $X$ е нова променлива.
  \end{itemize}
\end{proof}

\begin{thm}
  При дадена безконтекстна граматика $G$ с дължина $n$, можем да намерим еквивалентна
  на нея граматика $G'$ в нормална форма на Чомски за време $O(n^2)$,
  като получената граматика е с дължина $O(n^2)$.
\end{thm}


% \begin{problem}
%   Нека е дадена граматиката  $G = \pair{\{S,A,B,C,D,E\}, \{a,b\},S, R}$.
%   \begin{enumerate}[a)]
%   \item
%     Намерете множеството $\{X \in V \mid X \rightarrow^\star_G \varepsilon\}$.
%   \item
%     Вярно ли е, че $\varepsilon \in L(G)$?
%   \item
%     Постройте граматика $G_1$ без $\varepsilon$-правила, за която $L(G_1)=L(G)\setminus\{\varepsilon\}$.
%   \end{enumerate}
%   Множеството от правила $R$ на граматиката $G$ е зададено като:
%   \begin{enumerate}[a)]
%   \item
%     $R = \{S\rightarrow D,D\rightarrow AD|b,A\rightarrow ACB|BC|a, B\rightarrow ABCA|CEC,C\rightarrow \varepsilon|CA|a, E\rightarrow \varepsilon|aEb\}$;
%   \item
%     $R = \{S \rightarrow aD, D\rightarrow \varepsilon|ABBA|ADD,A\rightarrow DEB|a,B\rightarrow DDD|DC|b,C\rightarrow CCE|a, E\rightarrow \varepsilon|bEa\}$;
%   \item
%     $R = \{ S\rightarrow D,D\rightarrow AD|b,A\rightarrow AB|BC|a, B\rightarrow AB|CC, C\rightarrow \varepsilon|CA|a, E\rightarrow a|EB\}$;
%   \item
%     $R = \{ S \rightarrow AD|a, D\rightarrow \varepsilon|BB|AD,A\rightarrow DB|a,B\rightarrow DD|DC|b,C\rightarrow CE|a, E\rightarrow AB|b|EA\}$;
%   \item
%     $R =\{S\rightarrow AS|SB|SS,B\rightarrow CA|b, C\rightarrow AA|a|BA,A\rightarrow \varepsilon|BS\}$;
%   % \item
%   %   $R = \{S\rightarrow AB|AC,B\rightarrow \varepsilon |BC|b,A\rightarrow BB|CC|a,C\rightarrow CS|a\}$;
%   % \item
%   %   $R = \{S\rightarrow AS|SB|SS,B\rightarrow AC|b, C\rightarrow A|a|AB,A\rightarrow \varepsilon|BS\}$;
%   \item
%     $R = \{S\rightarrow BA|CA,B\rightarrow \varepsilon |BC|b,A\rightarrow BB|CC|a, C\rightarrow CS|a\}$;
%   \item
%     $R = \{S\rightarrow AS|b,A\rightarrow AC|BC|a, B\rightarrow BC|CC,C\rightarrow \varepsilon|CA|a\}$;
%   \item
%     $R = \{S\rightarrow \varepsilon|BA|AS,A\rightarrow SB|a,B\rightarrow SS|SC|b,
%     C\rightarrow CC|a\}$; 
%   \end{enumerate}
% \end{problem}


% \begin{problem}
%   Намерете безконтекстна граматика в нормална форма на Чомски за езиците от задача 6.
% \end{problem}


\subsection{Проблемът за принадлежност}

\begin{thm}
  Съществува {\em полиномиален} алгоритъм, който проверява дали дадена дума принадлежни на граматиката $G$.
  \marginpar{За дума $\alpha$, алгоритъмът работи за време $O(\abs{\alpha}^3)$}
\end{thm}
% \begin{proof}[стр. 154 от \cite{papadimitriou}]
Можем да приемем, че $G = \CFG$ е граматика в нормална форма на Чомски.
Нека $\alpha = a_1a_2\dots a_n$ е дума, за която искаме да проверим дали $\alpha \in \L(G)$.
\marginpar{Това е алгоритъм на Cocke, Younger и Kasami (CYK), който е пример за динамично програмиране (стр. 195 от \cite{kozen})}
\begin{algorithm}[H]
  \caption{Проверка дали $\alpha \in \L(G)$}
  \label{alg:belongs-to-grammar}
  \begin{algorithmic}[1]
    \State $n := \abs{\alpha}$ \Comment{Вход дума $\alpha = a_1\cdots a_n$}
    \ForAll{$i\in [1,n]$}
    \State $V[i,i] := \{A \in V \mid A\rightarrow a_i\}$
    \EndFor
    \ForAll{$i,j \in [1,n]\ \&\ i \neq j$}
    \State $V[i,j] := \emptyset$
    \EndFor      
    \ForAll{$s \in [1, n)$} \Comment{Дължина на интервала}
    \ForAll{$i \in [1, n-s]$}\Comment{Начало на интервала}
    \ForAll{$k \in [i, i + s)$}\Comment{Разделяне на интервала}
    \If{$\exists A\to BC \in R\ \&\ B \in V[i,k]\ \&\ C\in V[k+1,i+s]$}
    \State $V[i,i+s] := V[i,i+s] \cup \{A\}$
    \EndIf
    \EndFor
    \EndFor
    \EndFor
    \If{$S \in V[1,n]$}
    \State \Return \texttt{True}\Comment{Има извод на думата от $S$}
    \Else
    \State \Return \texttt{False}
    \EndIf
  \end{algorithmic}
\end{algorithm}

\begin{lemma}
  За дадена граматика в нормална форма на Чомски и дума $\alpha$, 
  за всяко $0 \leq s < \abs{\alpha}$, след $s$-тата итерация на алгоритъма (редове 6 - 10), за всяка позиция $i = 1,\dots,n-s$,
  \[V[i,i+s] = \{A \in V \mid A \rightarrow^\star_G a_i\dots a_{i+s}\}.\]
\end{lemma}
\begin{proof}
  Пълна индукция по $s$.
  За $s = 0$  е ясно. (Защо?)

  Нека твърдението е вярно за $s < n$. Ще докажем твърдението за $s+1$, т.е. за всяко $i = 1,\dots,n-s-1$,
  \[V[i,i+s+1] = \{A \in V \mid A \rightarrow^\star_G a_i\dots a_{i+s+1}\}.\]
  % Да разгледаме $A \in V[i,i+s+1]$.
  За едната посока, да разгледаме първoто правило в извода $A \to^\star_G a_i\cdots a_{i+s+1}$.
  Понеже $G$ е в НФЧ, то е от вида $A \to BC$ и тогава съществува някое $t$, за което 
  $B \to^\star a_i\cdots a_{i+t}$ и $C \to^\star a_{i+t+1}\cdots a_{i+s+1}$.
  От И.П. получаваме, че $B \in V[i,i+t]$ и $C \in V[i+t+1,i+s+1]$.
  Тогава от ред 10 на алгоритъма е ясно, че $A \in V[i,i+s+1]$.
  
  За другата посока, нека $A \in V[i,i+s+1]$.
  Единствената стъпка на алгоритъма, при която може да сме добавили $A$ към множеството $V[i,i+s+1]$ е ред 10.
  Тогава имаме, че съществува $k$, за което $B \in V[i,k]$, $C \in V[k+1,i+s+1]$, и $A\to BC$ е правило в граматиката $G$.
  От И.П. имаме, че $B \to^\star_G a_i\cdots a_k$ и $C \to^\star_G a_{k+1}\cdots a_{i+s+1}$.
  Заключаваме веднага, че $A \to^\star_G a_i\cdots a_{i+s+1}$.
\end{proof}

\begin{example}
  Нека е дадена граматиката $G$ с правила 
  \begin{align*}
    & S\rightarrow a\ |\ AB\ |\ AC\\
    & A\rightarrow a\\
    & B\rightarrow b\\
    & C\rightarrow SB\ |\ AS.
  \end{align*}
  Ще приложим $CYK$ алгоритъма за да проверим дали думата $aaabb \in \L(G)$.
  \begin{itemize}
  \item 
    $V[1,1] = V[2,2] = V[3,3] = \{S,A\}$;
    $V[4,4] = V[5,5] = \{B\}$.
  \item
    $V[1,2] = V[2,3] = \{C\}$;
    $V[3,4] = \{S,C\}$;
    $V[4,5] = \emptyset$.
  \item
    $V[1,3] = \{S\} \cup \emptyset$;
    $V[2,4] = \{S,C\} \cup \emptyset$;
    $V[3,5] = \emptyset \cup \{C\}$.
  \item
    $V[1,4] = \{S,C\} \cup \emptyset \cup \emptyset = \{S,C\}$;
    $V[2,5] = \{S\} \cup \emptyset \cup \{C\} = \{S,C\}$.
  \item
    $V[1,5] = \{S,C\} \cup \emptyset \cup \emptyset \cup \{C\}= \{S,C\}$.
  \end{itemize}
  Понеже $S \in V[1,5]$, то $aaabb \in \L(G)$.
\end{example}

\begin{thm}
  \marginpar{\cite[стр. 137]{hopcroft1}}
  Съществуват алгоритми, които определят по дадена безконтекстна граматика $G$ дали:
  \begin{enumerate}[a)]
  \item 
    $\abs{\L(G)} = 0$;
  \item
    $\abs{\L(G)} < \infty$;
  \item
    $\abs{\L(G)} = \infty$.
  \end{enumerate}
\end{thm}
\begin{proof}
  Нека е дадена една безконтекстна граматика $G$.
  \begin{description}
  \item[($\L(G) = \emptyset?$)]
    Прилагаме алгоритъма за премахване на безполезните променливи.
    Ако открием, че $S$ е безполезна променлива, то $\L(G) = \emptyset$.
  \item[($\abs{\L(G)} < \infty?$ или $\abs{\L(G)} = \infty?$)]
    Нека да разгледаме граматиката $G'$ в НФЧ без безполезни променливи, за която $\L(G) = \L(G')$.
    От граматиката $G' = \pair{V',\Sigma,S,R'}$ строим граф с възли променливите от $V'$ като
    за $A,B \in V'$ имаме ребро $A \to B$ точно тогава, когато съществува $C \in V'$,
    за което $A \to BC$ или $A \to CB$ е правило в $R'$.
    
    Ако в получения граф имаме цикъл, то $\L(G') = \infty$.
  \end{description}
\end{proof}

%%% Local Variables: 
%%% mode: latex
%%% TeX-master: "../eai"
%%% End: 

\newpage
\section{Най-ляв извод в граматика}
\index{граматика!най-ляв извод}
\mynote{Най-левият извод ще бъде важен за нас, когато разгледаме стековите автомати.}
В нашата дефиниция на извод, изборът върху коя променлива да приложим правило от граматиката е недетерминистичен.
В някои случаи, за нас ще бъде важно винаги да правим детерминистичен избор на това върху коя променлива прилагаме правило.

\begin{definition}
За две думи $\alpha,\beta \in (V\cup\Sigma)^\star$, дефинираме {\bf най-ляв извод} в граматиката $G$, $\alpha \lderive{\ell} \beta$, по следния начин:
\mynote{Новото е, че имаме изискването $\lambda \in \Sigma^\star$. Това на практика означава, че на всяка стъпка заместваме възможно най-лявата променлива.}
\begin{important}
\begin{figure}[H]
  \begin{subfigure}[b]{0.4\textwidth}
      \begin{prooftree}
        \AxiomC{}
        \RightLabel{\scriptsize{правило (0)}}
        \UnaryInfC{$\alpha \lderive{0} \alpha$}
      \end{prooftree}
    \end{subfigure}
    ~
    \begin{subfigure}[b]{0.4\textwidth}
      \begin{prooftree}
        \AxiomC{$(A,\alpha) \in R$}
        \AxiomC{$\lambda \alpha \rho \lderive{\ell} \beta$}
        \AxiomC{$\lambda \in \Sigma^\star$}
        \RightLabel{\scriptsize{правило (1)}}
        \TrinaryInfC{$\lambda A \rho \lderive{\ell+1} \beta$}
      \end{prooftree}
    \end{subfigure}
    \caption{Правила за най-ляв извод в безконтекстна граматика.}
  \end{figure}  
\end{important}
\end{definition}





\mynote{Интуитивно, $A \yield{\star} \alpha$ ни казва, че при обхождане на синтактичното дърво в широчина ние получаваме като листа думата $\alpha$. Ако
  $A\lderive{\star}\alpha$, то обхождаме синтактичното дърво с корен $A$ в дълбочина, като винаги избираме най-левия необходен клон.  Ако $A\derive{\star}\alpha$, то обхождаме синтактичното дърво в дълбочина без да имаме детерминистичен избор с кой необходен клон да продължим.}

\begin{proposition}\label{pr:left-derivation:padding}
  За проиволно естествено число $\ell$ имаме извода:
  \begin{prooftree}
    \AxiomC{$\alpha \lderive{\ell} \beta$}
    \AxiomC{$\lambda \in \Sigma^\star$}
    \AxiomC{$\rho \in (V\cup\Sigma)^\star$}
    \TrinaryInfC{$\lambda \alpha \rho \lderive{\ell} \lambda \beta \rho$}
  \end{prooftree}
\end{proposition}
\begin{proof}
  Индукция по $\ell$.
  \begin{itemize}
  \item
    $\ell = 0$. Директно следва от правило $(0)$.
  \item
    $\ell > 0$. Според правилата за извод имаме следния случай:
    \begin{prooftree}
      \AxiomC{$(A,\alpha') \in R$}
      \AxiomC{$\gamma\alpha'\delta \lderive{\ell-1} \beta$}
      \AxiomC{$\gamma \in \Sigma^\star$}
      \RightLabel{\scriptsize{правило (1)}}
      \TrinaryInfC{$\underbrace{\gamma A \delta}_{\alpha} \lderive{\ell} \beta$}
    \end{prooftree}
    Сега като използваме \IndHyp получаваме следния извод:
    \begin{prooftree}
      \AxiomC{$(A,\alpha') \in R$}
      \AxiomC{$\gamma\alpha'\delta \lderive{\ell-1} \beta$}
      \AxiomC{$\lambda \in \Sigma^\star$}
      \AxiomC{$\rho \in (V\cup\Sigma)^\star$}
      \RightLabel{\scriptsize{\IndHyp}}
      \TrinaryInfC{$\lambda\gamma\alpha'\delta\rho \lderive{\ell-1} \lambda\beta\rho$}
      \AxiomC{$\lambda\gamma \in \Sigma^\star$}
      \RightLabel{\scriptsize{правило (1)}}
      \TrinaryInfC{$\lambda\underbrace{\gamma A \delta}_{\alpha}\rho \lderive{\ell} \lambda\beta\rho$}
    \end{prooftree}
  \end{itemize}
\end{proof}



Удобно ще бъде да разгледаме следния аналог на \Proposition{unrestricted-grammar:context-general-step}.
\begin{proposition}\label{pr:left-derivation:context-step}
  Имаме извода:
  \begin{prooftree}
    \AxiomC{$\delta \lderive{\ell_1} \alpha B \gamma$}
    \AxiomC{$\alpha \in \Sigma^\star$}
    \AxiomC{$B \lderive{\ell_2}\beta$}
    \TrinaryInfC{$\delta \lderive{\ell_1+\ell_2}\alpha\beta\gamma$}
  \end{prooftree}
\end{proposition}
\begin{proof}
  Сега ще докажем твърдението с индукция по $\ell_1$.
  \begin{itemize}
  \item
    Нека $\ell_1 = 0$. Тук нещата са ясни, защото $\delta = \alpha B \gamma$ и имаме извода:
    \begin{prooftree}
      \AxiomC{$B \lderive{\ell_2} \beta$}
      \AxiomC{$\alpha \in \Sigma^\star$}
      \AxiomC{$\gamma \in (V \cup \Sigma)^\star$}
      \RightLabel{\scriptsize{(\Proposition{left-derivation:padding})}}
      \TrinaryInfC{$\underbrace{\alpha B \gamma}_{\delta} \lderive{\ell_2} \alpha\beta\gamma$}
    \end{prooftree}
  \item
    Нека $\ell_1 > 0$. Трябва да разбием извода с дължина $\ell_1$ за да можем да приложим индукционното предположение. От правилата за извод следва, че имаме следната ситуация:
    \begin{prooftree}
      \AxiomC{$(A,\alpha') \in R$}
      \AxiomC{$\lambda \alpha' \rho \lderive{\ell_1-1} \alpha B \gamma$}
      \AxiomC{$\lambda \in \Sigma^\star$}
      \RightLabel{\scriptsize{(1)}}
      \TrinaryInfC{$\underbrace{\lambda A \rho}_{\delta} \lderive{\ell_1} \alpha B \gamma$}
    \end{prooftree}

    Използвайки \IndHyp получаваме следния извод:
    \begin{prooftree}
      \AxiomC{$\lambda \in \Sigma^\star$}
      \AxiomC{$(A,\alpha') \in R$}
      \AxiomC{$\lambda\alpha'\rho \lderive{\ell_1-1} \alpha B \gamma$}
      \AxiomC{$B \lderive{\ell_2} \beta$}
      \AxiomC{$\alpha \in \Sigma^\star$}
      \RightLabel{\scriptsize{\IndHyp}}
      \TrinaryInfC{$\lambda\alpha'\rho \lderive{\ell_1+\ell_2-1} \alpha\beta\gamma$}
      \RightLabel{\scriptsize{(1)}}
      \TrinaryInfC{$\underbrace{\lambda A \rho}_{\delta} \lderive{\ell_1+\ell_2} \alpha\beta\gamma$}
    \end{prooftree}
  \end{itemize}  
\end{proof}

\begin{proposition}\label{pr:left-derivation:concat2}
  За произволни $\ell_1$ и $\ell_2$ имаме извода:
  \begin{prooftree}
    \AxiomC{$\alpha_1 \lderive{\ell_1} \beta_1$}
    \AxiomC{$\beta_1 \in \Sigma^\star$}
    \AxiomC{$\alpha_2 \lderive{\ell_2} \beta_2$}
    \TrinaryInfC{$\alpha_1\alpha_2 \lderive{\ell_1+\ell_2} \beta_1\beta_2$}
  \end{prooftree}
\end{proposition}


Следваща \Lemma{left-derivation-equivalence} е важна, защото тя на практика ни казва, че няма значение в какъв ред ще заместваме променливите в един извод на безконтекстна граматика.

\begin{important}
  \begin{lemma}\label{lem:left-derivation-equivalence}
% \mynote{Важно е да отбележим, че имаме горната еквивалентност само когато $\alpha \in \Sigma^\star$. Съобразете сами защо тази лема не е вярна в общия случай, когато позволим $\alpha$ да съдържа променливи.}
    За всяка безконтекстна граматика $G$, променлива $A$ и дума $\alpha \in \Sigma^\star$,
    \[A \lderive{\star} \alpha\text{ точно тогава, когато } A \derive{\star} \alpha.\]
    В частност, $\L(G) = \{\alpha\in\Sigma^\star \mid S \lderive{\star} \alpha\}$.
  \end{lemma}
\end{important}
\begin{proof}
  \mynote{Правило $(1)$ за $\lderive{\ell}$ е частен случай на правило $(1)$ за $\derive{\ell}$} 
  От правилата за извод е видно, че ако $A \lderive{\star} \alpha$, то $A \derive{\star} \alpha$.

  За другата посока, според \Theorem{grammar:yield-derive-equivalent} е достатъчно да се докаже, че ако $A \yield{\star} \alpha$ то $A \lderive{\star} \alpha$. С пълна индукция по $\ell$ ще докажем, че за произволно число $\ell$, ако $A \yield{\ell}\alpha$, то $A \lderive{\star}\alpha$.
  Да отбележим, че щом $A \in V$, а $\alpha \in \Sigma^\star$ е ясно, че $\ell > 0$.
  Това означава, че $A \yield{\ell}\alpha$ е получен чрез правило $(1)$ от дефиницията на релацията $\yield{\ell}$, т.е.
  \begin{prooftree}
    % \AxiomC{$A \to_G \alpha_1B_1\cdots\alpha_n B_n\alpha_{n+1}$}
    \AxiomC{$A \to_G X_1X_2\cdots X_n$}
    \AxiomC{$X_1 \yield{\ell_1} \alpha_1$}
    \AxiomC{$\cdots$}
    \AxiomC{$X_n \yield{\ell_n} \alpha_n$}
    \RightLabel{\scriptsize{(1)}}
    \QuaternaryInfC{$A \yield{\ell} \underbrace{\alpha_1\cdots\alpha_n}_{\alpha}$,}
  \end{prooftree}
  където $X_i \in V \cup \Sigma$ и $\ell = 1+\max\{\ell_1,\dots,\ell_n\}$.
  Започваме така:
  \begin{prooftree}
    \AxiomC{$A \to_G X_1X_2\cdots X_n$}
    \AxiomC{$X_1 \yield{\ell_1} \alpha_1$}
    \RightLabel{\scriptsize{\IndHyp}}
    \UnaryInfC{$X_1 \lderive{^\star}\alpha_1$}
    \LeftLabel{\scriptsize{(\Proposition{left-derivation:context-step})}}
    \BinaryInfC{$A \lderive{^\star}\alpha_1 X_2 \cdots X_n$}
  \end{prooftree}
  Продължаваме последователно за всяко $i < n$ правим да правим извода:
  \begin{prooftree}
    \AxiomC{$A \lderive{\star} \overbrace{\alpha_1\cdots\alpha_{i-1}}^{\in\Sigma^\star}X_{i}X_{i+1}\cdots X_n$}
    \AxiomC{$X_i \yield{\ell_{i+1}} \alpha_i$}
    \RightLabel{\scriptsize{\IndHyp}}
    \UnaryInfC{$X_i \lderive{^\star}\beta_i$}
    \LeftLabel{\scriptsize{(\Proposition{left-derivation:context-step})}}
    \BinaryInfC{$A \lderive{\star}\alpha_1\cdots\alpha_{i-1}\alpha_i X_{i+1} \cdots X_n$}
  \end{prooftree}
  Завършваме със следния извод:
  \begin{prooftree}
    \AxiomC{$A \lderive{\star} \overbrace{\alpha_1\alpha_2\cdots \alpha_{n-1}}^{\in\Sigma^\star} X_n$}
    \AxiomC{$X_n \yield{\ell_n} \alpha_n$}
    \RightLabel{\scriptsize{\IndHyp}}
    \UnaryInfC{$X_n \lderive{\star}\alpha_n$}
    \LeftLabel{\scriptsize{(\Proposition{left-derivation:context-step})}}
    \BinaryInfC{$A \lderive{^\star}\underbrace{\alpha_1\alpha_2\cdots\alpha_{n-1}\alpha_n}_{\alpha}$}
  \end{prooftree}
\end{proof}



\subsection*{Най-десен извод}

\begin{extra}
  \index{граматика!най-десен извод}

  \mynote{Единствената разлика между дефиницията на $\rderive{\star}$ и тази на $\derive{\star}$ е, че тук изискваме $\rho \in \Sigma^\star$.}

  За две думи $\alpha,\beta \in (V\cup\Sigma)^\star$, дефинираме {\bf най-десен извод} в граматиката $G$, $\alpha \rderive{\ell} \beta$, по следния начин:

  \begin{important}
    \begin{figure}[H]
      \begin{subfigure}[b]{0.4\textwidth}
        \begin{prooftree}
          \AxiomC{}
          \RightLabel{\scriptsize{правило (0)}}
          \UnaryInfC{$\alpha \rderive{0} \alpha$}
        \end{prooftree}
      \end{subfigure}
      ~
      \begin{subfigure}[b]{0.4\textwidth}
        \begin{prooftree}
          \AxiomC{$(A,\alpha) \in R$}
          \AxiomC{$\lambda \alpha \rho \rderive{\ell} \beta$}
          \AxiomC{$\rho \in \Sigma^\star$}
          \RightLabel{\scriptsize{правило (1)}}
          \TrinaryInfC{$\lambda A \rho \rderive{\ell+1} \beta$}
        \end{prooftree}
      \end{subfigure}
      \caption{Правила за най-десен извод в безконтекстна граматика.}
    \end{figure}
  \end{important}

  
%   \begin{framed}
%   \begin{figure}[H]
%     \begin{subfigure}[b]{0.5\textwidth}
%       \begin{prooftree}
%         \AxiomC{}
%         \RightLabel{\scriptsize{правило (0)}}
%         \UnaryInfC{$\alpha \rderive{0} \alpha$}
%       \end{prooftree}
%     \end{subfigure}
%     ~
%     \begin{subfigure}[b]{0.5\textwidth}
%       \begin{prooftree}
%         \AxiomC{$A \to_G \alpha$}
%         \AxiomC{$\alpha \rderive{\ell} \beta$}
%         \RightLabel{\scriptsize{правило (1)}}
%         \BinaryInfC{$A \rderive{\ell+1} \beta$}
%       \end{prooftree}
%     \end{subfigure}
%     \center
% \begin{prooftree}
%   \AxiomC{$\alpha_1 \rderive{\ell_1} \beta_1$}
%   \AxiomC{$\alpha_2 \rderive{\ell_2} \beta_2$}
%   \AxiomC{$\beta_2 \in \Sigma^\star$}
%   \RightLabel{\scriptsize{правило (2)}}
%   \TrinaryInfC{$\alpha_1\alpha_2 \rderive{\ell_1+\ell_2}\beta_1\beta_2$}
% \end{prooftree}
% \caption{Правила за най-десен извод в безконтекстна граматика.}
% \end{figure}
% \end{framed}

% \mynote{Интуитивно, $\yield{\star}$ е аналог на BFS, докато $\rderive{\star}$ е аналог на DFS като винаги се избира най-десния необходен клон.}

Удобно ще бъде да разгледаме следния аналог на \Proposition{unrestricted-grammar:context-general-step}.
\begin{problem}\label{prob:grammar:context-right-step}
  Докажете, че имаме извода:
  \begin{prooftree}
    \AxiomC{$\delta \rderive{\ell_1} \lambda B \rho$}
    \AxiomC{$B \rderive{\ell_2}\beta$}
    \AxiomC{$\rho \in \Sigma^\star$}
    \TrinaryInfC{$\delta \rderive{\ell_1+\ell_2}\lambda\beta\rho$}
  \end{prooftree}
\end{problem}

\begin{problem}
  Докажете, че за всяка безконтекстна граматика $G$, променлива $A$ и дума $\alpha \in \Sigma^\star$,
  \[A \rderive{\star} \alpha\text{ точно тогава, когато } A \derive{\star} \alpha.\]
\end{problem}

\end{extra}


%%% Local Variables:
%%% mode: latex
%%% TeX-master: "../eai"
%%% End:

\newpage
\section{Недетерминирани стекови автомати}

\index{автомат!недетерминиран стеков}
\mynote{На англ. {\em Push-down automaton}. В този курс няма да разглеждаме детерминирани стекови автомати. Когато кажем стеков автомат, ще имаме предвид недетерминиран стеков автомат.
  Означаваме $\Sigma_\varepsilon \df \Sigma \cup \{\varepsilon\}$ и $\Gamma^{\leq 2} \df \{\varepsilon\} \cup \Gamma \cup \Gamma^2$.}
{\bf Недетерминиран стеков автомат} е седморка от вида
\[P = \PDA,\] където:
\begin{itemize}
\item
  $Q$ е крайно множество от състояния;
\item  
  $\Sigma$ е крайна входна азбука;
\item
  $\Gamma$ е крайна стекова азбука;
\item
  $\sharp \in \Gamma$ е символ за дъно на стека;
\item
  \mynote{Дефиницията на стеков автомат има много вариации, всички еквивалентни помежду си}
  $\Delta:Q\times\Sigma_\varepsilon\times \Gamma \rightarrow \Ps(Q\times\Gamma^{\leq 2})$ 
  е функция на преходите;    
\item
  $\qstart \in Q$ е начално състояние;
\item
  $\qaccept \in Q$ е заключителното състояние.
\end{itemize}

\mynote{На англ. Instanteneous description}
{\em Моментно описание} (или конфигурация) на изчислението със стеков автомат представлява тройка от вида $(q,\alpha,\gamma) \in Q\times\Sigma^\star\times\Gamma^\star$,
т.е. автоматът се намира в състояние $q$, думата, която остава да се прочете е $\alpha$,
а съдържанието на стека е думата $\gamma$.
Удобно е да въведем бинарната релация $\vdash_P$ над $Q\times\Sigma^\star\times\Gamma^\star$,
която ще ни казва как моментното описание на автомата $P$ се променя след изпълнение на една стъпка:
\begin{align*}
  (p,\varepsilon) \in \Delta(q,x,A) & \implies (q,x\alpha,A\gamma) \vdash_P (p,\alpha,\gamma)\\
  (p,\beta) \in \Delta(q,\varepsilon,Y) & \implies (q,\alpha,Y\gamma) \vdash_P (p,\alpha,\beta\gamma).
\end{align*}
Рефлексивното и транзитивно затваряне на $\vdash_P$ ще означаваме с $\vdash^\star_P$.
Сега вече можем да дадем дефиниция на език, разпознаван от стеков автомат $P$.
\mynote{Възможно е да се даде и друга еквивалентна дефиниция - разпознаване с празен стек.}
Езикът $\L(P)$, който се разпознава от $P$, има следната дефиниция:
\[\L(P) = \{\ \omega \in \Sigma^\star \mid (\qstart,\omega,\sharp) \vdash^\star_P (\qaccept,\varepsilon,\varepsilon)\ \}.\]

\subsection{Примери}

\begin{extra}
\begin{example}
  \label{ex:anbn}
  За езика $L = \{a^nb^n\mid n\in\Nat\}$, да разгледаме $P = \PDA$, където
  \begin{itemize}
  \item
    $Q \df \{q,p,f\}$;
  \item
    $\qstart \df q$ и $\qaccept \df f$;
  \item
    $\Sigma \df \{a,b\}$ и $\Gamma \df \{\sharp,a\}$;
  \item
    \mynote{Тук получаваме детерминистичен стеков автомат.}
    Релацията на преходите $\Delta$ има следната дефиниция:
    \begin{enumerate}[(1)]
    \item
      $\Delta(q,a,\sharp) \df \{(q, a\sharp)\}$;
    \item
      $\Delta(q,a,a) \df \{(q, aa)\}$; \quad \comment{трупаме $a$-та в стека}
    \item 
      $\Delta(q,\varepsilon,\sharp) \df \{(f,\varepsilon)\}$;\quad \comment{трябва да разпознаем и думата $\varepsilon$}
    \item 
      $\Delta(q, b, a) \df \{(p,\varepsilon)\}$; \quad \comment{Започваме да четем само $b$-та}
    \item 
      $\Delta(p, b, a) \df \{(p,\varepsilon)\}$; \quad \comment{Чистим $a$-тата от стека}
    \item
      $\Delta(p, \varepsilon, \sharp) \df \{(f, \varepsilon)\}$.
    \item
      За всички останали тройки $(r,x,y)$, нека $\Delta(r,x,y) \df \emptyset$.
    \end{enumerate}
  \end{itemize}
  
  Да видим как думата $a^2b^2$ се разпознава от стековия автомат $P$:
  \begin{align*}
    (q, a^2b^2, \sharp) & \vdash_P (q, ab^2, a\sharp) & \comment{\text{правило }(1)}\\
                        & \vdash_P (q, b^2, aa\sharp) & \comment{\text{правило }(2)}\\
                        & \vdash_P (p, b, a\sharp) & \comment{\text{правило }(4)}\\
                        & \vdash_P (p, \varepsilon, \sharp) & \comment{\text{правило }(5)}\\
                        & \vdash_P (f, \varepsilon, \varepsilon) & \comment{\text{правило }(6)}
  \end{align*}
  % \mynote{\writedown Докажете, че $L = \L(P)$!}
  Получихме, че $(\qstart, a^2b^2, \sharp) \vdash^\star_P (\qaccept, \varepsilon, \varepsilon)$, откъдето следва, че $a^2b^2 \in \L(P)$.

  \begin{enumerate}[a)]
  \item
    Докажете с индукция по $n$, че за всяко естествено число $n$ са изпълнени свойствата:
    \begin{align}
      & (q, a^n\beta, \sharp) \vdash^n_P (q, \beta, a^n\sharp) \label{eq:anbn:1}\\
      & (p, b^n, a^n\sharp) \vdash^n_P (p, \varepsilon,\sharp). \label{eq:anbn:2}
    \end{align}
    Заключете, че $L \subseteq \L(P)$.
  \item
    Докажете, че с индукция по $n$, че за всяко естествено число $n$ са изпълнени свойствата:
    \begin{align}
      (q, \alpha\beta, \sharp) \vdash^n_P (q, \beta, \gamma\sharp)  & \implies \alpha = \gamma = a^n \label{eq:anbn:3}\\
      (p, \beta, \gamma\sharp) \vdash^n_P (p, \varepsilon, \sharp) & \implies \beta = b^n\ \&\ \gamma = a^n. \label{eq:anbn:4}
    \end{align}
    Оттук заключете, че $\L(P) \subseteq L$.    
  \end{enumerate}
\end{example}

\begin{example}
  \label{ex:omega-omega-r}
  Езикът $L = \{\ \omega\omega^{\rev} \mid \omega \in \{a,b\}^\star\ \}$ се разпознава от стеков автомат
  \[P = \PDA,\] където:
  \begin{itemize}
  \item 
    $Q \df \{q,p,f\}$ и $\qstart \df q$, $\qaccept \df f$;
  \item
    $\Sigma \df \{a,b\}$, $\Gamma \df \{a, b, \sharp\}$;
  \item
    Функцията на преходите $\Delta$ има следната дефиниция:
    \mynote{За всички липсващи твойки  в дефиницията на $\Delta$ приемаме, че $\Delta$ връща $\emptyset$}
    \begin{enumerate}[(1)]
    \item
     $\Delta(q, x, \sharp) \df \{(q, x\sharp)\}$, където $x \in \{a,b\}$;
    % \item 
    %   $\Delta(q, a, \sharp) \df \{(q, a\sharp)\}$;
    % \item 
    %   $\Delta(q, b, \sharp) \df \{(q, b\sharp)\}$;
    \item
      $\Delta(q, \varepsilon, \sharp) \df \{(q,\varepsilon)\}$;
    \item
      $\Delta(q, x, x) \df \{(q, xx), (p, \varepsilon)\}$, където $x \in \{a,b\}$;
    \item
      $\Delta(q, a, b) \df \{(q, ab)\}$;
    \item
      $\Delta(q, b, a) \df \{(q, ba)\}$;
    % \item
      % $\Delta(q, b, b) \df \{(q, bb), (p, \varepsilon)\}$;
    \item
      $\Delta(p, x, x) \df \{(p,\varepsilon)\}$, където $x \in \{a,b\}$;
    % \item
    %   $\Delta(p, b, b) \df \{(p,\varepsilon)\}$;
    \item
      $\Delta(p, \varepsilon, \sharp) \df \{(f,\varepsilon)\}$;
    \end{enumerate}
  \end{itemize}
  Основното наблюдение, което трябва да направим за да разберем конструкцията на автомата е, че
  всяка дума от вида $\omega\omega^{\texttt{rev}}$ може да се запише като $\omega_1aa\omega^{\texttt{rev}}_1$ или $\omega_1bb\omega^{\texttt{rev}}_1$.
  Да видим защо $P$ разпознава думата $abaaba$ с празен стек.
  Започваме по следния начин:
  \begin{align*}
    (q, abaaba,\sharp) & \vdash_P (q, baaba, a\sharp)   & \comment{\text{правило }(1)}\\
                       & \vdash_P (q, aaba, ba\sharp)   & \comment{\text{правило }(5)}\\
                       & \vdash_P (q, aba,  aba\sharp). & \comment{\text{правило }(4)}
  \end{align*}
  Сега можем да направим два избора как да продължим. Състоянието $p$ служи за маркер, което ни казва, че вече сме започнали 
  да четем $\omega^{\texttt{rev}}$. Поради тази причина, продължаваме така:
  \begin{align*}
    (q, aba, aba\sharp) & \vdash_P (p, ba, ba\sharp) & \comment{\text{правило }(3)}\\
                        & \vdash_P (p, a, a\sharp) & \comment{\text{правило }(6)}\\
                        & \vdash_P (p, \varepsilon, \sharp) & \comment{\text{правило }(6)}\\
                        & \vdash_P (f,\varepsilon,\varepsilon). & \comment{\text{правило }(7)}
  \end{align*}
  Да проиграем още един пример. Да видим защо думата $aba$ не се извежда от автомата.
  \begin{align*}
    (q, aba, \sharp) & \vdash_P (q, ba, a\sharp) & \comment{\text{правило }(1)}\\
                     & \vdash_P (q, a, ba\sharp) & \comment{\text{правило }(5)}\\
                     & \vdash_P (q, \varepsilon, aba\sharp). & \comment{\text{правило }(4)}
  \end{align*}
  От последното моментно описание на автомата нямаме нито един преход, следователно
  думата $aba$ не се разпознава от $P$.
  \mynote{\writedown Докажете, че $L = \L(P)$!}
  \begin{enumerate}[a)]
  \item
    \mynote{За (\ref{eq:omega-omega-r:1}) приложете индукция по дължината на думата $\alpha$. За индукционната стъпка разгледайте $\alpha$ като $\alpha = \alpha' x$.}
    Докажете с индукция пo $n$, за всяко естествено число $n$ са изпълнени свойствата:
    \begin{align}
      & |\alpha| = n\ \implies\ (q, \alpha\beta, \sharp) \vdash^n_P (q, \beta, \alpha^{\texttt{rev}}\sharp) \label{eq:omega-omega-r:1}\\
      & |\beta| = n\ \implies\ (p, \beta, \beta\sharp) \vdash^n_P (p, \varepsilon, \sharp). \label{eq:omega-omega-r:2}
    \end{align}
    Оттук заключете, че $L \subseteq \L(P)$.
  \item
    Докажете с индукця по $n$, че за всяко естествено число $n$ са изпълнени свойствата:
    \begin{align}
      (q, \alpha\beta, \sharp) \vdash^n_P (q, \beta, \gamma\sharp) & \implies \gamma = \alpha^{\rev}\ \&\ |\alpha| = n \label{eq:omega-omega-r:3}\\
      (p, \beta, \gamma\sharp) \vdash^n_P (p, \varepsilon, \sharp) & \implies \gamma = \beta\ \&\ |\beta| = n. \label{eq:omega-omega-r:4}
    \end{align}
    Оттук заключете, че $\L(P) \subseteq L$.
  \end{enumerate}
\end{example}

% \begin{problem}
%   Постройте \emph{детерминистичен} стеков автомат за езика $L = \{\omega \$ \omega^{\rev} \mid \omega \in \{a,b\}^\star \}$.
% \end{problem}

\begin{example}
  \mynote{От \Problem{equal-number-parentheses} знаем, че този език е безконтекстен.}
  Езикът $L = \{\ \omega \in \{a,b\}^\star \mid \abs{\omega}_a = \abs{\omega}_b\}$
  се разпознава от стековия автомат $P = \PDA$, където:
  \begin{itemize}
  \item 
    $Q = \{q,f\}$;
  \item
    $\Sigma = \{a,b\}$;
  \item
    $\Gamma = \{a, b, \sharp\}$;
  \item
    $\qstart = q$ и $\qaccept = f$;
  \item
    Можем да дефинираме релацията на преходите $\Delta$ по следния начин:
    \begin{enumerate}[(1)]
    \item 
      $\Delta(q, \varepsilon, \sharp) = \{(f, \varepsilon)\}$;
    \item
      $\Delta(q, x, \sharp) = \{(q, x\sharp)\}$, където $x \in \{a,b\}$;
    \item
      $\Delta(q, x, x) = \{(q, xx)\}$, където $x \in \{a,b\}$;
    \item
      $\Delta(q, a, b) = \{(q, \varepsilon)\}$;
    \item
      $\Delta(q, b, a) = \{(q, \varepsilon)\}$.
    \end{enumerate}
  \end{itemize}
  Да видим защо думата $abbbaa \in \L(P)$.
  \begin{align*}
    (q, abbbaa, \sharp) & \vdash_P (q, bbbaa,\ a\sharp) & \comment{\text{правило }(2)}\\
                        & \vdash_P (q, bbaa,\ \sharp) & \comment{\text{правило }(5)}\\
                        & \vdash_P (q, baa,\ b\sharp) & \comment{\text{правило }(2)}\\
                        & \vdash_P (q, aa, bb\sharp) & \comment{\text{правило }(3)}\\
                        & \vdash_P (q, a,\ b\sharp) & \comment{\text{правило }(4)}\\
                        & \vdash_P (q, \varepsilon,\ \sharp) & \comment{\text{правило }(4)}\\
                        & \vdash_P (f, \varepsilon,\ \varepsilon). & \comment{\text{правило }(1)}
  \end{align*}

  \begin{enumerate}[a)]
  \item
    Докажете с пълна индукция по дължината на думата $\gamma$, че за произволна дума $\gamma \in \{a, b\}^\star$, е изпълнено, че:
    \begin{align}
      (\forall n)[a^n\gamma \in L & \implies (q, \gamma, a^n\sharp) \vdash^\star_P (q, \varepsilon, \sharp)] \label{eq:omega-ab:1}\\
      (\forall n)[b^n\gamma \in L & \implies (q, \gamma, b^n\sharp) \vdash^\star_P (q, \varepsilon, \sharp)]. \label{eq:omega-ab:2}
    \end{align}
    % Ще докажем едновременно \Property{eq:omega-ab:1} и \Property{eq:omega-ab:2} с пълна индукция по дължината на думата $\gamma$.

    % Да разгледаме произволна дума $\gamma$. Случаят, когато $\abs{\gamma} = 0$ е тривиален. 
    % Нека $\abs{\gamma} > 0$ и да приемем, че $a^n\gamma \in L$.
    % Ясно е, че можем да представим $\gamma$ като $\gamma = a^kb\gamma'$, за някое $k$.
    % \begin{itemize}
    % \item
    %   Ако $n+k = 0$, то $a^n\gamma = b\gamma' \in L$ и прилагаме \IndHyp за \Property{eq:omega-ab:2} с думата $\gamma'$ и получаваме, че:
    %   \begin{align*}
    %     (q, \overbrace{b\gamma'}^{\gamma},\sharp) & \vdash (q,\gamma',b\sharp) & \comment{\text{правило (2)}}\\
    %                          & \vdash^\star (q, \varepsilon,\sharp) & \comment\text{\IndHyp}
    %   \end{align*}
    % \item
    %   Ако $n+k>0$, то $a^{n+k-1}\gamma' \in L$ и прилагаме \IndHyp за \Property{eq:omega-ab:1} с думата $\gamma'$ и получаваме, че:
    %   \begin{align*}
    %     (q, \overbrace{a^{k}b\gamma'}^{\gamma}, a^n\sharp) & \vdash^\star_P (q, b\gamma', a^{n+k}\sharp) & \comment{\text{правило (3)}}\\
    %                                                         & \vdash^\star_P (q, \gamma', a^{n+k-1}\sharp) & \comment\text{правило (5)}\\
    %                                                         & \vdash^\star_P (q, \varepsilon, \sharp). & \comment\text{\IndHyp}
    %   \end{align*}
    % \end{itemize}
    % Аналогично се доказва и \Property{eq:omega-ab:2}.
    
    Оттук заключете, че $L \subseteq \L(P)$.
  \item
    Докажете с пълна индукция по дължината на думата $\gamma$, че за произволна дума $\gamma \in \{a, b\}^\star$ е изпълнено, че:
    \begin{align}
      (\forall n)[(q, \gamma, a^n\sharp) \vdash^\star_P (q, \varepsilon, \sharp) & \implies a^n\gamma \in L] \label{eq:omega-ab:3}\\
      (\forall n)[(q, \gamma, b^n\sharp) \vdash^\star_P (q, \varepsilon, \sharp) & \implies b^n\gamma \in L]. \label{eq:omega-ab:4}
    \end{align}

    % Нека имаме изчислението $(q, \gamma, a^n\sharp) \vdash^\star_P (q, \varepsilon, \sharp)$.
    % Да представим думата $\gamma$ като $\gamma = a^kb\gamma'$.
    
    % \begin{itemize}
    % \item
    %   Ако $n+k = 0$, то можем да разбием изчислението по следния начин:
    %   \begin{align*}
    %     (q, b\gamma', \sharp) & \vdash_P (q, \gamma',b\sharp) \\
    %                           & \vdash^{\star} (q,\varepsilon,\sharp).
    %   \end{align*}
    %   Тогава от \IndHyp за \Property{eq:omega-ab:4} следва, че $a^n\gamma = b\gamma' \in L$.
    % \item
    %   Ако $n+k > 0$, то можем да разбием изчислението по следния начин:
    %   \begin{align*}
    %     (q, a^kb\gamma', a^n\sharp) & \vdash^\star_P (q, b\gamma',a^{n+k}\sharp) \\
    %                                 & \vdash_P (q, \gamma',a^{n+k-1}\sharp) \\
    %                                 & \vdash^{\star} (q,\varepsilon,\sharp).
    %   \end{align*}
    %   Тогава от \IndHyp за \Property{eq:omega-ab:3} следва, че $a^{n+k-1}\gamma' \in L$, но оттук
    %   веднага получаваме, че $a^na^kb\gamma' \in L$.
    % \end{itemize}
    % Аналогично се доказва и \Property{eq:omega-ab:4}.
    
    Оттук заключете, че $\L(P) \subseteq L$.
  \end{enumerate}
\end{example}

\begin{example}
  \mynote{От \Problem{balanced-parentheses} знаем, че този език е безконтекстен.}
  Езикът $L = \{\ \omega \in \{a,b\}^\star \mid \omega\text{ е балансирана дума}\ \}$
  се разпознава от стековия автомат $P = \PDA$, където:
  \begin{itemize}
  \item 
    $Q = \{q,f\}$;
  \item
    $\qstart = q$ и $\qaccept = f$;
  \item
    $\Sigma = \{a,b\}$ и $\Gamma = \{a, \sharp\}$;
  \item
    Можем да дефинираме релацията на преходите $\Delta$ по следния начин:
    \mynote{\writedown Докажете, че $L = \L(P)$!}
    \begin{enumerate}[(1)]
    \item 
      $\Delta(q, \varepsilon, \sharp) = \{(f, \varepsilon)\}$;
    \item
      $\Delta(q, a, \sharp) = \{(q, a\sharp)\}$;
    \item
      $\Delta(q, a, a) = \{(q, aa)\}$;
    \item
      $\Delta(q, b, a) = \{(q, \varepsilon)\}$;
    \end{enumerate}
  \end{itemize}  
  \begin{enumerate}[(a)]
  \item
    \mynote{Индукция по дължината на думата $\beta$.}
    Докажете, че за произволно естествено число $n$ и произволна дума $\beta \in \{a, b\}^\star$, 
    е изпълнено, че:
    \[a^n\beta \in L\ \implies (q, \beta, a^n\sharp) \vdash^\star_P (q, \varepsilon, \sharp).\]
    Оттук заключете, че $L \subseteq \L(P)$.
  \item
    \mynote{Индукция по броя на стъпките в изчислението на стековия автомат.}
    Докажете, че за произволно естествено число $n$ и произволна дума $\beta \in \{a, b\}^\star$, е изпълнено, че:
    \[(q,\beta,a^n\sharp) \vdash^\star_P (q, \varepsilon, \sharp)\ \implies\ a^n\beta \in L.\]
    Оттук заключете, че $\L(P) \subseteq L$.
  \end{enumerate}
\end{example}
\end{extra}


%%% Local Variables:
%%% mode: latex
%%% TeX-master: "../eai"
%%% End:


\begin{framed}
  \begin{lemma}
    За всяка безконтекстна граматика $G$,
    съществува стеков автомат $P$, такъв че $\L(G) = \L(P)$.
  \end{lemma}
\end{framed}
\begin{proof}
  \mynote{\cite[стр. 136]{papadimitriou}}
  % \mynote{Доказателството в \cite[стр. 117]{sipser3} не ми харесва}
  \mynote{Тук приемаме, че винаги правим най-ляв извод в граматиката}
  Нека е дадена безконтекстната граматика $G = \CFG$ в нормална форма на Чомски.
  Нашата цел е да построим стеков автомат
  \[P = \PDA,\] който разпознава $\L(G)$.
  \begin{itemize}
  \item
    $Q = \{\qstart,p,\qaccept\}$;
  \item
    $\Gamma = \Sigma \cup V \cup \{\sharp\}$;
  \item
    Релацията на преходите $\Delta$ дефинираме по следния начин:
    \mynote{Понеже граматиката е в нормална форма на Чомски, то $|\alpha| \leq 2$ и удовлетворяваме дефиницията на $\Delta$.}
    \begin{enumerate}[(1)]
    \item 
      $\Delta(\qstart, \varepsilon, \sharp ) = \{(p,S\sharp)\}$;
    \item
      $\Delta(p,\varepsilon,A) = \{(p,\alpha)\mid A\to_G \alpha\}, \text{ за всяка променлива }A \in V$;
    \item
      $\Delta(p,a,a) = \{(p,\varepsilon)\}, \text{ за всяка буква } a \in \Sigma$;
    \item
      $\Delta(p,\varepsilon,\sharp) = \{(\qaccept, \varepsilon)\}$.
    \end{enumerate}
  \end{itemize}
  
  Ще докажем, че за всяка променлива $A \in V$, за всяка дума $\alpha \in \Sigma^\star$ и $\gamma \in (\Sigma \cup V)^\star$, то е изпълнено, че:
  \mynote{Трябва да се каже, че $S \to^\star_G \alpha \gamma$ го правим като заместваме най-лявата променлива.}
  \begin{enumerate}[(a)]
  \item
    ако $S \derive{\star}_G \alpha \gamma$, то $(p, \alpha, S\sharp) \vdash^\star_P (p, \varepsilon, \gamma\sharp)$;
  \item
    ако $(p, \alpha, \gamma\sharp) \vdash^\star_P (p, \varepsilon, \sharp)$, то $\gamma \derive{\star}_G \alpha$.
  \end{enumerate}
  Тогава, ако вземем $\gamma = \varepsilon$, то ще получим, че
  \begin{align*}
    \alpha \in \L(G) & \iff S \to^\star_G \alpha\\
                     & \iff (p,\alpha,S\sharp) \vdash^\star_P (p, \varepsilon, \sharp) & \comment{\text{от (а) и (б)}}\\
                     & \iff (\qstart,\alpha,\sharp) \vdash^\star_P (\qaccept, \varepsilon, \varepsilon) & \comment{\text{от деф. на }\Delta}\\
                     & \iff \alpha \in \L(P).
  \end{align*}

  Сега преминаваме към доказателствата на двете твърдения.

  \begin{enumerate}[(a)]
  \item
    Индукция по дължината на извода $S \to^\star_G \alpha\gamma$.
    Нека $\ell = 0$. Този случай е тривиален, защото тогава $\alpha = \varepsilon$ и $\gamma = S$.
    Ясно е, че
    \[(p,\varepsilon,S\sharp) \vdash^0_P (p,\varepsilon,S\sharp).\]

    Нека $\ell > 0$ и $S \stackrel{\ell}{\to}_G \alpha\gamma$. Това означава, че този извод може да се запише по следния начин:
    \[S \stackrel{\ell-1}{\to}_G \alpha_1A\gamma_2 \to_G \underbrace{\alpha_1\alpha_2}_{\alpha}\underbrace{\gamma_1\gamma_2}_{\gamma},\]
    където $A \to_G \alpha_2\gamma_1$ е правилото в граматиката, което сме приложили най-накрая. Тогава от И.П. имаме, че
    \begin{equation}
      \label{eq:5}
      (p, \alpha_1, S\sharp) \vdash^\star_P (p, \varepsilon, A\gamma_2\sharp).
    \end{equation}
    Тогава имаме следното изчисление на стековия автомат:
    \begin{align*}
      (p, \alpha_1\alpha_2, S\sharp) & \vdash^\star_P (p, \alpha_2, A\gamma_2\sharp) & \comment{\text{от (\ref{eq:5})}}\\
                                     & \vdash_P (p, \alpha_2, \alpha_2\gamma_1\gamma_2\sharp) & \comment{\text{ред (2) от деф. на }\Delta}\\
                                     & \vdash^\star_P (p, \varepsilon, \gamma_1\gamma_2\sharp) & \comment{\text{ред (3) от деф. на }\Delta}.
    \end{align*}
  \item
    Индукция по броя на стъпките $\ell$ в изчислението на стековия автомат.
    Нека $\ell = 0$. Тогава е ясно, че единствената възможност $\alpha = \varepsilon$ и $\gamma = \varepsilon$.
    Тогава $\varepsilon \derive{\star}_G \varepsilon$.
    
    Нека $\ell > 0$ и $(p, \alpha, \gamma \sharp) \vdash^{\ell}_P (p, \varepsilon, \sharp)$.
    Имаме три избора за първата стъпка в това изчисление.
    \begin{itemize}
    \item
      $\Delta(p,a,a) \ni (p,\varepsilon)$.
      Това означава, че $\alpha = a\beta$, $\gamma = a\gamma_1$ и
      \[(p, \alpha, \gamma \sharp) \vdash_P (p,\beta,\gamma_1\sharp ) \vdash^{\ell-1}_P (p, \varepsilon, \sharp).\]
      Тогава от И.П. получаваме, че $\gamma_1 \derive{\star}_G \beta$ и оттук
      \[\underbrace{a \gamma_1}_{\gamma} \derive{\star}_G \underbrace{a\beta}_{\alpha}.\]
    \item
      $\Delta(p,\varepsilon,A) \ni (p,a)$. Това означава, че $A \to_G a$, $\gamma = A\gamma_1$ и
      \[(p, \alpha, \gamma \sharp) \vdash_P (p,\alpha,a\gamma_1\sharp ) \vdash^{\ell-1}_P (p, \varepsilon, \sharp).\]
      Тогава от И.П. получаваме, че $a\gamma_1 \derive{\star}_G \alpha$ и оттук
      \[\underbrace{A\gamma_1}_{\gamma} \derive{\star}_G \alpha.\]
    \item
      Нека $\Delta(p,\varepsilon,A) \ni (p,BC)$. Това означава, че $A \to_G BC$, $\gamma = A\gamma_1$ и
      \[(p, \alpha, \gamma \sharp) \vdash_P (p,\alpha, BC\gamma_1\sharp ) \vdash^{\ell-1}_P (p, \varepsilon, \sharp).\]
      Тогава от И.П. получаваме, че
      $BC\gamma_1 \derive{\star}_G \alpha$. Заключаваме, че
      \[ \underbrace{A\gamma_1}_{\gamma}\derive{\star}_G \alpha.\]
    \end{itemize}
  \end{enumerate}
\end{proof}

\begin{framed}
  \begin{lemma}
    За всеки стеков автомат $P$, съществува безконтекстна граматика $G$, такава че $\L(P) = \L(G)$.
  \end{lemma}
\end{framed}
\begin{proof}
  Нека е даден стековия автомат
  \[P = \PDA.\]
  Ще дефинираме безконтекстна граматика $G$, за която $\L(P) = \L(G)$.
  Променливите на граматика са 
  \[V = \{[q,A,p] \mid q,p \in Q, A \in \Gamma\}.\]
  Правилата на $G$ са следните:
  \begin{itemize}
  \item
    Началната променлива е $[\qstart,\sharp,\qaccept]$;
  \item
    Нека имаме $(r,BC) \in \Delta(q, a, A)$, където $a \in \Sigma_\varepsilon$.
    Тогава добавяме правилата в граматиката:
    \[[q,A,p] \to_G a[r,B_1,q'][q',B_2,p],\]
    за всеки две състояние $q'$ и $p$.
  \item
    Нека имаме $(r,B) \in \Delta(q, a, A)$, където $a \in \Sigma_\varepsilon$.
    Тогава добавяме правилата в граматиката:
    \[[q,A,p] \to_G a[r,B,p],\]
    за всяко състояние $p \in Q$.
  \item
    Нека имаме $(p,\varepsilon) \in \Delta(q,a,A)$, където $a \in \Sigma_\varepsilon$.
    Тогава добавяме правилата в граматиката:
    \[[q,A,p] \to a.\]
  \end{itemize}
  Трябва да докажем, че за произволна дума $\alpha \in \Sigma^\star$, произволни състояния $q,p \in Q$,
  и произволен символ $A \in \Gamma$, е изпълнено, че:
  \[[q,A,p] \derive{\star}_G \alpha\ \Leftrightarrow\ (q,\alpha,A) \vdash^\star_{P} (p,\varepsilon,\varepsilon).\]
  \begin{description}
  \item[$(\Rightarrow)$]
    С пълна индукция по броят на стъпките $\ell$ в изчислението на стековия автомат $P$ ще докажем, че:
    \[(q,\alpha,A) \vdash^\star_P (p,\varepsilon,\varepsilon)\ \implies\ [q,A,p] \derive{\star}_G \alpha.\]
    Ако $\ell = 1$, то е лесно, защото $\alpha = a \in \Sigma_\varepsilon$.
    Тогава $(p,\varepsilon) \in \Delta(q,a,A)$ и според конструкцията на граматиката $G$ имаме правилото $[q,A,p] \to_G a$.
    
    Ако $\ell > 1$, то в зависимост от първата стъпка на изчислението, имаме два случая.
    Нека $\alpha = a\beta$, където $a \in \Sigma_\varepsilon$.
    \begin{itemize}
    \item 
      Ако $\Delta(q,a,A) \ni (r,B)$, то имаме, че:
      \[(q,a\beta,A) \vdash_P (r,\beta,B) \vdash^{\ell-1}_P (p, \varepsilon, \varepsilon).\]
      Тогава от И.П. получаваме, че
      \[[r,B,p] \derive{\star}_G \beta.\]
    \item
      Ако $\Delta(q, a, A) \ni (r, BC)$, то имаме, че:
      \[(q, a\beta, A) \vdash_P (r, \beta, BC) \vdash^{\ell-1}_P (p, \varepsilon, \varepsilon).\]      
      За $\ell-1$ стъпки трябва да стигнем от стек с големина $2$ до празен стек.
      Това означава, че можем да разбием думата $\beta$ на две части, $\beta = \beta_1\beta_2$, със свойството, че след като прочетем $\beta_1$,
      то стекът има големина $1$ и след като прочетем $\beta_2$, то стекът е празен.
      \mynote{Да обърнем внимание, че в междинните стъпки от двете изчисления, стекът може да расте.}
      Това означава, че съществува състояние $q'$, за което можем да разбием изчислението по следния начин:
      \begin{align*}
        & (r, \beta_1, B) \vdash^{\ell_1}_P (q',\varepsilon,\varepsilon)\\
        & (q', \beta_2, C) \vdash^{\ell_2}_P (p,\varepsilon,\varepsilon).
      \end{align*}
      където $\ell_1 + \ell_2 = \ell - 1$.    
      Сега от {\bf И.П.} получаваме:
      \begin{align*}
        & (r, \beta_1, B) \vdash^{\ell_1}_P (q', \varepsilon, \varepsilon) \implies [r, B, q'] \derive{\star}_G \beta_1\\
        & (q', \beta_2, C) \vdash^{\ell_2}_P (p, \varepsilon, \varepsilon) \implies [q', C, p] \derive{\star}_G \beta_2.
      \end{align*}
      Тогава от правилата за извод в граматика получваме, че
      \[[r,B,q'][q',C,p] \derive{\star}_G \beta_1\beta_2.\]
      Понеже имаме, че $\Delta(q,a,A) \ni (r,BC)$, то в граматиката имаме правилото
      \[[q,A,p] \to_G a[r,B,q'][q',C,p].\]
      Обединявайки всичко, получаваме извода
      \[[q,A,p] \derive{\star}_G \underbrace{a\beta_1\beta_2}_{\alpha}.\]
    \end{itemize}
  \item[$(\Leftarrow)$]
    Този път с пълна индукция по дължината на извода $\ell$ в граматиката $G$ ще докажем, че
    \[[q,A,p] \derive{\star}_G \alpha \implies (q,\alpha,A) \vdash^\star_P (p,\varepsilon,\varepsilon).\]
    Ако $\ell = 1$, то имаме $[q,A,p] \rightarrow \alpha$, където $\alpha \in \Sigma_\varepsilon$.
    Този случай е ясен от дефиницията на граматиката $G$, т.е. $\Delta(q,a,A) \ni (p,\varepsilon)$.

    Ако $\ell > 1$, то имаме, че $\alpha = a\beta$ и според правилата на граматиката $G$ имаме два случая.
    \mynote{Тук отново е възможно $a = \varepsilon$. Това не е проблем, защото правим индукция по дължината на извода, а не по дължината на думата $\alpha$.}
    Ако
    \[[q,A,p] \derive{1}_G a[r,B,p] \derive{\ell-1}_G a\beta,\]
    то директно прилагаме И.П. и получаваме, че
    $(r, \beta, B) \vdash^\star_P (p, \varepsilon, \varepsilon)$ и накрая получаваме, че
    \[(q, a\beta, A) \vdash^\star_P (p, \varepsilon, \varepsilon).\]
    Сега да разгледаме втория случай:
    \[[q,A,p] \derive{1}_G a[r,B,q'][q',C,p] \derive{\ell-1}_G a\beta.\]
    От \Proposition{grammar:divide} следва, че имаме разбиване на думата $\beta$ като $\beta = \beta_1\beta_2$, където 
    \begin{align*}
      & [r,B,q'] \derive{\ell_1}_G \beta_1\\
      & [q',C,p] \derive{\ell_2}_G \beta_2.
    \end{align*}
    където $\ell_1 + \ell_2 = \ell - 1$.
    От {\bf И.П.} получаваме, че 
    \begin{align*}
      & [r,B,q'] \derive{\ell_1}_G \beta_1 \implies (r,\beta_1,B) \vdash^\star_P (q',\varepsilon,\varepsilon) \\
      & [q',C,p] \derive{\ell_2}_G \beta_2 \implies (q',\beta_2,C) \vdash^\star_P (p,\varepsilon,\varepsilon).
    \end{align*}
    Правилото
    \[[q,A,p] \rightarrow_G a[r,B,q'][q',C,p]\]
    е добавено в граматиката, защото $\Delta(q,a,A) \ni (r, BC)$. 
    Обединявайки всичко, което знаем, получаваме:
    \begin{align*}
      (q, a\beta, A) & \vdash_P (r, \beta_1\beta_2, BC)\\
                     & \vdash^\star_P (q', \beta_2, C)\\
                     & \vdash^\star_P (p, \varepsilon, \varepsilon).
    \end{align*}    
  \end{description}
\end{proof}

Предишните две леми ни дават следната теорема.
\begin{framed}
\begin{thm}
  \label{th:push-down-context-free}
  Класът на езиците, които се разпознават от краен стеков автомат съвпада с
  класа на безконтекстните езици.
\end{thm}
\end{framed}

\begin{example}
  Нека е дадена граматиката $G$ с правила 
  \begin{align*}
    & S \to AS\ |\ BS\ |\ \varepsilon\\
    & A \to aA\ |\ a\\
    & B \to Bb\ |\ b.
  \end{align*}
  Ще построим стеков автомат $P = \PDA$, такъв че $\L(P) = \L(G)$.
  \begin{itemize}
  \item
    $\Sigma = \{a,b\}$;
  \item 
    $\Gamma = \{A,S,B,a,b,\sharp\}$;
  \item
    $Q = \{\qstart,q,\qaccept\}$;
  \item
    Дефинираме релацията на преходите, следвайки конструкцията от \Theorem{push-down-context-free}:
    \begin{itemize}
    \item
      $\Delta(\qstart, \varepsilon, \sharp) = \{(q, S\sharp)\}$;
    \item 
      $\Delta(q, \varepsilon, S) = \{(q, AS), (q, BS), (q, \varepsilon)\}$;
    \item
      $\Delta(q, \varepsilon, A) = \{(q, aA), (q, a)\}$;
    \item
      $\Delta(q, \varepsilon, B) = \{(q, Bb), (q, b)\}$;
    \item
      $\Delta(q, a, a) = \{(q, \varepsilon)\}$;
    \item
      $\Delta(q, b, b) = \{(q, \varepsilon)\}$;
    \item
      $\Delta(q, \varepsilon, \sharp) = \{(\qaccept,\varepsilon)\}$;
    \end{itemize}
  \end{itemize}
\end{example}

\begin{framed}
  \begin{thm}
    \label{th:intersection-context-reg}
    Нека $L$ e безконтекстен език и $R$ е регулярен език.
    Тогава тяхното сечение $L \cap R$ е безконтекстен език.
  \end{thm}  
\end{framed}
\begin{hint}
  \mynote{\cite[стр. 144]{papadimitriou}}
  Нека имаме стеков автомат
  \[P = \pair{Q',\Sigma,\Gamma,\sharp, \Delta', \qstart', \qaccept'}, \text{ където } \L(P) = L,\]
  и краен тотален детерминиран автомат 
  \[\A = \pair{Q'', \Sigma, \qstart'', \delta'', F''}, \text{ където } \L(\A) = R.\]
  \mynote{Сравнете с конструкцията от \Proposition{automata-cap}.}
  Ще определим нов стеков автомат
  \[\M = \PDA,\]
  където
  \begin{itemize}
  \item 
    $Q = Q' \times Q''$;
  \item
    $\qstart = \pair{\qstart',\qstart''}$;
  \item
    $F = \{\qaccept'\} \times F''$;
  \item 
    Функцията на преходите $\Delta$ е дефинирана както следва:
    \begin{itemize}
    \item 
      \mynote{Симулираме едновременно изчислението и на двата автомата.}
      Ако $(r_1,Z) \in \Delta'(q_1, a, Y)$, то
      \[(\pair{r_1,\delta''(q_2,a)}, Z) \in \Delta(\pair{q_1,q_2},a,Y).\]
    \item
      \mynote{Нищо не четем от входната дума, следователно правим празен ход на $\A$}
      Ако $(r_1,Z) \in \Delta'(q_1,\varepsilon,Y)$ и всяко $q_2 \in Q''$, то
      \[(\pair{r_1,q_2}, Z) \in \Delta(\pair{q_1,q_2},\varepsilon,Y).\]
    \item
      \mynote{\writedown Докажете, че $\L(\M) = \L(P) \cap \L(\A)$ !}
      $\Delta$ не съдържа други преходи;
    \end{itemize}
  \end{itemize}

  \begin{itemize}
  \item
    \mynote{Индукция по броя стъпки в изчислението на $\M$.}
    Докажете, че ако $(\pair{q_1,q_2},\alpha,\gamma) \vdash^\star_\M (\pair{p_1,p_2},\varepsilon,\varepsilon)$, то
    $(q_1,\alpha,\gamma) \vdash^\star_P (p_1,\varepsilon,\varepsilon)$ и $(q_2,\alpha) \vdash^\star_\A (p_2,\varepsilon)$.
  \item
    \mynote{Индукция по броя стъпки в изчислението на $P$.}
    Докажете, че ако $(q_1,\alpha,\gamma) \vdash^\star_P (p_1,\varepsilon,\varepsilon)$ и $(q_2,\alpha) \vdash^\star_\A (p_2,\varepsilon)$, то
    $(\pair{q_1,q_2},\alpha,\gamma) \vdash^\star_\M (\pair{p_1,p_2},\varepsilon,\varepsilon)$.
  \end{itemize}
  
\end{hint}

\Theorem{intersection-context-reg} е удобна, когато искаме да докажем, че даден език не е безконтекстен.
С нейна помощ можем да сведем езика до друг, за който вече знаем, че не е безконтекстен.

\begin{example}
  Езикът $L = \{\omega \in \{a,b,c\}^\star \mid N_a(\omega) = N_b(\omega) = N_c(\omega)\}$ не е безконтекстен.
  Да допуснем, че $L$ е безконтекстен език.
  Тогава \[L^\prime = L \cap \L(a^\star b^\star c^\star)\] също е безконтекстен език.
  Но $L^\prime = \{a^nb^nc^n \mid n \in \Nat\}$, за който знаем от \Problem{anbncn}, че {\em не} е безконтекстен.
  Достигнахме до противоречие. Следователно, $L$ не е безконтекстен език.
\end{example}

\begin{framed}
  \begin{remark}
    Не е вярно, че сечението на всеки два безконтекстни езика е безконтекстен език.

    Например, $L_1 = \{a^nb^nc^k \mid n,k\in\Nat\}$ и $L_2 = \{a^kb^nc^n \mid n,k\in\Nat\}$
    са безконтекстни езици, но ние знаем, че
    \[L_1 \cap L_2 = \{a^nb^nc^n \mid n \in \Nat\}\]
    не е безконтекстен.
  \end{remark}
\end{framed}

%%% Local Variables: 
%%% mode: latex
%%% TeX-master: "../eai"
%%% End: 

\newpage
\section{Допълнителни задачи}

\subsection{Равен брой скоби}

Тук ще разглеждаме азбука $\Sigma$, която включва буквите $\texttt{[}$ и $\texttt{]}$.
Нека за по-голяма яснота да положим
\begin{align*}
  & \texttt{left}(\alpha) \df N_{\texttt{[}}(\alpha) & \comment{\text{брой срещания на $\texttt{[}$ в $\alpha$}}\\
  & \texttt{right}(\alpha) \df N_{\texttt{]}}(\alpha). & \comment{\text{брой срещания на $\texttt{]}$ в $\alpha$}}
\end{align*}

\begin{problem}
  \label{prob:nanb}
  Нека $\omega$ е произволна дума над азбуката $\{\texttt{[}, \texttt{]}\}$. 
  Тогава:
  \begin{enumerate}[a)]
  \item 
    ако $\texttt{left}(\omega) = \texttt{right}(\omega) + 1$, то съществуват думи $\omega_1$, $\omega_2$, за които е изпълнено:
    \begin{itemize}
    \item 
      $\omega = \omega_1 \texttt{[} \omega_2$;
    \item
      $\texttt{left}(\omega_1) = \texttt{right}(\omega_1)$;
    \item
      $\texttt{left}(\omega_2) = \texttt{right}(\omega_2)$.
    \end{itemize}
  \item
    ако $\texttt{right}(\omega) = \texttt{left}(\omega) + 1$, то съществуват думи $\omega_1$, $\omega_2$, за които е изпълнено:
    \begin{itemize}
    \item 
      $\omega = \omega_1 \texttt{]} \omega_2$;
    \item
      $\texttt{left}(\omega_1) = \texttt{right}(\omega_1)$;
    \item
      $\texttt{left}(\omega_2) = \texttt{right}(\omega_2)$.
    \end{itemize}
  \end{enumerate}
\end{problem}
\begin{hint}
  \marginpar{Другият случай е аналогичен}
  Ще се съсредоточим върху случая, когато $\omega$ е дума, за която $\texttt{left}(\omega) = \texttt{right}(\omega) + 1$.
  Ще докажем а) с индукция по дължината на думата.
  \begin{itemize}
  \item 
    $\abs{\omega} = 1$. Тогава $\omega_1 = \omega_2 = \varepsilon$ и $\omega = \texttt{[}$.
  \item
    Да приемем, че твърдението а) е вярно за думи с дължина $\leq n$.
  \item
    $\abs{\omega} = n+1$. Ще разгледаме два случая, в зависимост от първия символ на $\omega$.
    \begin{itemize}
    \item 
      Случаят $\omega = \texttt{[}\omega'$ е очевиден. (Защо?)
    \item
      Интересният случай е $\omega = \texttt{]}\omega'$.    
      Тогава $\omega = \texttt{]}^{i+1}\texttt{[}\omega'$, за някое $i \in \Nat$.
      Да разгледаме думата $\omega''$, която се получава от $\omega$
      като премахнем първото срещане на думата $\texttt{][}$, т.е. 
      $\omega'' = \texttt{]}^i\omega'$ и $\abs{\omega''} = n-1$.
      Понеже от $\omega$ сме премахнали равен брой леви и десни скоби, то
      $\texttt{left}(\omega'') = \texttt{right}(\omega'')+1$.
      Според {\bf И.П.} за $\omega''$ са изпълнени свойствата:
      \begin{itemize}
      \item 
        $\omega'' = \omega''_1\texttt{[}\omega''_2$;
      \item
        $\texttt{left}(\omega''_1) = \texttt{right}(\omega''_1)$;
      \item
        $\texttt{left}(\omega''_2) = \texttt{right}(\omega''_2)$.
      \end{itemize}
      Понеже $\texttt{]}^i$ е префикс на $\omega''_1$, за да получим обратно $\omega$, трябва 
      да прибавим премахнатата част $\texttt{][}$ веднага след $\texttt{]}^i$ в $\omega''_1$.
    \end{itemize}
  \end{itemize}
\end{hint}

\begin{problem}
  За произволна дума $\omega \in \{ \texttt{[}, \texttt{]} \}^\star$, 
  докажете, че ако $\texttt{left}(\omega) > \texttt{right}(\omega)$, то съществуват думи $\omega_1$ и $\omega_2$,
  за които са изпълнени свойствата:
  \begin{itemize}
  \item 
    $\omega = \omega_1 \texttt{[} \omega_2$;
  \item
    $\texttt{left}(\omega_1) \geq \texttt{right}(\omega_1)$;
  \item
    $\texttt{left}(\omega_2) \geq \texttt{right}(\omega_2)$.
  \end{itemize}
\end{problem}

\begin{framed}
  \begin{problem}
    Да се докаже, че езикът 
    \[L = \{\alpha \in \{\texttt{[}, \texttt{]}\}^\star\mid \texttt{left}(\alpha) = \texttt{right}(\alpha)\}\]
    е безконтекстен.
  \end{problem}  
\end{framed}
\begin{hint}
  \marginpar{  Алтернативна граматика за езика $L$ е
    \[S \to \varepsilon\ |\ \texttt{[}S\texttt{]}\ |\ \texttt{]}S\texttt{[}\ |\ SS.\]}
  Една възможна граматика $G$ е следната: 
  \[S \to \texttt{[}S\texttt{]}S\ |\ \texttt{]}S\texttt{[}S\ |\ \varepsilon.\]
  % Например, да разгледаме извода на думата $aabbba$ в тази граматика:
  % \begin{align*}
  %   S & \to aSbS \to aaSbSbS \to aa\varepsilon bSbS \to aab\varepsilon bS \to aabbbSaS\\
  %   & \to aabbb\varepsilon a S \to aabbba.
  % \end{align*}
  
  Като следствие от \Prob{nanb} може лесно да се изведе, че за думи $\omega$, за които $\texttt{left}(\omega) = \texttt{right}(\omega)$,
  е изпълнено следното:
  \begin{enumerate}[a)]
  \item 
    ако $\omega = \texttt{[}\omega'$, то са изпълнени свойствата:
    \begin{itemize}
    \item 
      $\omega = \texttt{[}\omega_1\texttt{]}\omega_2$;
    \item
      $\texttt{left}(\omega_1) = \texttt{right}(\omega_1)$;
    \item
      $\texttt{left}(\omega_2) = \texttt{right}(\omega_2)$.
    \end{itemize}
  \item
    ако $\omega = \texttt{]}\omega'$, то са изпълнени свойствата:
    \begin{itemize}
    \item 
      $\omega = \texttt{]}\omega_1\texttt{[}\omega_2$;
    \item
      $\texttt{left}(\omega_1) = \texttt{right}(\omega_1)$;
    \item
      $\texttt{left}(\omega_2) = \texttt{right}(\omega_2)$.
    \end{itemize}
  \end{enumerate}

  Сега първо ще проверим, че $L \subseteq \L(G)$.
  За целта ще докажем с {\em пълна индукция} по дължината на думата $\omega$, че за всяка дума $\omega$ със свойството $\texttt{left}(\omega) = \texttt{right}(\omega)$ е изпълнено
  $S \rightarrow^\star \omega$.
  \begin{itemize}
  \item 
    Нека $\abs{\omega} = 0$. Тогава $S \rightarrow \varepsilon$.
  \item
    Да приемем, че за всяка дума с дължина $\leq k$ твърдението е вярно.
  \item
    Нека $\abs{\omega} = k+1$. Имаме два случая.
    \begin{itemize}
    \item 
      $\omega = \texttt{[}\omega^\prime$, т.е. от а) на \Prob{nanb}, 
      $\omega = \texttt{[}\omega_1\texttt{]}\omega_2$ и $\texttt{left}(\omega_1) = \texttt{right}(\omega_1)$, $\texttt{left}(\omega_2) = \texttt{right}(\omega_2)$.
      Тогава $\abs{\omega_1} \leq k$ и по И.П. $S \rightarrow^\star \omega_1$.
      Аналогично, $S \rightarrow^\star \omega_2$.
      Понеже имаме правило $S \rightarrow \texttt{[}S\texttt{]}S$, заключаваме че 
      $S \to^\star \texttt{[}\omega_1\texttt{]}\omega_2$.
    \item
      $\omega = \texttt{]}\omega^\prime$, т.е. свойство б), $\omega = \texttt{]}\omega_1\texttt{[}\omega_2$ и 
      $\texttt{left}(\omega_1) = \texttt{right}(\omega_1)$, $\texttt{left}(\omega_2) = \texttt{right}(\omega_2)$.
      Този случай се разглежда аналогично.
    \end{itemize}
  \end{itemize}
  
  Преминаваме към доказателството на другата посока, т.е. $\L(G) \subseteq L$.
  Тук с индукция по дължината на извода $l$ ще докажем, че
  $S \stackrel{l}{\to} \omega$, то $\omega \in M$,
  където
  \[M = \{\omega \in \{a,b,S\}^\star \mid \texttt{left}(\omega) = \texttt{right}(\omega)\}.\]
  \begin{itemize}
  \item 
    Ясно, че $S \stackrel{0}{\rightarrow} S$ и $S \in M$.
  \item
    Да разгледаме дума $\omega$, за която $S \stackrel{k+1}{\to} \omega$.
    Това означава, че съществува дума $\alpha$, за която
    \[S \stackrel{k}{\to} \alpha \to \omega.\]
    От {\bf И.П.} имаме, че $\alpha \in M$.
    Нека $\omega$ се получава от $\alpha$ с прилагане на правило от вида $S \to \gamma$.
    Разглеждаме всички варианти за думата $\alpha \in M$ и за правилото $S \to \gamma$ в граматиката $G$
    за да докажем, че $\omega \in M$.
    Удобно е да представим всички случаи в таблица.
    \begin{center}
      \begin{tabular}{| c | c | c |}
        \hline
        От И.П. за $\alpha$ & правило на $G$ & $\omega$ \\ \hline
        $\in M$ & $S \to \texttt{[}S\texttt{]}S$ & $\in M$ \\ \hline
        $\in M$ & $S \to \texttt{]}S\texttt{[}S$ & $\in M$ \\ \hline
        $\in M$ & $S \to \varepsilon$ & $\in M$ \\ \hline
      \end{tabular}
    \end{center}    
    Във всички случаи лесно се установява, че $\omega \in M$.
  \end{itemize}
  Така за всяка дума $\omega \in \L(G)$ следва, че
  \[\omega \in \Sigma^\star \cap M = L.\]
\end{hint}

%%% Local Variables:
%%% mode: latex
%%% TeX-master: "../eai"
%%% End:

\subsection{Балансирани скоби}

Нека $\alpha$ е дума над азбука, която включва буквите $\texttt{[}$ и $\texttt{]}$. 
Ще казваме, че че $\alpha$ е {\bf балансирана}, ако са изпълнени свойствата:
\begin{itemize}
\item 
  $\texttt{left}(\alpha) = \texttt{right}(\alpha)$;
\item
  За всеки префикс $\gamma$ на $\alpha$,
  $\texttt{left}(\gamma) \geq \texttt{right}(\gamma)$.
\end{itemize}

% \begin{prop}
%   Нека $\alpha \in \{\texttt{[},\texttt{]}\}^\star$ е балансирана дума.
%   Тогава:
%   \begin{itemize}
%   \item 
%     $\alpha = \texttt{[}\beta\texttt{]}$, където $\beta$ е балансирана, или
%   \item
%     $\alpha = \beta\gamma$, където $\beta$ и $\gamma$ са балансирани думи, $\beta,\gamma \neq \varepsilon$.
%   \end{itemize}
% \end{prop}
% \begin{hint}
  
% \end{hint}

\begin{framed}
  \begin{problem}
    Докажете, че езикът 
    \[L = \{\alpha \in \{\texttt{[},\texttt{]}\}^\star \mid \alpha\text{ е балансирана дума}\}\]
    е безконтекстен.
  \end{problem}  
\end{framed}
\begin{hint}
  \marginpar{\cite[стр. 135]{kozen}}
  \marginpar{Очевидно е, че езикът не е регулярен}
  Да разгледаме граматиката $G$ с правила
  \[S \to \texttt{[}S\texttt{]}\ |\ SS\ |\ \varepsilon.\]
  Ще докажем, че $L = \L(G)$.
  
  Първо ще докажем включването $\L(G) \subseteq L$.
  Да разгледаме $M \df \{\alpha \in \{\texttt{[},\texttt{]}, S\}^\star \mid \alpha\text{ е балансирана}\}$.
  
  Нека $S \to^\star_G \alpha$. Ще докажем с индукция по дължината на извода на $\alpha$ от $S$,
  че $\alpha \in M$.
  Нека $S \to^{l}_G \beta \to^1_G \alpha$.
  От {\bf И.П.} имаме, че $\beta \in M$, т.е. $\beta$ е балансирана.

  Лесно се съобразява, че за всички случаи за $\beta$ и $\alpha$ имаме следното:
  \begin{center}
    \begin{tabular}{| c | c | c |}
      \hline
      $\text{от И.П. за }\beta$ & $\text{правило на }G$ & $\alpha$ \\ \hline
      $\in M$ & $S \rightarrow \texttt{[}S\texttt{]}$ & $\in M$ \\ \hline
      $\in M$ & $S \rightarrow SS$ & $\in M$ \\ \hline
      $\in M$ & $S \rightarrow \varepsilon$ & $\in M$ \\ \hline
    \end{tabular}
  \end{center}

  За включването $L \subseteq \L(G)$, нека $\alpha \in L$.
  Ще докажем с индукция по дължината на думата, че $\alpha \in \L(G)$.
  Ясно е, че за всички нетривиални случаи, можем да запишем думата $\alpha$ като $\alpha = \texttt{[}\beta\texttt{]}$.
  Проблемът е, че в общия случай не е ясно дали можем да приложим индукционното предположение за $\beta$,
  защото е възможно $\beta \not\in L$. Например, $\alpha = \texttt{[][]}$.
  Тогава $\beta = \texttt{][} \not \in L$.
  Поради тази причина, трябва да сме по-внимателни и да разгледаме два случая.
  \begin{itemize}
  \item 
    \marginpar{т.е. $\beta \neq \varepsilon$ и $\beta \neq \alpha$}
    Нека $\alpha$ има {\em същински} префикс $\beta \in L$.
    Понеже $\alpha \in L$, лесно се съобразява, че $\alpha = \beta\gamma$ и $\gamma \in L$
    Сега можем да приложим {\bf И.П.} за $\beta$ и $\gamma$ и да получим, че 
    $\beta \in \L(G)$ и $\gamma \in \L(G)$, т.е.
    $S \to^\star_G \beta$ и $S \to^\star_G \gamma$.
    Понеже имаме правило $S \to_G SS$, то е ясно, че $\alpha \in \L(G)$.
  \item
    Нека $\alpha$ да няма същински префикс $\gamma \in L$.
    Ясно е, че $\alpha = \texttt{[}\beta\texttt{]}$, за някое $\beta$
    и $\texttt{left}(\beta) = \texttt{right}(\beta)$. 
    Ако $\beta \in L$, то ще можем да приложим {\bf И.П.} за $\beta$ и ще сме готови.
    За всеки префикс $\gamma$ на $\beta$ имаме, че $\texttt{[}\gamma$ е префикс на $\alpha$,
    и понеже $\alpha \in L$, то $\texttt{left}(\texttt{[}\gamma) \geq \texttt{right}(\gamma)$.
    Възможно ли е $\texttt{left}(\gamma) < \texttt{right}(\gamma)$ ?
    Това може да се случи единствено ако $\texttt{left}(\texttt{[}\gamma) = \texttt{right}(\gamma)$.
    Но тогава $\texttt{[}\gamma$ е префикс на $\alpha$, за който $\texttt{[}\gamma \in L$,
    което противоречи на случая, който разглеждаме.
    Това означава, че за произволен префикс $\gamma$ на $\beta$,
    $\texttt{left}(\gamma) \geq \texttt{right}(\gamma)$ и оттук $\beta \in L$ и можем да приложим {\bf И.П.}
    Тогава $S \to^\star_G \beta$ и чрез правилото $S \to_G \texttt{[}S\texttt{]}$
    получаваме, че $\alpha \in \L(G)$.    
  \end{itemize}
\end{hint}

%%% Local Variables:
%%% mode: latex
%%% TeX-master: "../eai"
%%% End:

\subsection{Лесни задачи}

\begin{extra}

\begin{problem}
  Постройте регулярен израз за езика на следната граматика:
  \begin{align*}
    & S \to S + S\ |\ S * S\ |\ A\\
    & A \to KL\ |\ LK\\
    & K \to 0K\ |\ \varepsilon\\
    & L \to 1K\ |\ \varepsilon.
  \end{align*}
\end{problem}

\begin{problem}
  Докажете, че следните езици са безконтекстни.
  \mynote{
    \begin{enumerate}[a)]
    \item
      $S \to aSa\ \vert\ bSb\ \vert\ a\vert\ b\ \vert\ \varepsilon$
    \end{enumerate}}
  \begin{enumerate}[a)]
  \item
    $L = \{\omega \in \{a,b\}^\star \mid \omega = \omega^{\rev}\}$;
  \item
    $L = \{a^nb^{2m}c^{n} \mid m,n \in \Nat\}$;
  \item
    $L = \{a^nb^{m}c^{m}d^n \mid m,n \in \Nat\}$;
  \item
    \mynote{Обединение на два езика}
    $L = \{a^nb^{2k} \mid n,k \in \Nat\ \&\ n \neq k\}$;
  \item
    \mynote{$S \rightarrow aSb | aS | a$}
    $L = \{a^nb^k \mid n > k\}$;
  \item
    $L = \{a^nb^k \mid n \geq 2k\}$;
  \item
    \mynote{$S \rightarrow aSc | aS | aB | bB$,\\$B\rightarrow bBc | bB | \varepsilon$}
    $L = \{a^nb^kc^m \mid n + k \geq m+1\}$;
  \item
    $L = \{a^nb^kc^m \mid n + k \geq m+2\}$;
  \item
    \mynote{$S \rightarrow aSc | aS | B | Bc$,\\$B\rightarrow bBc | bB | \varepsilon$}
    $L = \{a^nb^kc^m \mid n + k + 1 \geq m\}$;
  \item
    $L = \{a^nb^mc^{2k} \mid n \neq 2m\ \&\ k \geq 1\}$;
  \item
    $L = \{a^nb^kc^m \mid n + k \leq m\}$;
  \item
    $L = \{a^nb^kc^m \mid n + k \leq m+1\}$;
  \item
    \mynote{Обединение на три езика}
    $L = \{a^nb^mc^k \mid n, m, k \text{ не са страни на триъгълник}\}$;
  \item
    $L = \{a,b\}^\star \setminus \{a^{2n}b^n \mid n\in\Nat\}$;
  \item
    \mynote{$S\to EaE$, $E \to aEbE\ |\ bEaE\ |\ \varepsilon$}
    $L = \{\omega \in \{a,b\}^\star\mid \card{\omega}{a} = \card{\omega}{b} + 1\}$;
  \item
    $L = \{\omega \in \{a,b\}^\star\mid \card{\omega}{a} \geq \card{\omega}{b}\}$;
  \item
    $L = \{\omega \in \{a,b\}^\star\mid \card{\omega}{a} > \card{\omega}{b}\}$;
  \item
    $L = \{\alpha \in \{a,b\}^\star \mid \text{ във всеки префикс $\beta$ на $\alpha$, } \card{\beta}{b} \leq \card{\beta}{a}\}$;
  \item
    $L = \{\alpha \sharp \beta \mid \alpha,\beta \in \{a,b\}^\star\ \&\ \alpha^{\rev}\mbox{ е поддума на }\beta \}$.
  \item
    $L = \{\omega_1 \sharp \omega_2 \sharp \cdots \sharp \omega_n \mid n\geq 2\ \&\ \omega_1,\omega_2,\dots,\omega_n \in \{a,b\}^\star\ \&\ \abs{\omega_1} = \abs{\omega_2}\}$;
  \item
    $L = \{\omega_1 \sharp \omega_2 \sharp \cdots \sharp \omega_n \mid n\geq 2\ \&\ \omega_1,\dots,\omega_n \in \{a,b\}^\star\ \&\ (\exists i \neq j)[\ \abs{\omega_i} = \abs{\omega_j}\ ]\}$;
  \item
    $L = \{\omega_1 \sharp \omega_2 \sharp \cdots \sharp \omega_n \mid n\geq 2\ \&\ (\forall i\in[1,n])[\ \omega_i \in \{a,b\}^\star\ \&\ \abs{\omega_i} = \abs{\omega_{n+1-i}}\ ]\}$.
  \end{enumerate}
\end{problem}

\begin{problem}
  Проверете дали следните езици са безконтекстни:
  \begin{enumerate}[a)]
  \item
    $\{a^nb^{2n}c^{3n}\ \mid\ n\in\Nat\}$;
  % \item
  %   $\{a^nb^{2n}c^{n}\ \mid\ n\in\Nat\}$;
  \item
    $\{a^nb^kc^ka^n\mid\ k \leq n\}$;
  \item
    $\{a^nb^mc^k\mid n < m < k\}$;
  \item
    $\{a^nb^nc^k\mid n \leq k \leq 2n\}$;
  \item
    $\{a^nb^mc^k\mid k = \min\{n,m\}\}$;
  \item
    $\{a^nb^nc^m\mid m \leq n\}$;
  \item
    $\{a^nb^mc^k\mid k = n\cdot m\}$;
  \item
    $L^\star$, където
    $L = \{\alpha\alpha^{\rev} \mid \alpha \in \{a,b\}^\star\}$;
  \item
    $\{\omega\omega\omega\mid \omega\in \{a,b\}^\star\}$;
  \item
    $\{a^{n^2}b^n\ \mid n \in \Nat\}$;
  \item
    $\{a^p\ \mid\ p\mbox{ е просто }\}$;
  \item
    $\{\omega \in \{a,b\}^\star \mid \omega = \omega^{\rev}\}$;
  \item
    $\{\omega^n \mid \omega \in \{a,b\}^\star\ \&\ \card{\omega}{b} = 2\ \&\ n \in \Nat\}$;
  \item
    $\{\omega c^n \omega^R \mid \omega \in \{a,b\}^\star\ \&\ n = \abs{\omega}\}$;
  \item
    % Дефиниция на подниз
    $\{\alpha c \beta \mid \alpha,\beta \in \{a,b\}^\star\ \&\ \alpha\mbox{ е подниз на }\beta\}$;
  \item
    $\{\omega_1 \sharp \omega_2 \sharp \dots \sharp \omega_k\mid k\geq 2\ \&\ \omega_i\in \{a\}^\star\ \&\ (\exists i,j)[i \neq j\ \&\ \omega_i = \omega_j]\}$;
  \item
    $\{\omega_1 \sharp \omega_2 \sharp \dots \sharp \omega_k\mid k\geq 2\ \&\ \omega_i\in \{a\}^\star\ \&\ (\forall i,j \leq k)[i \neq j \iff \omega_i \neq \omega_j]\}$;
  \item
    $\{a^ib^jc^k\mid i,j,k\geq 0\ \&\ (i = j \vee j = k)\}$;
  \item
    % \marginpar{Разгл. $L' = L \cap L(a^*b^*c^*)$.}
    $\{\omega \in \{a,b,c\}^\star\mid \card{\omega}{a} > \card{\omega}{b} > \card{\omega}{c}\}$;
  \item
    $\{a,b\}^\star \setminus \{a^nb^n\mid n\in \Nat\}$;
  \item
    $\{a^nb^mc^k \mid m^2 = 2nk\}$;

  \item
    $L = \{a^nb^mc^ma^n \mid m,n\in\Nat\ \&\ n = m+42\}$;
  \item
    $L = \{\sharp a \sharp aa \sharp aaa \sharp \cdots \sharp a^{n-1}\sharp a^n\sharp \mid n \geq 1\}$;
  \item
    $\{a^mb^nc^k\mid m = n \vee n = k \vee m = k\}$;
  \item
    $\{a^mb^nc^k\mid m \neq n \vee n \neq k \vee m \neq k\}$;
  % \item
  %   $\{a^mb^nc^k\mid m = n \wedge n = k \wedge m = k\}$;
  \item
    $\{\omega \in \{a,b,c\}^\star\mid \card{\omega}{a} \neq \card{\omega}{b} \vee \card{\omega}{a} \neq \card{\omega}{c} \vee \card{\omega}{b} \neq \card{\omega}{c}\}$.
  \end{enumerate}
\end{problem}

\end{extra}

%%% Local Variables:
%%% mode: latex
%%% TeX-master: "../eai"
%%% End:


\begin{problem}
  Докажете, че езикът $L = \{a^nb^{kn} \mid k,n > 0\}$ не е безконтекстен.
\end{problem}
\ifhints
\begin{hint}
  Да разгледаме ситуацията $\alpha \in L$ и
  $\alpha = xyuvw$, където $y = a^i$ и $v = b^j$
  Интересният случай е когато $0 < i,j < p$.
  \begin{itemize}
  \item 
    Нека $i = j$.
    Разглеждаме думата $xy^{p^2+1}uv^{p^2+1}w$.
    Това означава дали $p+p^2i$ дели $p^2+p^2i$, т.е.
    дали $1 + pi$ дели $p+pi$.
    \begin{align*}
      & p + pi = k(1+pi), \text{ за някое }1 \leq k < p\\
      & p = k + pi(k-1), \text{ за някое }1 \leq k < p\\
    \end{align*}
    Достигаме до противоречие.
  \item
    Нека $i > j$, т.е. $i \geq j+1$.
    Отново разглеждаме думата $xy^{p^2+1}uv^{p^2+1}w$.
    Това означава дали $p+p^2i$ дели $p^2+p^2j$, т.е.
    дали $1 + pi$ дели $p+pj$, но
    $1+pi \geq 1 + p(j+1) > p + pj$.
    Противоречие.
  \item
    Нека $i < j$ и тогава нека $j = mi + r$, където $p > i > r \geq 0$
    Разглеждаме думата $xy^{mp^2+1}uv^{mp^2+1}w$.
    Това означава дали $p+mp^2i$ дели $p^2+mp^2j$, т.е.
    дали $1 + mpi$ дели $p+mpj$.
    \[p+pmj = k(1+pmi), \text{ за някое }1 \leq k.\]
    \begin{itemize}
    \item 
      Възможно ли е $k \geq p$? Тогава:
      \begin{align*}
        & p + pmj = k(1+pmi) \geq p(1+pmi) \geq p(1+pj)\\
        & p(1+mj) \geq p(1+pj)\\
        & m \geq p.
      \end{align*}
      Достигаме до противоречие.
      Следователно, $1 \leq k < p$.
    \item
      Възможно ли е $k \leq m$? Тогава:
      \begin{align*}
        & p + pmj = k(1+pmi) \leq m(1+pmi) \leq m(1+pj)\\
        & p(1+mj) \leq m(1+pj)\\
        & p + pmj \leq m + pmj\\
        & p \leq m.
      \end{align*}
      Достигаме до противоречие, защото $m < p$.
    \item
      Заключаваме, че $1 \leq m < k < p$. Тогава:
      \begin{align*}
        & p + pmj = k(1+pmi)\\
        & p = k + pm(ki-j).
      \end{align*}
      Понеже $m < k$, то $j < (m+1)i \leq ki$,
      т.е. $ki-j > 0$. Достигаме до противоречие.
    \end{itemize}
  \end{itemize}
\end{hint}
\fi

За всеки две думи с равна дължина, дефинираме функцията $\texttt{diff}$ по следния начин:
\begin{align*}
  & \texttt{diff}(\varepsilon,\varepsilon) = 0\\
  & \texttt{diff}(a \cdot \alpha, b\cdot \beta) =
    \begin{cases}
      \texttt{diff}(\alpha,\beta), & \text{ако }a = b\\
      1 + \texttt{diff}(\alpha,\beta), & \text{ако }a \neq b\\
    \end{cases}
\end{align*}

\begin{problem}
  За всеки от следните езици, отговорете дали са безконтекстни, като се обосновете:
  \begin{enumerate}[a)]
  \item 
    \ifhints
    \marginpar{Да}
    \fi
    $\{\alpha \sharp \beta \mid \alpha,\beta \in \{a,b\}^\star\ \&\ |\alpha| = |\beta|\ \&\ \texttt{diff}(\alpha,\beta^{\rev}) = 1\}$;
  \item
    \ifhints
    \marginpar{Да}
    \fi
    $\{\alpha \sharp \beta \mid \alpha, \beta \in \{a,b\}^\star\ \&\ |\alpha| = |\beta|\ \&\ \texttt{diff}(\alpha,\beta^{\rev}) \geq 1\}$;
  \item
    \ifhints 
    \marginpar{Не}
    \fi
    $\{\alpha \sharp \beta \mid \alpha, \beta \in \{a,b\}^\star\ \&\ |\alpha| = |\beta|\ \&\ \texttt{diff}(\alpha,\beta) \geq 1\}$;
  \item
    \ifhints 
    \marginpar{Да}
    \fi
    $\{\alpha\beta \mid \alpha, \beta \in \{a,b\}^\star\ \&\ |\alpha| = |\beta|\ \&\ \texttt{diff}(\alpha,\beta) \geq 1\}$;
  \item
    \ifhints
    \marginpar{Не}
    \fi
    $\{\alpha\beta \mid \alpha, \beta \in \{a,b\}^\star\ \&\ |\alpha| = |\beta|\ \&\ \texttt{diff}(\alpha,\beta) = 1\}$;
  \end{enumerate}
\end{problem}    

\begin{problem}
    Да разгледаме езика
    \[D_1 \df \{\alpha_1 \sharp \cdots \sharp \alpha_n \mid n \geq 2\ \&\ |\alpha_i| = |\alpha_{i+1}|\ \&\ \texttt{diff}(\alpha_i,\alpha^{\rev}_{i+1}) = 1\}.\]
    \begin{itemize}
    \item 
      Докажете, че $D_1$ не е безконтекстен.
    \item
      Докажете, че $D_1$ може да се представи като сечението на два безконтекстни езика.
    \end{itemize}    
\end{problem}
\begin{hint}
  Да разгледаме безконтекстния език 
  \[L_1 = \{\alpha_1 \sharp \alpha_2 \mid |\alpha_1| = |\alpha_{2}|\ \&\ \texttt{diff}(\alpha_1,\alpha^{\rev}_{2}) = 1\}.\]
  Тогава
  \begin{align*}
    D_1 =\ & L_1 \cdot (\sharp L_1)^\star \cdot (\{\varepsilon\} \cup \sharp\{a,b\}^\star)\ \cap \\
           & \{a,b\}^\star \cdot (\sharp L_1)^\star \cdot (\{\varepsilon\} \cup \sharp\{a,b\}^\star).
  \end{align*}
\end{hint}

\begin{problem}
  За произволен език $L$, дефинираме езика
  \[\texttt{Diff}_n(L) = \{\alpha \in L \mid (\exists \beta \in L)[|\alpha| = |\beta|\ \&\ \texttt{diff}(\alpha,\beta) = n]\}.\]
  Вярно ли е, че:
  \begin{itemize}
  \item 
    ако $L$ е регулярен, то $\texttt{Diff}_n(L)$ е регулярен?
  \item
    ако $L$ е безконтекстен, то $\texttt{Diff}_n(L)$ е безконтекстен?
  \end{itemize}
\end{problem}
\begin{hint}
  Мисля, че най-лесно става като се разгледа съответно автомата или стековия автомат.
\end{hint}

\begin{problem}
  \marginpar{\cite[стр. 158]{sipser3}}
  Нека $L_1$ и $L_2$ са езици. Дефинираме
  \[L_1 \triangle L_2 \df \{\alpha\beta \mid \alpha \in L_1\ \&\ \beta \in L_2\ \&\ |\alpha| = |\beta|\}.\]
  Докажете, че
  \begin{enumerate}[a)]
  \item 
    ако $L_1$ и $L_2$ са регулярни езици, то е възможно $L_1 \triangle L_2$ да не е регулярен;
  \item
    ако $L_1$ и $L_2$ са регулярни езици, то $L_1 \triangle L_2$ е безконтекстен;
  \item
    ако $L_1$ и $L_2$ са безконтекстни езици, то е възможно $L_1 \triangle L_2$ да не е безконтекстен.
  \end{enumerate}
\end{problem}

\begin{problem}
  Докажете, че ако $L$ е безконтекстен език, то 
  \[L^{\rev} = \{\omega^{\rev} \mid \omega \in L\}\]
  също е безконтекстен.
\end{problem}


\begin{problem}
  Нека $\Sigma = \{a,b,c,d,f,e\}$.
  Докажете, че езикът $L$ е безконтекстен, където за думите $\omega \in L$ са изпълнени свойствата:
  \begin{itemize}[-]
  \item 
    за всяко $n\in\Nat$, след всяко срещане на $n$ последнователни $a$-та
    следват $n$ последователни $b$-та, и $b$-та не се срещат по друг повод в $\omega$, и
  \item
    за всяко $m\in\Nat$, след всяко срещане на $m$ последнователни $c$-та
    следват $m$ последователни $d$-та, и $d$-та не се срещат по друг повод в $\omega$, и
  \item
    за всяко $k\in\Nat$, след всяко срещане на $k$ последнователни $f$-а
    следват $k$ последователни $e$-та, и $e$-та не се срещат по друг повод в $\omega$.
  \end{itemize}
\end{problem}

\begin{problem}
  Да разгледаме езиците:
  \begin{align*}
    & P = \{\alpha\in\{a,b,c\}^*\,|\, \alpha \text{ е палиндром с четна дължина}\} \\
    & L =  \{\beta b^n\,|\, n\in\mathbb{N}, \beta\in P^n\}.
  \end{align*}
  Да се докаже, че:
  \begin{enumerate}[a)]
  \item 
    $L$ не е регулярен;
  \item 
    $L$ е безконтекстен.
  \end{enumerate}
\end{problem}

\begin{problem}
  Нека $L_1$ е произволен регулярен език над азбуката $\Sigma$, 
  а $L_2$ е езика от всички думи палиндроми над $\Sigma$.
  Докажете, че $L$ е безконтекстен език, където:
  \[L = \{\alpha_1 \cdots\alpha_{3n}\beta_1\cdots\beta_m\gamma_1\cdots\gamma_n \mid \alpha_i,\gamma_j \in L_1, \beta_k\in L_2, m,n \in \Nat\}.\]
\end{problem}

\begin{problem}
  Нека $L = \{\omega\in\{a,b\}^\star \mid \card{\omega}{a} = 2\}$.
  Да се докаже, че езикът $L' = \{\alpha^n \mid \alpha\in L, n \geq 0\}$ не е безконтекстен.
\end{problem}


\begin{problem}
  Нека $\Sigma = \{a,b,c\}$ и $L \subseteq \Sigma^\star$ е безконтестен език. Ако имаме дума 
  $\alpha \in \Sigma^\star$, тогава \emph{L-вариант} на $\alpha$ ще наричаме думата, която се получава като в $\alpha$ всяко едно 
  срещане на символа $a$ заменим с (евентуално различна) дума от $L$.
  Тогава, ако $M \subseteq \Sigma^*$ е произволен безконтестен език, да се докаже че езикът
  \begin{equation*}
    M' = \{\beta\in\Sigma^\star |\ \beta \text{ е $L$-вариант на } \alpha \in M \}
  \end{equation*}
  също е безконтекстен.
\end{problem}

\begin{problem}
  Докажете, че всеки безконтекстен език над азбуката $\Sigma = \{a\}$
  е регулярен.
\end{problem}
\ifhints
\begin{hint}
  
\end{hint}
\fi

\begin{problem}
%  \marginpar{\cite{papadimitriou} стр. 149}
  Да фиксираме азбуката $\Sigma$.
  Нека $L$ е безконтекстен език, а $R$ е регулярен език.
  Докажете, че езикът
  \[L/R = \{\alpha \in \Sigma^\star \mid (\exists \beta \in R)[\alpha \beta \in L]\}\]
  е безконтекстен.
\end{problem}
\ifhints
\begin{hint}
  
\end{hint}
\fi

\begin{problem}
  Нека е дадена граматиката $G = \pair{\{a,b\}, \{S,A,B,C\},S,R}$.
  Използвайте CYK-алгоритъма, за да проверите дали
  думата $\alpha$ принадлежи на $\L(G)$, където правилата на граматиката и думата $\alpha$
  са зададени като:
  \begin{enumerate}[a)]
  \item
    $S \to BA\ |\ CA\ |\ a,\ C\to BS\ |\ SA,\ A\to a,\ B\to b$,\\
    $\alpha=bbaaa$;
  \item
    $S \to AB\ |\ BC,\ A\to BA\ |\ a,\ B\to CC\ |\ b,\ C\to AB\ |\ a$,\\
    $\alpha=baaba$;
  \item
    $S \to AB,\ A\to AC\ |\ a\ |\ b,\ B\to CB\ |\ a,\ C\to a$,\\
    $\alpha=babaa$.
  \end{enumerate}
\end{problem}

\begin{problem}
  Нека $L$ е безконтекстен език над азбуката $\Sigma$.
  Докажете, че следните езици са безконтекстни:
  \begin{enumerate}[a)]
  \item 
    $\texttt{Pref}(L) = \{\alpha \in \Sigma^\star \mid (\exists \beta \in \Sigma^\star)[\alpha\cdot\beta \in L]\}$;
  \item 
    $\texttt{Suff}(L) = \{\beta \in \Sigma^\star \mid (\exists \alpha \in \Sigma^\star)[\alpha\cdot\beta \in L]\}$;
  \item
    $\texttt{Infix}(L) = \{\beta \in \Sigma^\star \mid (\exists \alpha,\gamma \in \Sigma^\star)[\alpha \cdot \beta \cdot \gamma \in L]\}$;
  \end{enumerate}
\end{problem}

\begin{problem}
  Докажете, че езикът
  \[L = \{\ \omega_1 \sharp \omega_2 \sharp \cdots \sharp \omega_{2n} \mid n \geq 1\ \&\ \sum^n_{i=1}\abs{\omega_{2i-1}} = \sum^{n}_{i=1}\abs{\omega_{2i}}\ \}\]
  е безконтекстен.
\end{problem}
\begin{hint}
  Най-лесно става със стеков автомат, като са необходими две допълнителни букви за азбуката на стека.

  Една възможна безконтекстна граматика е следната:
  \begin{align*}
    & E \to XEX\ |\ \sharp O\ |\ O\sharp\ |\ \sharp E\sharp\\
    & O \to OO\ |\ EE\ |\ \varepsilon\\
    & X \to a\ |\ b,
  \end{align*}
  където началната променлива е $E$.
\end{hint}

\begin{prop}
  Да разгледаме езиците:
  \begin{align*}
    & L_{\text{even}} = \{\ \omega_1 \sharp \omega_2 \sharp \cdots \sharp \omega_{2n} \mid n\in\Nat\ \&\ \sum^n_{i=1}\abs{\omega_{2i-1}} = \sum^{n}_{i=1}\abs{\omega_{2i}}\ \}\\
    & L_{\text{odd}} = \{\ \omega_1 \sharp \omega_2 \sharp \cdots \sharp \omega_{2n+1} \mid n\in\Nat\ \&\ \sum^n_{i=0}\abs{\omega_{2i+1}} = \sum^{n}_{i=1}\abs{\omega_{2i}}\ \}.
  \end{align*}
  Тогава:
  \begin{align*}
    (\forall \alpha)[\ \alpha\in L_{\text{even}}\ \implies\ (\exists \alpha_1,\alpha_2)[\ & \alpha \in \{a,b\}^\star\sharp\{a,b\}^\star\ \lor\\
                                                                                        & ( \alpha = \alpha_1\alpha_2\ \&\ \alpha_1 \in L_{\text{even}}\ \&\ \alpha_2 \in L_{\text{odd}})\ \lor\\
                                                                                        &( \alpha = \alpha_1\alpha_2\ \&\ \alpha_1 \in L_{\text{odd}}\ \&\ \alpha_2 \in L_{\text{even}})\ ]\ ]\\
    (\forall \alpha)[\ \alpha\in L_{\text{odd}}\ \implies\ (\exists \alpha_1,\alpha_2)[\ &( \alpha = \alpha_1\alpha_2\ \&\ \alpha_1 \in L_{\text{even}}\ \&\ \alpha_2 \in L_{\text{even}})\ \lor\\
                                                                                        &( \alpha = \alpha_1\alpha_2\ \&\ \alpha_1 \in L_{\text{odd}}\ \&\ \alpha_2 \in L_{\text{odd}})\ ]\ ].
  \end{align*}
\end{prop}
\begin{hint}
  Да разгледаме една дума $\alpha = \omega_1 \sharp \omega_2 \sharp \cdots \sharp \omega_{2n} \in L_{\text{even}}$.
  Не е възможно $|\omega_{2i-1}| < |\omega_{2i}|$ за $i = 1,\dots,n$, защото тогава бихме имали 
  \[\sum^n_{i=1}|\omega_{2i-1}|\ <\ \sum^n_{i=1}|\omega_{2i}|.\]
  Нека $|\omega_{2i-1}| \geq |\omega_{2i}|$ за първото такова $i$.
  Тогава
  \[\alpha = \omega_1 \sharp \omega_2 \sharp \cdots \sharp\omega'_{2i-1}\omega''_{2i-1}\sharp\omega_{2i}\sharp\cdots \omega_{2n},\]
  където $|\omega''_{2i-1}| = |\omega_{2i}|$.
  Като махнем от $\alpha$ подниза $\omega''_{2i-1}\sharp\omega_{2i}$, 
  получаваме думата
  \[\alpha' = \omega_1 \sharp \omega_2 \sharp \cdots \sharp\omega'_{2i-1}\sharp\cdots \omega_{2n} \in L_{\text{odd}}.\]
  Понеже $|\alpha'| < |\alpha|$, от И.П. знаем, че $\alpha' = \alpha_1\alpha_2$, такива че
  $\alpha_1,\alpha_2 \in L_{\text{even}}$ или $\alpha_1,\alpha_2 \in L_{\text{odd}}$.
\end{hint}


% \begin{problem}
%   Нека с $\code{\A}$ да означим думата над азбуката $\{0,1\}$, която кодира крайния автомат $\A$.
%   Посочете кои от следните езици са регулярни/безконтекстни/разрешими/полуразрешими, където:
%   \begin{enumerate}[a)]
%   \item
%     $L = \{\code{\A} \mid \A\text{ е краен автомат}\}$;
%   \item
%     $L = \{\code{\A} \mid \A\text{ е краен автомат с пет състояния}\}$;
%   \item
%     $L = \{\code{\A} \mid \A\text{ е краен детерминиран автомат}\}$;
%   \item
%     $L = \{\code{\A} \mid \A\text{ е краен детерминиран тотален автомат}\}$;
%   \item
%     $L = \{\code{\A}\cdot\omega \mid \omega \in \Sigma^\star\ \&\ \omega \in \L(\A)\}$;
%   \item
%     $L = \{\code{\A}\cdot\omega \mid \omega \in \Sigma^\star\ \&\ \omega \not\in \L(\A)\}$;
%   \item
%     $L = \{\code{\A} \cdot \code{\B} \mid \A, \B \text{ са крайни автомати и } \L(\A) = \L(\B)\}$;
%   \end{enumerate}
%   Обосновете се!
% \end{problem}

\begin{problem}
  Нека $G = \CFG$ е {\em регулярна} граматика, т.е.
  всички правила на $G$ са от вида $A \to bC$ и $A \to \varepsilon$.
  Посочете кои от следните езици са безконтекстни, където:
  \begin{enumerate}[a)]
  \item 
    $L = \{\alpha\sharp\beta^{\rev} \mid \alpha,\beta \in (V \cup \Sigma)^\star\ \&\ \alpha \vdash_G \beta\}$;
  \item 
    $L = \{\alpha\sharp\beta^{\rev} \mid \alpha,\beta \in (V \cup \Sigma)^\star\ \&\ \alpha \vdash^\star_G \beta\}$;
  \item
    $L = \{\alpha\sharp\beta^{\rev} \mid \alpha,\beta \in (V \cup \Sigma)^\star\ \&\ \alpha \not\vdash_G \beta\}$;
  \item
    $L = \{\alpha\sharp\beta^{\rev} \mid \alpha,\beta \in (V \cup \Sigma)^\star\ \&\ \alpha \not\vdash^\star_G \beta\}$.
  \end{enumerate}
  Обосновете се!
\end{problem}

\begin{problem}
  Да разгледаме една {\em безконтекстна} граматика $G = \CFG$.
  Посочете кои от следните езици са безконтекстни, където:
  \begin{enumerate}[a)]
  \item 
    $L = \{\alpha\sharp\beta^{\rev} \mid \alpha,\beta \in (V \cup \Sigma)^\star\ \&\ \alpha \vdash_G \beta\}$;
  \item
    $L = \{\alpha_1\sharp\alpha^{\rev}_2\sharp\alpha_3\sharp\alpha^{rev}_4\sharp \cdots \mid \alpha_i \in (V \cup \Sigma)^\star\ \&\ \alpha_i \vdash_G \alpha_{i+1} \text{ за }i<n\}$;
  \item 
    $L = \{\alpha\sharp\beta^{\rev} \mid \alpha,\beta \in (V \cup \Sigma)^\star\ \&\ \alpha \vdash^\star_G \beta\}$;
  \item
    $L = \{\alpha\sharp\beta^{\rev} \mid \alpha,\beta \in (V \cup \Sigma)^\star\ \&\ \alpha \not\vdash_G \beta\}$;
  \item
    $L = \{\alpha\sharp\beta^{\rev} \mid \alpha,\beta \in (V \cup \Sigma)^\star\ \&\ \alpha \not\vdash^\star_G \beta\}$.
  \end{enumerate}
  Обосновете се!
\end{problem}

\begin{problem}
  Да разгледаме една {\em неограничена} граматика $G = \CFG$.
  Посочете кои от следните езици са безконтекстни, където:
  \begin{enumerate}[a)]
  \item 
    $L = \{\alpha\sharp\beta^{\rev} \mid \alpha,\beta \in (V \cup \Sigma)^\star\ \&\ \alpha \vdash_G \beta\}$;
  \item 
    $L = \{\alpha\sharp\beta^{\rev} \mid \alpha,\beta \in (V \cup \Sigma)^\star\ \&\ \alpha \vdash^\star_G \beta\}$;
  \item
    $L = \{\alpha\sharp\beta^{\rev} \mid \alpha,\beta \in (V \cup \Sigma)^\star\ \&\ \alpha \not\vdash_G \beta\}$;
  \item
    $L = \{\alpha\sharp\beta^{\rev} \mid \alpha,\beta \in (V \cup \Sigma)^\star\ \&\ \alpha \not\vdash^\star_G \beta\}$.
  \end{enumerate}
  Обосновете се!
\end{problem}



%%% Local Variables:
%%% mode: latex
%%% TeX-master: "../eai"
%%% End:


%%% Local Variables:
%%% mode: latex
%%% TeX-master: "../eai"
%%% End:

\chapter{Машини на Тюринг}

\setlength{\epigraphwidth}{0.65\textwidth}\epigraph{Turing’s ‘Machines’. These machines are humans who calculate. (Wittgenstein 1947 [1980]: 1096).}

\newcommand{\tape}[1]{\cdots\blank\blank\blank{#1}\blank\blank\blank\cdots}

\newcommand{\goleft}{\lhd}
\newcommand{\goright}{\rhd}
\newcommand{\stay}{\Box}


% \begin{framed}
%   {\bf Теза на Чърч-Тюринг:} Всеки алгоритъм може да се осъществи като машина на Тюринг.
% \end{framed}

\section{Основни понятия}
\index{Тюринг}
\index{машина на Тюринг!детерминирана}
\marginpar{Тук до голяма степен следваме \cite[Глава 3]{sipser3}}
\marginpar{Понятието за машина на Тюринг има много еквивалентни дефиниции. }
{\em Детерминирана} машина на Тюринг ще наричаме осморка от вида 
\[\M = \TM,\] където:
\begin{itemize}
\item 
  $Q$ - крайно множество от състояния;
\item
  $\Sigma$ - крайна азбука за входа;
\item
  $\Gamma$ - крайна азбука за лентата, $\Sigma \subseteq \Gamma$;
\item
  $\qstart$ - начално състояние, $\qstart \in Q$;
\item
  $\blank$ - празен символ,  $\blank \in \Gamma \setminus \Sigma$;
\item
  \marginpar{Тези две състояния ще наричаме заключителни}
  $\qaccept \in Q$ - приемащо състояние;
\item
  $\qreject \in Q$ - отхвърлящо състояние, където $\qaccept \neq \qreject$;
\item
  % \marginpar{Няма нужда да изискваме главата да остава върху същата клетка от лентата}
  \marginpar{Това означава, че веднъж достигнем ли заключително състояние, не можем да правим повече преходи. Тук следваме \cite[стр. 169]{sipser3} и \cite[стр. 327]{hopcroft2}.}
  $\delta:Q'\times\Gamma \to Q\times \Gamma \times \{\goleft,\goright,\stay\}$ - функция на преходите, където
  $Q' = Q \setminus\{\qaccept, \qreject\}$.
\end{itemize}

Сега ще опишем как $\M$ работи върху вход думата $\alpha \in \Sigma^\star$.
Първоначално безкрайната лента съдържа само думата $\alpha$. Останалите клетки на лентата съдържат символа $\blank$.
Освен това, $\M$ се намира в началното състояние $\qstart$ и главата за четене е върху най-левия символ на $\alpha$.
Работата на $\M$ е описана от функцията на преходите $\delta$.
  
\begin{itemize}
\item 
  \marginpar{На англ. instanteneous description}
  Формално, {\bf моментната конфигурация} (или описание) на едно изчисление на машина на Тюринг
  е тройка от вида 
  \[(\alpha, q, \beta) \in \Gamma^\star\times Q \times \Gamma^\star,\]
  като интерпретацията на тази тройка е, че машината се намира в състояние $q$ и лентата има вида
  \[\tape{\alpha\underline{x}\beta'},\]
  \marginpar{Понякога за удобство ще означаваме моментната конфигурация като $(q,\alpha\underline{x}\beta)$}
  където $\beta = x\beta'$ и четящата глава на машината е поставена върху $x$.
\item
  \marginpar{Тази ситуация можем да опишем като $(q, \alpha\underline{\blank})$}
  Ако $\beta = \varepsilon$, това означава, че главата на машината е върху $\blank$, т.е.
  лентата има вида  \[\tape{\alpha\ \underline{\blank}}\]
\item
  Макар и да имаме безкрайна лента, моментната конфигурация, която може да се представи като {\em крайна} дума,
  описва цялото моментно състояние на машината на Тюринг.
  % Обърнете внимание, че тройката $(\blank\blank ab, q, \blank)$ описва същото моментно състояние както и 
  % тройката $(ab, q, \varepsilon)$.
\item
  {\bf Началната конфигурация} за входната дума $\alpha \in \Sigma^\star$ представлява тройката
  \[(\varepsilon, \qstart, \alpha).\]
  \begin{itemize}
  \item
    Ако $\alpha = x\alpha'$, за $x \in \Sigma$, то  лентата има вида \[\cdots\blank\blank\ \underline{x}\alpha'\blank\blank\cdots,\]
  \item
    Ако $\alpha = \varepsilon$, то началната конфигурация е $(\varepsilon, \qstart, \varepsilon)$ и лентата има вида
    \[\tape{\underline{\blank}}\]
  \end{itemize}
\item
  {\bf Заключителна конфигурация} представлява тройка от вида
  \[(\beta, \qaccept, \gamma), \text{ или }(\beta, \qreject, \gamma).\]
  Ако машината, която работи върху дадена входна дума, достигне до заключително състояние, ще казваме
  че машината {\em спира работа}.
\end{itemize}

Както за автомати, удобно е да дефинираме бинарна релация $\vdash_\M$ над $\Gamma^\star \times Q \times \Gamma^\star$,
която ще казва как моментната конфигурация на машината $\M$ се променя при изпълнение на една стъпка.
\begin{itemize}
\item
  \marginpar{Понеже $z \in \Gamma$, то е възможно и $z = \blank$.}
  Ако $\delta_\M(q,z) = (p,y,\goright)$, то дефинираме $(\alpha, q, z\beta) \vdash_\M (\alpha y, p, \beta)$.
  Това означава, че ако лентата е имала вида 
  \[\tape{\alpha\underline{z}\beta},\]
  след този преход на $\M$ лентата има следния вид:
  \[\tape{\alpha y\underline{x}\beta'}, \text{ ако }\beta = x\beta',\]
  или 
  \[\tape{\alpha y\underline{\blank}}, \text{ ако }\beta = \varepsilon.\]
\item
  Ако $\delta_\M(q,\blank) = (p,y,\goright)$, то дефинираме също така и $(\alpha, q, \varepsilon) \vdash_\M (\alpha y, p, \varepsilon)$.
  Това означава, че ако лентата е имала вида 
  \[\tape{\alpha\underline{\blank}},\]
  след този преход на $\M$ лентата има следния вид:
  \[\tape{\alpha y\underline{\blank}}\]
\item 
  Ако $\delta_\M(q,z) = (p,y,\goleft)$, то дефинираме $(\alpha x, q, z\beta) \vdash_\M (\alpha , p, xy\beta)$.
  Това означава, че ако лентата е имала вида 
  \[\tape{\alpha x\underline{z}\beta},\]
  след този преход на $\M$ лентата има вида 
  \[\tape{\alpha \underline{x}y\beta}\]
  Също така, в тази ситуация трябва да дефинираме и
  $(\varepsilon, q, z\beta) \vdash_\M (\varepsilon , p, \blank y\beta)$.
  Това означава, че ако лентата е имала вида 
  \[\tape{\underline{z}\beta},\]
  след този преход на $\M$ лентата има вида 
  \[\tape{ \underline{\blank}y\beta}\]
\item
  Ако $\delta_\M(q,\blank) = (p,y,\goleft)$, то дефинираме и $(\alpha x, q, \varepsilon) \vdash_\M (\alpha , p, xy)$,
  а също така и $(\varepsilon, q, \varepsilon) \vdash_\M (\varepsilon , p, \blank y)$,
\item
  Ако $\delta_\M(q, z) = (p, y, \stay)$, то дефинираме $(\alpha, q, z\beta) \vdash_\M (\alpha , p, y\beta)$.
  % Това означава, че ако лентата е имала вида 
  % \[\tape{\alpha \underline{z}\beta},\]
  % след този преход на $\M$ лентата има вида 
  % \[\tape{\alpha \underline{y}\beta}\]
\item
  Ако $\delta_\M(q, \blank) = (p, y, \stay)$, то дефинираме и $(\alpha, q, \varepsilon) \vdash_\M (\alpha , p, y)$.
\end{itemize}
С $\vdash^\star_\M$ ще означаваме рефлексивното и транзитивно затваряне на $\vdash_\M$.

\begin{itemize}
\item
  Машината на Тюринг $\M$ {\bf приема} думата $\alpha$, 
  ако 
  \[(\varepsilon, \qstart, \alpha) \vdash^\star_\M (\gamma_1, \qaccept, \gamma_2),\]
  за някои $\gamma_1, \gamma_2 \in \Gamma^\star$.
\item
  Машината на Тюринг $\M$ {\bf отхвърля} думата $\alpha$, 
  ако 
  \[(\varepsilon, \qstart, \alpha) \vdash^\star_\M (\gamma_1, \qreject, \gamma_2),\]
  за някои $\gamma_1, \gamma_2 \in \Gamma^\star$.
\item
  Машината на Тюринг $\M$ {\bf не приема} думата $\alpha$, 
  ако $\M$ отхвърля $\alpha$ или $\M$ никога не завършва при начална конфигурация $(\varepsilon,\qstart,\alpha)$.
\item
  \index{машина на Тюринг!разрешител}
  \marginpar{На англ. такава машина на Тюринг се нарича {\bf decider} \cite[стр. 170]{sipser3}. Може такива машини на Тюринг да се наричат и тотални \cite[стр. 213]{kozen}.}
  Една машина на Тюринг се нарича {\bf разрешител}, ако при всеки вход достига до заключително състояние,
  т.е. достига до $\qaccept$ или $\qreject$.
\item 
  Езикът, който се {\bf разпознава} от машината $\M$ е:
  \[\L(\M) = \{\alpha\in\Sigma^\star \mid (\varepsilon, \qstart, \alpha) \vdash^\star_\M (\beta, \qaccept, \gamma), \text{ за някои }\beta,\gamma\in\Gamma^\star\}.\]
\item
  \index{език!полуразрешим}
  \marginpar{На англ. {\bf semidecidable language}}
  Езикът $L$ се нарича {\bf полуразрешим}, ако съществува машина на Тюринг $\M$, за която
  $L = \L(\M)$.
  В този случай се казва, че $\M$ разпознава езика $L$.
  \marginpar{В литературата се използва и названието {\bf рекурсивно номеруем език}.}
  Ако една дума $\alpha \in L$, то след крайно много стъпки ще достигнем до състоянието $\qaccept$.
  Ако $\alpha \not\in L$, то не е ясно дали какво се случва с изчислението на $\M$ върху $\alpha$. Възможно е да достигнем до състоянието $\qreject$, но може да попаднем в безкрайно изчисление.
\item
  \index{език!разрешим}
  \marginpar{На англ. {\bf decidable language}}
  Един език $L$ се нарича {\bf разрешим}, ако за него съществува {\em разрешител} $\M$, за която
  $L = \L(\M)$.
  \marginpar{В литературата се използва и названието {\bf рекурсивен език}.}
  В този случай се казва, че $\M$ разрешава езика $L$.
\end{itemize}

\begin{framed}
  \begin{proposition}
    Ако $L$ е разрешим език над азбуката $\Sigma$, то $\Sigma^\star \setminus L$ също е разрешим език.
  \end{proposition}
\end{framed}

От дефинициите е ясно, че всеки разрешим език е полуразрешим.
По-късно, ще видим, че съществуват полуразрешими езици, чиито допълнения не са полуразрешими,
т.е. не всеки полуразрешим език е разрешим.
Една от основните ни задачи ще бъде да класифицираме различни езици като (не)раз\-ре\-ши\-ми и (не)полуразрешими.
За да придобием по-добра интуиция за тези нови понятия, ще разгледаме подробно няколко примера.
Ще видим също как можем да изобразяваме функцията на преходите на $\M$ графично.

\section{Примери за разрешими езици}

\begin{example}
  \marginpar{Знаем, че $L$ не е безконтекстен}
  Да разгледаме езика $L = \{a^nb^nc^n \mid n\in\Nat\}$.
 
  Нека да въведем нов символ $d$, с който ще маркираме обработените символи $a$, $b$, $c$.
  Идеята на алгоритъма, който ще разгледаме е да маркира на всяка итерация по едно $a$, $b$, и $c$.
  Той завършва успешно ако всички символи на думата са маркирани.
  Нека първоначално думата е копирана върху лентата и четящата глава е върху първия символ на думата.
  \begin{enumerate}[(1)]
  \item 
    Чете $d$-та надясно по лентата докато срещне първото $a$ и го замества с $d$. Отива на стъпка (2).
    Ако символите свършат (т.е. достигне се $\blank$) преди да се достигне $a$,
    то алгоритъмът завършва успешно.
  \item
    Чете $d$-та надясно по лентата докато срещне първото $b$ и го замества с $d$.
    Отива на стъпка (3).
  \item
    Чете $d$-та надясно по лентата докато срещне първото $c$ и го замества с $d$.
  \item
    Връща четящата глава в началото на лентата, т.е. чете наляво докато не срещне символа $\blank$.
    Връща се в стъпка (1). 
  \end{enumerate}

  Нека сега да видим, че този алгоритъм може да се опише съвсем формално с машина на Тюринг.
  Ще построим машина на Тюринг $\M$, за която $L = \L(\M)$, където
  \begin{itemize}
  \item 
    $\Sigma = \{a,b,c\}$;
  \item
    $\Gamma = \{a,b,c,d,\blank\}$, за някой нов символ $d$;
  \item
    $Q = \{1,2,\dots,5\}$;
  \item
    $q_{accept} = 5$;
  \item
    частичната функция на преходите $\delta:Q\times\Gamma \to Q\times\Gamma\times\{L,R\}$
    е описана на схемата отдолу.
  \end{itemize}

  \begin{figure}[H]
    \begin{center}
      \begin{tikzpicture}[->,>=stealth,thick,node distance=50pt]
        \tikzstyle{every state}=[circle,minimum size=10pt,auto]
        
        \node[state,initial]    (1) {$1$};
        \node[state]            (2) [right of=1]{$2$};
        \node[state]            (3) [right of=2]{$3$};
        \node[state]            (4) [right of=3]{$4$};
        \node[state,accepting]  (5) [below of=1]{$5$};
        % \node[state,accepting]  (6) [right of=5]{$6$};
        
        \begin{scope}[every node/.style={scale=.8}]
          \path
          (1) edge [loop above] node [above] {$d;R$} (1)
          (1) edge [bend right=15] node [left] {$\blank;R$} (5)
          % (1) edge [bend left=15] node [left] {$\{b,c\}$} (6)
          (1) edge [bend left=15] node [above] {$a/d;R$} (2)
          (2) edge [bend left=15] node [above] {$b/d;R$} (3)
          (2) edge [loop below] node [right] {$\{a,d\};R$} (2)
          (3) edge [bend left=15] node [above] {$c/d;L$} (4)
          (3) edge [loop below] node [right] {$\{b,d\};R$} (3)
          (4) edge [loop right] node [below right] {$\{a,b,d\};L$} (4)
          (4) edge [in=65,out=115,above] node [above] {$\blank;R$} (1);
        \end{scope}
      \end{tikzpicture}
    \end{center}
  \end{figure}


  % Да проследим изчислението на думата $aabbcc$:
  
  % \[_1aabbcc \vdash d_2abbcc \vdash da_2bbcc \vdash dad_3bcc \vdash dadb_3cc \vdash dad_4bdc \vdash da_4dbdc \vdash \cdots \vdash\]
  % \[_4dadbdc \vdash\ _4\blank dadbdc \vdash\ _1dadbdc \vdash d_1adbdc \vdash dd_2dbdc \vdash ddd_2bdc \vdash dddd_3dc \vdash \]
  % \[ ddddd_3c \vdash dddddd_4 \vdash \cdots \vdash\ _4\blank dddddd \vdash\ _1dddddd \vdash \cdots \vdash dddddd_1\blank \vdash dddddd_5\blank.\]

  Съобразете, че тази машина на Тюринг може да се направи тотална като се добави ново състояние $q_{reject}$
  и за всяка двойка $(q,x)$, за която функцията на преходите не е дефинирана, да сочи към $q_{reject}$.
  Така можем да получим {\em тотална} машина на Тюринг за езика $L$, което означава, че 
  $L$ е не само полуразрешим, но {\em разрешим} език.
\end{example}

\begin{example}
  \marginpar{Да напомним, че този език не е безконтекстен}
  \marginpar{В \cite[стр. 155]{hopcroft1} е дадено по-различно решение. Тук следваме \cite[стр. 173]{sipser3}. Там има малка грешка}
  Да разгледаме езика $L = \{\omega \sharp \omega \mid \omega\in\{a,b\}^\star\}$.
  Нека първо да видим, че можем неформално да опишем алгоритъм, който да разпознава думите на езика $L$.
  Нека една дума е копирана върху лентата и четящата глава е поставена върху първия символ от думата.
  \begin{enumerate}[(1)]
  \item 
    Чете $x$-ове надясно по лентата докато не срещне $a$ или $b$ и го замества с $x$.
    Запомня дали сме срещнали $a$ или $b$.
    Ако вместо $a$ или $b$ срещне $\sharp$, то отива на стъпка $(6)$.
  \item
    Чете $a$-та и $b$-та надясно по лентата докато не стигне $\sharp$. 
  \item
    Чете $c$-то надясно по лентата и всички следващи $x$-ове докато не срещне символа $a$ или $b$.
    Той трябва да е същия символ, който сме запаметили на стъпка $(1)$.
    Заместваме този символ с $x$.
  \item
    Чете $x$-ове наляво по лентата докато не стигне $\sharp$.
  \item
    Чете $a$-та и $b$-та по лентата докато не стигне $x$.
    Поставя четящата глава върху символа точно след първия $x$.
    Отива на стъпка $(1)$.
  \item
    Прочита $\sharp$ надясно по лентата и чете надясно $x$-ове докато не срещне $\blank$.
    Алгоритъмът завършва успешно.
  \end{enumerate}

  Ще построим машина на Тюринг $\M$, за която $L = \L(\M)$.
  \begin{itemize}
  \item 
    $\Sigma = \{a,b,\sharp\}$;
  \item
    $\Gamma = \{a,b,\sharp,x,\blank\}$;
  \item
    $Q = \{1,2,\dots,9\}$;
  \item
    $q_{accept} = 9$;
  \end{itemize}

  \begin{figure}[H]
    \begin{center}
      \begin{tikzpicture}[->,>=stealth,thick,node distance=50pt]
        \tikzstyle{every state}=[circle,minimum size=10pt,auto,scale=.9]
        
        \node[state,initial]    (1) {$1$};
        \node[state]            (2) [above right of=1]{$2$};
        \node[state]            (3) [below right of=1]{$3$};
        \node[state]            (4) [right of=2]{$4$};
        \node[state]            (5) [right of=3]{$5$};
        \node[state]            (6) [below right of=4]{$6$};
        \node[state]            (7) [above of=6]{$7$};
        \node[state]            (8) [left of=3]{$8$};
        \node[state,accepting]  (9) [below left of=3]{$9$};
        
        \begin{scope}[every node/.style={scale=.8}]
          \path
          (1) edge [bend left=15] node [below right] {$a/x;R$} (2)
              edge [bend right=15] node [above right] {$b/x;R$} (3)
              edge [bend right=15] node [left] {$\sharp;R$} (8)
          (2) edge [loop above] node [above] {$\{a,b\};R$} (2)
              edge [bend left=15] node [above] {$\sharp;R$} (4)
          (3) edge [loop below] node [below] {$\{a,b\};R$} (3)
              edge [bend right=15] node [below] {$\sharp;R$} (5)
          (4) edge [loop above] node [above] {$x;R$} (4)
              edge [bend left=15] node [below left] {$a/x;L$} (6)
          (5) edge [loop below] node [below] {$x;R$} (5)
              edge [bend right=15] node [above left] {$b/x;L$} (6)
          (6) edge [loop right] node [right] {$x;L$} (6)
              edge [bend right=15] node [right] {$\sharp;L$} (7)
          (7) edge [loop right] node [right] {$\{a,b\};L$} (7)
              edge [out=130,in=120,above,distance=2.5cm] node [above] {$x;R$} (1)
          (8) edge [loop left] node [left] {$x;R$} (8)
              edge [bend right=15] node [left] {$\blank;R$} (9);
        \end{scope}
      \end{tikzpicture}
    \end{center}
  \end{figure}

  % Да проследим изчислението на думата $ab\sharp ab$.
  
  % \begin{align*}
  %   (\varepsilon, 1, ab\sharp ab) & \to (x, 2, b\sharp ab) \to xb_2\sharp ab \to xb\sharp _4ab \to xb_6\sharp xb \to x_7b\sharp xb \to _7xb\sharp xb \to x_1b\sharp xb\\
  %   & \to xx_3\sharp xb \to xx\sharp _5xb \to xx\sharp x_5b \to xx\sharp _6xx \to xx_6\sharp xx \to x_7x\sharp xx \to xx_1\sharp xx \\
  %   & \to xx\sharp _8xx \to xx\sharp x_8x \to xx\sharp xx_8\blank \to xx\sharp xx\blank_9\blank
  % \end{align*}

  Може лесно да се съобрази, че тази машина на Тюринг може да се допълни до {\em тотална}.
  
\end{example}


%%% Local Variables:
%%% mode: latex
%%% TeX-master: "../eai"
%%% End:


\section{Изчислими функции}

\marginpar{Възможно е да се дефинира и за двулентова машина на Тюринг, като $f(\alpha)$ ще бъде върху втората лента.}
Една {\em тотална} функция $f:\Sigma^\star \to \Sigma^\star$ се нарича изчислима с машина на Тюринг $\M$, ако 
за всяка дума $\alpha \in \Sigma^\star$,
\[(\varepsilon, \qstart, \alpha) \vdash^\star_\M (\varepsilon, \qaccept, f(\alpha)\blank^k), \text{ за някое }k \in \Nat.\]
Това означава, че машината на Тюринг $\M$ винаги завършва. Лесно се съобразява, че езикът
\[\texttt{Graph}(f) = \{\ \alpha\sharp f(\alpha) \mid \alpha \in \Sigma^\star\ \}\]
е разрешим.

\begin{problem}
  Докажете, че съществуват функции от вида $f:\Sigma^\star\to\Sigma^\star$, които не са изчислими с машина на Тюринг.
\end{problem}
\begin{hint}
  Всяка машина на Тюринг може да се кодира с естествено число.
  Това означава, че съществуват изброимо безкрайно много машини на Тюринг.
  От друга страна, съществуват неизброимо много функции от вида $f:\Sigma^\star \to \Sigma^\star$.
\end{hint}

% Нека е дадена функцията $f:\Nat^k \to \Nat$.
% Ще казваме, че $f$ е изчислима с машината на Тюринг $\M$,
% ако за всяко $n_1,\dots,n_k$ е изпълнено:
% \begin{itemize}
% \item 
%   Представяме всяко от числата $n_1,\dots,n_k$ в монадична бройна система
%   като лентата на $\M$ има вида:  
%   \[\dots \blank \blank \underbrace{1111\dots 11}_{n} \blank\blank\dots,\]
%   като изискваме главата на $\M$ да е позиционирана върху най-лявата единица.
%   Такава конфигурация ще наричаме {\bf стандартна начална конфигурация}.
% \item
%   Ако $f(n_1,\dots,n_k) = m$, то $\M$ завършва с резултат върху лентата
%   \[\dots \blank \blank \underbrace{1111\dots 11}_{m} \blank\blank\dots,\]
%   като главата на $\M$ е върху най-лявата 1-ца.
%   Такава конфигурация се нарича {\bf стандартна финална конфигурация}.
% \item
%   Ако $f(n_1,\dots,n_k)$ е недефенирана, то $\M$ няма да завърши в стандартна конфигурация, т.е.
%   или $\M$ ще работи безкрайно време, или ще завърши в конфигурация, която не е стандартна.
% \end{itemize}

\begin{example}
  \marginpar{При двулентова машина на Тюринг тази задача е много по-лесна и сложността ще бъде $\mathcal{O}(n)$ вместо $\mathcal{O}(n^2)$.}
  Да разгледаме функцията $f:\{1\}^\star \to \{1\}^\star$
  дефинирана като
  \[f(1^n) = 1^{2n}.\]
  Да видим защо $f$ е изчислима с машина на Тюринг.
  \begin{framed}
    \begin{figure}[H]
      \begin{center}
        \begin{tikzpicture}[->,>=stealth,thick,node distance=50pt]
          \tikzstyle{every state}=[circle,minimum size=10pt,auto,scale=.7]
          
          \node[state,initial below]    (1) {$q_0$};
          \node[state]            (2) [right of=1]{$q_1$};
          \node[state]            (3) [right of=2]{$q_2$};
          \node[state]            (4) [right of=3]{$q_3$};
          \node[state]            (5) [right of=4]{$q_4$};
          \node[state]            (6) [right of=5]{$q_5$};
          \node[state]            (7) [right of=6]{$q_6$};
          \node[state]            (8) [right of=7]{$q_7$};
          \node[state]            (9) [right of=8]{$q_8$};
          \node[state]            (10) [right of=9]{$q_9$};
          \node[state]            (11) [below of=8]{$q_{10}$};
          \node[state,accepting]  (12) [below of=11]{$q_{11}$};
          
          \begin{scope}[every node/.style={scale=.8}]
            \path
            (1) edge [bend left=15] node [above] {$1;\goleft$} (2)
            (1) edge [bend right=30] node [above] {$\blank;\stay$} (12)
            (2) edge [bend left=15] node [above] {$1;\goleft$} (3)
            (2) edge [bend right=15] node [below] {$\blank;\goleft$} (3)
            (3) edge [bend left=15] node [above] {$\blank/1;\goleft$} (4)
            (4) edge [bend left=15] node [above] {$\blank/1;\goright$} (5)
            (5) edge [loop below] node [below] {$1;\goright$} (5)
            (5) edge [bend left=15] node [above] {$\blank;\goright$} (6)
            (6) edge [loop below] node [below] {$1;\goright$} (6)
            (6) edge [bend left=15] node [above] {$\blank;\goleft$} (7)
            (7) edge [bend left=15] node [above] {$1/\blank;\goleft$} (8)
            (8) edge [bend left=15] node [above] {$1;\goleft$} (9)
            (9) edge [loop below] node [below] {$1;\goleft$} (9)
            (9) edge [bend right=15] node [below] {$\blank;\goleft$} (10)
            (10) edge [loop below] node [below] {$1;\goleft$} (10)
            (10) edge [out=140,in=60, above] node [below] {$\blank;\goright$} (2)
            (8) edge [] node [right] {$\blank;\goleft$} (11)
            (11) edge [loop left] node [left] {$1;\goleft$} (11)
            (11) edge [] node [right] {$\blank;\goright$} (12);
          \end{scope}
        \end{tikzpicture}
        \caption{Машина на Тюринг за $f(1^n) = 1^{2n}$.}
      \end{center}
    \end{figure}
  \end{framed}

\begin{align*}
  (q_0, \underline{1}1) & \vdash (q_1, \underline{\blank}11) \vdash  (q_2, \underline{\blank} \blank 11) \vdash  (q_3, \underline{\blank} 1 \blank 11)\\
                        & \vdash (q_4, 1\underline{1}\blank 11) \vdash (q_4, 11 \underline{\blank} 11) \vdash (q_5, 11\blank \underline{1}1)\\
                        & \vdash \cdots \vdash (q_7, 11 \blank \underline{1}\blank) \vdash \cdots
\end{align*}

\end{example}

\begin{example}
  \marginpar{Това пак става много по-лесно с двулентова машина на Тюринг и сложността пак ще бъде $\mathcal{O}(n)$ вместо $\mathcal{O}(n^2)$.}
  Да видим защо тоталната функция $f:\{a,b\}^\star \to \{a,b\}^\star$, дефинирана като
  $f(\alpha) = \alpha\cdot\alpha$ е изчислима с машина на Тюринг.
  
  \begin{itemize}
  \item
    $\Sigma = \{a,b\}$;
  \item 
    $\Gamma = \{a,b,x,A,B\}$;
  \item
    $\qstart = q_0$;
  \item
    $\qaccept = q_6$
  \end{itemize}
  \begin{framed}
  \begin{figure}[H]
    \begin{center}
      \begin{tikzpicture}[->,>=stealth,thick,node distance=70pt]
        \tikzstyle{every state}=[circle,minimum size=10pt,auto,scale=.9]
        
        \node[state]            (1) {$q_0$};
        \node[state]            (2) [above of=1]{$q_1$};
        \node[state]            (3) [right of=2]{$q_2$};
        \node[state]            (4) [below of=1]{$q_3$};
        \node[state]            (5) [right of=4]{$q_4$};
        \node[state]            (6) [right of=1]{$q_5$};
        \node[state,accepting]  (7) [right of=6]{$q_6$};
        
        \begin{scope}[every node/.style={scale=.8}]
          \path
          (1) edge [bend left=15] node [left] {$a/x;\goright$} (2)
          (2) edge [loop above] node [above] {$\{a,b,A,B\};\goright$} (2)
          (2) edge [bend left=15] node [above] {$\blank/A;\goleft$} (3)
          (3) edge [loop right] node [right] {$\{a,b,A,B\};\goleft$} (3)
          (3) edge [bend right=15] node [right] {$x/a;\goright$} (1)
          (1) edge [bend right=15] node [left] {$b/x;\goright$} (4)
          (4) edge [loop below] node [below] {$\{a,b,A,B\};\goright$} (4)
          (4) edge [bend right=15] node [below] {$\blank/B;\goleft$} (5)
          (5) edge [loop right] node [right] {$\{a,b,A,B\};\goleft$} (5)
          (5) edge [bend left=15] node [right] {$x/b;\goright$} (1)
          (1) edge [loop left] node [left] {$A/a,B/b;\goright$} (1)
          (1) edge [bend left=15] node [above] {$\blank;\goleft$} (6)
          (6) edge [loop below] node [right] {$\{a,b\};\goleft$} (6)
          (6) edge [bend left=15] node [above] {$\blank;\goright$} (7);
        \end{scope}
      \end{tikzpicture}
      \caption{Машина на Тюринг за $f(\alpha) = \alpha\cdot \alpha$.}
    \end{center}
  \end{figure}
\end{framed}

Да проследим работата на $\M$ върху думата $ab$.
Първо копираме добавяме $AB$ и така лентата съдържа $abAB$. 
След това заменяме $A$ с $a$ и $B$ с $b$. Така най-накравя полуваме върху лентата думата $abab$.

\begin{align*}
  (q_0, \underline{a}b) & \vdash_\M (q_1, x\underline{b}) \vdash_\M (q_1,xb\underline{\blank}) \vdash_\M (q_2, x\underline{b}A) \vdash_\M (q_2, \underline{x}bA)\\
                        & \vdash_\M (q_0, a\underline{b}A) \vdash_\M (q_3, ax\underline{A}) \vdash_\M (q_3, axA\underline{\blank}) \vdash_\M (q_4, ax\underline{A}B)\\
                        & \vdash_\M (q_4, a\underline{x}AB) \vdash_\M (q_0, ab\underline{A}B) \vdash_\M (q_0,aba\underline{B}) \vdash_\M (q_0, abab\underline{\blank})\\
                        & \vdash_\M (q_5, aba\underline{b}) \vdash_\M (q_5, ab\underline{a}b) \vdash_\M (q_5, a\underline{b}ab) \vdash_\M (q_5, \underline{a}bab)\\
                        & \vdash_\M (q_5, \underline{\blank}abab) \vdash_\M (q_6, \underline{a}bab).
\end{align*}

Не можем директно да започнем да копираме $\alpha$, защото така няма да знаем къде е края на първото копие на $\alpha$.
Това можем да направим като първо запишем на лентата $\alpha \sharp \alpha$ и след това второто копие на $\alpha$ го изместим с една позиция наляво.
\end{example}

\begin{example}
  \marginpar{Изискваме $f(\alpha)$ да започва с $1$ за да може $f$ да бъде функция, т.е. $f(\alpha)$ е най-късият двоичен запис на числото $\ov{\alpha}_{(2)}+1$.}
  Да разгледаме тоталната функция 
  \[f:\{0,1\}^\star \to 1\cdot\{0,1\}^\star,\]
  дефинирана като
  \[\ov{f(\alpha)}_{(2)} = \ov{\alpha}_{(2)} + 1.\]
  Нека да видим, че тази функция е изчислима с машина на Тюринг.

  \begin{itemize}
  \item 
    $\Sigma = \{0,1\}$;
  \item
    $\Gamma = \{0,1,\blank\}$;
  \item
    $\qstart = q_0$;
  \item
    $\qaccept = q_4$.
  \end{itemize}

  \begin{framed}
  \begin{figure}[H]
    \begin{center}
      \begin{tikzpicture}[->,>=stealth,thick,node distance=70pt]
        \tikzstyle{every state}=[circle,minimum size=10pt,auto,scale=.9]
        
        \node[state,initial below]    (0) {$q_0$};
        \node[state]            (1) [right of=0]{$q_1$};
        \node[state]            (2) [right of=1]{$q_2$};
        \node[state]            (3) [right of=2]{$q_3$};
        \node[state,accepting]  (4) [right of=3]{$q_4$};
        
        \begin{scope}[every node/.style={scale=.8}]
          \path
          (0) edge [loop above] node [above] {$0/\blank;\goright$} (0)
          (0) edge [bend left=15] node [above] {$1;\goright$} (1)
          (0) edge [bend right=30] node [below] {$\blank;\stay$} (2)
          (1) edge [loop above] node [above] {$\{0,1\};\goright$} (1)
          (1) edge [bend left=15] node [above] {$\blank;\goleft$} (2)
          (2) edge [loop above] node [above] {$1/0;\goleft$} (2)
          (2) edge [bend left=15] node [above] {$0/1;\goleft$} (3)
          (2) edge [bend right=30] node [below] {$\blank/1;\stay$} (4)
          (3) edge [loop above] node [above] {$\{0,1\};\goleft$} (3)
          (3) edge [bend left=15] node [above] {$\blank;\goright$} (4);
        \end{scope}
      \end{tikzpicture}
    \end{center}
    \caption{Машина на Тюринг изчисляваща $f$, за която $\ov{f(\alpha)}_{(2)}= \ov{\alpha}_{(2)} + 1$.}
  \end{figure}
\end{framed}

Да проследим изчислението на $\M$ върху вход $01011$.

\begin{align*}
  (q_0, 0\underline{1}011) & \vdash_\M (q_0,\underline{1}011) \vdash_\M (q_1, 1\underline{0}11) \vdash_\M (q_1, 10\underline{1}1)\\
                           & \vdash_\M (q_1, 101\underline{1}) \vdash_\M (q_1, 1011\underline{\blank}) \vdash_\M (q_2, 101\underline{1})\\
                           & \vdash_\M (q_2, 10\underline{1}0) \vdash_\M (q_2, 1\underline{0}00) \vdash_\M (q_3, \underline{1}100)\\
                           & \vdash_\M (q_3, \underline{\blank}1100) \vdash_\M (q_4, \underline{1}100).
\end{align*}
\end{example}


\begin{problem}
  Да разгледаме азбуката $\Sigma = \{0,1,\dots,k-1\}$, където $k > 2$.
  Да разгледаме тоталната функция 
  \[f:\Sigma^\star \to (\Sigma\setminus\{0\})\cdot\Sigma^\star,\]
  дефинирана като
  \[\ov{f(\alpha)}_{(k)} = \ov{\alpha}_{(k)} + 1.\]
  Дефинирайте машина на Тюринг $\M$, която изчислява функцията $f$.
\end{problem}


%%% Local Variables:
%%% mode: latex
%%% TeX-master: "../eai"
%%% End:


\subsection*{Многолентови машини на Тюринг}
\index{машина на Тюринг!многолентова}
%Това е просто като имаш shift.
%Използват се при недет. машини

Машина на Тюринг с $k$ ленти има същата дефиниция като еднолентова машина на Тюринг
с единствената разлика, че
\[\delta: Q \times \Gamma^k\to Q \times \Gamma^k \times \{\goleft,\goright,\stay\}^k.\]
Тук добавяме и възможността главата върху някои от лентите да стои на място.

\begin{itemize}
\item
  $\Sigma \df \{a,b\}$;
\item
  $\Gamma \df \{a,b,\blank\}$
\item
  $\delta:Q\times\Gamma^2 \to Q\times\Gamma^2\times \{\goleft, \goright, \stay\}^2$;
\end{itemize}

\begin{framed}
  \begin{figure}[H]
    \begin{center}
      \begin{tikzpicture}[->,>=stealth,thick,node distance=70pt]
        \tikzstyle{every state}=[circle,minimum size=10pt,auto,scale=.9]
        
        \node[state,initial]    (1) {$q_1$};
        \node[state]            (2) [right of=1]{$q_2$};
        \node[state]            (3) [right of=2]{$q_3$};
        \node[state,accepting]  (4) [right of=3]{$q_4$};
        % \node[state]            (5) [below of=3]{$q_5$};
        
        \begin{scope}% [every node/.style={scale=.8}]
          \path
          (1) edge [loop above] node [above] {$\frac{a}{\blank} / \frac{a}{a};\frac{\goright}{\goright}$} (1)
          (1) edge [loop below] node [below] {$\frac{b}{\blank} / \frac{b}{b};\frac{\goright}{\goright}$} (1)
          (1) edge [bend left=15] node [above] {$\frac{\sharp}{\blank};\frac{\goright}{\stay}$} (2)
          (2) edge [loop above] node [above] {$\frac{a}{\blank};\frac{\goright}{\stay}$} (2)
          (2) edge [loop below] node [below] {$\frac{b}{\blank};\frac{\goright}{\stay}$} (2)
          (2) edge [bend left=15] node [above] {$\frac{\blank}{\blank};\frac{\goleft}{\goleft}$} (3)
          (3) edge [loop above] node [above] {$\frac{a}{a};\frac{\goleft}{\goleft}$} (3)
          (3) edge [loop below] node [below] {$\frac{b}{b};\frac{\goleft}{\goleft}$} (3)
          (3) edge [bend left=15] node [above] {$\frac{\sharp}{\blank};\frac{\stay}{\stay}$} (4);
        \end{scope}
      \end{tikzpicture}
    \end{center}
    \caption{двулентова детерминистична частична машина на Тюринг $\M$, за която $\L(\M) = \{\omega\sharp\omega \mid \omega \in \{a,b\}^\star\}$}
  \end{figure}
\end{framed}

В началото втората лента е празна. Имаме две глави, които се движат независимо една от друга.
\begin{align*}
  (q_1, \frac{\hat{a}}{\hat{\blank}}\frac{b}{\blank}\frac{\sharp}{\blank}\frac{a}{\blank}\frac{b}{\blank}) & \vdash (q_1, \frac{a}{a}\frac{\hat{b}}{\hat{\blank}}\frac{\sharp}{\blank}\frac{a}{\blank}\frac{b}{\blank}) \vdash (q_1, \frac{a}{a}\frac{b}{b}\frac{\hat{\sharp}}{\hat{\blank}}\frac{a}{\blank}\frac{b}{\blank}) \vdash (q_2, \frac{a}{a}\frac{b}{b}\frac{\sharp}{\hat{\blank}}\frac{\hat{a}}{\blank}\frac{b}{\blank})\\
                                                                                                           & \vdash (q_2, \frac{a}{a}\frac{b}{b}\frac{\sharp}{\hat{\blank}}\frac{\hat{a}}{\blank}\frac{b}{\blank}) \vdash (q_2, \frac{a}{a}\frac{b}{b}\frac{\sharp}{\hat{\blank}}\frac{a}{\blank}\frac{\hat{b}}{\blank}) \vdash (q_2, \frac{a}{a}\frac{b}{b}\frac{\sharp}{\hat{\blank}}\frac{a}{\blank}\frac{b}{\blank}\frac{\hat{\blank}}{\blank})\\
                                                                                                           & \vdash (q_3, \frac{a}{a}\frac{b}{\hat{b}}\frac{\sharp}{\blank}\frac{a}{\blank}\frac{\hat{b}}{\blank}) \vdash (q_3, \frac{a}{\hat{a}}\frac{b}{b}\frac{\sharp}{\blank}\frac{\hat{a}}{\blank}\frac{b}{\blank}) \vdash (q_3, \frac{\blank}{\hat{\blank}}\frac{a}{a}\frac{b}{b}\frac{\hat{\sharp}}{\blank}\frac{a}{\blank}\frac{b}{\blank})\\
                                                                                                           & \vdash (q_4, \frac{\blank}{\hat{\blank}}\frac{a}{a}\frac{b}{b}\frac{\hat{\sharp}}{\blank}\frac{a}{\blank}\frac{b}{\blank}).
\end{align*}

\begin{prop}
  За всяка $k$-лентова машина на Тюринг $\M$ съществува еднолентова машина на Тюринг $\M'$,
  такава че $\L(\M) = \L(\M')$.
\end{prop}
\begin{proof}
  \marginpar{В \cite[стр. 177]{sipser3} конструкцията е малко по-различна. Там съдържанието на всяка лента се поставя последователно върху една лента, като се разделят със специален символ. Тук следваме \cite[стр. 162]{hopcroft1}}
  Нека $\M$ е $k$-лентова машина на Тюринг.
  Ще построим еднолентова машина на Тюринг $\M'$, за която $\L(\M) = \L(\M')$.
  Да означим $\hat\Gamma = \{\hat X \mid X \in \Gamma\}$.
  Тогава азбуката на лентата на $\M'$ ще бъде $\Gamma' = (\hat\Gamma \cup \Gamma)^{k}$.
  Сега вместо да имаме $k$ ленти ще имаме една лента, която представлява $k$-орка.
  За да симулираме $\M$, използваме символите $\hat X$ за да маркират позицията на главите на $\M$,
  като във всяка компонента на лентата има точно по един символ от вида $\hat X$.
  % С $\$$ ще отблезяваме границите на всяка лента, в която можем да търсим маркера.
  За да определим следващия ход на машината $\M'$, трябва да сканираме лентата докато не 
  открием разположението на всичките $k$ на брой маркирани клетки. Тогава симулираме ход на $\M$
  и отново трябва да променим маркираните клетки.
\end{proof}

{\bf Да се обясни, че има квадратично забавяне при преход от многолентова машина към еднолентова.}


%%% Local Variables:
%%% mode: latex
%%% TeX-master: "../eai"
%%% End:

\newpage
\section{Недетерминистични машини на Тюринг}
\index{машина на Тюринг!недетерминистична}

Една машина на Тюринг $\N$ се нарича недетерминистична, ако функцията на преходите има вида
\[\Delta: Q'\times \Gamma \to \Ps(Q \times \Gamma\times \{\goleft,\goright,\stay\}), \]
където да напомним, че $Q' = Q \setminus \{\qaccept,\qreject\}$.

Отново можем да дефинираме бинарна релация $\vdash$ над $\Gamma^\star \times Q \times \Gamma^+$,
която ще казва как моментното описание на машината $\N$ се променя при изпълнение на една стъпка.

\begin{figure}[H]
  \begin{subfigure}[b]{0.5\textwidth}
    \begin{prooftree}
      \AxiomC{$\Delta(q,x) \ni (p,y,\goright)$}
      \RightLabel{\scriptsize{(right-1)}}
      \UnaryInfC{$(\alpha, q, xz\beta) \vdash (\alpha y, p, z\beta)$}
    \end{prooftree}
    \vspace*{2mm}
  \end{subfigure}
  ~
\begin{subfigure}[b]{0.5\textwidth}
\begin{prooftree}
  \AxiomC{$\Delta(q,x) \ni (p,y,\goleft)$}
  \RightLabel{\scriptsize{(left-1)}}
  \UnaryInfC{$(\alpha z, q, x\beta) \vdash (\alpha,p,zy\beta)$}
\end{prooftree}
\vspace*{2mm}
\end{subfigure}

\begin{subfigure}[b]{0.5\textwidth}
\begin{prooftree}
  \AxiomC{$\Delta(q,x) \ni (p,y,\goright)$}
  \RightLabel{\scriptsize{(right-2)}}
  \UnaryInfC{$(\alpha, q, x) \vdash (\alpha y, p, \blank)$}
\end{prooftree}
\vspace*{2mm}
\end{subfigure}
~
\begin{subfigure}[b]{0.5\textwidth}
  \begin{prooftree}
    \AxiomC{$\Delta(q,x) \ni (p,y,\goleft)$}
    \RightLabel{\scriptsize{(left-2)}}
    \UnaryInfC{$(\varepsilon,q,x\beta) \vdash (\varepsilon,p,\blank y\beta)$}
  \end{prooftree}
  \vspace*{2mm}
\end{subfigure}

\begin{prooftree}
  \AxiomC{$\Delta(q, z) \ni (p, y, \stay)$}
  \RightLabel{\scriptsize{(stay)}}
  \UnaryInfC{$(\alpha, q, z\beta) \vdash (\alpha , p, y\beta)$}
\end{prooftree}
\end{figure}

С $\vdash^\star$ ще означаваме рефлексивното и транзитивно затваряне на $\vdash$.
Тогава за недетерминистична машина на Тюринг $\N$, 
\[\L(\N) = \{\ \omega\in\Sigma^\star \mid (\exists \gamma_1,\gamma_2 \in \Gamma^\star)[(\varepsilon, \qstart, \omega\blank) \vdash^\star_\N (\gamma_1, \qaccept, \gamma_2) ]\ \}.\]

\begin{remark}
  Върху дадена дума $\omega$, недетерминистичната машина на Тюринг $\N$ може да има много различни изчисления.
  Думата $\omega$ принадлежи на $\L(\N)$ ако съществува {\em поне едно} изчисление, което завършва в състоянието $\qaccept$.
  Възможно е много други изчисления при вход $\omega$ да завършват в $\qreject$ или никога да не завършват.
\end{remark}

Аналогично, дефинираме една недетерминистична машина на Тюринг $\N$ да бъде {\bf разрешител}, ако за всяка дума $\omega$ и 
\emph{всяко изчисление} на $\N$ върху $\omega$ завършва в $\qaccept$ или $\qreject$.

% \begin{problem}
%   \mynote{\cite{hopcroft2}}
%   Нека
%   \[\N = (\{q_0,q_1,q_2,q_f\}, \{0,1\}, \{0,1,\blank\}, \Delta, q_0, \{q_f\}),\]
%   \begin{itemize}
%   \item 
%     $\Delta(q_0,0) = \{(q_0,1,\goright),(q_1,1,\goright)\}$;
%   \item
%     $\Delta(q_1,1) = \{(q_2,0,\goleft)\}$;
%   \item
%     $\Delta(q_2,1) = \{(q_0,1,\goright)\}$;
%   \item
%     $\Delta(q_1,\blank) = \{(q_f,\blank,\goright)\}$.
%   \end{itemize}
%   \mynote{$\{0^{n+1}1^k \mid n,k\in\Nat\}$}
%   Опишете $\L(\N)$.
% \end{problem}

\begin{extra}
\begin{example}
  \mynote{Не е обяснено защо е разрешим.}
  Нека да видим, че $L = \{\alpha\sharp\beta \mid \alpha,\beta \in \{a,b\}^\star\ \&\ \alpha\text{ е подниз на }\beta\}$
  е разрешим език като построим недетерминистична машина на Тюринг $\N$,
  която разрешава този език.
  \begin{framed}
    \begin{figure}[H]
      \begin{center}
        \begin{tikzpicture}[->,>=stealth,thick,node distance=65pt]
          \tikzstyle{every state}=[circle,minimum size=10pt,scale=.9]
          
          \node[state,initial below]    (1) {$q_0$};
          \node[state]            (2) [right of=1]{$q_1$};
          \node[state]            (3) [right of=2,node distance=80pt]{$q_2$};
          \node[state]            (4) [below of=3]{$q_3$};
          \node[state]            (5) [below right of=4,node distance=80pt]{$q_4$};
          \node[state]            (6) [right of=4]{$q_5$};
          \node[state]            (7) [above of=6]{$q_6$};
          \node[state]            (8) [right of=6,node distance=90pt]{$q_7$};
          \node[state]            (9) [right of=7,node distance=90pt]{$q_8$};
          \node[state,accepting]  (10)[below right of=5]{$q_{9}$};
          
          \begin{scope}[every node/.style={scale=.8}]
            \path
            (1) edge [loop above] node [above] {$\{a,b\};\goright$} (1)
            (1) edge [bend left=15] node [above] {$\sharp;\goright$} (2)
            (2) edge [loop above] node [above] {$a/\blank,b/\blank;\goright$} (2)
            (2) edge [bend left=15] node [above] {$\{a,b,\blank\};\goleft$} (3)
            (3) edge [loop above] node [above] {$\blank;\goleft$} (3)
            (3) edge [bend right=15] node [left] {$\sharp;\goleft$} (4)
            (4) edge [loop left] node [left] {$\{a,b\};\goleft$} (4)
            (4) edge [bend right=30] node [left] {$\blank;\goright$} (5)
            (5) edge [bend right=15] node [right] {$a/\blank;\goright$} (6)
            (6) edge [loop right] node [right] {$\{a,b\};\goright$} (6)
            (6) edge [bend right=15] node [right] {$\sharp;\goright$} (7)
            (7) edge [loop right] node [right] {$\blank;\goright$} (7)
            (7) edge [bend left=15] node [below] {$a/\blank;\goleft$} (3)
            (8) edge [loop right] node [right] {$\{a,b\};\goright$} (8)
            (5) edge [bend right=30] node [right] {$b/\blank;\goright$} (8)
            (8) edge [bend right=15] node [right] {$\sharp;\goright$} (9)
            (9) edge [loop right] node [above] {$\blank;\goright$} (9)
            (9) edge [bend right=45] node [above] {$b/\blank;\goleft$} (3)
            (5) edge [bend right=15] node [left] {$\sharp;\stay$} (10);
          \end{scope}
        \end{tikzpicture}
      \end{center}
    \end{figure}
  \end{framed}
  Да видим, че $\M$ успешно разпознава думата $ab\sharp aabb$, която принадлежи на $L$.
  \begin{align*}
    (q_0, \underline{a}b\sharp aabb\blank) & \vdash (q_0, a\underline{b}\sharp aabb\blank) \vdash (q_0, ab\underline{\sharp} aabb\blank) \vdash (q_1, ab\sharp\underline{a}abb\blank) \\
                                           & \vdash (q_1, ab\sharp\blank\underline{a}bb\blank) \vdash (q_2, ab\sharp\underline{\blank}abb\blank) \vdash (q_2, ab\underline{\sharp}\blank abb\blank)\\
                                           & \vdash (q_3, a\underline{b}\sharp\blank abb\blank) \vdash (q_3, \underline{a}b\sharp\blank abb\blank) \vdash (q_3, \underline{\blank}ab\sharp\blank abb\blank)\\
                                           & \vdash (q_4, \underline{a}b\sharp\blank abb\blank) \vdash (q_5, \blank\underline{b}\sharp \blank abb\blank) \vdash (q_5, \blank b\underline{\sharp} \blank abb\blank)\\
                                           & \vdash (q_6, \blank b \sharp \underline{\blank} abb\blank) \vdash (q_6, \blank b \sharp \blank \underline{a}bb\blank) \vdash (q_2, \blank b \sharp \underline{\blank}\blank bb\blank)\\
                                           & \vdash (q_2, \blank b \underline{\sharp} \blank\blank bb\blank) \vdash (q_3, \blank \underline{b} \sharp \blank\blank bb\blank) \vdash (q_3, \underline{\blank} b \sharp \blank\blank bb\blank)\\
                                           & \vdash (q_4,  \blank \underline{b} \sharp \blank\blank bb\blank) \vdash (q_7, \blank \blank \underline{\sharp} \blank\blank bb\blank) \\
                                           & \vdash (q_8, \blank \blank \sharp \underline{\blank}\blank bb\blank) \vdash (q_8, \blank \blank \sharp \blank \underline{\blank} bb\blank) \\
                                           & \vdash (q_8, \blank \blank \sharp \blank \blank \underline{b}b\blank) \vdash (q_2, \blank \blank \sharp \blank \underline{\blank} \blank b\blank)\\
                                           & \vdash \cdots \vdash (q_4, \blank\blank\underline{\sharp}\blank\blank\blank b\blank) \vdash (q_9, \blank\blank\underline{\sharp}\blank\blank\blank b\blank)
  \end{align*}
\end{example}
\end{extra}

\subsection*{Канонична наредба на $\Sigma^\star$}
\index{наредба!канонична}

\mynote{За доказателството, че всяка НМТ е еквивалентна на ДМТ, е необходимо да фиксираме канонична подредба на думите над дадена азбука}
Нека $\Sigma = \{a_0,a_1,\dots,a_{k-1}\}$.
Подреждаме думите по ред на тяхната дължина.
Думите с еднаква дължина подреждаме по техния числов ред, т.е.
гледаме на буквите $a_i$ като числото $i$ в $k$-ична бройна система.
Тогава думите с дължина $n$ са числата от $0$ до $k^n-1$ записани в $k$-ична бройна система.
Ще означаваме с $\omega_i$ $i$-тата дума в $\Sigma^\star$ при тази подредба.

Ако $\Sigma = \{0,1\}$, то наредбата започва така:
\[\varepsilon, 0, 1, \underbrace{00, 01, 10, 11}_{\text{от $0$ до $3$}}, \underbrace{000, 001, 010, 011, 100, 101, 110, 111}_{\text{от $0$ до $7$}}, 0000, 0001, \dots\]
В този случай, $\omega_0 = \varepsilon$, $\omega_7 = 000$, $\omega_{13} = 110$.
Обърнете внимание, че тази наредба отговаря на обхождане в широчина на едно пълно наредено двоично дърво.
Можем да дефинираме и релацията $<_{\text{can}}$ по следния начин:
\[\alpha <_{\text{can}} \beta \dff |\alpha| < |\beta| \lor ( |\alpha| = |\beta|\ \&\ \alpha <_{\text{lex}} \beta). \]

\begin{extra}
\begin{problem}
  \label{prob:canonical:function}
  Нека $\Sigma = \{a_0,\dots,a_{k-1}\}$.
  Да разгледаме функцията $f:\Sigma^\star \to \Sigma^\star$, за която 
  $f(\alpha)$ е думата веднага след $\alpha$ в каноничната подредба на $\Sigma^\star$.
  Докажете, че $f$ е изчислима с еднолетнова детерминистична машина на Тюринг.
\end{problem}
\begin{hint}
  Ако $\Sigma = \{0,1\}$, то машината на Тюринг има следния вид:
  \begin{framed}
    \begin{figure}[H]
      \begin{center}
        \begin{tikzpicture}[->,>=stealth,thick,node distance=70pt]
          \tikzstyle{every state}=[circle,minimum size=10pt,auto,scale=.9]
          
          \node[state,initial]    (1) [right of=0]{$q_1$};
          \node[state]            (2) [right of=1]{$q_2$};
          \node[state]            (3) [right of=2]{$q_3$};
          \node[state,accepting]  (4) [right of=3]{$q_4$};
          
          \begin{scope}[every node/.style={scale=.8}]
            \path
            (1) edge [loop above] node [above] {$\{0,1\};\goright$} (1)
            (1) edge [bend left=15] node [above] {$\blank;\goleft$} (2)
            (2) edge [loop above] node [above] {$1/0;\goleft$} (2)
            (2) edge [bend left=15] node [above] {$0/1;\goleft$} (3)
            (2) edge [bend right=30] node [below] {$\blank/0;\stay$} (4)
            (3) edge [loop above] node [above] {$\{0,1\};\goleft$} (3)
            (3) edge [bend left=15] node [above] {$\blank;\goright$} (4);
          \end{scope}
        \end{tikzpicture}
        \caption{Генериране на следващата дума в каноничната наредба.}
      \end{center}
    \end{figure}
  \end{framed}
\end{hint}
\end{extra}

\begin{important}
  \begin{theorem}
    Ако $L$ се разпознава от {\em недетерминистична} машина на Тюринг $\N$, то $L$
    е разпознава и от {\em детерминистична} машина на Тюринг $\D$.
  \end{theorem}
\end{important}
\begin{proof}
  \mynote{В \cite[стр. 164]{hopcroft1} не е добре обяснено.}
  Нека имаме недетерминистичната машина на Тюринг $\N$, за която $L = \L(\N)$.
  Една дума $\alpha$ принадлежи на $\L(\N)$ точно тогава, когато съществува изчисление,
  което започва с думата $\alpha$ върху лентата и след краен брой стъпки, следвайки функцията на преходите $\Delta_\N$,
  достига до състоянието $\qaccept$.
  Сложността идва от факта, че за думата $\alpha$ може да имаме много различни изчисления, 
  като само някои от тях завършват в $\qaccept$. Ще построим детерминистична машина на Тюринг $\D$,
  която последователно ще симулира всички възможни {\em крайни} изчисления за думата $\alpha$, докато 
  намери такова, което завършва в състоянието $\qaccept$.
  \mynote{На практика това, което правим е да представим всички възможни изчисления на $\N$ като $r$-разклонено дърво и да го обходим в широчина, докато не достигнем до $\qaccept$}
  
  Лесно се съобразява, че всяко изчисление на $\N$ може да се представи като 
  крайна редица от елементи на $Q \times \Gamma \times \{\goleft,\goright,\stay\}$.
  Понеже това множество е крайно, то можем на всяка такава тройка да
  съпоставим естествено число $ < r$, където 
  \[r = |Q| \cdot |\Gamma| \cdot 3.\]
  Например, нека $Q = \{q_0,q_1\}$, $\Gamma = \{a,b\}$. Тогава можем да направим следната съпоставка:
  \begin{align*}
    & (q_0,a,\stay) \to 0,\ (q_0,a,\goleft) \to 1,\ (q_0,a,\goright) \to 2,\\
    & (q_0,b,\stay) \to 3,\ (q_0,b,\goleft) \to 4, \ (q_0,b,\goright) \to 5,\\
    & (q_1,a,\stay) \to 6, (q_1,a,\goleft) \to 7, (q_1,a,\goright) \to 8,\\
    & (q_1,b,\stay) \to 9, (q_1,b,\goleft) \to 10,\ (q_1,b,\goright) \to 11.
  \end{align*}
  Ясно е, че всяко изчисление на $\N$ може да се представи като дума над азбуката $\Sigma = \{x_0,x_1,\dots,x_{r-1}\}$.
  Например, изчислението от три стъпки
  \[(\blank,q_0,aba) \vdash_N (b,q_1,ba) \vdash_\N (b,q_1,aa) \vdash_\N (ba,q_0,a)\]
  може да се опише като думата $x_{11}x_6x_2$ над азбуката $\Sigma = \{x_0,x_1,\dots,x_{11}\}$.
  
  Детерминистичната машина на Тюринг $\D$ има три ленти.
  \begin{itemize}
  \item 
    На първата лента съхраняваме входящия низ и {\em тя никога не се променя}.
  \item
    На втората лента ще записваме последователно думи следвайки каноничната наредба на 
    думите над азбуката $\{x_0,x_1,\dots,x_{r-1}\}$.
    От \Problem{canonical:function} знаем как последователно да генерираме тези думи върху една лента.
  \item
    На третата лента симулираме изчислението на $\N$ върху думата от първата лента, използвайки изчислението, 
    което е описано на втората лента. Например, ако съдържанието на втората лента е $x_{11}x_6x_2$,
    това означава, че симулираме изчисление от три стъпки като на първата стъпка избираме дванайсетата
    възможна тройка, на втората стъпка избираме седмата възможна тройка, на третата стъпка избираме третата възможна тройка.
    
    Ако симулацията завърши в състоянието $\qaccept$ на $\N$, то машината $\D$ завършва успешно.
    В противен случай, на втората лента генерираме чрез функцията от \Problem{canonical:function} следващия низ относно каноничната наредба на $\{x_0,x_1\dots,x_{r-1}\}$;
    изтриваме третата лента, копираме първата лента на третата и започваме нова детерминистична симулация като думата върху втората лента ни ръководи какъв преход да правим на всяка стъпка.
  \end{itemize}
\end{proof}


\begin{corollary}
  Ако $L$ се разпознава от {\em недетерминистичен} разрешител $\N$, то $L$
  също се разпознава от {\em детерминистичен} разрешител $\D$.
\end{corollary}
\begin{proof}
  Да разгледаме дървото $T$ с крайно разклонение $r$, което представя всички изчисления на разрешителя $\N$ при вход думата $\omega$.
  От \Lemma{konig} следва, че $T$ е крайно дърво, да кажем с височина $h$, защото ако допуснем, че $T$ е безкрайно, то ще има безкрайно дълго изчисление на $\N$,
  което е невъзможно, понеже $\N$ винаги достига до заключително състояние ($\qaccept$ или $\qreject$).
  \begin{itemize}
  \item 
    Ако $\N$ приема дадена дума $\omega$, то детерминистичната ни симулация на $\N$ ще достигне до изчисление, кодирано като път в $T$, 
    което завършва в състояние $\qaccept$.
  \item
    Ако $\N$ не приема дадена дума $\omega$, то детерминистичната ни симулация на $\N$ ще покаже, че всяко изчисление, кодирано като път в $T$, завършва в състояние $\qreject$.
    Един начин да направим това е да имаме една допълнителна лента, която използваме за брояч колко от възможните изчисления на $\N$ са завършили.
    Спираме, когато този брояч достигне $r^h$, където $h$ е дължината на думата на втората лента, т.е. дълбочината на дървото на изчисленията на $\N$.
  \end{itemize}
\end{proof}


%%% Local Variables:
%%% mode: latex
%%% TeX-master: "../eai"
%%% End:


\section{Основни свойства}

\begin{proposition}
  Ако езикът $L$ е разрешим, то $\overline{L}$ също е разрешим език.
\end{proposition}
\mynote{Означаваме $\overline{L} = \Sigma^\star \setminus L$.
  С други думи, твърдението ни казва, че разрешимите езици са затворени относно операцията допълнение.
  След малко в \Proposition{diagonal:accept} ще видим, че това твърдение не е изпълнено за полуразрешими езици.}
\begin{hint}
  Нека $L = \L(\M)$, където $\M$ е разрешител.
  Нека $\M'$ е същата като $\M$, само със сменени $\qaccept$ и $\qreject$ състояния.
  Тогава $\M'$ също е разрешител и $\ov{L} = \L(\M')$.
\end{hint}

\begin{proposition}
  Ако езиците $L_1$ и $L_2$ са разрешими , то $L_1 \cup L_2$ е разрешим език.
\end{proposition}
\mynote{С други думи, разрешимите езици са затворени относно операцията обединение.
  Като следствие получаваме, че всяко \emph{крайно} обединение на разрешими езици е разрешим език.
  
  \writedown Съобразете, че това твърдение е изпъленено и за полуразрешими езици.}
\begin{hint}
  Нека $L_1 = \L(\M_1)$ и $L_2 = \L(\M_2)$.
  Строим нова машина на Тюринг $\M$, която при вход думата $\alpha$
  симулира едновременно изчисленията на $\M_1$ и $\M_2$ върху $\alpha$.
  Това можем да направим като приемем, че $\M$ има две ленти - една за лентата на $\M_1$ и една за лентата на $\M_2$,
  като състоянията на $\M$ ще бъдат елементи на $Q_1 \times Q_2$.
  Ако една от двете машини достигне своето приемащо състояние, то $\M$ приема думата $\alpha$.
  Ако и двете машини достигнат своите отхвърлящи състояния, то $\M$ отхвърля думата $\alpha$.
\end{hint}

% \begin{proposition}
%   Ако $L_1$ и $L_2$ са полуразрешими езици, то $L_1 \cup L_2$ е полуразрешим език.
% \end{proposition}

\begin{proposition}
  Ако $L_1$ и $L_2$ са разрешими езици, то $L_1 \cap L_2$ е разрешим език.
  \mynote{С други думи, разрешимите езици са затворени относно операцията сечение.
    Като следствие получаваме, че всяко \emph{крайно} сечение на разрешими езици е разрешим език.

    \writedown Съобразете, че това твърдение е изпъленено и за полуразрешими езици.}
\end{proposition}
\begin{hint}
  Нека $L_1 = \L(\M_1)$ и $L_2 = \L(\M_2)$.
  Строим нова машина на Тюринг $\M$, която при вход думата $\alpha$
  симулира едновременно изчисленията на $\M_1$ и $\M_2$ върху $\alpha$.
  Ако и двете машини достигнат до приемащите си състояния, то $\M$ приема думата $\alpha$.
  Ако поне една от двете машини достигне до отхвърлящо състояние, то $\M$ отхвърля думата $\alpha$.
\end{hint}

% \begin{corollary}
%   Всяко крайно сечение на разрешими езици е разрешим език.
% \end{corollary}

% \begin{important}
%   \begin{theorem}
%     Разрешимите езици са затворени относно операциите обединение, сечение, допълнение.
%   \end{theorem}
% \end{important}


\index{Клини-Пост}
\begin{important}
  \begin{theorem}[Клини-Пост]
    \label{th:turing:kleene-post}
    Езиците $L$ и $\ov{L}$ са полуразрешими точно тогава, когато $L$ е разрешим език.
  \end{theorem}
\end{important}
\mynote{\todo Дефинирайте сами новата машина на Тюринг $\M$.}
\begin{hint}
  Посоката $(\Leftarrow)$ е ясна.
  За посоката $(\Rightarrow)$, нека $L = \L(\M_1)$ и $\ov{L} = \L(\M_2)$.
  Строим разрешител $\M$, която при вход думата $\alpha$ симулира едновременно изчисленията на $\M_1$ и $\M_2$ върху $\alpha$.
  Например, може $\M$ да има две ленти за симулацията на $\M_1$ и $\M_2$.
  Знаем със сигурност, че точно едно от двете симулирани изчисления ще завърши в приемащо състояние.
  Ако това е $\M_1$, то $\M$ приема $\alpha$.
  Ако това е $\M_2$, то $\M$ отхвърля $\alpha$.
\end{hint}


%%% Local Variables:
%%% mode: latex
%%% TeX-master: "../eai"
%%% End:


% \section{Машини на Тюринг като генератори}

% \marginpar{\cite[стр. 168]{hopcroft1}}
% \marginpar{\cite[стр. 180]{sipser3}}
% \marginpar{На англ. се наричат {\em enumerators}}

% Нека да разгледаме един вариант на многолентовите машини на Тюринг, които ще наричаме {\bf генератори}.
% Нека машината на Тюринг да има две ленти, като в началото и двете ленти са празни.
% \begin{itemize}
% \item 
%   Първата лента ще служи за работна лента - върху нея можем да пишем и четем;
% \item
%   Втората лента служи единствено за изход - върху нея можем само да пишем пишем думи; не можем да четем какво вече сме написали върху нея и не можем да пишем върху вече записана клетка. Думите са разделени със специален символ - $\#$.
%   Това означава, че втората лента има вида
%   \[\omega_1\#\omega_2\#\cdots\#\omega_n\#\blank\blank\cdots\]
% \item
%   Езикът, които се извежда от такъв генератор е съставен от думите, които са изписани на изходната лента.
%   Такива езици ще наричаме {\bf изчислимо изброими}.
%   Обърнете внимание, че измежду думите на изходната лента е възможно да има повторения.
%   Ако езикът е безкраен, то машината ще работи безкрайно много време.
% \end{itemize}

% \begin{framed}
%   \begin{thm}
%     Един език $L$ е полуразрешим точно тогава, когато $L$ е изчислимо номеруем.
%   \end{thm}
% \end{framed}
% \begin{proof}
%   $(\Leftarrow)$ Нека $L$ да се номерира от генераторът $E$.
%   Машината на Тюринг $\M$, за която $L = \L(\M)$ ще работи по следния начин:
%   \begin{enumerate}[1)]
%   \item 
%     При вход думата $\omega$, $\M$ започва да симулира $E$;
%   \item
%     Когато се появи дума $\gamma$ върху изходната лента на $E$, сравняваме $\omega$ с $\gamma$;
%   \item
%     Ако $\omega = \gamma$, то отиваме в състоянието $q_{accept}$ на $\M$ и завършваме;
%   \item
%     В противен случай, отиваме обратно на стъпка $2)$.
%   \end{enumerate}

%   $(\Rightarrow)$ Нека сега $L = \L(\M)$. Целта ни е да изведем всички думи на $L$ върху изходната лента.
%   Основният проблем е, че за дадена дума $\omega$, не знаем за колко стъпки трябва да симулираме $\M$ за да сме сигурни дали думата $\omega \in \L(\M)$ или не. Оказва се, че можем да разрешим този проблем като позволяме да извеждаме повторящи се думи.
%   За целта, да подредим всички думи $\omega_1, \omega_2, \dots $ над азбуката $\Sigma$ спрямо каноничната наредба.
%   \begin{enumerate}[1)]
%   \item
%     Нека $s = 1$;
%   \item 
%     Симулираме $\M$ върху думите $\omega_1,\dots,\omega_s$ за $s$ стъпки;
%   \item
%     За всяка от тези думи $\omega_i$, които се приемат от $\M$, записваме ги върху изходната лента.
%   \item
%     Нека $s = s+1$; Отиваме обратно на стъпка $2)$.
%   \end{enumerate}
% \end{proof}

% \begin{remark}
%   В последната конструкция позволяваме думите на един полуразрешим език $L$ да се 
%   извеждат върху изходната лента многократно. Можем лесно да осигурим условието всяка дума на $L$
%   да се извежда точно по веднъж.
%   На стъпка $s = \pair{i,j}$, то проверяваме дали думата $\omega_i$ се приема успешно от $\M$
%   за {\em точно} $j$ на брой стъпки. Само тогава думата се записва на изходната лента.
  
%   Обърнете внимание, че не можем да осигурим условието думите да се извеждат във възходящ ред
%   относно каноничната наредба.
% \end{remark}

% \begin{framed}
%   \begin{thm}
%     Един език $L$ е разрешим точно тогава, когато съществува генератор за $L$, 
%     който изписва думите на $L$ във възходящ ред относно каноничната наредба.
%   \end{thm}
% \end{framed}
% \begin{proof}
%   $(\Rightarrow)$ Нека $L = \L(\M)$. Тази посока е лесна, защото $\M$ е тотална машина,
%   т.е. за всеки вход $\M$ завършва или в $q_{accept}$ или в $q_{reject}$.
%   \begin{enumerate}[1)]
%   \item 
%     Нека $s = 1$;
%   \item
%     Симулираме $\M$ върху думата $\omega_s$.
%   \item
%     Ако симулацията завърши в състояние $q_{accept}$, то записваме $\omega_s$
%     върху изходната лента. 
%   \item
%     Иначе ако симулацията завърши в състояние $q_{reject}$, то нищо не записваме върху изходната лента. 
%   \item
%     Нека $s = s+1$. Отиваме на стъпка $2)$.
%   \end{enumerate}

%   \marginpar{Ако имам генератор $G$ за $L$ няма алгоритъм, който да ми каже дали $L$ е безкраен език или не. Това означава, че по код на $G$ няма как ефективно да получа код на $\M$}
%   $(\Leftarrow)$ Ако $L$ е краен, то е ясно, че мога да разпозная езика с краен автомат, което е частен случай на тотална машина на Тюринг.
%   По-интересният случай е когато $L$ е безкраен език.
%   Нека $L$ се генерира от машината на Тюринг $G$ като извежда думите на $L$ във възходящ ред.
%   \begin{itemize}
%   \item 
%     Вход дума $\omega$;
%   \item
%     Симулираме $G$ като гледаме думите, които се извеждат на изходната лента.
%     Ако срещнем думата $\omega$, то завършваме в състояние $q_{accept}$.
%   \item
%     Ако срещнем думата $\gamma$, която е по-голяма от $\omega$ относно каноничната наредба, 
%     то завършваме в състояние $q_{reject}$.
%   \end{itemize}
% \end{proof}

\subsection{Кодиране на машина на Тюринг}

Нека тук да приемем, че:
\begin{itemize}
\item
  $Q = \{q_1,q_2,\dots,q_n\}$;
\item
  $\Gamma = \{X_1,X_2,\dots,X_s\}$; 
\item
  $D_1 = \stay$, $D_2 = \goleft$, $D_3 = \goright$;
\item
  $\qstart = q_0$, $\qaccept = q_1$ и $\qreject = q_2$.
\end{itemize}

Така можем да кодираме преходите на машина на Тюринг като думи.
Да разгледаме прехода $\delta(q_i,X_j) = (q_k,X_\ell,D_m)$.
Кодираме този преход със следната дума:
\[0^i10^j10^k10^\ell10^m.\]
Да обърнем внимание, че в този двоичен код няма последователни единици и той 
започва и завършва с нула.

За да кодираме една машина на Тюринг $\M$ е достатъчно да кодираме функцията на преходите $\delta$.
Понеже $\delta$ е крайна функция, нека с числото $r$ да означим броя на всички възможни преходи.
По описания по-горе начин, нека $\texttt{code}_i$ е числото в двоичен запис, получено за $i$-тия преход на $\delta$.
Тогава кодът на $\M$ е следното число в двоичен запис:
\[\code{\M} \df 111\ \texttt{code}_1\ 11\ \texttt{code}_2\ 11\ \cdots\ 11\ \texttt{code}_r\ 111.\]
\begin{itemize}
\item
  Лесно се съобразява, че за две машини на Тюринг $\M$ и $\M'$ с различни функции на преходите, имаме $\code{\M} \neq \code{\M'}$.
% \item
%   Ще казваме, че числото $r\in\Nat$ е {\bf код на} $\M$, ако $r$, записано в двоичен запис представлява думата $\code{\M}$.
%   Оттук нататък, когато пишем $\M_r$, ще имаме предвид машината на Тюринг с код $r$.
% \item
%   Ясно е, че не всяко естествено число е код на машина на Тюринг, но по дадено число $n$
%   има ефективна процедура, която ни казва дали $n$ е код на машина на Тюринг или не.
% \item
  % С $\pair{\M,\omega}$ ще означаваме кода на $\M$ при вход $\omega$ е числото с двоичен запис описанието на $\M$ и след това прикрепена думата $\omega$.
  % При едно число $r = \pair{M,\omega}$, лесно се намира кода на $\M$.
  % Просто започваме да четем двоичния запис на $r$ докато не срещнем за втори път $111$.
  % След това започва думата $\omega$.
% \item
%   Да въведем означението $\M_i$ за произволно ествестено число $i$.
%   Ако $i$ е код на машина на Тюринг $\M$, то $\M_i \df \M$.
%   Ако $i$ не е код на машина на Тюринг, то $\M_i$ е машина на Тюринг с празна функция на преходите.
\end{itemize}

\begin{problem}
  Докажете, че следните езици са разрешими:
  \begin{itemize}
  \item 
    $L = \{\ \code{\M} \mid \M\text{ е машина на Тюринг } \}$;
  \item 
    $L = \{\ \code{\M} \mid \M\text{ е детерминистична машина на Тюринг }\}$.
  \end{itemize}
\end{problem}

% \begin{remark}
%   По-късно ще докажем, че следният език {\bf не} е разрешим:
%   \[L_{\texttt{tot}} = \{\ \code{\M} \mid \M\text{ е разрешител }\}.\]
% \end{remark}


%%% Local Variables:
%%% mode: latex
%%% TeX-master: "../eai"
%%% End:


\subsection{Диагоналният език $L_{\texttt{diag}}$}

% Нека $\omega_0,\omega_1,\dots,\omega_n,\dots$ е каноничната подредба на всички думи над азбуката $\{0,1\}$.
% Да разгледаме безкрайната таблица $\{a_{ij} \mid i,j \in \Nat\}$, където:
% \begin{align*}
%   a_{ij} = 
%   \begin{cases}
%     1, & \text{ ако $j$ е код на М.Т. и }\omega_i \in L(\M_j), \\
%     0, & \text{ ако $j$ не е код на М.Т. или } \omega_i \not\in L(\M_j).
%   \end{cases}
% \end{align*}

% Идеята е да вземем $0$-ите по диагонала на тази таблица.

\begin{framed}
  \begin{thm}
    Езикът 
    \[L_{\texttt{diag}} \df \{ \alpha \in \{0,1\}^\star \mid \alpha = \code{\M}\text{, където $\M$ е М.Т. и }\code{\M} \not\in L(\M)\}\]
    не се разпознава от машина на Тюринг, т.е. $L_{\texttt{diag}}$ {\bf не} е полуразрешим език.
  \end{thm}
\end{framed}
\begin{proof}
  Да допуснем, че $L_{\texttt{diag}}$ се разпознава от машина на Тюринг $\M$, т.е. 
  \[L_{\texttt{diag}} = \L(\M).\]
  Тогава:
  \begin{align*}
    & \code{\M} \in L_{\texttt{diag}} \implies \code{\M} \in \L(\M) \implies \code{\M} \not\in L_{\texttt{diag}},\\
    & \code{\M} \not\in L_{\texttt{diag}} \implies \code{\M} \not\in \L(\code{\M}) \implies \code{\M} \in L_{\texttt{diag}}.
  \end{align*}
  Достигаме до противоречие.
\end{proof}

\begin{prop}
  Езикът 
  \[L_{\texttt{halt}} \df \{\code{\M} \mid \text{$\M$ е М.Т. и }\code{\M} \in \L(\M)\}\]
  е полуразрешим, но не е разрешим.
\end{prop}
\begin{hint}
  Лесно се съобразява, че $L_{\texttt{halt}}$ е полуразрешим.
  Дефинираме машина на Тюринг $\M'$, която работи по следния начин:
  \begin{itemize}
  \item
    вход дума $\alpha$;
  \item 
    $\M'$ проверява дали $\alpha$ има вида $\code{\M}$,
    за някоя машина на Тюринг $\M$;
  \item
    Ако $\alpha = \code{\M}$, 
    то $\M'$ симулира работата на $\M$ върху $\alpha$.
    \begin{itemize}
    \item 
      Ако $\M$ завърши след краен брой стъпки като приема $\alpha$,
      то $\M'$ приема $\alpha$.
    \item
      Ако $\M$ завърши след краен брой стъпки като отхвърля $\alpha$,
      то $\M'$ отхвърля $\alpha$.
    \item
      Ако $\M$ никога не завършва върху $\alpha$,
      то $\M'$ също никога не завършва върху $\alpha$.
    \end{itemize}
  \item
    Ако $\alpha$ няма вида $\code{\M}$,
    то $\M'$ завършва като отхвърля думата $\alpha$.
  \end{itemize}
  Получаваме, че
  \[\alpha \in L_{\texttt{halt}} \iff \alpha \in \L(\M'),\]
  откъдето следва, че $L_{\texttt{halt}}$ е полуразрешим език.

  Ако допуснем, че $L_{\texttt{halt}}$ е разрешим,
  то 
  \[L_{\texttt{diag}} = (\{0,1\}^\star \setminus L_{\texttt{halt}}) \cap \{\code{\M} \mid \text{$\M$ е М.Т.}\}\]
  е разрешим език, което е противоречие.
\end{hint}


%%% Local Variables:
%%% mode: latex
%%% TeX-master: "../eai"
%%% End:


\subsection{Универсалният език $L_{\texttt{univ}}$}

\setlength{\epigraphwidth}{0.65\textwidth}\epigraph{A man provided with paper, pencil, and rubber, and subject to strict discipline, is in effect a universal machine. (Turing 1948: 416)}


\marginpar{Можем за простота да считаме, че всички разглеждани машини на Тюринг са дефинирани над азбуката $\{0,1\}$.}
\index{език!неразрешим}
\begin{framed}
  \begin{thm}
    Езикът 
    \[\Luniv \df \{\ \code{\M} \sharp \omega \mid \text{$\M$ е машина на Тюринг и }\omega\in \L(\M)\ \}\]
    е полуразрешим, но {\bf не} е разрешим.
  \end{thm}
\end{framed}
\begin{hint}
  \marginpar{Разсъждението е много сходно с това защо $\Laccept$ полуразрешим.}
  Първо да съобразим защо $\Luniv$ е полуразрешим език.
  Дефинираме (многолентова) машина на Тюринг $\M'$, която работи по следния начин:
  \begin{itemize}
  \item
    вход дума $\alpha$;
  \item 
    $\M'$ проверява дали $\alpha$ има вида $\code{\M} \cdot \omega$,
    за някоя машина на Тюринг $\M$ и дума $\omega$. Това става лесно, защото $\omega$
    започва веднага след второ срещане на $111$ в $\alpha$.
  \item
    Ако $\alpha = \code{\M} \sharp \omega$, 
    то $\M'$ симулира работата на $\M$ върху $\omega$.
    \begin{itemize}
    \item 
      Ако $\M$ завърши след краен брой стъпки като приеме $\omega$,
      то $\M'$ приема $\alpha$.
    \item
      Ако $\M$ завърши след краен брой стъпки като отхвърли $\omega$,
      то $\M'$ отхвърля $\alpha$.
    \item
      Ако $\M$ никога не завършва върху $\omega$,
      то очевидно $\M'$ също никога не завършва върху $\alpha$.
    \end{itemize}
  \item
    Ако $\alpha$ няма вида $\code{\M} \cdot \omega$,
    то $\M'$ завършва веднага като отхвърля думата $\alpha$.
  \end{itemize}
  Получаваме, че
  \[\alpha \in \Luniv \iff \alpha \in \L(\M').\]
  
  Сега да съобразим защо $\Luniv$ не е разрешим език.
  Имаме, че за произволна дума $\omega$,
  \[\omega \in \Laccept \iff \omega\sharp\omega \in \Luniv.\]
  Ако допуснем, че $\Luniv$ е разрешим, то тогава $\Laccept$ е разрешим език, което е противоречие.
\end{hint}

\begin{cor}
  Езикът
  \[\ov{\Luniv} \df \{\code{\M} \cdot \omega \mid \code{\M} \text{ е машина на Тюринг и }\omega\not\in \L(\M)\}\]
  {\bf не} е полуразрешим.
\end{cor}





%%% Local Variables:
%%% mode: latex
%%% TeX-master: "../eai"
%%% End:


\section{Критерий за разрешимост}

\mynote{Сипсър нарича $\leq_m$ \emph{mapping reducibility} \cite[235]{sipser3}.}

\begin{important}
  Доказателството, че $\Luniv$ не е разрешим е пример за една обща схема, с която можем да докажем, че даден език не е разрешим:
  \begin{itemize}
  \item 
    Нека имаме езика $K$, за който вече знаем, че не е разрешим.
    В нашия пример, $K = \Laccept$.
  \item
    Питаме се дали някой друг език $L$ е разрешим.
  \item
    Намираме изчислима тотална функция $f$, за която е изпълнено, че:
    \[\omega \in K \iff f(\omega) \in L.\]
    В \Theorem{universal}, това е функцията $f(\omega) = \omega \sharp \omega$.
  \item
    В този случай ще означаваме $K \leq_m L$.
  \item
    Тогава, ако $L$ е разрешим ще следва, че $K$ е разрешим, което е противоречие.
  \end{itemize}
\end{important}

Сега искаме да разгледаме един критерий, който ще ни казва кога един език съставен от кодове на машини на Тюринг е разрешим. С негова помощ ще можем директно да решаваме наглед трудни задачи. Например,
в момента не е очевидно защо следния език не е разрешим:
\begin{align*}
  L_{\texttt{palin}} \df \{\omega \in \{0,1\}^\star \mid & \ \omega\text{ е код на машина на Тюринг и }\L(\M_\omega)\\
                                                         & \text{ съдържа само думи палиндроми}\}.
\end{align*}
След малко ще видим, че според критерия, който ще разгледаме, директно ще можем да заключим, че $L_{\texttt{palin}}$ не е разрешим. Да започнем с няколко примера.

\begin{important}
  \begin{proposition}
    \label{pr:rice:sigma-star}
    Езикът
    \[L_{\Sigma^\star} \df \{\omega \in \{0,1\}^\star \mid \omega\text{ е код на машина на Тюринг и }\L(\M_\omega) = \Sigma^\star\}\]
    не е разрешим.
  \end{proposition}  
\end{important}
\begin{proof}
  \mynote{$L_{\Sigma^\star}$ не е дори полуразрешим, но за момента не знаем как да докажем това.}
  Ще дефинираме тотална изчислима функция $f$, която при вход думата $\omega \in \{0,1\}^\star$ работи по следния начин:
  \begin{itemize}
  \item
    Ако $\omega$ не е код на машина на Тюринг, то $f(\omega) \df \omega$.
  \item
    Ако $\omega$ е код на машина на Тюринг $\M_\omega$, то
    $f(\omega) \df \code{\M'}$, където $\M'$, при вход произволна дума $\alpha$, работи по следния начин:
    \begin{enumerate}[(1)]
    \item
      Първоначално $\M'$ не обръща внимание на $\alpha$, а $\M'$ симулира работата на $\M_\omega$ върху думата $\omega$.
    \item 
      Ако след краен брой стъпки симулацията завърши с резултат, че $\M_\omega$ приема думата $\omega$,
      то $\M'$ завършва като приема думата $\alpha$.
    \end{enumerate}    
  \end{itemize}
  Получаваме, че:
  \[\L(\M') =
    \begin{cases}
      \Sigma^\star, & \text{ако } \omega \in \L(\M_\omega)\\
      \emptyset, & \text{ако } \omega \not\in \L(\M_\omega).
    \end{cases}
  \]
  Можем да заключим, че за произволна дума $\omega$ са изпълнени импликациите:
  \begin{align*}
    \omega \in \Laccept & \implies \L(\M_{f(\omega)}) = \Sigma^\star \implies f(\omega) \in L_{\Sigma^\star},\\
    \omega \in \Lcode \setminus \Laccept & \implies \L(\M_{f(\omega)}) = \emptyset \implies f(\omega) \not\in L_{\Sigma^\star}\\
    \omega \not\in\Lcode & \implies f(\omega) = \omega \implies f(\omega) \not\in L_{\Sigma^\star},
  \end{align*}
  които можем да обобщим в следната еквивалентност:
  \[\omega \in \Laccept \iff f(\omega) \in L_{\Sigma^\star}\]
  Ако допуснем, че $L_{\Sigma^\star}$ е разрешим език, то $\Laccept$ също ще е разрешим, което е противоречие.
\end{proof}

\begin{corollary}
  Езикът
  \[\ov{L}_{\texttt{empty}} \df \{\omega \in \{0,1\}^\star \mid \omega \text{ е код на машина на Тюринг и }\L(\M_\omega) \neq \emptyset\}\]
  е полуразрешим, но не е разрешим.
\end{corollary}
\begin{hint}
  Съобразете, че в може да използвате функцията $f$ от доказателството на \Proposition{rice:sigma-star} за да получите, че:
  \[\omega \in \Laccept \iff f(\omega) \in \ov{L}_{\texttt{empty}}.\]
\end{hint}

\begin{corollary}
  Езикът
  \[\Lempty \df \{\omega \in \{0,1\}^\star \mid \omega\text{ е код на машина на Тюринг и }\L(\M_\omega) = \emptyset\}\]
  не е полуразрешим.
\end{corollary}
\begin{hint}
  Ако $\Lempty$ беше разрешим, то неговото допълнение
  \[\ov{L}_{\texttt{empty}} = \Lcode \setminus \Lempty\]
  щеше да е разрешим език, което е противоречие.

  Ако $\Lempty$ беше полуразрешим, тогава, използвайки, че $\ov{L}_{\texttt{empty}}$ е полуразрешим, от теоремата на Клини-Пост щеше да следва, че
  $\Lempty$ е разрешим, което е противоречие
\end{hint}


\begin{important}
  \begin{proposition}
    Езикът
    \[\Lreg \df \{\ \omega \mid \omega\text{ е код на машина на Тюринг и }\L(\M_\omega) \text{ е регулярен език}\ \}\]
    не е разрешим.
  \end{proposition}
\end{important}
\begin{proof}
  \mynote{\cite[стр. 219]{sipser3}}
  Да фиксираме един език, за който знаем, че не е регулярен, например, 
  $\{0^n1^n \mid n \in \Nat\}$.
  Ще дефинираме тотална изчислима функция $f$, която при вход думата $\omega \in \{0,1\}^\star$ работи по следния начин:
  \begin{itemize}
  \item
    Ако $\omega$ не е код на машина на Тюринг, то $f(\omega) \df \omega$.
  \item
    Ако $\omega$ е код на машината на Тюринг $\M_\omega$, то тогава $f(\omega) \df \code{\M'}$,
    където $\M'$, при вход произволна дума $\alpha$, работи така:
    \begin{enumerate}[(1)]
    \item
      Ако $\alpha = 0^n1^n$, за някое $n$, то $\M'$ приема думата $\alpha$.
    \item
      Ако $\alpha$ не е от вида $0^n1^n$, тогава $\M'$ симулира работата на $\M_\omega$ върху думата $\omega$.
    \item
      Ако след краен брой стъпки симулацията завърши с резултат, че $\M_\omega$ приема думата $\omega$, то $\M'$ завършва като приема $\alpha$.
    \end{enumerate}
  \end{itemize}
  \mynote{Използваме наготово, че $\{0,1\}^\star$ е регулярен език.}
  Получаваме, че:
  \[\L(\M') =
    \begin{cases}
      \{0,1\}^\star & \text{ако } \omega \in \L(\M_\omega)\\
      \{0^n1^n \mid n \in \Nat\} & \text{ако } \omega \not\in \L(\M_\omega).
    \end{cases}
  \]
  Сега можем да заключим, че за произволна дума $\omega$ са изполнени импликациите:
  \begin{align*}
    \omega \in \Laccept & \implies \L(\M_{f(\omega)}) = \{0,1\}^\star \implies f(\omega) \in \Lreg,\\
    \omega \in \Lcode \setminus \Laccept & \implies \L(\M_{f(\omega)}) = \{0^n1^n \mid n \in \Nat\} \implies f(\omega) \not\in \Lreg,\\
    \omega \not\in \Lcode & \implies f(\omega) = \omega \implies f(\omega) \not\in \Lreg,
  \end{align*}
  което можем да обединим в еквивалентността:
  \[\omega \in \Laccept \iff f(\omega) \in \Lreg\]
  и ако допуснем, че $\Lreg$ е разрешим език, то $\Laccept$ също ще е разрешим, което е противоречие.  
\end{proof}

Сега ще видим, че идеята, която следвахме в горните доказателства може да се обобщи.
Нека $\Ss$ е множество от полуразрешими езици над фиксирана азбука $\Sigma$.
Ще казваме, че $\Ss$ е свойство на полуразрешимите езици.
Например, 
\[\Ss = \{L \subseteq \Sigma^\star \mid L\text{ е регулярен език}\}.\]
$\Ss$ е {\bf тривиално свойство}, ако $\Ss = \emptyset$ или $\Ss$ съдържа точно всички полуразрешими езици.
Нека разгледаме изброимото множество от всички машини на Тюринг, които разпознават езиците от $\Ss$.
Ще представим това множество като език от кодовете на тези машини на Тюринг, т.е.
\index{$\texttt{Code}(\Ss)$}
\[\texttt{Code}(\Ss) \df \{\omega \mid \text{$\omega$ е код на машина на Тюринг и } \L(\M_\omega) \in \Ss\}.\]
\index{$\texttt{Code}(L)$}
\mynote{Можем да дефинираме и $\texttt{Code}(L)$, което е безкрайно изброимо множество, ако $L$ е полуразрешим език.}

\begin{problem}
  Докажете, че езикът
  \[L_{\texttt{Dec}} = \{\omega \in \{0,1\}^\star \mid \omega \text{ е код на машина на Тюринг и }\L(\M_\omega)\text{ е разрешим}\}\]
  не е разрешим.
\end{problem}

\begin{problem}
  Докажете, че езикът
  \begin{align*}
    L_{\texttt{palin}} \df \{\omega \in \{0,1\}^\star \mid & \ \omega\text{ е код на машина на Тюринг и }\L(\M_\omega)\\
                                                           & \text{ съдържа само думи палиндроми}\}.
  \end{align*}
  не е разрешим.
\end{problem}

Сега вече имаме достатъчно опит за да видим точно кои проблеми са разрешими.

\begin{important}
  \begin{theorem}[Райс 1953 \cite{rice}]
    \index{Райс}
    \mynote{\cite[стр. 188]{hopcroft1}}
    За всяко нетривиално свойство $\Ss$ на полуразрешимите езици,
    $\texttt{Code}(\Ss)$ е неразрешим.
  \end{theorem}
\end{important}
\begin{proof}
  \mynote{Цел: да сведем ефективно $\Laccept$ към $L_\Ss$}
  Без ограничение на общността, нека $\emptyset \not\in \Ss$.
  Понеже $\Ss$ е нетривиално свойство, да разгледаме езика $L \in \Ss$,
  като $\M_L$ е машина на Тюринг, за която $\L(\M_L) = L$.
  Ще дефинираме тотална изчислима функция $f$, която при вход думата $\omega \in \{0,1\}^\star$ работи по следния начин:
  \begin{itemize}
  \item
    Ако $\omega$ не е код на машина на Тюринг, то $f(\omega) \df \omega$.
  \item
    Ако $\omega$ е код на машината на Тюринг $\M_\omega$, то тогава $f(\omega) \df \code{\M'}$,
    където $\M'$, при вход произволна дума $\alpha$, работи така:
    \begin{enumerate}[(1)]
    \item
      първоначално $\M'$ не обръща внимание на входната дума $\alpha$, а започва да симулира работата на $\M_\omega$ върху $\omega$.
    \item
      % \mynote{в този случай ще получим, че $\L(\M') = L$}
      ако след краен брой стъпки симулацията завърши с резултат, че $\M_\omega$ приема думата $\omega$, то 
      $\M'$ започва да симулира работата на $\M_L$ върху входната дума $\alpha$;
    \item
      ако след краен брой стъпки симулацията завърши с резултат, че $\M_L$ приема думата $\alpha$, то 
      $\M'$ приема входната дума $\alpha$;
    \end{enumerate}
  \end{itemize}
  Така получаваме, че:
  \[\L(\M') =
    \begin{cases}
      L, & \text{ако }\omega \in \L(\M_\omega)\\
      \emptyset, & \text{ако }\omega \not\in \L(\M_\omega).
    \end{cases}
  \]
  Оттук заключаваме, че за произволна дума $\omega$ са изпълнени импликациите:
  \begin{align*}
    \omega \in \Laccept & \implies \L(\M_{f(\omega)}) = L \implies f(\omega) \in \texttt{Code}(\Ss),\\
    \omega \in \Lcode \setminus \Laccept & \implies \L(\M_{f(\omega)}) = \emptyset \implies f(\omega) \not\in\texttt{Code}(\Ss),\\
    \omega \not\in \Lcode & \implies f(\omega) = \omega \implies f(\omega) \not\in\texttt{Code}(\Ss),
  \end{align*}
  които можем да обобщим в следната еквивалентност:
  \[\omega \in \Laccept \iff f(\omega) \in \texttt{Code}(\Ss).\]
  Aко допуснем, че $\texttt{Code}(\Ss)$ е разрешимо множество, то ще следва, че $\Laccept$ е разрешимо, което е противоречие.

  Ако $\emptyset \in \Ss$, то правим горните разсъждения за класа от езици
  \[\ov{\Ss} = \{ L \subseteq \Sigma^\star \mid L\text{ е полуразрешим език и } L \not\in\Ss\ \}.\]
  По аналогичен начин доказваме, че $\texttt{Code}(\ov{\Ss})$ не е разрешим език.
  Понеже 
  \[\texttt{Code}(\ov{\Ss}) = \Lcode \setminus \texttt{Code}(\Ss),\]
  то $\texttt{Code}(\Ss)$ също не е разрешим език.
\end{proof}

\begin{corollary}
  За всяко от следните свойства $\Ss$ на полуразрешимите езици, 
  $\texttt{Code}(\Ss)$ {\bf не} е разрешим език, където:
  \mynote{Тук няма нужда нищо да доказваме. Просто съобразяваме, че всяко от тези свойства на полуразрешимите езици е нетривиално.}
  \begin{enumerate}[a)]
  \item 
    $\Ss$ е свойството празнота, т.е. езикът
    \[\texttt{Code}(\Ss) = \{\omega \mid \text{$\omega$ е код на машина на Тюринг и } \L(\M_\omega) = \emptyset\}\]
    не е разрешим;
  \item 
    $\Ss$ е свойството за пълнота, т.е. езикът
    \[\texttt{Code}(\Ss) = \{\omega \mid \text{$\omega$ е код на машина на Тюринг и } \L(\M_\omega) = \Sigma^\star\}\]
    не е разрешим;
  \item
    $\Ss$ е свойството крайност, т.е. езикът
    \[\texttt{Code}(\Ss) = \{\omega \mid \text{$\omega$ е код на машина на Тюринг и }|\L(\M_\omega)| < \infty\}\]
    не е разрешим;
  \item
    $\Ss$ е свойството безкрайност, т.е. езикът
    \[\texttt{Code}(\Ss) = \{\omega \mid \text{$\omega$ е код на машина на Тюринг и }|\L(\M_\omega)| = \infty\}\]
    не е разрешим;
  \item
    $\Ss$ е свойството регулярност, т.е. езикът
    \[\texttt{Code}(\Ss) = \{\omega \mid \text{$\omega$ е код на машина на Тюринг и $\L(\M_\omega)$ е регулярен език}\}\]
    не е разрешим;
  \item
    \mynote{Това свойство е нетривиално, защото вече показахме, че $\{a^nb^nc^n \mid n \in \Nat\}$ е полуразрешим (дори разрешим) език, а знаем отдавна, че този език не е безконтекстен.}
    $\Ss$ е свойството безконтекстност, т.е. езикът
    \[\texttt{Code}(\Ss) = \{\omega \mid \text{$\omega$ е код на машина на Тюринг и $\L(\M_\omega)$ е безконтекстен}\}\]
    не е разрешим;
  \item
    \mynote{Тук също - вече сме разгледали примери за полуразрешими езици, които не са разрешими.}
    $\Ss$ е свойството разрешимост, т.е. езикът
    \[\texttt{Code}(\Ss) = \{\omega \mid \text{$\omega$ е код на машина на Тюринг и $\L(\M_\omega)$ е разрешим}\}\]
    не е разрешим.
  \end{enumerate}
\end{corollary}


%%% Local Variables:
%%% mode: latex
%%% TeX-master: "../eai"
%%% End:


\section{Критерии за полуразрешимост}

\begin{lemma}
  Нека $\Ss$ е свойство на полуразрешимите езици.
  Ако съществува безкраен език $L_0 \in \Ss$, който няма крайно подмножество в $\Ss$,
  то $L_\Ss$ не е полуразрешим език.  
\end{lemma}
\begin{hint}
  Нека $L_0 = \L(\M_0)$.
  Ще опишем алгоритъм, който при вход дума $\pair{\M,\omega}$,
  извежда код на машина на Тюринг $\M'$, която работи така:
  \begin{itemize}
  \item 
    вход думата $\alpha$;
  \item
    за $\abs{\alpha}$ стъпки симулираме $\M$ върху $\omega$.
    \begin{itemize}
    \item 
      ако $\M$ не приема $\omega$ за $\leq \abs{\alpha}$ стъпки, то симулираме $\M_0$ върху $\alpha$;
    \item 
      ако $\M$ приема $\omega$ за $\leq \abs{\alpha}$ стъпки, то зацикляме и нищо не връщаме.
    \end{itemize}
  \end{itemize}

  Така получаваме, че 
  \begin{align*}
    \L(\M') = 
    \begin{cases}
      \{\alpha \in L_0 \mid \abs{\alpha} < k\}, & \M\text{ приема }\omega\\
      L, & \M\text{ не приема }\omega,
    \end{cases}
  \end{align*}
  където $k$ е минималната стъпка, при която $\M$ приема $\omega$.
  
  Заключаваме, че 
  \[\code{\M}\cdot \omega \not\in L_{\texttt{univ}} \iff \code{\M'} \in L_\Ss.\]
  Това означава, че ефективно можем да сведем въпрос за принадлежност в $\bar{L}_u$
  към въпрос за принадлежност в $L_\Ss$.
  Следователно, ако $L_\Ss$ е полуразрешим език, то $\bar{L}_{\texttt{univ}}$ е полуразрешим език, което е противоречие.
\end{hint}

\begin{cor}
  Следните езици {\bf не} са полуразрешими:
  \begin{itemize}
  \item 
    $L = \{\pair{\M} \mid \abs{\L(\M)} = \infty\}$;
  \item
    $L = \{\pair{\M} \mid \L(\M) = \Sigma^\star\}$;
  \item
    $L = \{\pair{\M} \mid \L(\M)\text{ не е разрешим}\}$;
  \item
    $L = \{\pair{\M} \mid \L(\M)\text{ не е полуразрешим}\}$;
  \item
    $L = \{\pair{\M} \mid \L(\M)\text{ не е регулярен}\}$.
  \end{itemize}
\end{cor}

\begin{lemma}
  Нека $L_1$ е език в $\Ss$ и нека $L_2$ е полуразрешимо множество, разширяващо $L_1$, и $L_2 \not\in\Ss$.
  Тогава $L_\Ss$ не е полуразрешимо.
\end{lemma}
\begin{hint}
  Нека $L_1 = \L(\M_1)$ и $L_2 = \L(\M_2)$.
  Ще опишем алгоритъм, който при вход дума $\code{\M}\cdot\omega$,
  извежда код на машина на Тюринг $\M'$, която работи така:
  \begin{itemize}
  \item 
    вход думата $\alpha$;
  \item
    Симулираме едновременно две изчисления - $\M_1$ върху $\alpha$ и $\M$ върху $\omega$
    докато намерим $s$, такова че:    
    \begin{itemize}
    \item 
      ако $\M_1$ приеме думата $\alpha$ за $s$ стъпки, то обявяваме, че $\M'$ приема $\alpha$ и завършваме.
    \item
      ако $\M_1$ не приема думата $\alpha$ за $s$ стъпки, но $\M$ приема $\omega$ за $s$ стъпки, 
      то започваме да симулираме $\M_2$ върху $\alpha$.
      Ако $\M_2$ приеме $\alpha$, то $\M'$ приема $\alpha$.
    \end{itemize}
    Ако не съществува такава стъпка $s$, то работата $\M'$ никога няма да завърши и 
    следователно $\M'$ не завършва върху вход $\alpha$.
  \end{itemize}
  
  Получаваме, че:
  \begin{align*}
    \L(\M') = 
    \begin{cases}
      L_2, & \M\text{ приема }\omega\\
      L_1, & \M\text{ не приема }\omega.
    \end{cases}
  \end{align*}
  Заключаваме, че:
  \[\code{\M}\cdot\omega \not\in L_{\texttt{univ}} \iff \code{\M'} \in L_\Ss.\]
  Това означава, че ефективно можем да сведем въпрос за принадлежност в $\bar{L}_{\texttt{univ}}$
  към въпрос за принадлежност в $L_\Ss$.
  Следователно, ако $L_\Ss$ е полуразрешим език, то $\bar{L}_{\texttt{univ}}$ е полуразрешим език, което е противоречие.  
\end{hint}

\begin{cor}
  Следните езици {\bf не} са полуразрешими:
  \begin{itemize}
  \item 
    $L = \{\pair{\M} \mid \L(\M) \text{ е регулярен} \}$;
  \item
    $L = \{\pair{\M} \mid \L(\M) \text{ е безконтекстен} \}$;
  \item
    $L = \{\pair{\M} \mid \L(\M) \text{ е разрешим} \}$;
  \item
    $L = \{\pair{\M} \mid \abs{\L(\M)} = 42\}$;
  \end{itemize}
\end{cor}


% % \section{Проблеми за безконтекстни езици}

% % \begin{lemma}
% %   Нека е дадена $\M = \TM$.
% %   Тогава езикът 
% %   \[L = \{\alpha\sharp\beta^R \mid \alpha,\beta \in \Gamma^\star Q \Gamma^\star\ \&\  \alpha \vdash_\M \beta\}\]
% %   е безконтекстен.
% % \end{lemma}
% % \begin{proof}
% %   Ще покажем, че съществува стеков автомат $P$, за който $\L_S(P) = L$.
% %   Четем буквата $X$. Тогава:
% %   \begin{itemize}
% %   \item 
% %     ако $\delta_\M(q,X) =(p,Y,R)$, то слагаме $Yp$ на върха на стека;
% %   \item
% %     ако $\delta_\M(q,X) =(p,Y,L)$, то ако $Z$ е върха на стека, заменяме $Z$ с $pZY$;
% %   \end{itemize}
% % \end{proof}

\begin{lemma}
  Нека е дадена $\M = \TM$.
  Тогава езикът 
  \[L = \{\alpha\sharp\beta^R \mid \alpha,\beta \in \Gamma^\star Q \Gamma^\star\ \&\  \alpha \not\vdash_\M \beta\}\]
  е безконтекстен.
\end{lemma}


% % \begin{thm}
% %   Неразрешим е проблемът за проверка дали при дадени две произволни безконтекстни граматики $G_1$ и $G_2$,
% %   $\L(G_1) \cap \L(G_2) = \emptyset$.  
% % \end{thm}

% % \begin{thm}
% %   Неразрешим е проблемът за проверка дали при дадена произволна безконтекстна граматика $G$,
% %   $\L(G) = \Sigma^\star$.  
% % \end{thm}


% % \section{Въпроси}

% % Вярно ли е, че следният проблем е {\em разрешим}:
% % \begin{itemize}
% % \item
% %   за произволна безконтекстна граматика $G$, проверява дали $\L(G) = \emptyset$?
% % \item
% %   за произволна безконтекстна граматика $G$, проверява дали $\L(G) = \Sigma^\star$?
% % \item
% %   за произволни безконтекстни граматики $G_1$ и $G_2$, проверява дали $\L(G_1) \cap \L(G_2) = \emptyset$?
% % \item
% %   за произволни безконтекстни граматики $G_1$ и $G_2$, проверява дали $\L(G_1) \cap \L(G_2) = \Sigma^\star$?
% % \item
% %   за произволни безконтекстни граматики $G_1$ и $G_2$, проверява дали $\L(G_1) = \L(G_2)$?
% % \item
% %   за произволни безконтекстни граматики $G_1$ и $G_2$, проверява дали $\L(G_1) \subseteq \L(G_2)$?
% % \item
% %   за произволна безконтекстна граматика $G$ и произволен регулярен израз $r$,
% %   проверява дали $\L(G) = \L(r)$?
% % \item
% %   за произволна безконтекстна граматика $G$ и произволен регулярен израз $r$,
% %   проверява дали $\L(G) \subseteq \L(r)$?
% % \item
% %   за произволна безконтекстна граматика $G$ и произволен регулярен израз $r$,
% %   проверява дали $\L(r) \subseteq \L(G)$?
% % \item
% %   за произволни безконтекстни граматики $G_1$ и $G_2$, проверява дали $\L(G_1) \subseteq \L(G_2)$ 
% %   е безконтекстен език ?
% % \item
% %   за произволна безконтекстна граматика $G$, проверява дали $\Sigma^\star \setminus \L(G)$
% %   е безконтекстен език ?
% % \item
% %   за произволна безконтекстна граматика $G$, проверява дали $\L(G)$ е регулярен език?
% % \end{itemize}

%%% Local Variables:
%%% mode: latex
%%% TeX-master: "../eai"
%%% End:


\subsection*{Безконтекстен език за преходите в машина на Тюринг}

% \subsection*{Валидни и невалидни изчисления на машини на Тюринг}
\marginpar{\cite{hopcroft1} стр. 201}
Да разгледаме машината на Тюринг $\M$.

Една дума $\omega$ описва конфигурация на машина на Тюринг,
ако $\omega \in \Gamma^\star Q \Gamma^\star$.

\begin{framed}
  \begin{prop}
    Да фиксираме една машина на Тюринг $\M$. 
    Тогава следните езици за безконтекстни:
    \begin{align*}
      & \texttt{Valid}(\M) \df \{\ \alpha\#\beta^{rev} \mid \alpha,\beta \in \Gamma^\star Q \Gamma^\star\ \&\ \alpha \vdash_\M \beta\ \} \\
      & \texttt{Valid}'(\M)\df \{\ \alpha^{rev}\#\beta \mid \alpha,\beta \in \Gamma^\star Q \Gamma^\star\ \&\ \alpha \vdash_\M \beta\ \} \\
      & \texttt{Invalid}(\M) \df \{\ \alpha\#\beta^{rev} \mid \alpha,\beta \in \Gamma^\star Q \Gamma^\star\ \&\  \alpha \not\vdash_\M \beta\ \}\\
      & \texttt{Invalid}'(\M) \df \{\ \alpha^{rev}\#\beta \mid \alpha,\beta \in \Gamma^\star Q \Gamma^\star\ \&\ \alpha \not\vdash_\M \beta\ \}.
    \end{align*}
  \end{prop}  
\end{framed}

\begin{hint}

  Да напомним първо как дефинираме релацията $\vdash_\M$:
  \begin{align*}
    & (\alpha_1z, q, x\alpha_2) \vdash_\M  (\alpha_1 zy, p, \alpha_2) & \comment{\text{ ако } q \overset{x/y;\goright}{\longrightarrow} p} \\
    & (\alpha_1z, q, x\alpha_2) \vdash_\M (\alpha_1, p ,zy\alpha_2) & \comment{\text{ ако } q \overset{x/y;\goleft}{\longrightarrow} p} \\
    & (\alpha_1z, q, x\alpha_2) \vdash_\M (\alpha_1z, p, y\alpha_2) & \comment{\text{ ако } q \overset{x/y;\stay}{\longrightarrow} p}.
  \end{align*}

  Думите в езика $\texttt{Valid}(\M)$ кодират релацията $\vdash_\M$. Това означава, че всяка дума на 
  $\texttt{Valid}(\M)$ има някое от следните представяния:
  \begin{align*}
    & \alpha_1zqx\alpha_2 \sharp \alpha^{rev}_2 p y z \alpha^{rev}_1 & \comment{\text{ ако } q \overset{x/y;\goright}{\longrightarrow} p} \\
    & \alpha_1zqx\alpha_2 \sharp \alpha^{rev}_2 y z p \alpha^{rev}_1 & \comment{\text{ ако } q \overset{x/y;\goleft}{\longrightarrow} p} \\
    & \alpha_1zqx\alpha_2 \sharp \alpha^{rev}_2 y p z\alpha^{rev}_1 & \comment{\text{ ако } q \overset{x/y;\stay}{\longrightarrow} p}
  \end{align*}

  Ще опишем неформално стеков автомат $P$ за езика $\texttt{Valid}(\M)$.
  Нека 
  \[Q^{P} \df \{r_q \mid q \in Q^\M\} \cup \{r, \hat{r}\}.\]

  \begin{itemize}
  \item
    Първо четем $\alpha_1$ и я записваме в стека като $\alpha^{rev}_1$.
    Това правим като дефинираме функцията на преходите като 
    \[(\forall a,z \in \Sigma)[\ \Delta_{P}(r,a,z) \df \{(r,az)\}\ ].\]
  \item 
    Правим това докато не срещнем някое $q \in Q^\M$. Тогава трябва да направим преход на $\M$.
    Тук трябва да внимаваме, защото за да направим преход, трябва да знаем състоянието $q$ и да прочетем следващия символ.
    Един начин да разрешим този проблем е като запомним кое състояние сме прочели на машината на Тюринг в състоянията на стековия автомат:
    \marginpar{$Q^\M$ са букви от азбуката на стековия автомат $P$.}
    \[(\forall q \in Q^\M)(\forall z \in \Sigma)[\ \Delta_{P}(r,q,z) = \{(r_q,z)\}\ ].\]
    \begin{itemize}
    \item 
      \marginpar{Стекът представлява $z\alpha^{rev}_1$}
      ако $\delta_\M(q,x) = (p,y,\goright)$, то слагаме $yp$ на върха на стека, т.е.
      \[\delta_{P}(r_q,x,z) = \{(\hat{r}, pyz)\}.\]
    \item
      ако $\delta_\M(q,x) =(p,y,\goleft)$, то ако $z$ е върха на стека, заменяме $z$ с $pzy$, т.е.
      \[\delta_{P}(r_q,x,z) = \{(\hat{r}, yzp)\}.\]
    \item
      ако $\delta_\M(q,x) =(p,y,\stay)$, то ако $z$ е върха на стека, заменяме $z$ с $ypz$, т.е.
      \[\delta_{P}(r_q,x,z) = \{(\hat{r}, ypz)\}.\]
    \end{itemize}
  \item
    Сега вече сме в състояние $\hat{r}$ и остава да прочетем $\alpha_2$ и да я запишем в стека като $\alpha^{rev}_2$:
    \[\delta_{P}(\hat{r},x,z) = \{(\hat{r}, xz)\}.\]
  \item
    \marginpar{За да разпознаем $\texttt{Invalid}(\M)$ трябва само да разменим условията за приемане и отхвърляне на думата.}
    Разбираме кога сме свършили с $\alpha_2$ когато стигнем до $\sharp$.
    Сега започваме да четем думата след $\sharp$ и сравняваме с това, което имаме в стека.
    \begin{itemize}
    \item
      Ако намерим разлика, то отхвърляме думата.
    \item
      Ако достигнем до дъното на стека, то приемаме думата.
    \end{itemize}
  \end{itemize}
\end{hint}

\begin{remark}
  Да обърнем внимание, че горната конструкция на стековия автомат $P$ е {\bf ефективна}, т.е.
  съществува алгоритъм, който при вход машина на Тюринг $\M$ връща като изход стеков автомат $P$ за езика $\texttt{Valid}(\M)$.
  % С други думи, езикът 
  % \[\{\code{\M} \cdot \code{P} \mid \L(P) = \texttt{Valid}(\M)\}\]
  % е разрешим.
\end{remark}

\subsection*{История на машина на Тюринг}
\index{история на приемащо изчисление}

Дума от вида  $\omega_1 \sharp \omega^{rev}_2 \sharp \omega_3 \sharp \omega^{rev}_4\sharp\omega_5\cdots$
се нарича {\bf история на приемащо изчисление} на машината на Тюринг $\M$, ако
\begin{itemize}
\item
  $\omega_i \in \Gamma^\star Q \Gamma^\star$, т.е. $\omega_i$ описва моментна конфигурация
  и $\omega_i$ не започва и не завършва на $\blank$.
\item
  $\omega_1 \in \qstart\Sigma^\star$ описва начална конфигурация.
\item
  $\omega_n \in \Gamma^\star \cdot\{\qaccept\} \cdot \Gamma^\star$ описва приемаща конфигурация.
\item 
  $\omega_i \vdash_\M \omega_{i+1}$ за $i = 1,\dots,n-1$.
\end{itemize}

\begin{lemma}
  \marginpar{\cite{hopcroft1}, стр. 201}
  Нека да означим с $\texttt{Accept}(\M)$ езикът от историите на всички приемащи изчисления за машината на Тюринг $\M$.
  Тогава 
  \[\texttt{Accept}(\M) = L_1 \cap L_2,\]
  където $L_1$ и $L_2$ са безконтекстни езици.
  Освен това, граматиките на $L_1$ и $L_2$ могат ефективно да бъдат построени от $\M$.
\end{lemma}
\begin{hint}
  Да разгледаме езиците:
  \begin{align*}
    & L_1 \df (\texttt{Valid}(\M)\sharp)^\star(\{\varepsilon\}\cup \Gamma^\star \cdot \{\qaccept\} \cdot \Gamma^\star\sharp)\\
    & L_2 \df \qstart\Sigma^\star \sharp (\texttt{Valid'}(\M)\sharp)^\star(\{\varepsilon\}\cup \Gamma^\star \cdot \{\qaccept\} \cdot \Gamma^\star\sharp),
  \end{align*}
  за които е ясно, че са безконтекстни.
\end{hint}

\begin{problem}
  Обяснете как може ефективно да се кодира всяка безконтекстна граматика $G$ като дума $\code{G}$ над азбуката $\{0,1\}$.
\end{problem}


\begin{framed}
\begin{thm}
  Езикът
  \[L = \{\code{G_1}\cdot\code{G_2} \mid \text{$G_1$ и $G_2$ са безконт. грам. и }\L(G_1) \cap \L(G_2) = \emptyset\}\]
  не е полуразрешим.
\end{thm}  
\end{framed}
\begin{hint}
  По дадена дума $\code{\M}$, можем ефективно да намерим $G_1$ и $G_2$, за които
  $\L(G_1) \cap \L(G_2) = \texttt{Accept}(\M)$, т.е. съществува тотална изчислима функция $f$, за която
  \[f(\code{\M}) = \code{G_1} \cdot \code{G_2}.\]
  Тогава ако $L$ е полуразрешим език, то $L_{\texttt{Empty}}$ е полуразрешим език, което е противоречие, защото
  \[\code{\M} \in L_{\texttt{Empty}} \iff f(\code{\M}) \in L.\]
\end{hint}

\begin{lemma}
  За всяка машина на Тюринг $\M$, $\overline{\texttt{Accept}(\M)}$ е безконтекстен език.
\end{lemma}
\begin{hint}
  Една дума $\alpha$ не е история на приемащо изчисление, ако е изпълнено някое от следните условия:
  \begin{itemize}
  \item 
    \marginpar{Можем да опишем това свойство с регулярен език}
    $\alpha$ не е от вида $\omega_1 \sharp \omega_2 \sharp \cdots \sharp \omega_n$,
    където $\omega_i \in \Gamma^\star Q \Gamma^\star$, или
  \item
    ако $\alpha$ е от вида $\omega_1 \sharp \omega_2 \sharp \cdots \sharp \omega_n$,
    където $\omega_i \in \Gamma^\star Q \Gamma^\star$, тогава:
    \begin{itemize}
    \item 
      $\omega_1 \not\in \qstart \Gamma^\star$, или
    \item
      $\omega_n \not\in \Gamma^\star \cdot \{\qaccept\} \cdot \Gamma^\star$, или
    \item
      $\omega_i \not\vdash_\M \omega^{rev}_{i+1}$, за някое нечетно $i$, или
    \item
      $\omega^{rev}_i \not\vdash_\M \omega_{i+1}$, за някое четно $i$.
    \end{itemize}
  \end{itemize}
  Думите притежаващи някое от тези свойства могат да се опишат като обединение на три регулярни езика и двата безконтекстни езика.
\end{hint}

\begin{framed}
  \begin{thm}
    За дадена азбука $\Sigma$, 
    езикът 
    \[\texttt{All}_{\texttt{CFG}} = \{\code{G} \mid G\text{ е безконтекстна граматика и }\L(G) = \Sigma^\star\}\]
    не е полуразрешим език.
  \end{thm}
\end{framed}
\begin{hint}
  \marginpar{Тук използваме, че ако $\L(\M) = \emptyset$, то $\overline{\texttt{Accept}(\M)} = \Sigma^\star$.}
  По дадена дума $\code{\M}$, можем ефективно да намерим $G$, за която
  $\L(G)$ са точно невалидните изчисления на $\M$.
  Тогава ако допуснем, че $L$ е полуразрешим език, то $L_{\texttt{Empty}}$ е полуразрешим, което е противоречие.
\end{hint}

\begin{cor}
  Следните езици не са разрешими:
  \begin{enumerate}[a)]
  \item
    $\{\code{G_1}\cdot\code{G_2} \mid \text{$G_1$ и $G_2$ са безконт. грам. и }\L(G_1) = \L(G_2)\}$;
  \item
    $\{\code{G_1}\cdot\code{G_2} \mid \text{$G_1$ и $G_2$ са безконт. грам. и }\L(G_1) \subseteq \L(G_2)\}$;
  \item 
    $\{\code{G}\cdot r \mid \text{$G$ е безконт. грам. и $r$ е рег. израз и }\L(G) = \L(r)\}$;
  \item
    $\{\code{G}\cdot \code{\A} \mid \text{$G$ е безконт. грам. и $\A$ е ДКА и }\L(G) = \L(\A)\}$;
  \item 
    $\{\code{G}\cdot r \mid \text{$G$ е безконт. грам. и $r$ е рег. израз и }\L(r) \subseteq \L(G)\}$;
  \item
    $\{\code{G}\cdot \code{\A} \mid \text{$G$ е безконт. грам. и $\A$ е ДКА и }\L(\A) \subseteq \L(G)\}$.
  \end{enumerate}
\end{cor}

\begin{remark}
  Добре е да обърнем внимание, че езикът 
  \[L = \{\code{G}\cdot \code{\A} \mid \text{$G$ е безконт. грам. и $\A$ е ДКА и }\L(G) \subseteq \L(\A)\}\]
  е разрешим.
  Това е така, защото $\L(G) \subseteq \L(\A) \iff \L(G) \cap \L(\ov{\A}) = \emptyset$,
  защото сечението на безконтекстен и регулярен език е безконтекстен език.
\end{remark}

\newpage

\begin{framed}
  \begin{prop}
    Езикът 
    \[\texttt{Reg} = \{\code{G} \mid G\text{ е безконт. грам. и $\L(G)$ е регулярен}\}\]
    не е разрешим.
  \end{prop}
\end{framed}
\begin{hint}
  Да фиксираме език $L_0$, който е безконтекстен, но не е регулярен.
  За произволен език $L$, да разгледаме езика
  \[\hat{L} \df L_0 \sharp \Sigma^\star\ \cup\ \Sigma^\star \sharp L.\]
  Първо ще докажем, че: 
  \begin{equation}
    \label{eq:2}
    L = \Sigma^\star\ \iff\ \hat{L}\text{ е регулярен}.
  \end{equation}
  Да отбележим, че можем ефективно да получим от безконтекстна граматика $G$ за $L$
  безконтекстна граматика $\hat{G}$ за $\hat{L}$.
  Нека да означим с $\texttt{conv}$ изчислимата функция, за която
  $\texttt{conv}(\code{G}) = \code{\hat{G}}$.

  \begin{itemize}
  \item 
    Ако $L = \Sigma^\star$, то $\hat{L}$ е регулярен, защото тогава
    $\hat{L} = \Sigma^\star \sharp \Sigma^\star$ е очевидно регулярен.
  \item
    \marginpar{Ако $L$ е регулярен, то $L/_\beta \df \{\alpha \mid \alpha\beta \in L\}$ е регулярен}  
    Ако $L \neq \Sigma^\star$, то нека да фиксираме дума $\omega \not\in L$.
    Ако допуснем, че $\hat{L}$ е регулярен, то езикът
    $\hat{L}/_{\sharp\omega} = L_0$ ще е регулярен, което е противоречие с избора на $L_0$.
  \end{itemize}
  
  Нека да означим
  \[\texttt{Full} \df \{\code{G} \mid G\text{ е безконтекстна граматика и }\L(G) = \Sigma^\star\}.\]
  От (\ref{eq:2}) имаме, че 
  \[\code{G} \in \texttt{Full}\ \iff\ \texttt{conv}(\code{G}) \in \texttt{Reg}.\]
  
  Ако допуснем, че $\texttt{Reg}$ е разрешим език, то тогава ще следва, че
  $\texttt{Full}$ е разрешим език, за което вече знаем, че не е вярно.
\end{hint}


%%% Local Variables:
%%% mode: latex
%%% TeX-master: "../eai"
%%% End:

\begin{theorem}[Грейбах 1963]
  \index{Грейбах}
  \marginpar{\cite[стр. 205]{hopcroft1}}
  \marginpar{на англ. Sheila Greibach}
  Нека $\mathcal{C}$ е клас от езици, за който съществува ефективно кодиране $\code{L}$ на езиците в $\mathcal{C}$ и който е:
  \marginpar{По дадена дума $\omega$ можем ефективно да проверим дали тя кодира език от $\mathcal{C}$ или не.}
  \begin{itemize}
  \item 
    ефективно затворен относно обединение;
  \item
    ефективно затворен относно конкатенация с регулярен език;
  \item
    "$= \Sigma^\star$" е неразрешим за достатъчно голяма $\Sigma$.
  \end{itemize}
  \marginpar{Съществуват езици от $\mathcal{C}$, които не притежават свойството $P$ и такива, които го притежават.}
  Нека $P$ е нетривиално свойство на $\mathcal{C}$, което е изпълнено за всеки регулярен език и ако $L \in P$,
  то $L/_a \in P$, където
  \[L/_a = \{\omega \mid \omega a \in L\}.\]
  Тогава езикът $\{\code{L} \mid P(L)\ \&\ L \in \mathcal{C}\}$ е неразрешим.
\end{theorem}
\begin{hint}
  Да фиксираме език $L_0 \in \Cs$, за който не е изпълнено свойството $P$.
  Нека да приемем, че $L_0 \subseteq \Sigma^\star$, която е достатъчно голяма азбука, за която
  въпроса ``$= \Sigma^\star$'' е неразрешим.
  За произволен език $L \subseteq \Sigma^\star$, да разгледаме езика
  \[\hat{L} \df L_0 \sharp \Sigma^\star\ \cup\ \Sigma^\star \sharp L.\]
  Ясно е, че $\hat{L}\in \mathcal{C}$, защото $\mathcal{C}$ е ефективно затворен относно конкатенация с регулярен език и относно обединение. 
  Първо ще докажем, че: 
  \begin{equation}
    \label{eq:2}
    L = \Sigma^\star\ \iff\ \code{\hat{L}} \in P.
  \end{equation}

  \begin{itemize}
  \item 
    Ако $L = \Sigma^\star$, то $\hat{L}$ е регулярен, защото тогава
    $\hat{L} = \Sigma^\star \sharp \Sigma^\star$ е очевидно регулярен и от избора на $P$, $\code{\hat{L}} \in \mathcal{C}$.
  \item
    \marginpar{Ако $\code{L} \in P$ , то за $L/_\beta \df \{\alpha \mid \alpha\beta \in L\}$ е изпълнено $P$.}  
    Ако $L \neq \Sigma^\star$, то нека да фиксираме дума $\omega \not\in L$.
    Ако допуснем, че $\code{\hat{L}} \in P$, то езикът
    за езикът $\hat{L}/_{\sharp\omega} = L_0$ също ще е изпълнено свойството $P$, което е противоречие с избора на $L_0$.
  \end{itemize}

  От (\ref{eq:2}) следва, че $P$ е разрешимо свойство точно тогава, когато въпросът ''$=\Sigma^\star$'' за езиците от $\mathcal{C}$ е разрешим, което е противоречие.
\end{hint}

\begin{corollary}
  Езикът
  \[\texttt{Reg} = \{\code{G} \mid G\text{ е безконтекстна граматика и }\L(G)\text{ е регулярен език}\}\]
  е неразрешим.
\end{corollary}
\begin{proof}
  Ясно е, че имаме ефективно кодиране на безконтекстните граматики $\code{G}$ и освен това те са
  ефективно затворени относно конкатенация с регулярен език и относно обединение.
  Вече знаем от ..., че $= \Sigma^\star$ за безконтекстни граматики е неразрешим за достатъчно голяма азбука $\Sigma$.
  Тогава от теоремата на Грейбах следва, че $\texttt{Reg}$ е неразрешим език.
\end{proof}


%%% Local Variables:
%%% mode: latex
%%% TeX-master: "../eai"
%%% End:

\newpage
\section{Неограничени граматики}
\index{граматика!неограничена}

\begin{definition}
  \mynote{\cite[стр. 220]{hopcroft1}}
  \mynote{На англ. unrestricted grammar}
  \mynote{Според йерархията на Чомски, това е граматика от тип 0}
  Граматиката $G = (V,\Sigma,R,S)$
  се нарича неограничена граматика, 
  ако правилата $R$ са от вида $\alpha \to \beta$,
  където $\alpha,\beta \in (V\cup\Sigma)^\star$.
\end{definition}

\begin{lemma}
  За всеки полуразрешим език $L$, $L = \L(G)$, за някоя неограничена граматика $G$.  
\end{lemma}
\begin{proof}
  Нека $L = \L(\M)$, където 
  \[\M = \TM\] е детерминистична машина на Тюринг,
  като искаме лентата да е безкрайна само отдясно и входната дума $\alpha$ е
  поставена в началото на лентата.
  Ще построим граматика $G = \CFG$, където 
  \[V = ((\Sigma\cup\{\varepsilon\})\times\Gamma) \cup \{A_1,A_2,A_3\}.\]
  Правилата на $G$ са следните:
  \begin{enumerate}[1)]
  \item 
    $A_1 \to sA_2$;
  \item
    $A_2 \to [a,a]A_2$, за всяка $a\in\Sigma$;
  \item
    $A_2 \to A_3$;
  \item
    $A_3 \to [\varepsilon,\blank]A_3$;
  \item
    $A_3 \to \varepsilon$;
  \item
    $q[a,X] \to [a,Y]p$, за всяка $a \in \Sigma\cup\{\varepsilon\}$, всяко $q\in Q$, $X,Y \in\Gamma$, 
    за които $\delta(q,X) = (p,Y,R)$;
  % \item
  %   $q[a,X] \to p[a,Y]$, за всяка $a \in \Sigma\cup\{\varepsilon\}$, всяко $q\in Q$, $X,Y \in\Gamma$, 
  %   за които $\delta(q,X) = (p,Y,N)$;
  \item
    $[b,Z]q[a,X] \to p[b,Z][z,Y]$, за всяко $X,Y,Z \in \Gamma$, $a,b\in\Sigma\cup\{\varepsilon\}$, $q\in Q$,
    за които $\delta(q,X) = (p,Y,L)$;
  \item
    $[a,X]q \to qaq$, $q[a,X] \to qaq$, $q \to \varepsilon$, за всяко $a\in\Sigma\cup\{\varepsilon\}$, $X\in\Gamma$,
    и $q \in F$.
  \end{enumerate}
  
  Лесно се вижда, че, използвайки правилата 1) и 2), за всяко $n$, имаме
  \[A_1 \to^\star s[a_1,a_1]\cdots[a_n,a_n]A_2,\]
  където $a_i \in \Sigma$.

  Нека $\M$ приема думата $\alpha = a_1\cdots a_n$.
  Това означава, че за някое $m$, $\M$ използва не повече от $m$ клетки от лентата отдясно на входната дума.
  Ясно е, че имаме
  \[A_1 \to^\star s[a_1,a_1]\cdots[a_n,a_n][\varepsilon,\blank]^m.\]
  Оттук нататък, можем да използваме само правилата 6), 7), 8), докато не срещнем финално състояние.
  С индукция по броя на стъпки в $\M$, можем да докаже, че ако е изпълнено
  $(\varepsilon,s,a_1\cdots a_n) \vdash^\star_\M (X_1\cdots X_{r-1},q,X_r\cdots X_l)$, 
  то \[s[a_1,a_1]\dots[a_n,a_n][\varepsilon,\blank]^m \rightarrow^\star_G [a_1,X_1]\cdots[a_{r-1},X_{r-1}]q[a_r,X_r]\cdots[a_{n+m},X_{n+m}],\]
  където $a_1,\dots,a_n \in \Sigma$, $a_{n+1},\dots,a_{n+m} = \varepsilon$, $X_1,\dots,X_{n+m} \in \Gamma$ и
  $X_{l+1} = X_{l+2} = \dots = X_{n+m} = \blank$.
  
  Най-накрая, ако $q \in F$, то можем да използваме правилата от 9) и да докажем, че
  \[[a_1,X_1]\cdots[a_{t-1},X_{t-1}]q[a_t,X_t]\cdots[a_{n+m},X_{n+m}] \rightarrow^\star_G a_1\cdots a_n.\]
  
  Така доказахме, че ако $\alpha \in \L(\M)$, то $\alpha \in \L(G)$, т.е. $\L(\M) \subseteq \L(G)$.
  За да докажем обратната посока, трябва да направим подобни разсъждения.
\end{proof}

\begin{lemma}
  Ако $L = \L(G)$, където $G$ е неограничена граматика, то $L$ е полуразрешим език.
\end{lemma}
\mynote{Доказателствата в \cite{hopcroft1} и \cite{papadimitriou} са различни}
\begin{proof}
  $\M$ ще бъде недетерминистична машина с три ленти.
  \begin{enumerate}[1)]
  \item
    Записваме входната дума $\omega$ на първата лента на $\M$.
    Тя никога не се променя.
  \item
    На втората лента ще имаме думата $\gamma \in (V\cup\Sigma)^\star$.
    В началото $\gamma := S$.
  \item 
    Недетерминистично избираме правило $\alpha \to \beta$ от граматиката $G$.
  \item
    Недетерминистично избираме $\gamma_0,\gamma_1 \in (V\cup\Sigma)^\star$, за които 
    $\gamma = \gamma_0\alpha\gamma_1$.
    Тогава $\gamma := \gamma_0\beta\gamma_1$.
    Ако няма такива $\gamma_0$ и $\gamma_1$, то $\M$ ,,зацикля'' - текущият опит за извеждане на $\omega$ пропада.
  \item
    Сравняваме съдържанието на първите две ленти, т.е. проверяваме дали $\omega = \gamma$.
    Ако $\omega = \gamma$, то спираме и казваме, че $\M$ разпознава думата $\omega$.
    Ако $\omega \neq \gamma$, то се връщаме на стъпка 3).
  \end{enumerate}

  \begin{algorithm}[H]
  \caption{}
%  \label{alg:}
  \begin{algorithmic}[1]
    \State $\gamma:= S$
    \ForAll{$\alpha\to\beta \in R$}
    \If{$(\exists \gamma_0,\gamma_1\in (V\cup\Sigma)^\star)[\gamma = \gamma_0\alpha\gamma_1]$}
    \State $\gamma := \gamma_0\beta\gamma_1$
    \Else ...
    \EndIf
    \EndFor
  \end{algorithmic}
\end{algorithm}

\end{proof}

\begin{example}
  Граматика за $L = \{a^nb^nc^n \mid n\in\Nat\}$.
\end{example}

%%% Local Variables:
%%% mode: latex
%%% TeX-master: "../eai"
%%% End:


% \section{Контекстни граматики}
\index{граматика!контекстна}
\mynote{На англ. context-sensitive \cite[стр. 223]{hopcroft1}. Евентуално позволяваме и правилото $S \to \varepsilon$,
ако искаме да включим $\varepsilon$ в езика, като $S$ не се среща в дясна страна на правило.}
Казваме, че $G = (V,\Sigma,R,S)$ е {\bf контекстна граматика}, ако правилата на $G$ са от вида
$\rho A \delta \to \rho \alpha \delta$, където $\rho,\delta \in (V\cup\Sigma)^\star$ и $\alpha \in (V\cup\Sigma)^+$.

\begin{extra}
\begin{example}
  Езикът $L = \{a^nb^nc^n \mid n > 0\}$ е контекстен.
\end{example}
\begin{hint}
  Разгледайте контекстната граматика $G$ зададена със следните правила:
  \mynote{Съобразете, че имаме извода $CB \derive{4}_G BC$.}
  \begin{align*}
    & S \to aSBC\ |\ aBC\\
    & CB \to CZ\\
    & CZ \to WZ\\
    & WZ \to WC\\
    & WC \to BC\\
    & aB \to ab\\
    & bB \to bb\\
    & bC \to bc\\
    & cC \to cc.
  \end{align*}

  Докажете, че за всяко $n > 0$ е изпълено следното:
  \begin{itemize}
  \item
    $S \derive{n} a^n(BC)^n$;
  \item
    $(BC)^n \derive{n-1} B^nC^n$;
  \item
    $aB^n \derive{n} ab^n$;
  \item
    $bC^n \derive{n} bc^n$.
  \end{itemize}
  Оттук лесно можем да докажем, че $L \subseteq \L(G)$.
\end{hint}
\end{extra}
%%% Local Variables:
%%% mode: latex
%%% TeX-master: "../eai"
%%% End:


\section{Сложност}

\begin{itemize}
\item 
  Детерминистичната машината на Тюринг $\M$ е {\bf полиномиално ограничена}, ако 
  същестува полином $p(x)$, такъв че за всеки вход $\omega$,
  машината $\M$ завършва след най-много $p(|\omega|)$ стъпки.
\item
  Езикът $L$ се нарича {\bf полиномиално разрешим},
  ако съществува полиномиално ограничена тотална детерминистична машина на Тюринг $\M$,
  за която $L = \L(\M)$.
% \item
\item
  Недетерминистичната машината на Тюринг $\M$ е {\bf полиномиално ограничена}, ако 
  същестува полином $p(x)$, такъв че за всеки вход $\omega$,
  съществува изчисление на машината $\M$ върху думата $\omega$,
  което завършва след най-много $p(|\omega|)$ стъпки.
\item
  Езикът $L$ се нарича {\bf недетерминистично полиномиално разрешим},
  ако съществува полиномиално ограничена тотална недетерминистична машина на Тюринг $\M$,
  за която $L = \L(\M)$.
% \item
%   $\mathcal{NP} \df \{L \subseteq \Sigma^\star \mid L\text{ е полиномиално разрешим с НМТ}\}$.
\end{itemize}

\begin{framed}
  \begin{dfn}
    \begin{align*}
      & \mathcal{P} \df \{L \subseteq \Sigma^\star \mid L\text{ е полиномиално разрешим с ДМТ}\};\\
      & \mathcal{EXP} \df \{L \subseteq \Sigma^\star \mid L\text{ е експоненциално разрешим с ДМТ}\};\\
      & \mathcal{NP} \df \{L \subseteq \Sigma^\star \mid L\text{ е полиномиално разрешим с НМТ}\}.
    \end{align*}
  \end{dfn}
\end{framed}

От Теорема ... знаем, че
\[\mathcal{NP} \subseteq \mathcal{EXP}.\]

\begin{prop}
  За азбука $\Sigma$ от поне две букви, можем да обобщим някои от резултатите от предишните глави:
  \[\texttt{REG} \subsetneqq \texttt{CFG} \subsetneqq \mathcal{P}.\]
\end{prop}
\begin{hint}
  Езикът $\{a^nb^nc^n \mid n \in \Nat\} \in \mathcal{P}$,
  но не е безконтекстен.
\end{hint}


%%% Local Variables:
%%% mode: latex
%%% TeX-master: "../eai"
%%% End:


\section{Задачи}

\begin{problem}
  Вярно ли е, че следните езици са разрешими?
  \begin{enumerate}[a)]
  \item
    $\{\code{\A}\cdot \omega \mid \A \text{ е ДКА и } \omega \in \L(\A)\}$;
  \item
    $\{\code{\A} \mid \A \text{ е ДКА и } \L(\A)\text{ е безкраен език}\}$;
  \item
    $\{\code{\A} \mid \A \text{ е ДКА и }\L(\A) = \{0,1\}^\star\}$;
  \item 
    $\{\code{\A} \mid \A \text{ е ДКА и }\L(\A)\text{ съдържа поне една дума с равен брой нули и единици}\}$;
  \item
    $\{\code{\A} \mid \A \text{ е ДКА и }\L(\A)\text{ съдържа поне една дума палиндром}\}$;
  \item
    $\{\code{\A} \mid \A \text{ е ДКА и }\L(\A)\text{ не съдържа дума с нечетен брой единици}\}$;
  \item
    $\{\code{\A}\cdot\code{\B} \mid \A\text{ и }\B \text{ са ДКА и }\L(\A) \subseteq \L(\B)\}$;
  \item
    $\{\code{\A}\cdot\code{\B} \mid \A\text{ и }\B \text{ са ДКА и }\L(\A) = \L(\B)\}$;
  \end{enumerate}
\end{problem}

\begin{problem}
  Вярно ли е, че следните езици са разрешими?
  \begin{enumerate}[a)]
  \item
    $\{\code{G} \cdot \omega \mid G \text{ е безконтекстна граматика и } \omega \in \L(G)\}$;
  \item
    $\{\code{G} \mid G \text{ е безконтекстна граматика и } \L(G) = \emptyset\}$;
  \item 
    $\{\code{G} \mid G \text{ е безконтекстна граматика над }\{0,1\}^\star\text{ и }\L(1^\star) \subseteq \L(G)\}$;
  \item 
    $\{\code{G} \mid G \text{ е безконтекстна граматика над }\{0,1\}^\star\text{ и }\L(1^\star) \subseteq \L(G)\}$;
  \item 
    $\{\code{G} \mid G \text{ е безконтекстна граматика над }\{0,1\}^\star\text{ и }\varepsilon \in \L(G)\}$;
  \item
    $\{\code{G}\cdot 0^k \mid G \text{ е безконтекстна граматика над }\{0,1\}^\star\text{ и }|\L(G)| \leq k\}$;
  \item
    $\{\code{G} \mid G \text{ е безконтекстна граматика над }\{0,1\}^\star\text{ и }|\L(G)| = \infty\}$;
  \end{enumerate}
\end{problem}


\begin{problem}
  Докажете, че езикът
  \[L = \{\code{\M}\sharp\omega \mid \M\text{ прави движение наляво при работата си върху вход }\omega\}\]
  е разрешим.
\end{problem}
\begin{hint}
  Нужно е да симулирате работата на $\M$ върху $\omega$ само за $|\omega| + |Q^\M| + 1$ на брой стъпки.
\end{hint}

\begin{problem}
  Докажете, че езикът съставен от думи от вида $\code{\M}\sharp\omega$, за които
  $\M$ прави опит за движение наляво от най-лявата клетка при работата си върху вход $\omega$ е разрешим.
\end{problem}

%%% Local Variables:
%%% mode: latex
%%% TeX-master: "../eai"
%%% End:


\section{Варианти на машини на Тюринг}

\begin{itemize}
\item
  Четящата глава не отива наляво от началната позиция;
\item
  Никога на пише $\blank$.
\end{itemize}

Така имаме по-лесен начин за представянето на моментното описание на едно изчисление на машина на Тюринг като дума:
\[\hat{\Gamma}^\star \cdot Q \cdot \hat{\Gamma}^\star \cup \hat{\Gamma}^\star \cdot \{\blank\}^\star \cdot Q \cdot \{\blank\},\]
където $\hat\Gamma = \Gamma \setminus\{\blank\}$.


%%% Local Variables:
%%% mode: latex
%%% TeX-master: "../eai"
%%% End:


\section{Линейни автомати}
\marginpar{На англ. linear bounded automaton}
\index{линеен автомат}

{\bf Линеен автомат} е машина на Тюринг, на която не се позволява четящата глава да излиза извън частта от лентата, върху която първоначално е записана входната дума.

\begin{theorem}
  Езикът
  \[L = \{\code{\M}\sharp \omega \mid \M\text{ е линеен автомат и } \omega \in \L(\M)\}\]
  е разрешим.
\end{theorem}
\begin{proof}
  Това е лесно, защото изчислението на $\M$ върху входната дума $\omega$
  може да се намира в една от $|Q|\cdot|\Gamma|^{|\omega|}\cdot |\omega|$ конфигурации.
\end{proof}


\begin{theorem}
  Езикът
  \[L = \{\code{\M} \mid \M\text{ е линеен автомат и } \L(\M) = \emptyset\}\]
  е неразрешим.
\end{theorem}
\begin{proof}
  
\end{proof}


%%% Local Variables:
%%% mode: latex
%%% TeX-master: "../eai"
%%% End:


\section{Проблемът за съответствието на Пост}\label{sect:turing:pcp}

\marginpar{На англ. Post's correspondence problem}
\marginpar{\cite[стр. 392]{hopcroft2}, но по-добре е обяснено в \cite[стр. 227]{sipser3}}

\marginpar{Ако $|\alpha_i| = |\beta_i|$ за всяко $i$, то задачата е тривиална.}
Пример за проблема за съответствието на Пост се нарича всяка крайна редица от елементи на $\Sigma^\star \times \Sigma^\star$.
Ние ще означаваме примерите по следния начин:
\[\begin{bmatrix} \alpha_1\\ \beta_1\end{bmatrix},\begin{bmatrix} \alpha_2\\ \beta_2\end{bmatrix},\dots,\begin{bmatrix} \alpha_n\\ \beta_n\end{bmatrix}.\]
Казваме, че този пример има решение, ако съществува непразна редица от индекси $i_1,i_2,\dots,i_n$, такава че:
\[\alpha_{i_1}\alpha_{i_2}\cdots\alpha_{i_n} = \beta_{i_1}\beta_{i_2}\cdots\beta_{i_n}.\]

\begin{problem}
  \marginpar{\cite[стр. 239]{sipser3}}
  Намерете решение на следния пример за PCP:
  \[\begin{bmatrix}ab\\ abab\end{bmatrix},\begin{bmatrix} b\\ a\end{bmatrix},\begin{bmatrix} aba\\ b\end{bmatrix},\begin{bmatrix} aa\\ a\end{bmatrix}.\]
\end{problem}
\begin{solution}
  \[\begin{bmatrix}ab\cdot ab \cdot aba \cdot b \cdot b \cdot aa \cdot aa\\abab \cdot abab \cdot b \cdot a \cdot a \cdot a \cdot a\end{bmatrix}\]
\end{solution}


\subsection*{Модифициран проблем за съответствието }

Казваме, че MPCP има решение, ако съществува произволна редица от индекси $i_1,i_2,\dots,i_n$ (може и празна), такава че:
\[\alpha_1\alpha_{i_1}\cdots\alpha_{i_n} = \beta_1\beta_{i_1}\cdots\beta_{i_n}.\]

\begin{lemma}
  Съществува алгоритъм, който свежда MPCP към PCP.
\end{lemma}
\begin{proof}
  Нека имаме пример за MPCP:
  \[\begin{bmatrix} \alpha_1\\ \beta_1\end{bmatrix},\begin{bmatrix} \alpha_2\\ \beta_2\end{bmatrix},\dots,\begin{bmatrix} \alpha_k\\ \beta_k\end{bmatrix} .\]
  Нека символите $*,\$$ не са от $\Sigma$.
  Нека за думата $\alpha = a_1\cdots a_n$ да дефинираме следните операции:
  \begin{align*}
    & \star\alpha = *a_1*a_2\cdots *a_n\\
    & \alpha\star = a_1*a_2*\cdots a_n*\\
    & \star\alpha\star = * a_1*a_2*\cdots a_n*.
  \end{align*}
  Тогава на базата на горния пример за MPCP, строим пример за PCP:
  \[\begin{bmatrix*}[l] \star\alpha_1\star\\ \star\beta_1\end{bmatrix*},\begin{bmatrix} \alpha_1\star\\ \star \beta_1\end{bmatrix},\dots,\begin{bmatrix} \gamma_k\star\\ \star\beta_k\end{bmatrix},\begin{bmatrix*}[r] \$\\ *\$\end{bmatrix*}.\]
\end{proof}


\begin{framed}
  \begin{theorem}[Е. Пост \cite{pcp}]\index{Пост}
    Проблемът за съответствието на Пост е неразрешим при азбука $\Sigma$ с поне два символа.
  \end{theorem}
\end{framed}
\begin{proof}
  \marginpar{Лесно се съобразява, че за азбука $\Sigma$ само с една буква проблемът е разрешим.}
  Свеждаме $\Luniv$ към MPCP. Вече знаем, че MPCP се свежда към PCP.
  Нека фиксираме символа $\sharp \not \in \Gamma$.
  \begin{enumerate}[1)]
  \item
    Нека имаме като вход дума $\gamma = \code{\M}\sharp \alpha$.
  \item
    Започваме като добавяме за думата $\alpha = a_1\cdots a_n$ над азбуката $\Sigma$ следната двойка
    $\begin{bmatrix*}[l] \sharp\\ \sharp qa_1\cdots a_n\sharp\end{bmatrix*}$.
  \item
    Ако $\delta(q,a) = (p,b,\goright)$, то добавяме двойката
    $\begin{bmatrix*}[l] qa\\ bp\end{bmatrix*}$.
    Тук трябва да внимаваме, защото имаме граничен случай при $a = \blank$. Тогава трябва да добавим и двойката $\begin{bmatrix*}[l] q\sharp\\ bp\sharp\end{bmatrix*}$.
  \item
    \marginpar{Тук е важно, че не позволяваме четящата глава да се мести по-наляво от първата клетка върху която е четящата глава при стартиране на изчислението.}
    Ако $\delta(q,a) = (p,b,\goleft)$, то добавяме двойката
    $\begin{bmatrix*}[l] xqa\\ pxb\end{bmatrix*}$.
  \item
    Ако $\delta(q,a) = (p,b,\stay)$, то добавяме двойката
    $\begin{bmatrix*}[l] qa\\ pb\end{bmatrix*}$.
  \item
    за всеки $x \in \Gamma$, добавяме $\begin{bmatrix} x\\ x\end{bmatrix}$.
    Освен това, добавяме и двойката $\begin{bmatrix} \sharp\\ \sharp\end{bmatrix}$.
  \item
    \marginpar{Тук идеята е, че когато достигнем до приемащо състояние, то започваме да трием съдържанието на лентата за да можем да изравним двете ленти.}
    За всеки $x \in \Gamma$, добавяме двойката
    $\begin{bmatrix*}[l] x\qaccept\\ \qaccept\end{bmatrix*}$ и $\begin{bmatrix*}[l] \qaccept x\\ \qaccept\end{bmatrix*}$.
  \item
    За да завършим, добавяме двойката
    $\begin{bmatrix*}[r] \qaccept\sharp\sharp\\ \sharp\end{bmatrix*}$.
  \end{enumerate}
\end{proof}

\begin{corollary}
  Проблемът за еднозначност на безконтекстна граматика е неразрешим.
\end{corollary}
\begin{proof}
  Нека да означим
  \[\text{AMBIG} = \{\code{G} \mid G \text{ е нееднозначна безконтекстна граматика}\}.\]
  Ще докажем, че $\text{PCP} \leq_m \text{AMBIG}$.
  Да разгледаме един пример за $\text{PCP}$ над азбуката $\Sigma$:
  \[\begin{bmatrix} \alpha_1\\ \beta_1\end{bmatrix},\begin{bmatrix} \alpha_2\\ \beta_2\end{bmatrix},\dots,\begin{bmatrix} \alpha_n\\ \beta_n\end{bmatrix}.\]
  По него можем ефективно да построим следната безконтекстна граматика:
  \begin{align*}
    & S \to A\ |\ B\\
    & A \to \alpha_1A c_1\ |\ \alpha_2 A c_2\ |\ \cdots\ |\ \alpha_n A c_n\ |\ \alpha_1c_2\ |\ \alpha_2c_2\ |\ \cdots\ |\ \alpha_nc_n\\
    & B \to \beta_1B c_1\ |\ \beta_2 B c_2\ |\ \cdots\ |\ \beta_n B c_n\ |\ \beta_1c_2\ |\ \beta_2c_2\ |\ \cdots\ |\ \beta_nc_n,
  \end{align*}
  където $c_1,\dots,c_n \not \in \Sigma$.
  Лесно се съобразява, че горния пример за $PCP$ има решение точно тогава, когато безконтекстната граматика е нееднозначна.
\end{proof}

\begin{corollary}
  Проблемът за сечение на две безконтекстни граматики е неразрешим.
\end{corollary}
\begin{proof}
  Нека да означим
  \[\text{INTERSECT} = \{\code{G_1}\sharp\code{G_2} \mid \L(G_1) \cap \L(G_2) \neq \emptyset \}.\]
  Ще докажем, че $\text{PCP} \leq_m \text{INTERSECT}$.
  Да разгледаме един пример за $\text{PCP}$ над азбуката $\Sigma$:
  \[\begin{bmatrix} \alpha_1\\ \beta_1\end{bmatrix},\begin{bmatrix} \alpha_2\\ \beta_2\end{bmatrix},\dots,\begin{bmatrix} \alpha_n\\ \beta_n\end{bmatrix}.\]
  По него можем ефективно да построим следната безконтекстна граматика:
  \begin{align*}
    & S_1 \to \alpha_1S_1 c_1\ |\ \alpha_2 S_1 c_2\ |\ \cdots\ |\ \alpha_n S_1 c_n\ |\ \alpha_1c_2\ |\ \alpha_2c_2\ |\ \cdots\ |\ \alpha_nc_n\\
    & S_2 \to \beta_1S_2 c_1\ |\ \beta_2 S_2 c_2\ |\ \cdots\ |\ \beta_n S_2 c_n\ |\ \beta_1c_2\ |\ \beta_2c_2\ |\ \cdots\ |\ \beta_nc_n,
  \end{align*}
  където $c_1,\dots,c_n \not \in \Sigma$.
\end{proof}




%%% Local Variables:
%%% mode: latex
%%% TeX-master: "../eai"
%%% End:


% \section*{Бележки}

% \begin{itemize}
% \item
%   За основните дефиниции следваме основно Глава 3 от \cite{sipser3}.
% \item 
%   За въпросите за неразрешимост следваме основно Глава 8 от \cite{hopcroft1}.
% % \item
% %   За по-задълбочено запознаване с теория на изчислимостта, добри уводни книги са
% %   \cite{ditchev-soskov} и \cite{nikolova-soskova}.
% \end{itemize}

%%% Local Variables:
%%% mode: latex
%%% TeX-master: "../eai"
%%% End:


% \include{outro}

\bibliographystyle{amsalpha}
\bibliography{eai}

\printindex

\end{document}

%%% Local Variables: 
%%% mode: latex
%%% TeX-master: t
%%% End: 