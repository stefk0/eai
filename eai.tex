\documentclass[a4paper, 11pt, oneside]{report}
%\usepackage[paperwidth=9cm, paperheight=12cm, top=0.5cm, bottom=0.5cm, left=0.0cm, right=0.5cm]{geometry}
%\special{papersize=9cm,12cm}


%%%%%%%%%%%%%
%% MARGINS %%
%%%%%%%%%%%%%

\setlength{\marginparsep}{0.5cm}
\setlength{\oddsidemargin}{0.3cm}
\setlength{\hoffset}{0cm}
\setlength{\marginparwidth}{110pt}

\let\oldmarginpar\marginpar

\renewcommand\marginpar[1]{\leavevmode\oldmarginpar{\raggedright\scriptsize #1}}

% \renewcommand\marginpar[1]{\-\oldmarginpar[\raggedleft\scriptsize #1]%
% {\raggedright\scriptsize #1}}
%\renewcommand\marginpar[1]{\oldmarginpar{\scriptsize #1}}

%%%%%%%%%%%%%%

%\usepackage{ucs}

% \usepackage{natbib}
% \usepackage{bibentry}

\usepackage[bulgarian]{babel}
\usepackage[utf8]{inputenc}
\usepackage[colorlinks=true, linkcolor=blue,pdfstartview=FitV,citecolor=green, urlcolor=blue]{hyperref}
\usepackage{pifont}
\usepackage{amssymb}
\usepackage{amsmath}
\usepackage{mathrsfs}
\usepackage{latexsym}
\usepackage{amsthm}
% \usepackage{paralist}
% \usepackage{enumerate}
\usepackage{makeidx}
\usepackage{layout}
\usepackage{framed}
\usepackage{bussproofs}
\usepackage{algorithm}
\floatname{algorithm}{Алгоритъм}
%\usepackage{algorithmic}
\usepackage{minted}
\usepackage[noend]{algpseudocode}
%\usepackage{algpseudocode}
\usepackage{float}

\usepackage{paralist}
\usepackage[shortlabels]{enumitem}
\setlist{leftmargin=*}

%%%%%%%%%%%%%%% TIKZ Package %%%%%%%%%%%%%%%%%%%%%%%
\usepackage{tikz}
\usepackage{pgf}
\usetikzlibrary{arrows,automata}
\usetikzlibrary{positioning}
\usetikzlibrary{backgrounds}
%%%%%%%%%%%%%%%%%%%%%%%%%%%%%%%%%%%%%%%%%%%%%%%%%%%%
\usepackage{caption}
\usepackage{subcaption}

\theoremstyle{definition}
\newtheorem{thm}{Теорема}[chapter]
\newtheorem{cor}{Следствие}[chapter]
\newtheorem{lemma}{Лема}[chapter]
\newtheorem{prop}{Твърдение}[chapter]
\newtheorem{dfn}{Определение}[chapter]
\newtheorem{problem}{Задача}[chapter]
\newtheorem{example}{Пример}[chapter]
\newtheorem{question}{Въпрос}[chapter]
\newtheorem*{remark}{Забележка}
\renewenvironment{proof}{\noindent{\bf Доказателство.}\hspace*{1em}}{\qed\par}
\newenvironment{hint}{\noindent{\bf Упътване.}\hspace*{1em}}{\qed\par}
\newenvironment{solution}{\noindent{\bf Решение.}\hspace*{1em}}{\qed\par}

\newcommand{\A}{\mathcal{A}}
\newcommand{\B}{\mathcal{B}}
\renewcommand{\C}{\mathcal{C}}
\newcommand{\M}{\mathcal{M}}
\renewcommand{\L}{\mathcal{L}}
\newcommand{\D}{\mathcal{D}}
\newcommand{\R}{\mathbb{R}}
\newcommand{\Z}{\mathbb{Z}}
\newcommand{\N}{\mathcal{N}}
\newcommand{\Q}{\mathbb{Q}}
\newcommand{\Ls}{\mathscr{L}}
\newcommand{\Fs}{\mathscr{F}}
\newcommand{\Rs}{\mathscr{R}}
\newcommand{\Ps}{\mathscr{P}}
\newcommand{\As}{\mathscr{A}}
\newcommand{\Bs}{\mathscr{B}}
\newcommand{\Es}{\mathscr{E}}
\newcommand{\Is}{\mathscr{I}}
\newcommand{\Ss}{\mathscr{S}}
\newcommand{\xn}{x_{1},\dots,x_{n}}

\newcommand{\Nat}{\mathbb{N}}
\newcommand{\Int}{\mathbb{Z}}
\newcommand{\Real}{\mathbb{R}}

\newcommand{\xs}{overline{x}}

\newcommand{\ys}{overline{y}}

\newcommand{\zs}{overline{z}}
\newcommand{\ov}[1]{\overline{#1}}
\newcommand{\abs}[1]{\lvert{#1}\rvert}
\newcommand{\pair}[1]{\langle{#1}\rangle}
\newcommand{\writedown}{\ding{45}\ }

\newcommand{\FA}{\langle{Q,\Sigma,s,\delta,F}\rangle}
\newcommand{\FAn}[1]{\langle{Q_#1,\Sigma,s_#1,\delta_#1,F_#1}\rangle}
\newcommand{\NFA}{\langle{Q,\Sigma,s,\Delta,F}\rangle}
\newcommand{\NFAn}[1]{\langle{Q_#1,\Sigma,s_#1,\Delta_#1,F_#1}\rangle}
\newcommand{\PDA}{\langle{Q,\Sigma,\Gamma,\#,s,\Delta,F}\rangle}
\newcommand{\PDAn}[1]{\langle{Q_#1,\Sigma,\Gamma,\#,s_#1,\Delta_#1,F_#1}\rangle}
\newcommand{\CFG}{\langle{V,\Sigma,R,S}\rangle}
\newcommand{\TM}{\langle{Q,\Sigma,\Gamma,\delta,s,\blank,q_{accept}, q_{reject}}\rangle}

\renewcommand{\iff}{\ \leftrightarrow\ }
\newcommand{\df}{\stackrel{\text{деф}}{=}}
\newcommand{\dff}{\stackrel{\text{деф}}{\iff}}

\newcommand{\Th}[1]{{\em Теорема~\ref{th:#1}}}
\newcommand{\Lem}[1]{{\em Лема~\ref{lem:#1}}}
\newcommand{\Cor}[1]{{\em Следствие~\ref{cor:#1}}}
\newcommand{\Prob}[1]{{\em Задача~\ref{pr:#1}}}
\newcommand{\Prop}[1]{{\em Твърдение~\ref{pr:#1}}}
\newcommand{\Ex}[1]{{\em Пример~\ref{ex:#1}}}

%\newcommand*{\blank}{\allowbreak\textvisiblespace\allowbreak} % visible space
\newcommand*{\blank}{\sqcup}

% \setsecheadstyle{\large\usefont{T2A}{fag}{b}{r}} %\scshape
% \setsubsecheadstyle{\bfseries\sffamily}

% \renewcommand\familydefault{\sfdefault}

\title{Записки по ,,Езици, автомати, изчислимост''}
\author{Стефан Вътев\thanks{ел. поща: \href{mailto:stefanv@fmi.uni-sofia.bg}{stefanv@fmi.uni-sofia.bg}}}
%\\{\em Софийски Университет ,,Св. Климент Охридски''}}
%, \LaTeX\ файловете са \href{https://github.com/stefk0/EAI}{тук}}}
%, Факултет по математика и информатика, Софийски университет ,,Св. Климент Охридски''}}

% \abstract{test}

\makeindex
\begin{document}
\maketitle
% \layout
\begin{tikzpicture}
\pgftransformscale{.8}

%%% HELP LINES - uncomment to design/extend
% \draw[step=1cm,gray,very thin] (-10,0) grid (10,12);
% \node at (0,0) {\textbf{(0,0)}};

%% Horizontal bar
\draw[very thick] (10,0) -- (-10,0);

% LOG TIME
% \draw (-1,0) parabola bend (0,2) (1,0) ;
% \node at (0,1) {
%   \begin{tabular}{c}
%     Крайни \\ езици
%   \end{tabular}
% };

% LOG SPACE
\draw (-3,0) parabola bend (0,3.5) (3,0);
\node at (0,2) {
	\begin{tabular}{c}
          Крайни \\ езици
	\end{tabular}
};

\draw (-5.5,0) parabola bend (0,6) (5.5,0);
\node at (0,5) {
  \begin{tabular}{c}
    Регулярни \\ езици
  \end{tabular}
};

\draw (-7.5,0) parabola bend (0,8) (7.5,0);
\node at (0,7) {  
  \begin{tabular}{c}
    Безконтекстни \\ езици
  \end{tabular}
};

\draw (-9.5,0) parabola bend (0,10) (9.5,0);
\node at (0,9) {  
  \begin{tabular}{c}
    Разрешими \\ езици
  \end{tabular}
};

\draw[very thick] (-9.5,0) parabola bend (0,12.5) (9.5,0);
\node at (0,11) {Полуразрешими езици};
\end{tikzpicture}


\tableofcontents

\chapter{Увод}
\label{ch:intro}

\section{Съждително смятане}
\label{sect:propositional}
\marginpar{На англ. Propositional calculus}

Съждителното смятане наподобява аритметичното смятане, като вместо аритметичните операции $+,-,\cdot,/$, 
имаме съждителни операции като $\neg, \wedge, \vee$.
Например, $(p\vee q) \wedge \neg  r$ е съждителна формула.
Освен това, докато аритметичните променливи приемат стойности числа, то
съждителните променливи приемат само стойности {\bf истина (1)} или {\bf неистина (0)}.

{\bf Съждителна формула} наричаме съвкупността от съждителни променливи $p,q,r,\dots$, свързани със знаците за логически операции
$\neg, \vee, \wedge, \rightarrow, \leftrightarrow$ и скоби, определящи реда на операциите.

\subsection*{Съждителни операции}

\begin{itemize}
\item
  Отрицание $\neg$
\item 
  Дизюнкция $\vee$
\item
  Конюнкция $\wedge$
\item
  Импликация $\rightarrow$
\item
  Еквивалентност $\iff$
\end{itemize}

Ще използваме таблица за истинност за да определим стойностите на основните съждителни операции
при всички възможни набори на стойностите на променливите.

\[
\begin{array}{|c|c|c|c|c|c|c|c|c|}
  \hline
  p & q & \neg p & p \vee q & p \wedge q & p \rightarrow q & \neg p \vee q & p \iff q & (\neg{p}\wedge q)\ \vee\ (p\wedge \neg q) \\
  \hline
  0 & 0 & 1 & 0 & 0 & 1 & 1 & 1 & 1\\
  \hline
  0 & 1 & 1 & 1 & 0 & 1 & 1 & 0 & 0\\
  \hline
  1 & 0 & 0 & 1 & 0 & 0 & 0 & 0 & 0\\
  \hline
  1 & 1 & 0 & 1 & 1 & 1 & 1 & 1 & 1\\
  \hline
\end{array}
\]


{\bf Съждително верен} (валиден) е този логически израз, който има верностна стойност {\bf 1} при всички възможни набори на
стойностите на съждителните променливи в израза, т.е. стълбът на израза в таблицата за истинност трябва да съдържа само 
стойности {\bf 1}. 

Два съждителни израза $\varphi$ и $\psi$ са {\bf еквивалентни}, което означаваме $\varphi \equiv \psi$, ако са съставени от 
едни и същи съждителни променливи и двата израза имат едни и същи верностни стойности при всички комбинации от верностни 
стойности на променливите. С други думи, колоните на двата израза в съответните им таблици за истинност трябва да съвпадат.
Така например, от горната таблица се вижда, че 
$p\to q \equiv \neg p \vee q$ и $p \iff q \equiv (\neg{p}\wedge q)\ \vee\ (p\wedge \neg q)$.

\subsection*{Съждителни закони}

\begin{enumerate}[I)]
  \item
    {\bf Комутативен закон}
    \[p\vee q \equiv q\vee p\] 
    \[p \wedge q \equiv q \wedge p\]
  \item
    {\bf Асоциативен закон}
    \[(p\vee q)\vee r \equiv p\vee(q\vee r)\]
    \[(p\ \wedge\ q)\ \wedge\ r \equiv p\ \wedge\ (q\ \wedge\ r)\]
  \item
    {\bf Дистрибутивен закон}
    \[p\ \wedge\ (q \vee r) \equiv (p\ \wedge q)\vee (p\ \wedge\ r)\]
    \[p\vee (q\ \wedge\ r) \equiv (p\vee q)\ \wedge\ (p\vee r)\]
  \item
    {\bf Закони на де Морган}
    \[\neg(p \wedge q) \equiv (\neg p \vee \neg q)\]
    \[\neg(p\vee q) \equiv (\neg p \wedge \neg q)\]
  \item
    {\bf Закон за контрапозицията}
    \[p\rightarrow q \equiv \neg q \rightarrow \neg p\]
  \item
    {\bf Обобщен закон за контрапозицията}
    \[(p \wedge q)\rightarrow r \equiv (p \wedge \neg r) \rightarrow \neg q\]
  \item
    {\bf Закон за изключеното трето}
    \[p\vee \neg p \equiv {\mathbf 1}\]
  \item
    {\bf Закон за силогизма (транзитивност)}
    \[[(p\rightarrow q)\ \wedge\ (q\rightarrow r)] \rightarrow (p\rightarrow r) \equiv {\mathbf 1}\]
\end{enumerate}

Лесно се проверява с таблиците за истинност, че законите са валидни.

\section{Предикати и квантори}

\subsection*{Квантори}

Свойствата или отношенията на елементите в произволно множество се наричат {\bf предикати}.
Нека да разгледаме един едноместен предикат $P(\cdot)$.

\bigskip
\begin{tabular}{|l|p{4.2cm}|p{4.5cm}|}
  \hline
  твърдение & Кога е истина? & Кога е неистина?\\
  \hline
  $\forall x P(x)$ & $P(x)$ е вярно за всяко $x$ & съществува $x$, за което $P(x)$ {\bf не} е вярно \\
  \hline
  $\exists x P(x)$ & съществува $x$, за което $P(x)$ е вярно & $P(x)$ {\bf не} е вярно за всяко $x$\\
  \hline
\end{tabular}  
\bigskip

\begin{enumerate}[(I)]
\item 
  {\bf Квантор за общност} $\forall x$.
  Записът $(\forall x \in A) P(x)$ означава, че за всеки елемент $a$ в $A$, 
  твърдението $P(a)$ има стойност истина.
  Например, $(\forall x\in\Real)[(x+1)^2 = x^2+2x+1]$.
\item
  {\bf Квантор за съществуване} $\exists x$.
  Записът $(\exists x \in A) P(x)$ означава, че съществува елемент $a$ в $A$, 
  за който твърдението $P(a)$ има стойност истина.
  Например, $(\exists x \in\mathbb{C})[x^2 = -1]$, но $(\forall x\in\Real)[x^2 \neq -1]$.
\end{enumerate}

% \begin{example}
%   \begin{itemize}
%   \item
%     За всяко естествено число, съществува по-голямо от него:
%     \[(\forall x\in\Nat)(\exists z\in\Nat)[x < z].\]
%   \item
%     Съществува естествено число, от което няма по-малко:
%     \[(\exists x\in\Nat)(\forall y\in\Nat)[x < y \vee x = y].\]
%     Нека да означим с $Zero(x)$ предиката, който казва, че $x$ е най-малкото число, т.е.
%     \[Zero(x) \equiv (\forall y)[x < y \vee x =y].\]
%   \item
%     Нека $S(x,y)$ да бъде предиката, който казва, че $y = x+1$ в естествените числа:
%     \[S(x,y) \equiv (x < y\ \wedge\ (\forall z\in\Nat)[x < z\ \rightarrow (z = y\ \vee\ y < z)].\]
%   \item
%     $One(x)$ - $x$ е числото $1$:
%     \[One(x) \equiv (\exists y)[Z(y)\ \wedge\ S(y,x)].\]
%   \item
%     $Div(x,y)$ - $x$ се дели на $y$:
%     \[Div(x,y) \equiv (\exists z)[x = y.z].\]
%   \item
%     $Prime(x)$ - $x$ е просто число:
%     \[Prime(x) \equiv x \geq 2\ \wedge\ (\forall y\in\Nat)[\neg (O(y)\ \wedge Z(y))\ \rightarrow\ \neg Div(x,y)].\]
%   \end{itemize}
% \end{example}


\subsection*{Закони на предикатното смятане}

\begin{enumerate}[(I)]
  \item
    $\neg\forall x P(x) \iff \exists x \neg P(x)$
  \item
    $\neg\exists x P(x) \iff \forall x \neg P(x)$
  \item
    $\forall x P(x) \iff \neg\exists x \neg P(x)$
  \item
    $\exists x P(x) \iff \neg\forall x \neg P(x)$
  \item
    $\forall x \forall y P(x) \iff \forall y\forall x P(x)$
  \item
    $\exists x\exists y P(x,y) \iff \exists y \exists x P(x)$  
  \item
    $\exists x\forall y P(x,y) \rightarrow \forall y \exists x P(x,y)$
\end{enumerate}

\bigskip
\begin{tabular}{|l|p{2.5cm}|p{3.2cm}|p{3cm}|}
  \hline
  \multicolumn{4}{|c|}{{\bf Закони на Де Морган за квантори}}\\
  \hline
  твърдение & Еквивалентно твърдение & Кога е истина? & Кога е неистина?\\
  \hline
  $\neg \exists x P(x)$ & $\forall x \neg P(x)$ & за всяко $x$ $P(x)$ {\bf не} е вярно & съществува $x$, за което $P(x)$ е вярно \\
  \hline
  $\neg \forall x P(x)$ & $\exists x \neg P(x)$ & съществува $x$, за което $P(x)$ {\bf не} е вярно & $P(x)$ е вярно за всяко $x$\\
  \hline
\end{tabular}  
\bigskip

\begin{problem}
  Да означим с $K(x,y)$ твърдението ``$x$ познава $y$''.
  Изразете като предикатна формула следните твърдения.
  \begin{enumerate}[1)]
  \item
    \marginpar{$\forall x \exists y K(x,y)$}
    Всеки познава някого.
  \item
    \marginpar{$\exists x \forall y K(x,y)$}
    Някой познава всеки.
  \item
    \marginpar{$\exists x\forall y K(y,x)$}
    Някой е познаван от всички.
  \item
    \marginpar{$\forall x \exists y(K(x,y)\wedge \neg K(y,x)) $}
    Всеки знае някой, който не го познава.
  \item
    \marginpar{$\exists x \forall y(K(y,x)\ \rightarrow K(x,y))$}
    Има такъв, който знае всеки, който го познава.
  \item
    \marginpar{$(\forall x,y)(K(x,y)\ \&\ K(y,x) \to \exists z(K(x,z)\ \&\ K(y,z))$}
    Всеки двама познати имат общ познат.
  \end{enumerate}
\end{problem}

\begin{example}
  Нека $D \subseteq \Real$.
  Казваме, че $f:D \to \Real$ е {\em непрекъсната} в точката $x_0 \in D$, ако 
  \[(\forall \varepsilon > 0)(\exists \delta >0)(\forall x\in D)[|x_0 - x| < \delta \implies |f(x_0) - f(x)| < \varepsilon].\]
  Да видим какво означава $f$ да бъде {\em прекъсната} в точката $x_0 \in D$:
  \marginpar{$f$ е прекъсната в $x_0$ ако $f$ не е непрекъсната в $x_0$}
  \begin{align*}
    & \neg (\forall \varepsilon > 0)(\exists \delta >0)(\forall x\in D)[|x_0 - x| < \delta \implies |f(x_0) - f(x)| < \varepsilon] \iff\\
    & (\exists \varepsilon > 0) \neg (\exists \delta >0)(\forall x\in D)[|x_0 - x| < \delta \implies |f(x_0) - f(x)| < \varepsilon] \iff \\
    & (\exists \varepsilon > 0)(\forall \delta >0)\neg(\forall x\in D)[|x_0 - x| < \delta \implies |f(x_0) - f(x)| < \varepsilon] \iff \\
    & (\exists \varepsilon > 0)(\forall \delta >0)(\exists x\in D)\neg[|x_0 - x| < \delta \implies |f(x_0) - f(x)| < \varepsilon] \iff \\
    & (\exists \varepsilon > 0)(\forall \delta >0)(\exists x\in D)\neg[\neg (|x_0 - x| <\delta) \vee |f(x_0) - f(x)| < \varepsilon] \iff \\
    & (\exists \varepsilon > 0)(\forall \delta >0)(\exists x\in D)[\neg\neg (|x_0 - x| <\delta) \land \neg (|f(x_0) - f(x)| < \varepsilon)] \iff \\
    & (\exists \varepsilon > 0)(\forall \delta >0)(\exists x\in D)[|x_0 - x| < \delta\ \land\ |f(x_0) - f(x)| \geq \varepsilon].
  \end{align*}
\end{example}

\section{Доказателства на твърдения}

\subsection*{Допускане на противното}

Да приемем, че искаме да докажем, че свойството $P(x)$
е вярно за всяко естествено число.
Един начин да направим това е следният:
\begin{itemize}
\item 
  Допускаме, че съществува елемент $n$, за който $\neg P(n)$.
\item
  Използвайки, че $\neg P(n)$ правим извод, от който следва факт, за който знаем, че винаги е лъжа.
  Това означава, че доказваме следното твърдение
  \[\exists x \neg P(x) \rightarrow \mathbf{0}.\]
\item
  Тогава можем да заключим, че $\forall x P(x)$, защото имаме следния извод:
  \begin{prooftree}
    \AxiomC{$\exists x \neg P(x) \rightarrow \mathbf{0}$}
    \UnaryInfC{$\mathbf{1} \rightarrow \neg \exists x \neg P(x)$}
    \UnaryInfC{$\neg \exists x \neg P(x)$}
    \UnaryInfC{$\forall x P(x)$}
  \end{prooftree}
\end{itemize}

Ще илюстрираме този метод като решим няколко прости задачи.

\begin{problem}
  \label{prob:even-number-square}
  За всяко $a \in \Int$, ако $a^2$ е четно, то $a$ е четно.
\end{problem}
\begin{proof}
  Ние искаме да докажем твърдението $P$, където:
  \[P \equiv (\forall a\in\Z)[a^2\mbox{ е четно}\ \rightarrow\ a\mbox{ е четно}].\]
  \marginpar{$\neg (\forall x)(A(x) \rightarrow B(x))$ е еквивалентно на $(\exists x)(A(x) \wedge \neg B(x))$}
  Да допуснем противното, т.е. изпълнено е $\neg P$. Лесно се вижда, че
  \[\neg P \iff (\exists a\in\Z)[a^2\mbox{ е четно}\ \wedge\ a\mbox{ не е четно}].\]
  Да вземем едно такова нечетно $a$, за което $a^2$ е четно.
  Това означава, че $a = 2k+1$, за някое $k \in \Z$,
  и \[a^2 = (2k+1)^2 = 4k^2 + 4k + 1,\]
  което очевидно е нечетно число.
  Но ние допуснахме, че $a^2$ е четно.
  Така достигаме до противоречие, следователно нашето допускане е грешно 
  и 
  \[(\forall a\in\Z)[a^2\mbox{ е четно}\ \rightarrow\ a\mbox{ е четно}].\]
\end{proof}

\begin{problem}
  Докажете, $\sqrt{2}$ {\bf не} е рационално число.
\end{problem}
\begin{proof}
  Да допуснем, че $\sqrt{2}$ е рационално число. Тогава  съществуват $a,b \in \Z$, такива че
  \[\sqrt{2} = \frac{a}{b}.\]
  Без ограничение, можем да приемем, че $a$ и $b$ са естествени числа,
  които нямат общи делители, т.е. не можем да съкратим дробта $\frac{a}{b}$.
  Получаваме, че \[2b^2 = a^2.\]
  Тогава $a^2$ е четно число и от Задача \ref{prob:even-number-square}, $a$ е четно число.
  Нека $a = 2k$. Получаваме, че
  \[2b^2 = 4k^2,\]
  от което следва, че
  \[b^2 = 2k^2.\]
  Това означава, че $b$ също е четно число, $b = 2n$, за някое $n \in \Z$.
  Следователно, $a$ и $b$ са четни числа и имат общ делител $2$,
  което е противоречие с нашето допускане, че $a$ и $b$ нямат общи делители.
  Така достигаме до противоречие.
  Накрая заключаваме, че $\sqrt{2}$ не е рационално число.
\end{proof}


\subsection*{Индукция върху естествените числа}

\marginpar{Да напомним, че естествените числа са $\Nat = \{0,1,2,\dots\}$}
Доказателството с индукция по $\Nat$ представлява следната схема:
\begin{prooftree}
  \AxiomC{$P(0)$}
  \AxiomC{$(\forall x\in\Nat)[P(x)\rightarrow P(x+1)]$}
  \BinaryInfC{$(\forall x\in\Nat) P(x)$}
\end{prooftree}

Това означава, че ако искаме да докажем, че свойството $P(x)$ е вярно за всяко естествено число $x$,
то трябва да докажем първо, че е изпълнено $P(0)$ и след това, за произволно естествено число $x$, ако $P(x)$ вярно, то също така е вярно $P(x+1)$.

\begin{problem}
  \label{prob:number-prod-prime}  
  Всяко естествено число $n \geq 2$ може да се запише като произведение на прости числа.
\end{problem}
\begin{proof}
  Доказателството протича с индукция по $n \geq 2$.
  \begin{enumerate}[a)]
  \item 
    За $n = 2$  е ясно.
  \item
    Ако $n+1$ е просто число, то всичко е ясно.
    Ако $n+1$ е съставно, то \[n + 1 = n_1\cdot n_2.\]
    Тогава $n_1 = p^{n_1}_1\cdots p^{n_k}_k$ и $n_2 = q^{m_1}_1\cdots q^{m_r}_r$,
    където $p_1,\dots,p_k$ и $q_1,\dots,q_r$ са прости числа.
    Тогава е ясно, че $n+1$ също е произведение на прости числа.
  \end{enumerate}
\end{proof}

\begin{problem}
  Докажете, че за всяко $n$, 
  \[\sum^n_{i=0} 2^i = 2^{n+1} - 1.\]
\end{problem}
\begin{proof}
  Доказателството протича с индукция по $n$.
  \begin{itemize}
  \item 
    За $n = 0$, $\sum^0_{i=0}2^i = 1 = 2^{1} - 1$.
  \item
    Нека твърдението е вярно за $n$.
    Ще докажем, че твърдението е вярно за $n+1$.
    \begin{align*}
      \sum^{n+1}_{i=0} 2^i & = \sum^{n}_{i=0}2^i + 2^{n+1}\\
      & = 2^{n+1} - 1 + 2^{n+1} & (\text{от И.П.})\\
      & = 2.2^{n+1} - 1 \\
      & = 2^{(n+1)+1} - 1.
    \end{align*}
  \end{itemize}
\end{proof}

%\subsection*{Пълна индукция върху естествените числа}

\section{Множества, релации, функции}

\subsection*{Основни операции върху множества}

Ще разгледаме няколко операции върху произволни множества $A$ и $B$.
\begin{itemize}
\item
  {\bf Сечение}
  \[A\cap B = \{x\ \mid\ x\in A\ \wedge\ x\in B\}.\]
  % Казано по-формално, $A\cap B$ е множеството, за което е изпълнена формулата
  % \[(\forall x)[x \in A\cap B \iff (x\in A\ \wedge\ x \in B)].\]
  % Примери:
  % \begin{itemize}
  % \item
  %   $A \cap A = A$, за всяко множество $A$.
  % \item
  %   $A \cap \emptyset = \emptyset$, за всяко множество $A$.
  % \item
  %   $\{1,\emptyset,\{\emptyset\}\} \cap \{\emptyset\} = \{\emptyset\}$.
  %   \item
  %     $\{1,2,\{1,2\}\} \cap \{1,\{1\}\} = \{1\}$.
  %   \end{itemize}
  \item
    {\bf Обединение}
    \[A\cup B = \{x\ \mid x\in A\ \vee\ x\in B\}.\]
    % $A\cup B$ е множеството, за което е изпълнена формулата
    % \[(\forall x)[x \in A\cup B \iff (x\in A\ \vee\ x \in B)].\]
    % Примери:
    % \begin{itemize}
    % \item
    %   $A \cup A = A$, за всяко множество $A$.
    % \item 
    %   $A \cup \emptyset = A$, за всяко множество $A$.
    % \item
    %   $\{1,2,\emptyset\} \cup \{1,2,\{\emptyset\}\} = \{1,2,\emptyset,\{\emptyset\}\}$.
    % \item
    %   $\{1,2,\{1,2\}\} \cup \{1,\{1\}\} = \{1,2,\{1\},\{1,2\}\}$.
    % \end{itemize}
  \item
    {\bf Разлика}
    \[A\setminus B = \{x\ \mid\ x\in A\ \wedge\ x\not\in B\}.\]
    % $A\setminus B$ е множеството, за което е изпълнена формулата
    % \[(\forall x)[x \in A\setminus B \iff (x\in A\ \wedge\ x \not\in B)].\]
    % Примери:
    % \begin{itemize}
    % \item
    %   $A \setminus A = \emptyset$, за всяко множество $A$.
    % \item 
    %   $A \setminus \emptyset = A$, за всяко множество $A$.
    % \item 
    %   $\emptyset \setminus A = \emptyset$, за всяко множество $A$.
    % \item
    %   $\{1,2,\emptyset\} \setminus \{1,2,\{\emptyset\}\} = \{\emptyset\}$.
    % \item
    %   $\{1,2,\{1,2\}\} \setminus \{1,\{1\}\} = \{2,\{1,2\}\}$.
    % \end{itemize}
  % \item
  %   {\bf Симетрична разлика}
  %   \[A\triangle B = (A\backslash B)\cup (B\backslash A).\]
  %   % $A\triangle B$ е множеството, за което е изпълнена формулата
  %   % \[(\forall x)[x \in A\triangle B \iff [(x\in A\ \wedge\ x \not\in B) \vee (x \in B\ \wedge\ x\not\in A)]].\]
  %   Примери:
  %   \begin{itemize}
  %   \item 
  %     $A \triangle \emptyset = A$, за всяко множество $A$.
  %   \item
  %     $A \triangle A = \emptyset$, за всяко множество $A$.
  %   \item
  %     $A\triangle B = B \triangle A$, за всеки две множества $A$ и $B$.
  %   \item
  %     $\{1,2,\emptyset\} \triangle \{1,2,\{\emptyset\}\} = \{\emptyset\} \cup \{\{\emptyset\}\} = \{\emptyset,\{\emptyset\}\}$.
  %   \item
  %     $\{1,2,\{1,2\}\} \triangle \{1,\{1\}\} = \{2,\{1,2\}\} \cup \{\{1\}\} = \{2,\{1\},\{1,2\}\}$.
  %   \end{itemize}
  \item
    {\bf Степенно множество}
    \[\Ps(A) = \{x\mid x\subseteq A\}.\]
    % $\Ps(A)$ е множеството, за което е изпълнена формулата
    % \[(\forall x)[x \in \Ps(A) \iff (\forall y)[y\in x\rightarrow y \in A]].\]
    Примери:
    \begin{itemize}
    \item 
      $\Ps(\emptyset) = \{\emptyset\}$.
    \item
      $\Ps(\{\emptyset\}) = \{\emptyset,\{\emptyset\}\}$.
    \item
      $\Ps(\{\emptyset,\{\emptyset\}\}) = \{\emptyset,\{\emptyset\},\{\{\emptyset\}\},\{\emptyset,\{\emptyset\}\}\}$.
    \item
      $\Ps(\{1,2\}) = \{\emptyset,\{1\},\{2\},\{1,2\}\}$.
    \end{itemize}
  \end{itemize}
  Нека имаме редица от множества $\{A_1,A_2,\dots,A_n\}$.
  Тогава имаме следните операции:
  \begin{itemize}
  \item
    {\bf Обединение на редица от множества}
    \[\bigcup^{n}_{i=1} A_i = \{x \mid \exists i (1\leq i\leq n\ \&\ x\in A_i)\}.\]
    % \[(\forall x)[x \in \bigcup^n_{i=1}A_i \iff (\exists i)[1 \leq i \leq n\ \wedge\ x \in A_i]].\]
  \item
    {\bf Сечение на редица от множества}
    \[\bigcap^{n}_{i=1} A_i = \{x \mid \forall i (1\leq i\leq n \rightarrow x\in A_i)\}.\]
    % \[(\forall x)[x \in \bigcap^n_{i=1}A_i \iff (\forall i)[1 \leq i \leq n\ \rightarrow\ x \in A_i]].\]
  \end{itemize}

% \begin{example}
%   Нека $A = \{x\in\Nat\mid x > 1\}$ и $B = \{x\in\Nat\mid x>3\}$. Тогава :
%   \begin{itemize}
%     \item
%       $A\cap B = \{x\in\Nat\mid x > 3\}$,
%     \item
%       $A\cup B = \{x\in\Nat\mid x > 1\}$,
%     \item
%       $A\setminus B = \{x\in\Nat\mid 1<x\leq 3\}$,
%     \item
%       $B\setminus A = \emptyset$,
%     % \item
%     %   $A\triangle B = \{x\in\Nat\mid 1<x\leq 3\}$
%     \end{itemize}
% \end{example}


\begin{problem}
  Проверете верни ли са свойствата:
  \begin{enumerate}[a)]
  \item
    $A\subseteq B \iff A\setminus B = \emptyset \iff A\cup B = B \iff A\cap B = A$;
  \item
    $A\setminus \emptyset = A$, $\emptyset\setminus A=\emptyset$, $A\setminus B = B\setminus A$.
  \item
    $A\cap (B\cup A) = A \cap B$;
  \item
    $A\cup(B\cap C) = (A\cup B)\cap(A\cup C)$ и $A \cap (B \cup C) = (A \cup B) \cap (A \cup C)$;
  % \item
  %   $C\subseteq A\ \&\ C\subseteq B \rightarrow C\subseteq A\cap B$;
  % \item
  %   $A\subseteq C\ \&\ B\subseteq C \rightarrow A\cup B\subseteq C$;
  \item
    $A\backslash B = A \iff A\cap B = \emptyset$;
  \item
    $A\backslash B = A\backslash (A\cap B)$ и $A\backslash B = A\backslash (A\cup B)$;
  \item
    $(A\cup B)\setminus C = (A\setminus C) \cup (B\setminus C)$;
  % \item
  %   \marginpar{Не е вярно!}
  %   $A\setminus (B\setminus C) = (A\setminus B)\setminus C$;
  \item
    \marginpar{Закони на Де Морган}
    $C\setminus (A\cup B) = (C\backslash A)\cap(C\backslash B)$ и $C \backslash (A\cap B) = (C\backslash A)\cup(C\backslash B)$
  \item
    $C\backslash(\bigcup^{n}_{i=1} A_i) = \bigcap^{n}_{i=1} (C\backslash A_i)$ и $C \backslash(\bigcap^{n}_{i=1} A_i) = \bigcup^{n}_{i=1} (C\backslash A_i)$;
  \item
    $(A\backslash B)\backslash C = (A\backslash C)\backslash(B \backslash C)$ и $A\backslash (B\backslash C) = (A\backslash B) \cup (A\cap C)$;
  \item
    $A\subseteq B \Rightarrow \Ps(A) \subseteq \Ps(B)$;
  \item
    \marginpar{$X \subseteq A\cup B \stackrel{?}{\Rightarrow} X\subseteq A \vee X \subseteq B$}
    $\Ps(A\cap B) = \Ps(A) \cap \Ps(B)$ и $\Ps(A\cup B) = \Ps(A) \cup \Ps(B)$;
  \end{enumerate}
\end{problem}

За да дадем определение на понятието релация, трябва първо 
да въведем понятието декартово произведение на множества,
което пък от своя страна се основава на понятието наредена двойка.

\subsection*{Наредена двойка}
\index{наредена двойка}
За два елемента $a$ и $b$ въвеждаме опрецията {\bf наредена двойка} $\pair{a,b}$.
Наредената двойка $\pair{a,b}$ има следното характеристичното свойство:
\[a_1 = a_2\ \wedge\ b_1 = b_2\ \iff\ \pair{a_1,b_1} = \pair{a_2,b_2}.\]
Понятието наредена двойка може да се дефинира по много начини, стига да изпълнява харектеристичното свойство.
Ето примери как това може да стане:
\begin{enumerate}[1)]
\item
  \marginpar{Norbert Wiener (1914)}
  Първото теоретико-множествено определение на понятието наредена двойка е
  дадено от Норберт Винер:
  \[\pair{a,b} \df \{\{\{a\},\emptyset\},\{\{b\}\}\}.\]
\item
  \marginpar{Kazimierz Kuratowski (1921)}
  Определението на Куратовски се приема за ,,стандартно'' в наши дни:
  \[\pair{a,b} \df \{\{a\},\{a,b\}\}.\]
\end{enumerate}

\begin{problem}
  Докажете, че горните дефиниции наистина изпълняват харектеристичното свойство за наредени двойки.
\end{problem}

\begin{dfn}
  \marginpar{Пример за индуктивна (рекурсивна) дефиниция}
  Сега можем, за всяко естествено число $n \geq 1$,
  да въведем понятието наредена $n$-орка $\pair{a_1,\dots,a_n}$:
  \begin{align*}
    & \pair{a_1} \df a_1,\\
    & \pair{a_1,a_2,\dots,a_n} \df \pair{a_1,\pair{a_2,\dots,a_n}}
  \end{align*}
\end{dfn}

Оттук нататък ще считаме, че имаме операцията наредена $n$-орка, без да се интересуваме от нейната формална дефиниция.
 
\subsection*{Декартово произведение}
\marginpar{На англ. cartesian product}
\index{декартово произведение}
\marginpar{Считаме, че $(A\times B)\times C = A\times (B\times C) = A\times B \times C$}

За две множества $A$ и $B$, определяме тяхното декартово произведение като
\[A\times B = \{\pair{a,b}\mid a\in A\ \&\ b\in B\}.\]
За краен брой множества $A_1,A_2,\dots,A_n$, определяме
\[A_1\times A_2\times\cdots\times A_n = \{\pair{a_1,a_2,\dots,a_n}\mid a_1 \in A_1\ \&\ a_2\in A_2\ \&\ \dots\ \&\ a_n \in A_n\}.\]

\begin{problem}
  Проверете, че:
  \begin{enumerate}[a)]
  \item
    $A\times(B\cup C) = (A\times B) \cup (A\times C)$.
  \item
    $(A\cup B)\times C = (A\times C)\cup (B\times C)$.
  \item 
    $A\times(B\cap C) = (A\times B) \cap (A\times C)$.
  \item
    $(A \cap B)\times C = (A \times C)\cap(B\times C)$.
  \item 
    $A\times(B\setminus C) = (A\times B) \setminus (A\times C)$.
  \item
    $(A\setminus B)\times C = (A\times C)\setminus (B\times C)$.
  \end{enumerate}
\end{problem}

\subsection*{Основни видове бинарни релации}
% \marginpar{Бинарни релации}

Подмножествата $R$ от вида $R \subseteq A\times A\times\cdots\times A$ се наричат релации.
Релациите от вида $R\ \subseteq\ A\times A$ са важен клас, който ще срещаме често.
Да разгледаме няколко основни видове релации от този клас:
\begin{enumerate}[I)]
\item
  {\bf рефликсивна}, ако
  \[(\forall x\in A)[\pair{x,x}\in R].\]
  Например, релацията $\leq\ \subseteq\ \Nat\times\Nat$ е рефлексивна, защото
  \[(\forall x\in \Nat)[x \leq x].\]
% \item
%   {\bf антирефлексивна}, ако
%   \[(\forall x\in A)[\pair{x,x}\not\in R].\]
%   Например, релацията $<\ \subseteq\ \Nat\times\Nat$ е антирефлексивна, защото
%   \[(\forall x\in \Nat)[x \not< x].\]
\item
  {\bf транзитивна}, ако
  \[(\forall x,y,z\in A)[\pair{x,y}\in R\ \&\ \pair{y,z}\in R \rightarrow \pair{x,z}\in R].\]
  Например, релацията $\leq\ \subseteq\ \Nat\times\Nat$ е транзитивна, защото
  \[(\forall x,y,z\in A)[x \leq y\ \&\ y \leq z\ \rightarrow\ x\leq z].\]
\item
  {\bf симетрична}, ако
  \[(\forall x,y\in A)[\pair{x,y}\in R \rightarrow \pair{y,x}\in R].\]
  Например, релацията $=\ \subseteq\ \Nat\times\Nat$ е рефлексивна, защото
  \[(\forall x,y\in \Nat)[x = y\ \rightarrow\ y = x].\]
\item
  {\bf антисиметрична}, ако
  \[(\forall x,y\in A)[\pair{x,y}\in R\ \&\ \pair{y,x}\in R \rightarrow x = y].\]
  Например, релацията $\leq\ \subseteq\ \Nat\times\Nat$ е антисиметрична, защото
  \[(\forall x,y,z\in A)[x \leq y\ \&\ y \leq x\ \rightarrow\ x = y].\]
% \item
%   {\bf асиметрична}, ако
%   \[(\forall x,y)[\pair{x,y}\in R \rightarrow \pair{y,x}\not\in R].\]
%   Например, релацията $\leq\ \subseteq\ \Nat\times\Nat$ е асиметрична, защото
%   \[(\forall x,y\in \Nat)[x < y\ \rightarrow\ y \not< x].\]
\end{enumerate}

% \begin{remark}
%   Добре е да запомните как се наричат тези основни видове релации,
%   защото ще ги използваме често.
% \end{remark}

% \begin{example}
%   Да обобщим примерите от по-горе.
%   \begin{enumerate}[a)]
%   \item
%     Релацията $\leq\ \subseteq\ \Nat\times\Nat$ е рефлексивна, транзитивна и антисиметрична.
%   \item
%     Релацията $<\ \subseteq\ \Nat\times\Nat$ е антирефлексивна, транзитивна и асиметрична.
%   \item
%     Релацията $=\ \subseteq\ \Nat\times\Nat$ е рефлексивна, транзитивна и симетрична.
%   \end{enumerate}
% \end{example}

\begin{itemize}
\item
  Една бинарна релация $R$ над множеството $A$ се нарича {\bf релация на еквивалентност}, 
  ако $R$ е рефлексивна, транзитивна и симетрична.
\item 
  За всеки елемент $a \in A$, определяме неговия 
  {\bf клас на еквивалентност} относно релацията на еквивалентност $R$ по следния начин:
  \[[a]_R \df \{b\in A \mid \pair{a,b} \in R\}.\]
\end{itemize}

\begin{remark}
  Лесно се съобразява, че за всеки два елемента $a, b\in A$,
  \[\pair{a,b} \in R \iff [a]_R = [b]_R.\]
\end{remark}

\begin{example}
  За всяко естествено число $n\geq 2$, дефинираме релацията $R_n$ като
  \[\pair{x,y}\in R_n \iff x \equiv y\ (\bmod\ n).\]
  Ясно е, че $R_n$ са релации на еквивалентност.
\end{example}


\subsection*{Операции върху бинарни релации}

\begin{enumerate}[I)]
\item
  {\bf Композиция} на две релации $R \subseteq B\times C$ и $P \subseteq A\times B$ е релацията $R\circ P \subseteq A\times C$,
  определена като:
  \[R\circ P \df \{\pair{a,c} \in A\times C \mid (\exists b \in B)[\pair{a,b}\in P\ \&\ \pair{b,c} \in R]\}.\]
\item
  {\bf Обръщане} на релацията $R \subseteq A\times B$ е релацията $R^{-1}\subseteq B\times A$, 
  определена като:
  \[R^{-1} \df \{\pair{x,y} \in B\times A \mid \pair{y,x} \in R\}.\]
\item
  \marginpar{Очевидно е, че $P$ е рефлексивна релация, дори ако $R$ не е.}
  {\bf Рефлексивно затваряне} на релацията $R \subseteq A\times A$ е релацията
  \[P \df R \cup \{\pair{a,a}\mid a \in A\}.\]
\item
  {\bf Итерация} на релацията $R \subseteq A\times A$ дефинираме като за всяко естествено число $n$,
  дефинираме релацията $R^n$ по следния начин:
  \marginpar{Лесно се вижда, че  $R^1 = R$}
  \begin{align*}
    R^0 & \df \{\pair{a,a} \mid a \in A\}\\
    R^{n+1} & \df R^n \circ R.
  \end{align*}
\item
  \marginpar{\ding{45} Проверете, че $R^+$ е транзитивна релация!}
  {\bf Транзитивно затваряне} на $R \subseteq A\times A$ е релацията
  \[R^+ \df \bigcup_{n\geq 1} R^n.\]
\end{enumerate}

\index{$R^\star$}
За дадена релация $R$, с $R^\star$ ще означаваме нейното {\em рефлексивно и транзитивно затваряне}.
От дефинициите е ясно, че \[R^\star = \bigcup_{n\geq 0} R^n.\]

\subsection*{Видове функции}

Функцията $f:A \to B$ е:
\begin{itemize}
\item
  \marginpar{(или $f$ е {\bf обратима})}
  {\bf инекция}\index{функция!инекция}, ако 
  \[(\forall a_1,a_2\in A)[a_1\neq a_2 \rightarrow f(a_1)\neq f(a_2)],\]
  или еквивалентно,
  \[(\forall a_1,a_2\in A)[f(a_1) = f(a_2) \rightarrow a_1 = a_2].\]
\item
  \marginpar{(или $f$ е {\bf върху} $B$)}
  {\bf сюрекция}\index{функция!сюрекция}, ако 
  \[(\forall b\in B)(\exists a\in A)[f(a) = b].\]
\item
  {\bf биекция}\index{функция!биекция}, ако е инекция и сюрекция.
\end{itemize}

\begin{problem}
  \marginpar{Канторово кодиране. Най-добре се вижда като се нарисува таблица}
  Докажете, че $f: \Nat \times \Nat\rightarrow \Nat$ е биекция, където
  \[f(x, y) = \frac{(x+y)(x+y+1)}{2} + x.\]
\end{problem}

\section{Азбуки, думи, езици}

\subsection*{Основни понятия}

\begin{itemize}
\item 
  \index{азбука}
  {\bf Азбука} ще наричаме всяко крайно множество,
  като обикновено ще я означаваме със $\Sigma$.
  \marginpar{Често ще използваме буквите $a$, $b$, $c$ за да означаваме букви.}
  Елементите на азбуката $\Sigma$ ще наричаме {\bf букви} или символи.
\item
  \index{дума}
  {\bf Дума} над азбуката $\Sigma$ е произволна крайна редица от елементи на $\Sigma$.
  Например, за $\Sigma = \{a,b\}$, $aababba$ е дума над $\Sigma$ с дължина $7$.
  С $\abs{\alpha}$ ще означаваме дължината на думата $\alpha$.
  \marginpar{Обикновено ще означаваме думите с $\alpha$, $\beta$, $\gamma$, $\omega$.}
\item
  Обърнете внимание, че имаме единствена дума с дължина $0$.
  Тази дума ще означаваме с $\varepsilon$ и ще я наричаме {\bf празната дума},
  т.е. $\abs{\varepsilon} = 0$.
\item
  С $a^n$ ще означаваме думата съставена от $n$ $a$-та.
  Формалната индуктивна дефиниция е следната:
  \begin{align*}
    a^0 & \df \varepsilon,\\
    a^{n+1} & \df a^na.
  \end{align*}
\item
  Множеството от всички думи над азбуката $\Sigma$ ще означаваме със $\Sigma^\star$.
  Например, за $\Sigma = \{a,b\}$,
  \[\Sigma^\star = \{\varepsilon,a,b,aa,ab,ba,bb,aaa,aab,\dots\}.\]
  Обърнете внимание, че $\emptyset^\star = \{\varepsilon\}$.
% \item
%   {\bf Лексикографска наредба}
\end{itemize}

\subsection*{Операции върху думи}

\begin{itemize}
\item 
  \index{конкатенация}
  Операцията {\bf конкатенация} взима две думи $\alpha$ и $\beta$ и образува 
  новата дума $\alpha\cdot\beta$ като слепва двете думи.
  Например $aba\cdot bb = ababb$.
  Обърнете внимание, че в общия 
  случай $\alpha\cdot\beta \neq \beta\cdot\alpha$. 
  \marginpar{Често ще пишем $\alpha\beta$ вместо $\alpha\cdot\beta$}
  Можем да дадем формална индуктивна дефиниция на операцията конкатенация по
  дължината на думата $\beta$.
  \begin{itemize}
  \item 
    Ако $\abs{\beta} = 0$, то $\beta = \varepsilon$.
    Тогава $\alpha\cdot \varepsilon \df \alpha$.
  \item
    Ако $\abs{\beta} = n+1$, то $\beta = \gamma b$, $\abs{\gamma} = n$.
    Тогава $\alpha\cdot\beta \df (\alpha\cdot\gamma)b$.
  \end{itemize}
\item
  Друга често срещана операция върху думи е {\bf обръщането} на дума.
  Дефинираме думата $\alpha^R$ като обръщането на $\alpha$ по следния начин.
  \begin{itemize}
  \item 
    Ако $\abs{\alpha} = 0$, то $\alpha = \varepsilon$ и $\alpha^R \df \varepsilon$.
  \item
    Ако $\abs{\alpha} = n+1$, то $\alpha = a\beta$, където $\abs{\beta} = n$.
    Тогава $\alpha^R \df (\beta^R)a$.
  \end{itemize}
  Например, $reverse^R = esrever$.
\item
  \index{дума!префикс}
  \index{дума!суфикс}
  Казваме, че думата $\alpha$ е {\bf префикс} на думата $\beta$,
  ако съществува дума $\gamma$, такава че $\beta = \alpha\cdot\gamma$.
  $\alpha$ е {\bf суфикс} на $\beta$, ако $\beta = \gamma\cdot\alpha$, за някоя дума $\gamma$.
\item
  \marginpar{Обърнете внимание, че $\emptyset\cdot A = A\cdot\emptyset = \emptyset$}
  \marginpar{Също така, $\{\varepsilon\}\cdot A = A\cdot\{\varepsilon\} = A$}
  Нека $A$ и $B$ са множества от думи.
  Дефинираме конкатенацията на $A$ и $B$ като
  \[A\cdot B \df \{\alpha\cdot\beta \mid \alpha\in A\ \&\ \beta \in B\}.\]
\item
  Сега за едно множество от думи $A$, дефинираме $A^n$ индуктивно:
  \begin{align*}
    A^0 & \df \{\varepsilon\},\\
    A^{n+1} & \df A^n \cdot A.
  \end{align*}
  Ако $A = \{ab, ba\}$, то
  $A^0 = \{\varepsilon\}$, $A^1 = A$, $A^2 = \{abab, abba, baba, baab\}$.
  Ако $A = \{a,b\}$, то $A^n = \{\alpha \in \{a,b\}^\star \mid \abs{\alpha} = n\}$.
\item
  За едно множеството от думи $A$, дефинираме:
  \marginpar{Операцията $\star$ е известна като звезда на Клини}
  \marginpar{Обърнете внимание, че $\emptyset^\star = \{\varepsilon\}$}
  \begin{align*}
    A^\star & \df \bigcup_{n\geq 0} A^n\\
    & = A^0 \cup A^1 \cup A^2 \cup A^3 \cup \dots\\
    A^+ & \df A\cdot A^\star.
  \end{align*}
\end{itemize}

\begin{problem}
  Проверете:
  \begin{enumerate}[a)]
  \item 
    операцията конкатенация е {\em асоциативна}, т.е. за всеки три думи $\alpha$, $\beta$, $\gamma$,
    \[(\alpha\cdot\beta)\cdot\gamma = \alpha\cdot(\beta\cdot\gamma);\]
  \item
    за множествата от думи $A$, $B$ и $C$,
    \[(A\cdot B)\cdot C = A\cdot (B\cdot C);\]
  \item
    $\{\varepsilon\}^\star = \varepsilon$;
  \item
    за произволно множество от думи $A$,
    $A^\star = A^\star\cdot A^\star$ и $(A^\star)^\star = A^\star$;
  \item
    за произволни букви $a$ и $b$,
    $\{a,b\}^\star = \{a\}^\star\cdot(\{b\}\cdot\{a\}^\star)^\star$.
  \end{enumerate}
\end{problem}


\begin{problem}
  Докажете, че за всеки две думи $\alpha$ и $\beta$ е изпълено:
  \begin{enumerate}[a)]
  \item 
    $(\alpha\cdot\beta)^R = \beta^R\cdot\alpha^R$;
  \item
    $\alpha$ е префикс на $\beta$ точно тогава, когато $\alpha^R$ е суфикс на $\beta^R$;
  \item
    $(\alpha^R)^R = \alpha$;
  \item
    $(\alpha^n)^R = (\alpha^R)^n$, за всяко $n \geq 0$.
  \end{enumerate}
\end{problem}

\begin{problem}
  \marginpar{С други думи, ако $\varepsilon \not\in X$, то $Z = X^\star Y$ е най-малкото решение на уравнението $Z = XZ \cup Y$.}
  Нека $X, Y, Z \subseteq \Sigma^\star$ със свойството, че $Z = XZ \cup Y$.
  \begin{enumerate}[a)]
  \item 
    Докажете, че за всяко $n \in \Nat$, $X^nY \subseteq Z$.
    Заключете, че $X^\star Y \subseteq Z$.
  \item
    Да предположим, че $\varepsilon \not\in X$.
    Докажете, че за всяка дума $\omega \in Z$ е изпълнено, че $\omega \in X^\star Y$.
  \end{enumerate}
\end{problem}

\section*{Бележки}

Повечето книги в тази област започват с уводна глава, в която въвеждат понятията множества, релации и езици.
\begin{itemize}
\item 
  Глава 1 от \cite{rosen}.
\item
  Глава 1 от \cite{papadimitriou}.
\item
  За описанието на думи и азбуки следваме \cite[Глава 2]{kozen}.
\end{itemize}

%%% Local Variables: 
%%% mode: latex
%%% TeX-master: "EAI"
%%% End: 


\chapter{Регулярни езици и автомати}
\label{ch:regular}

\section{Автоматни езици}

\begin{definition}
  \mynote{Често ще пишем съкратено ДКА вместо детерминиран краен автомат.}
  Детерминиран краен автомат е петорка $\A = \FA$, където
  \begin{itemize}
  \item
    $\Sigma$ е азбука;
  \item
    $Q$ е крайно множество от състояния;
  \item
    $\delta:Q\times\Sigma\to Q$ е тотална функция, която ще наричаме
    \emph{функция на преходите};
  \item
    $\qstart\in Q$ е начално състояние;
  \item
    $F\subseteq Q$ е множеството от финални състояния
  \end{itemize}
\end{definition}
\index{автомат!детерминиран}

Нека имаме една дума $\alpha \in \Sigma^\star$, $\alpha = a_0a_1\cdots a_{n-1}$.
Казваме, че $\alpha$ се {\bf разпознава} от автомата $\A$, ако
съществува редица от състояния $q_0,q_1,q_2,\dots,q_n$, такива че:
\begin{itemize}
\item
  $q_0 = \qstart$, началното състояние на автомата;
\item
  $\delta(q_i,a_{i}) = q_{i+1}$, за всяко $i = 0, \dots, n-1$;
\item
  $q_n \in F$.
\end{itemize}

Казваме, че $\A$ {\bf разпознава} езика $L$, ако $\A$ разпознава точно думите от $L$, т.е.
$L = \{\alpha \in \Sigma^\star \mid \A\mbox{ разпознава }\alpha\}$.
Обикновено означаваме езика, който се разпознава от даден автомат $\A$ с $\L(\A)$.
\index{език!автоматен}
В такъв случай ще казваме, че езикът $L$ е {\bf автоматен}.

При дадена функция на преходите $\delta$,
често е удобно да разглеждаме функция $\delta^\star:Q\times\Sigma^\star \to Q$, която е дефинирана за всяко $q\in Q$ и $\alpha \in \Sigma^\star$ по следния начин:
% \mynote{Това е пример за индуктивна (рекурсивна) дефиниция по дължината на думата $\alpha$}
\begin{itemize}
\item
  Ако $\alpha = \varepsilon$, то $\delta^\star(q,\varepsilon) \df q$;
\item
  Ако $\alpha = \beta $, то $\delta^\star(q,\beta a) \df \delta(\delta^\star(q,\beta), a)$.
\end{itemize}
Лесно се съобразява, че една дума $\alpha$ се {\em разпознава} от автомата $\A$ точно тогава, когато $\delta^\star(\qstart,\alpha) \in F$.
Оттук следва, че
\begin{framed}
\[\L(\A) \df \{\ \alpha\in\Sigma^\star \mid \delta^\star(\qstart,\alpha) \in F\ \}.\]
\end{framed}

\mynote{Обърнете внимание, че $\delta(q,a) = \delta^\star(q,a)$ за $a\in\Sigma$}

\begin{proposition}
  \label{pr:dfa:delta-star}
  Нека $\A$ е ДКА. Тогава за всяко състояние $q$ и произволни думи $\alpha$ и $\beta$ е изпълнено, че
  \[\delta^\star(q,\alpha\beta) = \delta^\star(\delta^\star(q,\alpha),\beta).\]
\end{proposition}
\begin{hint}
  Индукция по дължината на $\beta$.

  \begin{itemize}
  \item
    $|\beta| = 0$, т.е. $\beta = \varepsilon$. Тогава за всяко състояние $q$ и произволна дума $\alpha$ имаме:
    \[\delta^\star(q, \alpha\varepsilon) = \delta^\star( \underbrace{\delta^\star(q, \alpha)}_{q}, \varepsilon).\]
  \item
    Да приемем, че твърдението е изпълнено за думи $\beta$ с дължина $n$.
  \item
    Нека $|\beta| = n+1$, т.е. $\beta = \gamma b$, където $|\gamma| = n$. Тогава за всяко състояние $q$ и произволна дума $\alpha$ имаме:
    \begin{align*}
      \delta^\star(q, \alpha\beta) & = \delta^\star(q, \alpha \gamma b)\\
                                   & = \delta( \delta^\star(q, \alpha\gamma),b) & \comment\text{от деф. на }\delta^\star\\
                                   & = \delta(\delta^\star(\underbrace{\delta^\star(q,\alpha)}_{p}, \gamma), b) & \comment{\text{от И.П. приложено за }\gamma}\\
                                   & = \delta( \delta^\star(p, \gamma), b) \\
                                   & = \delta^\star( p, \gamma b) & \comment\text{от деф. на }\delta^\star\\
                                   & = \delta^\star(\delta^\star(q, \alpha), \beta). & \comment{p = \delta^\star(q,\alpha)}
    \end{align*}
  \end{itemize}
\end{hint}

\begin{remark}
Формално погледнато, доказателството на \Proposition{dfa:delta-star} протича по следния начин:
  \begin{prooftree}
    \AxiomC{$P(0)$}
    \AxiomC{$(\forall n\in\Nat)[P(n) \implies P(n+1)]$}
    \BinaryInfC{$(\forall n \in \Nat)[P(n)]$,}
  \end{prooftree}
  където
  \[P(n) \df (\forall \beta\in\Sigma^n)(\forall \alpha\in\Sigma^\star)(\forall q\in Q)[\delta^\star(q,\alpha\beta) = \delta^\star(\delta^\star(q,\alpha),\beta)].\]
\end{remark}


\index{моментно описание}
\mynote{(На англ. {\em instantaneous description}). В случая въвеждането на това понятие не е напълно необходимо, но по-късно, когато въведем стекови автомати и машини на Тюринг, ще работим с такива моментни описания на изчисления и затова е добре още отначало да свикнем с него.}
{\bf Моментното описание} на изчисление с краен автомат представлява двойка от вида $(q,\alpha) \in Q\times\Sigma^\star$,
т.е. автоматът се намира в състояние $q$, а думата, която остава да се прочете е $\alpha$.
Удобно е да въведем бинарната релация $\vdash_\A$ над $Q\times\Sigma^\star$,
която ще ни казва как моментното описание на автомата $\A$ се променя след изпълнение на една стъпка:
\[(q,b\alpha) \vdash_\A (p,\alpha) \stackrel{\text{деф}}{\iff} \delta(q,b) = p.\]
Рефлексивното и транзитивно затваряне на $\vdash_\A$ ще означаваме с $\vdash^\star_\A$.
За да дадем по-удобна дефиниция на $\vdash^\star_\A$, първо ще дефинираме релацията $\vdash^n_\A$, която
определя работата на автомата $\A$ върху моментни описания за $n$ стъпки.
\mynote{Рефл. и транз. затваряне на една релация е разгледано в Глава \ref{ch:intro}.}

\begin{prooftree}
  \AxiomC{}
  \UnaryInfC{$(q,\alpha) \vdash^0_\A (q,\alpha)$}
\end{prooftree}

\begin{prooftree}
  \AxiomC{$(q,\alpha) \vdash^n_\A (r, b\beta)$}
  \AxiomC{$\delta(r,b) = p$}
  \BinaryInfC{$(q,\alpha) \vdash^{n+1}_\A (p,\beta)$}
\end{prooftree}
% \begin{itemize}
% \item 
%   $(q,\alpha) \vdash^0_\A (q,\alpha)$, защото за $0$ стъпки се случва нищо.
% \item
%   Нека $\delta(q,x) = q'$ и $(q',\alpha) \vdash^n_\A (p, \beta)$. Тогава
%   $(q,x\alpha) \vdash^{n+1}_\A (p,\beta)$, защото за $n+1$ стъпки първо правим една стъпка 
%   и отиваме в моментното описание $(q',\alpha)$ и след това правим още $n$ стъпки.
% \end{itemize}
Сега можем да дефинираме $\vdash^\star_\A$ като:
\[(q,\alpha) \vdash^\star_\A (p,\beta) \dff (\exists n\in\Nat)[(q,\alpha) \vdash^n_\A (p,\beta)].\]

\begin{problem}
  Докажете, че за произволно $\beta \in \Sigma^\star$,
  \[(q,\alpha\beta) \vdash^\star_\A (p, \beta) \iff \delta^\star(q,\alpha) = p.\]  
\end{problem}

Получаваме, че 
\[\L(\A) = \{\alpha\in\Sigma^\star \mid (\exists q \in F)[(\qstart,\alpha) \vdash^\star_\A(q,\varepsilon)]\}.\]


%%% Local Variables:
%%% mode: latex
%%% TeX-master: "../eai"
%%% End:

\section{Регулярни изрази и езици}

Да фиксираме една непразна азбука $\Sigma$.
\index{регулярен израз}
{\bf Регулярните изрази} $\mathbf{r}$ могат да се опишат със следната абстрактна граматика
\[\mathbf{r} ::= \bm{\emptyset}\ |\ \bm{\varepsilon}\ |\ \mathbf{a}\ |\ \bm{r}_1 \cdot \bm{r}_2\ |\ \bm{r}_1 + \bm{r_2}\ |\ \bm{r}^\star_1.\]

Регулярните изрази могат да се опишат и по следния начин:
\marginpar{Това е пример за индуктивна дефиниция}
\begin{itemize}
\item 
  Символите $\bm{\emptyset}$, $\bm{\varepsilon}$ са регулярни изрази;
\item
  за всяка буква $a \in \Sigma$, символът $\bm{a}$ е регулярен израз;
\item
  \marginpar{В литературата също се среща записът $(\bm{r}_1\ |\ \bm{r}_2)$ вместо $(\bm{r}_1 + \bm{r}_2)$}
  ако $\mathbf{r_1}$ и $\mathbf{r_2}$ са регулярни изрази, то думите $(\bm{r}_1 \cdot \bm{r}_2)$, $(\bm{r}_1 + \bm{r}_2)$ и $\bm{r}^\star_1$
  също са регулярни изрази;
\item
  Всеки регулярен израз е получен по някое от горните правила.
\end{itemize}

\index{език!регулярен}
\marginpar{Това е друг пример за индуктивна (рекурсивна) дефиниция.}
Сега ще дефинираме езиците, които се описват с регулярни изрази.
Тези езици се наричат {\bf регулярни}.
Това ще направим следвайки индуктивната дефиниция на регулярните изрази,
т.е. за всеки регулярен израз $\mathbf{r}$ ще определим език $\L(\mathbf{r})$.
\begin{itemize}
\item
  $\emptyset$ е регулярен език,
  който се описва от регулярния израз $\bm{\emptyset}$. Означаваме $\L(\bm{\emptyset}) = \emptyset$;
\item
  $\{\varepsilon\}$ е регулярен език,
  който се описва от регулярния израз $\bm{\varepsilon}$.
  Означаваме $\L(\bm{\varepsilon}) = \{\varepsilon\}$;
\item
  за всяка буква $a \in \Sigma$, $\{a\}$ е регулярен език,
  който се описва от регулярния израз $\mathbf{a}$.
  Означаваме $\L(\mathbf{a}) = \{a\}$;
\item
  Нека $L_1$ и $L_2$ са регулярни езици, т.е. съществуват регулярни изрази $\mathbf{r}_1$
  и $\mathbf{r}_2$, за които $\L(\mathbf{r}_1) = L_1$ и $\L(\mathbf{r}_2) = L_2$.
  Тогава:
  \begin{itemize}
  \item 
    \index{регулярни операции!обединение}
    $L_1 \cup L_2$ е регулярен език, който се описва с регулярния израз $(\mathbf{r}_1 + \mathbf{r}_2)$.
    Това означава, че $\L(\mathbf{r}_1) \cup \L(\mathbf{r}_2) = \L(\mathbf{r}_1 + \mathbf{r}_2)$.
  \item
    \index{регулярни операции!конкатенация}
    \marginpar{Тази операция се нарича конкатенация. Обикновено изпускаме знака $\cdot$}
    $L_1 \cdot L_2$ е регулярен език, който се описва с регулярния израз $(\mathbf{r}_1 \cdot \mathbf{r}_2)$.
    Това означава, че $\L(\mathbf{r}_1) \cdot \L(\mathbf{r}_2) = \L(\mathbf{r}_1 \cdot \mathbf{r}_2)$.
  \item
    \marginpar{Звезда на Клини}
    \index{регулярни операции!звезда на Клини}
    $L^\star_1$ е регулярен език, който се описва с регулярния израз $\mathbf{r}^\star_1$.
    Това означава, че $\L(\mathbf{r}_1)^\star = \L(\mathbf{r}^\star_1)$.
  \end{itemize}
\end{itemize}

\begin{remark}
  Ние знаем, че:
  \begin{itemize}
  \item
    Всеки регулярен израз представлява крайна дума над крайна азбука.
    Това означава, че множеството от всички регулярни изрази е изброимо безкрайно.
    Оттук следва, че всички регулярни езици образуват изброимо безкрайно множество.
  \item 
    Понеже $\Sigma$ е крайна азбука, то $\Sigma^\star$ е изброимо безкрайно множество;
  \item
    Един език над азбуката $\Sigma$ представлява елемент на $\Ps(\Sigma^\star)$.
    Това означава, че всички езици над азбуката $\Sigma$ представляват неизброимо безкрайно множество.
  \end{itemize}
  От всичко това следва, че има езици, които не са регулярни.
  По-нататък ще видим примери за такива езици.
\end{remark}

\begin{example}
  \marginpar{В \cite[стр. 73]{sipser1} е показан алгоритъм, за който по един автомат може да се получи регулярен израз описващ езика на автомата. Ние няма да разглеждаме този алгоритъм. }
  Нека да построим регулярни изрази за всеки от езиците от \Ex{automata-pictures}.
  \begin{enumerate}[a)]
  \item 
    Нека $\mathbf{r} \equiv \mathbf{(a+b)^\star bab(a+b)^\star}$. Тогава
    \[\L(\mathbf{r}) = \{\omega \in \{a,b\}^\star \mid \omega \text{ съдържа } bab\}.\]
  \item
    Нека $\mathbf{r} \equiv \mathbf{b^\star ab^\star a(a+b)^\star}$. Тогава
    \[\L(\mathbf{r}) = \{\omega \in \{a,b\}^\star \mid N_a(\omega) \geq 2\}.\]
  \item
    Нека $\mathbf{r} \equiv \mathbf{(b^\star abb^\star)^\star}$. Тогава
    \[\L(\mathbf{r}) = \{\omega \in \{a,b\}^\star \mid \text{ всяко $a$ в $\omega$ се следва от поне едно $b$}\}.\]
  \item
    Нека $\mathbf{r} \equiv \mathbf{(b^\star ab^\star ab^\star ab^\star)^\star}$. Тогава
    \[\L(\mathbf{r}) = \{\omega \in \{a,b\}^\star \mid N_a(\omega) \equiv 0 \bmod 3\}.\]
  \end{enumerate}
\end{example}

\begin{problem}
  \marginpar{Когато пишем $\bm{r} \equiv \bm{s}$ имаме предвид, че $\L(\bf{r}) = \L(\bm{s})$.}
  За произволни регулярни изрази $\bm{r}$ и $\bm{s}$, проверете дали са изпълнени следните равенства:
  \begin{enumerate}[a)]
  \item 
    $\bm{r + s} \equiv \bm{s + r}$;
  \item
    $\bm{(\varepsilon + r)^\star} \equiv \bm{r^\star}$;
  \item
    $\bm{\emptyset^\star} \equiv \bm{\varepsilon}$;
  \item
    $\bm{(r^\star s^\star)^\star} \equiv \bm{(r+s)^\star}$;
  \item
    $\bm{(r^\star)^\star} \equiv \bm{r^\star}$;
  \item
    $\bm{(rs + r)^\star r} \equiv \bm{r(sr+r)^\star}$;
  \item
    $\bm{s(rs+s)^\star r} \equiv \bm{rr^\star s(rr^\star s)^\star}$;
  \item
    $\bm{(r+s)^\star} \equiv \bm{r^\star + s^\star}$;
  \item
    $\bm{(r+s)^\star s} \equiv \bm{(r^\star s)^\star}$;
  \item
    $\bm{(rs + r)^\star rs} \equiv \bm{(rr^\star s)^\star}$;
  \item
    $\bm{\emptyset^\star} \equiv \bm{\varepsilon^\star}$;
  \end{enumerate}
\end{problem}



%%% Local Variables: 
%%% mode: latex
%%% TeX-master: "../eai"
%%% End: 

\section{Недетерминирани крайни автомати}
\index{автомат!недетерминиран}
\begin{dfn}
  \marginpar{Въведени от Рабин и Скот \cite{rabin-scott}}
  \marginpar{За по-голяма яснота, често ще означаваме с $\N$ недетерминирани автомати}
  Недетерминиран краен автомат представлява
  \[\N = \NFA,\]
  \begin{itemize}
  \item
    $Q$ е крайно множество от състояния;
  \item
    $\Sigma$ е крайна азбука;
  \item
    $\Delta: Q\times\Sigma \to \Ps(Q)$ е функцията на преходите.
    \marginpar{Да напомним, че $\Ps(Q) = \{R\mid R\subseteq Q\}$, $\abs{\Ps(Q)} = 2^{\abs{Q}}$}
    Да обърнем внимание, че е възможно за някоя двойка $(q,a)$ да няма нито един преход в автомата.
    Това е възможно, когато $\Delta(q,a) = \emptyset$;
  \item
    $\qstart \in Q$ е началното състояние;
  \item
    $F\subseteq Q$ е множеството от финални състояния.
  \end{itemize}
\end{dfn}

Удобно е да разширим функцията на преходите $\Delta: Q\times\Sigma \to \Ps(Q)$ 
до функцията $\Delta^\star: \Ps(Q)\times\Sigma^\star \to \Ps(Q)$,
която дефинираме по следния начин:
\marginpar{Обърнете внимание, че $\Delta^\star(R,a) = \bigcup_{r\in R}\Delta(r,a)$.}
\begin{itemize}
\item 
  $\Delta^\star(R, \varepsilon) = R$, за произволно $R \subseteq Q$;
\item
  $\Delta^\star(R, \alpha x) = \bigcup_{p \in \Delta^\star(R,\alpha)} \Delta(p, x)$, за произволни $x \in \Sigma$, $\alpha \in \Sigma^\star$, $q\in Q$.
\end{itemize}

\begin{framed}
  \[\L(\N) \df \{\omega \in \Sigma^\star \mid \Delta^\star(\{\qstart\},\omega) \cap F \neq \emptyset \}.\]
\end{framed}

\begin{prop}
  За всеки две думи $\alpha,\beta \in \Sigma^\star$ и всяко $R \subseteq Q$,
  \[ \Delta^\star(R, \alpha\beta) = \Delta^\star( \Delta^\star(R,\alpha),\beta).\]
\end{prop}
\begin{proof}
  Отново индукция по дължината на $\beta$.
  \begin{itemize}
  \item
    Нека $\beta = \varepsilon$. Тогава:
    \begin{align*}
      \Delta^\star(R,\alpha\varepsilon) & = \Delta^\star(R,\alpha) \\
                                        & = \Delta^\star( \Delta^\star(R,\alpha), \varepsilon). & \comment\text{деф. на }\Delta^\star
    \end{align*}
  \item
    Да приемем, че твърдението е вярно за думи с дължина $n$.
  \item
    Нека $\beta = \gamma b$, където $|\gamma| = n$.
    \begin{align*}
      \Delta^\star(R, \alpha\gamma b) & = \bigcup_{q \in \Delta^\star(R,\alpha\gamma)} \Delta(q, b) & \comment\text{от деф. на }\Delta^\star\\
                                      & = \bigcup_{q \in \Delta^\star(\Delta^\star(R,\alpha),\gamma)} \Delta(q,b) & \comment\text{от И.П.}\\
                                      & = \Delta^\star( \Delta^\star(R,\alpha), \gamma b) & \comment\text{от деф. на }\Delta^\star.
    \end{align*}
    
  \end{itemize}
\end{proof}

\begin{problem}
  Докажете, че за произволни $R_i \subseteq Q$, където $i < k$, е изпълнено, че:
  \[\Delta^\star( \bigcup_{i<k} R_i, \alpha) = \bigcup_{i<k} \Delta^\star( R_i, \alpha).\]
\end{problem}

И тук е удобно да въведем бинарната релация $\vdash_\N$ над $Q\times\Sigma^\star$,
която ще ни казва как моментното описание на автомата $\N$ се променя след изпълнение на една стъпка:
\[(q,x\alpha) \vdash_\N (p,\alpha), \text{ ако } p \in \Delta(q,x).\]
\marginpar{Рефл. и транз. затваряне на една релация е разгледано в Глава \ref{ch:intro}}
Рефлексивното и транзитивно затваряне на $\vdash_\N$ ще означаваме с $\vdash^\star_\N$.
За да дадем точна дефиниция на $\vdash^\star_\N$, първо ще дефинираме релацията $\vdash^n_\N$, която
определя работата на автомата $\N$ за $n$ стъпки.
\begin{itemize}
\item 
  $(q,\alpha) \vdash^0_\N (q,\alpha)$, защото за $0$ стъпки се случва нищо.
\item
  Нека $\Delta(q,x) \ni q'$ и $(q',\alpha) \vdash^n_\N (p, \beta)$. Тогава
  $(q,x\alpha) \vdash^{n+1}_\N (p,\beta)$, защото за $n+1$ стъпки първо правим една стъпка 
  и отиваме в моментното описание $(q',\alpha)$ и след това правим още $n$ стъпки.
\end{itemize}
Сега можем да дефинираме $\vdash^\star_\N$ като:
\[(q,\alpha) \vdash^\star_\N (p,\beta) \dff (\exists n\in\Nat)[(q,\alpha) \vdash^n_\N (p,\beta)].\]
Друг начин да дефинираме релацията $\vdash^\star_\N$ е следния:
\[(q,\alpha\beta) \vdash^\star_\N (p, \beta) \iff p \in \Delta^\star(\{q\},\alpha).\]
Получаваме, че 
\[\L(\N) = \{\alpha\in\Sigma^\star \mid (\exists q \in F)[(\qstart,\alpha) \vdash^\star_\N (q,\varepsilon)]\}.\]


\begin{framed}
\begin{thm}[Рабин-Скот \cite{rabin-scott}]
  За всеки недетерминиран краен автомат $\N$ съществува еквивалентен на него детерминиран краен автомат $\D$, т.е. $\L(\N) = \L(\D)$.
\end{thm}
\end{framed}
\begin{hint}
  Нека $\N = \NFA$. Ще построим детерминиран автомат
  \[\D = (Q',\Sigma,\delta,\qstart',F'),\]
  за който $\L(\N) = \L(\D)$.
  Конструкцията е следната:
  \marginpar{Да отбележим, че детерминираният автомат $\D$ има не повече от $2^{\abs{Q}}$ на брой състояния $Q'$}
  \begin{itemize}
  \item
    $Q' = \Ps(Q)$. Някои от тези състояния може да са недостижими и следователно да са излишни, но в общия случай трябва да имаме
    всички подмножества на $Q$.
  \item
    За произволна буква $a\in\Sigma$ и произволно $R \subseteq Q$,
    \begin{align*}
      \delta(R,a) & = \{q\in Q\mid (\exists r\in R)[q\in\Delta(r,a)]\}\\
                  & = \bigcup_{r\in R}\Delta(r,a)\\
                  & = \Delta^\star(R,a).
    \end{align*}
  \item
    $\qstart' = \{\qstart\}$;
  \item
    $F' = \{R \subseteq Q \mid R\cap F \neq \emptyset\}$.
  \end{itemize}
  Ще докажем с индукция по дължината на думата $\alpha$, че
  \begin{equation}
    \label{eq:6}
    (\forall \alpha\in\Sigma^\star)(\forall R \subseteq Q)[\ \Delta^\star(R,\alpha) = \delta^\star(R,\alpha)\ ].
  \end{equation}

  \begin{itemize}
  \item
    Ако $|\alpha| = 0$, т.е. $\alpha = \varepsilon$, то е ясно от дефиницията на $\Delta^\star$ и $\delta^\star$.
  \item
    Да приемем, че (\ref{eq:6}) е изпълнено за думи $\alpha$ с дължина $n$.
  \item
    Нека сега $\alpha$ има дължина $n+1$, т.е. $\alpha = \beta a$, където $|\beta| =n$ и $a \in \Sigma$.
    Тогава:
    \begin{align*}
      \Delta^\star(R, \beta a) & = \Delta^\star(\Delta^\star(R,\beta), a) & \comment\text{деф. на }\Delta^\star\\
                               & = \Delta^\star(\delta^\star(R,\beta), a) & \comment\text{от И.П.}\\
                               & = \delta( \delta^\star(R, \beta), a) & \comment\text{деф. на }\delta\\
                               & = \delta^\star( R, \beta a) & \comment\text{деф. на }\delta^\star
    \end{align*}
  \end{itemize}
  Сега вече е лесно да съобразим, че
  \begin{align*}
    \omega \in \L(\D) & \iff \delta^\star(\{\qstart\},\omega) \in F' & \comment\text{деф. на }\L(\D)\\
                      & \iff \delta^\star(\{\qstart\},\omega) \cap F \neq \emptyset & \comment\text{деф. на }F'\\
                      & \iff \Delta^\star(\{\qstart\},\omega) \cap F \neq \emptyset & \comment\text{от (\ref{eq:6})}\\
                      & \iff \omega \in \L(\N) & \comment\text{деф. на }\L(\N).
  \end{align*}
\end{hint}

\begin{problem}
  За всеки НКА $\N$ съществува НКА $\N'$ с едно финално състояние, 
  за който $\L(\N) = \L(\N')$.
\end{problem}
\begin{hint}
  Вместо формална конструкция, да разгледаме един пример, който илюстрира идеята.
  \begin{figure}[H]
    \begin{subfigure}[b]{0.3\textwidth}
      \begin{tikzpicture}[framed,->,>=stealth,thick,node distance=45pt]
        \tikzstyle{every state}=[circle,minimum size=20pt,auto]
        \node[initial below,state]      (1) {$q_0$};
        \node[state,accepting]     [above right of=1] (2) {$q_1$};
        \node[state,accepting]     [below right of=1] (3) {$q_2$};
        \path
        (1) edge [bend left=15] node  [above] {$a$} (2)
        (2) edge [bend left=15] node  [right] {$b$} (1)
        % (2) edge [loop above] node  [above] {$a$} (2)
        (2) edge [bend left=15] node  [right] {$a$} (3)
        (3) edge [bend left=15] node  [below] {$a$} (1)
        (3) edge [loop below] node  [right] {$b$} (3);
        % (1) edge [bend right=15] node [below] {$b$} (3);
      \end{tikzpicture}
      \caption{автомат $\N$}
    \end{subfigure}
    \begin{subfigure}[b]{0.4\textwidth}
      \begin{tikzpicture}[framed,->,>=stealth,thick,node distance=45pt]
        \tikzstyle{every state}=[circle,minimum size=20pt,auto]
        \node[initial below,state]                        (1) {$q_0$};
        \node[state]     [above right of=1]         (2) {$q_1$};
        \node[state]     [below right of=1]         (3) {$q_2$};
        \node[state,accepting]     [right=3cm of 1] (4) {$f$};
        \path
        (1) edge [bend left=15] node  [above] {$a$} (2)
        % (2) edge [loop above] node  [above] {$a$} (2)
        (2) edge [bend left=15] node  [right] {$b$} (1)
        (2) edge [bend left=15] node  [right] {$a$} (3)
        (3) edge [loop below] node  [right] {$b$} (3)
        (3) edge [bend left=15] node  [below] {$a$} (1)
        (1) edge [dashed,bend left=15] node  [above] {$a$} (4)
        (2) edge [dashed,bend left=15] node  [above] {$a$} (4)
        (3) edge [dashed,bend right=15] node  [below] {$b$} (4);
        % (1) edge [bend right=15] node [below] {$b$} (3);
      \end{tikzpicture}
    \caption{автомат $\N'$, $\L(\N') = \L(\N)$}
  \end{subfigure}
\end{figure}  
За произволен автомат $\N$, формулирайте точно конструкцията на $\N'$ с едно финално състояние и докажете, че наистина $\L(\N) = \L(\N')$.
Обърнете внимание, че примера показва, че е възможно $\N$ да е детерминиран автомат, но полученият $\N'$ да бъде недетерминиран.
\end{hint}

\begin{problem}
  \marginpar{Това означава, че автоматните езици са затворени относно операцията $\texttt{rev}$. По-късно ще видим, че можем да дадем и друго доказателство на това твърдение, като направим индукция по построението на регулярните езици.}
  Докажете, че ако $L$ е автоматен език, то $L^{rev} = \{\omega^{rev} \mid \omega \in L\}$
  също е автоматен.
\end{problem}
% \begin{hint}
%   Нека $\A = \FA$, $L = \L(\A)$, е само с едно финално състояние, т.е. $F = \{\qaccept\}$.
%   Разгледайте недетерминистичния краен автомат $\N = \pair{\Sigma,Q,\qstart',\Delta,F'}$, където
%   \begin{itemize}
%   \item
%     $\Delta(q,a) = \{p \in Q \mid \delta(p,a) = q\}$;
%   \item
%     $\qstart' = \qaccept$;
%   \item
%     $F' = \{\qstart\}$;
%   \end{itemize}

% \end{hint}


\begin{lemma}
  \label{lem:automata-basic}
  Съществува детерминистичен автомат $\A = \FA$, който разпознава езика $L$, където
  \begin{itemize}
  \item
    $L = \emptyset$,
  \item
    $L = \{\varepsilon\}$, или
  \item
    $L = \{a\}$, за произволна буква $a\in\Sigma$.
  \end{itemize}
\end{lemma}
\begin{hint}
  \begin{figure}[H]
    \begin{subfigure}[b]{0.2\textwidth}
      \label{subf:a1}
      \begin{tikzpicture}[->,>=stealth,thick,node distance=35pt]
        \tikzstyle{every state}=[circle,minimum size=15pt,auto]
        \node[initial below,state]      (1) {$q_0$};
      \end{tikzpicture}
      \caption{$L(\A) = \emptyset$}
    \end{subfigure}
    \qquad
    \begin{subfigure}[b]{0.2\textwidth}
      \begin{tikzpicture}[->,>=stealth,thick,node distance=35pt]
        \tikzstyle{every state}=[circle,minimum size=15pt,auto]
        \node[initial below,state,accepting]      (1) {$q_0$};
      \end{tikzpicture}
      \caption{$\L(\A) = \{\varepsilon\}$}
    \end{subfigure}
    \qquad
    \begin{subfigure}[b]{0.3\textwidth}
      \begin{tikzpicture}[->,>=stealth,thick,node distance=45pt]
        \tikzstyle{every state}=[circle,minimum size=15pt,auto]
        \node[initial below,state]      (1)              {$q_0$};
        \node[accepting,state]    (2) [right of=1] {$q_1$};
        \path 
        (1) edge  node [above] {$a$} (2);
      \end{tikzpicture}
      \caption{$\L(\A) = \{a\}$}
    \end{subfigure}
  \end{figure}
\end{hint}

\begin{lemma}
  \label{lem:concat}
  Класът на автоматните езици е затворен относно операцията {\em конкатенация}.
  Това означава, че ако $L_1$ и $L_2$ са два произволни автоматни езика, то $L_1\cdot L_2$
  също е автоматен език.
\end{lemma}
\begin{proof}
  Нека са дадени детерминистичните автомати:
  \begin{itemize}
  \item
    $\A_1 = \pair{\Sigma,Q_1,\delta_1,\qstart',F_1}$, където $\L(\A_1) = L_1$;
  \item
    $\A_2 = \pair{\Sigma,Q_2,\delta_2,\qstart'', F_2}$, където $\L(\A_2) = L_2$.
  \end{itemize}
  Ще дефинираме автомата $\N = \NFA$ по такъв начин, че
  \[\L(\N) = L_1\cdot L_2 = \L(\A_1)\cdot\L(\A_2).\]
  \begin{itemize}
  \item
    $Q = Q_1 \cup Q_2$;
  \item
    $\qstart = \qstart''$;
  \item
    $F = 
    \begin{cases}
      F_1 \cup F_2, & \text{ ако } \qstart'' \in F_2\\
      F_2,          & \text{ иначе}.
    \end{cases}$
  \item 
    $\Delta(q,a) = 
    \begin{cases}
      \{\delta_1(q,a)\},                      & \text{ ако }q\in Q_1\setminus F_1\ \&\ a\in\Sigma\\
      \{\delta_1(q,a), \delta_2(\qstart'',a)\}, & \text{ ако }q \in F_1\ \&\ a\in\Sigma\\
      \{\delta_2(q,a)\},                      & \text{ ако }q\in Q_2\ \&\ a\in\Sigma.
    \end{cases}$
  \end{itemize}

  % \begin{enumerate}[(1)]
  % \item 
  %   $(\forall q,p \in Q_1)[(q, \alpha\beta) \vdash^\star_{\A_1} (p, \beta) \iff (q, \alpha\beta) \vdash^\star_{\N} (p, \beta)]$;
  % \item
  %   $(\forall q,p \in Q_2)[ (q,\alpha\beta) \vdash^\star_{\A_2} (p, \beta) \implies (q,\alpha\beta) \vdash^\star_\N (p, \beta)]$;
  % % \item
  % %   Нека $q \in Q_1$ и $p \in Q_2$ са произволни състояния, за които $(q,\omega) \vdash^\star_\N (p,\varepsilon)$.
  % %   Тогава съществува $f_1 \in F_1$ и $\omega = \omega_1 b\omega_2$, такива че:
  % %   $(q,\omega_1\omega_2) \vdash^\star_{\A_1} (f_1,b\omega_2) \vdash_\N (r, \omega_2) \vdash^\star_{\A_2} (p,\varepsilon)$.
  % \end{enumerate}
  % \marginpar{\writedown Разгледайте и случая, когато $\beta = \varepsilon$!}
  % Нека $\alpha \in \L(\A_1)$ и $\beta = b\gamma \in \L(\A_2)$.
  % Тогава имаме, че:
  % \begin{enumerate}[(a)]
  % \item
  %   $(\qstart',\alpha) \vdash^\star_{\A_1}(f_1,\varepsilon)$, за някое $f_1 \in F_1$;
  % \item
  %   $(\qstart'',b\gamma) \vdash_{\A_2} (p, \gamma) \vdash^\star_{\A_2}(f_2,\varepsilon)$, за някое $p \in Q_2$ и $f_2 \in F_2$.
  % \end{enumerate}
  
  % \begin{align*}
  %   (q,\alpha b\gamma) & \vdash^\star_\N(f_1,b\gamma) & \comment\text{от (1) и (а) }\\
  %                      & \vdash_\N(p,\gamma) & \comment\text{от деф. на $\N$ и (б)}\\
  %                      & \vdash^\star_\N(f_2,\varepsilon) & \comment\text{от (2) и (б)}.
  % \end{align*}

  % Оттук заключаваме, че $\L(\A_1) \cdot \L(\A_2) \subseteq \L(\N)$.

  % Нека сега $\alpha \in \L(\N)$, т.е.
  % $(\qstart,\alpha) \vdash^\star_\N (p, \varepsilon)$, за някое $p \in F$.
  % Нека $p \in F_2$.
  % За $\alpha = a_0a_1\cdots a_{n-1}$, да разгледаме редицата $q_0,q_1,\dots,q_n$,
  % която отговаря на някое от успешните изчисления на $\N$ върху $\alpha$, завършващи в $p$, т.е.
  % \begin{itemize}
  % \item
  %   $q_0 = \qstart$;
  % \item
  %   $q_{i+1} \in \Delta(q_i,a_i)$;
  % \item
  %   $q_n = p \in F_2$.
  % \end{itemize}
  % От конструкцията на $\N$ следва, че редицата $(q_i)^n_{i=0}$ може да се разбие на две части:
  % $q_i \in Q_1$ за $0 \leq i \leq l$ и $q_i \in Q_2$ за $l < i \leq n$.
  % Тогава от конструкцията на $\N$ следва, че:
  % \[(q_0,\alpha) \vdash^\star_{\A_1} ( q_l, a_la_{l+1}\cdots a_n) \vdash_\N (q_{l+1},a_{l+1}\cdots a_n) \vdash^\star_{\A_2} (q_n,\varepsilon).\]
  % От конструкцията на $\N$ следва, че $q_l \in F_1$ и щом $q_{l+1} \in Q_2$, то $\delta_2(\qstart'',a_l) = q_{l+1}$.
  % Оттук заключаваме, че $a_0a_1\cdots a_l \in \L(\A_1)$ и $a_{l+1}\cdots a_{n-1} \in \L(\A_2)$.

  
  Първо ще докажем, че $\L(\A_1)\cdot\L(\A_2) \subseteq \L(\N)$.
  За целта, нека разгледаме думата $\alpha \in \L(\A_1)$, където $\alpha = a_0a_1\cdots a_{l-1}$, и нека редицата $(q_i)^n_{i=0}$ описва приемащото изчисление на $\A_1$ върху думата $\alpha$.
  Това означава, че:
  \begin{itemize}
  \item
    $q_0 = \qstart'$;
  \item
    $q_{i+1} = \delta_1(q_i,a_i)$ за $i < n$;
  \item
    $q_n \in F_1$.
  \end{itemize}  
  Също така, нека разгледаме думата $\beta \in \L(\A_2)$, където $\beta = b_0b_1\cdots b_{m-1}$, и нека редицата $(p_i)^m_{i=0}$ описва приемащото изчисление на $\A_2$ върху думата $\beta$.
  Това означава, че:
  \begin{itemize}
  \item
    $p_0 = \qstart''$;
  \item
    $p_{i+1} = \delta_2(p_i,b_i)$ за $i < m$;
  \item
    $p_m \in F_2$.
  \end{itemize}  
  От конструкцията на $\N$ се вижда лесно, че $(q_i)^l_{i=0}$ описва изчисление на $\N$ върху $\alpha$ и
  $(p_i)^{m}_{i=0}$ описва изчисление на $\N$ върху $\beta$.
  Това означава, че:
  \begin{itemize}
  \item
    $q_{i+1} \in \Delta(q_i,a_i)$ за $i < n$;
  \item
    $p_{i+1} \in \Delta(p_i,b_i)$ за $i < m$.
  \end{itemize}
  Тогава:
  \[(q_0,\alpha\beta) \vdash^\star_\N (q_l,\beta)\text{ и } (p_0, \underbrace{b_0b_1\cdots b_{m-1}}_{\beta}) \vdash_\N (p_1,b_1\cdots b_{m-1}) \vdash^\star_{\N} (p_m,\varepsilon).\]
  Понеже $q_l \in F_1$, а $p_0 = \qstart''$, то от конструкцията на $\N$ следва, че
  \[(q_l, \underbrace{b_0b_1\cdots b_{m-1}}_{\beta}) \vdash_\N (p_1,b_1\cdots b_{m-1}),\]
  защото $\delta_2(\qstart'',b_0) \in \Delta(q_l,b_0)$.
  
  Обединявайки всичко това, получаваме, че:
  \begin{align*}
    (\qstart, \alpha\beta) & \vdash^\star_\N (q_l,\beta)\\
                           & \vdash_\N (p_1,b_1\cdots b_{m-1})\\
                           & \vdash^\star_\N(p_m,\varepsilon).
  \end{align*}
  Понеже $p_m \in F_2$, то $\alpha\beta \in \L(\N)$.
  

  Сега ще докажем, че $\L(\N) \subseteq \L(\A_1) \cdot \L(\A_2)$.
  За целта, нека разгледаме думата $\omega \in \L(\N)$, където $\omega = a_0a_1\cdots a_{n-1}$.
  Да разгледаме редицата от състояния $(q_i)^{n}_{i=0}$, която описва едно приемащо изчисление на $\N$ върху $\omega$.
  \marginpar{Възможно е да има и други редици от състояния $(p_i)^{n}_{i=0}$, които да описват приемащи изчисления на $\N$ върху $\omega$.}
  Това означава, че:
  \begin{itemize}
  \item
    $q_0 = \qstart$;
  \item
    $q_{i+1} \in \Delta(q_i,a_i)$ за $i < n$;
  \item
    $q_n \in F$.
  \end{itemize}

  \marginpar{Ако $q_n \in F_1$, то според конструкцията на $\N$, $\varepsilon \in \L(\A_2)$ и всяко състояние от $(q_i)^{n}_{i=0}$ принадлежи на $Q_1$ и оттам $\omega \in \L(\A_1)$.}
  Интересният случай е когато $q_n \in F_2$.
  Според конструкцията на $\N$, можем да разбием редицата от състояния $(q_i)^n_{i=0}$ на две непразни подредици:
  \begin{itemize}
  \item
    $(q_{i})^{l}_{i=0}$ - тези които са от $Q_1$,
  \item
    $(q_i)^{n}_{i=l+1}$ - тези, които са от $Q_2$.
  \end{itemize}
  Нека $\alpha = a_0a_1\cdots a_{l-1}$ и $\beta = a_la_{l+1}\cdots a_{n-1}$.
  Ясно е, че:
  \[(q_0,\alpha\beta) \vdash^\star_\N (q_{l}, a_{l}a_{l+1}\cdots a_{n-1}) \vdash_\N (q_{l+1},a_{l+1}\cdots a_{n-1}) \vdash^\star_\N (q_n,\varepsilon).\]
  От конструкцията на $\N$ следва, че редицата от състояния $(q_i)^{l}_{i=0}$ описва изчислението на $\A_1$ върху $\alpha$.
  \marginpar{Това е единственият начин да направим преход от състояние на $Q_1$ към състояние на $Q_2$.}
  Също така от конструкцията следва, че щом $q_{l+1} \in \Delta(q_l,a_l)$, то $q_l \in F_1$ и $\delta_2(\qstart'',a_l) = q_{l+1}$. Заключаваме, че:
  \begin{itemize}
  \item
    $(q_0, \alpha) \vdash^\star_{\A_1} (q_l,\varepsilon)$.
    Понеже $q_0 = \qstart'$ и $q_l \in F_1$, то $\alpha \in \L(\A_1)$.
  \item
    $(\qstart'', \beta) \vdash^\star_{\A_2} (q_n,\varepsilon)$.
    Понеже $q_n \in F_2$, то $\beta \in \L(\A_2)$.
  \end{itemize}
\end{proof}

\begin{figure}[H]
  \center
  \begin{subfigure}[b]{0.3\textwidth}
    \label{subf:a1}
    \begin{tikzpicture}[framed,->,>=stealth,thick,node distance=45pt]
      \tikzstyle{every state}=[circle,minimum size=15pt,auto]
      \node[initial,state,accepting]      (1) {$q'_0$};
      \node[state]                        (2) [right of=1] {$q_1$};
      \node[state]                        (3) [above right of=2] {$q_2$};
      \node[state,accepting]              (4) [below right of=2] {$q_3$};
      \path
      (1) edge [loop above] node [above] {$b$} (1)
      (1) edge node [above] {$a,b$} (2)
      (2) edge node [above] {$a$} (3)
      (2) edge node [below] {$b$} (4)
      (3) edge [bend right=30] node [above] {$a$} (1)
      (4) edge [bend right=15] node [right] {$a$} (3)
      (4) edge [bend left=30] node [below] {$b$} (1);
    \end{tikzpicture}
    \caption{автомат $\A_1$}
  \end{subfigure}
  \qquad
  \qquad
  \qquad
  \begin{subfigure}[b]{0.3\textwidth}
    \begin{tikzpicture}[framed,->,>=stealth,thick,node distance=45pt]
      \tikzstyle{every state}=[circle,minimum size=15pt,auto]
      \node[initial,state]                (1) {$q''_0$};
      \node[state]     [above right of=1] (2) {$q_4$};
      \node[state,accepting]     [below right of=1] (3) {$q_5$};
      \path
      (1) edge [bend left=15] node  [above] {$a$} (2)
      (2) edge [bend left=15] node  [right] {$a$} (3)
      (3) edge [loop right]  node [right] {$a,b$} (3)
      (1) edge [bend right=15] node [below] {$b$} (3);
    \end{tikzpicture}
    \caption{автомат $\A_2$}
  \end{subfigure}
\end{figure}

\begin{example}
    За да построим автомат, който разпознава конкатенацията на $\L(\N_1)$ и $\L(\N_2)$,
    трябва да свържем финалните състояния на $\N_1$ с изходящите от $s_2$ състояния на $\N_2$.
    
    \begin{figure}[H]
      \center
      % \begin{subfigure}[b]{0.3\textwidth}
      \begin{tikzpicture}[framed,->,>=stealth,thick,node distance=2cm]
        \tikzstyle{every state}=[circle,minimum size=15pt,auto]
        \node[initial,state]                      (1) {$s_1$};
        \node[state] [right of=1]                 (2) {$q_1$};
        \node[state] [above right of=2]           (3) {$q_2$};
        \node[state] [below right of=2]           (4) {$q_3$};
        \node[state] [right=4cm of 1]             (5) {$s_2$};
        \node[state] [above right of=5]           (6) {$q_4$};
        \node[state,accepting] [below right of=5] (7) {$q_5$};
        \path
        (1) edge node [above]                         {$a$} (2)
        (2) edge node [above]                         {$a$} (3)
        (2) edge node [below]                         {$b$} (4)
        (3) edge [bend right=15] node [above]         {$a$} (1)
        (4) edge [bend left=15] node [below]          {$b$} (1)
        (5) edge [bend left=15] node [below]          {$a$} (6)
        (6) edge [bend left=15] node [right]          {$a$} (7)
        (5) edge [bend right=15] node [above]         {$b$} (7)
        (1) edge [dashed, bend left=45] node [above]  {$a$} (6)
        (1) edge [dashed, bend right=45] node [below] {$b$} (7)
        (4) edge [dashed, bend left=45] node [above]  {$a$} (6)
        (4) edge [dashed, bend left=10] node [above]  {$b$} (7);
      \end{tikzpicture}
      \caption{$\L(\N) = \L(\A_1)\cdot\L(\A_2)$}
  \end{figure}  
  Обърнете внимание, че $\A_1$ и $\A_2$ са детерминирани автомати, но $\N$ е недетерминиран.
  Също така, в този пример се оказва, че вече $q''_0$ е недостижимо състояние, но в общия случай не можем да 
  го премахнем, защото може да има преходи влизащи в $q''_0$.
\end{example}

\begin{lemma}
  \label{lem:union}
  \marginpar{Второ доказателство на това твърдение.}
  Класът от автоматните езици е затворен относно операцията {\em обединение}.
\end{lemma}
\begin{hint}
  Нека са дадени детерминистичните автомати:
  \begin{itemize}
  \item 
    $\A_1 = \pair{\Sigma,Q_1,\delta_1,\qstart',F_1}$, като $L(\A_1) = L_1$;
  \item
    $\A_2=\pair{\Sigma,Q_2,\delta_2,\qstart'',F_2}$, като $L(\A_2) = L_2$.
  \end{itemize}
  Ще дефинираме автомата $\N=\NFA$, така че
  \[L(\N) = L(\A_1) \cup L(\A_2).\]
  \begin{itemize}
  \item 
    $Q = Q_1 \cup Q_2 \cup \{\qstart\}$, където $\qstart\not\in Q_1\cup Q_2$;
  \item
    $F = 
    \begin{cases}
      F_1 \cup F_2 \cup \{\qstart\}, & \text{ ако } \qstart' \in F_1 \vee \qstart'' \in F_2\\
      F_1 \cup F_2,            & \text{ иначе } 
    \end{cases}$
  \item
    $\Delta(q,a) = 
    \begin{cases}
      \{\delta_1(q,a)\},                       & \text{ ако } q\in Q_1\ \&\ a\in\Sigma\\
      \{\delta_2(q,a)\},                       & \text{ ако } q\in Q_2\ \&\  a\in\Sigma\\
      \{\delta_1(\qstart',a), \delta_2(\qstart'',a)\}, & \text{ ако } q = \qstart\ \&\ a \in\Sigma.
    \end{cases}$
  \end{itemize}
\end{hint}


\begin{example}
    За да построим автомат, който разпознава обединението на $\L(\N_1)$ и $\L(\N_2)$,
    трябва да добавим ново начално състояние, което да свържем с наследниците на началните състояния на $\N_1$ и $\N_2$.
    
    \begin{figure}[H]
      \center
      % \begin{subfigure}[b]{0.3\textwidth}
      \begin{tikzpicture}[framed,->,>=stealth,thick,node distance=2cm]
        \tikzstyle{every state}=[circle,minimum size=20pt,auto]
        \node[initial,state,accepting]      (0) {$q_0$};
        \node[state,accepting] [above right of=0]        (1) {$q'_0$};
        \node[state]    [right of=1]        (2) {$q_1$};
        \node[state]                        (3) [above right of=2] {$q_2$};
        \node[state,accepting]                        (4) [below right of=2] {$q_3$};
        \node[state]    [below right=2cm of 0] (5) {$q''_0$};
        \node[state]     [above right of=5] (6) {$q_4$};
        \node[state,accepting]     [below right of=5] (7) {$q_5$};
        \path
        (1) edge [loop above] node [above] {$b$} (1)
        (1) edge node [above]                  {$a$} (2)
        (2) edge node [above]                  {$a$} (3)
        (2) edge node [below]                  {$b$} (4)
        (3) edge [bend right=15] node [above]  {$a,b$} (1)
        (4) edge [bend left=15]  node [below]  {$b$} (1)
        (4) edge [bend right=15]  node [right]  {$a$} (3)
        (5) edge [bend left=15] node [below]   {$a$} (6)
        (6) edge [bend left=15] node  [right] {$a,b$} (7)
        (5) edge [bend right=15] node [above]  {$b$} (7)
        (7) edge [loop right] node [right] {$a,b$} (7)
        (0) edge [dashed, bend right=15] node [below]  {$a$} (2)
        (0) edge [dashed, bend left=15] node [above]  {$b$} (1)
        (0) edge [dashed, bend right=15] node [below]  {$a$} (6)
        (0) edge [dashed, bend right=45] node [below]  {$b$} (7);
      \end{tikzpicture}
      \caption{$\L(\N) = \L(\A_1)\cup\L(\A_2)$}
  \end{figure}  
  Обърнете внимание, че $\A_1$ и $\A_2$ са детерминирани автомати, но $\N$ е недетерминиран.
  Освен това, новото състояние $q_0$ трябва да бъде маркирано като финално, защото $q'_0$ е финално.
\end{example}

\begin{lemma}
  \label{lem:kleene-star}
  Класът от автоматните езици е затворен относно операцията {\em звезда на Клини}, т.е.
  ако $\L(\A) = L$, то съществува $\N$, за който $\L(\N) = L^\star$.
\end{lemma}
\begin{proof}
  Нека е даден автомата $\A = \pair{\Sigma,Q,\qstart,\delta,F}$, за който е изпънено, че
  $\L(\A) = \L(\mathbf{r})$.
  
  Ще построим $\N = \pair{\Sigma,Q', \qstart', \Delta, F'}$, такъв че
  \[\L(\N) = (\L(\A))^\star.\]


  

  % Първата стъпка е да построим $\N_1 = \NFAn{1}$, такъв че 
  % \[\L(\N_1) = \bigcup_{n\geq 1} (\L(\N))^n = \bigcup_{n\geq 1} (\L(\mathbf{r}))^n = \L(\mathbf{r^+}).\]
  Това можем да направим по следния начин:
  \begin{itemize}
  \item
    $Q' = Q \cup \{\qstart'\}$;
  \item
    $F' = F \cup \{\qstart'\}$;
  \item
    За $q \in Q$ и $a \in \Sigma$, определяме функцията на преходите $\Delta$ по следния начин:
    \begin{align*}
      \Delta(q,a) \df
      \begin{cases}
        \{\delta(q,a)\}, & \text{ако }\delta(q,a) \not\in F\\
        \{\delta(q,a), \qstart\}, & \text{ако }\delta(q,a) \in F.
      \end{cases}
    \end{align*}
    За $a \in \Sigma$,
    \begin{align*}
      \Delta(\qstart',a) \df
      \begin{cases}
        \{\delta(\qstart, a)\}, & \text{ако }\delta(\qstart,a) \not\in F\\
        \{\delta(\qstart, a), \qstart\}, & \text{ако }\delta(\qstart,a) \in F.
      \end{cases}
    \end{align*}
  \end{itemize}


  Нека $\alpha = a_0a_1\cdots a_{n-1} \in \L(\N)$.
  Това означава, че $(\qstart',\alpha) \vdash^\star_\N (f,\varepsilon)$ за някое $f \in F$.
  Нека редицата от състояния $(q_i)^n_{i=0}$ описва едно приемащо изчисление на $\N$ върху $\alpha$, т.е.
  \begin{itemize}
  \item
    $q_0 = \qstart'$;
  \item
    $q_{i+1} \in \Delta(q_i,a_i)$;
  \item
    $q_n \in F$.
  \end{itemize}
  \marginpar{Възможно е $l = 0$ и съответно тази подредица да е празна.}
  Да разгледаме подредицата $(q_{i_j})^{l-1}_{j = 0}$ на $(q_i)^{n}_{i=0}$ съставена от тези състояния, за които
  \[\delta(q_{i_j}, a_{i_j}) \in F\ \&\ q_{i_j+1} = \qstart.\]

  Да положим:
  \begin{align*}
    & \alpha_{i_0} \df a_0\cdots a_{i_0};\\
    & \alpha_{i_{j+1}} \df a_{i_j+1}\cdots a_{i_{j+1}}\text{, за }0\leq j \leq l-2\\
    & \alpha_{i_l} \df a_{i_{l-1}+1}\cdots a_n.
  \end{align*}
  Ясно е, че $\alpha = \alpha_{i_0}\alpha_{i_1}\cdots\alpha_{i_l}$.

  Сега можем да разбием изчислението на $\N$ върху $\alpha$ по следния начин:
  \begin{align*}
    (\qstart',\alpha_{i_0}\alpha_{i_1}\cdots \alpha_{i_l}) & \vdash^\star_\N (q_{i_0+1}, \alpha_{i_1}\cdots \alpha_{i_l})\\
                                                           & \vdash^\star_\N (q_{i_1+1},\alpha_{i_2}\cdots \alpha_{i_l})\\
                                                           & \vdash^\star_\N\\
                                                           & \cdots\\
                                                           & \vdash^\star_\N (q_{i_{l-1}+1}, \alpha_{i_l})\\
                                                           & \vdash^\star_\N(q_n,\varepsilon).
  \end{align*}

  За $j = 0$ имаме, че:
  \[(\qstart',\alpha_{i_0}) \vdash_\N (q_1, a_1\cdots a_{i_0}) \vdash^\star_\N (q_{i_0}, a_{i_0}) \vdash_\N (q_{i_0+1}, \varepsilon).\]

  Понеже, според конструкцията на $\N$, $\delta(\qstart, a_0) = q_1$ и освен това $\delta(q_{i_0}, a_{i_0}) = p$, за някое $p \in F$, то 
  \[(\qstart,\alpha_{i_0}) \vdash_\A (q_1, a_1\cdots a_{i_0}) \vdash^\star_\A (q_{i_0}, a_{i_0}) \vdash_\A (p, \varepsilon).\]
  Оттук следва, че думата $\alpha_{i_0} \in \L(\A)$.
  В случая, когато $\alpha_{i_0} = a_0$, то $\delta(\qstart,a_0) \in F$ и $q_{i_0} = \qstart$
  
  
  За всяко $j$, където $0<j<l$, понеже $\delta(q_{i_j}, a_{i_j}) = p$, за някое $p \in F$, то 
  \[(q_{i_j+1}, \alpha_{i_j}) \vdash^\star_\A (q_{i_{j+1}}, a_{i_{j+1}}) \vdash_\A (p, \varepsilon).\]
  Понеже $q_{i_j+1} = \qstart$, то оттук следва, че думата $\alpha_{i_j} \in \L(\A)$.

  За $j = l$ имаме, че
  \[(q_{i_l+1}, \alpha_{i_l}) \vdash^\star_\A (q_n, \varepsilon).\]
  Понеже, $q_{i_{l+1}} = \qstart$ и $q_n \in F$, то $\alpha_{i_l} \in F$.

  Накрая заключаваме, че $\alpha \in \L(\A)^\star$.
\end{proof}


\begin{example}
  Нека да приложим конструкцията за да намерим автомат разпознаващ $\L(\N)^\star$.
  
  \begin{figure}[H]
    % \center
    \begin{subfigure}[b]{0.3\textwidth}
      \begin{tikzpicture}[framed,->,>=stealth,thick,node distance=45pt]
        \tikzstyle{every state}=[circle,minimum size=20pt,auto]
        \node[initial,state]      (1) {$s$};
        \node[state]              (2) [right of=1] {$q_1$};
        \node[state,accepting]    (3) [right of=2] {$q_2$};
        \node[state,accepting]    (4) [above of=2] {$q_3$};
        \path
        (1) edge node [above] {$a$} (2)
        (1) edge [bend left=15] node [above] {$b$} (4)
        (2) edge node [above] {$b$} (3)
        (3) edge [bend left=45] node [below] {$a$} (1);
      \end{tikzpicture}
      \caption{автомат $\N$}
    \end{subfigure}
    \hspace{2cm}
    \begin{subfigure}[b]{0.5\textwidth}
      \begin{tikzpicture}[framed,->,>=stealth,thick,node distance=45pt]
        \tikzstyle{every state}=[circle,minimum size=20pt,auto]
        % \node[initial above,state,accepting] (0) {$s'$};
        \node[initial, state]                (1) [below right of=0] {$s$};
        \node[state]                         (2) [right of=1] {$q_1$};
        \node[state,accepting]               (3) [right of=2] {$q_2$};
        \node[state,accepting]               (4) [above of=2] {$q_3$};
        \path
        % (0) edge [dashed, bend left=15] node [above] {$a$} (2)
        % (0) edge [dashed, bend left=15] node [above] {$b$} (4)
        (1) edge [bend left=15] node [above] {$b$} (4)
        (1) edge node [above] {$a$} (2)
        (2) edge node [below] {$b$} (3)
        (3) edge [bend left=45] node [below] {$a$} (1)
        (3) edge [dashed, bend right=45] node [above] {$b$} (4)
        (4) edge [dashed] node [left] {$a$} (2)
        (4) edge [dashed, loop above] node {$b$} (4)
        (3) edge [dashed, bend right=45] node [above] {$a$} (2);        
      \end{tikzpicture}
      \caption{$\L(\N_1) = \L(\N)^+$}
    \end{subfigure}
  \end{figure}
    
  \marginpar{Лесно се вижда, че $\L(\N) = \{b\} \cup \{aba\}^\star \cdot ab$}
  След като построим автомат за езика $\L(\N)^+$, трябва да приложим
  конструкцията за обединение на автомата за езика $\L(\N)^+$ с автомата за езика $\{\varepsilon\}$.
  Защо трябва да добавим ново начално състояние $s'$?
  Да допуснем, че вместо това сме направили $s$ финално.
  Тогава има опасност да разпознаем повече думи. Например, думата $aba$ би се разпознала от този автомат,
  но $aba \not\in\L(\N)^\star$.
  
  \begin{figure}[H]
    \centering
    % \begin{subfigure}[b]{0.5\textwidth}
      \begin{tikzpicture}[framed,->,>=stealth,thick,node distance=45pt]
        \tikzstyle{every state}=[circle,minimum size=20pt,auto]
        \node[initial, state, accepting]     (0) {$s'$};
        \node[state]                         (1) [below right of=0] {$s$};
        \node[state]                         (2) [right of=1] {$q_1$};
        \node[state, accepting]              (3) [right of=2] {$q_2$};
        \node[state, accepting]              (4) [above of=2] {$q_3$};
        \path
        (0) edge [dashed, bend left=15] node [above] {$a$} (2)
        (0) edge [dashed, bend left=15] node [above] {$b$} (4)
        (1) edge [bend left=15] node [above] {$b$} (4)
        (1) edge node [above] {$a$} (2)
        (2) edge node [below] {$b$} (3)
        (3) edge [bend left=45] node [below] {$a$} (1)
        (3) edge [dashed, bend right=45] node [above] {$b$} (4)
        (4) edge [dashed] node [left] {$a$} (2)
        (4) edge [dashed, loop above] node {$b$} (4)
        (3) edge [dashed, bend right=45] node [above] {$a$} (2);        
      \end{tikzpicture}
      \caption{$\L(\hat\N) = \L(\N)^\star = \L(\N)^+ \cup \{\varepsilon\}$}    
  \end{figure}

\end{example}



%%% Local Variables:
%%% mode: latex
%%% TeX-master: "../eai"
%%% End:

\section{Един критерий за нерегулярност}

\begin{lemma}[Лема за покачването]
  \index{лема за покачването!регулярни езици}
  \label{lem:pumping-reg}
  \marginpar{На англ. се нарича \\ Pumping Lemma}
  \marginpar{Има подобна лема и за безконтекстни езици, която ще разгледаме по-нататък.}
  Нека $L$ да бъде {\em безкраен} регулярен език.
  Съществува число $p\geq 1$, зависещо само от $L$, 
  за което за всяка дума $\alpha\in L, \abs{\alpha}\geq p$ може да 
  бъде записана във вида $\alpha = xyz$ и 
  \begin{enumerate}[1)]
  \item
    $|y|\geq 1$;
  \item
    $|xy|\leq p$;
    \marginpar{Обърнете внимание, че $0 \in \Nat$ и $xy^0z =  xz$}
  \item
    $(\forall i\in\Nat)[xy^iz \in L]$.
  \end{enumerate}
\end{lemma}
\begin{hint}
  \marginpar{Тази лема я има във всеки учебник по този предмет. Например, \cite[стр. 88]{papadimitriou}, \cite[стр. 77]{sipser3}.}
  Понеже $L$ е регулярен, то $L$ е и автоматен език. Нека
  \[\A = \FA\]
  е краен детерминиран автомат, за който $L = \L(\A)$.
  Да положим $p = \abs{Q}$ и нека $\alpha = a_1a_2\cdots a_k$ е дума, за която $k \geq p$.
  Да разгледаме първите $p$ стъпки от изпълнението на $\alpha$ върху $\A$:
  \[\qstart\stackrel{a_1}{\rightarrow} q_1 \stackrel{a_2}{\rightarrow}q_2 \dots \stackrel{a_p}{\rightarrow} q_p.\]
  Тъй като $\abs{Q} = p$, а по този път участват $p+1$ състояния $q_0,q_1,\dots,q_p$,
  то съществуват числа $i, j$, за които $0\leq i < j\leq p$ и $q_i = q_j$.
  Нека разделим думата $\alpha$ на три части по следния начин:
  \[\underbrace{a_1\cdots a_i}_{x}\quad \underbrace{a_{i+1}\cdots a_j}_{y}\quad \underbrace{a_{j+1}\cdots a_k}_{z}.\]
  Ясно е, че $\abs{y} \geq 1$ и $\abs{xy} = j \leq p$.
  Да разгледаме случая за $i = 0$.
  Думата $xy^0z = xz \in L$, защото имаме следното изчисление:
  \[\qstart\underbrace{\stackrel{a_1}{\rightarrow}q_1 \cdots \stackrel{a_i}{\rightarrow}}_{x} q_i\underbrace{\stackrel{a_{j+1}}{\rightarrow}q_{j+1}\cdots\stackrel{a_{k}}{\rightarrow}}_{z}q_k\in F,\]
  защото $q_i = q_j$.
  Да разгледаме и случая $i = 2$. Тогава думата $xy^2z \in L$, защото имаме следното изчисление:
  \[\qstart\underbrace{\stackrel{a_1}{\rightarrow}q_1 \cdots \stackrel{a_i}{\rightarrow}}_{x} q_i\underbrace{\stackrel{a_{i+1}}{\rightarrow}q_{i+1}\cdots\stackrel{a_{j}}{\rightarrow}}_{y}q_j\underbrace{\stackrel{a_{i+1}}{\rightarrow}q_{i+1}\cdots\stackrel{a_{j}}{\rightarrow}}_{y}q_j\underbrace{\stackrel{a_{j+1}}{\rightarrow}\cdots\stackrel{a_{k}}{\rightarrow}}_{z}q_k\in F.\]
  \marginpar{\writedown Докажете!}
  Вече лесно можем да съобразим, че за всяко естествено число $i$, е изпълнено $xy^iz \in \L(\A)$.
\end{hint}

Практически е по-полезно да разглеждаме следната еквивалентна формулировка на лемата за покачването.

\begin{cor}[Контрапозиция на лемата за покачването]
  \label{cor:pumping-reg}
  \marginpar{Ясно е, че всеки краен език е регулярен. Нали?}
  Нека $L$ е произволен {\bf безкраен} език. Нека също така е изпълнено, че:
  \begin{description}
  \item[($\forall$)]
    за {\em всяко} естествено число $p \geq 1$,
  \item[($\exists$)]
    можем да намерим дума $\alpha \in L$, $\abs{\alpha}\geq p$, такава че
  \item[($\forall$)]
    за {\em всяко} разбиване на думата на три части, $\alpha = xyz$, със свойствата $\abs{xy} \leq p$ и $\abs{y} \geq 1$,
  \item[($\exists$)]
    можем да посочим $i \in \Nat$, за което е изпълнено, че $xy^iz \not\in L$.
  \end{description}  
  Тогава $L$ {\bf не} е регулярен език.
\end{cor}
\begin{proof}
  Да означим с $P_{\text{reg}}(L)$ следната формула:
  \begin{align*}
    (\exists p \geq 1)(\forall \alpha \in L)[\abs{\alpha} \geq p \Rightarrow (\exists x,y,z\in\Sigma^\star)[\ & \alpha = xyz\ \& \\
                                                                                                              & \abs{y} \geq 1\ \&\\
                                                                                                              & \abs{xy} \leq p\ \&\\
                                                                                                              & (\forall i\in\Nat)[xy^iz \in L]]].
  \end{align*}
  Така условието на \hyperref[lem:pumping-reg]{Лемата за покачването} представлява твърдението:
  \begin{center}
    {\em ,,Aко $L$ е регулярен език, то е изпълнено $P_{\text{reg}}(L)$.''}
  \end{center}
  \marginpar{Контрапозиция на твърдението $P \to Q$ е твърдението $\neg Q \to \neg P$}
  \noindent
  Лемата може да се запише по следния еквивалентен начин:
  
  \begin{center}
    {\em ,,Ако $P_{\text{reg}}(L)$ не е изпълнено, то $L$ не е регулярен език.''}
  \end{center}
  \marginpar{Използваме, че $\neg \exists \forall \exists \forall (\dots) \equiv \forall \exists \forall \exists \neg(\dots)$}
  Отрицанието на $P_{\text{reg}}(L)$ преставлява формулата
  \begin{align*}
    (\forall p \geq 1)(\exists \alpha \in L)[\ \abs{\alpha} \geq p\ \&\ (\forall x,y,z\in\Sigma^\star)[\ & \alpha \neq xyz\ \vee\\
                                                                                                         & \abs{y} \not\geq 1\ \vee\\
                                                                                                         & \abs{xy} \not\leq p\ \vee\\
                                                                                                         & (\exists i\in\Nat)[xy^iz \not\in L]\ ]\ ].
  \end{align*}
  Горната формула е еквивалентна на:
  \marginpar{Използваме, че
    \begin{align*}
      \neg P \vee \neg Q \vee R & \equiv \neg(P\ \&\ Q) \vee R\\
                                & \equiv (P \wedge Q) \to R
    \end{align*}}
  \begin{align*}
    (\forall p \geq 1)(\exists \alpha \in L)[\ \abs{\alpha} \geq p\ \&\ (\forall x,y,z\in\Sigma^\star)[\ & (\alpha = xyz \wedge \abs{y} \geq 1\wedge \abs{xy} \leq p)\\
                                                                                                         & \Rightarrow (\exists i\in\Nat)[xy^iz \not\in L]\ ]\ ].
  \end{align*}

  Това означава, че ако
  \begin{description}
  \item[($\forall$)]
    вземем произволна константа $p \geq 1$,
  \item[($\exists$)]
    за нея намерим дума $\alpha \in L$, такава че $\abs{\alpha} \geq p$ и 
  \item[($\forall$)]
    докажем, че за всяко нейно разбиване на три части $x,y,z$, със свойствата
    $\abs{y} \geq 1$ и $\abs{xy} \leq p$,
  \item[($\exists$)]
    можем да намерим естествено число $i$, за което $xy^iz \not\in L$,
  \end{description}
  то можем да заключим, че езикът $L$ не е регулярен.
\end{proof}

%%% Local Variables:
%%% mode: latex
%%% TeX-master: "../eai"
%%% End:

\section{Минимален автомат}

Нека да започнем като въведем няколко означения.
Нека $\alpha$ е дума над азбуката $\Sigma$  и $L$ е език. Означаваме 
\[\alpha^{-1}L = \{\omega \in \Sigma^\star \mid \alpha\omega \in L\}.\]
Освен това, да означим 
\[\alpha L = \{\alpha\omega \in \Sigma^\star \mid \omega \in L\}.\]
Имаме свойството, че
\[L = \{\omega \in \Sigma^\star \mid \varepsilon \in \omega^{-1}L\}.\]

\begin{prop}
  За всеки две думи $\alpha, \beta \in \Sigma^\star$ е изпълнено, че 
  \[(\alpha\cdot\beta)^{-1}L = \beta^{-1}(\alpha^{-1}L).\]
\end{prop}

\subsection{Минимален автомат по даден регулярен език}
Да разгледаме езика $L = \L(\mathbf{((a+b)^+\cdot a)^\star})$.
\marginpar{
Удобно е да представим 
\begin{align*}
  L & = \{\varepsilon\} \cup \{a,b\}^+ a L\\
  & = \{\varepsilon\} \cup a\{a,b\}^\star aL \cup b\{a,b\}^\star aL
\end{align*}}

Да видим как можем директно да построим минимален тотален детерминиран автомат $\A$ разпознаващ $L$, където
\[\A = \pair{Q^\A,\Sigma,s,\delta_\A,F^\A}.\]
Състоянията на автомата ще бъдат от вида $q_B$, където $B \subseteq \Sigma^\star$, така че накрая искаме да имаме свойството
\[B = \{\omega \in \Sigma^\star \mid \delta^\star_\A(q_B,\omega) \in F^\A\}.\]
Тогава началното състояние ще бъде $q_L$, защото 
\[L = \L(\A) = \{\omega \in \Sigma^\star \mid \delta^\star_\A(q_L,\omega) \in F^\A\}.\]
Сега едновременно ще строим състоянията на автомата $Q^\A$ и функцията на преходите $\delta_\A$.
\begin{itemize}
\item 
\marginpar{
  Удобно е да представим
  \begin{align*}
    M & \df \{a,b\}^\star aL\\
    & = aL \cup \{a,b\}^+aL\\
    & = aL \cup aM \cup bM
  \end{align*}
  Тогава 
  \begin{align*}
    L & = \{\varepsilon\} \cup aM \cup bM
  \end{align*}
  Освен това,
  \begin{align*}
    N & \df L \cup M \\
    & = \{\varepsilon\} \cup aM \cup bM \cup M\\
    & = \{\varepsilon\} \cup aM \cup bM \cup aL\\
  \end{align*}}
  $a^{-1}L = \{a,b\}^\star aL \df M$.
  Имаме, че $M \neq L$, защото $\varepsilon \in L$, но $\varepsilon \not\in M$.
  Понеже $M \neq L$, имаме ново състояние $q_M$ в автомата и 
  $\delta_\A(q_L,a) = q_M$.
\item
  $b^{-1}L = \{a,b\}^\star aL = M$;
  Следователно, $\delta_\A(q_L,b) = q_M$.
\item
  $a^{-1}M = L \cup M \df N$. 
  Имаме, че $N \neq L$, защото $a\in N$, но $a \not\in L$.
  Освен това, $N \neq M$, защото $\varepsilon \in N$, но $\varepsilon \not\in M$.
  Понеже $N \neq L, M$, имаме ново състояние $q_N$ в автомата и 
  $\delta_\A(q_M,a) = q_N$.
\item
  $b^{-1}M = M$. Следователно, $\delta_\A(q_M,b) = q_M$.
\item
  $a^{-1}N = L \cup M = N$. Следователно, $\delta_\A(q_N,a) = q_N$.
\item
  $b^{-1}N = M$.
  Следователно, $\delta_\A(q_M,b) = q_M$.
\end{itemize}
От горните сметки следва, че 
\[(\forall \omega \in \Sigma^\star)[\omega^{-1}L \in \{L, N, M\}].\]
Така получихме, че 
\[Q^\A = \{q_L, q_N, q_M\}.\]
Нека сега да съобразим кои са финалните състояния.
Понеже имаме свойството 
\[L = \{\omega \in \Sigma^\star \mid \varepsilon \in \omega^{-1}L\},\]
то следва, че финалните състояния на автомата $\A$ са  $q_L$ и $q_M$,
защото $\varepsilon \in L,M$. 
Сега вече сме готови да нарисуваме картинка на автомата.

\begin{figure}[H]
  \begin{subfigure}[b]{.4\textwidth}
    \begin{tikzpicture}[->,>=stealth,thick,node distance=55pt]
      \tikzstyle{every state}=[circle,minimum size=20pt,auto]
      
      \node[initial above, state,accepting]   (L) {$q_L$};
      \node[state]                            (M) [right of=L]{$q_M$};
      \node[state,accepting]                  (N) [right of=M]{$q_N$};
      
      
      \path 
      (L) edge [bend left=15] node [above] {$a,b$}   (M)
      (M) edge [loop above] node [above] {$b$} (M)
      (M) edge [bend left=15] node [above] {$a$} (N)
      (N) edge [bend left=15] node [below] {$b$} (M)
      (N) edge [loop above] node [above] {$a$} (N);
    \end{tikzpicture}
    \caption{Минимален автомат за езика $\L(\mathbf{((a+b)^+a)^\star})$}
  \end{subfigure}
  \quad
  ~
  \quad
  \begin{subfigure}[b]{.4\textwidth}
    \begin{tikzpicture}[->,>=stealth,thick,node distance=55pt]
      \tikzstyle{every state}=[circle,minimum size=20pt,auto]
      
      \node[initial above, state,accepting]   (L) {$[\varepsilon]_L$};
      \node[state]                            (M) [right of=L]{$[a]_L$};
      \node[state,accepting]                  (N) [right of=M]{$[aa]_L$};
      
      
      \path 
      (L) edge [bend left=15] node [above] {$a,b$}   (M)
      (M) edge [loop above] node [above] {$b$} (M)
      (M) edge [bend left=15] node [above] {$a$} (N)
      (N) edge [bend left=15] node [below] {$b$} (M)
      (N) edge [loop above] node [above] {$a$} (N);
    \end{tikzpicture}
    \caption{Минимален автомат за езика $\L(\mathbf{((a+b)^+a)^\star})$}
  \end{subfigure}
\end{figure}

Остава да се уверим, че нашата конструкция е коректна, т.е. наистина автоматът $\A$ е минимален за езика $L$.
Според \hyperref[th:myhill-nerode]{Теоремата на Майхил-Нероуд}, 
състоянията на минималния автомат $\M$, разпознаващ $L$, са класовете на еквивалентност на релацията $\approx_L$.
Да разгледаме изображението $f$ с дефиниционна област $\Sigma^\star/_{\approx_L} = \{[\alpha]_L \mid \alpha \in \Sigma^\star\}$, дефинирано като:
\[f([\alpha]_L) = q_K, \text{ където }K = \alpha^{-1}L.\]
\begin{itemize}
\item 
  Ако $[\alpha]_L \neq [\beta]_L$, то $\alpha^{-1}L \neq \beta^{-1}L$ и оттук следва, че $f([\alpha]_L) \neq f([\beta]_L)$.
  Това означава, че $f$ е инективна.
\item
  Да разгледаме $q_K \in Q^\A$. От конструкцията на автомата $\A$ следва, че съществува дума $\alpha \in \Sigma^\star$,
  за която $K = \alpha^{-1}L$. Това означава, че $f([\alpha]_L) = q_K$.
  Следователно, $f$ е сюрективна.
\item
  Имаме и свойството:
  \begin{align*}
    f(\delta_\M([\alpha]_L,x)) & = f([\alpha\cdot x]_L) & (\text{деф. на }\delta_\M)\\
    & = q_N & (N = (\alpha\cdot x)^{-1}L = x^{-1}(\alpha^{-1}L))\\
    & = \delta_\A(q_M, x) & (M = \alpha^{-1}L)\\
    & = \delta_\A(f([\alpha]_L), x),
  \end{align*}
  от което следва, че $f$ е биекция.
\end{itemize}

Според горните разсъждения, 
\begin{align*}
  Q^\M & = \{f(q_L),f(q_M),f(q_N)\} = \{[\varepsilon]_L, [a]_L, [aa]_L\},\\
  F^\M & = \{f(q_L), f(q_N)\} = \{[\varepsilon]_L, [aa]_L\}.
\end{align*}

Проверете, че наистина $L = [\varepsilon]_L \cup [aa]_L$.

\begin{example}
  Да разгледаме езика $L = \L(\mathbf{a\cdot(a+b)^\star\cdot b})$.
  \begin{itemize}
  \item 
    $a^{-1}L = \{a,b\}^\star b = M$;
  \item
    $b^{-1}L = \emptyset$;
  \item
    $a^{-1}M = \{b\} \cup \{a,b\}^\star b = N$;
  \item
    $b^{-1}M = \{\varepsilon\} \cup \{b\} \cup \{a,b\}^\star b = P$;
  \item
    $a^{-1}N = \{b\} \cup \{a,b\}^\star b = M$;
  \item
    $b^{-1}N = \{\varepsilon\} \cup \{b\} \cup \{a,b\}^\star b = P$;
  \item
    $a^{-1}P = \{b\} \cup \{a,b\}^\star b = N$;
  \item
    $b^{-1}P = \{\varepsilon\} \cup \{b\} \cup \{a,b\}^\star b = P$.
  \end{itemize}

  \begin{figure}[H]
    \centering
    \begin{tikzpicture}[->,>=stealth,thick,node distance=55pt]
      \tikzstyle{every state}=[circle,minimum size=20pt,auto]
      
      \node[initial, state]                   (L) {$q_L$};
      \node[state]                            (M) [above right of=L]{$q_M$};
      \node[state]                            (E) [below right of=L]{$q_\emptyset$};
      \node[state]                            (N) [right of=M]{$q_N$};
      \node[state, accepting]                 (P) [below of=N]{$q_P$};
      
      
      \path 
      (L) edge [bend left=15]  node [above] {$a$} (M)
      (L) edge [bend right=15] node [above] {$b$} (E)
      (E) edge [loop right]    node [right] {$a,b$} (E) 
      (M) edge [bend right=15] node [below] {$a$} (N)
      (N) edge [bend right=30] node [above] {$a$} (M)
      (M) edge [bend right=15] node [below] {$b$} (P)
      (N) edge [bend left=15]  node [right] {$b$} (P)
      (P) edge [bend left=15]  node [left]  {$a$} (N)
      (P) edge [loop right]    node [right] {$b$} (P);      
    \end{tikzpicture}
    \caption{Минимален автомат за $\L(\mathbf{a\cdot (a+b)^\star\cdot b})$}
  \end{figure}
\end{example}

\begin{example}
  Да разгледаме езика 
  \[L = \{\omega \in \{a,b\}^\star \mid \omega \text{ съдържа четен брой $a$ и точно едно $b$}\}.\]
  Нека да видим дали можем да построим автомат за този език.
  \begin{itemize}
  \item 
    $a^{-1}L \df M$ е езика съставен от думите с нечетен брой $a$ и точно едно $b$;
  \item 
    $b^{-1}L \df N$ е езика съставен от думите с четен брой $a$ и нито едно $b$;
  \item
    $a^{-1}M = L$;
  \item
    $b^{-1}M \df P$ е езика съставен от думите с нечетен брой $a$ и нито едно $b$;
  \item
    $a^{-1}N = P$;
  \item
    $b^{-1}N = \emptyset$;
  \item
    $a^{-1}P = N$;
  \item
    $b^{-1}P = \emptyset$;
  \end{itemize}

  \begin{figure}[H]
    \centering
    \begin{tikzpicture}[->,>=stealth,thick,node distance=55pt]
      \tikzstyle{every state}=[circle,minimum size=20pt,auto]
      
      \node[initial, state]        (L) {$q_L$};
      \node[state]                 (M) [above right of=L]{$q_M$};
      \node[state, accepting]      (N) [below right of=M]{$q_N$};
      \node[state]                 (P) [right of=M]{$q_P$};
      \node[state]                 (E) [right of=N]{$q_\emptyset$};
            
      \path 
      (L) edge [bend right=15]  node [below] {$a$} (M)
      (M) edge [bend right=15]  node [above] {$a$} (L)
      (L) edge [bend right=15] node [above] {$b$} (N)
      (M) edge [bend left=15] node [above] {$b$} (P)
      (N) edge [bend left=15] node [left] {$a$} (P)
      (P) edge [bend left=15] node [right] {$a$} (N)
      (P) edge [bend left=15] node [right] {$b$} (E)
      (N) edge [bend right=15] node [below] {$b$} (E);
    \end{tikzpicture}
    \caption{Минимален автомат, който приема думи с четен брой $a$ и точно едно $b$}
  \end{figure}  
\end{example}


\begin{example}
  Да разгледаме езика $L = \{a^nb^n\mid n \in \Nat\}$. Да се опитаме да построим автомат, който го разпознава.
  Нека да означим $L_k = \{a^nb^{n+k}\mid n \in \Nat\}$. Да видим какво се получава като приложим процедурата за строене 
  на минимален автомат.
  \begin{itemize}
  \item 
    $a^{-1}L = L_1$;
  \item
    $b^{-1}L = \emptyset$;
  \item
    $a^{-1}L_1 = L_2$;
  \item
    $b^{-1}L_1 = \{\varepsilon\}$;
  \item
    $a^{-1}\{\varepsilon\} = b^{-1}\{\varepsilon\} = \emptyset$;
  \item
    Вижда се, че $a^{-1}L_k = L_{k+1}$, за всяко $k$.
  \item
    Вижда се, че $b^{-1}L_{k+1} = \{b^k\}$, за всяко $k$.
    Освен това е ясно, че $b^{-1}\{b^{k}\} = \{b^{k-1}\}$, за всяко $k \geq 1$.
  \end{itemize}
  Получаваме, че езикът $L$ се разпознава от автомат с {\em безкрайно много състояния}.
  
  \begin{figure}[H]
    \centering
    \begin{tikzpicture}[->,>=stealth,thick,node distance=55pt]
      \tikzstyle{every state}=[circle,minimum size=15pt,auto]
      
      \node[initial, state]                   (0) {$L$};
      \node[state]                            (1) [right of=0]{$L_1$};
      \node[state]                            (2) [right of=1]{$L_2$};
      \node[state]                            (3) [right of=2]{$L_3$};
      \node[state]                            (A) [below of=1]{$\emptyset$};
      \node[state]                            (B) [below right of=1]{$\{b\}$};
      \node[state]                            (BB) [below right of=2]{$\{bb\}$};
      \node[state, accepting]                 (E) [below of=A]{$\{\varepsilon\}$};
      
      \coordinate[right of=3] (4);
      \coordinate[below right of=3] (BBB);
      \coordinate[below of=4] (BBBA);

      \path 
      (0) edge [bend left=15]   node [above] {$a$} (1)
      (1) edge [bend left=15]   node [above] {$a$} (2)
      (2) edge [bend left=15]   node [above] {$a$} (3)
      (0) edge [bend right=30]  node [left] {$b$} (E)
      (E) edge [loop left]      node [left] {$a,b$} (E)
      (1) edge [bend right=30]  node [left] {$b$} (E)
      (2) edge [bend right=15]  node [left] {$b$} (B)
      (3) edge [bend right=15]  node [left] {$b$} (BB)
      (B) edge [bend right=15]  node [above] {$b$} (A)
      (B) edge [bend left=15]  node [right] {$a$} (E)
      (A) edge [bend right=15]   node [right] {$a,b$} (E)
      (BB) edge [bend right=15] node [above] {$b$} (B)
      (BB) edge [bend left=15]  node [below] {$a$} (E);
      
      \draw [dashed,->,shorten >=0pt] (3) to[bend left=15] node[auto] {$a$} (4);
      \draw [dashed,->,shorten >=0pt] (BBB) to[bend right=15] node[above] {$b$} (BB);
      \draw [dashed,->,shorten >=0pt] (BBBA) to[bend left=30] node[below] {$a$} (E);
    \end{tikzpicture}
    \caption{Получаваме {\em безкраен} автомат за $\{a^nb^n \mid n \in \Nat\}$}
  \end{figure}    
\end{example}

\subsection{Проверка за регулярност на език}

  \begin{prop}
    Езикът $L$ е регулярен точно тогава, когато релацията $\approx_L$ има {\em крайно много} класове на еквивалентност.
  \end{prop}
\begin{proof}
  Ако $L$ е регулярен, то той се разпознава от някой ДКА $\A$, който има крайно много състояния 
  и следователно крайно много класове на еквивалентност относно $\sim_\A$.
  Релацията $\approx_L$ е по-груба от $\sim_\A$ и има по-малко класове на еквивалентност.
  Следователно, $\approx_L$ има крайно много класове на еквивалентност.
  
  За другата посока, ако $\approx_L$ има крайно много класове на еквивалентност, то можем да 
  построим ДКА $\A$ както в доказателството на \hyperref[th:myhill-nerode]{Теоремата на Майхил-Нероуд}, който разпознава $L$.
\end{proof}

Това следствие ни дава още един начин за проверка дали даден език е регулярен.
За разлика от \Lem{pumping-reg}, сега имаме {\bf необходимо и достатъчно условие}.
При даден език $L$, ние разглеждаме неговата релация $\approx_L$.
Ако тя има крайно много класове, то езикът $L$ е регулярен.
В противен случай, езикът $L$ не е регулярен.

\begin{example}
  За езика $L = \{a^nb^n\mid n \in \Nat\}$ имаме, че $\abs{\approx_L} = \infty$,
  защото \[(\forall k,j\in\Nat)[k \neq j \implies [a^kb]_L \neq [a^jb]_L].\]
  Проверете, че $[a^kb]_L = \{a^kb,a^{k+1}b^{2},\dots,a^{k+l}b^{l+1},\dots\}$.
  Така получаваме, че релацията $\approx_L$ има безкрайно много класове на еквивалентност.
  Заключаваме, че този език {\bf не} е регулярен.
\end{example}

\begin{example}
  За езика $L = \{a^{n^2} \mid n \in \Nat\}$ имаме, че $\abs{\approx_L} = \infty$,
  защото \[(\forall m,n\in\Nat)[m \neq n \implies [a^{n^2}]_L \neq [a^{m^2}]_L].\]
  
  Без ограничение на общността, да разгледаме $n < m$ и думата $\gamma = a^{2n+1}$.
  Тогава $a^{n^2}\gamma = a^{(n+1)^2} \in L$, но 
  $m^2 < m^2 + 2n + 1 < (m+1)^2$ и следователно $a^{m^2}\gamma = a^{m^2+2n+1}\not\in L$.
\end{example}

\begin{example}
  За езика $L = \{a^{n!} \mid n \in \Nat\}$ имаме, че $\abs{\approx_L} = \infty$,
  защото \[(\forall m,n\in\Nat)[m \neq n \implies [a^{n!}]_L \neq [a^{m!}]_L].\]
  
  Без ограничение на общността, да разгледаме $n < m$ и думата $\gamma = a^{(n!)n}$.
  Тогава $a^{n!}\gamma = a^{(n+1)!} \in L$, но 
  $m! < m! + (n!)n < m! + (m!)m = (m+1)!$ и следователно $a^{m!}\gamma = a^{m!+(n!)n}\not\in L$.
\end{example}

\begin{problem}
  Да разгледаме езика
  \[L = \{a^{f_n} \mid f_0 = f_1 = 1\ \&\ f_{n+2} = f_{n+1} + f_{n}\}.\]
  Докажете, че $\abs{\approx_L} = \infty$.
\end{problem}

%\marginpar{\href{http://en.wikipedia.org/wiki/DFA_minimization}{Уикипедия}}

\subsection{Релация на Майхил-Нероуд}

\begin{itemize}
\item
  \index{Майхил-Нероуд!релация}
  \marginpar{$\approx_L$ е известна като релация на Майхил-Нероуд}
  Нека $L \subseteq \Sigma^\star$ е език и нека $\alpha,\beta \in \Sigma^\star$.
  Казваме, че $\alpha$ и $\beta$ са {\bf еквивалентни относно} $L$, което записваме 
  като $\alpha \approx_L \beta$, когато:
  \[\alpha \approx_L \beta \dff \alpha^{-1}L = \beta^{-1}L.\]
  С други думи, 
  \[\alpha \approx_L \beta \iff (\forall \omega \in \Sigma^\star)[\alpha\omega \in L \iff \beta\omega \in L].\]
\item
  \marginpar{Трябва ли $\A$ да е тотален?}
  Нека $\A = \FA$ е ДКА.
  Казваме, че две думи $\alpha,\beta \in \Sigma^\star$ са {\bf еквивалентни относно $\A$},
  което означаваме с $\alpha \sim_\A \beta$, ако 
  \[\delta^\star(s,\alpha) = \delta^\star(s,\beta).\]
\item
  Проверете, че $\approx_L$ и $\sim_\A$ са {\bf релации на еквивалентност}, т.е.
  те са рефлексивни, транзитивни и симетрични.
\item
  Класът на еквивалентност на думата $\alpha$ относно релацията $\approx_L$ означаваме като
  \[[\alpha]_L \df \{\beta \in \Sigma^\star \mid \alpha \approx_L \beta\}.\]
  Означаваме 
  \[\Sigma^\star/_{\approx_L} \df \{[\alpha]_L \mid \alpha \in \Sigma^\star\}.\]
  Тогава с $\abs{\Sigma^\star/_{\approx_L}}$ ще означаваме броя на класовете на еквивалентност на релацията $\approx_L$.
\item
  Класът на еквивалентност на думата $\alpha$ относно релацията $\sim_\A$ означаваме като
  \[[\alpha]_\A \df \{\beta \in \Sigma^\star \mid \alpha \sim_\A \beta\}.\]
  Означаваме 
  \[\Sigma^\star/_{\sim_\A} \df \{[\alpha]_\A \mid \alpha \in \Sigma^\star\}.\]
  С $\abs{\Sigma^\star/_{\sim_\A}}$ ще означаваме броя на класовете на еквивалентност на релацията $\sim_\A$.
\item
  Съобразете, че всяко състояние на $\A$, което е достижимо от началното състояние, определя клас на еквивалентност относно 
  релацията $\sim_\A$. Това означава, че ако за всяка дума означим  $q_\alpha = \delta^\star_\A(s,\alpha)$, то
  $\alpha \sim_\A \beta$ точно тогава, когато $q_\alpha = q_\beta$. Заключаваме, че броят на класовете на еквивалентност
  на $\sim_\A$ е равен на броя на достижимите от $s$ състояния. Следователно,
  \[|\Sigma^\star/_{\sim_\A}| \leq |Q^\A|.\]
\item
  Релациите $\approx_\L$ и $\sim_\A$ са дясно-инвариантни, т.е. за всеки две думи $\alpha$ и $\beta$
  е изпълнено:
  \begin{align*}
    \alpha \sim_\A \beta  &\implies (\forall \gamma\in\Sigma^\star)[\alpha\gamma \sim_\A \beta\gamma],\\
    \alpha \approx_\L \beta & \implies (\forall \gamma\in\Sigma^\star)[\alpha\gamma \approx_\L \beta\gamma].
  \end{align*}
\end{itemize}



\begin{prop}
  \label{pr:rel-finer}
  За всеки ДКА $\A = \FA$ е изпълнено:
  \[(\forall \alpha,\beta \in \Sigma^\star)[\alpha\sim_\A\beta \implies \alpha\approx_{\L(\A)}\beta].\]
  С други думи, 
  $[\alpha]_\A \subseteq [\alpha]_{\L(\A)}$, за всяка дума $\alpha \in \Sigma^\star$.
\end{prop}
\begin{proof}
%  \marginpar{стр. 95 от \cite{papadimitriou}}
  Да означим за всяка дума $\alpha$, $q_\alpha = \delta^\star_\A(s, \alpha)$.
  Лесно се съобразява, че за всеки две думи $\alpha$ и $\beta$ имаме 
  \begin{align*}
    \alpha \sim_\A \beta & \iff \delta^\star(s,\alpha) = \delta^\star(s,\beta) & (\text{по деф. на }\sim_\A)\\
    & \iff q_\alpha = q_\beta.
  \end{align*}
  Нека $\alpha \sim_\A \beta$. Ще проверим, че  $\alpha \approx_{\L(\A)} \beta$.
  За произволно $\gamma \in \Sigma^\star$ имаме:
  \begin{align*}
    \alpha\gamma \in \L(\A) & \iff \delta^\star(s,\alpha\gamma)\in F & (\text{по деф. на }\L(\A))\\
    & \iff \delta^\star(\delta^\star(s,\alpha),\gamma) \in F & (\text{по деф. на }\delta^\star)\\
    & \iff \delta^\star(q_\alpha, \gamma) \in F & (q_\alpha = \delta^\star(s,\alpha))\\
    & \iff \delta^\star(q_\beta, \gamma) \in F & (q_\alpha = q_\beta, \text{ защото }\alpha \sim_\A \beta)\\
    & \iff \delta^\star(\delta^\star(s,\beta),\gamma) \in F & (q_\beta = \delta^\star(s,\beta))\\
    & \iff \delta^\star(s,\beta\gamma) \in F & (\text{по деф. на }\delta^\star)\\
    & \iff \beta\gamma \in \L(\A) & (\text{по деф. на }\L(\A)).
  \end{align*}
  Заключаваме, че 
  \[(\forall \alpha,\beta \in \Sigma^\star)[\alpha\sim_\A\beta \implies \alpha\approx_{\L(\A)}\beta].\]
\end{proof}

\begin{problem}
  Докажете, че за всеки тотален ДКА $\A$ и всяка дума $\alpha$,
  \[[\alpha]_{\L(\A)} = \bigcup_{\beta \in [\alpha]_{\L(\A)}}[\beta]_\A.\]
\end{problem}


\begin{cor}
  \label{cor:approx-less-sim}
  За всеки тотален ДКА $\A$ е изпълнено, че
  \[\abs{\Sigma^\star/_{\approx_{\L(\A)}}} \leq \abs{\Sigma^\star/_{\sim_\A}}.\]
\end{cor}
\begin{hint}
  Да означим $L = \L(\A)$ и да разгледаме изображението 
  \[f([\alpha]_L) \df \{[\gamma]_\A \mid \gamma \approx_L \alpha\}.\]

  \begin{itemize}
  \item
    Ясно е, че за всяко $\alpha$, $f([\alpha]_L) \neq \emptyset$.
  \item 
    Съобразете, че $f$ е {\bf функция}, т.е. 
    \[(\forall\alpha,\beta\in\Sigma^\star)[\alpha \approx_L \beta \implies f([\alpha]_L) = f([\beta]_L)].\]
  \item
    Използвайте \Prop{rel-finer} за да съобразите, че 
    \[(\forall\alpha,\beta\in\Sigma^\star)[\alpha \not\approx_L \beta \implies f([\alpha]_L) \cap f([\beta]_L) = \emptyset].\]
  \item
    Заключете, че \[\abs{\Sigma^\star/_{\approx_{\L(\A)}}} \leq \abs{\Sigma^\star/_{\sim_\A}}.\]
  \end{itemize}
\end{hint}

\begin{cor}
  \label{cor:upper-bound}
  Нека $L$ е произволен регулярен език $L$.  
  Всеки тотален ДКА $\A$, който разпознава $L$ има свойството
  \[\abs{\Sigma^\star/_{\approx_L}} \leq \abs{Q},\]
  т.е. броят на класовете на еквивалентност на релацията $\approx_L$
  не надвишава броя на състоянията на автомата.
\end{cor}
\begin{proof}
  Да изберем $\A$, който разпознава $L$, бъде такъв, че да {\em няма недостижими състояния}.
  Тъй като всяко достижимо състояние определя клас на еквивалентност относно $\sim_\A$,
  то получаваме, че $\abs{Q} = \abs{\sim_\A}$.
  Комбинирайки със \Cor{approx-less-sim},
  \[\abs{Q} = \abs{\Sigma^\star/_{\sim_\A}} \geq \abs{\Sigma^\star/_{\approx_L}}.\]
\end{proof}
Така получаваме {\em долна граница} за броя на състоянията в тотален автомат разпознаващ езика $L$.
Този брой е не по-малък от броя на класовете на еквивалентност на $\approx_L$.
В следващия раздел ще видим, че тази долна граница може да бъде достигната.

\subsection{Теорема за съществуване на минимален автомат}

\begin{thm}[Майхил-Нероуд]
  \label{th:myhill-nerode}
  \index{Майхил-Нероуд!теорема}
  % \index{Майхил}
  % \index{Нероуд}
  \marginpar{на англ. Myhill-Nerode}
  Нека $L\subseteq \Sigma^\star$ е регулярен език.
  Тогава съществува ДКА $\A = \FA$, който разпознава $L$,
  с точно толкова състояния, колкото са класовете на еквивалентност на релацията $\approx_L$,
  т.е. $\abs{Q} = \abs{\Sigma^\star/_{\approx_L}}$.
\end{thm}
\begin{proof}
%  \marginpar{стр. 96 от \cite{papadimitriou}}
  Да фиксираме регулярния език $L$.
  Ще дефинираме тотален ДКА $\A = \FA$, разпознаващ $L$, като:
  \begin{itemize}
  \item
    $Q = \{[\alpha]_L\mid \alpha\in \Sigma^\star\}$;
  \item
    $s = [\varepsilon]_L$;
  \item
    $F = \{[\alpha]_L\mid \alpha\in L\} = \{[\alpha]_L \mid [\alpha]_L \cap L \neq \emptyset\}$;
  \item
    Определяме изображението $\delta$ като 
    за всяка буква $x \in \Sigma$ и всяко състояние $[\alpha]_L\in Q$, 
    \[\delta([\alpha]_L,x) = [\alpha x]_L.\]
  \end{itemize}
  
  Първо, трябва да се уверим, че множеството от състояния $Q$ е крайно, т.е.
  релацията $\approx_L$ има крайно много класове на еквивалентност.
  И така, тъй като $L$ е регулярен език, то той се разпознава от някой тотален ДКА $\A'$.
  От \Cor{upper-bound} имаме, че $\abs{Q^{\A'}} \geq \abs{\Sigma^\star/_{\approx_L}}$.
  Понеже $Q^{\A'}$ е крайно множество, то $\approx_L$ има крайно много класове и 
  следователно $Q$ също е крайно множество.

  Второ, трябва да се уверим, че изображението $\delta$ задава функция, т.е. 
  да проверим, че за всеки две думи $\alpha$, $\beta$ и всяка буква $x$,
  \[[\alpha]_L = [\beta]_L \implies \delta([\alpha]_L,x) = \delta([\beta]_L,x).\]
  Но това се вижда веднага, защото от определението на релацията $\approx_L$ следва, че
  ако $\alpha \approx_L \beta$, то за всяка буква $x$, $\alpha x \approx_L \beta x$,
  т.е. $[\alpha x]_L = [\beta x]_L$ и 
  \begin{align*}
    [\alpha]_L = [\beta]_L & \implies [\alpha x]_L = [\beta x]_L & (\text{свойство на }\approx_L)\\
    & \implies \delta([\alpha]_L,x) = [\alpha x]_L = [\beta x]_L = \delta([\beta]_L,x) & (\text{деф. на }\delta)
  \end{align*}
  
  Така вече сме показали, че $\A$ е коректно зададен тотален ДКА.
  Остава да покажем, че $\A$ разпознава езика $L$, т.е. $\L(\A) = L$.
  За целта, първо ще докажем едно помощно твърдение.
  \begin{prop}
    За всеки две думи $\alpha$ и $\beta$,
    $\delta^\star([\alpha]_L,\beta) = [\alpha\beta]_L$.
  \end{prop}
  \begin{proof}
    Ще докажем това свойство с индукция по дължината на $\beta$.
    \begin{itemize}
    \item
      За $\beta = \varepsilon$ свойството следва директно от дефиницията на $\delta^\star$ като рефлексивно и транзитивно затваряне на $\delta$,
      защото $\delta^\star([\alpha]_L,\varepsilon) = [\alpha]_L$.
    \item
      Нека $\abs{\beta} = n+1$ и да приемем, че сме доказали твърдението за думи с дължина $n$.
      Тогава $\beta = \gamma a$, където $\abs{\gamma} = n$. Свойството следва от следните равенства:
      \begin{align*}
        \delta^\star([\alpha]_L, \gamma a) & = \delta(\delta^\star([\alpha]_L,\gamma),a) & (\text{деф. на }\delta^\star)\\
                                          & = \delta([\alpha\gamma]_L,a) & (\text{от {\bf И.П.} за }\gamma)\\
                                          & = [\alpha\gamma a]_L & (\text{от деф. на }\delta)\\
                                          & = [\alpha\beta]_L & (\beta = \gamma a).
      \end{align*}
    \end{itemize}
  \end{proof}
  \noindent За да се убедим, че $L = \L(\A)$ е достатъчно да проследим еквивалентностите:
  \begin{align*}
    \alpha\in \L(\A) & \iff \delta^\star(s,\alpha) \in F & (\text{от деф. на }\L(\A))\\
                     & \iff \delta^\star([\varepsilon]_L,\alpha) \in F & (\text{по деф. }s = [\varepsilon]_L)\\
                     & \iff \delta^\star([\varepsilon]_L,\alpha) = [\alpha]_L\ \&\ \alpha\in L & (\text{от деф. на }F)\\
                     & \iff \alpha \in L & (\text{от последното твърдение}).
  \end{align*}
\end{proof}

\begin{dfn}
  \index{изоморфизъм}
  Нека $\A_1 = \FAn{1}$ и $\A_2 = \FAn{2}$.
  Казваме, че $\A_1$ и $\A_2$ са {\bf изоморфни}, което означаваме с $\A_1 \cong \A_2$, ако
  съществува биекция $f: Q_1\to Q_2$, за която:
  \begin{itemize}
  \item
    $f(s_1) = s_2$;
  \item
    $q \in F_1 \iff f(q) \in F_2$;
  \item
    $(\forall a\in\Sigma)(\forall q\in Q_1)[f(\delta_1(q,a)) = \delta_2(f(q),a)]$.
  \end{itemize}
  Ще казваме, че $f$ задава изоморфизъм на $\A_1$ върху $\A_2$.
\end{dfn}

Това означава, че два автомата $\A_1$ и $\A_2$ са изоморфни, ако можем да получим $\A_2$
като преименуваме състоянията на $\A_1$.

\begin{cor}
  Нека е даден регулярния език $L$.
  Всички минимални автомати за $L$ са изоморфни на $\A_0$, автоматът построен в \hyperref[th:myhill-nerode]{Теоремата на Майхил-Нероуд}.
\end{cor}
\begin{proof}
  Нека $\A = \FA$ е произволен тотален автомат, за който $\L(\A) = L$ и $\abs{Q} = \abs{\Sigma^\star/_{\approx_L}}$.
  Съобразете, че $\A$ е {\em свързан}, т.е. всяко състояние на $\A$ е достижимо от началното.
  Искаме да докажем, че $\A \cong \A_0$.
  Понеже $\A$ е свързан, за всяко състояние $q$ можем да намерим дума $\omega_q$,
  за която $\delta^\star(s,\omega_q) = q$.
  Да дефинираме изображението $f:Q\to \Sigma^\star/_{\approx_L}$ като $f(q) = [\omega_q]_L$.
  Ще докажем, че
  $f$ задава изоморфизъм на $\A$ върху $\A_0$. 
  \begin{itemize}
  \item
    Първо да съобразим, че ако $\delta^\star_\A(s,\alpha) = q$, то $[\omega_q]_L = [\alpha]_L$.
    Понеже $\delta^\star_\A(s,\alpha) = q = \delta^\star_\A(s,\omega_q)$, то $\omega_q \sim_\A \alpha$.
    От \Prop{rel-finer} имаме, че
    \[\omega_q \sim_\A \alpha \implies \omega_q \approx_L \alpha.\]
    Това означава, $[\omega_q]_L = [\alpha]_L$ и следователно $f$ е определена коректно, т.е. $f$ е {\bf функция}.
  \item
    Ще проверим, че $f$ е {\bf инективна}, т.е.
    \[(\forall q_1,q_2 \in Q)[q_1\neq q_2 \implies f(q_1) \neq f(q_2)].\]
    Да допуснем, че има състояния $q_1 \neq q_2$, за които 
    \[f(q_1) = [\omega_{q_1}]_L = [\omega_{q_2}]_L = f(q_2).\]
    Тогава $\omega_{q_1} \not\sim_\A \omega_{q_2}$ и $\omega_{q_1} \approx_L \omega_{q_2}$.
    \marginpar{\writedown Обяснете!}
    Но тогава от \Cor{upper-bound} получаваме, че $\abs{\Sigma^\star/_{\sim_\A}} > \abs{\Sigma^\star/_{\approx_L}}$,
    което противоречи с минималността на $\A$.
  \item
    За да бъде $f$ {\bf сюрективна} трябва за всеки клас $[\beta]_L$ да съществува състояние $q$, за което $f(q) = [\beta]_L$.
    Понеже $\A$ е свързан, съществува състояние $q$, за което $\delta^\star_\A(s,\beta) = q$.
    Вече се убедихме, че в този случай $\beta \approx_L \omega_q$, защото $\beta \sim_\A \omega_q$.
    Тогава $f(q) = [\omega_q]_L = [\beta]_L$.
  \item
    За последно оставихме проверката, че $f$ наистина е {\bf изоморфизъм}:
    \begin{align*}
      f(\delta_\A(q,a)) & = f(\delta_\A(\delta^\star_\A(s,\omega_q),a)) & (\text{от избора на }\omega_q)\\
      & = f(\delta^\star_\A(s,\omega_qa)) & (\text{от деф. на }\delta^\star_\A)\\
      & = [\omega_qa]_L & (\text{от деф. на }f)\\
      & = \delta^\star_{\A_0}([\varepsilon]_L, \omega_qa) & (\text{от деф. на }\A_0)\\ 
      & = \delta_{\A_0}(\delta^\star_{\A_0}([\varepsilon]_L, \omega_q),a) & (\text{от деф. на }\delta^\star_{\A_0})\\
      & = \delta_{\A_0}([\omega_q]_L, a) & (\text{свойство на }\delta^\star_{\A_0})\\
      & = \delta_{\A_0}(f(q), a) & ( f(q) = [\omega_q]_L).
    \end{align*}
  \end{itemize}
\end{proof}


%%% Local Variables: 
%%% mode: latex
%%% TeX-master: "EAI"
%%% End: 

\subsection{Алгоритъм за строене на минимален автомат}
\begin{itemize}
\item
  Да фиксираме произволен тотален ДКА $\A = \FA$.
\item
  За състояние $p$ в автомата $\A$, да означим с $\L_\A(p)$ езикът, който се разпознава от автомата $\A$,
  ако приемем, че $p$ е началното състояние на автомата, т.е.
  \[\L_\A(p) \df \{\omega \in \Sigma^\star \mid \delta^\star(p,\omega) \in F\}.\]
  В частност, $\L(\A) = \L_\A(\qstart)$.
\item
  Сега дефинираме следната релация между състояния на автомата $\A$:
  \[p \equiv_\A q\ \dff\ \L_\A(p) = \L_\A(q).\]
  Това означава, че $p \equiv_\A q$ точно тогава, когато
  \begin{equation}
    \label{eq:1}
    (\forall \omega\in \Sigma^\star)[\delta^\star(p,\omega) \in F\ \iff\ \delta^\star(q,\omega) \in F].
  \end{equation}
\item
  Релацията $\equiv_\A$ между състояния на автомата $\A$ е релация на еквивалентност. 
\item
  Нека $q_\alpha$ е състоянието, което съответства на думата $\alpha$ в $\A$, т.е.
  $\delta^\star_\A(s,\alpha) = q_\alpha$. Тогава 
  \[\L_\A(q_\alpha) = \alpha^{-1}\L(\A).\]
  Оттук получаваме, че 
  \begin{align*}
    q_\alpha \equiv_\A q_\beta & \iff \L_\A(q_\alpha) = \L_\A(q_\beta)\\
    & \iff \alpha^{-1}\L(\A) = \beta^{-1}\L(\A)\\
    & \iff \alpha \sim_{\L(\A)} \beta.
  \end{align*}
  % \[q_\alpha \equiv_\A q_\beta\ \iff\ \alpha\approx_{\L(\A)} \beta.\]
  Това означава, че ако в $\A$ няма недостижими състояния от началното състояние $s$, то 
  \[\abs{\Sigma^\star/_{\equiv_\A}} = \abs{\Sigma^\star/_{\approx_{\L(\A)}}}.\]
\end{itemize}

При даден език $L$ и тотален ДКА $\A = \FA$, който го разпознава, нашата цел е да построим нов ДКА $\A_0$,
който има толкова състояния колкото са класовете на еквивалентност на релацията $\approx_\L$.
Това ще направим като ``слеем'' състоянията на $\A$, които са еквивалентни относно релацията $\equiv_\A$.
Това означава, че всяко състояние на $\A_0$ ще отговаря на един клас на еквивалентност на релацията $\equiv_\A$.
Проблемът с намирането на класовете на еквивалентност на релацията $\equiv_\A$ е кванторът $\forall \gamma \in \Sigma^\star$
във нейната дефиниция чрез (Формула \ref{eq:1}), защото $\Sigma^\star$ е безкрайно множество от думи.

Да фиксираме автомата $\A$ и $L = \L(\A)$.
Да означим 
\[\L^n_\A(p) \df \{\omega \in \Sigma^\star \mid \abs{\omega} \leq n\ \&\ \delta^\star(p,\omega) \in F\}.\]
Според тази дефиниция, $L = \bigcup_{n\geq 0} \L^n_\A(\qstart)$.

За всяко естествено число $n$, дефинираме бинарните релации $\equiv_n$ върху $Q$ по следния начин:
\[p \equiv_n q \dff \L^n_\A(p) = \L^n_\A(q).\]

% Алгоритъмът представлява намирането на релации $\equiv_n$, където
% \[p\equiv_n q \iff (\forall\gamma\in\Sigma^\star)[\abs{\gamma}\leq n\ \rightarrow\ (\delta^\star(p,\gamma) \in F\ \iff\ \delta^\star(q,\gamma) \in F)].\]
Релациите $\equiv_n$ представляват апроксимации на релацията $\equiv_\A$.
Обърнете внимание, че за всяко $n$, $\equiv_n$ е {\em по-груба} релация от $\equiv_{n+1}$, 
която на свой ред е по-груба от $\equiv_\A$.
Алгоритъмът строи $\equiv_n$ докато не срещнем $n$, за което $\equiv_n\ =\ \equiv_{n+1}$.
Тъй като броят на класовете на еквивалентност на $\equiv_\A$ е краен (той е $\leq \abs{Q}$), то 
със сигурност ще намерим такова $n$, за което $\equiv_n\ =\ \equiv_{n+1}$.
Тогава заключаваме, че $\equiv_\A\ =\ \equiv_n$.

Понеже единствената дума с дължина $0$ e $\varepsilon$ и по определение $\delta^\star(p,\varepsilon) = p$, 
лесно се съобразява, че $\equiv_0$ има два класа на еквивалентност.
Единият е $F$, а другият е $Q\setminus F$.

\begin{prop}
  \label{pr:one-letter-test}
  За всеки две състояния $p,q \in Q$, и всяко $n$, $p \equiv_{n+1} q$ точно тогава, когато
  \begin{enumerate}[a)]
  \item
    $p \equiv_{n} q$ и
  \item
    $(\forall a \in \Sigma)[\delta(q,a) \equiv_{n} \delta(p,a)]$.
  \end{enumerate}
\end{prop}
\begin{hint}
  \marginpar{\cite[стр. 99]{papadimitriou}}
  \begin{align*}
    p \equiv_{n+1} q & \iff \L^{n+1}_\A(p) = \L^{n+1}_\A(q)\\
    & \iff \L^n_\A(p) = \L^n_\A(q)\ \&\ (\forall a \in \Sigma)[\L^n_\A(\delta(p,a)) = \L^n_\A(\delta(q,a))]\\
    & \iff p \equiv_n q\ \&\ (\forall a \in \Sigma)[\delta(p,a) \equiv_n \delta(q,a)].
  \end{align*}
\end{hint}

Нека е даден автомата $A = \FA$.
След като сме намерили релацията $\equiv_\A$ за $\A$, 
строим автомата $\A' = (Q',\Sigma,s',\delta',F')$, където:
\begin{itemize}
\item
  $Q' = \{[q]_{\equiv_\A} \mid q\in Q\}$;
\item
  $s' = [s]_{\equiv_\A}$;
\item
  $\delta'([q]_{\equiv_\A}, a) = [\delta(q,a)]_{\equiv_\A}$;
\item
  $F' = \{[q]_{\equiv_\A}\mid F\cap [q]_{\equiv_\A} \neq \emptyset\}$;
\end{itemize}

От всичко казано дотук знаем, че $\A'$ е минимален автомат разпознаващ езика $\L(\A)$.

\begin{example}
  Да разгледаме следния краен детерминиран автомат $\A$.
  \begin{figure}[H]
    \begin{subfigure}[b]{.4\textwidth}
      \begin{tikzpicture}[->,>=stealth,thick,node distance=55pt]
        \tikzstyle{every state}=[circle,minimum size=20pt,auto]
        
        \node[initial above, state]   (0) {$0$};
        \node[state]            (1) [above right of=0]{$1$};
        \node[state]            (2) [below right of=0]{$2$};
        \node[state,accepting]  (3) [right of=1]{$3$};
        \node[state,accepting]  (4) [right of=2]{$4$};
        \node[state,accepting]  (5) [below right of=3]{$5$};
        
        
        \path 
        (0) edge  node [above] {$a$}   (1)
        (0) edge  node [below] {$b$}   (2)
        (1) edge node [above] {$a$}    (3)
        (1) edge [bend left=15] node [below] {$b$}    (4)
        (2) edge [bend left=15] node [left] {$b$}    (3)
        (2) edge node [below] {$a$}   (4)
        (4) edge  node [below] {$a,b$} (5)
        (3) edge  node [left] {$a,b$}  (5)
        (5) edge [loop above]   node [above] {$a,b$}  (5);
      \end{tikzpicture}
      \caption{Ще построим минимален автомат разпознаващ $\L(\A)$}
    \end{subfigure}
    \qquad
    \qquad
    \begin{subfigure}[b]{0.5\textwidth}
      \begin{tikzpicture}[->,>=stealth,thick,node distance=45pt]
        \tikzstyle{every state}=[circle,minimum size=20pt,auto,scale=.9]
        
        \node[initial above, state]   (0) {$B_0$};
        \node[state]            (1) [right of=0]{$B_1$};
        \node[state,accepting]  (2) [right of=1]{$B_2$};
        
        \path 
        (0) edge [bend left=15] node [above] {$a,b$}   (1)
        (1) edge [bend left=15] node [above] {$a,b$}   (2)
        (2) edge [loop above] node [above] {$a,b$}   (2);
      \end{tikzpicture}
      \caption{Получаваме следния минимален автомат $\A_0$, $\L(\A_0) = \L(\A)$}
      \label{sub:min1}
    \end{subfigure}
  \end{figure}
  \marginpar{Съобразете, че $\L(\A) = \{\alpha \in \{a,b\}^\star \mid \abs{\alpha} \geq 2\}$.}

Ще приложим алгоритъма за минимизация за да получим минималния автомат за езика $L$.
За всяко $n = 0,1,2,\dots$, ще намерим класовете на еквивалентност на $\equiv_n$,
докато не намерим $n$, за което $\equiv_n\ =\ \equiv_{n+1}$.

\begin{itemize}
\item 
  Класовете на еквивалентност на $\equiv_0$ са два.
  Те са $A_0 = Q\setminus F = \{0,1,2\}$ и $A_1 = F = \{3,4,5\}$.
\item
  Сега да видим дали можем да разбием някои от класовете на еквивалентност на $\equiv_0$.
  
  \begin{tabular}{|c|c|c|c|c|c|c|}
    \hline
    $Q$ & $0$ & $1$ & $2$ & $3^\star$ & $4^\star$ & $5^\star$ \\
    \hline
    \hline
    $\equiv_0$ & $A_0$ & $A_0$ & $A_0$ & $A_1$ & $A_1$ & $A_1$\\
    \hline
    $a$ & $A_0$& $A_1$ & $A_1$ & $A_1$ & $A_1$ & $A_1$\\
    \hline
    $b$ & $A_0$& $A_1$ & $A_1$ & $A_1$ & $A_1$ & $A_1$\\
    \hline
  \end{tabular}

  Виждаме, че $0 \not\equiv_1 1$ и $1 \equiv_1 2$.
  Класовете на еквивалентност на $\equiv_1$ са 
  $B_0 = \{0\}$, $B_1 = \{1,2\}$, $B_2 = \{3,4,5\}$.
\item
  Сега да видим дали можем да разбием някои от класовете на еквивалентност на $\equiv_1$.
  
  \begin{tabular}{|c|c|c|c|c|c|c|}
    \hline
    $Q$ & $0$ & $1$ & $2$ & $3^\star$ & $4^\star$ & $5^\star$ \\
    \hline
    \hline
    $\equiv_1$ & $B_0$ & $B_1$ & $B_1$ & $B_2$ & $B_2$ & $B_2$\\
    \hline
    $a$ & $B_1$ & $B_2$ & $B_2$ & $B_2$ & $B_2$ & $B_2$\\
    \hline
    $b$ & $B_1$ & $B_2$ & $B_2$ & $B_2$ & $B_2$ & $B_2$\\
    \hline
  \end{tabular}

  Виждаме, че $\equiv_1\ =\ \equiv_2$.
  \marginpar{Получаваме, че $\equiv_\A\ =\ \equiv_1$}
  Следователно, минималният автомат има три състояния.
  Той е изобразен на Фигура \ref{sub:min1}.  
  Минималният автомат може да се представи и таблично:
  
  \begin{tabular}{|c|c|c|c|c|c|c|}
    % \hline
    % $Q$ & $0$ & $1$ & $2$ & $3^\star$ & $4^\star$ & $5^\star$ \\
    % \hline
    \hline
    $\delta$ & $B_0$ & $B_1$ & $B_2$ \\
    \hline
    $a$ & $B_1$ & $B_2$ & $B_2$ \\
    \hline
    $b$ & $B_1$ & $B_2$ & $B_2$ \\
    \hline
  \end{tabular}
\end{itemize}
\end{example}

\begin{example}
  Да разгледаме следния краен детерминиран автомат $\A$.
  \begin{figure}[H]
    % \begin{center}
    \begin{subfigure}[b]{0.4\textwidth}
      \begin{tikzpicture}[->,>=stealth,thick,node distance=55pt]
        \tikzstyle{every state}=[circle,minimum size=20pt,auto]
        
        \node[initial above, state]   (0) {$0$};
        \node[state,accepting]        (1) [above right of=0]{$1$};
        \node[state,accepting]        (2) [below right of=0]{$2$};
        \node[state]                  (3) [right of=1]{$3$};
        \node[state]                  (4) [right of=2]{$4$};
        \node[state,accepting]        (5) [below right of=3]{$5$};
        
        \path 
        (0) edge node [below] {$a$}   (1)
            edge node [below] {$b$}   (2)
        (1) edge node [above] {$a$}    (3)
            edge [bend left=15] node [below] {$b$}    (4)
        (2) edge [bend left=15] node [left] {$b$}    (3)
            edge node [below] {$a$}   (4)
        (4) edge node [below] {$a,b$} (5)
        (3) edge node [left] {$a,b$}  (5)
        (5) edge [loop above]   node [above] {$a,b$}  (5);
      \end{tikzpicture}
      \caption{Ще построим минимален автомат разпознаващ $\L(\A)$}
    \end{subfigure}
    \qquad
    \qquad
    \begin{subfigure}[b]{0.4\textwidth}
      \begin{tikzpicture}[->,>=stealth,thick,node distance=45pt]
        \tikzstyle{every state}=[circle,minimum size=20pt,auto,scale=.9]
        
        \node[initial above, state]   (0) {$C_0$};
        \node[state,accepting]  (1) [right of=0]{$C_1$};
        \node[state]            (2) [right of=1]{$C_2$};
        \node[state,accepting]  (3) [right of=2]{$C_3$};
                
        \path 
        (0) edge [bend left=15] node [above] {$a,b$}   (1)
        (1) edge [bend left=15] node [above] {$a,b$}   (2)
        (2) edge [bend left=15] node [above] {$a,b$}   (3)
        (3) edge [loop above]   node [above] {$a,b$}   (3);
      \end{tikzpicture}
      \caption{Получаваме следния минимален автомат $\A_0$, $\L(\A_0) = \L(\A)$}
      \label{sub:min2}
    \end{subfigure}
  \end{figure}

  \marginpar{Съобразете, че $\L(\A) = \{a,b\} \cup \{\alpha \in \{a,b\}^\star \mid \abs{\alpha} \geq 3\}$.}
  
  Отново следваме същата процедура за минимизация.
  Ще намерим класовете на еквивалентност на $\equiv_n$,
  докато не намерим $n$, за което $\equiv_n\ =\ \equiv_{n+1}$.
  \begin{itemize}
  \item
    Класовете на екиваленост на $\equiv_0$ са 
    $A_0 = Q\setminus F = \{0,3,4\}$ и $A_1 = F = \{1,2,5\}$.
  \item
    Разбиваме класовете на еквивалентност на $\equiv_0$ като използваме \Prop{one-letter-test}.
    
    \begin{tabular}{|c|c|c|c|c|c|c|}
      \hline
      $Q$ & 0 & $1^\star$ & $2^\star$ & 3 & 4 & $5^\star$ \\
      \hline
      \hline
      $\equiv_0$ & $A_0$ & $A_1$ & $A_1$ & $A_0$ & $A_0$ & $A_1$\\
      \hline
      $a$ & $A_1$& $A_0$ & $A_0$ & $A_1$ & $A_1$ & $A_1$\\
      \hline
      $b$ & $A_1$& $A_0$ & $A_0$ & $A_1$ & $A_1$ & $A_1$\\
      \hline
    \end{tabular}
    
    Виждаме, че $1 \not\equiv_1 5$ и $1 \equiv_0 5$.
    Следователно, $\equiv_0\ \neq\ \equiv_1$.
    Класовете на еквивалентност на $\equiv_1$ са 
    $B_0 = \{0,3,4\}$, $B_1 = \{1,2\}$, $B_2 = \{5\}$.
  \item
    Сега се опитваме да разбием класовете на еквивалентност на $\equiv_1$.

    \begin{tabular}{|c|c|c|c|c|c|c|}
      \hline
      $Q$ & 0 & $1^\star$ & $2^\star$ & 3 & 4 & $5^\star$ \\
      \hline
      \hline
      $\equiv_1$ & $B_0$ & $B_1$ & $B_1$ & $B_0$ & $B_0$ & $B_2$\\
      \hline
      $a$ & $B_1$ & $B_0$ & $B_0$ & $B_2$ & $B_2$ & $B_2$\\
      \hline
      $b$ & $B_1$ & $B_0$ & $B_0$ & $B_2$ & $B_2$ & $B_2$\\
      \hline
    \end{tabular}
    
    Имаме, че $0 \equiv_1 3$, но $0 \not\equiv_2 3$. Следователно $\equiv_1\ \neq\ \equiv_2$.
    Класовете на еквивалентност на $\equiv_2$ са 
    $C_0 = \{0\}$, $C_1 = \{1,2\}$, $C_2 = \{3,4\}$, $C_3 = \{5\}$.
  \item
    Отново опитваме да разбием класовете на $\equiv_2$.

      \begin{tabular}{|c|c|c|c|c|c|c|}
        \hline
        $Q$ & 0 & $1^\star$ & $2^\star$ & 3 & 4 & $5^\star$ \\
        \hline
        \hline
        $\equiv_2$ & $C_0$ & $C_1$ & $C_1$ & $C_2$ & $C_2$ & $C_3$\\
        \hline
        $a$ & $C_1$ & $C_2$ & $C_2$ & $C_3$ & $C_3$ & $C_3$\\
        \hline
        $b$ & $C_1$ & $C_2$ & $C_2$ & $C_3$ & $C_3$ & $C_3$\\
        \hline
      \end{tabular}
      
      Виждаме, че не можем да разбием $C_1$ или $C_2$.
      \marginpar{Получаваме, че $\equiv_\A\ =\ \equiv_2$}
      Следователно, $\equiv_2\ =\ \equiv_3$ и минималният автомат разпознаващ езика $L$
      има четири състояния. Вижте Фигура \ref{sub:min2} за преходите на минималния автомат.
      Минималният автомат може да се представи и таблично:

      \begin{tabular}{|c|c|c|c|c|}
        \hline
        $\delta$ & $C_0$ & $C_1$ & $C_2$ & $C_3$ \\
        \hline
        $a$ & $C_1$ & $C_2$ & $C_3$ & $C_3$ \\
        \hline
        $b$ & $C_1$ & $C_2$ & $C_3$ & $C_3$ \\
        \hline
      \end{tabular}
      
  \end{itemize}
\end{example}


%%% Local Variables:
%%% mode: latex
%%% TeX-master: "../eai"
%%% End:

\section{Регулярни граматики}
\index{граматика!неограничена}
\label{sect:regular-grammar}
{\bf Неограничена граматика} e наредена четворка от вида
\[G = (V, \Sigma, R, S),\]
където
\begin{itemize}
\item
  $V$ е крайно множество от {\em променливи} (нетерминали);
\item
  $\Sigma$ е крайно множество от {\em букви} (терминали), $\Sigma \cap V = \emptyset$;
\item
  \marginpar{В \cite{hopcroft1} правилата се наричат {\em productions}}
  $R \subseteq (V\cup\Sigma)^+ \times (V \cup \Sigma)^\star$ е крайно множество от {\em правила}.
  Обикновено правилата $(\alpha, \beta) \in R$ ще означаваме като 
  $\alpha \to_G \beta$, където $\alpha \in (V \cup \Sigma)^+, \beta \in (V \cup \Sigma)^\star$;
\item
  $S \in V$ е началната променлива (нетерминал). 
\end{itemize}

Казваме, че имаме {\bf извод} $\omega \to_G \omega'$, ако $\omega = \alpha\beta\gamma \in (V\cup\Sigma)^\star$,
$\omega' = \alpha\beta'\gamma \in (V\cup\Sigma)^\star$ и имаме правило $\beta \to \beta'$ в граматиката $G$.
Нека $\to^\star_G$ е рефлексивното и транзитивно затваряне на релацията $\to_G$, т.е.
\begin{align*}
  & \alpha \to^\star_G \alpha\\
  & \alpha \to^\star_G \alpha'\ \&\ \alpha' \to_G \alpha'' \implies \alpha \to^\star_G \alpha''.
\end{align*}

Езикът, който се поражда от граматиката $G$ е
\[\L(G) = \{\omega \in \Sigma^\star \mid S \to^\star_G \omega\}.\]

Граматиките се разделят на няколко вида в зависимост от това какви {\em ограничения} налагаме върху правилата $R$.
В следващите няколко глави ще разгледаме различни ограничения. Сега ще разгледаме граматики с такъв вид правила,
които пораждат точно регулярните (или еквивалентно автоматни) езици.

\index{граматика!регулярна}
Граматиката $G = (V, \Sigma, R, S)$ се нарича {\bf регулярна граматика},
ако правилата са от вида 
\begin{align*}
  & A \to \varepsilon,\\
  % & A \to a,\\
  & A \to aB,
\end{align*}
където $A, B \in V$ и $a \in \Sigma$.

\begin{prop}
  Нека $G = \CFG$ е регулярна граматика и $L = \L(G)$.
  Съществува краен автомат $\A$, такъв че $L = \L(\A)$.
\end{prop}
\begin{hint}
  Нека $V = \{A_1,\dots,A_k\}$. Тогава:
  \begin{itemize}
  \item 
    $Q = \{q_1,\dots,q_k\}$;
  \item
    $F = \{q_i \mid A_i \to \varepsilon\}$.
  \item
    $\delta(q_i,a) = q_j\ \iff\ A_i \to aA_j$.
  \end{itemize}
\end{hint}

\begin{prop}
  Нека $\A$ е краен автомат и $L = \L(\A)$.
  Съществува регулярна граматика $G$, такава че $L = \L(G)$.
\end{prop}
\begin{hint}
  Нека $Q = \{q_1,\dots,q_k\}$. Тогава:
  \begin{itemize}
  \item 
    $V = \{A_1,\dots,A_k\}$;
  \item
    $A_i \to aA_j\ \iff\ \delta(q_i,a) = q_j$;
  \item
    $A_{i} \to \varepsilon\ \iff\ q_{i} \in F$.
  \end{itemize}
\end{hint}


%%% Local Variables:
%%% mode: latex
%%% TeX-master: "../eai"
%%% End:

\section{Допълнителни задачи}

\begin{problem}
  Постройте регулярен израз за езика на следната граматика:
  \begin{align*}
    & S \to S + S\ |\ S * S\ |\ A\\
    & A \to KL\ |\ LK\\
    & K \to 0K\ |\ \varepsilon\\
    & L \to 1K\ |\ \varepsilon.
  \end{align*}
\end{problem}


\begin{problem}
  Докажете, че следните езици са безконтекстни.
  \begin{enumerate}[a)]
  \item
    \marginpar{$S \rightarrow aSa\ \vert\ bSb\ \vert\ \varepsilon$}
    $L = \{ww^R \mid w \in \{a,b\}^\star\}$;
  \item
    \marginpar{$S \rightarrow aSa\ \vert\ bSb\ \vert\ a\vert\ b\ \vert\ \varepsilon$}
    $L = \{w \in \{a,b\}^\star \mid w = w^R\}$;
  \item
    $L = \{a^nb^{2m}c^{n} \mid m,n \in \Nat\}$;
  \item
    $L = \{a^nb^{m}c^{m}d^n \mid m,n \in \Nat\}$;
  \item
    \marginpar{Обединение на два езика}
    $L = \{a^nb^{2k} \mid n,k \in \Nat\ \&\ n \neq k\}$;
  \item
    \marginpar{$S \rightarrow aSb | aS | a$}
    $L = \{a^nb^k \mid n > k\}$;
  \item
    $L = \{a^nb^k \mid n \geq 2k\}$;
  % \item
  %   \marginpar{$S \rightarrow aSc | B,\ B \rightarrow bBc | \varepsilon$}
  %   $L = \{a^nb^mc^{n+m}\mid n,m \in \Nat\}$;
  % \item
  %   \marginpar{$S \rightarrow aSc | aS | B$, $B\rightarrow bBc | bB | \varepsilon$}
  %   $L = \{a^nb^kc^m \mid n + k \geq m\}$;
  \item
    \marginpar{$S \rightarrow aSc | aS | aB | bB$,\\$B\rightarrow bBc | bB | \varepsilon$}
    $L = \{a^nb^kc^m \mid n + k \geq m+1\}$;
  \item
    $L = \{a^nb^kc^m \mid n + k \geq m+2\}$;
  \item
    \marginpar{$S \rightarrow aSc | aS | B | Bc$,\\$B\rightarrow bBc | bB | \varepsilon$}
    $L = \{a^nb^kc^m \mid n + k + 1 \geq m\}$;
  \item
    $L = \{a^nb^mc^{2k} \mid n \neq 2m\ \&\ k \geq 1\}$;
  \item
    $L = \{a^nb^kc^m \mid n + k \leq m\}$;
  \item
    $L = \{a^nb^kc^m \mid n + k \leq m+1\}$;
  \item
    \marginpar{Обединение на три езика}
    $L = \{a^nb^mc^k \mid n, m, k \text{ не са страни на триъгълник}\}$.
  \item
    $L = \{a,b\}^\star \setminus \{a^{2n}b^n \mid n\in\Nat\}$;
  \item
    \marginpar{$S\to EaE$, $E \to aEbE | bEaE | \varepsilon$}
    $L = \{\alpha \in \{a,b\}^\star\mid N_a(\alpha) = N_b(\alpha) + 1\}$;
  \item
    \marginpar{$S\to E | SaS$, $E \to aEbE | bEaE | \varepsilon$}
    $L = \{\alpha \in \{a,b\}^\star\mid N_a(\alpha) \geq N_b(\alpha)\}$;
  \item
    $L = \{\alpha \in \{a,b\}^\star\mid N_a(\alpha) > N_b(\alpha)\}$;
  \item
    $L = \{\omega_1 a \omega_2 b \mid \omega_1,\omega_2 \in \{a,b\}^\star\ \&\ \abs{\omega_1} = \abs{\omega_2}\}$;
  \item
    $L = \{\alpha \sharp \beta \mid \alpha,\beta \in \{a,b\}^\star\ \&\ \alpha^R\mbox{ е поддума на }\beta \}$.
  \item 
    $L = \{\omega_1\sharp\omega_2 \mid \omega_1,\omega_2 \in \{a,b\}^\star\ \&\ \abs{\omega_1} = \abs{\omega_2}\}$;
  \item
    $L = \{\omega_1 \sharp \omega_2 \sharp \cdots \sharp \omega_n \mid n\geq 2\ \&\ \omega_1,\omega_2,\dots,\omega_n \in \{a,b\}^\star\ \&\ \abs{\omega_1} = \abs{\omega_2}\}$;
  \item
    $L = \{\omega_1 \sharp \omega_2 \sharp \cdots \sharp \omega_n \mid n\geq 2\ \&\ \omega_1,\dots,\omega_n \in \{a,b\}^\star\ \&\ (\exists i \neq j)[\abs{\omega_i} = \abs{\omega_j}]\}$;
  \item
    $L = \{\omega_1 \sharp \omega_2 \sharp \cdots \sharp \omega_n \mid n\geq 2\ \&\ (\forall i\in[1,n])[\omega_i \in \{a,b\}^\star\ \&\ \abs{\omega_i} = \abs{\omega_{n+1-i}}]\}$.
  \end{enumerate}
\end{problem}


\begin{problem}
  Проверете дали следните езици са безконтекстни:
  \begin{enumerate}[a)]
  \item
    $\{a^nb^{2n}c^{3n}\ \mid\ n\in\Nat\}$;
  \item
    $\{a^nb^{2n}c^{n}\ \mid\ n\in\Nat\}$;
  \item
    $\{a^nb^kc^ka^n\mid\ k \leq n\}$;
  \item
    $\{a^nb^mc^k\mid n < m < k\}$;
  \item
    $\{a^nb^nc^k\mid n \leq k \leq 2n\}$;
  \item
    $\{a^nb^mc^k\mid k = \min\{n,m\}\}$;
  \item
    $\{a^nb^nc^m\mid m \leq n\}$;
  \item
    $\{a^nb^mc^k\mid k = n\cdot m\}$;
  \item
    $L^\star$, където
    $L = \{\alpha\alpha^R \mid \alpha \in \{a,b\}^\star\}$;
  \item
    $\{www\mid w\in \{a,b\}^\star\}$;
  % \item
  %   $\{ww^R\mid w\in \{a,b\}^\star\}$;
  \item
    $\{a^{n^2}b^n\ \mid n \in \Nat\}$;
  \item
    $\{a^p\ \mid\ p\mbox{ е просто }\}$;
  \item
    $\{\omega \in \{a,b\}^\star \mid \omega = \omega^R\}$;
  \item
    $\{\omega^n \mid \omega \in \{a,b\}^\star\ \&\ N_b(\omega) = 2\ \&\ n \in \Nat\}$;
  \item
    $\{\omega c^n \omega^R \mid \omega \in \{a,b\}^\star\ \&\ n = \abs{\omega}\}$;
  \item
    % Дефиниция на подниз
    $\{w c x\mid w,x\in \{a,b\}^\star\ \&\ w\mbox{ е подниз на }x\}$;
  \item
    $\{x_1 c x_2 c \dots c x_k\mid k\geq 2\ \&\ x_i\in\{a,b\}^\star\ \&\ (\exists i,j)[i \neq j\ \&\ x_i = x_j]\}$;
  \item
    $\{a^ib^jc^k\mid i,j,k\geq 0\ \&\ (i = j \vee j = k)\}$;
  \item
    % \marginpar{Разгл. $L' = L \cap L(a^*b^*c^*)$.}
    $\{\alpha \in \{a,b,c\}^\star\mid N_a(\alpha) > N_b(\alpha) > n_c(\alpha)\}$;
  \item
    $\{a,b\}^\star \setminus \{a^nb^n\mid n\in \Nat\}$;
  \item
    $\{a^nb^mc^k \mid m^2 = 2nk\}$;

  \item
    $L = \{a^nb^mc^ma^n \mid m,n\in\Nat\ \&\ n = m+42\}$;
  \item
    $L = \{babaabaaab\cdots ba^{n-1}ba^nb \mid n \geq 1\}$;
  \item
    $\{a^mb^nc^k\mid m = n \vee n = k \vee m = k\}$;
  \item
    $\{a^mb^nc^k\mid m \neq n \vee n \neq k \vee m \neq k\}$;
  \item
    $\{a^mb^nc^k\mid m = n \wedge n = k \wedge m = k\}$;
  \item
    $\{w \in \{a,b,c\}^\star\mid N_a(w) \neq N_b(w) \vee N_a(w) \neq N_c(w) \vee N_b(w) \neq N_c(w)\}$.
  \end{enumerate}
\end{problem}

\begin{problem}
  Докажете, че ако $L$ е безконтекстен език, то $L^R = \{\omega^R \mid \omega \in L\}$ 
  също е безконтекстен.
\end{problem}

\begin{problem}
  Нека $\Sigma = \{a,b,c,d,f,e\}$.
  Докажете, че езикът $L$ е безконтекстен, където за думите $\omega \in L$ са изпълнени свойствата:
  \begin{itemize}[-]
  \item 
    за всяко $n\in\Nat$, след всяко срещане на $n$ последнователни $a$-та
    следват $n$ последователни $b$-та, и $b$-та не се срещат по друг повод в $\omega$, и
  \item
    за всяко $m\in\Nat$, след всяко срещане на $m$ последнователни $c$-та
    следват $m$ последователни $d$-та, и $d$-та не се срещат по друг повод в $\omega$, и
  \item
    за всяко $k\in\Nat$, след всяко срещане на $k$ последнователни $f$-а
    следват $k$ последователни $e$-та, и $e$-та не се срещат по друг повод в $\omega$.
  \end{itemize}
\end{problem}

\begin{problem}
  Да разгледаме езиците:
  \begin{align*}
    & P = \{\alpha\in\{a,b,c\}^*\,|\, \alpha \text{ е палиндром с четна дължина}\} \\
    & L =  \{\beta b^n\,|\, n\in\mathbb{N}, \beta\in P^n\}.
  \end{align*}
  Да се докаже, че:
  \begin{enumerate}[a)]
  \item 
    $L$ не е регулярен;
  \item 
    $L$ е безконтекстен.
  \end{enumerate}
\end{problem}

\begin{problem}
  Нека $L_1$ е произволен регулярен език над азбуката $\Sigma$, 
  а $L_2$ е езика от всички думи палиндроми над $\Sigma$.
  Докажете, че $L$ е безконтекстен език, където:
  \[L = \{\alpha_1\alpha_2\cdots\alpha_{3n}\beta_1\cdots\beta_m\gamma_1\cdots\gamma_n \mid \alpha_i,\gamma_j \in L_1, \beta_k\in L_2, m,n \in \Nat\}.\]
\end{problem}

\begin{problem}
  Нека $L = \{\omega\in\{a,b\}^\star \mid N_a(\omega) = 2\}$.
  Да се докаже, че езикът $L' = \{\alpha^n \mid \alpha\in L, n \geq 0\}$ не е безконтекстен.
\end{problem}


\begin{problem}
  Нека $\Sigma = \{a,b,c\}$ и $L \subseteq \Sigma^\star$ е безконтестен език. Ако имаме дума 
  $\alpha \in \Sigma^\star$, тогава \emph{L-вариант} на $\alpha$ ще наричаме думата, която се получава като в $\alpha$ всяко едно 
  срещане на символа $a$ заменим с (евентуално различна) дума от $L$.
  Тогава, ако $M \subseteq \Sigma^*$ е произволен безконтестен език, да се докаже че езикът
  \begin{equation*}
    M' = \{\beta\in\Sigma^\star |\ \beta \text{ е $L$-вариант на } \alpha \in M \}
  \end{equation*}
  също е безконтекстен.
\end{problem}

\begin{problem}
  Докажете, че всеки безконтекстен език над азбуката $\Sigma = \{a\}$
  е регулярен.
\end{problem}

\begin{problem}
%  \marginpar{\cite{papadimitriou} стр. 149}
  Да фиксираме азбуката $\Sigma$.
  Нека $L$ е безконтекстен език, а $R$ е регулярен език.
  Докажете, че езикът
  $L/R = \{\omega \in \Sigma^\star \mid (\exists u \in R)[\omega u \in L]\}$
  е безконтекстен.
\end{problem}


\begin{problem}
  Нека е дадена граматиката $G = \pair{\{a,b\}, \{S,A,B,C\},S,R}$.
  Използвайте CYK-алгоритъма, за да проверите дали
  думата $\alpha$ принадлежи на $\L(G)$, където правилата на граматиката $R$ и думата $\alpha$
  са зададени като:
  \begin{enumerate}[a)]
  \item
    $R = \{S\rightarrow BA| CA|a, C\rightarrow BS|SA,A\rightarrow a, B\rightarrow b\}$, $\alpha=bbaaa$;
  \item
    $R =\{S\rightarrow AB|BC, A\rightarrow BA|a,B\rightarrow CC|b, C\rightarrow AB|a\}$, $\alpha=baaba$;
  \item
    $R = \{S\rightarrow AB, A\rightarrow AC|a|b,B\rightarrow CB|a, C\rightarrow a\}$, $\alpha=babaa$;
  \end{enumerate}
\end{problem}

\begin{problem}
  \marginpar{Интересно е също да се направи и безконтекстна граматика за $L$}
  Постройте стеков автомат за езика над азбуката $\{a,b,\sharp\}$:
  \[L = \{\omega_1 \sharp \omega_2 \sharp \cdots \sharp \omega_{2n} \mid n\in\Nat\ \&\ \sum^n_{i=1}\abs{\omega_{2i}} = \sum^{n}_{i=1}\abs{\omega_{2i-1}}\}.\]
\end{problem}


%%% Local Variables: 
%%% mode: latex
%%% TeX-master: "EAI"
%%% End: 



\chapter{Безконтекстни езици и стекови автомати}


\section{Безконтекстни граматики}
\index{граматика!безконтекстна}
% От Сипсер, същото е в слайдовете на Сашка
% Малко е тъпо, че в Пападимитриу дефиницията е различна. Там \Sigma \subseteq V

\begin{dfn}
  \marginpar{На англ. {\em context-free grammar}}
  \marginpar{Други срещани наименования на български са {\em контекстно-свободна}, {\em контекстно-независима}}
%  \marginpar{Според йерархията на Чомски, това са граматики тип }
  Безконтекстна граматика e четворка от вида
  \[G = (V,\Sigma,R,S),\]
  където
  \begin{itemize}
  \item
    \marginpar{Променливите се наричат също нетерминали}
    $V$ е крайно множество от {\em променливи};
  \item
    \marginpar{Буквите се наричат също терминали.}
    $\Sigma$ е крайно множество от {\em букви}, $\Sigma \cap V = \emptyset$;
  \item
    $R \subseteq V\times (V\cup\Sigma)^\star$, крайно множество от {\em правила};
  \item
    $S \in V$ е началната променлива. 
  \end{itemize}
\end{dfn}

При дадена граматика $G$, за правилата на граматиката обикновено ще пишем $A \rightarrow \alpha$ вместо $(A,\alpha) \in R$.
Ще въведем и релация между думи $\alpha,\beta\in (V \cup \Sigma)^\star$, която ще казва, че думата $\beta$
се получаава от $\alpha$ като приложим правло от граматиката.
За две думи $u,v\in (V\cup\Sigma)^\star$ ще пишем $u \rightarrow_G v$, ако съществуват думи $x,y\in (\Sigma\cup V)^\star$, $A\in V$,
правило $A\rightarrow \alpha$ и $u = xAy$, $v = x\alpha y$.
С $\rightarrow^\star_G$ ще означаваме рефлексивното и транзитивно затваряне на релацията $\rightarrow_G$.
% \marginpar{Да се дефинира $\rightarrow^\star_G$}

Езикът породен от граматиката $G$ е множеството от думи
\[\L(G) = \{\alpha\in\Sigma^\star\mid S \rightarrow^\star_G \alpha\}.\]
  
\begin{problem}
  Докажете, че езикът $L = \{a^mb^nc^k\mid m+n \geq k\}$ е безконтекстен.
\end{problem}
\begin{proof}
  Да разгледаме граматиката $G$ с правила
  \begin{align*}
    S& \rightarrow aSc\vert aS \vert B\\
    B& \rightarrow bBc\vert bB\vert\varepsilon.
  \end{align*}
  
  Лесно се вижда с индукция по $n$, че за всяко $n$ имаме свойствата:
  \marginpar{\ding{45} Докажете!}
  \begin{itemize}
  \item 
    $S \rightarrow^\star a^nSc^n$,
  \item
    $S \rightarrow^\star a^nS$,
  \item
    $B \rightarrow^\star a^nBc^n$,
  \item
    $B \rightarrow^\star b^nB$.
  \end{itemize}
  Комбинирайки горните свойства, можем да видим, че за всяко $n \geq k$,
  \begin{itemize}
  \item 
    $S \rightarrow^\star a^nSc^k$,
  \item
    $B \rightarrow^\star b^nBc^k$.
  \end{itemize}
  За да докажем, че $L \subseteq L(G)$, 
  да разгледаме една дума $\omega \in L$, т.е. $\omega = a^mb^nc^k$, където $m+n \geq k$.
  Имаме два случая:
  \begin{itemize}
  \item 
    $k \leq m$, т.е. $m = k+l$ и $m+n = k+l+n$.
    Тогава имаме изводите:
    \[S \rightarrow^\star a^kSc^k,\ S \rightarrow^\star a^lS,\ S \rightarrow B,\ B \rightarrow^\star b^nB,\ B \rightarrow \varepsilon.\]
    Обединявайки всичко това, получаваме:
    \[S \rightarrow^\star a^mb^nc^k.\]
  \item
    $k > m$, т.е. $k = m+l$, за някое $l > 0$, и $m+n = k+r = m+l+r$, за някое $r$.
    Тогава имаме изводите:
    \[S \rightarrow^\star a^mSc^m,\ S\rightarrow B,\ B\rightarrow^\star b^lBc^l,\ B\rightarrow b^rB,\ B\rightarrow\varepsilon,\]
    и отново получаваме $S \rightarrow^\star a^mb^nc^k$.
  \end{itemize}
  Така доказахме, че $\omega \in \L(G)$.
  
  Сега ще докажем, че $\L(G) \subseteq L$.
  С индукция по дължината на извода $l$,
  ще докажем, че ако $S \stackrel{l}{\rightarrow}\omega$, то $\omega \in M$, където
  \[M = \{a^nSc^k\mid n\geq k\}\cup\{a^nb^mBc^k\mid n+m\geq k\}\cup\{a^nb^mc^k\mid n+m\geq k\}.\]
  
  Ако $l = 0$, то е ясно, че $S \stackrel{0}{\rightarrow} S$ и $S \in M$.

  Нека $l > 0$ и $S \stackrel{l-1}{\rightarrow} \alpha \rightarrow \omega$.
  От {\bf И.П.} имаме, че $\alpha \in M$. Нека $\omega$ се получава от $\alpha$ с прилагане на правилото $C \rightarrow \gamma$.
  Разглеждаме всички варианти за думата $\alpha \in M$ и за правилото $C\rightarrow \gamma$ в граматиката $G$
  за да докажем, че  $\omega \in M$.
  Удобно е да представим всички случаи в таблица.
  \begin{center}
    \begin{tabular}{| c | c | c |}
      \hline
      $\alpha\in M$ & $C \rightarrow \gamma$ & $\omega \in M?$ \\ \hline
      $a^nSc^k$ & $S \rightarrow aSc$ & $a^{n+1}Sc^{k+1}$ \\ \hline
      $a^nSc^k$ & $S \rightarrow aS$ & $a^{n+1}Sc^{k}$ \\ \hline
      $a^nSc^k$ & $S \rightarrow B$ & $a^{n}Bc^{k}$ \\ \hline
      $a^nb^mBc^k$ & $B \rightarrow bBc$ & $a^nb^{m+1}Bc^{k+1}$\\ \hline
      $a^nb^mBc^k$ & $B \rightarrow bB$ & $a^nb^{m+1}Bc^{k}$\\ \hline
      $a^nb^mBc^k$ & $B \rightarrow \varepsilon$ & $a^nb^{m}c^{k}$\\ \hline
    \end{tabular}
  \end{center}
  Във всички случаи се установява, че $\omega \in M$.
  Сега, за всяка дума $\omega \in L(G)$ следва, че
  \[\omega \in \Sigma^\star \cap M = \{a^mb^nc^k\mid m+n \geq k\}.\]
\end{proof}


\begin{problem}
  \marginpar{
    $S \to aS \mid aSc \mid aB \mid bB$\\
    $B \to bB \mid bBc \mid \varepsilon$
}
  Докажете, че езикът $L = \{a^mb^nc^k\mid m+n \geq k + 1\}$ е безконтекстен.  
\end{problem}

\begin{problem}
  \label{pr:nanb}
  Нека $\omega$ е произволна дума над азбуката $\{a,b\}$. 
  Тогава:
  \begin{enumerate}[a)]
  \item 
    ако $n_a(\omega) = n_b(\omega) + 1$, то съществуват думи $\omega_1$, $\omega_2$, за които
    $\omega = \omega_1 a \omega_2$, $n_a(\omega_1) = n_b(\omega_1)$ и $n_a(\omega_2) = n_b(\omega_2)$.
  \item
    ако $n_b(\omega) = n_a(\omega) + 1$, то съществуват думи $\omega_1$, $\omega_2$, за които
    $\omega = \omega_1 b \omega_2$, $n_a(\omega_1) = n_b(\omega_1)$ и $n_a(\omega_2) = n_b(\omega_2)$.
  \end{enumerate}
\end{problem}
\begin{proof}
  Пълна индукция по дължината на думата $\omega$, за които $n_a(\omega) = n_b(\omega)+1$.
  \begin{itemize}
  \item 
    $\abs{\omega} = 1$. Тогава $\omega_1 = \omega_2 = \varepsilon$ и $\omega = a$.
  \item
    $\abs{\omega} = n+1$. Ще разгледаме два случая, в зависимост от първия символ на $\omega$.
    \begin{itemize}
    \item 
      Случаят $\omega = a\omega'$ е лесен. (Защо?)
    \item
      Интересният случай е $\omega = b\omega'$.    
      Тогава $\omega = b^{i+1}a\omega'$. Да разгледаме думата $\omega''$, която се получава от $\omega$
      като премахнем първото срещане на думата $ba$, т.е. 
      $\omega'' = b^i\omega'$ и $\abs{\omega''} = n-1$.
      Понеже от $\omega$ сме премахнали равен брой $a$-та и $b$-та, $n_a(\omega'') = n_b(\omega'')+1$.
      Според {\bf И.П.} за $\omega''$, можем да запишем думата като $\omega'' = \omega''_1a\omega''_2$
      и $n_a(\omega''_1) = n_b(\omega''_1)$, $n_a(\omega''_2) = n_b(\omega''_2)$.
      Понеже $b^i$ е префикс на $\omega''_1$, за да получим обратно $\omega$, трябва 
      да прибавим премахнатата част $ba$ веднага след $b^i$ в $\omega''_1$.
    \end{itemize}
  \end{itemize}
\end{proof}

\begin{problem}
  За произволна дума $\omega \in \{a,b\}^\star$, 
  докажете, че ако $n_a(\omega) > n_b(\omega)$, то съществуват думи $\omega_1$ и $\omega_2$,
  за които $\omega = \omega_1 a \omega_2$ и $n_a(\omega_1) \geq n_b(\omega_1)$, $n_a(\omega_2) \geq n_b(\omega_2)$.
\end{problem}

\begin{problem}
  Да се докаже, че езикът $L = \{\alpha \in \{a,b\}^\star\mid n_a(\alpha) = n_b(\alpha)\}$ 
  е безконтекстен.
\end{problem}
\begin{proof}
  \marginpar{  Алтернативна граматика за езика $L$ е
  \begin{align*}
    S& \rightarrow aB\vert bA\\
    A& \rightarrow a\vert aS\vert bAA\\
    B& \rightarrow b\vert bS\vert aBB
  \end{align*}}
  Една възможна граматика $G$ е следната: 
  \[S \rightarrow aSbS\vert bSaS \vert\varepsilon.\]
  Например, да разгледаме извода на думата $aabbba$ в тази граматика:
  \begin{align*}
    S & \to aSbS \to aaSbSbS \to aa\varepsilon bSbS \to aab\varepsilon bS \to aabbbSaS\\
    & \to aabbb\varepsilon a S \to aabbba.
  \end{align*}
  
  Като следствие от \Prob{nanb} може лесно да се изведе, че за думи $\omega$, за които $n_a(\omega) = n_b(\omega)$,
  е изпълнено следното:
  \begin{enumerate}[a)]
  \item 
    ако $\omega = a\omega'$, то
    $\omega = a\omega_1b\omega_2$ и $n_a(\omega_1) = n_b(\omega_1)$, $n_a(\omega_2) = n_b(\omega_2)$;
  \item
    ако $\omega = b\omega'$, то
    $\omega = b\omega_1a\omega_2$ и $n_a(\omega_1) = n_b(\omega_1)$, $n_a(\omega_2) = n_b(\omega_2)$.
  \end{enumerate}

  Сега първо ще проверим, че $L \subseteq L(G)$.
  За целта ще докажем с {\em пълна индукция} по дължината на думата $\omega$, че за всяка дума $\omega$ със свойството $n_a(\omega) = n_b(\omega)$ е изпълнено
  $S \rightarrow^\star \omega$.
  \begin{itemize}
  \item 
    Нека $\abs{\omega} = 0$. Тогава $S \rightarrow \varepsilon$.
  \item
    Нека $\abs{\omega} = k+1$. Имаме два случая.
    \begin{itemize}
    \item 
      $\omega = a\omega^\prime$, т.е. от свойство а), $\omega = a\omega_1b\omega_2$ и $n_a(\omega_1) = n_b(\omega_1)$, $n_a(\omega_2) = n_b(\omega_2)$.
      Тогава $\abs{\omega_1} \leq k$ и по И.П. $S \rightarrow^\star \omega_1$.
      Аналогично, $S \rightarrow^\star \omega_2$.
      Понеже имаме правило $S \rightarrow aSbS$, заключаваме че $S \rightarrow^\star a\omega_1b\omega_2$.
    \item
      $\omega = b\omega^\prime$, т.е. свойство б), $\omega = b\omega_1a\omega_2$ и $n_a(\omega_1) = n_b(\omega_1)$, $n_a(\omega_2) = n_b(\omega_2)$.
      Този случай се разглежда аналогично.
    \end{itemize}
  \end{itemize}
  
  Преминаваме към доказателството на другата посока, т.е. $L(G) \subseteq L$.
  Тук с индукция по дължината на извода $l$ ще докажем, че
  $S \stackrel{l}{\rightarrow} \omega$, то $\omega \in M$,
  където
  \[M = \{\omega \in \{a,b,S\}^\star \mid n_a(\omega) = n_b(\omega)\}.\]
  За $l = 0$  е ясно, че $S \stackrel{0}{\rightarrow^\star} S$.
  За $l = k+1$, то $S \stackrel{k}{\rightarrow^\star} \alpha \rightarrow \omega$.
  От {\bf И.П.} имаме, че $\alpha \in M$.
  Нека $\omega$ се получава от $\alpha$ с прилагане на правилото $C \rightarrow \gamma$.
  Разглеждаме всички варианти за думата $\alpha \in M$ и за правилото $C\rightarrow \gamma$ в граматиката $G$
  за да докажем, че  $\omega \in M$.
  Удобно е да представим всички случаи в таблица.
  \begin{center}
    \begin{tabular}{| c | c | c |}
      \hline
      $\alpha$ & $C \rightarrow \gamma$ & $\omega$ \\ \hline
      $\in M$ & $S \rightarrow aSbS$ & $\in M$ \\ \hline
      $\in M$ & $S \rightarrow bSaS$ & $\in M$ \\ \hline
      $\in M$ & $S \rightarrow \varepsilon$ & $\in M$ \\ \hline
    \end{tabular}
  \end{center}
  Във всички случаи лесно се установява, че $\omega \in M$.
  Така за всяка дума $\omega \in L(G)$ следва, че
  \[\omega \in \Sigma^\star \cap M = L.\]
\end{proof}

\begin{problem}
  Докажете, че следните езици са безконтекстни.
  \begin{enumerate}[a)]
  \item
    \marginpar{$S \rightarrow aSa\ \vert\ bSb\ \vert\ \varepsilon$}
    $L = \{ww^R \mid w \in \{a,b\}^\star\}$;
  \item
    \marginpar{$S \rightarrow aSa\ \vert\ bSb\ \vert\ a\vert\ b\ \vert\ \varepsilon$}
    $L = \{w \in \{a,b\}^\star \mid w = w^R\}$;
  \item
    $L = \{a^nb^{2m}c^{n} \mid m,n \in \Nat\}$;
  \item
    $L = \{a^nb^{m}c^{m}d^n \mid m,n \in \Nat\}$;
  \item
    $L = \{a^nb^{2k} \mid n,k \in \Nat\ \&\ n \neq k\}$;
  \item
    \marginpar{$S \rightarrow aSb | aS | a$}
    $L = \{a^nb^k \mid n > k\}$;
  \item
    $L = \{a^nb^k \mid n \geq 2k\}$;
  \item
    \marginpar{$S \rightarrow aSc | B,\ B \rightarrow bBc | \varepsilon$}
    $L = \{a^nb^mc^{n+m}\mid n,m \in \Nat\}$;
  \item
    \marginpar{$S \rightarrow aSc | aS | B$, $B\rightarrow bBc | bB | \varepsilon$}
    $L = \{a^nb^kc^m \mid n + k \geq m\}$;
  \item
    \marginpar{$S \rightarrow aSc | aS | aB | bB$,\\$B\rightarrow bBc | bB | \varepsilon$}
    $L = \{a^nb^kc^m \mid n + k \geq m+1\}$;
  \item
    $L = \{a^nb^kc^m \mid n + k \geq m+2\}$;
  \item
    \marginpar{$S \rightarrow aSc | aS | B | Bc$,\\$B\rightarrow bBc | bB | \varepsilon$}
    $L = \{a^nb^kc^m \mid n + k + 1 \geq m\}$;
  \item
    $L = \{a^nb^kc^m \mid n + k + 2 \geq m\}$;
  \item
    $L = \{a^nb^kc^m \mid n + k \leq m\}$;
  \item
    $L = \{a^nb^kc^m \mid n + k \leq m+1\}$;
  \item
    \marginpar{Обединение на три езика}
    $L = \{a^nb^mc^k \mid n, m, k \text{ не са страни на триъгълник}\}$.
  \item
    $L = \{a,b\}^\star \setminus \{a^{2n}b^n \mid n\in\Nat\}$;
  \item
    \marginpar{$S\to EaE$, $E \to aEbE | bEaE | \varepsilon$}
    $L = \{\alpha \in \{a,b\}^\star\mid n_a(\alpha) = n_b(\alpha) + 1\}$;
  \item
    \marginpar{$S\to E | SaS$, $E \to aEbE | bEaE | \varepsilon$}
    $L = \{\alpha \in \{a,b\}^\star\mid n_a(\alpha) \geq n_b(\alpha)\}$;
  \item
    $L = \{\alpha \in \{a,b\}^\star\mid n_a(\alpha) > n_b(\alpha)\}$;
  \item
    $L = \{\omega_1 a \omega_2 b \mid \omega_1,\omega_2 \in \{a,b\}^\star\ \&\ \abs{\omega_1} = \abs{\omega_2}\}$;
  \item
    $L = \{\alpha c \beta \mid \alpha,\beta \in \{a,b\}^\star\ \&\ \alpha^R\mbox{ е поддума на }\beta \}$.
  \end{enumerate}
\end{problem}

\begin{problem}
  \marginpar{от Владислав}
  Да разгледаме граматиката $G = \CFG$, където  $V = \{S,A,B\}$, $\Sigma = \{a,b\}$, а правилата $R$ са
  \[S \to AA | B, A \to B | bb, B \to aa | aB.\]
  Да се намери езика на тази граматика и да се докаже, че граматиката разпознава точно този език.
\end{problem}



\section{Езици, които не са безконтекстни}

\begin{lemma}[за покачването (безконтекстни езици)]
  \index{лема за покачването!безконтекстни езици}
  \label{lem:pumping-context} 
  \marginpar{(стр. 123 от \cite{sipser1}; стр. 125 от \cite{hopcroft1})}
  За всеки безконтекстен език $L$ съществува $p>0$, такова
  че ако $\alpha\in L, \abs{\alpha} \geq p$, то съществува разбиване на думата на пет части, $\alpha=xyuvw$,
  за което е изпълнено:
  \begin{enumerate}[1)]
  \item
    $\abs{yv}\geq 1$,
  \item
    $\abs{yuv}\leq p$, и
  \item
    $(\forall i\geq 0)[xy^iuv^iw\in L]$.
\end{enumerate}
\end{lemma}
\begin{proof}
  Нека $G$ е граматиката за езика $L$.
  \marginpar{За простота, можем да си мислим, че $G$ е в НФЧ. Тогава $b=2$.}
  Нека \[b = \max\{\abs{\beta} \mid A\rightarrow_G \beta\}.\]
  Можем да приемем, че $b \geq 2$.
  \marginpar{Възлите във вътрешността на дървото са променливи, а листата са букви или $\varepsilon$}
  Това означава, че във всяко дърво на извод, всеки възел има
  не повече от $b$ наследника.
  Нека $p = b^{\abs{V}}+1$. Ще покажем, че $p$ е константа на покачването за граматиката $G$.
  Това означава, че всяка дума с дължина поне $p$ в езика $L$ има дърво на извод с височина
  поне $\abs{V} + 1$.
  
  Нека $\abs{\alpha} \geq p$ и $T$ е дърво на извода за думата $\alpha$.
  Понеже думата $\alpha$ може да има много дървета на извод, нека $T$ също така да бъде {\em с минимален брой възли}. 
  От направените по-горе разсъждения е ясно, че височината на $T$ е поне $\abs{V} + 1$,
  Следователно, по най-дългия път $\pi$ в $T$ имаме поне $\abs{V}+2$ възела, от които
  поне $\abs{V}+1$ са променливи, защото само листата могат да не са променливи.
  Да разгледаме последните $\abs{V}+1$ променливи по пътя $\pi$.
  От принципа на Дирихле следва, че измежду тези $\abs{V}+1$ променливи има поне една повтаряща се.
  Нека $R$ да бъде една такава променлива.
  Последните две повтаряния на $R$ разделят думата $\alpha$ на пет части.
  Нека $\alpha = xyuvw$.
  \begin{enumerate}[1)]
  \item
    $\abs{yv}\geq 1$,
    защото ако допуснем, че $\abs{yv} = 0$,
    то ще достигнем до противоречие с минималността на $T$.
  \item
    $\abs{yuv} \leq p$, защото сме избрали най-долното $R$.
  \item
    $xy^iuv^iw \in L$, защото можем да заменим поддървото 
    с корен последното $R$ за поддървото с корен предпоследното $R$.
    В случая $i = 0$, правим обратното.
  \end{enumerate}
\end{proof}

\begin{cor}
  \label{cor:pumping-context-free}
  \marginpar{Ако $L$ е краен език, то е ясно, че $L$ е безконтекстен.}
  Нека $L$ е произволен {\bf безкраен} език. Нека също така е изпълнено, че за всяко естествено число $p \geq 1$ можем да намерим дума $\alpha \in L$, $\abs{\alpha}\geq p$, такава че за всяко разбиване на думата на пет части, $\alpha = xyuvw$,
  със свойствата $\abs{yv} \geq 1$ и $\abs{yuv} \leq p$, е изпълнено, че $(\exists i)[xy^iuv^iw \not\in L]$.
  \marginpar{\writedown Докажете! Аналогично е на \Cor{pumping-reg}}
  Тогава $L$ {\bf не} е безконтекстен език.
\end{cor}

\begin{cor}
  \marginpar{\writedown Докажете!}
  Нека $G$ е безконтекстна граматика и $p$ е константата на покачването за $G$, $L = \L(G)$.
  Тогава $\abs{L} = \infty$ точно тогава, когато съществува $\alpha \in L$, за която $p \leq \abs{\alpha} < 2p$.
\end{cor}
% \begin{proof}
%   Ако съществува дума $\alpha \in L$, за която $\abs{\alpha} \geq p$, то от \Lem{pumping-context} следва,
%   че $\abs{L} = \infty$, защото $\alpha = xyuvw$ и $xy^iuv^iw \in L$, за всяко $i\in\Nat$.

%   За другата посока, нека сега $\abs{L} = \infty$.
%   Да изберем най-късата дума $\alpha \in L$, за която $\abs{\alpha} \geq p$.
%   Ще докажем, че $p \leq \abs{\alpha} < 2p$. За целта да допуснем, че $\abs{\alpha} \geq 2p$.
%   Тогава от \Lem{pumping-context} следва, че $\alpha = xyuvw$, $\abs{yv} \geq 1$, $\abs{yuv} \leq p$, $xy^0uv^0w = xuw \in L$.
%   Ако $\abs{xuw} < p$, то $\abs{yv} > p$, защото $\abs{yv} + \abs{xuw} = \abs{\alpha} \geq 2p$, и следователно $\abs{yuv} > p$, което е противоречие.
%   Следва, че $\abs{\alpha} > \abs{xuw} \geq p$.
%   Получихме, че думата $xuw\in L$ и $\abs{xuw} \geq p$. Това е противоречие с минималността на $\alpha$.
% \end{proof}

% \begin{framed}
%   \Lem{pumping-context} е полезна, когато искаме да докажем, че даден език $L$ {\bf не} е безконтекстен.
%   За целта, доказваме отрицанието на свойствата от \Lem{pumping-context} за $L$, т.е.
%   за всяка константа $p$, ние намираме дума $\alpha \in L$, $\abs{\alpha}\geq p$, такава че за всяко разбиване на думата на пет части, $\alpha = xyuvw$,
%   със свойствата $\abs{yv} \geq 1$ и $\abs{yuv} \leq p$, е изпълнено, че $(\exists i)[xy^iuv^iw \not\in L]$.
% \end{framed}


\begin{example}
  \label{example:anbncn}
  Езикът $L = \{a^nb^nc^n\ \mid\ n\in\Nat\}$ не е безконтекстен.
\end{example}
\begin{proof}
  \begin{itemize}
  \item 
    Разглеждаме произволна константа $p \geq 1$.
  \item
    Избираме дума $\alpha \in L$, $\abs{\alpha} \geq p$.
    В случая, нека $\alpha = a^pb^pc^p$.
  \item
    Разглеждаме произволно разбиване $xyuvw = \alpha$, за което $\abs{xyv} \leq p$ и $1 \leq \abs{yv}$.
  \item
    Трябва да изберем $i$, за което $xy^iuv^iw \not\in L$.
    Знаем, че поне едно от $y$ и $v$ не е празната дума.
    Имаме няколко случая за $y$ и $v$.
    \begin{itemize}
    \item
      $y$ и $v$ са думи съставени от една буква.
      В този случай получаваме, че $xy^2uv^2w$ има различен брой букви $a$, $b$ и $c$.
    \item
      $y$ или $v$ е съставена от две букви.
      Тогава е възможно да се окаже, че $xy^2uv^2w$ да има равен брой $a$, $b$ и $c$,
      но тогава редът на буквите е нарушен.
    \item
      понеже $\abs{yuv} \leq p$, то не е възможно в $y$ или $v$ да се срещат и трите букви.
    \end{itemize}  
    Оказа се, че във всички възможни случаи за $y$ и $v$, 
    $xy^2uv^2w \not\in L$.
  \end{itemize}
  Така от \Cor{pumping-context-free} следва, че езикът $L$ не е безконтекстен.
\end{proof}

% \begin{problem}
%   Да се даде пример за език $L$, който {\bf не} е безконтекстен, но удовлетворява
%   лемата за разрастването.
% \end{problem}

\begin{example}
  Приложете лемата за покачването за да докажете, че
  езикът $L$ не е безконтекстен, където:
  \begin{enumerate}[a)]
  \item
    $L = \{a^ib^jc^k\ \mid\ 0 \leq i \leq j \leq k\}$;
  \item
    $L = \{\beta\beta\mid \beta\in \{a,b\}^\star\}$;
  \item
    $L = \{a^{n^2}\mid n\in\Nat\}$.
  \end{enumerate}
\end{example}
\begin{proof}
  \begin{enumerate}[a)]
  \item
    Да фиксираме думата $\alpha = a^pb^pc^p$ и да разгледаме
    едно произволно нейно разбиване, $\alpha = xyuvw$, за което
    $\abs{yuv} \leq p$ и $1 \leq \abs{yv}$.
    Знаем, че поне една от $y$ и $v$ не е празната дума.
    \begin{itemize}
    \item
      $y$ и $v$ са съставени от една буква.
      Имаме три случая.
      \begin{enumerate}[i)]
      \item
        $a$ не се среща в $y$ и $v$.
        Тогава $xy^0vu^0w$ съдържа повече $a$ от $b$ или $c$.
      \item
        $b$ не се среща в $y$ и $v$.
        Ако $a$ се среща в $y$ или $v$, тогава $xy^2uv^2w$ съдържа повече $a$ от $b$
        Ако $c$ се среща в $y$ или $v$, тогава $xy^0uv^0w$ съдържа по-малко $c$ от $b$.
      \item
        $c$ не се среща в $y$ и $v$.
        Тогава $xy^2uv^2w$ съдържа повече $a$ или $b$ от $c$.
      \end{enumerate}      
    \item
      $y$ или $v$ е съставена от две букви.
      Тук разглеждаме $xy^2uv^2w$ и съобразяваме, че редът на буквите е нарушен.
    \end{itemize}
  \item
    \marginpar{Защо $\alpha = a^pba^pb$ не е добър кандидат?}
    Разгледайте $\alpha = a^pb^pa^pb^p$, т.е. $\beta = a^pb^p$ и $\alpha = \beta\beta$.
    Нека $xyuvw = \alpha$ е произволно разбиване на $\alpha$, за което е изпълнено, че
    $\abs{yuv} \leq p$ и $1\leq \abs{yv}$.
    \begin{itemize}
    \item
      Ако $yuv$ е в първата част на думата, то 
      $xy^0uv^0w = a^ib^ja^pb^p \not\in L$.
      Аналогично ако $yuv$ е във втората част на думата.
    \item
      Ако $yuv$ е в двете части на думата, то 
      $xy^0uv^0w = a^pb^ia^jb^p \not\in L$.
    \end{itemize}    
  \item
    Решава се аналогично както за регулярни езици.
  \end{enumerate}
\end{proof}


\begin{thm}
  Безконтекстните езици {\bf не} са затворени относно сечение и допълнение.
\end{thm}
\begin{proof}
  Да разгледаме езика
  \[L_0 = \{a^nb^nc^n\mid n\in\Nat\},\] за който вече знаем от Пример \ref{example:anbncn}, че не е безконтекстен.
  Да вземем също така и безконтекстните езици 
  \marginpar{\writedown Защо са безконтекстни?}
  \[L_1 = \{a^nb^nc^m\mid n,m\in\Nat\},\ L_2 = \{a^mb^nc^n\mid n,m\in\Nat\},\]
  \begin{itemize}
  \item 
    Понеже $L_0 = L_1\cap L_2$, то заключаваме, че безконтекстните езици не са затворени 
    относно операцията сечение.
  \item
    \marginpar{Озн. $\ov{L} = \Sigma^\star \setminus L$}
    Да допуснем, че безконтекстните езици са затворени относно операцията допълнение.
    Тогава  $\ov{L}_1$ и $\ov{L}_2$ са безконтекстни.
    Знаем, че безконтекстните езици са затворени относно обединение. 
    Следователно, езикът $L_3 = \ov{L}_1 \cup \ov{L}_2$ също е безконтекстен.
    Ние допуснахме, че безконтекстните са затворени относно допълнение, следователно $\ov{L}_3$
    също е безконтекстен.
    Но тогава получаваме, че езикът
    \[L_0 = L_1 \cap L_2 = \ov{\ov{L}_1 \cup \ov{L}_2} = \ov{L}_3\]
    е безконтекстен, което е противоречие.
  \end{itemize}
\end{proof}


\section{Алгоритми}

\subsection{Опростяване на безконтекстни граматики}

\subsubsection*{Премахване на безполезните променливи}

Нека е дадена безконтекстната граматика $G = \CFG$.
\marginpar{\cite{hopcroft1} стр. 88}
Една променлива $A$ се нарича {\bf полезна}, ако съществува извод от следния вид:
\[S \to^\star \alpha A \beta \to^\star \gamma,\]
където $\gamma \in \Sigma^\star$, а $\alpha,\beta \in (V \cup \Sigma)^\star$.
Това означава, че една променлива е полезна, ако участва в извода на някоя дума в езика на граматиката.
Една променлива се нарича {\bf безполезна}, ако не е полезна.
Целта ни е да получим еквивалентна граматика $G'$ без безполезни променливи.
Ще решим задачата като разгледаме две леми.

\begin{lemma}
  \label{lem:useless1}
  Нека е дадена безконтекстната граматика $G = \CFG$ и $\L(G) \neq \emptyset$.
  Съществува алгоритъм, който намира граматика $G' = \pair{V',\Sigma,S,R'}$, за която 
  $\L(G) = \L(G')$, и за всяка променлива $A' \in V'$, съществува дума $\alpha \in \Sigma^\star$,
  за която $A' \to^\star \alpha$.
\end{lemma}
\begin{proof}
  Да разгледаме следната проста итеративна процедура.
  \begin{algorithm}[H]
    \caption{Намираме $V' = \{A \in V\mid (\exists \alpha \in \Sigma^\star)[A \to^\star \alpha]\}$}
    \label{alg:useless}
    \begin{algorithmic}[1]
      \State $V' := \emptyset$
      \State $V'' := \{A \in V \mid (\exists \alpha \in \Sigma^\star)[A \to \alpha]\}$
      \While{$V' \neq V''$}
      \State $V' := V''$
      \State $V'' := V' \cup \{A \in V \mid (\exists \alpha \in (\Sigma \cup V')^\star)[A \to \alpha]\}$
      \EndWhile
      \State \Return $V'$
    \end{algorithmic}
  \end{algorithm}
  Трябва да докажем, че във $V'$ са точно полезните променливи за $G$.
  Очевидно е, че ако $A \in V'$, то $A$ е полезна променлива.
  \marginpar{\writedown Докажете!}
  За другата посока, с индукция по дължината на извода се доказва, че ако $A \to^\star_G \omega$,
  то $A \in V'$.
  
  Правилата на $G'$ са всички правила на $G$, в които участват променливи от $V'$ и букви от $\Sigma$.
\end{proof}

\begin{lemma}
  \label{lem:useless2}
  Съществува алгоритъм, който по дадена безконтекстна граматика $G = \CFG$, намира $G' = \pair{V',\Sigma',S,R'}$, $\L(G') = \L(G)$,
  със свойството, че за всяко $x \in V' \cup \Sigma'$ съществуват $\alpha, \beta \in (V'\cup\Sigma')^\star$,
  за които $S \to^\star \alpha x \beta$,
  т.е. всяка променлива или буква в $G'$ е достижима от началната променлива $S$.
\end{lemma}
\begin{proof}
  Намираме $V'$ и $\Sigma'$ итеративно, като в началото $V' = \{S\}$, $\Sigma' = \emptyset$.
  Ако $A \in V'$ и имаме правила $A \to \alpha_0 | \alpha_1 | \dots | \alpha_n$ в $G$,
  то за всяко $i = 0,\dots,n$ добавяме всички променливи на $\alpha_i$ към $V'$ и всички нетерминали на $\alpha_i$ към $\Sigma'$.
\end{proof}

\begin{thm}
  Всеки непразен безконтекстен език $L$ се поражда от безконтекстна граматика $G$
  без безполезни правила.
\end{thm}
\begin{proof}
  \marginpar{Защо е важна последователността на прилагане?}
  Нека е дадена безконтекстна граматика $G$ пораждаща $L$.
  Прилагаме върху $G$ първо процедурата от \Lem{useless1} и след това върху резултата прилагаме процедурата от \Lem{useless2}.
\end{proof}

\subsubsection*{Премахване на $\varepsilon$-правила}
\index{$\varepsilon$-правила}
За да премахнем правилата от вида $A \to \varepsilon$, следваме процедурата:
\marginpar{Броят на правилата може да се увеличи експоненциално, защото в най-лошия случай извеждаме всички подмножества на дадено множество от променливи}
\begin{enumerate}[1)]
\item 
  Намираме множеството $E = \{A \in V \mid A \to^\star \varepsilon\}$ по следния начин.
  Първо, $E := \{A \in V \mid A \to \varepsilon\}$.
  След това, за всяко правило от вида $B \to X_1\cdots X_k$, 
  ако всяко $X_i \in E$, то добавяме $B$ към $E$.
\item
  Строим множеството от правила $R'$, в което няма правила $\varepsilon$-правила по следния начин.
  За всяко правило $A \to X_1\cdots X_k$ в $R$,
  добавяме към $R'$ всички правила от вида $A \to \alpha_1\cdots\alpha_k$, където:
  \begin{itemize}[-]
  \item 
    ако $X_i \not\in E$, то $\alpha_i = X_i$;
  \item
    ако $X_i \in E$, то $\alpha_i = X_i$ или $\alpha_i = \varepsilon$;
  \item
    не всички $\alpha_i$-та са $\varepsilon$.
  \end{itemize}
\end{enumerate}

\begin{example}
  Нека е дадена граматиката $G$ с правила
  \[S\rightarrow D,D\rightarrow AD|b,A\rightarrow AB|BC|a, B\rightarrow AA|EC,C\rightarrow \varepsilon|CA|a, E\rightarrow \varepsilon|aEb.\]
  Тогава $E = \{X \in V \mid X \rightarrow^\star_G \varepsilon\} = \{A,B,C,E\}$.
  Това означава, че $\varepsilon \not\in \L(G)$.
  Граматиката $G'$ без $\varepsilon$-правила, за която $\L(G') = \L(G)$ има следните правила
  $S \to D, D\to AD|D|b, A \to A|B|C|AB|BC|a,B\to A|E|C|AA|EC, C \to C|A|CA|a, E \to aEb|ab$.
\end{example}

\subsubsection*{Премахване на преименуващи правила}
\index{преименуващи правила}
Преименуващите правила са от вида $A \to B$.
Нека е дадена граматика $G = \CFG$, в която има преименуващи правила.
Ще построим еквивалентна граматика $G'$ без преименуващи правила.
В началото нека в $R'$ да добавим всички правила от $R$, които не са преименуващи.
След това, за всякa променлива $A$, за която $A \to^\star_G B$,
ако $B \to \alpha$ е правило в $R$, което не е преименуващо,
то добавяме към $R'$ правилото $A \to \alpha$.

\begin{example}
  Нека е дадена граматиката $G$ с правила  
  \[A\rightarrow B|S,B\rightarrow C|BC,C\rightarrow AB|a|b,S\rightarrow B|CC|b.\]
  Първо добавяме към $R'$ правилата $B \to BC, C \to AB|a|b, S \to CC|b$.
  \begin{itemize}
  \item 
    Лесно се съобразява, че $A \to^\star_G B,S,C$.
    Добавяме правилата $A \to BC|AB|a|b|CC$.
  \item
    Имаме $B \to^\star_G C$.
    Добавяме правилата $B \to AB|a|b$.
  \item
    Имаме $S \to^\star_G B,C$.
    Добавяме правилата $S \to BC|AB|a|b$.
  \end{itemize}
  Накрая получаваме, че граматиката $G'$ има правила
  $A \to BC|AB|a|b|CC, B \to AB|a|b|BC, C \to AB|a|b, S \to BC|AB|CC|a|b$.
\end{example}

\subsection{Нормална Форма на Чомски}

\begin{dfn}
%[стр. 99 от \cite{sipser}]
\index{Нормална форма на Чомски}
Една безконтекстна граматика е в {\em нормална форма на Чомски}, ако
всяко правило е от вида
\[A \rightarrow BC\mbox{ и }A \rightarrow a,\]
като $B, C$ {\em не могат} да бъдат променливата за начало $S$.
Освен това, позволяваме правилото $S\to\varepsilon$.
\footnote{На стр. 151 в \cite{papadimitriou} дефиницията е малко по-различна.
Там дефинират $G$ да бъде в нормална форма на Чомски ако $R \subseteq V\times(V\cup\Sigma)^2$.
В този случай губим езиците $\{\varepsilon\}$ и $\{a\}$, за $a\in\Sigma$.}
\end{dfn}

\begin{thm}
  Всеки безконтекстен език $L$ е генериран от контекстно-свободна
  граматика в нормална форма на Чомски.
\end{thm}
\begin{proof}
%  \marginpar{Броят на правилата може да се увеличи експоненциално.}
  Нека имаме контекстно-свободна граматика $G$, за която $L = L(G)$.
  Ще построим контекстно-свободна граматика $G^\prime$ в нормална форма на Чомски, $L = L(G^\prime)$.
  % [стр. 99 от \cite{sipser}]
  Следваме следната процедура:
  \begin{itemize}
  \item
    Добавяме нов начален символ $S_0$ и правило $S_0 \to S$.
  \item
    \marginpar{Време $O(n)$}
    Съкращаваме дължината на правилата.
    Заменяме правилата от вида $A\to u_1\dots u_n$, $n\geq 3$, $u_i \in V\cup\Sigma$, с
    правилата \[A\to u_1A_1,\ A_1\to u_2A_2,\ \dots,\ A_{n-2} \to u_{n-1}u_n.\]
    където $A_i$ са нови променливи.
  \item
    \marginpar{Време $O(n^2)$}
    За всяка променлива $A \neq S_0$ премахваме правилата от вида $A\to\varepsilon$.
    Това правим по следния начин.
    
    Ако имаме правило от вида $R \to Au$ или $R\to u A$, $u \in V \cup \Sigma$,
    то добавяме правилото $R\to u$.
    %Правим това за всяко срещане на променливата $A$ в дясната страна на правило.
    Например, 
    \begin{itemize}
    \item 
      ако имаме правило $R\to aA$, то добавяме правилото $R \to a$;
    \item
      ако имаме правило $R\to AA$, то добавяме правилото $R \to A$.
    \end{itemize}
    Ако имаме правило от вида $R\to A$, то добавяме правилото $R\to\varepsilon$
    само ако променливата $R$ още не е преминала през процедурата за премахване на $\varepsilon$.
  \item
    \marginpar{Време $O(n^2)$}
    \marginpar{Памет $O(n^2)$}
    Премахваме преименуващите правила, т.е. правила от вида $A\to B$.
    Заменяме всяко правило от вида $B \to \beta$ с $A\to \beta$,
    освен ако $A \to \beta$ е вече премахнато преименуващо правило.
  \item
    \marginpar{Време $O(n)$}
    За правила от вида $A\to u_1 u_2$, където $u_1, u_2 \in V \cup \Sigma$, 
    заменяме всяка буква $u_i$ с новата променлива $U_i$
    и добавяме правилото $U_i\to u_i$.
    Например, правилото $A \to aB$ се заменя с правилото $A \to XB$ и добавяме правилото $X \to a$,
    където $X$ е нова променлива.
  \end{itemize}
\end{proof}

\begin{thm}
  При дадена безконтекстна граматика $G$ с дължина $n$, можем да намерим еквивалентна
  на нея граматика $G'$ в нормална форма на Чомски за време $O(n^2)$,
  като получената граматика е с дължина $O(n^2)$.
\end{thm}


% \begin{problem}
%   Нека е дадена граматиката  $G = \pair{\{S,A,B,C,D,E\}, \{a,b\},S, R}$.
%   \begin{enumerate}[a)]
%   \item
%     Намерете множеството $\{X \in V \mid X \rightarrow^\star_G \varepsilon\}$.
%   \item
%     Вярно ли е, че $\varepsilon \in L(G)$?
%   \item
%     Постройте граматика $G_1$ без $\varepsilon$-правила, за която $L(G_1)=L(G)\setminus\{\varepsilon\}$.
%   \end{enumerate}
%   Множеството от правила $R$ на граматиката $G$ е зададено като:
%   \begin{enumerate}[a)]
%   \item
%     $R = \{S\rightarrow D,D\rightarrow AD|b,A\rightarrow ACB|BC|a, B\rightarrow ABCA|CEC,C\rightarrow \varepsilon|CA|a, E\rightarrow \varepsilon|aEb\}$;
%   \item
%     $R = \{S \rightarrow aD, D\rightarrow \varepsilon|ABBA|ADD,A\rightarrow DEB|a,B\rightarrow DDD|DC|b,C\rightarrow CCE|a, E\rightarrow \varepsilon|bEa\}$;
%   \item
%     $R = \{ S\rightarrow D,D\rightarrow AD|b,A\rightarrow AB|BC|a, B\rightarrow AB|CC, C\rightarrow \varepsilon|CA|a, E\rightarrow a|EB\}$;
%   \item
%     $R = \{ S \rightarrow AD|a, D\rightarrow \varepsilon|BB|AD,A\rightarrow DB|a,B\rightarrow DD|DC|b,C\rightarrow CE|a, E\rightarrow AB|b|EA\}$;
%   \item
%     $R =\{S\rightarrow AS|SB|SS,B\rightarrow CA|b, C\rightarrow AA|a|BA,A\rightarrow \varepsilon|BS\}$;
%   % \item
%   %   $R = \{S\rightarrow AB|AC,B\rightarrow \varepsilon |BC|b,A\rightarrow BB|CC|a,C\rightarrow CS|a\}$;
%   % \item
%   %   $R = \{S\rightarrow AS|SB|SS,B\rightarrow AC|b, C\rightarrow A|a|AB,A\rightarrow \varepsilon|BS\}$;
%   \item
%     $R = \{S\rightarrow BA|CA,B\rightarrow \varepsilon |BC|b,A\rightarrow BB|CC|a, C\rightarrow CS|a\}$;
%   \item
%     $R = \{S\rightarrow AS|b,A\rightarrow AC|BC|a, B\rightarrow BC|CC,C\rightarrow \varepsilon|CA|a\}$;
%   \item
%     $R = \{S\rightarrow \varepsilon|BA|AS,A\rightarrow SB|a,B\rightarrow SS|SC|b,
%     C\rightarrow CC|a\}$; 
%   \end{enumerate}
% \end{problem}

\begin{problem}
  Нека е дадена граматиката  $G = \pair{\{S,A,B,C\}, \{a,b\}, S, R}$.
  Използвайте обща конструкция, за да премахнете ,,дългите'' правила 
  (т.е. правила с дължина поне 2, които не са в н.ф. на Чомски) от $ G$ като при това получите 
  безконтестна граматика $G_1$ с език $L(G)=L(G_1)$, където:
  \begin{enumerate}[a)]
  \item
    $R = \{S \rightarrow \varepsilon|ab|aAba, A\rightarrow aBCb, B\rightarrow bbb, C\rightarrow aC\vert aCaC\}\rangle$;
  \item
    $R = \{S \rightarrow \varepsilon|ab|baAb, A\rightarrow BaBb,B\rightarrow b,C\rightarrow AbA\vert aCCa\}$;
  \item
    $R = \{A\rightarrow BSB|a,B\rightarrow ba|BC,C\rightarrow BaSA|a|b,S\rightarrow CC|b\}$;
  \item
    $R = \{A\rightarrow BAS,B\rightarrow CB,C\rightarrow ab|ABbS,S\rightarrow CC|b\}$;
  \end{enumerate}
\end{problem}


% \begin{problem}
%   Намерете безконтекстна граматика в нормална форма на Чомски за езиците от задача 6.
% \end{problem}


\subsection{Проблемът за принадлежност}

\begin{thm}
  Съществува {\em полиномиален} алгоритъм , който проверява дали дадена дума принадлежни на граматиката $G$.
  \marginpar{За дума $\alpha$, алгоритъмът работи за време $O(\abs{\alpha}^3)$}
\end{thm}
% \begin{proof}[стр. 154 от \cite{papadimitriou}]
Можем да приемем, че $G = \CFG$ е граматика в нормална форма на Чомски.
Нека $\alpha = a_1a_2\dots a_n$ е дума, за която искаме да проверим дали $\alpha \in L(G)$.
\marginpar{Това е алгоритъм на Cocke, Younger и Kasami (CYK), който е пример за динамично програмиране (стр. 195 от \cite{kozen})}
\begin{algorithm}[H]
  \caption{Проверка за $\alpha \in L(G)$}
  \label{alg:belongs-to-grammar}
  \begin{algorithmic}[1]
    \State $n := \abs{\alpha}$ \Comment{Вход дума $\alpha = a_1\cdots a_n$}
    \ForAll{$i\in [1,n]$}
    \State $V[i,i] = \{A \in V \mid A\rightarrow a_i\}$
    \EndFor
    \ForAll{$i,j \in [1,n]\ \&\ i \neq j$}
    \State $V[i,j] = \emptyset$
    \EndFor      
    \ForAll{$s \in [1, n)$} \Comment{Дължина на интервала}
    \ForAll{$i \in [1, n-s]$}\Comment{Начало на интервала}
    \ForAll{$k \in [i, i + s)$}\Comment{Разделяне на интервала}
    \If{$\exists A\to BC \in R\ \&\ B \in V[i,k]\ \&\ C\in V[k+1,i+s]$}
    \State $V[i,i+s] := V[i,i+s] \cup \{A\}$
    \EndIf
    \EndFor
    \EndFor
    \EndFor
    \If{$S \in V[1,n]$}
    \State \Return \texttt{True}\Comment{Има извод на думата от $S$}
    \Else
    \State \Return \texttt{False}
    \EndIf
  \end{algorithmic}
\end{algorithm}

\begin{lemma}
  За дадена граматика в нормална форма на Чомски и дума $\alpha$, 
  за всяко $0 \leq s < \abs{\alpha}$, след $s$-тата итерация на алгоритъма (редове 6 - 10), за всяка позиция $i = 1,\dots,n-s$,
  \[V[i,i+s] = \{A \in V \mid A \rightarrow^\star_G a_i\dots a_{i+s}\}.\]
\end{lemma}
\begin{proof}
  Пълна индукция по $s$.
  За $s = 0$  е ясно. (Защо?)

  Нека твърдението е вярно за $s < n$. Ще докажем твърдението за $s+1$, т.е. за всяко $i = 1,\dots,n-s-1$,
  \[V[i,i+s+1] = \{A \in V \mid A \rightarrow^\star_G a_i\dots a_{i+s+1}\}.\]
  % Да разгледаме $A \in V[i,i+s+1]$.
  За едната посока, да разгледаме първoто правило в извода $A \to^\star_G a_i\cdots a_{i+s+1}$.
  Понеже $G$ е в НФЧ, то е от вида $A \to BC$ и тогава съществува някое $t$, за което 
  $B \to^\star a_i\cdots a_{i+t}$ и $C \to^\star a_{i+t+1}\cdots a_{i+s+1}$.
  От И.П. получаваме, че $B \in V[i,i+t]$ и $C \in V[i+t+1,i+s+1]$.
  Тогава от ред 10 на алгоритъма е ясно, че $A \in V[i,i+s+1]$.
  
  За другата посока, нека $A \in V[i,i+s+1]$.
  Единствената стъпка на алгоритъма, при която може да сме добавили $A$ към множеството $V[i,i+s+1]$ е ред 10.
  Тогава имаме, че съществува $k$, за което $B \in V[i,k]$, $C \in V[k+1,i+s+1]$, и $A\to BC$ е правило в граматиката $G$.
  От И.П. имаме, че $B \to^\star_G a_i\cdots a_k$ и $C \to^\star_G a_{k+1}\cdots a_{i+s+1}$.
  Заключаваме веднага, че $A \to^\star_G a_i\cdots a_{i+s+1}$.
\end{proof}

\begin{example}
  Нека е дадена граматиката $G$ с правила 
  $S\rightarrow a|AB|AC, C\rightarrow SB|AS,A\rightarrow a, B\rightarrow b$.
  Ще приложим $CYK$ алгоритъма за да проверим дали думата $aaabb \in \L(G)$.
  \begin{itemize}
  \item 
    $V[1,1] = V[2,2] = V[3,3] = \{S,A\}$.
    $V[4,4] = V[5,5] = \{B\}$.
  \item
    $V[1,2] = V[2,3] = \{C\}$.
    $V[3,4] = \{S,C\}$.
    $V[4,5] = \emptyset$.
  \item
    $V[1,3] = \{S\} \cup \emptyset$.
    $V[2,4] = \{S,C\} \cup \emptyset$.
    $V[3,5] = \emptyset \cup \{C\}$.
  \item
    $V[1,4] = \{S,C\} \cup \emptyset \cup \emptyset = \{S,C\}$.
    $V[2,5] = \{S\} \cup \emptyset \cup \{C\} = \{S,C\}$
  \item
    $V[1,5] = \{S,C\} \cup \emptyset \cup \emptyset \cup \{C\}= \{S,C\}$.
  \end{itemize}
  Понеже $S \in V[1,5]$, то $aaabb \in \L(G)$.
\end{example}

\begin{thm}
  \marginpar{\cite{hopcroft1}, стр. 137}
  Съществуват алгоритми, които определят по дадена безконтекстна граматика $G$ дали:
  \begin{enumerate}[a)]
  \item 
    $\abs{\L(G)} = 0$;
  \item
    $\abs{\L(G)} < \infty$;
  \item
    $\abs{\L(G)} = \infty$.
  \end{enumerate}
\end{thm}
\begin{proof}
  Нека е дадена една безконтекстна граматика $G$.
  \begin{description}
  \item[($\L(G) = \emptyset?$)]
    Прилагаме алгоритъма за премахване на безполезните променливи.
    Ако открием, че $S$ е безполезна променлива, то $\L(G) = \emptyset$.
  \item[($\abs{\L(G)} < \infty?$ или $\abs{\L(G)} = \infty?$)]
    Нека да разгледаме граматиката $G'$ в НФЧ без безполезни променливи, за която $\L(G) = \L(G')$.
    От граматиката $G' = \pair{V',\Sigma,S,R'}$ строим граф с възли променливите от $V'$ като
    за $A,B \in V'$ имаме ребро $A \to B$ точно тогава, когато съществува $C \in V'$,
    за което $A \to BC$ или $A \to CB$ е правило в $R'$.
    
    Ако в получения граф имаме цикъл, то $\L(G') = \infty$.
  \end{description}
\end{proof}

\section{Недетерминирани стекови автомати}

\index{автомат!недетерминиран стеков}
\marginpar{На англ. {\bf Push-down automaton}}% (стр. 157 от \cite{kozen})}
%Sipser p.102
\begin{dfn}
  Недетерминиран стеков автомат е 7-орка от вида
  \[P = \PDA,\] където:
  \begin{itemize}
  \item
    $Q$ е крайно множество от състояния;
  \item  
    $\Sigma$ е крайна входна азбука;
  \item
    $\Gamma$ е крайна стекова азбука;
  \item
    $\# \in \Gamma$ е символ за дъно на стека;
  \item
    $s\in Q$ е начално състояние;
  \item
    \marginpar{Озн. $\Ps_{fin}(A)$ - крайните подмножества на $A$}
    $\Delta:Q\times(\Sigma \cup \{\varepsilon\})\times\Gamma\rightarrow \Ps_{fin}(Q\times\Gamma^\star)$ 
    е функция на преходите;    
  \item
    $F\subseteq Q$ е множество от заключителни състояния.
  \end{itemize}
\end{dfn}

\marginpar{Instanteneous description}
{\em Моментно описание} (или конфигурация) на изчислението със стеков автомат представлява тройка от вида $(q,\alpha,\gamma) \in Q\times\Sigma^\star\times\Gamma^\star$,
т.е. автоматът се намира в състояние $q$, думата, която остава да се прочете е $\alpha$,
а съдържанието на стека е думата $\gamma$.
Удобно е да въведем бинарната релация $\vdash_P$ над $Q\times\Sigma^\star\times\Gamma^\star$,
която ще ни казва как моментното описание на автомата $P$ се променя след изпълнение на една стъпка:
\[(q,x\alpha,Y\gamma) \vdash_P (p,\alpha,\beta\gamma), \text{ ако } \Delta(q,x,Y) \ni (p,\beta),\]
\[(q,\alpha,Y\gamma) \vdash_P (p,\alpha,\beta\gamma), \text{ ако } \Delta(q,\varepsilon,Y) \ni (p,\beta).\]
Рефлексивното и транзитивно затваряне на $\vdash_P$ ще означаваме с $\vdash^\star_P$.
Сега вече можем да дадем дефиниция на език, разпознаван от стеков автомат $P$.
\begin{itemize}
\item
  $\L_F(P)$ е езика, който се разпознава от $P$ {\bfс финално състояние},
  \[\L_F(P) = \{\omega \mid (q_0,\omega,\#) \vdash^\star_P (q,\varepsilon,\alpha)\ \&\ q \in F\}.\]    
\item
  $\L_S(P)$ е езика, който се разпознава от $P$  {\bf с празен стек},
  \[\L_S(P) = \{w\mid (q_0,w,\#) \vdash^\star_P (q,\varepsilon,\varepsilon)\}.\]    
\end{itemize}

\begin{example}
  \label{ex:anbn}
  За езика $L = \{a^nb^n\mid n\in\Nat\}$ съществува стеков автомат $P$, такъв че
  $L = \L_S(P)$.
  Да разгледаме $P = \PDA$, където
  \begin{itemize}
  \item
    $Q = \{q\}$;
  \item
    $\Sigma = \{a,b\}$;
  \item
    $\Gamma = \{\#,A\}$, където символът $\#$ служи за дъно на стека, а броят на $A$-тата в стека ще показват колко букви $a$ сме прочели от думата;
  \item
    $F = \emptyset$, защото разпознаваме с празен стек, а не с финално състояние;
  \item 
    $\Delta(q,a,\#) = \{(q, A\#)\}$;
  \item 
    $\Delta(q,\varepsilon,\#) = \{(q,\varepsilon)\}$;
  \item 
    $\Delta(q,b,A) = \{(q,\varepsilon)\}$.
  \end{itemize}
  Вместо доказтелство, да видим как думата $a^2b^2$ се разпознава от автомата с празен стек:
  \marginpar{\writedown Докажете, че $L = \L_S(P)$!}
  \begin{align*}
    (q,a^2b^2,\#) & \vdash_P (q,ab^2,A\#) \\
    & \vdash_P (q,b^2, AA\#)\\
    & \vdash_P (q,b,A\#)\\
    & \vdash_P (q,\varepsilon,\#)\\
    & \vdash_P (q,\varepsilon,\varepsilon).
  \end{align*}
\end{example}

\begin{example}
  За езика $L = \{\omega\omega^R \mid \omega \in \{a,b\}^\star\}$ съществува стеков автомат $P$, такъв че
  $L = \L_S(P)$.
  Нека $P = \PDA$, където:
  \begin{itemize}
  \item 
    $\Delta(q, a, \#) = \{(q, A\#)\}$;
  \item 
    $\Delta(q, b, \#) = \{(q, B\#)\}$;
  \item
    $\Delta(q, a, A) = \{(q, AA), (p, \varepsilon)\}$;
  \item
    $\Delta(q, a, B) = \{(q, AB)\}$;
  \item
    $\Delta(q, b, B) = \{(q, BB), (p, \varepsilon)\}$;
  \item
    $\Delta(q, b, A) = \{(q, BA)\}$;
  \item
    $\Delta(p, a, A) = \{(p,\varepsilon)\}$;
  \item
    $\Delta(p, b, B) = \{(p,\varepsilon)\}$;
  \item
    $\Delta(q, \varepsilon, \#) = \{(q,\varepsilon)\}$;
  \item
    $\Delta(p, \varepsilon, \#) = \{(p,\varepsilon)\}$;
  \end{itemize}
  Основното наблюдение, което трябва да направим за да разберем конструкцията на автомата е, че
  всяка дума от вида $\omega\omega^R$ може да се запише като $\omega_1aa\omega^R_1$ или $\omega_1bb\omega^R_1$.
  Да видим защо $P$ разпознава думата $abaaba$ с празен стек.
  Започваме по следния начин:
  \begin{align*}
    (q,abaaba,\#) & \vdash_P (q,baaba,A\#)\\
    & \vdash_P (q, aaba, BA\#) \\
    & \vdash_P (q, aba, ABA\#).
  \end{align*}
  Сега можем да направим два избора как да продължим. Състоянието $p$ служи за маркер, което ни казва, че вече сме започнали 
  да четем $\omega^R$. Поради тази причина, продължаваме така:
  \begin{align*}
    (q, aba, ABA\#) & \vdash_P (p, ba, BA\#)\\
    & \vdash_P (p, a, A\#)\\
    & \vdash_P (p, \varepsilon, \#) \\
    & \vdash_P (p,\varepsilon,\varepsilon).
  \end{align*}
  Да проиграем още един пример. Да видим защо думата $aba$ не се извежда от автомата.
  \begin{align*}
    (q,aba,\#) & \vdash_P (q, ba,A\#)\\
    & \vdash_P (q, a, BA\#)\\
    & \vdash_P (q, \varepsilon, ABA\#).
  \end{align*}
  \marginpar{\writedown Докажете, че $\L_S(P) = L$ !}
  От последното моментно описание на автомата нямаме нито един преход, следователно
  думата $aba$ не се разпознава от $P$ с празен стек.
\end{example}


\begin{thm}
  \marginpar{(\cite{hopcroft1}, стр. 114) }
  Нека $L$ е произволен език над азбука $\Sigma$.
  \begin{enumerate}[1)]
  \item 
    Ако съществува НСА $P$, за който $L = \L_F(P)$, то съществува НСА $P^\prime$, за който $L = \L_S(P^\prime)$.
  \item
    Ако съществува НСА $P$, за който $L = \L_S(P)$, то съществува НСА $P^\prime$, за който $L = \L_F(P^\prime)$.
  \end{enumerate}
  С други думи, езиците разпознавани от НСА с празен стек са точно езиците разпознавани от НСА с финално състояние.
\end{thm}
\begin{proof}
  \begin{enumerate}[1)]
  \item 
    Нека $L = \L_F(P)$, където $P = \PDA$.
    Ще построим $P^\prime$, така че да симулира $P$ и като отидем във финално състояние ще изпразним стека.
    Нека
    \[P^\prime = \langle{Q\cup\{q_e,s^\prime\},\Sigma,\Gamma \cup \{\$\},\$,s^\prime,\Delta^\prime,\emptyset}\rangle,\]
    където $\$ \not\in \Gamma$.
    Важно е $P^\prime$ да има собствен нов символ за дъно на стека, защото е възможно за някоя дума $\alpha \not\in \L_F(P)$
    стековият автомат $P$ да си изчисти стека и така да разпознаем повече думи.
    \begin{itemize}
    \item 
      \marginpar{- започваме симулацията}
      $\Delta'(s^\prime,\varepsilon,\$) = \{(s,\#\$)\}$;
    \item
      \marginpar{- симулираме $P$}
      $\Delta'(q,a,X)$ включва множеството $\Delta(q,a,X)$, за всяко $q\in Q$, $a\in\Sigma_\varepsilon$, $X\in\Gamma$;
    \item
      \marginpar{- ако сме във финално, започваме да чистим стека}
      $\Delta'(q,\varepsilon,X)$ съдържа също и елемента $(q_e,\varepsilon)$, за всяко $q\in F$, $X \in \Gamma \cup \{\$\}$;
    \item
      \marginpar{- изчистваме стека}
      $\Delta'(q_e,\varepsilon,X) = \{(q_e,\varepsilon)\}$, за всяко $X \in \Gamma \cup \{\$\}$;
    \item
      $\Delta'$ няма други правила.
    \end{itemize}
  \item
    Сега имаме $L = \L_S(P)$, където $P = \langle{Q,\Sigma,\Gamma,\#,s,\Delta,\emptyset}\rangle$. 
    Да положим
    \[P^\prime = \langle{Q\cup\{s^\prime,q_f\}, \Sigma, \Gamma \cup \{\$\}, \Delta^\prime, \$, \{q_f\}}\rangle.\]
    $P^\prime$ ще симулира $P$ като ще внимаваме кога $P$ изчиства символа $\#$. Тогава ще искаме да отидем във финалното състояние $q_f$.
    \begin{itemize}
    \item 
      \marginpar{- започваме симулацията}
      $\Delta'(s',\varepsilon,\$) = \{(s, \#\$)\}$;
    \item
      \marginpar{- симулираме $P$}
      $\Delta'(q,a,X) = \Delta(q,a,X)$, за всяко $q \in Q$, $a \in \Sigma_\varepsilon$, $X \in \Gamma$;
    \item
      \marginpar{- щом сме стигнали до $\$$, значи $P$ е изчистил стека си}
      $\Delta'(q,\varepsilon,\$) = \{(q_f,\varepsilon)\}$.
    \end{itemize}
  \end{enumerate}
\end{proof}

\begin{problem}
  Като използвате стековия автомат от Пример \ref{ex:anbn}, дефинирайте автомат $P'$, за който
  $\L_F(P') = \{a^nb^n \mid n\in\Nat\}$.
\end{problem}

\begin{framed}
\begin{thm}
  \label{th:push-down-context-free}
  Класът на езиците, които се разпознават от краен стеков автомат съвпада с
  класа на безконтекстните езици.
\end{thm}
\end{framed}
\begin{proof}
  \marginpar{(\cite{hopcroft1}, стр. 117)}
  Ще разгледаме двете посоки на твърдението поотделно.
  \begin{enumerate}[1)]
  \item 
    Нека е дадена безконтекстна граматика $G = \CFG$.
    Нашата цел е да построим стеков автомат $P$, така че $\L_S(P) = \L(G)$.
    Нека  \[P = \langle{\{q\},\Sigma,\Sigma\cup V,S,q,\Delta,\emptyset}\rangle,\]
    където функцията на преходите е:
    \begin{align*}
      & \Delta(q,\varepsilon,A) = \{(q,\alpha)\mid A\to\alpha\mbox{ е правило в граматиката }G\}\\
      & \Delta(q,a,a) = \{(q,\varepsilon)\}
    \end{align*}
  \item
    Нека имаме $P = \langle{Q, \Sigma, \Gamma, \Delta, s, \#, \emptyset}\rangle$.
    Ще дефинираме безконтекстна граматика $G$, за която $\L_S(P) = \L(G)$.
    Променливите на граматика са 
    \[V = \{[q,A,p] \mid q,p \in Q, A \in \Gamma\}.\]
    Правилата на $G$ са следните:
    \begin{itemize}
    \item
      $S \to [s,\#,q]$, за всяко $q \in Q$;
    \item
      $[q,A,q_{m+1}] \to a[q_1,B_1,q_2][q_2,B_2,q_3]\dots [q_m,B_m,q_{m+1}]$,
      където 
      \[(q_1,B_1\dots B_m) \in \Delta(q, a, A)\]
      и произволни $q,q_1,\dots,q_{m+1} \in Q$,
      $a \in \Sigma \cup \{\varepsilon\}$.

      Да обърнем внимание, че е възможно $m = 0$.
      Това означава, че $(q_1,\varepsilon) \in \Delta(q, a, A)$ и тогава имаме правилото $[q,A,q_{1}] \to a$, където $a \in \Sigma \cup \{\varepsilon\}$.
    \end{itemize}
    Трябва да докажем, че:
    \[[q,A,p] \rightarrow^\star_G \alpha\ \Leftrightarrow\ (q,\alpha,A) \vdash^\star_P (p,\varepsilon,\varepsilon).\]
    \begin{description}
    \item[$(\Rightarrow)$]
      С пълна индукция по $i$, ще докажем, че 
      \[(q,\alpha,A) \vdash^i_P (p,\varepsilon,\varepsilon)\ \implies\ [q,A,p] \Rightarrow^\star_G \alpha.\]
      Ако $i = 1$, то е лесно, защото $\alpha \in \Sigma \cup\{\varepsilon\}$ и $m = 0$.

      Ако $i > 1$, нека $\alpha = a\beta$. Тогава:
      \marginpar{Възможно е $a = \varepsilon$}
      \[(q,a\beta,A) \vdash_P (q_1,\beta,B_1\dots B_n) \vdash^{i-1}_P (p, \varepsilon, \varepsilon).\]
      Да разбием думата $\beta$ на $n$ части, $\beta = \beta_1\cdots \beta_n$, със свойството, че след като прочетем $\beta_i$ 
      сме премахнали променливата $B_i$ от върха на стека. Това означава, че :
      \begin{align*}
        & (q_j, \beta_j, B_j) \vdash^{l_j}_P (q_{j+1},\varepsilon,\varepsilon), \text{ за }j = 1,\dots,n-1,\\
        & (q_n, \beta_n, B_n) \vdash^{l_n}_P (p,\varepsilon,\varepsilon),
      \end{align*}
      където $l_1+l_2+\cdots+l_n = i-1$.
      Сега по {\bf И.П.} получаваме:
      \begin{align*}
        & (q_j, \beta_j, B_j) \vdash^{l_j}_P (q_{j+1},\varepsilon,\varepsilon) \implies [q_j,B_j, q_{j+1}] \rightarrow^\star_G \beta_j, \text{ за }за j = 1,\dots,n-1,\\
        & (q_n, \beta_n, B_n) \vdash^{l_n}_P (p,\varepsilon,\varepsilon) \implies [q_n,B_n, p] \rightarrow^\star_G \beta_n.
      \end{align*}
      Обединявайки тези изводи с правилото
      \[[q,A,p] \rightarrow_G a[q_1,B_1,q_2]\dots[q_n,B_n,p],\]
      получаваме извода
      \[[q,A,p] \rightarrow^\star_G a\beta.\]
    \item[$(\Leftarrow)$]
      Отново с пълна индукция по $i$ ще докажем, че
      \[[q,A,p] \rightarrow^i_G \alpha \implies (q,\alpha,A) \vdash^\star_P (p,\varepsilon,\varepsilon).\]
      Ако $i = 1$, то имаме $[q,A,p] \rightarrow \alpha$, където $\alpha = a$ или $\alpha = \varepsilon$.
      Ако $i > 1$, то имаме, че $\alpha = a\beta$ и за някое $n$, 
      \[[q,A,p] \rightarrow_G a[q_1,B_1,q_2][q_2,B_2,q_3]\dots[q_n,B_n,p] \rightarrow^{i-1}_G \beta.\]
      Отново нека $\beta = \beta_1\dots \beta_n$, където 
      \begin{align*}
        & [q_j,B_j,q_{j+1}] \rightarrow^{i_j}_G \beta_j, \text{ за } j = 1,\dots,n-1,\\
        & [q_{n},B_n,p ] \rightarrow^{i_n}_G \beta_n,
      \end{align*}
      където $i_1 + i_2 + \cdots + i_n = i-1$.
      От {\bf И.П.} получаваме, че 
      \begin{align*}
        & [q_j,B_j,q_{j+1}] \rightarrow^{i_j}_G \beta_j \implies (q_j,\beta_j,B_j) \vdash^\star_P (q_{j+1},\varepsilon,\varepsilon),\ j = 1,\dots,n-1\\
        & [q_n,B_n,p] \rightarrow^{i_n}_G \beta_n \implies (q_n,\beta_n,B_n) \vdash^\star_P (p,\varepsilon,\varepsilon),
      \end{align*}
      Обединявайки всичко, което знаем, получаваме:
      \begin{align*}
        (q, a\beta, A) & \vdash_P (q_1, \beta_1\cdots\beta_n, B_1\cdots B_n)\\
        & \vdash^\star_P (q_2, \beta_{2}\cdots\beta_n, B_2\cdots B_n)\\
        & \dots\\
        & \vdash^\star_P (q_n, \beta_n, B_n)\\
        & \vdash^\star_P (p, \varepsilon, \varepsilon)
      \end{align*}
    \end{description}
  \end{enumerate}
\end{proof}

% \begin{problem}
%   Нека е дадена граматиката $G = \pair{\{S,A,B\},\{a,b\},S,R\}}$.
%   Постройте стеков автомат $P = \PDA$, такъв че $\L_S(P) = \L(G)$, където правилата $R$ на граматиката $G$ са зададени като:
%   \begin{enumerate}[a)]
%     % За едно тези двете да се даде пример как става 
%   \item
%     $R = \{S\rightarrow ASB\vert \varepsilon, A\rightarrow aAa\vert a, B\rightarrow bBb\vert b\}$;
%   \item
%     $R = \{S\rightarrow ASB\vert \varepsilon, A\rightarrow aA\vert a, B\rightarrow Bb\vert b\}$;
%   \item
%     $R =\{S\rightarrow SA|\varepsilon,A\rightarrow BSa|B, B\rightarrow b|BS|ab\}$;
%   \item
%     $R = \{S\rightarrow AS|\varepsilon,A\rightarrow SaBB|A, B\rightarrow b|BBbS|AA\}$;
%   \end{enumerate}
% \end{problem}

\begin{example}
  Нека е дадена граматиката $G$ с правила 
  $S\rightarrow ASB\vert \varepsilon, A\rightarrow aAa\vert a, B\rightarrow bBb\vert b$.
  Ще построим стеков автомат $P = \PDA$, такъв че $\L_S(P) = \L(G)$.
  \begin{itemize}
  \item
    $\Sigma = \{a,b\}$;
  \item 
    $\Gamma = \{A,S,B,a,b\}$;
  \item
    $\# = S$;
  \item
    $Q = \{q\}$;
  \item
    $F = \emptyset$;
  \item
    Дефинираме релацията на преходите, следвайки конструкцията от \Th{push-down-context-free}:
    \begin{itemize}
    \item 
      $\Delta(q,\varepsilon, S) = \{\pair{q,ASB}, \pair{q,\varepsilon}\}$;
    \item
      $\Delta(q, \varepsilon, A) = \{\pair{q, aAa}, \pair{q, a}\}$;
    \item
      $\Delta(q, \varepsilon, B) = \{\pair{q, bBb}, \pair{q, b}\}$;
    \item
      $\Delta(q, a, a) = \{\pair{q,\varepsilon}\}$;
    \item
      $\Delta(q, b, b) = \{\pair{q,\varepsilon}\}$.
    \end{itemize}
  \end{itemize}
\end{example}


\begin{thm}
  \marginpar{(стр. 144 от \cite{papadimitriou})}
  Нека $L$ e безконтекстен език и $R$ е регулярен език.
  Тогава тяхното сечение $L \cap R$ е безконтекстен език.
\end{thm}
\begin{proof}
  Нека имаме стеков автомат
  \[\M_1 = \PDAn{1}, \text{ където } \L_F(\M_1) = L,\]
  \marginpar{всъщност няма нужда да е детерминиран}
  и краен детерминиран автомат 
  \[\M_2 = \FAn{2}, \text{ където } \L(\M_2) = R.\]
  Ще определим нов стеков автомат $\M = \PDA$, където
  \begin{itemize}
  \item 
    $Q = Q_1 \times Q_2$;
  \item
    $s = \pair{s_1,s_2}$;
  \item
    $F = F_1 \times F_2$;
  \item 
    Функцията на преходите $\Delta$ е дефинирана както следва:
    \begin{itemize}
    \item 
      \marginpar{симулираме едновременно изчислението и на двата автомата}
      Ако $\Delta_1(q_1, a, b) \ni \pair{r_1,c}$
      и $\delta_2(q_2,a) = r_2$, то
      \[\Delta(\pair{q_1,q_2},a,b) \ni \pair{\pair{r_1,r_2}, c}.\]
    \item
      \marginpar{празен ход на автомата $M_2$}
      Ако $\Delta_1(q_1,\varepsilon,b) \ni \pair{r_1,c}$,
      то за всяко $q_2 \in Q_2$,
      \[\Delta(\pair{q_1,q_2},\varepsilon,b) \ni \pair{\pair{r_1,q_2},c}.\]    
    \item
      \marginpar{\writedown Докажете, че $\L(\M) = \L(\M_1) \cap \L(\M_2)$ !}
      $\Delta$ не съдържа други преходи;
    \end{itemize}
  \end{itemize}
\end{proof}

\begin{example}
  Езикът $L = \{w \in \{a,b,c\}^\star \mid n_a(w) = n_b(w) = n_c(w)\}$ не е безконтекстен.
\end{example}
\begin{proof}
  Да допуснем, че $L$ е безконтекстен език.
  Тогава \[L^\prime = L \cap \L(a^\star b^\star c^\star)\] също е безконтекстен език.
  Но $L^\prime = \{a^nb^nc^n \mid n \in \Nat\}$, за който знаем от Пример \ref{example:anbncn}, че {\em не} е безконтекстен.
  Достигнахме до противоречие. Следователно, $L$ не е безконтекстни език.
\end{proof}



\section*{Библиография}

Основни източници в тази глава са:
\begin{itemize}
\item 
  глава 4 от \cite{hopcroft1}, глави 5, 6 и 7 от \cite{hopcroft2};
\item
  глава 2 от \cite{sipser1};
\item
  глава 3 от \cite{papadimitriou}.
\end{itemize}


% \section{Въпроси}
% Вярно ли е, че:
% \begin{itemize}
% \item
% %  \marginpar{Да} 
%   ако $L$ е безконтекстен език, то езикът $L \cap \{a^{2n}b^{2k}\mid n,k\in\Nat\}$ е безконтекстен ?
% \item
%  % \marginpar{Да}
%   ако $L$ е безкраен безконтекстен език, то съществува безкрайна редица от регулярни езици $L_1,L_2,\dots$,
%   за които $L = \bigcup_{i\in\Nat}L_i$ ?
% \item
%   \marginpar{Не}
%   за всяка безкрайна редица от регулярни езици $L_1,L_2,\dots$, то 
%   езикът $L = \bigcup_{i\in\Nat}L_i$ е безконтекстен ?
% \item
%   %\marginpar{Да}
%   за всеки регулярен език $R$ и всеки безконтекстен език $L$, то $L \cap R$ е безконтекстен ?
% \item
%   за всеки регулярен език $R$ и всеки безконтекстен език $L$, то $L \cup R$ е безконтекстен ?
% \item
%   за всеки регулярен език $R$ и всеки безконтекстен език $L$, то $L \setminus R$ е безконтекстен ?
% \item
%   за всеки регулярен език $R$ и всеки безконтекстен език $L$, то $R \setminus L$ е безконтекстен ?
% \item
%   съществува регулярен език $R$ и безконтекстен език $L$, за които $L \cap R$ не е безконтекстен ?
% \item
%   съществува регулярен език $R$ и нерегулярен, но безконтекстен език $L$, за които $L \cap R$ е регулярен ?
% \item
%   за всеки два нерегулярни, но контекстно-свободни езика $L_1,L_2$, то $L_1\cup L_2$ е регулярен ?
% \item
%   съществуват два нерегулярни, но безконтекстни езика $L_1,L_2$, за които $L_1\setminus L_2$ е регулярен ?
% \item
%   съществуват два нерегулярни, но безконтекстни езика $L_1,L_2$, за които $L_1\cap L_2$ е регулярен ?
% \item
%   съществуват два нерегулярни, но безконтекстни езика $L_1,L_2$, за които $L_1\cup L_2$ е регулярен ?
% \item
%   съществува регулярен език $R$, който може да се представи като $R = L_1 \cup L_2$, където
%   $L_1 \cap L_2 = \emptyset$, $L_1,L_2$ са нерегулярни, но контекстно-свободни ?
% \item
%   езикът $\{a,b\}^\star \setminus \{a^nb^n \mid n\in\Nat\}$ е регулярен ?
% \item
%   езикът $\{a,b\}^\star \setminus \{a^nb^n \mid n\in\Nat\}$ е безконтекстен ?
% \item
%   езикът $\{a,b\}^\star \setminus \{a^nb^{2k+1} \mid n,k\in\Nat\}$ е регулярен ?
% \item
%   езикът $\{a,b\}^\star \setminus \{a^nb^{k} \mid n > k\}$ е регулярен ?
% \item
%   езикът $\{a,b\}^\star \setminus \{a^nbba^{n} \mid n \in \Nat\}$ е регулярен ?
% \item
%   езикът $\{a,b\}^\star \setminus \{a^nb^n \mid n\in\Nat\}$ е безконтекстен ?
% \item
%   езикът $\{a,b,c\}^\star \setminus \{a^nb^mc^k \mid m < n\ \&\ m < k\}$ е безконтекстен ?
% \item
%   \marginpar{Не. $\alpha = b^pa^pbba^p$.}
%   езикът $L = \{uvv^R \mid u,v \in \{a,b\}^\star\ \&\ \abs{u} \leq \abs{v}\}$ е регулярен ?
% \item
%   \marginpar{Да.}
%   езикът $L = \{uvv^R \mid u,v \in \{a,b\}^\star\ \&\ \abs{u} \leq \abs{v}\}$ е безконтекстен ?
% \item
%   съществува алгоритъм, който за даден вход регулярен израз $r$ и безконтекстна граматика $G$
%   проверява дали $\L(r) = \L(G)$?
% \item
%   съществува алгоритъм, който за даден вход регулярен израз $r$ и безконтекстна граматика $G$
%   проверява дали $\L(r) \cap \L(G) = \emptyset$?
% \item
%   съществува алгоритъм, който за даден вход регулярен израз $r$ и безконтекстна граматика $G$
%   проверява дали $\abs{\L(r) \cap \L(G)} < \infty$?
% \item
%   съществува алгоритъм, който за даден вход регулярен израз $r$ и безконтекстна граматика $G$
%   проверява дали $\abs{\L(r) \cap \L(G)} = \infty$?
% \item
%   съществува алгоритъм, който за даден вход регулярен израз $r$, безконтекстна граматика $G$
%   и число $k$, проверява дали $\abs{\L(r) \cap \L(G)} = k$?
% \item
%   съществува алгоритъм, който за даден вход регулярен израз $r$ и безконтекстна граматика $G$
%   проверява дали $\L(r) \setminus \L(G) = \emptyset$?
% \item
%   съществува алгоритъм, който за даден вход регулярен израз $r$ и безконтекстна граматика $G$
%   проверява дали $\L(G) \setminus \L(r) = \emptyset$?
% \item
%   съществува алгоритъм, който за даден вход регулярен израз $r$ и безконтекстна граматика $G$
%   проверява дали $\abs{\L(r) \setminus \L(G)} < \infty$?
% \item
%   съществува алгоритъм, който за даден вход регулярен израз $r$ и безконтекстна граматика $G$
%   проверява дали $\abs{\L(G) \setminus \L(r)} < \infty$?
% \item
%   съществува алгоритъм, който за даден вход регулярен израз $r$ и безконтекстна граматика $G$
%   проверява дали $\abs{\L(r) \setminus \L(G)} = \infty$?
% \item
%   съществува алгоритъм, който за даден вход регулярен израз $r$ и безконтекстна граматика $G$
%   проверява дали $\abs{\L(G) \setminus \L(r)} = \infty$?
% \item
%   съществува алгоритъм, който за даден вход регулярен израз $r$, безконтекстна граматика $G$
%   и число $k$, проверява дали $\abs{\L(r) \setminus \L(G)} = k$?
% \item
%   съществува алгоритъм, който за даден вход регулярен израз $r$, безконтекстна граматика $G$
%   и число $k$, проверява дали $\abs{\L(G) \setminus \L(r)} = k$?
% \end{itemize}

% Нека е дадена безконтекстна граматика $G$ с правила \[S\rightarrow a\vert AB \vert AC, A \rightarrow a, B\rightarrow b, C\rightarrow SB.\]
% Вярно ли е, че ако приложим CYK алгоритъма върху думата $\alpha$, където
% \begin{itemize}
% \item 
%   $\alpha = aabb$, то $N[1,1] = \{S\}$.
% \item 
%   $\alpha = aabb$, то $N[3,3] = \{B\}$.
% \item 
%   $\alpha = aabb$, то $N[1,4] = \{\}$.
% \item
%   $\alpha = baab$, то $N[2,4] = \{\}$.
% \item
%   $\alpha = baab$, то $N[1,3] = \{\}$.
% \end{itemize}



%%% Local Variables: 
%%% mode: latex
%%% TeX-master: "EAI"
%%% End: 

\section{Допълнителни задачи}

\begin{problem}
  Докажете, че следните езици са безконтекстни:
  \begin{enumerate}[a)]
  \item 
    $L = \{\omega_1\sharp\omega_2 \mid \omega_1,\omega_2 \in \{a,b\}^\star\ \&\ \abs{\omega_1} = \abs{\omega_2}\}$;
  \item
    $L = \{\omega_1 \sharp \omega_2 \sharp \cdots \sharp \omega_n \mid n\geq 2\ \&\ \omega_1,\omega_2,\dots,\omega_n \in \{a,b\}^\star\ \&\ \abs{\omega_1} = \abs{\omega_2}\}$;
  \item
    $L = \{\omega_1 \sharp \omega_2 \sharp \cdots \sharp \omega_n \mid n\geq 2\ \&\ \omega_1,\dots,\omega_n \in \{a,b\}^\star\ \&\ (\exists i \neq j)[\abs{\omega_i} = \abs{\omega_j}]\}$;
  \item
    $L = \{\omega_1 \sharp \omega_2 \sharp \cdots \sharp \omega_n \mid n\geq 2\ \&\ (\forall i\in[1,n])[\omega_i \in \{a,b\}^\star\ \&\ \abs{\omega_i} = \abs{\omega_{n+1-i}}]\}$.
  \end{enumerate}
\end{problem}


\begin{problem}
  Проверете дали следните езици са безконтекстни:
  \begin{enumerate}[a)]
  \item
    $\{a^nb^{2n}c^{3n}\ \mid\ n\in\Nat\}$;
  \item
    $\{a^nb^{2n}c^{n}\ \mid\ n\in\Nat\}$;
  \item
    $\{a^mb^n\mid\ m \neq n\}$;
  \item
    $\{a^nb^mc^k\mid n < m < k\}$;
  \item
    $\{a^nb^mc^k\mid k = \min\{n,m\}\}$;
  \item
    $\{a^nb^nc^m\mid m \leq n\}$;
  \item
    $\{a^nb^mc^k\mid k = n\cdot m\}$;
  \item
    $L^\star$, където
    $L = \{\alpha\alpha^R \mid \alpha \in \{a,b\}^\star\}$;
  \item
    $\{www\mid w\in \{a,b\}^\star\}$;
  \item
    $\{ww^R\mid w\in \{a,b\}^\star\}$;
  \item
    $\{a^{n^2}b^n\ \mid n \in \Nat\}$;
  \item
    $\{a^p\ \mid\ p\mbox{ е просто }\}$;
  \item
    $\{\omega \in \{a,b\}^\star \mid \omega = \omega^R\}$;
  \item
    $\{\omega^n \mid \omega \in \{a,b\}^\star\ \&\ n \in \Nat\}$;
  \item
    $\{a^{n^3 + 2n^2} \mid n \in \Nat\}$;
  \item
    % Дефиниция на подниз
    $L = \{w c x\mid w,x\in \{a,b\}^\star\ \&\ w\mbox{ е подниз на }x\}$;
  \item
    $L = \{x_1 c x_2 c \dots c x_k\mid k\geq 2\ \&\ x_i\in\{a,b\}^\star\ \&\ (\exists i,j)[i \neq j\ \&\ x_i = x_j]\}$;
  \item
    $L = \{a^ib^jc^k\mid i,j,k\geq 0\ \&\ (i = j \vee j = k)\}$;
  \item
    % \marginpar{Разгл. $L' = L \cap L(a^*b^*c^*)$.}
    $L = \{\alpha \in \{a,b,c\}^\star\mid n_a(\alpha) > n_b(\alpha) > n_c(\alpha)\}$;
  \item
    $L = \{a,b\}^\star \setminus \{a^nb^n\mid n\in \Nat\}$;
  \item
%    \marginpar{Разгл. $L' = L \cap L(a^*b^*a^*)$.}
    $L = \{\omega \in \{a,b\}^\star \mid n_a(\omega) = 2n_b(\omega)\}$;
  \item
    $L = \{a^nb^mc^ma^n \mid m,n\in\Nat\ \&\ n = m+42\}$;
  \item
    $L = \{babaabaaab\cdots ba^{n-1}ba^nb \mid n \geq 1\}$;
%   \end{enumerate}
% \end{problem}

% \begin{problem}
%   Проверете кои от следните езици са безконтекстен:
%   \begin{enumerate}[a)]
  \item
    $\{a^mb^nc^k\mid m = n \vee n = k \vee m = k\}$;
  \item
    $\{a^mb^nc^k\mid m \neq n \vee n \neq k \vee m \neq k\}$;
  \item
    $\{a^mb^nc^k\mid m = n \wedge n = k \wedge m = k\}$;
  \item
    $\{w \in \{a,b,c\}^\star\mid n_a(w) \neq n_b(w) \vee n_a(w) \neq n_c(w) \vee n_b(w) \neq n_c(w)\}$.
  \end{enumerate}
\end{problem}

\begin{problem}
  Докажете, че ако $L$ е безконтекстен език, то $L^R = \{\omega^R \mid \omega \in L\}$ 
  също е безконтекстен.
\end{problem}

\begin{problem}
  Нека $\Sigma = \{a,b,c,d,f,e\}$.
  Докажете, че езикът $L$ е безконтекстен, където за думите $\omega \in L$ са изпълнени свойствата:
  \begin{itemize}[-]
  \item 
    за всяко $n\in\Nat$, след всяко срещане на $n$ последнователни $a$-та
    следват $n$ последователни $b$-та, и $b$-та не се срещат по друг повод в $\omega$, и
  \item
    за всяко $m\in\Nat$, след всяко срещане на $m$ последнователни $c$-та
    следват $m$ последователни $d$-та, и $d$-та не се срещат по друг повод в $\omega$, и
  \item
    за всяко $k\in\Nat$, след всяко срещане на $k$ последнователни $f$-а
    следват $k$ последователни $e$-та, и $e$-та не се срещат по друг повод в $\omega$.
  \end{itemize}
\end{problem}

\begin{problem}
  Да разгледаме езиците:
  \begin{align*}
    & P = \{\alpha\in\{a,b,c\}^*\,|\, \alpha \text{ е палиндром с четна дължина}\} \\
    & L =  \{\beta b^n\,|\, n\in\mathbb{N}, \beta\in P^n\}.
  \end{align*}
  Да се докаже, че:
  \begin{enumerate}[a)]
  \item 
    $L$ не е регулярен;
  \item 
    $L$ е безконтекстен.
  \end{enumerate}
\end{problem}

\begin{problem}
  Нека $L_1$ е произволен регулярен език над азбуката $\Sigma$, 
  а $L_2$ е езика от всички думи палиндроми над $\Sigma$.
  Докажете, че $L$ е безконтекстен език, където:
  \[L = \{\alpha_1\alpha_2\cdots\alpha_{3n}\beta_1\cdots\beta_m\gamma_1\cdots\gamma_n \mid \alpha_i,\gamma_j \in L_1, \beta_k\in L_2, m,n \in \Nat\}.\]
\end{problem}

\begin{problem}
  Нека $L = \{\omega\in\{a,b\}^\star \mid N_a(\omega) = 2\}$.
  Да се докаже, че езикът $L' = \{\alpha^n \mid \alpha\in L, n \geq 0\}$ не е безконтекстен.
\end{problem}


\begin{problem}
  Нека $\Sigma = \{a,b,c\}$ и $L \subseteq \Sigma^\star$ е безконтестен език. Ако имаме дума 
  $\alpha \in \Sigma^\star$, тогава \emph{L-вариант} на $\alpha$ ще наричаме думата, която се получава като в $\alpha$ всяко едно 
  срещане на символа $a$ заменим с (евентуално различна) дума от $L$.
  Тогава, ако $M \subseteq \Sigma^*$ е произволен безконтестен език, да се докаже че езикът
  \begin{equation*}
    M' = \{\beta\in\Sigma^\star |\ \beta \text{ е $L$-вариант на } \alpha \in M \}
  \end{equation*}
  също е безконтекстен.
\end{problem}

\begin{problem}
  Докажете, че всеки безконтекстен език над азбуката $\Sigma = \{a\}$
  е регулярен.
\end{problem}

\begin{problem}
%  \marginpar{\cite{papadimitriou} стр. 149}
  Да фиксираме азбуката $\Sigma$.
  Нека $L$ е безконтекстен език, а $R$ е регулярен език.
  Докажете, че езикът
  $L/R = \{\omega \in \Sigma^\star \mid (\exists u \in R)[\omega u \in L]\}$
  е безконтекстен.
\end{problem}


\begin{problem}
  Нека е дадена граматиката $G = \pair{\{a,b\}, \{S,A,B,C\},S,R}$.
  Използвайте CYK-алгоритъма, за да проверите дали
  думата $\alpha$ принадлежи на $\L(G)$, където правилата на граматиката $R$ и думата $\alpha$
  са зададени като:
  \begin{enumerate}[a)]
  \item
    $R = \{S\rightarrow BA| CA|a, C\rightarrow BS|SA,A\rightarrow a, B\rightarrow b\}$, $\alpha=bbaaa$;
  \item
    $R =\{S\rightarrow AB|BC, A\rightarrow BA|a,B\rightarrow CC|b, C\rightarrow AB|a\}$, $\alpha=baaba$;
  \item
    $R = \{S\rightarrow AB, A\rightarrow AC|a|b,B\rightarrow CB|a, C\rightarrow a\}$, $\alpha=babaa$;
  \end{enumerate}
\end{problem}

\begin{problem}
  \marginpar{Интересно е също да се направи и безконтекстна граматика за $L$}
  Постройте стеков автомат за езика $L$ над азбуката $\{a,b,\sharp\}$, където
  \[L = \{\omega_1 \sharp \omega_2 \sharp \cdots \sharp \omega_{2n} \mid n\in\Nat\ \&\ \sum^n_{i=1}\abs{\omega_{2i}} = \sum^{n}_{i=1}\abs{\omega_{2i-1}}\}.\]
\end{problem}


%%% Local Variables: 
%%% mode: latex
%%% TeX-master: "EAI"
%%% End: 


\chapter{Машини на Тюринг}

\newcommand{\tape}[1]{\dots\blank\blank\blank{#1}\blank\blank\blank\dots}
\marginpar{Тук най-вече следваме \cite{sipser3}}
\section{Основни понятия}
\index{Тюринг}
{\em Детерминистична} машина на Тюринг ще наричаме седморка от вида 
\[\M = \TM,\] където:
\begin{itemize}
\item 
  $Q$ - състояния;
\item
  $\Sigma$ - азбука за входа;
\item
  $\Gamma$ - азбука за лентата, $\Sigma \subseteq \Gamma$;
\item
  \marginpar{Няма нужда да изискваме главата да остава върху същата клетка от лентата}
  $\delta:Q\times\Gamma \to Q\times \Gamma \times \{L,R\}$ - (частична) функция на преходите;
\item
  $s$ - начално състояние, $s \in Q$;
\item
  $\blank$ - празен символ,  $\blank \in \Gamma \setminus \Sigma$;
\item
  \marginpar{Тези две състояния ще наричаме заключителни}
  $q_{accept} \in Q$ - приемащо състояние;
\item
  \marginpar{$q_{accept} \neq q_{reject}$}
  $q_{reject} \in Q$ - отхвърлящо състояние.
\end{itemize}

Сега ще опишем как $\M$ работи върху вход думата $\alpha \in \Sigma^\star$.
Първоначално, безкрайната лента съдържа само $\alpha$. Останалите клетки на лентата съдържат $\blank$.
Освен това, $\M$ се намира в началното състояние $s$ и главата е върху най-левия символ на $\alpha$.
Работата $\M$ е описана от функцията на преходите.
  
\begin{itemize}
\item 
  % \marginpar{(На англ. instanteneous description)}
  {\bf Моментната конфигурация} (или описание) на едно изчисление на МТ е тройка от вида $(\alpha, q, \beta) \in \Gamma^\star\times Q \times \Gamma^\star$. Това означава, че
  машината се намира в състояние $q$ и лентата има вида
  \[\tape{\alpha\beta},\]
  като главата на машината е поставена върху първия символ на $\beta$.  
\item
  {\bf Началната конфигурация} за входа $\alpha \in \Sigma^\star$ представлява 
  \[(\varepsilon, s, \alpha).\]
  Това означава, че лентата има вида \[\cdots\blank\blank\alpha\blank\blank\cdots\]
  и четящата глава е на първия символ от думата $\alpha$.
\item
  {\bf Заключителна конфигурация} представлява тройка от вида
  \[(\alpha, q_{accept}, \beta), \text{ или }(\alpha, q_{reject}, \beta).\]
  Ако машината, която работи върху дадена входа дума, достигне до заключително състояние, ще казваме
  че машината {\em спира работа}.
\end{itemize}

Както за автомати, удобно е да дефинираме бинарна релация $\vdash_\M$ над $\Gamma^\star \times Q \times \Gamma^\star$,
която ще казва как моментната конфигурация на машината $\M$ се променя при изпълнение на една стъпка.
\begin{itemize}
\item
  Ако $\delta_\M(q,Z) = (p,Y,R)$, то дефинираме $(\alpha, q, Z\beta) \vdash_\M (\alpha Y, p, \beta)$.
  % При $Z = \blank$ също така можем да запишем 
  % $(\alpha, q, \varepsilon) \vdash_\M (\alpha Y, p, \varepsilon)$ (???)
\item 
  Ако $\delta_\M(q,Z) = (p,Y,L)$, то дефинираме $(\alpha X, q, Z\beta) \vdash_\M (\alpha , p, XY\beta)$.
  % При $X = \blank$, имаме $(\varepsilon, q, Z\beta) \vdash_\M (\varepsilon, p, \blank Y\beta)$.
% \item
%   Ако $\delta_\M(q,Z) = (p,Y,N)$, то 
  
\end{itemize}
С $\vdash^\star_\M$ ще означаваме рефлексивното и транзитивно затваряне на $\vdash_\M$.

\begin{itemize}
\item
  \marginpar{Обърнете внимание, че това не е същото като да изискваме функцията на преходите $\delta$ да бъде тотална.}
  Една машина на Тюринг се нарича {\bf тотална}, ако при всеки вход достига до заключително състояние.
\item 
  Езикът, който се {\bf разпознава} от машината $\M$ е:
  \[\L(\M) = \{\alpha\in\Sigma^\star \mid (\varepsilon, s, \alpha) \vdash^\star_\M (\beta, q_{accept}, \gamma), \text{ за някои }\beta,\gamma\in\Gamma^\star\}.\]
\item
  \index{език!полуразрешим}
  Езикът $L$ се нарича {\bf полуразрешим}, ако съществува машина на Тюринг $\M$, за която
  $L = \L(\M)$.
\item
  \index{език!разрешим}
  Един език $L$ се нарича {\bf разрешим}, ако за него съществува {\em тотална} машина на Тюринг $\M$, за която
  $L = \L(\M)$.
\end{itemize}

\begin{framed}
  \begin{prop}
    Ако $L$ е разрешим език над азбуката $\Sigma$, то $\Sigma^\star \setminus L$ също е разрешим език.
  \end{prop}
\end{framed}

\begin{remark}
  По-късно, ще видим, че съществуват полуразрешими езици, чиито допълнения не са полуразрешими.
\end{remark}

% \marginpar{Това трябва ли ми ?}
% Езикът, който се {\bf разпознава чрез спиране} от $M$ е:
% \[\L_H(\M) = \{\alpha \in \Sigma^\star \mid (\varepsilon, s, \alpha) \vdash^\star (\beta, q, X\gamma)\ \&\ \neg !\delta(q,X)\}\]
% Може да се докаже, че се разпознават едни и същи езици.


\section{Примери за разрешими езици}

\begin{example}
  \marginpar{Знаем, че $L$ не е безконтекстен}
  Да разгледаме езика $L = \{a^nb^nc^n \mid n\in\Nat\}$.
 
  Нека да въведем нов символ $d$, с който ще маркираме обработените символи $a$, $b$, $c$.
  Идеята на алгоритъма, който ще разгледаме е да маркира на всяка итерация по едно $a$, $b$, и $c$.
  Той завършва успешно ако всички символи на думата са маркирани.
  Нека първоначално думата е копирана върху лентата и четящата глава е върху първия символ на думата.
  \begin{enumerate}[(1)]
  \item 
    Чете $d$-та надясно по лентата докато срещне първото $a$ и го замества с $d$. Отива на стъпка (2).
    Ако символите свършат (т.е. достигне се $\blank$) преди да се достигне $a$,
    то алгоритъмът завършва успешно.
  \item
    Чете $d$-та надясно по лентата докато срещне първото $b$ и го замества с $d$.
    Отива на стъпка (3).
  \item
    Чете $d$-та надясно по лентата докато срещне първото $c$ и го замества с $d$.
  \item
    Връща четящата глава в началото на лентата, т.е. чете наляво докато не срещне символа $\blank$.
    Връща се в стъпка (1). 
  \end{enumerate}

  Нека сега да видим, че този алгоритъм може да се опише съвсем формално с машина на Тюринг.
  Ще построим машина на Тюринг $\M$, за която $L = \L(\M)$, където
  \begin{itemize}
  \item 
    $\Sigma = \{a,b,c\}$;
  \item
    $\Gamma = \{a,b,c,d,\blank\}$, за някой нов символ $d$;
  \item
    $Q = \{1,2,\dots,5\}$;
  \item
    $q_{accept} = 5$;
  \item
    частичната функция на преходите $\delta:Q\times\Gamma \to Q\times\Gamma\times\{L,R\}$
    е описана на схемата отдолу.
  \end{itemize}

  \begin{figure}[H]
    \begin{center}
      \begin{tikzpicture}[->,>=stealth,thick,node distance=50pt]
        \tikzstyle{every state}=[circle,minimum size=10pt,auto]
        
        \node[state,initial]    (1) {$1$};
        \node[state]            (2) [right of=1]{$2$};
        \node[state]            (3) [right of=2]{$3$};
        \node[state]            (4) [right of=3]{$4$};
        \node[state,accepting]  (5) [below of=1]{$5$};
        % \node[state,accepting]  (6) [right of=5]{$6$};
        
        \begin{scope}[every node/.style={scale=.8}]
          \path
          (1) edge [loop above] node [above] {$d;R$} (1)
          (1) edge [bend right=15] node [left] {$\blank;R$} (5)
          % (1) edge [bend left=15] node [left] {$\{b,c\}$} (6)
          (1) edge [bend left=15] node [above] {$a/d;R$} (2)
          (2) edge [bend left=15] node [above] {$b/d;R$} (3)
          (2) edge [loop below] node [right] {$\{a,d\};R$} (2)
          (3) edge [bend left=15] node [above] {$c/d;L$} (4)
          (3) edge [loop below] node [right] {$\{b,d\};R$} (3)
          (4) edge [loop right] node [below right] {$\{a,b,d\};L$} (4)
          (4) edge [in=65,out=115,above] node [above] {$\blank;R$} (1);
        \end{scope}
      \end{tikzpicture}
    \end{center}
  \end{figure}


  % Да проследим изчислението на думата $aabbcc$:
  
  % \[_1aabbcc \vdash d_2abbcc \vdash da_2bbcc \vdash dad_3bcc \vdash dadb_3cc \vdash dad_4bdc \vdash da_4dbdc \vdash \cdots \vdash\]
  % \[_4dadbdc \vdash\ _4\blank dadbdc \vdash\ _1dadbdc \vdash d_1adbdc \vdash dd_2dbdc \vdash ddd_2bdc \vdash dddd_3dc \vdash \]
  % \[ ddddd_3c \vdash dddddd_4 \vdash \cdots \vdash\ _4\blank dddddd \vdash\ _1dddddd \vdash \cdots \vdash dddddd_1\blank \vdash dddddd_5\blank.\]

  Съобразете, че тази машина на Тюринг може да се направи тотална като се добави ново състояние $q_{reject}$
  и за всяка двойка $(q,x)$, за която функцията на преходите не е дефинирана, да сочи към $q_{reject}$.
  Така можем да получим {\em тотална} машина на Тюринг за езика $L$, което означава, че 
  $L$ е не само полуразрешим, но {\em разрешим} език.
\end{example}

\begin{example}
  \marginpar{Да напомним, че този език не е безконтекстен}
  \marginpar{В \cite[стр. 155]{hopcroft1} е дадено по-различно решение. Тук следваме \cite[стр. 173]{sipser3}. Там има малка грешка}
  Да разгледаме езика $L = \{\omega \sharp \omega \mid \omega\in\{a,b\}^\star\}$.
  Нека първо да видим, че можем неформално да опишем алгоритъм, който да разпознава думите на езика $L$.
  Нека една дума е копирана върху лентата и четящата глава е поставена върху първия символ от думата.
  \begin{enumerate}[(1)]
  \item 
    Чете $x$-ове надясно по лентата докато не срещне $a$ или $b$ и го замества с $x$.
    Запомня дали сме срещнали $a$ или $b$.
    Ако вместо $a$ или $b$ срещне $\sharp$, то отива на стъпка $(6)$.
  \item
    Чете $a$-та и $b$-та надясно по лентата докато не стигне $\sharp$. 
  \item
    Чете $c$-то надясно по лентата и всички следващи $x$-ове докато не срещне символа $a$ или $b$.
    Той трябва да е същия символ, който сме запаметили на стъпка $(1)$.
    Заместваме този символ с $x$.
  \item
    Чете $x$-ове наляво по лентата докато не стигне $\sharp$.
  \item
    Чете $a$-та и $b$-та по лентата докато не стигне $x$.
    Поставя четящата глава върху символа точно след първия $x$.
    Отива на стъпка $(1)$.
  \item
    Прочита $\sharp$ надясно по лентата и чете надясно $x$-ове докато не срещне $\blank$.
    Алгоритъмът завършва успешно.
  \end{enumerate}

  Ще построим машина на Тюринг $\M$, за която $L = \L(\M)$.
  \begin{itemize}
  \item 
    $\Sigma = \{a,b,\sharp\}$;
  \item
    $\Gamma = \{a,b,\sharp,x,\blank\}$;
  \item
    $Q = \{1,2,\dots,9\}$;
  \item
    $q_{accept} = 9$;
  \end{itemize}

  \begin{figure}[H]
    \begin{center}
      \begin{tikzpicture}[->,>=stealth,thick,node distance=50pt]
        \tikzstyle{every state}=[circle,minimum size=10pt,auto,scale=.9]
        
        \node[state,initial]    (1) {$1$};
        \node[state]            (2) [above right of=1]{$2$};
        \node[state]            (3) [below right of=1]{$3$};
        \node[state]            (4) [right of=2]{$4$};
        \node[state]            (5) [right of=3]{$5$};
        \node[state]            (6) [below right of=4]{$6$};
        \node[state]            (7) [above of=6]{$7$};
        \node[state]            (8) [left of=3]{$8$};
        \node[state,accepting]  (9) [below left of=3]{$9$};
        
        \begin{scope}[every node/.style={scale=.8}]
          \path
          (1) edge [bend left=15] node [below right] {$a/x;R$} (2)
              edge [bend right=15] node [above right] {$b/x;R$} (3)
              edge [bend right=15] node [left] {$\sharp;R$} (8)
          (2) edge [loop above] node [above] {$\{a,b\};R$} (2)
              edge [bend left=15] node [above] {$\sharp;R$} (4)
          (3) edge [loop below] node [below] {$\{a,b\};R$} (3)
              edge [bend right=15] node [below] {$\sharp;R$} (5)
          (4) edge [loop above] node [above] {$x;R$} (4)
              edge [bend left=15] node [below left] {$a/x;L$} (6)
          (5) edge [loop below] node [below] {$x;R$} (5)
              edge [bend right=15] node [above left] {$b/x;L$} (6)
          (6) edge [loop right] node [right] {$x;L$} (6)
              edge [bend right=15] node [right] {$\sharp;L$} (7)
          (7) edge [loop right] node [right] {$\{a,b\};L$} (7)
              edge [out=130,in=120,above,distance=2.5cm] node [above] {$x;R$} (1)
          (8) edge [loop left] node [left] {$x;R$} (8)
              edge [bend right=15] node [left] {$\blank;R$} (9);
        \end{scope}
      \end{tikzpicture}
    \end{center}
  \end{figure}

  % Да проследим изчислението на думата $ab\sharp ab$.
  
  % \begin{align*}
  %   (\varepsilon, 1, ab\sharp ab) & \to (x, 2, b\sharp ab) \to xb_2\sharp ab \to xb\sharp _4ab \to xb_6\sharp xb \to x_7b\sharp xb \to _7xb\sharp xb \to x_1b\sharp xb\\
  %   & \to xx_3\sharp xb \to xx\sharp _5xb \to xx\sharp x_5b \to xx\sharp _6xx \to xx_6\sharp xx \to x_7x\sharp xx \to xx_1\sharp xx \\
  %   & \to xx\sharp _8xx \to xx\sharp x_8x \to xx\sharp xx_8\blank \to xx\sharp xx\blank_9\blank
  % \end{align*}

  Може лесно да се съобрази, че тази машина на Тюринг може да се допълни до {\em тотална}.
  
\end{example}


\subsection*{Канонична подредба на $\Sigma^\star$}

Нека $\Sigma = \{a_0,a_1,\dots,a_{k-1}\}$.
Подреждаме думите по ред на тяхната дължина.
Думите с еднаква дължина подреждаме по техния числов ред, т.е.
гледаме на буквите $a_i$ като числото $i$ в $k$-ична бройна система.
Тогава думите с дължина $n$ са числата от $0$ до $k^n-1$ записани в $k$-ична бройна система.
Ще означаваме с $\omega_i$ $i$-тата дума в $\Sigma^\star$ при тази подредба.

\begin{example}
  Ако $\Sigma = \{0,1\}$, то наредбата започва така:
  \[\varepsilon, 0, 1, \underbrace{00, 01, 10, 11}_{\text{от $0$ до $3$}}, \underbrace{000, 001, 010, 011, 100, 101, 110, 111}_{\text{от $0$ до $7$}}, 0000, 0001, \dots\]
  В този случай, $\omega_0 = \varepsilon$, $\omega_7 = 000$, $\omega_{13} = 110$.
\end{example}

\subsection*{Многолентови машини на Тюринг}

%Това е просто като имаш shift.
%Използват се при недет. машини

Машина на Тюринг с $k$ ленти има същата дефиниция като еднолентова машина на Тюгинг
с единствената разлика, че
\[\delta: Q \times \Gamma^k\to Q \times \Gamma^k \times \{L,R,S\}^k.\]
Тук добавяме и възможността главата върху някои от лентите да стои на място.
\marginpar{В \cite[стр. 177]{sipser3} конструкцията е малко по-различна. Там съдържанието на всяка лента се поставя последователно върху една лента, като се разделят със специален символ}
\begin{prop}
  За всяка $k$-лентова машина на Тюринг $\M$ съществува еднолентова машина на Тюринг $\M'$,
  такава че $\L(\M) = \L(\M')$.
\end{prop}
\begin{proof}
  \marginpar{Тук на практика следваме \cite[стр. 162]{hopcroft1}}
  Нека $\M$ е $k$-лентова машина на Тюринг.
  Ще построим еднолентова машина на Тюринг $\M'$, за която $\L(\M) = \L(\M')$.
  Да означим $\hat\Gamma = \{\hat X \mid X \in \Gamma\}$.
  Тогава азбуката на лентата на $\M'$ ще бъде $\Gamma' = (\hat\Gamma \cup \Gamma)^{k}$.
  Сега вместо да имаме $k$ ленти ще имаме една лента, която представлява $k$-орка.
  За да симулираме $\M$, използваме символите $\hat X$ за да маркират позицията на главите на $\M$,
  като във всяка координата на лентата има точно по един символ от вида $\hat X$.
  % С $\$$ ще отблезяваме границите на всяка лента, в която можем да търсим маркера.
  Да да определим следващия ход на машината $\M'$, ние трябва да сканираме лентата докато не 
  открием разположението на всичките $k$ на брой маркирани клетки. Тогава симулираме ход на $\M$
  и отново трябва да променим маркираните клетки.
\end{proof}

\subsection*{Недетерминистични машини на Тюринг}

Една машина на Тюринг $\N$ се нарича недетерминистична, ако функцията на преходите има вида
\[\Delta_\N: Q\times \Gamma \to \Ps(Q \times \Gamma\times \{L,R\}). \]

Отново можем да дефинираме бинарна релация $\vdash_\N$ над $\Gamma^\star \times Q \times \Gamma^\star$,
която ще казва как моментното описание на машината $\N$ се променя при изпълнение на една стъпка.
\begin{itemize}
\item
  Ако $\Delta_\N(q,Z) \ni (p,Y,R)$, то дефинираме $(\alpha, q, Z\beta) \vdash_\N (\alpha Y, p, \beta)$.
\item 
  Ако $\Delta_\N(q,Z) \ni (p,Y,L)$, то дефинираме $(\alpha X, q, Z\beta) \vdash_\N (\alpha , p, XY\beta)$.
\end{itemize}
С $\vdash^\star_\N$ ще означаваме рефлексивното и транзитивно затваряне на $\vdash_\N$.

Тогава за недетерминистична машина на Тюринг $\N$, 
\[\L(\N) = \{\alpha\in\Sigma^\star \mid (\varepsilon, s, \alpha) \vdash^\star_\N (\beta, q_{accept}, \gamma), \text{ за някои }\beta,\gamma\in\Gamma^\star\}.\]

\begin{remark}
  Върху дадена дума $\omega$, недетерминистичната машина на Тюринг $\N$ може да има много различни изчисления.
  Думата $\omega$ принадлежи на $\L(\N)$ ако съществува {\em поне едно} изчисление, което завършва в състоянието $q_{accept}$.
  Възможно е много други изчисления за $\omega$ да завършват в $q_{reject}$ или да зациклят.
\end{remark}

Аналогично, дефинираме една недетерминистична машина на Тюринг $\N$ да бъде {\bf тотална}, ако за всяка дума и 
всяко изчисление на $\N$ върху $\omega$ завършва в $q_{accept}$ или $q_{reject}$.

\begin{framed}
  \begin{thm}
    Ако $L$ се разпознава от {\em недетерминистична} машина на Тюринг $\N$, то $L$
    също се разпознава и от {\em детерминистична} машина на Тюринг $\D$.
  \end{thm}
\end{framed}
\begin{proof}
  \marginpar{В \cite[стр. 164]{hopcroft1} не е добре обяснено.}
  Нека имаме недетерминистичната машина на Тюринг $\N$, за която $L = \L(\N)$.
  Една дума $\alpha$ принадлежи на $\L(\N)$ точно тогава, когато съществува изчисление 
  започващо с думата $\alpha$ върху лентата, което достига до състоянието $q_{accept}$.
  Сложността идва от факта, че за думата $\alpha$ може да имаме много различни изчисления, 
  като сямо някои от тях завършват в $q_{accept}$. Ще построим детерминистична машина на Тюринг,
  която последователно ще симулира всички възможни {\em крайни} изчисления за думата $\alpha$, докато 
  намери такова, което завършва в състоянието $q_{accept}$.
  \marginpar{На практика това, което е правим е да представим всички възможни изчисления на $\N$ като $r$-разклонено дърво и да го обходим в широчина, докато не достигнем до приемащо състояние}
  Лесно се съобразява, че всяко изчисление на $\N$ може да се представи като 
  крайна редица от елементи на $Q \times \Gamma \times \{L,R\}$.
  Понеже това множество е крайно, то можем на всяка такава тройка да
  съпоставим число в интервала $[1,r]$, където $r = |Q| \cdot |\Gamma|\cdot2$.
  Оттук следва, че всяко изчисление на $\N$ може да се представи като крайна 
  редица от числа в интервала $[1,r]$.

  Машината на Тюринг $\D$ има три ленти.
  
  \begin{itemize}
  \item 
    На първата лента съхраняваме входящия низ и тя никога не се променя.
  \item
    На втората лента ще записваме последователно низове следвайки каноничната подредба на 
    думите над азбуката $\{1,2,\dots,r\}$.
  \item
    На третата лента симулираме изчислението на $\N$ върху думата от първата лента, използвайки изчислението, 
    което е описано на втората лента. Например, ако съдържанието на втората лента е $4,1,2$,
    това означава, че симулираме изчисление от три стъпки като на първата стъпка избираме четвъртата
    възможна тройка, на втората стъпка избираме първата възможна тройка, на третата стъпка избираме втората възможна тройка.
    
    Ако симулацията завърши в състоянието $q_{accept}$ на $\N$, то машината $\D$ завършва успешно.
    В противен случай, на втората лента записваме следващия низ; изтриваме третата лента и започваме нова симулация.
  \end{itemize}
\end{proof}

\begin{prop}[Лема на Кьониг]
  Ако $T$ е безкрайно дърво с крайно разклонение, то $T$ съдържа безкраен път.
\end{prop}

\begin{cor}
  Ако $L$ се разпознава от {\em тотална недетерминистична} машина на Тюринг $\N$, то $L$
  също се разпознава и от {\em тотална детерминистична} машина на Тюринг $\D$.
\end{cor}
\begin{proof}
  Да разгледаме дървото $T$, което представя всички изчисления на $\N$ при вход думата $\omega$.
  От лемата на Кьониг следва, че $T$ е крайно дърво, защото ако допуснем, че $T$ е безкрайно, то ще има безкрайно дълго изчисление на $\N$,
  което е невъзможно, понеже $\N$ винаги достига до финално състояние.
  \begin{itemize}
  \item 
    Ако $\N$ приема дадена дума $\omega$, то детерминистичната ни симулация на $\N$ ще достигне до изчисление, кодирано като път в $T$, 
    което завършва в състояние $q_{accept}$.
  \item
    Ако $\N$ не приема дадена дума $\omega$, то детерминистичната ни симулация на $\N$ ще покаже, че всяко изчисление, кодирано като път в $T$, завършва в състояние $q_{reject}$.
  \end{itemize}
\end{proof}


\section{Машини на Тюринг като генератори}

\marginpar{\cite[стр. 168]{hopcroft1}}
\marginpar{\cite[стр. 180]{sipser3}}
\marginpar{На англ. се наричат {\em enumerators}}

Нека да разгледаме един вариант на многолентовите машини на Тюринг, които ще наричаме {\bf генератори}.
Нека машината на Тюринг да има две ленти, като в началото и двете ленти са празни.
\begin{itemize}
\item 
  Първата лента ще служи за работна лента - върху нея можем да пишем и четем;
\item
  Втората лента служи единствено за изход - върху нея можем само да пишем пишем думи; не можем да четем какво вече сме написали върху нея и не можем да пишем върху вече записана клетка. Думите са разделени със специален символ - $\#$.
  Това означава, че втората лента има вида
  \[\omega_1\#\omega_2\#\cdots\#\omega_n\#\blank\blank\cdots\]
\item
  Езикът, които се извежда от такъв генератор е съставен от думите, които са изписани на изходната лента.
  Такива езици ще наричаме {\bf изчислимо изброими}.
  Обърнете внимание, че измежду думите на изходната лента е възможно да има повторения.
  Ако езикът е безкраен, то машината ще работи безкрайно много време.
\end{itemize}

\begin{framed}
  \begin{thm}
    Един език $L$ е полуразрешим точно тогава, когато $L$ е изчислимо номеруем.
  \end{thm}
\end{framed}
\begin{proof}
  $(\Leftarrow)$ Нека $L$ да се номерира от генераторът $E$.
  Машината на Тюринг $\M$, за която $L = \L(\M)$ ще работи по следния начин:
  \begin{enumerate}[1)]
  \item 
    При вход думата $\omega$, $\M$ започва да симулира $E$;
  \item
    Когато се появи дума $\gamma$ върху изходната лента на $E$, сравняваме $\omega$ с $\gamma$;
  \item
    Ако $\omega = \gamma$, то отиваме в състоянието $q_{accept}$ на $\M$ и завършваме;
  \item
    В противен случай, отиваме обратно на стъпка $2)$.
  \end{enumerate}

  $(\Rightarrow)$ Нека сега $L = \L(\M)$. Целта ни е да изведем всички думи на $L$ върху изходната лента.
  Основният проблем е, че за дадена дума $\omega$, не знаем за колко стъпки трябва да симулираме $\M$ за да сме сигурни дали думата $\omega \in \L(\M)$ или не. Оказва се, че можем да разрешим този проблем като позволяме да извеждаме повторящи се думи.
  За целта, да подредим всички думи $\omega_1, \omega_2, \dots $ над азбуката $\Sigma$ спрямо каноничната наредба.
  \begin{enumerate}[1)]
  \item
    Нека $s = 1$;
  \item 
    Симулираме $\M$ върху думите $\omega_1,\dots,\omega_s$ за $s$ стъпки;
  \item
    За всяка от тези думи $\omega_i$, които се приемат от $\M$, записваме ги върху изходната лента.
  \item
    Нека $s = s+1$; Отиваме обратно на стъпка $2)$.
  \end{enumerate}
\end{proof}

\begin{remark}
  В последната конструкция позволяваме думите на един полуразрешим език $L$ да се 
  извеждат върху изходната лента многократно. Можем лесно да осигурим условието всяка дума на $L$
  да се извежда точно по веднъж.
  На стъпка $s = \pair{i,j}$, то проверяваме дали думата $\omega_i$ се приема успешно от $\M$
  за {\em точно} $j$ на брой стъпки. Само тогава думата се записва на изходната лента.
  
  Обърнете внимание, че не можем да осигурим условието думите да се извеждат във възходящ ред
  относно каноничната наредба.
\end{remark}

\begin{framed}
  \begin{thm}
    Един език $L$ е разрешим точно тогава, когато съществува генератор за $L$, 
    който изписва думите на $L$ във възходящ ред относно каноничната наредба.
  \end{thm}
\end{framed}
\begin{proof}
  $(\Rightarrow)$ Нека $L = \L(\M)$. Тази посока е лесна, защото $\M$ е тотална машина,
  т.е. за всеки вход $\M$ завършва или в $q_{accept}$ или в $q_{reject}$.
  \begin{enumerate}[1)]
  \item 
    Нека $s = 1$;
  \item
    Симулираме $\M$ върху думата $\omega_s$.
  \item
    Ако симулацията завърши в състояние $q_{accept}$, то записваме $\omega_s$
    върху изходната лента. 
  \item
    Иначе ако симулацията завърши в състояние $q_{reject}$, то нищо не записваме върху изходната лента. 
  \item
    Нека $s = s+1$. Отиваме на стъпка $2)$.
  \end{enumerate}

  \marginpar{Ако имам генератор $G$ за $L$ няма алгоритъм, който да ми каже дали $L$ е безкраен език или не. Това означава, че по код на $G$ няма как ефективно да получа код на $\M$}
  $(\Leftarrow)$ Ако $L$ е краен, то е ясно, че мога да разпозная езика с краен автомат, което е частен случай на тотална машина на Тюринг.
  По-интересният случай е когато $L$ е безкраен език.
  Нека $L$ се генерира от машината на Тюринг $G$ като извежда думите на $L$ във възходящ ред.
  \begin{itemize}
  \item 
    Вход дума $\omega$;
  \item
    Симулираме $G$ като гледаме думите, които се извеждат на изходната лента.
    Ако срещнем думата $\omega$, то завършваме в състояние $q_{accept}$.
  \item
    Ако срещнем думата $\gamma$, която е по-голяма от $\omega$ относно каноничната наредба, 
    то завършваме в състояние $q_{reject}$.
  \end{itemize}
\end{proof}




\subsection*{Тезис на Чърч-Тюринг}

\section{Универсална машина на Тюринг}
За простота, нека $\Sigma = \{0,1\}$ и $\Gamma = \{0,1,\blank\}$.
\begin{itemize}
\item 
  $X_1 = 0$, $X_2 = 1$, $X_3 = \blank$;
\item
  $D_1 = L$, $D_2 = R$
\end{itemize}

\subsection*{Кодиране на преход}
Да разгледаме прехода $\delta(q_i,X_j) = (q_k,X_l,D_m)$.
Кодираме този преход по следния начин:
\[0^i10^j10^k10^l10^m\]
Да обърнем внимание, че в този двоичен код няма последователни единици и той 
започва и завършва с нула.
\subsection*{Кодиране на машина на Тюринг}
За да кодираме една машина на Тюринг $\M$ е достатъчно да кодираме функцията на преходите $\delta$.
Понеже $\delta$ е крайна функция, нека с числото $r$ да означим броя на всички възможни преходи.
По описания по-горе начин, нека $code_i$ е числото в двоичен запис, получено за $i$-тия преход на $\delta$.
Тогава кодът на $\M$ е следното число в двоичен запис:
\[\pair{\M} = 111\ code_1\ 11\ code_2\ 11\ \cdots\ 11\ code_r\ 111.\]
\begin{itemize}
\item
  Лесно се съобразява, че за две МТ $\M$ и $\M'$ с различни функции на преходите, имаме $\pair{\M} \neq \pair{\M'}$.
\item
  Ще казваме, че числото $r\in\Nat$ е {\bf код на} $\M$, ако $r$, записано в двоичен запис представлява думата $\pair{\M}$.
  Оттук нататък, когато пишем $\M_r$, ще имаме предвид машината на Тюринг с код $r$.
\item
  Ясно е, че не всяко естествено число е код на машина на Тюринг, но по дадено число $n$
  има ефективна процедура, която ни казва дали $n$ е код на машина на Тюринг или не.
\item
  С $\pair{\M,w}$ ще означаваме кода на МТ $\M$ при вход $w$ е числото с двоичен запис описанието на $\M$ и след това прикрепена думата $w$.
  При едно число $r = \pair{M,w}$, лесно се намира кода на $\M$.
  Просто започваме да четем двоичния запис на $r$ докато не срещнем за втори път $111$.
  След това започва думата $w$.
\end{itemize}


\section{Изчислими функции}

Нека е дадена функцията $f:\Nat^k \to \Nat$.
Ще казваме, че $f$ е изчислима с машината на Тюринг $\M$,
ако за всяко $n_1,\dots,n_k$ е изпълнено:
\begin{itemize}
\item 
  Представяме всяко от числата $n_1,\dots,n_k$ в монадична бройна система
  като лентата на $\M$ има вида:  
  \[\dots \blank \blank \underbrace{1111\dots 11}_{n} \blank\blank\dots,\]
  като изискваме главата на $\M$ да е позиционирана върху най-лявата единица.
  Такава конфигурация ще наричаме {\bf стандартна начална конфигурация}.
\item
  Ако $f(n_1,\dots,n_k) = m$, то $\M$ завършва с резултат върху лентата
  \[\dots \blank \blank \underbrace{1111\dots 11}_{m} \blank\blank\dots,\]
  като главата на $\M$ е върху най-лявата 1-ца.
  Такава конфигурация се нарича {\bf стандартна финална конфигурация}.
\item
  Ако $f(n_1,\dots,n_k)$ е недефенирана, то $\M$ няма да завърши в стандартна конфигурация, т.е.
  или $\M$ ще работи безкрайно време, или ще завърши в конфигурация, която не е стандартна.
\end{itemize}

\begin{example}

  Да видим, че функцията $f(n) = 2n$ е изчислима.
  
% \[\stackrel{1}{1}11\ \Rightarrow\ \stackrel{2}{\blank} 111\ \Rightarrow\ \stackrel{3}{\blank}\blank 111\ \Rightarrow\ \stackrel{4}{\blank}1\blank 111\ \Rightarrow\ 1\stackrel{5}{1}\blank 111\ \Rightarrow\  11\stackrel{5}{\blank}111\ \Rightarrow\  11 \blank \stackrel{6}{1}11\]
% \[\dots \Rightarrow\ 11\blank 111\stackrel{6}{\blank}\ \Rightarrow\ 11\blank 11\stackrel{7}{1}\blank\ \Rightarrow\ 11\blank 1\stackrel{8}{1}\blank\blank\ \Rightarrow\ 11\blank \stackrel{9}{1}1\blank\blank\ \Rightarrow\  1\stackrel{10}{1}\blank 11\blank\blank\]
% \[\dots \Rightarrow\ \stackrel{10}{\blank}11\blank 11\blank\blank\ \Rightarrow\ \stackrel{2}{1}1\blank 11\blank\blank\ \Rightarrow\ \cdots\]

\begin{figure}[H]
  \begin{center}
    \begin{tikzpicture}[->,>=stealth,thick,node distance=45pt]
      \tikzstyle{every state}=[circle,minimum size=10pt,auto,scale=.7]
      
      \node[state]   (1) {$1$};
      \node[state]            (2) [right of=1]{$2$};
      \node[state]            (3) [right of=2]{$3$};
      \node[state]            (4) [right of=3]{$4$};
      \node[state]            (5) [right of=4]{$5$};
      \node[state]            (6) [right of=5]{$6$};
      \node[state]            (7) [right of=6]{$7$};
      \node[state]            (8) [right of=7]{$8$};
      \node[state]            (9) [right of=8]{$9$};
      \node[state]            (10) [right of=9]{$10$};
      \node[state]            (11) [below of=8]{$11$};
      \node[state,accepting]  (12) [below of=11]{$12$};
      
      \begin{scope}[every node/.style={scale=.8}]
      \path
      (1) edge [bend left=15] node [above] {$1;L$} (2)
      (2) edge [bend left=15] node [above] {$1;L$} (3)
      (2) edge [bend right=15] node [below] {$\blank;L$} (3)
      % (3) edge [bend left=15] node [above] {$1/1;L$} (4)
      (3) edge [bend left=15] node [above] {$\blank/1;L$} (4)
      (4) edge [bend left=15] node [above] {$\blank/1;R$} (5)
      % (4) edge [bend right=15] node [below] {$1/1;R$} (5)
      (5) edge [loop below] node [below] {$1;R$} (5)
      (5) edge [bend left=15] node [above] {$\blank;R$} (6)
      (6) edge [loop below] node [below] {$1;R$} (6)
      (6) edge [bend left=15] node [above] {$\blank;L$} (7)
      (7) edge [bend left=15] node [above] {$1/\blank;L$} (8)
      % (7) edge [bend right=15] node [below] {$\blank;L$} (8)
      (8) edge [bend left=15] node [above] {$1;L$} (9)
      (9) edge [loop below] node [below] {$1;L$} (9)
      (9) edge [bend right=15] node [below] {$\blank;L$} (10)
      (10) edge [loop below] node [below] {$1;L$} (10)
      % (10) [out=130,in=120,above,distance=2.5cm] node [above] {$x;R$} (2)
      (10) edge [out=140,in=60, above] node [below] {$\blank;R$} (2)
      (8) edge [] node [right] {$\blank;L$} (11)
      (11) edge [loop left] node [left] {$1;L$} (11)
      (11) edge [] node [right] {$\blank;R$} (12);
      \end{scope}
    \end{tikzpicture}
  \end{center}
\end{figure}

\begin{align*}
  _1111 & \to\ _2\blank 111 \to\  _3\blank \blank 111 \to\ _4\blank 1 \blank 111 \to 1_51\blank 111 \to 11_5\blank 111\\
  & \to 11\blank_6 111 \to \cdots \to 11 \blank 11_71 \to 11 \blank 1_81\blank \to 11 \blank _911\blank \to 11_9 \blank 11\blank\\
  & \to 1_{10}1\blank 11 \blank \to \cdots \to\ _211\blank 11\blank \to \cdots \to 1_5111\blank 11 \blank \to \cdots \\
  & \to 1111\blank_6 11\blank \to \cdots
\end{align*}

\end{example}

\begin{problem}
  За произволно естествено число $n$, дефинирайте МТ $\M_n$ с $n+11$ състояния, за която,
  ако главата е на най-лявата $1$-ца върху блок от $1$-ци, то $\M_n$
  завършва като записва $2n$ единици на лентата и завършва в стандартна конфигурация.
\end{problem}

\begin{thm}
  За всяко $k$, съществуват функции от вида $f:\Nat^k\to\Nat$, които не са изчислими с МТ.
\end{thm}
\begin{proof}
  Знаем, че всяка МТ може да се кодира с естествено число.
  Това означава, че съществуват изброимо безкрайно много различни машини на Тюринг.
  Също така, ние знаем, че съществуват неизброимо много различни функции от вида $f:\Nat^k\to\Nat$.
  Заключаваме, че със сигурност съществуват функции, които не са изчислими с МТ.
\end{proof}

\section{Примери за разрешими и полуразрешими езици}

Да напомним, че:
\begin{itemize}
\item
  {\bf полуразрешими} са тези езици, които се разпознават от машина на Тюринг.
\item
  {\bf разрешими} са тези езици, които се разпознават от тотална машина на Тюринг.
\end{itemize}

Ще разгледаме два основни примера за езици. Единият ще бъде за език, който не е полуразрешим, а другият за език, който е 
полуразрешим, но не е разрешим.

\subsection{Диагоналният език $L_d$}

Нека $\omega_0,\omega_1,\dots,\omega_n,\dots$ е каноничната подредба на всички думи над азбуката $\{0,1\}$.
Да разгледаме безкрайната таблица $\{a_{ij} \mid i,j \in \Nat\}$, където:
\begin{align*}
  a_{ij} = 
  \begin{cases}
    1, & \text{ ако } \omega_i \in L(\M_j), \\
    0, & \text{ ако } \omega_i \not\in L(\M_j).
  \end{cases}
\end{align*}
Идеята е да вземем $0$-ите по диагонала на тази таблица.

\begin{framed}
  Езикът 
  $L_d = \{w_i \mid w_i \not\in L(\M_i)\}$ не се разпознава от MT,
  т.е. $L_d$ {\bf не} е полуразрешим.
\end{framed}
Да допуснем, че $L_d$ се разпознава от машина на Тюринг, т.е. $L_d = \L(\M_i)$, за някоя машина на Тюринг с код $i$.
Тогава:
\begin{align*}
  & \omega_i \in L_d \implies \omega_i \in \L(\M_i) \implies \omega_i \not\in L_d,\\
  & \omega_i \not\in L_d \implies \omega_i \not\in \L(\M_i) \implies \omega_i \in L_d.
\end{align*}
Достигаме до противоречие.

\begin{remark}
  Да обърнем внимание, че $\bar{L}_d = \{\omega_i \mid \omega_i \in \L(\M_i)\}$ е полуразрешим език,
  който очевидно не е разрешим.
\end{remark}

\subsection{Универсалният език $L_u$}

Да разгледаме езика $L_u = \{\pair{\M,\omega} \mid \omega\in \L(\M)\}$.

\begin{lemma}
  $L_u$ е полуразрешим език.
\end{lemma}

\begin{lemma}
  $\bar{L}_u = \{\pair{\M,\omega} \mid \omega\not\in \L(\M)\}$ {\bf не} е полуразрешим език.
\end{lemma}

\begin{framed}
  \begin{thm}
    Универсалният език $L_u$ е полуразрешим, но {\bf не} е разрешим.
  \end{thm}
\end{framed}

Да допуснем, че $L_u$ е разрешим.
\begin{itemize}
\item 
  Вход думата $\omega$;
\item
  Намираме каноничния индекс $i$ на $\omega$, т.е. $\omega_i = \omega$;
\item
  Намираме машината на Тюринг $\M_i$ с код $i$;
\item
  Симулираме $\pair{\M_i,\omega_i}$. % върху $\M$.
\end{itemize}

Така получваме, че:
\[\omega_i\in \bar{L}_d \iff \M_i\text{ приема }\omega_i \iff \pair{\M_i,\omega_i} \in L_u.\]
Заключаваме, че $\bar{L}_d$ е разрешим език, което е противоречие.

% \section{Проблемът за съответствие на Пост (PCP)}

% \subsection*{MPCP}

\section{Критерий за разрешимост}
\marginpar{\cite{hopcroft1}, стр. 188}

Нека $\Ss$ е множество от полуразрешими езици над азбуката $\{0,1\}$.
Ще казваме, че $\Ss$ е свойство на полуразрешимите езици.
$\Ss$ е тривиално свойство, ако $\Ss = \emptyset$ или $\Ss$ съдържа точно всички полуразрешими езици.
Нека $L_\Ss = \{\pair{\M} \mid \L(\M) \in \Ss\}$.

\begin{thm}[Райс-Успенски]
  \index{Райс-Успенски}
  Всяко нетривиално свойство $\Ss$ на полуразрешимите езици е неразрешимо.
\end{thm}
\begin{proof}
  \marginpar{Цел: да сведем ефективно $L_u$ към $L_\Ss$}
  Без ограничение на общността, нека $\emptyset \not\in \Ss$.
  Понеже $\Ss$ е нетривиално свойство, да разгледаме $L \in \Ss$,
  като $\M_L$ е машина на Тюринг, за която $\L(\M_L) = L$.
  Да разгледаме алгоритъм $A$, който за дадена дума $\pair{\M,w}$
  връща код на машина на Тюринг $\M'$, която работи по следния начин:
  \begin{itemize}
  \item
    \marginpar{Неформално описваме функцията $\delta$ за $\M'$}
    имаме вход - произволна дума $x$;
  \item
    \marginpar{$\M$ и $w$ са предварително фиксирани. Кодът на $\M'$ зависи от тях}
    първоначално не обръщаме внимание на $x$, а питаме дали $\pair{\M,w} \in L_u$, т.е. дали $\M$ приема думата $w$;
    \begin{itemize}
    \item 
      ако съществува стъпка $s$, за която $\pair{\M,w} \in L^s_u$, то симулираме $\M_L$ върху входната дума $x$;
      в този случай получаваме $\L(\M') = L$;
    \item
      ако не съществува стъпка $s$, за която $\pair{\M,w} \in L^s_u$, то 
      няма да разпознаем нито една дума;
      в този случай получаваме $\L(\M') = \emptyset$.      
    \end{itemize}
  \end{itemize}
  От всичко това следва, че ако $A(\pair{\M,w}) = \pair{\M'}$, то:
  \begin{align*}
    & \pair{\M,w} \in L_u \implies (\exists s)[\pair{\M,w} \in \L^s_u] \implies \L(\M') = L \implies \L(\M') \in \Ss,\\
    & \pair{\M,w} \not\in L_u \implies (\forall s)[\pair{\M,w} \not\in \L^s_u] \implies \L(\M') = \emptyset \implies \L(\M') \not\in \Ss.
  \end{align*}
  Да допуснем, че $\Ss$ е разрешимо множество от полуразрешими езици.
  Тогава от еквивалентността,
  \[\pair{\M,w} \in \L_u \iff \pair{\M'} \in L_\Ss,\]
  получаваме, че $\L_u$ е разрешимо множество, което е противоречие.
\end{proof}

\begin{cor}
  Следните свойства $\Ss$ на полуразрешимите множества {\bf не} са разрешими:
  \begin{enumerate}[a)]
  \item 
    празнота, т.е. $\Ss = \{\pair{\M} \mid \L(\M) = \emptyset\}$;
  \item
    крайност, т.е. $\Ss = \{\pair{\M} \mid |\L(\M)| < \infty\}$;
  \item
    регулярност, т.е. $\Ss = \{\pair{\M} \mid (\exists \text{ рег. израз }r)[\L(\M) = \L(r)]\}$;
  \item
    безконтекстност, т.е. $\Ss = \{\pair{\M} \mid (\exists\text{ безконт. грам. }G)[\L(\M) = \L(G)]\}$.
  \end{enumerate}
\end{cor}

\subsection*{Валидни и невалидни изчисления на машини на Тюринг}

\begin{lemma}
  \marginpar{\cite{hopcroft1}, стр. 201}
  Множеството от валидни изчисления на машина на Тюринг $\M$ е сечението на два безконтекстни езика $L_1$ и $L_2$.
  Освен това, граматиките на $L_1$ и $L_2$ могат ефективно да бъдат построени от $\M$.
\end{lemma}
\begin{hint}
  Да разгледаме езика
  \[L_3 = \{\alpha\#\beta^R \mid \alpha \vdash_\M \beta\}.\]
  Лесно е да построим стеков автомат $P_3$, който разпознава езика $L_3$.
  Четем буквата $X$. Тогава:
  \begin{itemize}
  \item 
    ако $\delta_\M(q,X) =(p,Y,R)$, то слагаме $Yp$ на върха на стека;
  \item
    ако $\delta_\M(q,X) =(p,Y,L)$, то ако $Z$ е върха на стека, заменяме $Z$ с $pZY$;
  \end{itemize}
  Аналогично разглеждаме безконтекстния език
  \[L_4 = \{\alpha^R\#\beta \mid \alpha \vdash_\M \beta\}.\]
  Сега можем да дефинираме езиците
  \begin{align*}
    & L_1 = (L_3\#)^\star(\{\varepsilon\}\cup \Gamma^\star F \Gamma^\star\#)\\
    & L_2 = q_0\Sigma^\star(L_4\#)^\star(\{\varepsilon\}\cup \Gamma^\star F \Gamma^\star\#),
  \end{align*}
  за които е ясно, че са безконтекстни.
\end{hint}

\begin{thm}
  Въпросът дали две произволни безконтекстни граматики $G_1$ и $G_2$, $\L(G_1) \cap \L(G_2) = \emptyset$,
  е неразрешим.
\end{thm}

\begin{lemma}
  Множеството от невалидни изчисления на машина на Тюринг е безконтекстен език.
\end{lemma}

\begin{thm}
  Въпросът дали за произволна безконтекстна граматика $G$, $\L(G) = \Sigma^\star$,
  е неразрешим.
\end{thm}

\begin{cor}
  Нека $G_1$ и $G_2$ са произволни безконтекстни граматики, а $r$ е произволен регулярен израз.
  Следните проблеми са неразрешими:
  \begin{enumerate}
  \item 
    $\L(G_1) = \L(G_2)$;
  \item
    $\L(G_2) \subseteq \L(G_1)$;
  \item
    $\L(G_1) = \L(r)$;
  \item
    $\L(r) \subseteq \L(G_1)$.
  \end{enumerate}
\end{cor}

\section{Критерии за полуразрешимост}

\begin{lemma}
  Нека $\Ss$ е свойство на полуразрешимите езици.
  Ако съществува безкраен език $L_0 \in \Ss$, който няма крайно подмножество в $\Ss$,
  то $L_\Ss$ не е полуразрешим език.  
\end{lemma}
\begin{hint}
  Нека $L_0 = \L(\M_0)$.
  Ще опишем алгоритъм, който при вход дума $\pair{\M,\omega}$,
  извежда код на машина на Тюринг $\M'$, която работи така:
  \begin{itemize}
  \item 
    вход думата $\alpha$;
  \item
    за $\abs{\alpha}$ стъпки симулираме $\M$ върху $\omega$.
    \begin{itemize}
    \item 
      ако $\M$ приема $\omega$ за $\leq \abs{x}$ стъпки, то симулираме $\M_0$ върху $\alpha$;
    \item 
      ако $\M$ не приема $\omega$ за $\leq \abs{x}$ стъпки, то зацикляме и нищо не връщаме.
    \end{itemize}
  \end{itemize}

  Така получаваме, че 
  \begin{align*}
    \L(\M') = 
    \begin{cases}
      \{\alpha \in L_0 \mid \abs{\alpha} < k\}, & \M\text{ приема }\omega\\
      L, & \M\text{ не приема }\omega,
    \end{cases}
  \end{align*}
  където $k$ е минималната стъпка, при която $\M$ приема $\omega$.
  
  Заключаваме, че 
  \[\pair{\M,\omega} \not\in L_u \iff \L(\M') \in \Ss.\]
  Това означава, че ефективно можем да сведем въпрос за принадлежност в $\bar{L}_u$
  към въпрос за принадлежност в $L_\Ss$.
  Следователно, ако $L_\Ss$ е полуразрешим език, то $\bar{L}_u$ е полуразрешим език, което е противоречие.
\end{hint}

\begin{cor}
  Следните езици {\bf не} са полуразрешими:
  \begin{itemize}
  \item 
    $L = \{\pair{\M} \mid \abs{\L(\M)} = \infty\}$;
  \item
    $L = \{\pair{\M} \mid \L(\M) = \Sigma^\star\}$;
  \item
    $L = \{\pair{\M} \mid \L(\M)\text{ не е разрешим}\}$;
  \item
    $L = \{\pair{\M} \mid \L(\M)\text{ не е полуразрешим}\}$;
  \item
    $L = \{\pair{\M} \mid \L(\M)\text{ не е регулярен}\}$.
  \end{itemize}
\end{cor}

\begin{lemma}
  Нека $L_1$ е език в $\Ss$ и нека $L_2$ е полуразрешимо множество, разширяващо $L_1$, и $L_2 \not\in\Ss$.
  Тогава $L_\Ss$ не е полуразрешимо.
\end{lemma}
\begin{hint}
  Нека $L_1 = \L(\M_1)$ и $L_2 = \L(\M_2)$.
  Ще опишем алгоритъм, който при вход дума $\pair{\M,\omega}$,
  извежда код на машина на Тюринг $\M'$, която работи така:
  \begin{itemize}
  \item 
    вход думата $\alpha$;
  \item
    симулираме едновременно две изчисления - $\M_1$ върху $\alpha$ и $\M$ върху $\omega$:
    \begin{itemize}
    \item 
      ако $\M_1$ приеме думата $\alpha$, то обявяваме, че $\M'$ приема $\alpha$ и завършваме.
    \item
      ако достигнем стъпка $s$, за която $\M^s_1$ все още не приема думата $\alpha$,
      но $\M^s$ приема $\omega$, то започваме да симулираме $\M_2$ върху $\alpha$.
      Ако $\M_2$ приеме $\alpha$, то $\M'$ приема $\alpha$.
    \end{itemize}
  \end{itemize}
  
  Получаваме, че:
  \begin{align*}
    \L(\M') = 
    \begin{cases}
      L_2, & \M\text{ приема }\omega\\
      L_1, & \M\text{ не приема }\omega.
    \end{cases}
  \end{align*}
  Заключаваме, че:
  \[\pair{\M,\omega} \not\in L_u \iff \L(\M') \in \Ss.\]
  Това означава, че ефективно можем да сведем въпрос за принадлежност в $\bar{L}_u$
  към въпрос за принадлежност в $L_\Ss$.
  Следователно, ако $L_\Ss$ е полуразрешим език, то $\bar{L}_u$ е полуразрешим език, което е противоречие.  
\end{hint}

\begin{cor}
  Следните езици {\bf не} са полуразрешими:
  \begin{itemize}
  \item 
    $L = \{\pair{\M} \mid \L(\M) \text{ е регулярен} \}$;
  \item
    $L = \{\pair{\M} \mid \L(\M) \text{ е безконтекстен} \}$;
  \item
    $L = \{\pair{\M} \mid \L(\M) \text{ е разрешим} \}$;
  \item
    $L = \{\pair{\M} \mid \abs{\L(\M)} = 42\}$;
  \end{itemize}
\end{cor}


% \section{Проблеми за безконтекстни езици}

% \begin{lemma}
%   Нека е дадена $\M = \TM$.
%   Тогава езикът 
%   \[L = \{\alpha\sharp\beta^R \mid \alpha,\beta \in \Gamma^\star Q \Gamma^\star\ \&\  \alpha \vdash_\M \beta\}\]
%   е безконтекстен.
% \end{lemma}
% \begin{proof}
%   Ще покажем, че съществува стеков автомат $P$, за който $\L_S(P) = L$.
%   Четем буквата $X$. Тогава:
%   \begin{itemize}
%   \item 
%     ако $\delta_\M(q,X) =(p,Y,R)$, то слагаме $Yp$ на върха на стека;
%   \item
%     ако $\delta_\M(q,X) =(p,Y,L)$, то ако $Z$ е върха на стека, заменяме $Z$ с $pZY$;
%   \end{itemize}
% \end{proof}

% \begin{lemma}
%   Нека е дадена $\M = \TM$.
%   Тогава езикът 
%   \[L = \{\alpha\sharp\beta^R \mid \alpha,\beta \in \Gamma^\star Q \Gamma^\star\ \&\  \alpha \not\vdash_\M \beta\}\]
%   е безконтекстен.
% \end{lemma}


% \begin{thm}
%   Неразрешим е проблемът за проверка дали при дадени две произволни безконтекстни граматики $G_1$ и $G_2$,
%   $\L(G_1) \cap \L(G_2) = \emptyset$.  
% \end{thm}

% \begin{thm}
%   Неразрешим е проблемът за проверка дали при дадена произволна безконтекстна граматика $G$,
%   $\L(G) = \Sigma^\star$.  
% \end{thm}


% \section{Въпроси}

% Вярно ли е, че следният проблем е {\em разрешим}:
% \begin{itemize}
% \item
%   за произволна безконтекстна граматика $G$, проверява дали $\L(G) = \emptyset$?
% \item
%   за произволна безконтекстна граматика $G$, проверява дали $\L(G) = \Sigma^\star$?
% \item
%   за произволни безконтекстни граматики $G_1$ и $G_2$, проверява дали $\L(G_1) \cap \L(G_2) = \emptyset$?
% \item
%   за произволни безконтекстни граматики $G_1$ и $G_2$, проверява дали $\L(G_1) \cap \L(G_2) = \Sigma^\star$?
% \item
%   за произволни безконтекстни граматики $G_1$ и $G_2$, проверява дали $\L(G_1) = \L(G_2)$?
% \item
%   за произволни безконтекстни граматики $G_1$ и $G_2$, проверява дали $\L(G_1) \subseteq \L(G_2)$?
% \item
%   за произволна безконтекстна граматика $G$ и произволен регулярен израз $r$,
%   проверява дали $\L(G) = \L(r)$?
% \item
%   за произволна безконтекстна граматика $G$ и произволен регулярен израз $r$,
%   проверява дали $\L(G) \subseteq \L(r)$?
% \item
%   за произволна безконтекстна граматика $G$ и произволен регулярен израз $r$,
%   проверява дали $\L(r) \subseteq \L(G)$?
% \item
%   за произволни безконтекстни граматики $G_1$ и $G_2$, проверява дали $\L(G_1) \subseteq \L(G_2)$ 
%   е безконтекстен език ?
% \item
%   за произволна безконтекстна граматика $G$, проверява дали $\Sigma^\star \setminus \L(G)$
%   е безконтекстен език ?
% \item
%   за произволна безконтекстна граматика $G$, проверява дали $\L(G)$ е регулярен език?
% \end{itemize}

\section{Неограничени граматики}
\index{граматика!неограничена}

\begin{dfn}
  \marginpar{(стр. 220 от \cite{hopcroft1})}
  \marginpar{На англ. unrestricted grammar}
  \marginpar{Според йерархията на Чомски, това е граматика от тип 0}
  Граматиката $G = (V,\Sigma,R,S)$
  се нарича неограничена граматика, 
  ако правилата $R$ са от вида $\alpha \to \beta$,
  където $\alpha,\beta \in (V\cup\Sigma)^\star$.
\end{dfn}

\begin{lemma}
  За всеки полуразрешим език $L$, $L = \L(G)$, за някоя неограничена граматика $G$.  
\end{lemma}
\begin{proof}
  Нека $L = \L(\M)$, където 
  \[\M = \TM\] е детерминистична машина на Тюринг,
  като искаме лентата да е безкрайна само отдясно и входната дума $\alpha$ е
  поставена в началото на лентата.
  Ще построим граматика $G = \CFG$, където 
  \[V = ((\Sigma\cup\{\varepsilon\})\times\Gamma) \cup \{A_1,A_2,A_3\}.\]
  Правилата на $G$ са следните:
  \begin{enumerate}[1)]
  \item 
    $A_1 \to sA_2$;
  \item
    $A_2 \to [a,a]A_2$, за всяка $a\in\Sigma$;
  \item
    $A_2 \to A_3$;
  \item
    $A_3 \to [\varepsilon,\blank]A_3$;
  \item
    $A_3 \to \varepsilon$;
  \item
    $q[a,X] \to [a,Y]p$, за всяка $a \in \Sigma\cup\{\varepsilon\}$, всяко $q\in Q$, $X,Y \in\Gamma$, 
    за които $\delta(q,X) = (p,Y,R)$;
  % \item
  %   $q[a,X] \to p[a,Y]$, за всяка $a \in \Sigma\cup\{\varepsilon\}$, всяко $q\in Q$, $X,Y \in\Gamma$, 
  %   за които $\delta(q,X) = (p,Y,N)$;
  \item
    $[b,Z]q[a,X] \to p[b,Z][z,Y]$, за всяко $X,Y,Z \in \Gamma$, $a,b\in\Sigma\cup\{\varepsilon\}$, $q\in Q$,
    за които $\delta(q,X) = (p,Y,L)$;
  \item
    $[a,X]q \to qaq$, $q[a,X] \to qaq$, $q \to \varepsilon$, за всяко $a\in\Sigma\cup\{\varepsilon\}$, $X\in\Gamma$,
    и $q \in F$.
  \end{enumerate}
  
  Лесно се вижда, че, използвайки правилата 1) и 2), за всяко $n$, имаме
  \[A_1 \to^\star s[a_1,a_1]\cdots[a_n,a_n]A_2,\]
  където $a_i \in \Sigma$.

  Нека сега да приемем, че $\M$ приема думата $\alpha = a_1\cdots a_n$.
  Това означава, че за някое $m$, $\M$ използва не повече от $m$ клетки от лентата отдясно на входната дума.
  Ясно е, че имаме
  \[A_1 \to^\star s[a_1,a_1]\cdots[a_n,a_n][\varepsilon,\blank]^m.\]
  Оттук нататък, можем да използваме само правилата 6), 7), 8), докато не срещнем финално състояние.
  С индукция по броя на стъпки в $\M$, можем да докаже, че ако е изпълнено
  $(\varepsilon,s,a_1\cdots a_n) \vdash^\star_\M (X_1\cdots X_{r-1},q,X_r\cdots X_l)$, 
  то \[s[a_1,a_1]\dots[a_n,a_n][\varepsilon,\blank]^m \rightarrow^\star_G [a_1,X_1]\cdots[a_{r-1},X_{r-1}]q[a_r,X_r]\cdots[a_{n+m},X_{n+m}],\]
  където $a_1,\dots,a_n \in \Sigma$, $a_{n+1},\dots,a_{n+m} = \varepsilon$, $X_1,\dots,X_{n+m} \in \Gamma$ и
  $X_{l+1} = X_{l+2} = \dots = X_{n+m} = \blank$.
  
  Най-накрая, ако $q \in F$, то можем да използваме правилата от 9) и да докажем, че
  \[[a_1,X_1]\cdots[a_{t-1},X_{t-1}]q[a_t,X_t]\cdots[a_{n+m},X_{n+m}] \rightarrow^\star_G a_1\cdots a_n.\]
  
  Така доказахме, че ако $\alpha \in \L(\M)$, то $\alpha \in \L(G)$, т.е. $\L(\M) \subseteq \L(G)$.
  За да докажем обратната посока, трябва да направим подобни разсъждения.
\end{proof}

\begin{lemma}
  Ако $L = \L(G)$, където $G$ е неограничена граматика, то $L$ е полуразрешим език.
\end{lemma}
\marginpar{Доказателствата в \cite{hopcroft1} и \cite{papadimitriou} са различни}
\begin{proof}
  $\M$ ще бъде недетерминистична машина с три ленти.
  \begin{enumerate}[1)]
  \item
    Записваме входната дума $\omega$ на първата лента на $\M$.
    Тя никога не се променя.
  \item
    На втората лента ще имаме думата $\gamma \in (V\cup\Sigma)^\star$.
    В началото $\gamma := S$.
  \item 
    Недетерминистично избираме правило $\alpha \to \beta$ от граматиката $G$.
  \item
    Недетерминистично избираме $\gamma_0,\gamma_1 \in (V\cup\Sigma)^\star$, за които 
    $\gamma = \gamma_0\alpha\gamma_1$.
    Тогава $\gamma := \gamma_0\beta\gamma_1$.
    Ако няма такива $\gamma_0$ и $\gamma_1$, то $\M$ ,,зацикля'' - текущият опит за извеждане на $\omega$ пропада.
  \item
    Сравняваме съдържанието на първите две ленти, т.е. проверяваме дали $\omega = \gamma$.
    Ако $\omega = \gamma$, то спираме и казваме, че $\M$ разпознава думата $\omega$.
    Ако $\omega \neq \gamma$, то се връщаме на стъпка 3).
  \end{enumerate}

  \begin{algorithm}[H]
  \caption{}
%  \label{alg:}
  \begin{algorithmic}[1]
    \State $\gamma:= S$
    \ForAll{$\alpha\to\beta \in R$}
    \If{$(\exists \gamma_0,\gamma_1\in (V\cup\Sigma)^\star)[\gamma = \gamma_0\alpha\gamma_1]$}
    \State $\gamma := \gamma_0\beta\gamma_1$
    \Else ...
    \EndIf
    \EndFor
  \end{algorithmic}
\end{algorithm}

\end{proof}

\begin{example}
  Граматика за $L = \{a^nb^nc^n \mid n\in\Nat\}$.
\end{example}


\section*{Библиография}

\begin{itemize}
\item 
\item
\item

\end{itemize}

%%% Local Variables: 
%%% mode: latex
%%% TeX-master: "EAI"
%%% End: 


% \include{outro}

\bibliographystyle{amsalpha}
\bibliography{EAI}

\printindex

\end{document}

%%% Local Variables: 
%%% mode: latex
%%% TeX-master: t
%%% End: 
