\section{Линейни автомати}
\marginpar{На англ. linear bounded automaton}
\index{линеен автомат}

{\bf Линеен автомат} е машина на Тюринг, на която не се позволява четящата глава да излиза извън частта от лентата, върху която първоначално е записана входната дума.

\begin{theorem}
  Езикът
  \[L = \{\code{\M}\sharp \omega \mid \M\text{ е линеен автомат и } \omega \in \L(\M)\}\]
  е разрешим.
\end{theorem}
\begin{proof}
  Това е лесно, защото изчислението на $\M$ върху входната дума $\omega$
  може да се намира в една от $|Q|\cdot|\Gamma|^{|\omega|}\cdot |\omega|$ конфигурации.
\end{proof}


\begin{theorem}
  Езикът
  \[L = \{\code{\M} \mid \M\text{ е линеен автомат и } \L(\M) = \emptyset\}\]
  е неразрешим.
\end{theorem}
\begin{proof}
  
\end{proof}


%%% Local Variables:
%%% mode: latex
%%% TeX-master: "../eai"
%%% End:
