\section{Сложност}

\begin{itemize}
\item 
  Детерминистичната машината на Тюринг $\M$ е {\bf полиномиално ограничена}, ако 
  същестува полином $p(x)$, такъв че за всеки вход $\omega$,
  машината $\M$ завършва след най-много $p(|\omega|)$ стъпки.
\item
  Езикът $L$ се нарича {\bf полиномиално разрешим},
  ако съществува полиномиално ограничена тотална детерминистична машина на Тюринг $\M$,
  за която $L = \L(\M)$.
% \item
\item
  Недетерминистичната машината на Тюринг $\M$ е {\bf полиномиално ограничена}, ако 
  същестува полином $p(x)$, такъв че за всеки вход $\omega$,
  съществува изчисление на машината $\M$ върху думата $\omega$,
  което завършва след най-много $p(|\omega|)$ стъпки.
\item
  Езикът $L$ се нарича {\bf недетерминистично полиномиално разрешим},
  ако съществува полиномиално ограничена тотална недетерминистична машина на Тюринг $\M$,
  за която $L = \L(\M)$.
% \item
%   $\mathcal{NP} \df \{L \subseteq \Sigma^\star \mid L\text{ е полиномиално разрешим с НМТ}\}$.
\end{itemize}

\begin{framed}
  \begin{dfn}
    \begin{align*}
      & \mathcal{P} \df \{L \subseteq \Sigma^\star \mid L\text{ е полиномиално разрешим с ДМТ}\};\\
      & \mathcal{EXP} \df \{L \subseteq \Sigma^\star \mid L\text{ е експоненциално разрешим с ДМТ}\};\\
      & \mathcal{NP} \df \{L \subseteq \Sigma^\star \mid L\text{ е полиномиално разрешим с НМТ}\}.
    \end{align*}
  \end{dfn}
\end{framed}

От Теорема ... знаем, че
\[\mathcal{NP} \subseteq \mathcal{EXP}.\]

\begin{prop}
  За азбука $\Sigma$ от поне две букви, можем да обобщим някои от резултатите от предишните глави:
  \[\texttt{REG} \subsetneqq \texttt{CFG} \subsetneqq \mathcal{P}.\]
\end{prop}
\begin{hint}
  Езикът $\{a^nb^nc^n \mid n \in \Nat\} \in \mathcal{P}$,
  но не е безконтекстен.
\end{hint}


%%% Local Variables:
%%% mode: latex
%%% TeX-master: "../eai"
%%% End:
