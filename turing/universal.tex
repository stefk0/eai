\subsection{Универсалният език $L_u$}

Да разгледаме езика $L_{\texttt{univ}} \df \{\code{\M} \cdot \omega \mid \text{$\M$ е М.Т. и }\omega\in \L(\M)\}$.

% \begin{lemma}
%   $L_{\texttt{univ}}$ е полуразрешим език.
% \end{lemma}


\begin{framed}
  \begin{thm}
    Езикът $L_{\texttt{univ}}$ е полуразрешим, но {\bf не} е разрешим.
  \end{thm}
\end{framed}

\begin{hint}
  Първо да съобразим защо $L_{\texttt{univ}}$ е полуразрешим език.
  \begin{itemize}
  \item 
    По дадена дума $\alpha$, ние можем ефективно да проверим
    дали тя има вида $\alpha = \code{\M} \cdot \omega$.
  \item
    Ако $\alpha = \code{\M} \cdot \omega$, 
    то симулираме $\M$ върху $\omega$.
  \end{itemize}
  
  Сега да съобразим защо $L_{\texttt{univ}}$ не е разрешим език.
  Имаме, че за произволна дума $\omega$,
  \begin{align*}
    \omega \in \bar{L}_{\texttt{diag}} & \iff (\exists \M)[\M\text{ е М.Т. и }\omega = \code{\M}\ \&\ \omega \in \L(\M)]\\
                                       & \iff \omega \cdot \omega \in L_{\texttt{univ}}.
  \end{align*}
  Ако допуснем, че $L_{\texttt{univ}}$ е разрешим, то тогава $\bar{L}_{\texttt{diag}}$ е разрешим език, което е противоречие.
\end{hint}

\begin{remark}
  $\bar{L}_{\texttt{univ}} \df \{\code{\M} \cdot \omega \mid \code{\M} \text{ е М.Т. и }\omega\not\in \L(\M)\}$ {\bf не} е полуразрешим език.
\end{remark}



%%% Local Variables:
%%% mode: latex
%%% TeX-master: "../eai"
%%% End:
