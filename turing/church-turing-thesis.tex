

Wittgenstein put this point in a striking way:
Turing’s ‘Machines’. These machines are humans who calculate. (Wittgenstein 1947 [1980]: 1096.)


A man provided with paper, pencil, and rubber, and subject to strict discipline, is in effect a universal machine. (Turing 1948: 416)


Turing machines are supposed to be a precise replacement for the concept of
an effective procedure. Turing took it that anyone who grasped the concept of
an effective procedure and the concept of a Turing machine would have the
intuition that anything that could be done via an effective procedure could be
done by Turing machine. This claim is given support by the fact that all the
other proposed precise replacements for the concept of an effective procedure
turn out to be extensionally equivalent to the concept of a Turing machine—
that is, they can compute exactly the same set of functions. This claim is called
the Church-Turing thesis.

\begin{definition}
  The Church-Turing Thesis states that anything computable via an effective procedure is Turing computable.  
\end{definition}



%%% Local Variables:
%%% mode: latex
%%% TeX-master: "../eai"
%%% End:
