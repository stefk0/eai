\section{Основни свойства}

\begin{proposition}
  Ако езикът $L$ е разрешим, то $\overline{L}$ също е разрешим език.
\end{proposition}
\mynote{Означаваме $\overline{L} = \Sigma^\star \setminus L$.
  С други думи, твърдението ни казва, че разрешимите езици са затворени относно операцията допълнение.
  След малко в \Proposition{diagonal:accept} ще видим, че това твърдение не е изпълнено за полуразрешими езици.}
\begin{hint}
  Нека $L = \L(\M)$, където $\M$ е разрешител.
  Нека $\M'$ е същата като $\M$, само със сменени $\qaccept$ и $\qreject$ състояния.
  Тогава $\M'$ също е разрешител и $\ov{L} = \L(\M')$.
\end{hint}

\begin{proposition}
  Ако езиците $L_1$ и $L_2$ са разрешими , то $L_1 \cup L_2$ е разрешим език.
\end{proposition}
\mynote{С други думи, разрешимите езици са затворени относно операцията обединение.
  Като следствие получаваме, че всяко \emph{крайно} обединение на разрешими езици е разрешим език.
  
  \writedown Съобразете, че това твърдение е изпъленено и за полуразрешими езици.}
\begin{hint}
  Нека $L_1 = \L(\M_1)$ и $L_2 = \L(\M_2)$.
  Строим нова машина на Тюринг $\M$, която при вход думата $\alpha$
  симулира едновременно изчисленията на $\M_1$ и $\M_2$ върху $\alpha$.
  Това можем да направим като приемем, че $\M$ има две ленти - една за лентата на $\M_1$ и една за лентата на $\M_2$,
  като състоянията на $\M$ ще бъдат елементи на $Q_1 \times Q_2$.
  Ако една от двете машини достигне своето приемащо състояние, то $\M$ приема думата $\alpha$.
  Ако и двете машини достигнат своите отхвърлящи състояния, то $\M$ отхвърля думата $\alpha$.
\end{hint}

% \begin{proposition}
%   Ако $L_1$ и $L_2$ са полуразрешими езици, то $L_1 \cup L_2$ е полуразрешим език.
% \end{proposition}

\begin{proposition}
  Ако $L_1$ и $L_2$ са разрешими езици, то $L_1 \cap L_2$ е разрешим език.
  \mynote{С други думи, разрешимите езици са затворени относно операцията сечение.
    Като следствие получаваме, че всяко \emph{крайно} сечение на разрешими езици е разрешим език.

    \writedown Съобразете, че това твърдение е изпъленено и за полуразрешими езици.}
\end{proposition}
\begin{hint}
  Нека $L_1 = \L(\M_1)$ и $L_2 = \L(\M_2)$.
  Строим нова машина на Тюринг $\M$, която при вход думата $\alpha$
  симулира едновременно изчисленията на $\M_1$ и $\M_2$ върху $\alpha$.
  Ако и двете машини достигнат до приемащите си състояния, то $\M$ приема думата $\alpha$.
  Ако поне една от двете машини достигне до отхвърлящо състояние, то $\M$ отхвърля думата $\alpha$.
\end{hint}

% \begin{corollary}
%   Всяко крайно сечение на разрешими езици е разрешим език.
% \end{corollary}

% \begin{important}
%   \begin{theorem}
%     Разрешимите езици са затворени относно операциите обединение, сечение, допълнение.
%   \end{theorem}
% \end{important}


\index{Клини-Пост}
\begin{important}
  \begin{theorem}[Клини-Пост]
    \label{th:turing:kleene-post}
    Езиците $L$ и $\ov{L}$ са полуразрешими точно тогава, когато $L$ е разрешим език.
  \end{theorem}
\end{important}
\mynote{\todo Дефинирайте сами новата машина на Тюринг $\M$.}
\begin{hint}
  Посоката $(\Leftarrow)$ е ясна.
  За посоката $(\Rightarrow)$, нека $L = \L(\M_1)$ и $\ov{L} = \L(\M_2)$.
  Строим разрешител $\M$, която при вход думата $\alpha$ симулира едновременно изчисленията на $\M_1$ и $\M_2$ върху $\alpha$.
  Например, може $\M$ да има две ленти за симулацията на $\M_1$ и $\M_2$.
  Знаем със сигурност, че точно едно от двете симулирани изчисления ще завърши в приемащо състояние.
  Ако това е $\M_1$, то $\M$ приема $\alpha$.
  Ако това е $\M_2$, то $\M$ отхвърля $\alpha$.
\end{hint}


%%% Local Variables:
%%% mode: latex
%%% TeX-master: "../eai"
%%% End:
