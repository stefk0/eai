\chapter{Машини на Тюринг}

\newcommand{\tape}[1]{\dots\blank\blank\blank{#1}\blank\blank\blank\dots}
\newcommand{\qaccept}{q_{\texttt{accept}}}
\newcommand{\qreject}{q_{\texttt{reject}}}

\newcommand{\goleft}{\lhd}
\newcommand{\goright}{\rhd}
\newcommand{\stay}{\Box}

\marginpar{Тук най-вече следваме \cite[Глава 3]{sipser3}}

% \begin{framed}
%   {\bf Теза на Чърч-Тюринг:} Всеки алгоритъм може да се осъществи като машина на Тюринг.
% \end{framed}

\section{Основни понятия}
\index{Тюринг}
\index{машина на Тюринг!детерминистична}

\marginpar{Понятието за машина на Тюринг има много еквивалентни дефиниции}
{\em Детерминистична} машина на Тюринг ще наричаме седморка от вида 
\[\M = \TM,\] където:
\begin{itemize}
\item 
  $Q$ - крайно множество от състояния;
\item
  $\Sigma$ - крайна азбука за входа;
\item
  $\Gamma$ - крайна азбука за лентата, $\Sigma \subseteq \Gamma$;
\item
  % \marginpar{Няма нужда да изискваме главата да остава върху същата клетка от лентата}
  $\delta:Q\times\Gamma \to Q\times \Gamma \times \{\goleft,\goright,\stay\}$ - (частична) функция на преходите;
\item
  $s$ - начално състояние, $s \in Q$;
\item
  $\blank$ - празен символ,  $\blank \in \Gamma \setminus \Sigma$;
\item
  \marginpar{Тези две състояния ще наричаме заключителни}
  $\qaccept \in Q$ - приемащо състояние;
\item
  \marginpar{$\qaccept \neq \qreject$}
  $\qreject \in Q$ - отхвърлящо състояние.
\end{itemize}

Сега ще опишем как $\M$ работи върху вход думата $\alpha \in \Sigma^\star$.
Първоначално, безкрайната лента съдържа само $\alpha$. Останалите клетки на лентата съдържат $\blank$.
Освен това, $\M$ се намира в началното състояние $s$ и главата е върху най-левия символ на $\alpha$.
Работата $\M$ е описана от функцията на преходите.
  
\begin{itemize}
\item 
  % \marginpar{(На англ. instanteneous description)}
  Формално, {\bf моментната конфигурация} (или описание) на едно изчисление на машина на Тюринг
  е тройка от вида 
  \[(\alpha, q, x\beta) \in \Gamma^\star\times Q \times \Gamma^\star,\]
  като интерпретацията на тази тройка е, че машината се намира в състояние $q$ и лентата има вида
  \[\tape{\alpha\underline{x}\beta},\]
  като четящата глава на машината  е поставена върху първия символ на $\beta$.  
  Понякога за удобство ще означаваме моментната конфигурация като $(q,\alpha\underline{x}\beta)$.
\item
  Ако $\beta = \varepsilon$, това означава, че главата на машината е върху $\blank$, т.е.
  лентата има вида  \[\tape{\alpha\underline{\blank}},\]
\item
  {\bf Началната конфигурация} за входа $\alpha \in \Sigma^\star$ представлява 
  \[(\varepsilon, s, \alpha).\]
  Това означава, че лентата има вида \[\cdots\blank\blank\underline{x}\alpha\blank\blank\cdots,\]
  където $\alpha = x\alpha'$, ако $\alpha \in \Sigma^+$.
\item
  Ако $\alpha = \varepsilon$, то началната конфигурация е $(\varepsilon, s, \varepsilon)$ и лентата има вида
  \[\tape{\underline{\blank}}.\]
\item
  {\bf Заключителна конфигурация} представлява тройка от вида
  \[(\alpha, \qaccept, \beta), \text{ или }(\alpha, \qreject, \beta).\]
  Ако машината, която работи върху дадена входа дума, достигне до заключително състояние, ще казваме
  че машината {\em спира работа}.
\end{itemize}

Както за автомати, удобно е да дефинираме бинарна релация $\vdash_\M$ над $\Gamma^\star \times Q \times \Gamma^\star$,
която ще казва как моментната конфигурация на машината $\M$ се променя при изпълнение на една стъпка.
\begin{itemize}
\item
  Ако $\delta_\M(q,z) = (p,y,\goright)$, то дефинираме $(\alpha, q, z\beta) \vdash_\M (\alpha y, p, \beta)$.
  Това означава, че ако лентата е имала вида 
  \[\tape{\alpha\underline{z}\beta},\]
  след тази стъпка лентата има вида 
  \[\tape{\alpha y\underline{b}\beta'},\]
  ако $\beta = b\beta'$, или 
  \[\tape{\alpha y\underline{\blank}},\]
  ако $\beta = \varepsilon$.
\item 
  Ако $\delta_\M(q,z) = (p,y,\goleft)$, то дефинираме $(\alpha x, q, z\beta) \vdash_\M (\alpha , p, xy\beta)$.
  Това означава, че ако лентата е имала вида 
  \[\tape{\alpha x\underline{z}\beta},\]
  след тази стъпка лентата има вида 
  \[\tape{\alpha \underline{x}y\beta}\]
\item
  Ако $\delta_\M(q, z) = (p, y, \stay)$, то дефинираме $(\alpha, q, z\beta) \vdash_\M (\alpha , p, y\beta)$.
  Това означава, че ако лентата е имала вида 
  \[\tape{\alpha \underline{z}\beta},\]
  след тази стъпка лентата има вида 
  \[\tape{\alpha \underline{y}\beta}\]
  
\end{itemize}
С $\vdash^\star_\M$ ще означаваме рефлексивното и транзитивно затваряне на $\vdash_\M$.

\begin{itemize}
\item
  Машината на Тюринг $\M$ {\bf приема} думата $\alpha$, 
  ако 
  \[(\varepsilon, s, \alpha) \vdash^\star_\M (\gamma_1, \qaccept, \gamma_2),\]
  за някои $\gamma_1, \gamma_2 \in \Gamma^\star$.
\item
  Машината на Тюринг $\M$ {\bf отхвърля} думата $\alpha$, 
  ако 
  \[(\varepsilon, s, \alpha) \vdash^\star_\M (\gamma_1, \qreject, \gamma_2),\]
  за някои $\gamma_1, \gamma_2 \in \Gamma^\star$.
\item
  Машината на Тюринг $\M$ {\bf не приема} думата $\alpha$, 
  ако $\M$ отхвърля $\alpha$ или $\M$ никога не завършва при начална конфигурация $(\varepsilon,s,\alpha)$.
\item
  \marginpar{Обърнете внимание, че това не е същото като да изискваме функцията на преходите $\delta$ да бъде тотална.}
  Една машина на Тюринг се нарича {\bf тотална}, ако при всеки вход достига до заключително състояние,
  т.е. достига до $\qaccept$ или $\qreject$.
\item 
  Езикът, който се {\bf разпознава} от машината $\M$ е:
  \[\L(\M) = \{\alpha\in\Sigma^\star \mid (\varepsilon, s, \alpha) \vdash^\star_\M (\beta, \qaccept, \gamma), \text{ за някои }\beta,\gamma\in\Gamma^\star\}.\]
\item
  \index{език!полуразрешим}
  Езикът $L$ се нарича {\bf полуразрешим}, ако съществува машина на Тюринг $\M$, за която
  $L = \L(\M)$.
\item
  \index{език!разрешим}
  Един език $L$ се нарича {\bf разрешим}, ако за него съществува {\em тотална} машина на Тюринг $\M$, за която
  $L = \L(\M)$.
\end{itemize}

\begin{framed}
  \begin{prop}
    Ако $L$ е разрешим език над азбуката $\Sigma$, то $\Sigma^\star \setminus L$ също е разрешим език.
  \end{prop}
\end{framed}

\begin{remark}
  По-късно, ще видим, че съществуват полуразрешими езици, чиито допълнения не са полуразрешими.
\end{remark}

\section{Примери за разрешими езици}

\begin{example}
  \marginpar{Знаем, че $L$ не е безконтекстен}
  Да разгледаме езика $L = \{a^nb^nc^n \mid n\in\Nat\}$.
 
  Нека да въведем нов символ $d$, с който ще маркираме обработените символи $a$, $b$, $c$.
  Идеята на алгоритъма, който ще разгледаме е да маркира на всяка итерация по едно $a$, $b$, и $c$.
  Той завършва успешно ако всички символи на думата са маркирани.
  Нека първоначално думата е копирана върху лентата и четящата глава е върху първия символ на думата.
  \begin{enumerate}[(1)]
  \item 
    Чете $d$-та надясно по лентата докато срещне първото $a$ и го замества с $d$. Отива на стъпка (2).
    Ако символите свършат (т.е. достигне се $\blank$) преди да се достигне $a$,
    то алгоритъмът завършва успешно.
  \item
    Чете $d$-та надясно по лентата докато срещне първото $b$ и го замества с $d$.
    Отива на стъпка (3).
  \item
    Чете $d$-та надясно по лентата докато срещне първото $c$ и го замества с $d$.
  \item
    Връща четящата глава в началото на лентата, т.е. чете наляво докато не срещне символа $\blank$.
    Връща се в стъпка (1). 
  \end{enumerate}

  Нека сега да видим, че този алгоритъм може да се опише съвсем формално с машина на Тюринг.
  Ще построим машина на Тюринг $\M$, за която $L = \L(\M)$, където
  \begin{itemize}
  \item 
    $\Sigma = \{a,b,c\}$;
  \item
    $\Gamma = \{a,b,c,d,\blank\}$, за някой нов символ $d$;
  \item
    $Q = \{1,2,\dots,5\}$;
  \item
    $q_{accept} = 5$;
  \item
    частичната функция на преходите $\delta:Q\times\Gamma \to Q\times\Gamma\times\{L,R\}$
    е описана на схемата отдолу.
  \end{itemize}

  \begin{figure}[H]
    \begin{center}
      \begin{tikzpicture}[->,>=stealth,thick,node distance=50pt]
        \tikzstyle{every state}=[circle,minimum size=10pt,auto]
        
        \node[state,initial]    (1) {$1$};
        \node[state]            (2) [right of=1]{$2$};
        \node[state]            (3) [right of=2]{$3$};
        \node[state]            (4) [right of=3]{$4$};
        \node[state,accepting]  (5) [below of=1]{$5$};
        % \node[state,accepting]  (6) [right of=5]{$6$};
        
        \begin{scope}[every node/.style={scale=.8}]
          \path
          (1) edge [loop above] node [above] {$d;R$} (1)
          (1) edge [bend right=15] node [left] {$\blank;R$} (5)
          % (1) edge [bend left=15] node [left] {$\{b,c\}$} (6)
          (1) edge [bend left=15] node [above] {$a/d;R$} (2)
          (2) edge [bend left=15] node [above] {$b/d;R$} (3)
          (2) edge [loop below] node [right] {$\{a,d\};R$} (2)
          (3) edge [bend left=15] node [above] {$c/d;L$} (4)
          (3) edge [loop below] node [right] {$\{b,d\};R$} (3)
          (4) edge [loop right] node [below right] {$\{a,b,d\};L$} (4)
          (4) edge [in=65,out=115,above] node [above] {$\blank;R$} (1);
        \end{scope}
      \end{tikzpicture}
    \end{center}
  \end{figure}


  % Да проследим изчислението на думата $aabbcc$:
  
  % \[_1aabbcc \vdash d_2abbcc \vdash da_2bbcc \vdash dad_3bcc \vdash dadb_3cc \vdash dad_4bdc \vdash da_4dbdc \vdash \cdots \vdash\]
  % \[_4dadbdc \vdash\ _4\blank dadbdc \vdash\ _1dadbdc \vdash d_1adbdc \vdash dd_2dbdc \vdash ddd_2bdc \vdash dddd_3dc \vdash \]
  % \[ ddddd_3c \vdash dddddd_4 \vdash \cdots \vdash\ _4\blank dddddd \vdash\ _1dddddd \vdash \cdots \vdash dddddd_1\blank \vdash dddddd_5\blank.\]

  Съобразете, че тази машина на Тюринг може да се направи тотална като се добави ново състояние $q_{reject}$
  и за всяка двойка $(q,x)$, за която функцията на преходите не е дефинирана, да сочи към $q_{reject}$.
  Така можем да получим {\em тотална} машина на Тюринг за езика $L$, което означава, че 
  $L$ е не само полуразрешим, но {\em разрешим} език.
\end{example}

\begin{example}
  \marginpar{Да напомним, че този език не е безконтекстен}
  \marginpar{В \cite[стр. 155]{hopcroft1} е дадено по-различно решение. Тук следваме \cite[стр. 173]{sipser3}. Там има малка грешка}
  Да разгледаме езика $L = \{\omega \sharp \omega \mid \omega\in\{a,b\}^\star\}$.
  Нека първо да видим, че можем неформално да опишем алгоритъм, който да разпознава думите на езика $L$.
  Нека една дума е копирана върху лентата и четящата глава е поставена върху първия символ от думата.
  \begin{enumerate}[(1)]
  \item 
    Чете $x$-ове надясно по лентата докато не срещне $a$ или $b$ и го замества с $x$.
    Запомня дали сме срещнали $a$ или $b$.
    Ако вместо $a$ или $b$ срещне $\sharp$, то отива на стъпка $(6)$.
  \item
    Чете $a$-та и $b$-та надясно по лентата докато не стигне $\sharp$. 
  \item
    Чете $c$-то надясно по лентата и всички следващи $x$-ове докато не срещне символа $a$ или $b$.
    Той трябва да е същия символ, който сме запаметили на стъпка $(1)$.
    Заместваме този символ с $x$.
  \item
    Чете $x$-ове наляво по лентата докато не стигне $\sharp$.
  \item
    Чете $a$-та и $b$-та по лентата докато не стигне $x$.
    Поставя четящата глава върху символа точно след първия $x$.
    Отива на стъпка $(1)$.
  \item
    Прочита $\sharp$ надясно по лентата и чете надясно $x$-ове докато не срещне $\blank$.
    Алгоритъмът завършва успешно.
  \end{enumerate}

  Ще построим машина на Тюринг $\M$, за която $L = \L(\M)$.
  \begin{itemize}
  \item 
    $\Sigma = \{a,b,\sharp\}$;
  \item
    $\Gamma = \{a,b,\sharp,x,\blank\}$;
  \item
    $Q = \{1,2,\dots,9\}$;
  \item
    $q_{accept} = 9$;
  \end{itemize}

  \begin{figure}[H]
    \begin{center}
      \begin{tikzpicture}[->,>=stealth,thick,node distance=50pt]
        \tikzstyle{every state}=[circle,minimum size=10pt,auto,scale=.9]
        
        \node[state,initial]    (1) {$1$};
        \node[state]            (2) [above right of=1]{$2$};
        \node[state]            (3) [below right of=1]{$3$};
        \node[state]            (4) [right of=2]{$4$};
        \node[state]            (5) [right of=3]{$5$};
        \node[state]            (6) [below right of=4]{$6$};
        \node[state]            (7) [above of=6]{$7$};
        \node[state]            (8) [left of=3]{$8$};
        \node[state,accepting]  (9) [below left of=3]{$9$};
        
        \begin{scope}[every node/.style={scale=.8}]
          \path
          (1) edge [bend left=15] node [below right] {$a/x;R$} (2)
              edge [bend right=15] node [above right] {$b/x;R$} (3)
              edge [bend right=15] node [left] {$\sharp;R$} (8)
          (2) edge [loop above] node [above] {$\{a,b\};R$} (2)
              edge [bend left=15] node [above] {$\sharp;R$} (4)
          (3) edge [loop below] node [below] {$\{a,b\};R$} (3)
              edge [bend right=15] node [below] {$\sharp;R$} (5)
          (4) edge [loop above] node [above] {$x;R$} (4)
              edge [bend left=15] node [below left] {$a/x;L$} (6)
          (5) edge [loop below] node [below] {$x;R$} (5)
              edge [bend right=15] node [above left] {$b/x;L$} (6)
          (6) edge [loop right] node [right] {$x;L$} (6)
              edge [bend right=15] node [right] {$\sharp;L$} (7)
          (7) edge [loop right] node [right] {$\{a,b\};L$} (7)
              edge [out=130,in=120,above,distance=2.5cm] node [above] {$x;R$} (1)
          (8) edge [loop left] node [left] {$x;R$} (8)
              edge [bend right=15] node [left] {$\blank;R$} (9);
        \end{scope}
      \end{tikzpicture}
    \end{center}
  \end{figure}

  % Да проследим изчислението на думата $ab\sharp ab$.
  
  % \begin{align*}
  %   (\varepsilon, 1, ab\sharp ab) & \to (x, 2, b\sharp ab) \to xb_2\sharp ab \to xb\sharp _4ab \to xb_6\sharp xb \to x_7b\sharp xb \to _7xb\sharp xb \to x_1b\sharp xb\\
  %   & \to xx_3\sharp xb \to xx\sharp _5xb \to xx\sharp x_5b \to xx\sharp _6xx \to xx_6\sharp xx \to x_7x\sharp xx \to xx_1\sharp xx \\
  %   & \to xx\sharp _8xx \to xx\sharp x_8x \to xx\sharp xx_8\blank \to xx\sharp xx\blank_9\blank
  % \end{align*}

  Може лесно да се съобрази, че тази машина на Тюринг може да се допълни до {\em тотална}.
  
\end{example}


%%% Local Variables:
%%% mode: latex
%%% TeX-master: "../eai"
%%% End:


\section{Основни свойства}

\begin{prop}
  Ако $L$ е разрешим език, то $\ov{L}$ е разрешим език.
\end{prop}
\begin{hint}
  Нека $L = \L(\M)$, където $\M$ е тотална машина на Тюринг.
  Нека $\M'$ е същата като $\M$, само със сменени $\qaccept$ и $\qreject$ състояния.
  Тогава $\ov{L} = \L(\M')$.
\end{hint}

\begin{prop}
  Ако $L_1$ и $L_2$ са разрешими езици, то $L_1 \cup L_2$ е разрешим език.
\end{prop}
\begin{hint}
  Нека $L_1 = \L(\M_1)$ и $L_2 = \L(\M_2)$.
  Симулираме двете изчисления едновременно.
  Ако едната машина достигне accept състоянието си, то връщаме accept.
\end{hint}

\begin{prop}
  Ако $L_1$ и $L_2$ са полуразрешими езици, то $L_1 \cup L_2$ е полуразрешим език.
\end{prop}

\begin{prop}
  Ако $L_1$ и $L_2$ са разрешими езици, то $L_1 \cap L_2$ е разрешим език.
\end{prop}
\begin{hint}
  Нека $L_1 = \L(\M_1)$ и $L_2 = \L(\M_2)$.
  Симулираме двете изчисления едновременно.
  Ако и двете машини достигнат accept състоянията си, то връщаме accept.
\end{hint}

\begin{framed}
  \begin{thm}
    $L$ и $\ov{L}$ са полуразрешими езици точно тогава, когато $L$ е разрешим език.
  \end{thm}
\end{framed}
\begin{hint}
  Посоката $(\Rightarrow)$ е ясна.
  За посоката $(\Leftarrow)$, нека $L = \L(\M_1)$ и $\ov{L} = \L(\M_2)$.
  Симулираме едновременно и двете изчисления.
  Знаем със сигурност, че точно едно от тях ще завърши в accept състояние.
  Ако това е $\M_1$, връщаме accept.
  Ако това е $\M_2$, връщаме reject.
\end{hint}

\subsection*{Канонична подредба на $\Sigma^\star$}

\marginpar{Тази канонична подредба е необходима единствено за доказателството, че всяка НМТ е еквивалентна на ДМТ}
Нека $\Sigma = \{a_0,a_1,\dots,a_{k-1}\}$.
Подреждаме думите по ред на тяхната дължина.
Думите с еднаква дължина подреждаме по техния числов ред, т.е.
гледаме на буквите $a_i$ като числото $i$ в $k$-ична бройна система.
Тогава думите с дължина $n$ са числата от $0$ до $k^n-1$ записани в $k$-ична бройна система.
Ще означаваме с $\omega_i$ $i$-тата дума в $\Sigma^\star$ при тази подредба.

\begin{example}
  Ако $\Sigma = \{0,1\}$, то наредбата започва така:
  \[\varepsilon, 0, 1, \underbrace{00, 01, 10, 11}_{\text{от $0$ до $3$}}, \underbrace{000, 001, 010, 011, 100, 101, 110, 111}_{\text{от $0$ до $7$}}, 0000, 0001, \dots\]
  В този случай, $\omega_0 = \varepsilon$, $\omega_7 = 000$, $\omega_{13} = 110$.
\end{example}

\subsection*{Многолентови машини на Тюринг}
\index{машина на Тюринг!многолентова}
%Това е просто като имаш shift.
%Използват се при недет. машини

Машина на Тюринг с $k$ ленти има същата дефиниция като еднолентова машина на Тюринг
с единствената разлика, че
\[\delta: Q \times \Gamma^k\to Q \times \Gamma^k \times \{\goleft,\goright,\stay\}^k.\]
Тук добавяме и възможността главата върху някои от лентите да стои на място.

\begin{itemize}
\item
  $\Sigma \df \{a,b\}$;
\item
  $\Gamma \df \{a,b,\blank\}$
\item
  $\delta:Q\times\Gamma^2 \to Q\times\Gamma^2\times \{\goleft, \goright, \stay\}^2$;
\end{itemize}

\begin{framed}
  \begin{figure}[H]
    \begin{center}
      \begin{tikzpicture}[->,>=stealth,thick,node distance=70pt]
        \tikzstyle{every state}=[circle,minimum size=10pt,auto,scale=.9]
        
        \node[state,initial]    (1) {$q_1$};
        \node[state]            (2) [right of=1]{$q_2$};
        \node[state]            (3) [right of=2]{$q_3$};
        \node[state,accepting]  (4) [right of=3]{$q_4$};
        % \node[state]            (5) [below of=3]{$q_5$};
        
        \begin{scope}% [every node/.style={scale=.8}]
          \path
          (1) edge [loop above] node [above] {$\frac{a}{\blank} / \frac{a}{a};\frac{\goright}{\goright}$} (1)
          (1) edge [loop below] node [below] {$\frac{b}{\blank} / \frac{b}{b};\frac{\goright}{\goright}$} (1)
          (1) edge [bend left=15] node [above] {$\frac{\sharp}{\blank};\frac{\goright}{\stay}$} (2)
          (2) edge [loop above] node [above] {$\frac{a}{\blank};\frac{\goright}{\stay}$} (2)
          (2) edge [loop below] node [below] {$\frac{b}{\blank};\frac{\goright}{\stay}$} (2)
          (2) edge [bend left=15] node [above] {$\frac{\blank}{\blank};\frac{\goleft}{\goleft}$} (3)
          (3) edge [loop above] node [above] {$\frac{a}{a};\frac{\goleft}{\goleft}$} (3)
          (3) edge [loop below] node [below] {$\frac{b}{b};\frac{\goleft}{\goleft}$} (3)
          (3) edge [bend left=15] node [above] {$\frac{\sharp}{\blank};\frac{\stay}{\stay}$} (4);
        \end{scope}
      \end{tikzpicture}
    \end{center}
    \caption{двулентова детерминистична частична машина на Тюринг $\M$, за която $\L(\M) = \{\omega\sharp\omega \mid \omega \in \{a,b\}^\star\}$}
  \end{figure}
\end{framed}

В началото втората лента е празна. Имаме две глави, които се движат независимо една от друга.
\begin{align*}
  (q_1, \frac{\hat{a}}{\hat{\blank}}\frac{b}{\blank}\frac{\sharp}{\blank}\frac{a}{\blank}\frac{b}{\blank}) & \vdash (q_1, \frac{a}{a}\frac{\hat{b}}{\hat{\blank}}\frac{\sharp}{\blank}\frac{a}{\blank}\frac{b}{\blank}) \vdash (q_1, \frac{a}{a}\frac{b}{b}\frac{\hat{\sharp}}{\hat{\blank}}\frac{a}{\blank}\frac{b}{\blank}) \vdash (q_2, \frac{a}{a}\frac{b}{b}\frac{\sharp}{\hat{\blank}}\frac{\hat{a}}{\blank}\frac{b}{\blank})\\
                                                                                                           & \vdash (q_2, \frac{a}{a}\frac{b}{b}\frac{\sharp}{\hat{\blank}}\frac{\hat{a}}{\blank}\frac{b}{\blank}) \vdash (q_2, \frac{a}{a}\frac{b}{b}\frac{\sharp}{\hat{\blank}}\frac{a}{\blank}\frac{\hat{b}}{\blank}) \vdash (q_2, \frac{a}{a}\frac{b}{b}\frac{\sharp}{\hat{\blank}}\frac{a}{\blank}\frac{b}{\blank}\frac{\hat{\blank}}{\blank})\\
                                                                                                           & \vdash (q_3, \frac{a}{a}\frac{b}{\hat{b}}\frac{\sharp}{\blank}\frac{a}{\blank}\frac{\hat{b}}{\blank}) \vdash (q_3, \frac{a}{\hat{a}}\frac{b}{b}\frac{\sharp}{\blank}\frac{\hat{a}}{\blank}\frac{b}{\blank}) \vdash (q_3, \frac{\blank}{\hat{\blank}}\frac{a}{a}\frac{b}{b}\frac{\hat{\sharp}}{\blank}\frac{a}{\blank}\frac{b}{\blank})\\
                                                                                                           & \vdash (q_4, \frac{\blank}{\hat{\blank}}\frac{a}{a}\frac{b}{b}\frac{\hat{\sharp}}{\blank}\frac{a}{\blank}\frac{b}{\blank}).
\end{align*}

\begin{prop}
  За всяка $k$-лентова машина на Тюринг $\M$ съществува еднолентова машина на Тюринг $\M'$,
  такава че $\L(\M) = \L(\M')$.
\end{prop}
\begin{proof}
  \marginpar{В \cite[стр. 177]{sipser3} конструкцията е малко по-различна. Там съдържанието на всяка лента се поставя последователно върху една лента, като се разделят със специален символ. Тук следваме \cite[стр. 162]{hopcroft1}}
  Нека $\M$ е $k$-лентова машина на Тюринг.
  Ще построим еднолентова машина на Тюринг $\M'$, за която $\L(\M) = \L(\M')$.
  Да означим $\hat\Gamma = \{\hat X \mid X \in \Gamma\}$.
  Тогава азбуката на лентата на $\M'$ ще бъде $\Gamma' = (\hat\Gamma \cup \Gamma)^{k}$.
  Сега вместо да имаме $k$ ленти ще имаме една лента, която представлява $k$-орка.
  За да симулираме $\M$, използваме символите $\hat X$ за да маркират позицията на главите на $\M$,
  като във всяка компонента на лентата има точно по един символ от вида $\hat X$.
  % С $\$$ ще отблезяваме границите на всяка лента, в която можем да търсим маркера.
  За да определим следващия ход на машината $\M'$, трябва да сканираме лентата докато не 
  открием разположението на всичките $k$ на брой маркирани клетки. Тогава симулираме ход на $\M$
  и отново трябва да променим маркираните клетки.
\end{proof}

{\bf Да се обясни, че има квадратично забавяне при преход от многолентова машина към еднолентова.}


%%% Local Variables:
%%% mode: latex
%%% TeX-master: "../eai"
%%% End:

\section{Недетерминистични машини на Тюринг}
\index{машина на Тюринг!недетерминистична}

Една машина на Тюринг $\N$ се нарича недетерминистична, ако функцията на преходите има вида
\[\Delta: Q'\times \Gamma \to \Ps(Q \times \Gamma\times \{\goleft,\goright,\stay\}), \]
където да напомним, че $Q' = Q \setminus \{\qaccept,\qreject\}$.

Отново можем да дефинираме бинарна релация $\vdash$ над $\Gamma^\star \times Q \times \Gamma^+$,
която ще казва как моментното описание на машината $\N$ се променя при изпълнение на една стъпка.

\begin{figure}[H]
  \begin{subfigure}[b]{0.5\textwidth}
    \begin{prooftree}
      \AxiomC{$\Delta(q,x) \ni (p,y,\goright)$}
      \RightLabel{\scriptsize{(right-1)}}
      \UnaryInfC{$(\alpha, q, xz\beta) \vdash (\alpha y, p, z\beta)$}
    \end{prooftree}
    \vspace*{2mm}
  \end{subfigure}
  ~
\begin{subfigure}[b]{0.5\textwidth}
\begin{prooftree}
  \AxiomC{$\Delta(q,x) \ni (p,y,\goleft)$}
  \RightLabel{\scriptsize{(left-1)}}
  \UnaryInfC{$(\alpha z, q, x\beta) \vdash (\alpha,p,zy\beta)$}
\end{prooftree}
\vspace*{2mm}
\end{subfigure}

\begin{subfigure}[b]{0.5\textwidth}
\begin{prooftree}
  \AxiomC{$\Delta(q,x) \ni (p,y,\goright)$}
  \RightLabel{\scriptsize{(right-2)}}
  \UnaryInfC{$(\alpha, q, x) \vdash (\alpha y, p, \blank)$}
\end{prooftree}
\vspace*{2mm}
\end{subfigure}
~
\begin{subfigure}[b]{0.5\textwidth}
  \begin{prooftree}
    \AxiomC{$\Delta(q,x) \ni (p,y,\goleft)$}
    \RightLabel{\scriptsize{(left-2)}}
    \UnaryInfC{$(\varepsilon,q,x\beta) \vdash (\varepsilon,p,\blank y\beta)$}
  \end{prooftree}
  \vspace*{2mm}
\end{subfigure}

\begin{prooftree}
  \AxiomC{$\Delta(q, z) \ni (p, y, \stay)$}
  \RightLabel{\scriptsize{(stay)}}
  \UnaryInfC{$(\alpha, q, z\beta) \vdash (\alpha , p, y\beta)$}
\end{prooftree}
\end{figure}

С $\vdash^\star$ ще означаваме рефлексивното и транзитивно затваряне на $\vdash$.
Тогава за недетерминистична машина на Тюринг $\N$, 
\[\L(\N) = \{\ \omega\in\Sigma^\star \mid (\exists \gamma_1,\gamma_2 \in \Gamma^\star)[(\varepsilon, \qstart, \omega\blank) \vdash^\star_\N (\gamma_1, \qaccept, \gamma_2) ]\ \}.\]

\begin{remark}
  Върху дадена дума $\omega$, недетерминистичната машина на Тюринг $\N$ може да има много различни изчисления.
  Думата $\omega$ принадлежи на $\L(\N)$ ако съществува {\em поне едно} изчисление, което завършва в състоянието $\qaccept$.
  Възможно е много други изчисления при вход $\omega$ да завършват в $\qreject$ или никога да не завършват.
\end{remark}

Аналогично, дефинираме една недетерминистична машина на Тюринг $\N$ да бъде {\bf разрешител}, ако за всяка дума $\omega$ и 
\emph{всяко изчисление} на $\N$ върху $\omega$ завършва в $\qaccept$ или $\qreject$.

% \begin{problem}
%   \mynote{\cite{hopcroft2}}
%   Нека
%   \[\N = (\{q_0,q_1,q_2,q_f\}, \{0,1\}, \{0,1,\blank\}, \Delta, q_0, \{q_f\}),\]
%   \begin{itemize}
%   \item 
%     $\Delta(q_0,0) = \{(q_0,1,\goright),(q_1,1,\goright)\}$;
%   \item
%     $\Delta(q_1,1) = \{(q_2,0,\goleft)\}$;
%   \item
%     $\Delta(q_2,1) = \{(q_0,1,\goright)\}$;
%   \item
%     $\Delta(q_1,\blank) = \{(q_f,\blank,\goright)\}$.
%   \end{itemize}
%   \mynote{$\{0^{n+1}1^k \mid n,k\in\Nat\}$}
%   Опишете $\L(\N)$.
% \end{problem}

\begin{extra}
\begin{example}
  \mynote{Не е обяснено защо е разрешим.}
  Нека да видим, че $L = \{\alpha\sharp\beta \mid \alpha,\beta \in \{a,b\}^\star\ \&\ \alpha\text{ е подниз на }\beta\}$
  е разрешим език като построим недетерминистична машина на Тюринг $\N$,
  която разрешава този език.
  \begin{framed}
    \begin{figure}[H]
      \begin{center}
        \begin{tikzpicture}[->,>=stealth,thick,node distance=65pt]
          \tikzstyle{every state}=[circle,minimum size=10pt,scale=.9]
          
          \node[state,initial below]    (1) {$q_0$};
          \node[state]            (2) [right of=1]{$q_1$};
          \node[state]            (3) [right of=2,node distance=80pt]{$q_2$};
          \node[state]            (4) [below of=3]{$q_3$};
          \node[state]            (5) [below right of=4,node distance=80pt]{$q_4$};
          \node[state]            (6) [right of=4]{$q_5$};
          \node[state]            (7) [above of=6]{$q_6$};
          \node[state]            (8) [right of=6,node distance=90pt]{$q_7$};
          \node[state]            (9) [right of=7,node distance=90pt]{$q_8$};
          \node[state,accepting]  (10)[below right of=5]{$q_{9}$};
          
          \begin{scope}[every node/.style={scale=.8}]
            \path
            (1) edge [loop above] node [above] {$\{a,b\};\goright$} (1)
            (1) edge [bend left=15] node [above] {$\sharp;\goright$} (2)
            (2) edge [loop above] node [above] {$a/\blank,b/\blank;\goright$} (2)
            (2) edge [bend left=15] node [above] {$\{a,b,\blank\};\goleft$} (3)
            (3) edge [loop above] node [above] {$\blank;\goleft$} (3)
            (3) edge [bend right=15] node [left] {$\sharp;\goleft$} (4)
            (4) edge [loop left] node [left] {$\{a,b\};\goleft$} (4)
            (4) edge [bend right=30] node [left] {$\blank;\goright$} (5)
            (5) edge [bend right=15] node [right] {$a/\blank;\goright$} (6)
            (6) edge [loop right] node [right] {$\{a,b\};\goright$} (6)
            (6) edge [bend right=15] node [right] {$\sharp;\goright$} (7)
            (7) edge [loop right] node [right] {$\blank;\goright$} (7)
            (7) edge [bend left=15] node [below] {$a/\blank;\goleft$} (3)
            (8) edge [loop right] node [right] {$\{a,b\};\goright$} (8)
            (5) edge [bend right=30] node [right] {$b/\blank;\goright$} (8)
            (8) edge [bend right=15] node [right] {$\sharp;\goright$} (9)
            (9) edge [loop right] node [above] {$\blank;\goright$} (9)
            (9) edge [bend right=45] node [above] {$b/\blank;\goleft$} (3)
            (5) edge [bend right=15] node [left] {$\sharp;\stay$} (10);
          \end{scope}
        \end{tikzpicture}
      \end{center}
    \end{figure}
  \end{framed}
  Да видим, че $\M$ успешно разпознава думата $ab\sharp aabb$, която принадлежи на $L$.
  \begin{align*}
    (q_0, \underline{a}b\sharp aabb\blank) & \vdash (q_0, a\underline{b}\sharp aabb\blank) \vdash (q_0, ab\underline{\sharp} aabb\blank) \vdash (q_1, ab\sharp\underline{a}abb\blank) \\
                                           & \vdash (q_1, ab\sharp\blank\underline{a}bb\blank) \vdash (q_2, ab\sharp\underline{\blank}abb\blank) \vdash (q_2, ab\underline{\sharp}\blank abb\blank)\\
                                           & \vdash (q_3, a\underline{b}\sharp\blank abb\blank) \vdash (q_3, \underline{a}b\sharp\blank abb\blank) \vdash (q_3, \underline{\blank}ab\sharp\blank abb\blank)\\
                                           & \vdash (q_4, \underline{a}b\sharp\blank abb\blank) \vdash (q_5, \blank\underline{b}\sharp \blank abb\blank) \vdash (q_5, \blank b\underline{\sharp} \blank abb\blank)\\
                                           & \vdash (q_6, \blank b \sharp \underline{\blank} abb\blank) \vdash (q_6, \blank b \sharp \blank \underline{a}bb\blank) \vdash (q_2, \blank b \sharp \underline{\blank}\blank bb\blank)\\
                                           & \vdash (q_2, \blank b \underline{\sharp} \blank\blank bb\blank) \vdash (q_3, \blank \underline{b} \sharp \blank\blank bb\blank) \vdash (q_3, \underline{\blank} b \sharp \blank\blank bb\blank)\\
                                           & \vdash (q_4,  \blank \underline{b} \sharp \blank\blank bb\blank) \vdash (q_7, \blank \blank \underline{\sharp} \blank\blank bb\blank) \\
                                           & \vdash (q_8, \blank \blank \sharp \underline{\blank}\blank bb\blank) \vdash (q_8, \blank \blank \sharp \blank \underline{\blank} bb\blank) \\
                                           & \vdash (q_8, \blank \blank \sharp \blank \blank \underline{b}b\blank) \vdash (q_2, \blank \blank \sharp \blank \underline{\blank} \blank b\blank)\\
                                           & \vdash \cdots \vdash (q_4, \blank\blank\underline{\sharp}\blank\blank\blank b\blank) \vdash (q_9, \blank\blank\underline{\sharp}\blank\blank\blank b\blank)
  \end{align*}
\end{example}
\end{extra}

\subsection*{Канонична наредба на $\Sigma^\star$}
\index{наредба!канонична}

\mynote{За доказателството, че всяка НМТ е еквивалентна на ДМТ, е необходимо да фиксираме канонична подредба на думите над дадена азбука}
Нека $\Sigma = \{a_0,a_1,\dots,a_{k-1}\}$.
Подреждаме думите по ред на тяхната дължина.
Думите с еднаква дължина подреждаме по техния числов ред, т.е.
гледаме на буквите $a_i$ като числото $i$ в $k$-ична бройна система.
Тогава думите с дължина $n$ са числата от $0$ до $k^n-1$ записани в $k$-ична бройна система.
Ще означаваме с $\omega_i$ $i$-тата дума в $\Sigma^\star$ при тази подредба.

Ако $\Sigma = \{0,1\}$, то наредбата започва така:
\[\varepsilon, 0, 1, \underbrace{00, 01, 10, 11}_{\text{от $0$ до $3$}}, \underbrace{000, 001, 010, 011, 100, 101, 110, 111}_{\text{от $0$ до $7$}}, 0000, 0001, \dots\]
В този случай, $\omega_0 = \varepsilon$, $\omega_7 = 000$, $\omega_{13} = 110$.
Обърнете внимание, че тази наредба отговаря на обхождане в широчина на едно пълно наредено двоично дърво.
Можем да дефинираме и релацията $<_{\text{can}}$ по следния начин:
\[\alpha <_{\text{can}} \beta \dff |\alpha| < |\beta| \lor ( |\alpha| = |\beta|\ \&\ \alpha <_{\text{lex}} \beta). \]

\begin{extra}
\begin{problem}
  \label{prob:canonical:function}
  Нека $\Sigma = \{a_0,\dots,a_{k-1}\}$.
  Да разгледаме функцията $f:\Sigma^\star \to \Sigma^\star$, за която 
  $f(\alpha)$ е думата веднага след $\alpha$ в каноничната подредба на $\Sigma^\star$.
  Докажете, че $f$ е изчислима с еднолетнова детерминистична машина на Тюринг.
\end{problem}
\begin{hint}
  Ако $\Sigma = \{0,1\}$, то машината на Тюринг има следния вид:
  \begin{framed}
    \begin{figure}[H]
      \begin{center}
        \begin{tikzpicture}[->,>=stealth,thick,node distance=70pt]
          \tikzstyle{every state}=[circle,minimum size=10pt,auto,scale=.9]
          
          \node[state,initial]    (1) [right of=0]{$q_1$};
          \node[state]            (2) [right of=1]{$q_2$};
          \node[state]            (3) [right of=2]{$q_3$};
          \node[state,accepting]  (4) [right of=3]{$q_4$};
          
          \begin{scope}[every node/.style={scale=.8}]
            \path
            (1) edge [loop above] node [above] {$\{0,1\};\goright$} (1)
            (1) edge [bend left=15] node [above] {$\blank;\goleft$} (2)
            (2) edge [loop above] node [above] {$1/0;\goleft$} (2)
            (2) edge [bend left=15] node [above] {$0/1;\goleft$} (3)
            (2) edge [bend right=30] node [below] {$\blank/0;\stay$} (4)
            (3) edge [loop above] node [above] {$\{0,1\};\goleft$} (3)
            (3) edge [bend left=15] node [above] {$\blank;\goright$} (4);
          \end{scope}
        \end{tikzpicture}
        \caption{Генериране на следващата дума в каноничната наредба.}
      \end{center}
    \end{figure}
  \end{framed}
\end{hint}
\end{extra}

\begin{important}
  \begin{theorem}
    Ако $L$ се разпознава от {\em недетерминистична} машина на Тюринг $\N$, то $L$
    е разпознава и от {\em детерминистична} машина на Тюринг $\D$.
  \end{theorem}
\end{important}
\begin{proof}
  \mynote{В \cite[стр. 164]{hopcroft1} не е добре обяснено.}
  Нека имаме недетерминистичната машина на Тюринг $\N$, за която $L = \L(\N)$.
  Една дума $\alpha$ принадлежи на $\L(\N)$ точно тогава, когато съществува изчисление,
  което започва с думата $\alpha$ върху лентата и след краен брой стъпки, следвайки функцията на преходите $\Delta_\N$,
  достига до състоянието $\qaccept$.
  Сложността идва от факта, че за думата $\alpha$ може да имаме много различни изчисления, 
  като само някои от тях завършват в $\qaccept$. Ще построим детерминистична машина на Тюринг $\D$,
  която последователно ще симулира всички възможни {\em крайни} изчисления за думата $\alpha$, докато 
  намери такова, което завършва в състоянието $\qaccept$.
  \mynote{На практика това, което правим е да представим всички възможни изчисления на $\N$ като $r$-разклонено дърво и да го обходим в широчина, докато не достигнем до $\qaccept$}
  
  Лесно се съобразява, че всяко изчисление на $\N$ може да се представи като 
  крайна редица от елементи на $Q \times \Gamma \times \{\goleft,\goright,\stay\}$.
  Понеже това множество е крайно, то можем на всяка такава тройка да
  съпоставим естествено число $ < r$, където 
  \[r = |Q| \cdot |\Gamma| \cdot 3.\]
  Например, нека $Q = \{q_0,q_1\}$, $\Gamma = \{a,b\}$. Тогава можем да направим следната съпоставка:
  \begin{align*}
    & (q_0,a,\stay) \to 0,\ (q_0,a,\goleft) \to 1,\ (q_0,a,\goright) \to 2,\\
    & (q_0,b,\stay) \to 3,\ (q_0,b,\goleft) \to 4, \ (q_0,b,\goright) \to 5,\\
    & (q_1,a,\stay) \to 6, (q_1,a,\goleft) \to 7, (q_1,a,\goright) \to 8,\\
    & (q_1,b,\stay) \to 9, (q_1,b,\goleft) \to 10,\ (q_1,b,\goright) \to 11.
  \end{align*}
  Ясно е, че всяко изчисление на $\N$ може да се представи като дума над азбуката $\Sigma = \{x_0,x_1,\dots,x_{r-1}\}$.
  Например, изчислението от три стъпки
  \[(\blank,q_0,aba) \vdash_N (b,q_1,ba) \vdash_\N (b,q_1,aa) \vdash_\N (ba,q_0,a)\]
  може да се опише като думата $x_{11}x_6x_2$ над азбуката $\Sigma = \{x_0,x_1,\dots,x_{11}\}$.
  
  Детерминистичната машина на Тюринг $\D$ има три ленти.
  \begin{itemize}
  \item 
    На първата лента съхраняваме входящия низ и {\em тя никога не се променя}.
  \item
    На втората лента ще записваме последователно думи следвайки каноничната наредба на 
    думите над азбуката $\{x_0,x_1,\dots,x_{r-1}\}$.
    От \Problem{canonical:function} знаем как последователно да генерираме тези думи върху една лента.
  \item
    На третата лента симулираме изчислението на $\N$ върху думата от първата лента, използвайки изчислението, 
    което е описано на втората лента. Например, ако съдържанието на втората лента е $x_{11}x_6x_2$,
    това означава, че симулираме изчисление от три стъпки като на първата стъпка избираме дванайсетата
    възможна тройка, на втората стъпка избираме седмата възможна тройка, на третата стъпка избираме третата възможна тройка.
    
    Ако симулацията завърши в състоянието $\qaccept$ на $\N$, то машината $\D$ завършва успешно.
    В противен случай, на втората лента генерираме чрез функцията от \Problem{canonical:function} следващия низ относно каноничната наредба на $\{x_0,x_1\dots,x_{r-1}\}$;
    изтриваме третата лента, копираме първата лента на третата и започваме нова детерминистична симулация като думата върху втората лента ни ръководи какъв преход да правим на всяка стъпка.
  \end{itemize}
\end{proof}


\begin{corollary}
  Ако $L$ се разпознава от {\em недетерминистичен} разрешител $\N$, то $L$
  също се разпознава от {\em детерминистичен} разрешител $\D$.
\end{corollary}
\begin{proof}
  Да разгледаме дървото $T$ с крайно разклонение $r$, което представя всички изчисления на разрешителя $\N$ при вход думата $\omega$.
  От \Lemma{konig} следва, че $T$ е крайно дърво, да кажем с височина $h$, защото ако допуснем, че $T$ е безкрайно, то ще има безкрайно дълго изчисление на $\N$,
  което е невъзможно, понеже $\N$ винаги достига до заключително състояние ($\qaccept$ или $\qreject$).
  \begin{itemize}
  \item 
    Ако $\N$ приема дадена дума $\omega$, то детерминистичната ни симулация на $\N$ ще достигне до изчисление, кодирано като път в $T$, 
    което завършва в състояние $\qaccept$.
  \item
    Ако $\N$ не приема дадена дума $\omega$, то детерминистичната ни симулация на $\N$ ще покаже, че всяко изчисление, кодирано като път в $T$, завършва в състояние $\qreject$.
    Един начин да направим това е да имаме една допълнителна лента, която използваме за брояч колко от възможните изчисления на $\N$ са завършили.
    Спираме, когато този брояч достигне $r^h$, където $h$ е дължината на думата на втората лента, т.е. дълбочината на дървото на изчисленията на $\N$.
  \end{itemize}
\end{proof}


%%% Local Variables:
%%% mode: latex
%%% TeX-master: "../eai"
%%% End:


% \section{Машини на Тюринг като генератори}

% \marginpar{\cite[стр. 168]{hopcroft1}}
% \marginpar{\cite[стр. 180]{sipser3}}
% \marginpar{На англ. се наричат {\em enumerators}}

% Нека да разгледаме един вариант на многолентовите машини на Тюринг, които ще наричаме {\bf генератори}.
% Нека машината на Тюринг да има две ленти, като в началото и двете ленти са празни.
% \begin{itemize}
% \item 
%   Първата лента ще служи за работна лента - върху нея можем да пишем и четем;
% \item
%   Втората лента служи единствено за изход - върху нея можем само да пишем пишем думи; не можем да четем какво вече сме написали върху нея и не можем да пишем върху вече записана клетка. Думите са разделени със специален символ - $\#$.
%   Това означава, че втората лента има вида
%   \[\omega_1\#\omega_2\#\cdots\#\omega_n\#\blank\blank\cdots\]
% \item
%   Езикът, които се извежда от такъв генератор е съставен от думите, които са изписани на изходната лента.
%   Такива езици ще наричаме {\bf изчислимо изброими}.
%   Обърнете внимание, че измежду думите на изходната лента е възможно да има повторения.
%   Ако езикът е безкраен, то машината ще работи безкрайно много време.
% \end{itemize}

% \begin{framed}
%   \begin{thm}
%     Един език $L$ е полуразрешим точно тогава, когато $L$ е изчислимо номеруем.
%   \end{thm}
% \end{framed}
% \begin{proof}
%   $(\Leftarrow)$ Нека $L$ да се номерира от генераторът $E$.
%   Машината на Тюринг $\M$, за която $L = \L(\M)$ ще работи по следния начин:
%   \begin{enumerate}[1)]
%   \item 
%     При вход думата $\omega$, $\M$ започва да симулира $E$;
%   \item
%     Когато се появи дума $\gamma$ върху изходната лента на $E$, сравняваме $\omega$ с $\gamma$;
%   \item
%     Ако $\omega = \gamma$, то отиваме в състоянието $q_{accept}$ на $\M$ и завършваме;
%   \item
%     В противен случай, отиваме обратно на стъпка $2)$.
%   \end{enumerate}

%   $(\Rightarrow)$ Нека сега $L = \L(\M)$. Целта ни е да изведем всички думи на $L$ върху изходната лента.
%   Основният проблем е, че за дадена дума $\omega$, не знаем за колко стъпки трябва да симулираме $\M$ за да сме сигурни дали думата $\omega \in \L(\M)$ или не. Оказва се, че можем да разрешим този проблем като позволяме да извеждаме повторящи се думи.
%   За целта, да подредим всички думи $\omega_1, \omega_2, \dots $ над азбуката $\Sigma$ спрямо каноничната наредба.
%   \begin{enumerate}[1)]
%   \item
%     Нека $s = 1$;
%   \item 
%     Симулираме $\M$ върху думите $\omega_1,\dots,\omega_s$ за $s$ стъпки;
%   \item
%     За всяка от тези думи $\omega_i$, които се приемат от $\M$, записваме ги върху изходната лента.
%   \item
%     Нека $s = s+1$; Отиваме обратно на стъпка $2)$.
%   \end{enumerate}
% \end{proof}

% \begin{remark}
%   В последната конструкция позволяваме думите на един полуразрешим език $L$ да се 
%   извеждат върху изходната лента многократно. Можем лесно да осигурим условието всяка дума на $L$
%   да се извежда точно по веднъж.
%   На стъпка $s = \pair{i,j}$, то проверяваме дали думата $\omega_i$ се приема успешно от $\M$
%   за {\em точно} $j$ на брой стъпки. Само тогава думата се записва на изходната лента.
  
%   Обърнете внимание, че не можем да осигурим условието думите да се извеждат във възходящ ред
%   относно каноничната наредба.
% \end{remark}

% \begin{framed}
%   \begin{thm}
%     Един език $L$ е разрешим точно тогава, когато съществува генератор за $L$, 
%     който изписва думите на $L$ във възходящ ред относно каноничната наредба.
%   \end{thm}
% \end{framed}
% \begin{proof}
%   $(\Rightarrow)$ Нека $L = \L(\M)$. Тази посока е лесна, защото $\M$ е тотална машина,
%   т.е. за всеки вход $\M$ завършва или в $q_{accept}$ или в $q_{reject}$.
%   \begin{enumerate}[1)]
%   \item 
%     Нека $s = 1$;
%   \item
%     Симулираме $\M$ върху думата $\omega_s$.
%   \item
%     Ако симулацията завърши в състояние $q_{accept}$, то записваме $\omega_s$
%     върху изходната лента. 
%   \item
%     Иначе ако симулацията завърши в състояние $q_{reject}$, то нищо не записваме върху изходната лента. 
%   \item
%     Нека $s = s+1$. Отиваме на стъпка $2)$.
%   \end{enumerate}

%   \marginpar{Ако имам генератор $G$ за $L$ няма алгоритъм, който да ми каже дали $L$ е безкраен език или не. Това означава, че по код на $G$ няма как ефективно да получа код на $\M$}
%   $(\Leftarrow)$ Ако $L$ е краен, то е ясно, че мога да разпозная езика с краен автомат, което е частен случай на тотална машина на Тюринг.
%   По-интересният случай е когато $L$ е безкраен език.
%   Нека $L$ се генерира от машината на Тюринг $G$ като извежда думите на $L$ във възходящ ред.
%   \begin{itemize}
%   \item 
%     Вход дума $\omega$;
%   \item
%     Симулираме $G$ като гледаме думите, които се извеждат на изходната лента.
%     Ако срещнем думата $\omega$, то завършваме в състояние $q_{accept}$.
%   \item
%     Ако срещнем думата $\gamma$, която е по-голяма от $\omega$ относно каноничната наредба, 
%     то завършваме в състояние $q_{reject}$.
%   \end{itemize}
% \end{proof}




% \subsection*{Тезис на Чърч-Тюринг}

% \section{Универсална машина на Тюринг}
% За простота, нека $\Sigma = \{0,1\}$ и $\Gamma = \{0,1,\blank\}$.
% \begin{itemize}
% \item 
%   $X_1 = 0$, $X_2 = 1$, $X_3 = \blank$;
% \item
%   $D_1 = L$, $D_2 = R$
% \end{itemize}


\subsection{Кодиране на краен автомат}

Да разгледаме крайния автомат $\A = \NFA$, където 
\begin{itemize}
\item 
  $Q = \{q_1,\dots,q_n\}$,
\item
  $F = \{q_{k_1},\dots,q_{k_l}\}$,
\item
  $\Sigma = \{a_1,\dots,a_m\}$.
\end{itemize}
Да разгледаме прехода $\Delta(q_i,a_j) \ni q_k$.
Кодираме този преход по следния начин:
\[0^i10^j10^k.\]

\[\code{\Delta} \df \texttt{code}_1\ 11\ \texttt{code}_2\ 11 \cdots 11\ \texttt{code}_r\]

Да обърнем внимание, че в този двоичен код няма последователни единици и той 
започва и завършва с нула.

Да означим 
\[\code{F} \df 0^{k_1}10^{k_2}1\cdots 0^{k_l}.\]

Тогава кодът на $\A$ е думата
\[\code{\A} \df 111\ \code{F}\ 11\ \code{\Delta}\ 111.\]


\subsection{Кодиране на машина на Тюринг}

\subsection*{Кодиране на преход}
Да разгледаме прехода $\delta(q_i,X_j) = (q_k,X_l,D_m)$.
Кодираме този преход по следния начин:
\[0^i10^j10^k10^l10^m\]
Да обърнем внимание, че в този двоичен код няма последователни единици и той 
започва и завършва с нула.


За да кодираме една машина на Тюринг $\M$ е достатъчно да кодираме функцията на преходите $\delta$.
Понеже $\delta$ е крайна функция, нека с числото $r$ да означим броя на всички възможни преходи.
По описания по-горе начин, нека $code_i$ е числото в двоичен запис, получено за $i$-тия преход на $\delta$.
Тогава кодът на $\M$ е следното число в двоичен запис:
\[\code{\M} \df 111\ \texttt{code}_1\ 11\ \texttt{code}_2\ 11\ \cdots\ 11\ \texttt{code}_r\ 111.\]
\begin{itemize}
\item
  Лесно се съобразява, че за две МТ $\M$ и $\M'$ с различни функции на преходите, имаме $\code{\M} \neq \code{\M'}$.
% \item
%   Ще казваме, че числото $r\in\Nat$ е {\bf код на} $\M$, ако $r$, записано в двоичен запис представлява думата $\code{\M}$.
%   Оттук нататък, когато пишем $\M_r$, ще имаме предвид машината на Тюринг с код $r$.
% \item
%   Ясно е, че не всяко естествено число е код на машина на Тюринг, но по дадено число $n$
%   има ефективна процедура, която ни казва дали $n$ е код на машина на Тюринг или не.
% \item
  % С $\pair{\M,\omega}$ ще означаваме кода на $\M$ при вход $\omega$ е числото с двоичен запис описанието на $\M$ и след това прикрепена думата $\omega$.
  % При едно число $r = \pair{M,\omega}$, лесно се намира кода на $\M$.
  % Просто започваме да четем двоичния запис на $r$ докато не срещнем за втори път $111$.
  % След това започва думата $\omega$.
% \item
%   Да въведем означението $\M_i$ за произволно ествестено число $i$.
%   Ако $i$ е код на машина на Тюринг $\M$, то $\M_i \df \M$.
%   Ако $i$ не е код на машина на Тюринг, то $\M_i$ е машина на Тюринг с празна функция на преходите.
\end{itemize}

\begin{example}
  Да се даде пример за кода на конкретна машина на Тюринг.
\end{example}


\begin{prop}
  Следните езици са разрешими:
  \begin{itemize}
  \item 
    $L = \{\code{\M} \mid \M\text{ е машина на Тюринг}\}$;
  \item
    $L = \{\code{\M} \mid \M\text{ е детерминистична машина на Тюринг}\}$.
  \end{itemize}
\end{prop}

\begin{remark}
  Следният език {\bf не} е разрешим:
  \[L_{\texttt{tot}} = \{\code{\M} \mid \M\text{ е тотална машина на Тюринг}\}.\]
\end{remark}


% \section{Изчислими функции}

% Нека е дадена функцията $f:\Nat^k \to \Nat$.
% Ще казваме, че $f$ е изчислима с машината на Тюринг $\M$,
% ако за всяко $n_1,\dots,n_k$ е изпълнено:
% \begin{itemize}
% \item 
%   Представяме всяко от числата $n_1,\dots,n_k$ в монадична бройна система
%   като лентата на $\M$ има вида:  
%   \[\dots \blank \blank \underbrace{1111\dots 11}_{n} \blank\blank\dots,\]
%   като изискваме главата на $\M$ да е позиционирана върху най-лявата единица.
%   Такава конфигурация ще наричаме {\bf стандартна начална конфигурация}.
% \item
%   Ако $f(n_1,\dots,n_k) = m$, то $\M$ завършва с резултат върху лентата
%   \[\dots \blank \blank \underbrace{1111\dots 11}_{m} \blank\blank\dots,\]
%   като главата на $\M$ е върху най-лявата 1-ца.
%   Такава конфигурация се нарича {\bf стандартна финална конфигурация}.
% \item
%   Ако $f(n_1,\dots,n_k)$ е недефенирана, то $\M$ няма да завърши в стандартна конфигурация, т.е.
%   или $\M$ ще работи безкрайно време, или ще завърши в конфигурация, която не е стандартна.
% \end{itemize}

% \begin{example}

%   Да видим, че функцията $f(n) = 2n$ е изчислима.
  
% % \[\stackrel{1}{1}11\ \Rightarrow\ \stackrel{2}{\blank} 111\ \Rightarrow\ \stackrel{3}{\blank}\blank 111\ \Rightarrow\ \stackrel{4}{\blank}1\blank 111\ \Rightarrow\ 1\stackrel{5}{1}\blank 111\ \Rightarrow\  11\stackrel{5}{\blank}111\ \Rightarrow\  11 \blank \stackrel{6}{1}11\]
% % \[\dots \Rightarrow\ 11\blank 111\stackrel{6}{\blank}\ \Rightarrow\ 11\blank 11\stackrel{7}{1}\blank\ \Rightarrow\ 11\blank 1\stackrel{8}{1}\blank\blank\ \Rightarrow\ 11\blank \stackrel{9}{1}1\blank\blank\ \Rightarrow\  1\stackrel{10}{1}\blank 11\blank\blank\]
% % \[\dots \Rightarrow\ \stackrel{10}{\blank}11\blank 11\blank\blank\ \Rightarrow\ \stackrel{2}{1}1\blank 11\blank\blank\ \Rightarrow\ \cdots\]

% \begin{figure}[H]
%   \begin{center}
%     \begin{tikzpicture}[->,>=stealth,thick,node distance=45pt]
%       \tikzstyle{every state}=[circle,minimum size=10pt,auto,scale=.7]
      
%       \node[state]   (1) {$1$};
%       \node[state]            (2) [right of=1]{$2$};
%       \node[state]            (3) [right of=2]{$3$};
%       \node[state]            (4) [right of=3]{$4$};
%       \node[state]            (5) [right of=4]{$5$};
%       \node[state]            (6) [right of=5]{$6$};
%       \node[state]            (7) [right of=6]{$7$};
%       \node[state]            (8) [right of=7]{$8$};
%       \node[state]            (9) [right of=8]{$9$};
%       \node[state]            (10) [right of=9]{$10$};
%       \node[state]            (11) [below of=8]{$11$};
%       \node[state,accepting]  (12) [below of=11]{$12$};
      
%       \begin{scope}[every node/.style={scale=.8}]
%       \path
%       (1) edge [bend left=15] node [above] {$1;L$} (2)
%       (2) edge [bend left=15] node [above] {$1;L$} (3)
%       (2) edge [bend right=15] node [below] {$\blank;L$} (3)
%       % (3) edge [bend left=15] node [above] {$1/1;L$} (4)
%       (3) edge [bend left=15] node [above] {$\blank/1;L$} (4)
%       (4) edge [bend left=15] node [above] {$\blank/1;R$} (5)
%       % (4) edge [bend right=15] node [below] {$1/1;R$} (5)
%       (5) edge [loop below] node [below] {$1;R$} (5)
%       (5) edge [bend left=15] node [above] {$\blank;R$} (6)
%       (6) edge [loop below] node [below] {$1;R$} (6)
%       (6) edge [bend left=15] node [above] {$\blank;L$} (7)
%       (7) edge [bend left=15] node [above] {$1/\blank;L$} (8)
%       % (7) edge [bend right=15] node [below] {$\blank;L$} (8)
%       (8) edge [bend left=15] node [above] {$1;L$} (9)
%       (9) edge [loop below] node [below] {$1;L$} (9)
%       (9) edge [bend right=15] node [below] {$\blank;L$} (10)
%       (10) edge [loop below] node [below] {$1;L$} (10)
%       % (10) [out=130,in=120,above,distance=2.5cm] node [above] {$x;R$} (2)
%       (10) edge [out=140,in=60, above] node [below] {$\blank;R$} (2)
%       (8) edge [] node [right] {$\blank;L$} (11)
%       (11) edge [loop left] node [left] {$1;L$} (11)
%       (11) edge [] node [right] {$\blank;R$} (12);
%       \end{scope}
%     \end{tikzpicture}
%   \end{center}
% \end{figure}

% \begin{align*}
%   _1111 & \to\ _2\blank 111 \to\  _3\blank \blank 111 \to\ _4\blank 1 \blank 111 \to 1_51\blank 111 \to 11_5\blank 111\\
%   & \to 11\blank_6 111 \to \cdots \to 11 \blank 11_71 \to 11 \blank 1_81\blank \to 11 \blank _911\blank \to 11_9 \blank 11\blank\\
%   & \to 1_{10}1\blank 11 \blank \to \cdots \to\ _211\blank 11\blank \to \cdots \to 1_5111\blank 11 \blank \to \cdots \\
%   & \to 1111\blank_6 11\blank \to \cdots
% \end{align*}

% \end{example}

% \begin{problem}
%   За произволно естествено число $n$, дефинирайте МТ $\M_n$ с $n+11$ състояния, за която,
%   ако главата е на най-лявата $1$-ца върху блок от $1$-ци, то $\M_n$
%   завършва като записва $2n$ единици на лентата и завършва в стандартна конфигурация.
% \end{problem}

% \begin{thm}
%   За всяко $k$, съществуват функции от вида $f:\Nat^k\to\Nat$, които не са изчислими с МТ.
% \end{thm}
% \begin{proof}
%   Знаем, че всяка МТ може да се кодира с естествено число.
%   Това означава, че съществуват изброимо безкрайно много различни машини на Тюринг.
%   Също така, ние знаем, че съществуват неизброимо много различни функции от вида $f:\Nat^k\to\Nat$.
%   Заключаваме, че със сигурност съществуват функции, които не са изчислими с МТ.
% \end{proof}

% \section{Примери за разрешими и полуразрешими езици}

% Да напомним, че:
% \begin{itemize}
% \item
%   {\bf полуразрешими} са тези езици, които се разпознават от машина на Тюринг.
% \item
%   {\bf разрешими} са тези езици, които се разпознават от тотална машина на Тюринг.
% \end{itemize}

% Ще разгледаме два основни примера за езици. Единият ще бъде за език, който не е полуразрешим, а другият за език, който е 
% полуразрешим, но не е разрешим.


% % \section{Проблемът за съответствие на Пост (PCP)}

% % \subsection*{MPCP}

\subsection{Диагоналният език $L_{\texttt{diag}}$}

% Нека $\omega_0,\omega_1,\dots,\omega_n,\dots$ е каноничната подредба на всички думи над азбуката $\{0,1\}$.
% Да разгледаме безкрайната таблица $\{a_{ij} \mid i,j \in \Nat\}$, където:
% \begin{align*}
%   a_{ij} = 
%   \begin{cases}
%     1, & \text{ ако $j$ е код на М.Т. и }\omega_i \in L(\M_j), \\
%     0, & \text{ ако $j$ не е код на М.Т. или } \omega_i \not\in L(\M_j).
%   \end{cases}
% \end{align*}

% Идеята е да вземем $0$-ите по диагонала на тази таблица.

\begin{framed}
  \begin{thm}
    Езикът 
    \[L_{\texttt{diag}} \df \{ \alpha \in \{0,1\}^\star \mid \alpha = \code{\M}\text{, където $\M$ е М.Т. и }\code{\M} \not\in L(\M)\}\]
    не се разпознава от машина на Тюринг, т.е. $L_{\texttt{diag}}$ {\bf не} е полуразрешим език.
  \end{thm}
\end{framed}
\begin{proof}
  Да допуснем, че $L_{\texttt{diag}}$ се разпознава от машина на Тюринг $\M$, т.е. 
  \[L_{\texttt{diag}} = \L(\M).\]
  Тогава:
  \begin{align*}
    & \code{\M} \in L_{\texttt{diag}} \implies \code{\M} \in \L(\M) \implies \code{\M} \not\in L_{\texttt{diag}},\\
    & \code{\M} \not\in L_{\texttt{diag}} \implies \code{\M} \not\in \L(\code{\M}) \implies \code{\M} \in L_{\texttt{diag}}.
  \end{align*}
  Достигаме до противоречие.
\end{proof}

\begin{prop}
  Езикът 
  \[L_{\texttt{halt}} \df \{\code{\M} \mid \text{$\M$ е М.Т. и }\code{\M} \in \L(\M)\}\]
  е полуразрешим, но не е разрешим.
\end{prop}
\begin{hint}
  Лесно се съобразява, че $L_{\texttt{halt}}$ е полуразрешим.
  Дефинираме машина на Тюринг $\M'$, която работи по следния начин:
  \begin{itemize}
  \item
    вход дума $\alpha$;
  \item 
    $\M'$ проверява дали $\alpha$ има вида $\code{\M}$,
    за някоя машина на Тюринг $\M$;
  \item
    Ако $\alpha = \code{\M}$, 
    то $\M'$ симулира работата на $\M$ върху $\alpha$.
    \begin{itemize}
    \item 
      Ако $\M$ завърши след краен брой стъпки като приема $\alpha$,
      то $\M'$ приема $\alpha$.
    \item
      Ако $\M$ завърши след краен брой стъпки като отхвърля $\alpha$,
      то $\M'$ отхвърля $\alpha$.
    \item
      Ако $\M$ никога не завършва върху $\alpha$,
      то $\M'$ също никога не завършва върху $\alpha$.
    \end{itemize}
  \item
    Ако $\alpha$ няма вида $\code{\M}$,
    то $\M'$ завършва като отхвърля думата $\alpha$.
  \end{itemize}
  Получаваме, че
  \[\alpha \in L_{\texttt{halt}} \iff \alpha \in \L(\M'),\]
  откъдето следва, че $L_{\texttt{halt}}$ е полуразрешим език.

  Ако допуснем, че $L_{\texttt{halt}}$ е разрешим,
  то 
  \[L_{\texttt{diag}} = (\{0,1\}^\star \setminus L_{\texttt{halt}}) \cap \{\code{\M} \mid \text{$\M$ е М.Т.}\}\]
  е разрешим език, което е противоречие.
\end{hint}


%%% Local Variables:
%%% mode: latex
%%% TeX-master: "../eai"
%%% End:


\subsection{Универсалният език $L_{\texttt{univ}}$}

\setlength{\epigraphwidth}{0.65\textwidth}\epigraph{A man provided with paper, pencil, and rubber, and subject to strict discipline, is in effect a universal machine. (Turing 1948: 416)}


\marginpar{Можем за простота да считаме, че всички разглеждани машини на Тюринг са дефинирани над азбуката $\{0,1\}$.}
\index{език!неразрешим}
\begin{framed}
  \begin{thm}
    Езикът 
    \[\Luniv \df \{\ \code{\M} \sharp \omega \mid \text{$\M$ е машина на Тюринг и }\omega\in \L(\M)\ \}\]
    е полуразрешим, но {\bf не} е разрешим.
  \end{thm}
\end{framed}
\begin{hint}
  \marginpar{Разсъждението е много сходно с това защо $\Laccept$ полуразрешим.}
  Първо да съобразим защо $\Luniv$ е полуразрешим език.
  Дефинираме (многолентова) машина на Тюринг $\M'$, която работи по следния начин:
  \begin{itemize}
  \item
    вход дума $\alpha$;
  \item 
    $\M'$ проверява дали $\alpha$ има вида $\code{\M} \cdot \omega$,
    за някоя машина на Тюринг $\M$ и дума $\omega$. Това става лесно, защото $\omega$
    започва веднага след второ срещане на $111$ в $\alpha$.
  \item
    Ако $\alpha = \code{\M} \sharp \omega$, 
    то $\M'$ симулира работата на $\M$ върху $\omega$.
    \begin{itemize}
    \item 
      Ако $\M$ завърши след краен брой стъпки като приеме $\omega$,
      то $\M'$ приема $\alpha$.
    \item
      Ако $\M$ завърши след краен брой стъпки като отхвърли $\omega$,
      то $\M'$ отхвърля $\alpha$.
    \item
      Ако $\M$ никога не завършва върху $\omega$,
      то очевидно $\M'$ също никога не завършва върху $\alpha$.
    \end{itemize}
  \item
    Ако $\alpha$ няма вида $\code{\M} \cdot \omega$,
    то $\M'$ завършва веднага като отхвърля думата $\alpha$.
  \end{itemize}
  Получаваме, че
  \[\alpha \in \Luniv \iff \alpha \in \L(\M').\]
  
  Сега да съобразим защо $\Luniv$ не е разрешим език.
  Имаме, че за произволна дума $\omega$,
  \[\omega \in \Laccept \iff \omega\sharp\omega \in \Luniv.\]
  Ако допуснем, че $\Luniv$ е разрешим, то тогава $\Laccept$ е разрешим език, което е противоречие.
\end{hint}

\begin{cor}
  Езикът
  \[\ov{\Luniv} \df \{\code{\M} \cdot \omega \mid \code{\M} \text{ е машина на Тюринг и }\omega\not\in \L(\M)\}\]
  {\bf не} е полуразрешим.
\end{cor}





%%% Local Variables:
%%% mode: latex
%%% TeX-master: "../eai"
%%% End:


\section{Критерий за разрешимост}

\mynote{Сипсър нарича $\leq_m$ \emph{mapping reducibility} \cite[235]{sipser3}.}

\begin{important}
  Доказателството, че $\Luniv$ не е разрешим е пример за една обща схема, с която можем да докажем, че даден език не е разрешим:
  \begin{itemize}
  \item 
    Нека имаме езика $K$, за който вече знаем, че не е разрешим.
    В нашия пример, $K = \Laccept$.
  \item
    Питаме се дали някой друг език $L$ е разрешим.
  \item
    Намираме изчислима тотална функция $f$, за която е изпълнено, че:
    \[\omega \in K \iff f(\omega) \in L.\]
    В \Theorem{universal}, това е функцията $f(\omega) = \omega \sharp \omega$.
  \item
    В този случай ще означаваме $K \leq_m L$.
  \item
    Тогава, ако $L$ е разрешим ще следва, че $K$ е разрешим, което е противоречие.
  \end{itemize}
\end{important}

Сега искаме да разгледаме един критерий, който ще ни казва кога един език съставен от кодове на машини на Тюринг е разрешим. С негова помощ ще можем директно да решаваме наглед трудни задачи. Например,
в момента не е очевидно защо следния език не е разрешим:
\begin{align*}
  L_{\texttt{palin}} \df \{\omega \in \{0,1\}^\star \mid & \ \omega\text{ е код на машина на Тюринг и }\L(\M_\omega)\\
                                                         & \text{ съдържа само думи палиндроми}\}.
\end{align*}
След малко ще видим, че според критерия, който ще разгледаме, директно ще можем да заключим, че $L_{\texttt{palin}}$ не е разрешим. Да започнем с няколко примера.

\begin{important}
  \begin{proposition}
    \label{pr:rice:sigma-star}
    Езикът
    \[L_{\Sigma^\star} \df \{\omega \in \{0,1\}^\star \mid \omega\text{ е код на машина на Тюринг и }\L(\M_\omega) = \Sigma^\star\}\]
    не е разрешим.
  \end{proposition}  
\end{important}
\begin{proof}
  \mynote{$L_{\Sigma^\star}$ не е дори полуразрешим, но за момента не знаем как да докажем това.}
  Ще дефинираме тотална изчислима функция $f$, която при вход думата $\omega \in \{0,1\}^\star$ работи по следния начин:
  \begin{itemize}
  \item
    Ако $\omega$ не е код на машина на Тюринг, то $f(\omega) \df \omega$.
  \item
    Ако $\omega$ е код на машина на Тюринг $\M_\omega$, то
    $f(\omega) \df \code{\M'}$, където $\M'$, при вход произволна дума $\alpha$, работи по следния начин:
    \begin{enumerate}[(1)]
    \item
      Първоначално $\M'$ не обръща внимание на $\alpha$, а $\M'$ симулира работата на $\M_\omega$ върху думата $\omega$.
    \item 
      Ако след краен брой стъпки симулацията завърши с резултат, че $\M_\omega$ приема думата $\omega$,
      то $\M'$ завършва като приема думата $\alpha$.
    \end{enumerate}    
  \end{itemize}
  Получаваме, че:
  \[\L(\M') =
    \begin{cases}
      \Sigma^\star, & \text{ако } \omega \in \L(\M_\omega)\\
      \emptyset, & \text{ако } \omega \not\in \L(\M_\omega).
    \end{cases}
  \]
  Можем да заключим, че за произволна дума $\omega$ са изпълнени импликациите:
  \begin{align*}
    \omega \in \Laccept & \implies \L(\M_{f(\omega)}) = \Sigma^\star \implies f(\omega) \in L_{\Sigma^\star},\\
    \omega \in \Lcode \setminus \Laccept & \implies \L(\M_{f(\omega)}) = \emptyset \implies f(\omega) \not\in L_{\Sigma^\star}\\
    \omega \not\in\Lcode & \implies f(\omega) = \omega \implies f(\omega) \not\in L_{\Sigma^\star},
  \end{align*}
  които можем да обобщим в следната еквивалентност:
  \[\omega \in \Laccept \iff f(\omega) \in L_{\Sigma^\star}\]
  Ако допуснем, че $L_{\Sigma^\star}$ е разрешим език, то $\Laccept$ също ще е разрешим, което е противоречие.
\end{proof}

\begin{corollary}
  Езикът
  \[\ov{L}_{\texttt{empty}} \df \{\omega \in \{0,1\}^\star \mid \omega \text{ е код на машина на Тюринг и }\L(\M_\omega) \neq \emptyset\}\]
  е полуразрешим, но не е разрешим.
\end{corollary}
\begin{hint}
  Съобразете, че в може да използвате функцията $f$ от доказателството на \Proposition{rice:sigma-star} за да получите, че:
  \[\omega \in \Laccept \iff f(\omega) \in \ov{L}_{\texttt{empty}}.\]
\end{hint}

\begin{corollary}
  Езикът
  \[\Lempty \df \{\omega \in \{0,1\}^\star \mid \omega\text{ е код на машина на Тюринг и }\L(\M_\omega) = \emptyset\}\]
  не е полуразрешим.
\end{corollary}
\begin{hint}
  Ако $\Lempty$ беше разрешим, то неговото допълнение
  \[\ov{L}_{\texttt{empty}} = \Lcode \setminus \Lempty\]
  щеше да е разрешим език, което е противоречие.

  Ако $\Lempty$ беше полуразрешим, тогава, използвайки, че $\ov{L}_{\texttt{empty}}$ е полуразрешим, от теоремата на Клини-Пост щеше да следва, че
  $\Lempty$ е разрешим, което е противоречие
\end{hint}


\begin{important}
  \begin{proposition}
    Езикът
    \[\Lreg \df \{\ \omega \mid \omega\text{ е код на машина на Тюринг и }\L(\M_\omega) \text{ е регулярен език}\ \}\]
    не е разрешим.
  \end{proposition}
\end{important}
\begin{proof}
  \mynote{\cite[стр. 219]{sipser3}}
  Да фиксираме един език, за който знаем, че не е регулярен, например, 
  $\{0^n1^n \mid n \in \Nat\}$.
  Ще дефинираме тотална изчислима функция $f$, която при вход думата $\omega \in \{0,1\}^\star$ работи по следния начин:
  \begin{itemize}
  \item
    Ако $\omega$ не е код на машина на Тюринг, то $f(\omega) \df \omega$.
  \item
    Ако $\omega$ е код на машината на Тюринг $\M_\omega$, то тогава $f(\omega) \df \code{\M'}$,
    където $\M'$, при вход произволна дума $\alpha$, работи така:
    \begin{enumerate}[(1)]
    \item
      Ако $\alpha = 0^n1^n$, за някое $n$, то $\M'$ приема думата $\alpha$.
    \item
      Ако $\alpha$ не е от вида $0^n1^n$, тогава $\M'$ симулира работата на $\M_\omega$ върху думата $\omega$.
    \item
      Ако след краен брой стъпки симулацията завърши с резултат, че $\M_\omega$ приема думата $\omega$, то $\M'$ завършва като приема $\alpha$.
    \end{enumerate}
  \end{itemize}
  \mynote{Използваме наготово, че $\{0,1\}^\star$ е регулярен език.}
  Получаваме, че:
  \[\L(\M') =
    \begin{cases}
      \{0,1\}^\star & \text{ако } \omega \in \L(\M_\omega)\\
      \{0^n1^n \mid n \in \Nat\} & \text{ако } \omega \not\in \L(\M_\omega).
    \end{cases}
  \]
  Сега можем да заключим, че за произволна дума $\omega$ са изполнени импликациите:
  \begin{align*}
    \omega \in \Laccept & \implies \L(\M_{f(\omega)}) = \{0,1\}^\star \implies f(\omega) \in \Lreg,\\
    \omega \in \Lcode \setminus \Laccept & \implies \L(\M_{f(\omega)}) = \{0^n1^n \mid n \in \Nat\} \implies f(\omega) \not\in \Lreg,\\
    \omega \not\in \Lcode & \implies f(\omega) = \omega \implies f(\omega) \not\in \Lreg,
  \end{align*}
  което можем да обединим в еквивалентността:
  \[\omega \in \Laccept \iff f(\omega) \in \Lreg\]
  и ако допуснем, че $\Lreg$ е разрешим език, то $\Laccept$ също ще е разрешим, което е противоречие.  
\end{proof}

Сега ще видим, че идеята, която следвахме в горните доказателства може да се обобщи.
Нека $\Ss$ е множество от полуразрешими езици над фиксирана азбука $\Sigma$.
Ще казваме, че $\Ss$ е свойство на полуразрешимите езици.
Например, 
\[\Ss = \{L \subseteq \Sigma^\star \mid L\text{ е регулярен език}\}.\]
$\Ss$ е {\bf тривиално свойство}, ако $\Ss = \emptyset$ или $\Ss$ съдържа точно всички полуразрешими езици.
Нека разгледаме изброимото множество от всички машини на Тюринг, които разпознават езиците от $\Ss$.
Ще представим това множество като език от кодовете на тези машини на Тюринг, т.е.
\index{$\texttt{Code}(\Ss)$}
\[\texttt{Code}(\Ss) \df \{\omega \mid \text{$\omega$ е код на машина на Тюринг и } \L(\M_\omega) \in \Ss\}.\]
\index{$\texttt{Code}(L)$}
\mynote{Можем да дефинираме и $\texttt{Code}(L)$, което е безкрайно изброимо множество, ако $L$ е полуразрешим език.}

\begin{problem}
  Докажете, че езикът
  \[L_{\texttt{Dec}} = \{\omega \in \{0,1\}^\star \mid \omega \text{ е код на машина на Тюринг и }\L(\M_\omega)\text{ е разрешим}\}\]
  не е разрешим.
\end{problem}

\begin{problem}
  Докажете, че езикът
  \begin{align*}
    L_{\texttt{palin}} \df \{\omega \in \{0,1\}^\star \mid & \ \omega\text{ е код на машина на Тюринг и }\L(\M_\omega)\\
                                                           & \text{ съдържа само думи палиндроми}\}.
  \end{align*}
  не е разрешим.
\end{problem}

Сега вече имаме достатъчно опит за да видим точно кои проблеми са разрешими.

\begin{important}
  \begin{theorem}[Райс 1953 \cite{rice}]
    \index{Райс}
    \mynote{\cite[стр. 188]{hopcroft1}}
    За всяко нетривиално свойство $\Ss$ на полуразрешимите езици,
    $\texttt{Code}(\Ss)$ е неразрешим.
  \end{theorem}
\end{important}
\begin{proof}
  \mynote{Цел: да сведем ефективно $\Laccept$ към $L_\Ss$}
  Без ограничение на общността, нека $\emptyset \not\in \Ss$.
  Понеже $\Ss$ е нетривиално свойство, да разгледаме езика $L \in \Ss$,
  като $\M_L$ е машина на Тюринг, за която $\L(\M_L) = L$.
  Ще дефинираме тотална изчислима функция $f$, която при вход думата $\omega \in \{0,1\}^\star$ работи по следния начин:
  \begin{itemize}
  \item
    Ако $\omega$ не е код на машина на Тюринг, то $f(\omega) \df \omega$.
  \item
    Ако $\omega$ е код на машината на Тюринг $\M_\omega$, то тогава $f(\omega) \df \code{\M'}$,
    където $\M'$, при вход произволна дума $\alpha$, работи така:
    \begin{enumerate}[(1)]
    \item
      първоначално $\M'$ не обръща внимание на входната дума $\alpha$, а започва да симулира работата на $\M_\omega$ върху $\omega$.
    \item
      % \mynote{в този случай ще получим, че $\L(\M') = L$}
      ако след краен брой стъпки симулацията завърши с резултат, че $\M_\omega$ приема думата $\omega$, то 
      $\M'$ започва да симулира работата на $\M_L$ върху входната дума $\alpha$;
    \item
      ако след краен брой стъпки симулацията завърши с резултат, че $\M_L$ приема думата $\alpha$, то 
      $\M'$ приема входната дума $\alpha$;
    \end{enumerate}
  \end{itemize}
  Така получаваме, че:
  \[\L(\M') =
    \begin{cases}
      L, & \text{ако }\omega \in \L(\M_\omega)\\
      \emptyset, & \text{ако }\omega \not\in \L(\M_\omega).
    \end{cases}
  \]
  Оттук заключаваме, че за произволна дума $\omega$ са изпълнени импликациите:
  \begin{align*}
    \omega \in \Laccept & \implies \L(\M_{f(\omega)}) = L \implies f(\omega) \in \texttt{Code}(\Ss),\\
    \omega \in \Lcode \setminus \Laccept & \implies \L(\M_{f(\omega)}) = \emptyset \implies f(\omega) \not\in\texttt{Code}(\Ss),\\
    \omega \not\in \Lcode & \implies f(\omega) = \omega \implies f(\omega) \not\in\texttt{Code}(\Ss),
  \end{align*}
  които можем да обобщим в следната еквивалентност:
  \[\omega \in \Laccept \iff f(\omega) \in \texttt{Code}(\Ss).\]
  Aко допуснем, че $\texttt{Code}(\Ss)$ е разрешимо множество, то ще следва, че $\Laccept$ е разрешимо, което е противоречие.

  Ако $\emptyset \in \Ss$, то правим горните разсъждения за класа от езици
  \[\ov{\Ss} = \{ L \subseteq \Sigma^\star \mid L\text{ е полуразрешим език и } L \not\in\Ss\ \}.\]
  По аналогичен начин доказваме, че $\texttt{Code}(\ov{\Ss})$ не е разрешим език.
  Понеже 
  \[\texttt{Code}(\ov{\Ss}) = \Lcode \setminus \texttt{Code}(\Ss),\]
  то $\texttt{Code}(\Ss)$ също не е разрешим език.
\end{proof}

\begin{corollary}
  За всяко от следните свойства $\Ss$ на полуразрешимите езици, 
  $\texttt{Code}(\Ss)$ {\bf не} е разрешим език, където:
  \mynote{Тук няма нужда нищо да доказваме. Просто съобразяваме, че всяко от тези свойства на полуразрешимите езици е нетривиално.}
  \begin{enumerate}[a)]
  \item 
    $\Ss$ е свойството празнота, т.е. езикът
    \[\texttt{Code}(\Ss) = \{\omega \mid \text{$\omega$ е код на машина на Тюринг и } \L(\M_\omega) = \emptyset\}\]
    не е разрешим;
  \item 
    $\Ss$ е свойството за пълнота, т.е. езикът
    \[\texttt{Code}(\Ss) = \{\omega \mid \text{$\omega$ е код на машина на Тюринг и } \L(\M_\omega) = \Sigma^\star\}\]
    не е разрешим;
  \item
    $\Ss$ е свойството крайност, т.е. езикът
    \[\texttt{Code}(\Ss) = \{\omega \mid \text{$\omega$ е код на машина на Тюринг и }|\L(\M_\omega)| < \infty\}\]
    не е разрешим;
  \item
    $\Ss$ е свойството безкрайност, т.е. езикът
    \[\texttt{Code}(\Ss) = \{\omega \mid \text{$\omega$ е код на машина на Тюринг и }|\L(\M_\omega)| = \infty\}\]
    не е разрешим;
  \item
    $\Ss$ е свойството регулярност, т.е. езикът
    \[\texttt{Code}(\Ss) = \{\omega \mid \text{$\omega$ е код на машина на Тюринг и $\L(\M_\omega)$ е регулярен език}\}\]
    не е разрешим;
  \item
    \mynote{Това свойство е нетривиално, защото вече показахме, че $\{a^nb^nc^n \mid n \in \Nat\}$ е полуразрешим (дори разрешим) език, а знаем отдавна, че този език не е безконтекстен.}
    $\Ss$ е свойството безконтекстност, т.е. езикът
    \[\texttt{Code}(\Ss) = \{\omega \mid \text{$\omega$ е код на машина на Тюринг и $\L(\M_\omega)$ е безконтекстен}\}\]
    не е разрешим;
  \item
    \mynote{Тук също - вече сме разгледали примери за полуразрешими езици, които не са разрешими.}
    $\Ss$ е свойството разрешимост, т.е. езикът
    \[\texttt{Code}(\Ss) = \{\omega \mid \text{$\omega$ е код на машина на Тюринг и $\L(\M_\omega)$ е разрешим}\}\]
    не е разрешим.
  \end{enumerate}
\end{corollary}


%%% Local Variables:
%%% mode: latex
%%% TeX-master: "../eai"
%%% End:


\section{Критерии за полуразрешимост}

\begin{lemma}
  Нека $\Ss$ е свойство на полуразрешимите езици.
  Ако съществува безкраен език $L_0 \in \Ss$, който няма крайно подмножество в $\Ss$,
  то $L_\Ss$ не е полуразрешим език.  
\end{lemma}
\begin{hint}
  Нека $L_0 = \L(\M_0)$.
  Ще опишем алгоритъм, който при вход дума $\pair{\M,\omega}$,
  извежда код на машина на Тюринг $\M'$, която работи така:
  \begin{itemize}
  \item 
    вход думата $\alpha$;
  \item
    за $\abs{\alpha}$ стъпки симулираме $\M$ върху $\omega$.
    \begin{itemize}
    \item 
      ако $\M$ не приема $\omega$ за $\leq \abs{\alpha}$ стъпки, то симулираме $\M_0$ върху $\alpha$;
    \item 
      ако $\M$ приема $\omega$ за $\leq \abs{\alpha}$ стъпки, то зацикляме и нищо не връщаме.
    \end{itemize}
  \end{itemize}

  Така получаваме, че 
  \begin{align*}
    \L(\M') = 
    \begin{cases}
      \{\alpha \in L_0 \mid \abs{\alpha} < k\}, & \M\text{ приема }\omega\\
      L, & \M\text{ не приема }\omega,
    \end{cases}
  \end{align*}
  където $k$ е минималната стъпка, при която $\M$ приема $\omega$.
  
  Заключаваме, че 
  \[\code{\M}\cdot \omega \not\in L_{\texttt{univ}} \iff \code{\M'} \in L_\Ss.\]
  Това означава, че ефективно можем да сведем въпрос за принадлежност в $\bar{L}_u$
  към въпрос за принадлежност в $L_\Ss$.
  Следователно, ако $L_\Ss$ е полуразрешим език, то $\bar{L}_{\texttt{univ}}$ е полуразрешим език, което е противоречие.
\end{hint}

\begin{cor}
  Следните езици {\bf не} са полуразрешими:
  \begin{itemize}
  \item 
    $L = \{\pair{\M} \mid \abs{\L(\M)} = \infty\}$;
  \item
    $L = \{\pair{\M} \mid \L(\M) = \Sigma^\star\}$;
  \item
    $L = \{\pair{\M} \mid \L(\M)\text{ не е разрешим}\}$;
  \item
    $L = \{\pair{\M} \mid \L(\M)\text{ не е полуразрешим}\}$;
  \item
    $L = \{\pair{\M} \mid \L(\M)\text{ не е регулярен}\}$.
  \end{itemize}
\end{cor}

\begin{lemma}
  Нека $L_1$ е език в $\Ss$ и нека $L_2$ е полуразрешимо множество, разширяващо $L_1$, и $L_2 \not\in\Ss$.
  Тогава $L_\Ss$ не е полуразрешимо.
\end{lemma}
\begin{hint}
  Нека $L_1 = \L(\M_1)$ и $L_2 = \L(\M_2)$.
  Ще опишем алгоритъм, който при вход дума $\code{\M}\cdot\omega$,
  извежда код на машина на Тюринг $\M'$, която работи така:
  \begin{itemize}
  \item 
    вход думата $\alpha$;
  \item
    Симулираме едновременно две изчисления - $\M_1$ върху $\alpha$ и $\M$ върху $\omega$
    докато намерим $s$, такова че:    
    \begin{itemize}
    \item 
      ако $\M_1$ приеме думата $\alpha$ за $s$ стъпки, то обявяваме, че $\M'$ приема $\alpha$ и завършваме.
    \item
      ако $\M_1$ не приема думата $\alpha$ за $s$ стъпки, но $\M$ приема $\omega$ за $s$ стъпки, 
      то започваме да симулираме $\M_2$ върху $\alpha$.
      Ако $\M_2$ приеме $\alpha$, то $\M'$ приема $\alpha$.
    \end{itemize}
    Ако не съществува такава стъпка $s$, то работата $\M'$ никога няма да завърши и 
    следователно $\M'$ не завършва върху вход $\alpha$.
  \end{itemize}
  
  Получаваме, че:
  \begin{align*}
    \L(\M') = 
    \begin{cases}
      L_2, & \M\text{ приема }\omega\\
      L_1, & \M\text{ не приема }\omega.
    \end{cases}
  \end{align*}
  Заключаваме, че:
  \[\code{\M}\cdot\omega \not\in L_{\texttt{univ}} \iff \code{\M'} \in L_\Ss.\]
  Това означава, че ефективно можем да сведем въпрос за принадлежност в $\bar{L}_{\texttt{univ}}$
  към въпрос за принадлежност в $L_\Ss$.
  Следователно, ако $L_\Ss$ е полуразрешим език, то $\bar{L}_{\texttt{univ}}$ е полуразрешим език, което е противоречие.  
\end{hint}

\begin{cor}
  Следните езици {\bf не} са полуразрешими:
  \begin{itemize}
  \item 
    $L = \{\pair{\M} \mid \L(\M) \text{ е регулярен} \}$;
  \item
    $L = \{\pair{\M} \mid \L(\M) \text{ е безконтекстен} \}$;
  \item
    $L = \{\pair{\M} \mid \L(\M) \text{ е разрешим} \}$;
  \item
    $L = \{\pair{\M} \mid \abs{\L(\M)} = 42\}$;
  \end{itemize}
\end{cor}


% % \section{Проблеми за безконтекстни езици}

% % \begin{lemma}
% %   Нека е дадена $\M = \TM$.
% %   Тогава езикът 
% %   \[L = \{\alpha\sharp\beta^R \mid \alpha,\beta \in \Gamma^\star Q \Gamma^\star\ \&\  \alpha \vdash_\M \beta\}\]
% %   е безконтекстен.
% % \end{lemma}
% % \begin{proof}
% %   Ще покажем, че съществува стеков автомат $P$, за който $\L_S(P) = L$.
% %   Четем буквата $X$. Тогава:
% %   \begin{itemize}
% %   \item 
% %     ако $\delta_\M(q,X) =(p,Y,R)$, то слагаме $Yp$ на върха на стека;
% %   \item
% %     ако $\delta_\M(q,X) =(p,Y,L)$, то ако $Z$ е върха на стека, заменяме $Z$ с $pZY$;
% %   \end{itemize}
% % \end{proof}

\begin{lemma}
  Нека е дадена $\M = \TM$.
  Тогава езикът 
  \[L = \{\alpha\sharp\beta^R \mid \alpha,\beta \in \Gamma^\star Q \Gamma^\star\ \&\  \alpha \not\vdash_\M \beta\}\]
  е безконтекстен.
\end{lemma}


% % \begin{thm}
% %   Неразрешим е проблемът за проверка дали при дадени две произволни безконтекстни граматики $G_1$ и $G_2$,
% %   $\L(G_1) \cap \L(G_2) = \emptyset$.  
% % \end{thm}

% % \begin{thm}
% %   Неразрешим е проблемът за проверка дали при дадена произволна безконтекстна граматика $G$,
% %   $\L(G) = \Sigma^\star$.  
% % \end{thm}


% % \section{Въпроси}

% % Вярно ли е, че следният проблем е {\em разрешим}:
% % \begin{itemize}
% % \item
% %   за произволна безконтекстна граматика $G$, проверява дали $\L(G) = \emptyset$?
% % \item
% %   за произволна безконтекстна граматика $G$, проверява дали $\L(G) = \Sigma^\star$?
% % \item
% %   за произволни безконтекстни граматики $G_1$ и $G_2$, проверява дали $\L(G_1) \cap \L(G_2) = \emptyset$?
% % \item
% %   за произволни безконтекстни граматики $G_1$ и $G_2$, проверява дали $\L(G_1) \cap \L(G_2) = \Sigma^\star$?
% % \item
% %   за произволни безконтекстни граматики $G_1$ и $G_2$, проверява дали $\L(G_1) = \L(G_2)$?
% % \item
% %   за произволни безконтекстни граматики $G_1$ и $G_2$, проверява дали $\L(G_1) \subseteq \L(G_2)$?
% % \item
% %   за произволна безконтекстна граматика $G$ и произволен регулярен израз $r$,
% %   проверява дали $\L(G) = \L(r)$?
% % \item
% %   за произволна безконтекстна граматика $G$ и произволен регулярен израз $r$,
% %   проверява дали $\L(G) \subseteq \L(r)$?
% % \item
% %   за произволна безконтекстна граматика $G$ и произволен регулярен израз $r$,
% %   проверява дали $\L(r) \subseteq \L(G)$?
% % \item
% %   за произволни безконтекстни граматики $G_1$ и $G_2$, проверява дали $\L(G_1) \subseteq \L(G_2)$ 
% %   е безконтекстен език ?
% % \item
% %   за произволна безконтекстна граматика $G$, проверява дали $\Sigma^\star \setminus \L(G)$
% %   е безконтекстен език ?
% % \item
% %   за произволна безконтекстна граматика $G$, проверява дали $\L(G)$ е регулярен език?
% % \end{itemize}

%%% Local Variables:
%%% mode: latex
%%% TeX-master: "../eai"
%%% End:


% \subsection*{Безконтекстен език за преходите в машина на Тюринг}

% \subsection*{Валидни и невалидни изчисления на машини на Тюринг}
\marginpar{\cite{hopcroft1} стр. 201}
Да разгледаме машината на Тюринг $\M$.

Една дума $\omega$ описва конфигурация на машина на Тюринг,
ако $\omega \in \Gamma^\star Q \Gamma^\star$.

\begin{framed}
  \begin{prop}
    Да фиксираме една машина на Тюринг $\M$. 
    Тогава следните езици за безконтекстни:
    \begin{align*}
      & \texttt{Valid}(\M) \df \{\ \alpha\#\beta^{rev} \mid \alpha,\beta \in \Gamma^\star Q \Gamma^\star\ \&\ \alpha \vdash_\M \beta\ \} \\
      & \texttt{Valid}'(\M)\df \{\ \alpha^{rev}\#\beta \mid \alpha,\beta \in \Gamma^\star Q \Gamma^\star\ \&\ \alpha \vdash_\M \beta\ \} \\
      & \texttt{Invalid}(\M) \df \{\ \alpha\#\beta^{rev} \mid \alpha,\beta \in \Gamma^\star Q \Gamma^\star\ \&\  \alpha \not\vdash_\M \beta\ \}\\
      & \texttt{Invalid}'(\M) \df \{\ \alpha^{rev}\#\beta \mid \alpha,\beta \in \Gamma^\star Q \Gamma^\star\ \&\ \alpha \not\vdash_\M \beta\ \}.
    \end{align*}
  \end{prop}  
\end{framed}

\begin{hint}

  Да напомним първо как дефинираме релацията $\vdash_\M$:
  \begin{align*}
    & (\alpha_1z, q, x\alpha_2) \vdash_\M  (\alpha_1 zy, p, \alpha_2) & \comment{\text{ ако } q \overset{x/y;\goright}{\longrightarrow} p} \\
    & (\alpha_1z, q, x\alpha_2) \vdash_\M (\alpha_1, p ,zy\alpha_2) & \comment{\text{ ако } q \overset{x/y;\goleft}{\longrightarrow} p} \\
    & (\alpha_1z, q, x\alpha_2) \vdash_\M (\alpha_1z, p, y\alpha_2) & \comment{\text{ ако } q \overset{x/y;\stay}{\longrightarrow} p}.
  \end{align*}

  Думите в езика $\texttt{Valid}(\M)$ кодират релацията $\vdash_\M$. Това означава, че всяка дума на 
  $\texttt{Valid}(\M)$ има някое от следните представяния:
  \begin{align*}
    & \alpha_1zqx\alpha_2 \sharp \alpha^{rev}_2 p y z \alpha^{rev}_1 & \comment{\text{ ако } q \overset{x/y;\goright}{\longrightarrow} p} \\
    & \alpha_1zqx\alpha_2 \sharp \alpha^{rev}_2 y z p \alpha^{rev}_1 & \comment{\text{ ако } q \overset{x/y;\goleft}{\longrightarrow} p} \\
    & \alpha_1zqx\alpha_2 \sharp \alpha^{rev}_2 y p z\alpha^{rev}_1 & \comment{\text{ ако } q \overset{x/y;\stay}{\longrightarrow} p}
  \end{align*}

  Ще опишем неформално стеков автомат $P$ за езика $\texttt{Valid}(\M)$.
  Нека 
  \[Q^{P} \df \{r_q \mid q \in Q^\M\} \cup \{r, \hat{r}\}.\]

  \begin{itemize}
  \item
    Първо четем $\alpha_1$ и я записваме в стека като $\alpha^{rev}_1$.
    Това правим като дефинираме функцията на преходите като 
    \[(\forall a,z \in \Sigma)[\ \Delta_{P}(r,a,z) \df \{(r,az)\}\ ].\]
  \item 
    Правим това докато не срещнем някое $q \in Q^\M$. Тогава трябва да направим преход на $\M$.
    Тук трябва да внимаваме, защото за да направим преход, трябва да знаем състоянието $q$ и да прочетем следващия символ.
    Един начин да разрешим този проблем е като запомним кое състояние сме прочели на машината на Тюринг в състоянията на стековия автомат:
    \marginpar{$Q^\M$ са букви от азбуката на стековия автомат $P$.}
    \[(\forall q \in Q^\M)(\forall z \in \Sigma)[\ \Delta_{P}(r,q,z) = \{(r_q,z)\}\ ].\]
    \begin{itemize}
    \item 
      \marginpar{Стекът представлява $z\alpha^{rev}_1$}
      ако $\delta_\M(q,x) = (p,y,\goright)$, то слагаме $yp$ на върха на стека, т.е.
      \[\delta_{P}(r_q,x,z) = \{(\hat{r}, pyz)\}.\]
    \item
      ако $\delta_\M(q,x) =(p,y,\goleft)$, то ако $z$ е върха на стека, заменяме $z$ с $pzy$, т.е.
      \[\delta_{P}(r_q,x,z) = \{(\hat{r}, yzp)\}.\]
    \item
      ако $\delta_\M(q,x) =(p,y,\stay)$, то ако $z$ е върха на стека, заменяме $z$ с $ypz$, т.е.
      \[\delta_{P}(r_q,x,z) = \{(\hat{r}, ypz)\}.\]
    \end{itemize}
  \item
    Сега вече сме в състояние $\hat{r}$ и остава да прочетем $\alpha_2$ и да я запишем в стека като $\alpha^{rev}_2$:
    \[\delta_{P}(\hat{r},x,z) = \{(\hat{r}, xz)\}.\]
  \item
    \marginpar{За да разпознаем $\texttt{Invalid}(\M)$ трябва само да разменим условията за приемане и отхвърляне на думата.}
    Разбираме кога сме свършили с $\alpha_2$ когато стигнем до $\sharp$.
    Сега започваме да четем думата след $\sharp$ и сравняваме с това, което имаме в стека.
    \begin{itemize}
    \item
      Ако намерим разлика, то отхвърляме думата.
    \item
      Ако достигнем до дъното на стека, то приемаме думата.
    \end{itemize}
  \end{itemize}
\end{hint}

\begin{remark}
  Да обърнем внимание, че горната конструкция на стековия автомат $P$ е {\bf ефективна}, т.е.
  съществува алгоритъм, който при вход машина на Тюринг $\M$ връща като изход стеков автомат $P$ за езика $\texttt{Valid}(\M)$.
  % С други думи, езикът 
  % \[\{\code{\M} \cdot \code{P} \mid \L(P) = \texttt{Valid}(\M)\}\]
  % е разрешим.
\end{remark}

\subsection*{История на машина на Тюринг}
\index{история на приемащо изчисление}

Дума от вида  $\omega_1 \sharp \omega^{rev}_2 \sharp \omega_3 \sharp \omega^{rev}_4\sharp\omega_5\cdots$
се нарича {\bf история на приемащо изчисление} на машината на Тюринг $\M$, ако
\begin{itemize}
\item
  $\omega_i \in \Gamma^\star Q \Gamma^\star$, т.е. $\omega_i$ описва моментна конфигурация
  и $\omega_i$ не започва и не завършва на $\blank$.
\item
  $\omega_1 \in \qstart\Sigma^\star$ описва начална конфигурация.
\item
  $\omega_n \in \Gamma^\star \cdot\{\qaccept\} \cdot \Gamma^\star$ описва приемаща конфигурация.
\item 
  $\omega_i \vdash_\M \omega_{i+1}$ за $i = 1,\dots,n-1$.
\end{itemize}

\begin{lemma}
  \marginpar{\cite{hopcroft1}, стр. 201}
  Нека да означим с $\texttt{Accept}(\M)$ езикът от историите на всички приемащи изчисления за машината на Тюринг $\M$.
  Тогава 
  \[\texttt{Accept}(\M) = L_1 \cap L_2,\]
  където $L_1$ и $L_2$ са безконтекстни езици.
  Освен това, граматиките на $L_1$ и $L_2$ могат ефективно да бъдат построени от $\M$.
\end{lemma}
\begin{hint}
  Да разгледаме езиците:
  \begin{align*}
    & L_1 \df (\texttt{Valid}(\M)\sharp)^\star(\{\varepsilon\}\cup \Gamma^\star \cdot \{\qaccept\} \cdot \Gamma^\star\sharp)\\
    & L_2 \df \qstart\Sigma^\star \sharp (\texttt{Valid'}(\M)\sharp)^\star(\{\varepsilon\}\cup \Gamma^\star \cdot \{\qaccept\} \cdot \Gamma^\star\sharp),
  \end{align*}
  за които е ясно, че са безконтекстни.
\end{hint}

\begin{problem}
  Обяснете как може ефективно да се кодира всяка безконтекстна граматика $G$ като дума $\code{G}$ над азбуката $\{0,1\}$.
\end{problem}


\begin{framed}
\begin{thm}
  Езикът
  \[L = \{\code{G_1}\cdot\code{G_2} \mid \text{$G_1$ и $G_2$ са безконт. грам. и }\L(G_1) \cap \L(G_2) = \emptyset\}\]
  не е полуразрешим.
\end{thm}  
\end{framed}
\begin{hint}
  По дадена дума $\code{\M}$, можем ефективно да намерим $G_1$ и $G_2$, за които
  $\L(G_1) \cap \L(G_2) = \texttt{Accept}(\M)$, т.е. съществува тотална изчислима функция $f$, за която
  \[f(\code{\M}) = \code{G_1} \cdot \code{G_2}.\]
  Тогава ако $L$ е полуразрешим език, то $L_{\texttt{Empty}}$ е полуразрешим език, което е противоречие, защото
  \[\code{\M} \in L_{\texttt{Empty}} \iff f(\code{\M}) \in L.\]
\end{hint}

\begin{lemma}
  За всяка машина на Тюринг $\M$, $\overline{\texttt{Accept}(\M)}$ е безконтекстен език.
\end{lemma}
\begin{hint}
  Една дума $\alpha$ не е история на приемащо изчисление, ако е изпълнено някое от следните условия:
  \begin{itemize}
  \item 
    \marginpar{Можем да опишем това свойство с регулярен език}
    $\alpha$ не е от вида $\omega_1 \sharp \omega_2 \sharp \cdots \sharp \omega_n$,
    където $\omega_i \in \Gamma^\star Q \Gamma^\star$, или
  \item
    ако $\alpha$ е от вида $\omega_1 \sharp \omega_2 \sharp \cdots \sharp \omega_n$,
    където $\omega_i \in \Gamma^\star Q \Gamma^\star$, тогава:
    \begin{itemize}
    \item 
      $\omega_1 \not\in \qstart \Gamma^\star$, или
    \item
      $\omega_n \not\in \Gamma^\star \cdot \{\qaccept\} \cdot \Gamma^\star$, или
    \item
      $\omega_i \not\vdash_\M \omega^{rev}_{i+1}$, за някое нечетно $i$, или
    \item
      $\omega^{rev}_i \not\vdash_\M \omega_{i+1}$, за някое четно $i$.
    \end{itemize}
  \end{itemize}
  Думите притежаващи някое от тези свойства могат да се опишат като обединение на три регулярни езика и двата безконтекстни езика.
\end{hint}

\begin{framed}
  \begin{thm}
    За дадена азбука $\Sigma$, 
    езикът 
    \[\texttt{All}_{\texttt{CFG}} = \{\code{G} \mid G\text{ е безконтекстна граматика и }\L(G) = \Sigma^\star\}\]
    не е полуразрешим език.
  \end{thm}
\end{framed}
\begin{hint}
  \marginpar{Тук използваме, че ако $\L(\M) = \emptyset$, то $\overline{\texttt{Accept}(\M)} = \Sigma^\star$.}
  По дадена дума $\code{\M}$, можем ефективно да намерим $G$, за която
  $\L(G)$ са точно невалидните изчисления на $\M$.
  Тогава ако допуснем, че $L$ е полуразрешим език, то $L_{\texttt{Empty}}$ е полуразрешим, което е противоречие.
\end{hint}

\begin{cor}
  Следните езици не са разрешими:
  \begin{enumerate}[a)]
  \item
    $\{\code{G_1}\cdot\code{G_2} \mid \text{$G_1$ и $G_2$ са безконт. грам. и }\L(G_1) = \L(G_2)\}$;
  \item
    $\{\code{G_1}\cdot\code{G_2} \mid \text{$G_1$ и $G_2$ са безконт. грам. и }\L(G_1) \subseteq \L(G_2)\}$;
  \item 
    $\{\code{G}\cdot r \mid \text{$G$ е безконт. грам. и $r$ е рег. израз и }\L(G) = \L(r)\}$;
  \item
    $\{\code{G}\cdot \code{\A} \mid \text{$G$ е безконт. грам. и $\A$ е ДКА и }\L(G) = \L(\A)\}$;
  \item 
    $\{\code{G}\cdot r \mid \text{$G$ е безконт. грам. и $r$ е рег. израз и }\L(r) \subseteq \L(G)\}$;
  \item
    $\{\code{G}\cdot \code{\A} \mid \text{$G$ е безконт. грам. и $\A$ е ДКА и }\L(\A) \subseteq \L(G)\}$.
  \end{enumerate}
\end{cor}

\begin{remark}
  Добре е да обърнем внимание, че езикът 
  \[L = \{\code{G}\cdot \code{\A} \mid \text{$G$ е безконт. грам. и $\A$ е ДКА и }\L(G) \subseteq \L(\A)\}\]
  е разрешим.
  Това е така, защото $\L(G) \subseteq \L(\A) \iff \L(G) \cap \L(\ov{\A}) = \emptyset$,
  защото сечението на безконтекстен и регулярен език е безконтекстен език.
\end{remark}

\newpage

\begin{framed}
  \begin{prop}
    Езикът 
    \[\texttt{Reg} = \{\code{G} \mid G\text{ е безконт. грам. и $\L(G)$ е регулярен}\}\]
    не е разрешим.
  \end{prop}
\end{framed}
\begin{hint}
  Да фиксираме език $L_0$, който е безконтекстен, но не е регулярен.
  За произволен език $L$, да разгледаме езика
  \[\hat{L} \df L_0 \sharp \Sigma^\star\ \cup\ \Sigma^\star \sharp L.\]
  Първо ще докажем, че: 
  \begin{equation}
    \label{eq:2}
    L = \Sigma^\star\ \iff\ \hat{L}\text{ е регулярен}.
  \end{equation}
  Да отбележим, че можем ефективно да получим от безконтекстна граматика $G$ за $L$
  безконтекстна граматика $\hat{G}$ за $\hat{L}$.
  Нека да означим с $\texttt{conv}$ изчислимата функция, за която
  $\texttt{conv}(\code{G}) = \code{\hat{G}}$.

  \begin{itemize}
  \item 
    Ако $L = \Sigma^\star$, то $\hat{L}$ е регулярен, защото тогава
    $\hat{L} = \Sigma^\star \sharp \Sigma^\star$ е очевидно регулярен.
  \item
    \marginpar{Ако $L$ е регулярен, то $L/_\beta \df \{\alpha \mid \alpha\beta \in L\}$ е регулярен}  
    Ако $L \neq \Sigma^\star$, то нека да фиксираме дума $\omega \not\in L$.
    Ако допуснем, че $\hat{L}$ е регулярен, то езикът
    $\hat{L}/_{\sharp\omega} = L_0$ ще е регулярен, което е противоречие с избора на $L_0$.
  \end{itemize}
  
  Нека да означим
  \[\texttt{Full} \df \{\code{G} \mid G\text{ е безконтекстна граматика и }\L(G) = \Sigma^\star\}.\]
  От (\ref{eq:2}) имаме, че 
  \[\code{G} \in \texttt{Full}\ \iff\ \texttt{conv}(\code{G}) \in \texttt{Reg}.\]
  
  Ако допуснем, че $\texttt{Reg}$ е разрешим език, то тогава ще следва, че
  $\texttt{Full}$ е разрешим език, за което вече знаем, че не е вярно.
\end{hint}


%%% Local Variables:
%%% mode: latex
%%% TeX-master: "../eai"
%%% End:


% \section{Неограничени граматики}
\index{граматика!неограничена}

\begin{definition}
  \mynote{\cite[стр. 220]{hopcroft1}}
  \mynote{На англ. unrestricted grammar}
  \mynote{Според йерархията на Чомски, това е граматика от тип 0}
  Граматиката $G = (V,\Sigma,R,S)$
  се нарича неограничена граматика, 
  ако правилата $R$ са от вида $\alpha \to \beta$,
  където $\alpha,\beta \in (V\cup\Sigma)^\star$.
\end{definition}

\begin{lemma}
  За всеки полуразрешим език $L$, $L = \L(G)$, за някоя неограничена граматика $G$.  
\end{lemma}
\begin{proof}
  Нека $L = \L(\M)$, където 
  \[\M = \TM\] е детерминистична машина на Тюринг,
  като искаме лентата да е безкрайна само отдясно и входната дума $\alpha$ е
  поставена в началото на лентата.
  Ще построим граматика $G = \CFG$, където 
  \[V = ((\Sigma\cup\{\varepsilon\})\times\Gamma) \cup \{A_1,A_2,A_3\}.\]
  Правилата на $G$ са следните:
  \begin{enumerate}[1)]
  \item 
    $A_1 \to sA_2$;
  \item
    $A_2 \to [a,a]A_2$, за всяка $a\in\Sigma$;
  \item
    $A_2 \to A_3$;
  \item
    $A_3 \to [\varepsilon,\blank]A_3$;
  \item
    $A_3 \to \varepsilon$;
  \item
    $q[a,X] \to [a,Y]p$, за всяка $a \in \Sigma\cup\{\varepsilon\}$, всяко $q\in Q$, $X,Y \in\Gamma$, 
    за които $\delta(q,X) = (p,Y,R)$;
  % \item
  %   $q[a,X] \to p[a,Y]$, за всяка $a \in \Sigma\cup\{\varepsilon\}$, всяко $q\in Q$, $X,Y \in\Gamma$, 
  %   за които $\delta(q,X) = (p,Y,N)$;
  \item
    $[b,Z]q[a,X] \to p[b,Z][z,Y]$, за всяко $X,Y,Z \in \Gamma$, $a,b\in\Sigma\cup\{\varepsilon\}$, $q\in Q$,
    за които $\delta(q,X) = (p,Y,L)$;
  \item
    $[a,X]q \to qaq$, $q[a,X] \to qaq$, $q \to \varepsilon$, за всяко $a\in\Sigma\cup\{\varepsilon\}$, $X\in\Gamma$,
    и $q \in F$.
  \end{enumerate}
  
  Лесно се вижда, че, използвайки правилата 1) и 2), за всяко $n$, имаме
  \[A_1 \to^\star s[a_1,a_1]\cdots[a_n,a_n]A_2,\]
  където $a_i \in \Sigma$.

  Нека $\M$ приема думата $\alpha = a_1\cdots a_n$.
  Това означава, че за някое $m$, $\M$ използва не повече от $m$ клетки от лентата отдясно на входната дума.
  Ясно е, че имаме
  \[A_1 \to^\star s[a_1,a_1]\cdots[a_n,a_n][\varepsilon,\blank]^m.\]
  Оттук нататък, можем да използваме само правилата 6), 7), 8), докато не срещнем финално състояние.
  С индукция по броя на стъпки в $\M$, можем да докаже, че ако е изпълнено
  $(\varepsilon,s,a_1\cdots a_n) \vdash^\star_\M (X_1\cdots X_{r-1},q,X_r\cdots X_l)$, 
  то \[s[a_1,a_1]\dots[a_n,a_n][\varepsilon,\blank]^m \rightarrow^\star_G [a_1,X_1]\cdots[a_{r-1},X_{r-1}]q[a_r,X_r]\cdots[a_{n+m},X_{n+m}],\]
  където $a_1,\dots,a_n \in \Sigma$, $a_{n+1},\dots,a_{n+m} = \varepsilon$, $X_1,\dots,X_{n+m} \in \Gamma$ и
  $X_{l+1} = X_{l+2} = \dots = X_{n+m} = \blank$.
  
  Най-накрая, ако $q \in F$, то можем да използваме правилата от 9) и да докажем, че
  \[[a_1,X_1]\cdots[a_{t-1},X_{t-1}]q[a_t,X_t]\cdots[a_{n+m},X_{n+m}] \rightarrow^\star_G a_1\cdots a_n.\]
  
  Така доказахме, че ако $\alpha \in \L(\M)$, то $\alpha \in \L(G)$, т.е. $\L(\M) \subseteq \L(G)$.
  За да докажем обратната посока, трябва да направим подобни разсъждения.
\end{proof}

\begin{lemma}
  Ако $L = \L(G)$, където $G$ е неограничена граматика, то $L$ е полуразрешим език.
\end{lemma}
\mynote{Доказателствата в \cite{hopcroft1} и \cite{papadimitriou} са различни}
\begin{proof}
  $\M$ ще бъде недетерминистична машина с три ленти.
  \begin{enumerate}[1)]
  \item
    Записваме входната дума $\omega$ на първата лента на $\M$.
    Тя никога не се променя.
  \item
    На втората лента ще имаме думата $\gamma \in (V\cup\Sigma)^\star$.
    В началото $\gamma := S$.
  \item 
    Недетерминистично избираме правило $\alpha \to \beta$ от граматиката $G$.
  \item
    Недетерминистично избираме $\gamma_0,\gamma_1 \in (V\cup\Sigma)^\star$, за които 
    $\gamma = \gamma_0\alpha\gamma_1$.
    Тогава $\gamma := \gamma_0\beta\gamma_1$.
    Ако няма такива $\gamma_0$ и $\gamma_1$, то $\M$ ,,зацикля'' - текущият опит за извеждане на $\omega$ пропада.
  \item
    Сравняваме съдържанието на първите две ленти, т.е. проверяваме дали $\omega = \gamma$.
    Ако $\omega = \gamma$, то спираме и казваме, че $\M$ разпознава думата $\omega$.
    Ако $\omega \neq \gamma$, то се връщаме на стъпка 3).
  \end{enumerate}

  \begin{algorithm}[H]
  \caption{}
%  \label{alg:}
  \begin{algorithmic}[1]
    \State $\gamma:= S$
    \ForAll{$\alpha\to\beta \in R$}
    \If{$(\exists \gamma_0,\gamma_1\in (V\cup\Sigma)^\star)[\gamma = \gamma_0\alpha\gamma_1]$}
    \State $\gamma := \gamma_0\beta\gamma_1$
    \Else ...
    \EndIf
    \EndFor
  \end{algorithmic}
\end{algorithm}

\end{proof}

\begin{example}
  Граматика за $L = \{a^nb^nc^n \mid n\in\Nat\}$.
\end{example}

%%% Local Variables:
%%% mode: latex
%%% TeX-master: "../eai"
%%% End:


\section{Сложност}

\begin{itemize}
\item 
  Детерминистичната машината на Тюринг $\M$ е {\bf полиномиално ограничена}, ако 
  същестува полином $p(x)$, такъв че за всеки вход $\omega$,
  машината $\M$ завършва след най-много $p(|\omega|)$ стъпки.
\item
  Езикът $L$ се нарича {\bf полиномиално разрешим},
  ако съществува полиномиално ограничена тотална детерминистична машина на Тюринг $\M$,
  за която $L = \L(\M)$.
% \item
\item
  Недетерминистичната машината на Тюринг $\M$ е {\bf полиномиално ограничена}, ако 
  същестува полином $p(x)$, такъв че за всеки вход $\omega$,
  съществува изчисление на машината $\M$ върху думата $\omega$,
  което завършва след най-много $p(|\omega|)$ стъпки.
\item
  Езикът $L$ се нарича {\bf недетерминистично полиномиално разрешим},
  ако съществува полиномиално ограничена тотална недетерминистична машина на Тюринг $\M$,
  за която $L = \L(\M)$.
% \item
%   $\mathcal{NP} \df \{L \subseteq \Sigma^\star \mid L\text{ е полиномиално разрешим с НМТ}\}$.
\end{itemize}

\begin{framed}
  \begin{dfn}
    \begin{align*}
      & \mathcal{P} \df \{L \subseteq \Sigma^\star \mid L\text{ е полиномиално разрешим с ДМТ}\};\\
      & \mathcal{EXP} \df \{L \subseteq \Sigma^\star \mid L\text{ е експоненциално разрешим с ДМТ}\};\\
      & \mathcal{NP} \df \{L \subseteq \Sigma^\star \mid L\text{ е полиномиално разрешим с НМТ}\}.
    \end{align*}
  \end{dfn}
\end{framed}

От Теорема ... знаем, че
\[\mathcal{NP} \subseteq \mathcal{EXP}.\]

\begin{prop}
  За азбука $\Sigma$ от поне две букви, можем да обобщим някои от резултатите от предишните глави:
  \[\texttt{REG} \subsetneqq \texttt{CFG} \subsetneqq \mathcal{P}.\]
\end{prop}
\begin{hint}
  Езикът $\{a^nb^nc^n \mid n \in \Nat\} \in \mathcal{P}$,
  но не е безконтекстен.
\end{hint}


%%% Local Variables:
%%% mode: latex
%%% TeX-master: "../eai"
%%% End:


\section*{Бележки}

\begin{itemize}
\item
  За основните дефиниции следваме основно Глава 3 от \cite{sipser3}.
\item 
  За въпросите за неразрешимост следваме основно Глава 8 от \cite{hopcroft1}.
\end{itemize}

% \begin{itemize}
% \item 
% \item
% \item

% \end{itemize}


%%% Local Variables:
%%% mode: latex
%%% TeX-master: "../eai"
%%% End:
