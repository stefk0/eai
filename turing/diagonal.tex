\subsection{Диагоналният език $L_d$}

Нека $\omega_0,\omega_1,\dots,\omega_n,\dots$ е каноничната подредба на всички думи над азбуката $\{0,1\}$.
Да разгледаме безкрайната таблица $\{a_{ij} \mid i,j \in \Nat\}$, където:
\begin{align*}
  a_{ij} = 
  \begin{cases}
    1, & \text{ ако } \omega_i \in L(\M_j), \\
    0, & \text{ ако } \omega_i \not\in L(\M_j).
  \end{cases}
\end{align*}

Идеята е да вземем $0$-ите по диагонала на тази таблица.

\begin{framed}
  Езикът 
  $L_d = \{w_i \mid w_i \not\in L(\M_i)\}$ не се разпознава от машина на Тюринг,
  т.е. $L_d$ {\bf не} е полуразрешим език.
\end{framed}
Да допуснем, че $L_d$ се разпознава от машина на Тюринг, т.е. $L_d = \L(\M_i)$, за някоя машина на Тюринг с код $i$.
Тогава:
\begin{align*}
  & \omega_i \in L_d \implies \omega_i \in \L(\M_i) \implies \omega_i \not\in L_d,\\
  & \omega_i \not\in L_d \implies \omega_i \not\in \L(\M_i) \implies \omega_i \in L_d.
\end{align*}
Достигаме до противоречие.

\begin{remark}
  Да обърнем внимание, че $\bar{L}_d = \{\omega_i \mid \omega_i \in \L(\M_i)\}$ е полуразрешим език,
  който очевидно не е разрешим.
\end{remark}


%%% Local Variables:
%%% mode: latex
%%% TeX-master: "../eai"
%%% End:
