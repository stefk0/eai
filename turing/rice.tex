\section{Критерий за разрешимост}

Нека $\Ss$ е множество от полуразрешими езици над азбуката $\{0,1\}$.
Ще казваме, че $\Ss$ е свойство на полуразрешимите езици.
$\Ss$ е тривиално свойство, ако $\Ss = \emptyset$ или $\Ss$ съдържа точно всички полуразрешими езици.
Нека $L_\Ss = \{\pair{\M} \mid \L(\M) \in \Ss\}$.

\begin{thm}[Райс-Успенски]
  \index{Райс-Успенски}
  \marginpar{\cite[стр. 188]{hopcroft1}}
  Всяко нетривиално свойство $\Ss$ на полуразрешимите езици е неразрешимо.
\end{thm}
\begin{proof}
  \marginpar{Цел: да сведем ефективно $L_u$ към $L_\Ss$}
  Без ограничение на общността, нека $\emptyset \not\in \Ss$.
  Понеже $\Ss$ е нетривиално свойство, да разгледаме $L \in \Ss$,
  като $\M_L$ е машина на Тюринг, за която $\L(\M_L) = L$.
  Да разгледаме алгоритъм $A$, който за дадена дума $\pair{\M,w}$
  връща код на машина на Тюринг $\M'$, която работи по следния начин:
  \begin{itemize}
  \item
    \marginpar{Неформално описваме функцията $\delta$ за $\M'$}
    имаме вход - произволна дума $x$;
  \item
    \marginpar{$\M$ и $w$ са предварително фиксирани. Кодът на $\M'$ зависи от тях}
    първоначално не обръщаме внимание на $x$, а питаме дали $\pair{\M,w} \in L_u$, т.е. дали $\M$ приема думата $w$;
    \begin{itemize}
    \item 
      ако съществува стъпка $s$, за която $\pair{\M,w} \in L^s_u$, то симулираме $\M_L$ върху входната дума $x$;
      в този случай получаваме $\L(\M') = L$;
    \item
      ако не съществува стъпка $s$, за която $\pair{\M,w} \in L^s_u$, то 
      няма да разпознаем нито една дума;
      в този случай получаваме $\L(\M') = \emptyset$.      
    \end{itemize}
  \end{itemize}
  От всичко това следва, че ако $A(\pair{\M,w}) = \pair{\M'}$, то:
  \begin{align*}
    & \pair{\M,w} \in L_u \implies (\exists s)[\pair{\M,w} \in \L^s_u] \implies \L(\M') = L \implies \L(\M') \in \Ss,\\
    & \pair{\M,w} \not\in L_u \implies (\forall s)[\pair{\M,w} \not\in \L^s_u] \implies \L(\M') = \emptyset \implies \L(\M') \not\in \Ss.
  \end{align*}
  Да допуснем, че $\Ss$ е разрешимо множество от полуразрешими езици.
  Тогава от еквивалентността,
  \[\pair{\M,w} \in \L_u \iff \pair{\M'} \in L_\Ss,\]
  получаваме, че $\L_u$ е разрешимо множество, което е противоречие.
\end{proof}

\begin{cor}
  Следните свойства $\Ss$ на полуразрешимите множества {\bf не} са разрешими:
  \begin{enumerate}[a)]
  \item 
    празнота, т.е. $\Ss = \{\pair{\M} \mid \L(\M) = \emptyset\}$;
  \item
    крайност, т.е. $\Ss = \{\pair{\M} \mid |\L(\M)| < \infty\}$;
  \item
    регулярност, т.е. $\Ss = \{\pair{\M} \mid (\exists \text{ рег. израз }r)[\L(\M) = \L(r)]\}$;
  \item
    безконтекстност, т.е. $\Ss = \{\pair{\M} \mid (\exists\text{ безконт. грам. }G)[\L(\M) = \L(G)]\}$.
  \end{enumerate}
\end{cor}

%%% Local Variables:
%%% mode: latex
%%% TeX-master: "../eai"
%%% End:
