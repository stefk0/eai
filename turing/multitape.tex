\subsection*{Многолентови машини на Тюринг}
\index{машина на Тюринг!многолентова}
%Това е просто като имаш shift.
%Използват се при недет. машини

Машина на Тюринг с $k$ ленти има същата дефиниция като еднолентова машина на Тюринг
с единствената разлика, че
\[\delta: Q \times \Gamma^k\to Q \times \Gamma^k \times \{\goleft,\goright,\stay\}^k.\]
Тук добавяме и възможността главата върху някои от лентите да стои на място.

\begin{itemize}
\item
  $\Sigma \df \{a,b\}$;
\item
  $\Gamma \df \{a,b,\blank\}$
\item
  $\delta:Q\times\Gamma^2 \to Q\times\Gamma^2\times \{\goleft, \goright, \stay\}^2$;
\end{itemize}

\begin{framed}
  \begin{figure}[H]
    \begin{center}
      \begin{tikzpicture}[->,>=stealth,thick,node distance=70pt]
        \tikzstyle{every state}=[circle,minimum size=10pt,auto,scale=.9]
        
        \node[state,initial]    (1) {$q_1$};
        \node[state]            (2) [right of=1]{$q_2$};
        \node[state]            (3) [right of=2]{$q_3$};
        \node[state,accepting]  (4) [right of=3]{$q_4$};
        % \node[state]            (5) [below of=3]{$q_5$};
        
        \begin{scope}% [every node/.style={scale=.8}]
          \path
          (1) edge [loop above] node [above] {$\frac{a}{\blank} / \frac{a}{a};\frac{\goright}{\goright}$} (1)
          (1) edge [loop below] node [below] {$\frac{b}{\blank} / \frac{b}{b};\frac{\goright}{\goright}$} (1)
          (1) edge [bend left=15] node [above] {$\frac{\sharp}{\blank};\frac{\goright}{\stay}$} (2)
          (2) edge [loop above] node [above] {$\frac{a}{\blank};\frac{\goright}{\stay}$} (2)
          (2) edge [loop below] node [below] {$\frac{b}{\blank};\frac{\goright}{\stay}$} (2)
          (2) edge [bend left=15] node [above] {$\frac{\blank}{\blank};\frac{\goleft}{\goleft}$} (3)
          (3) edge [loop above] node [above] {$\frac{a}{a};\frac{\goleft}{\goleft}$} (3)
          (3) edge [loop below] node [below] {$\frac{b}{b};\frac{\goleft}{\goleft}$} (3)
          (3) edge [bend left=15] node [above] {$\frac{\sharp}{\blank};\frac{\stay}{\stay}$} (4);
        \end{scope}
      \end{tikzpicture}
    \end{center}
    \caption{двулентова детерминистична частична машина на Тюринг $\M$, за която $\L(\M) = \{\omega\sharp\omega \mid \omega \in \{a,b\}^\star\}$}
  \end{figure}
\end{framed}

В началото втората лента е празна. Имаме две глави, които се движат независимо една от друга.
\begin{align*}
  (q_1, \frac{\hat{a}}{\hat{\blank}}\frac{b}{\blank}\frac{\sharp}{\blank}\frac{a}{\blank}\frac{b}{\blank}) & \vdash (q_1, \frac{a}{a}\frac{\hat{b}}{\hat{\blank}}\frac{\sharp}{\blank}\frac{a}{\blank}\frac{b}{\blank}) \vdash (q_1, \frac{a}{a}\frac{b}{b}\frac{\hat{\sharp}}{\hat{\blank}}\frac{a}{\blank}\frac{b}{\blank}) \vdash (q_2, \frac{a}{a}\frac{b}{b}\frac{\sharp}{\hat{\blank}}\frac{\hat{a}}{\blank}\frac{b}{\blank})\\
                                                                                                           & \vdash (q_2, \frac{a}{a}\frac{b}{b}\frac{\sharp}{\hat{\blank}}\frac{\hat{a}}{\blank}\frac{b}{\blank}) \vdash (q_2, \frac{a}{a}\frac{b}{b}\frac{\sharp}{\hat{\blank}}\frac{a}{\blank}\frac{\hat{b}}{\blank}) \vdash (q_2, \frac{a}{a}\frac{b}{b}\frac{\sharp}{\hat{\blank}}\frac{a}{\blank}\frac{b}{\blank}\frac{\hat{\blank}}{\blank})\\
                                                                                                           & \vdash (q_3, \frac{a}{a}\frac{b}{\hat{b}}\frac{\sharp}{\blank}\frac{a}{\blank}\frac{\hat{b}}{\blank}) \vdash (q_3, \frac{a}{\hat{a}}\frac{b}{b}\frac{\sharp}{\blank}\frac{\hat{a}}{\blank}\frac{b}{\blank}) \vdash (q_3, \frac{\blank}{\hat{\blank}}\frac{a}{a}\frac{b}{b}\frac{\hat{\sharp}}{\blank}\frac{a}{\blank}\frac{b}{\blank})\\
                                                                                                           & \vdash (q_4, \frac{\blank}{\hat{\blank}}\frac{a}{a}\frac{b}{b}\frac{\hat{\sharp}}{\blank}\frac{a}{\blank}\frac{b}{\blank}).
\end{align*}

\begin{prop}
  За всяка $k$-лентова машина на Тюринг $\M$ съществува еднолентова машина на Тюринг $\M'$,
  такава че $\L(\M) = \L(\M')$.
\end{prop}
\begin{proof}
  \marginpar{В \cite[стр. 177]{sipser3} конструкцията е малко по-различна. Там съдържанието на всяка лента се поставя последователно върху една лента, като се разделят със специален символ. Тук следваме \cite[стр. 162]{hopcroft1}}
  Нека $\M$ е $k$-лентова машина на Тюринг.
  Ще построим еднолентова машина на Тюринг $\M'$, за която $\L(\M) = \L(\M')$.
  Да означим $\hat\Gamma = \{\hat X \mid X \in \Gamma\}$.
  Тогава азбуката на лентата на $\M'$ ще бъде $\Gamma' = (\hat\Gamma \cup \Gamma)^{k}$.
  Сега вместо да имаме $k$ ленти ще имаме една лента, която представлява $k$-орка.
  За да симулираме $\M$, използваме символите $\hat X$ за да маркират позицията на главите на $\M$,
  като във всяка компонента на лентата има точно по един символ от вида $\hat X$.
  % С $\$$ ще отблезяваме границите на всяка лента, в която можем да търсим маркера.
  За да определим следващия ход на машината $\M'$, трябва да сканираме лентата докато не 
  открием разположението на всичките $k$ на брой маркирани клетки. Тогава симулираме ход на $\M$
  и отново трябва да променим маркираните клетки.
\end{proof}

{\bf Да се обясни, че има квадратично забавяне при преход от многолентова машина към еднолентова.}


%%% Local Variables:
%%% mode: latex
%%% TeX-master: "../eai"
%%% End:
