\section{Критерии за полуразрешимост}

Вече знаем, че на практика всички интересни въпроси за полуразрешими езици не са разрешими.
Сега да видим какво можем да кажем, ако се ограничим само до позитивната част на тези въпроси.

\begin{important}
  \begin{lemma}
    \label{lem:rice-shapiro:finite}
    Нека $\Ss$ е свойство на полуразрешимите езици.
    Ако съществува безкраен език $L_0 \in \Ss$, който няма краен подезик в $\Ss$,
    то $\Code(\Ss)$ {\em не е} полуразрешим език.  
  \end{lemma}
\end{important}
\mynote{Това означава, че ако $\Code(\Ss)$ е полуразрешим език, то всеки език $L_0 \in \Ss$ притежава краен подезик, който също принадлежи на $\Ss$.}
\begin{hint}
  Нека $L_0 = \L(\M_0)$ като $L_0 \in \Ss$, но всеки краен подезик на $L_0$ не принадлежи на $\Ss$.
  Ще докажем, че имаме следната връзка:
  \[\Ldiag \leq_m \Code(\Ss).\]
  Ще дефинираме тотална изчислима функция $f$, която при вход думата $\omega \in \{0,1\}^\star$ работи по следния начин:
  \begin{itemize}
  \item
    Ако $\omega$ не е код на машина на Тюринг, то $f(\omega) \df \omega$.
  \item
    Ако $\omega$ е код на машината на Тюринг $\M_\omega$, то тогава $f(\omega) \df \code{\M'}$,
    където $\M'$, при вход произволна дума $\alpha$, работи така:
    \begin{enumerate}[(1)]
    \item
      Първоначално $\M'$ не обръща внимание на $\alpha$, а симулира работата на $\M_\omega$ върху думата $\omega$.
    \item
      Ако симулацията завърши за по-малко от $|\alpha|$ на брой стъпки с резултат, че $\M_\omega$ приема $\omega$, то $\M'$ завършва веднага като \emph{отхвърля} $\alpha$.
    \item
      В противен случай, $\M'$ симулира работата на $\M_0$ върху $\alpha$.
    \item
      Ако симулацията завърши с резултат, че $\M_0$ приема $\alpha$, то $\M'$ завършва като приеме думата $\alpha$.
    \end{enumerate}
  \end{itemize}
  Така получаваме, че:
  \begin{align*}
    \L(\M') = 
    \begin{cases}
      \{\alpha \in L_0 \mid \abs{\alpha} < \texttt{step}_\omega\}, & \text{ако } \omega \in \L(\M_\omega)\\
      L_0, & \text{ако }\omega \not\in \L(\M_\omega),
    \end{cases}
  \end{align*}
  където $\texttt{step}_\omega$ е минималният брой стъпки необходими на $\M_\omega$ за да приеме думата $\omega$.
  Заключаваме, че за произволна дума $\omega \in \{0,1\}^\star$ са изпълнени импликациите:
  \begin{align*}
    \omega \in \Ldiag & \implies \L(\M_{f(\omega)}) = L_0 \implies f(\omega) \in \Code(\Ss)\\
    \omega \in \Lcode \setminus \Ldiag & \implies \L(\M_{f(\omega)}) \text{ е краен подезик на }L_0 \implies f(\omega) \not\in \Code(\Ss)\\
    \omega \not\in \Lcode & \implies f(\omega) = \omega \implies f(\omega) \not\in \Code(\Ss).
  \end{align*}
  които можем да обединим в следната еквивалентност:
  \[\omega \in \Ldiag \iff f(\omega) \in \Code(\Ss),\]
  Оттук следва, че $\Code(\Ss)$ не е полуразрешим, защото от \Theorem{diagonal} ние знаем, че $\Ldiag$ не е полуразрешим.
\end{hint}

\begin{corollary}
  Директно от \Lemma{rice-shapiro:finite} следва, че за всяко от следните свойства $\Ss$ на полуразрешимите езици, 
  $\Code(\Ss)$ {\bf не} е полуразрешим език, където:
  \mynote{Защо не можем да използваме \Lemma{rice-shapiro:finite} за да докажем, че свойството празнота не е полуразрешимо, както и свойството регулярност, разрешимост, полуразрешимост?}
  \begin{itemize}
  \item
    $\Ss$ е свойството безкрайност, т.е. езикът
    \[\Code(\Ss) = \{\omega \mid \text{$\omega$ е код на машина на Тюринг и }|\L(\M_\omega)| = \infty\}\]
    не е полуразрешим;
  \item
    $\Ss$ е свойството за пълнота, т.е. езикът
    \[\Code(\Ss) = \{\omega \mid \text{$\omega$ е код на машина на Тюринг и } \L(\M_\omega) = \Sigma^\star\}\]
    не е полуразрешим;
  \item
    $\Ss$ е свойството неразрешимост, т.е. езикът
    \[\Code(\Ss) = \{\omega \mid \text{$\omega$ е код на машина на Тюринг и $\L(\M_\omega)$ не е разрешим}\}\]
    не е полуразрешим;
  \item
    $\Ss$ е свойството неполуразрешимост, т.е. езикът
    \[\Code(\Ss) = \{\omega \mid \text{$\omega$ е код на машина на Тюринг и $\L(\M_\omega)$ не е полуразрешим}\}\]
    не е полуразрешим;
  \item
    $\Ss$ е свойството нерегулярност, т.е. езикът
    \[\Code(\Ss) = \{\omega \mid \text{$\omega$ е код на машина на Тюринг и $\L(\M_\omega)$ не е регулярен}\}\]
    не е полуразрешим.
  \end{itemize}
\end{corollary}

\begin{important}
  \begin{lemma}
    \label{lem:rice-shapiro:extension}
    Нека $L_1$ е език в $\Ss$ и нека $L_2$ е полуразрешим език, като $L_1 \subsetneqq L_2$ и $L_2 \not\in\Ss$. Тогава $\Code(\Ss)$ не е полуразрешим език.
  \end{lemma}  
\end{important}
\mynote{Това означава, че за полуразрешим език $\Code(\Ss)$, ако $L_1 \in \Ss$ и $L_1 \subseteq L_2$, като $L_2$ е полуразрешим, то $L_2 \in \Ss$.}
\begin{hint}
  Нека $L_1 = \L(\M_1)$ и $L_2 = \L(\M_2)$. Ще докажем, че
  \[\Ldiag \leq_m \Code(\Ss).\]
  Ще дефинираме тотална изчислима функция $f$, която при вход думата $\omega \in \{0,1\}^\star$ работи по следния начин:
  \begin{itemize}
  \item
    Ако $\omega$ не е код на машина на Тюринг, то $f(\omega) \df \omega$.
  \item
    Ако $\omega$ е код на машината на Тюринг $\M_\omega$, тогава $f(\omega) \df \code{\M'}$, където $\M'$, при вход произволна дума $\alpha$, работи така:
    \begin{enumerate}[(1)]
    \item
      $\M'$ симулира едновременно две изчисления - работата на $\M_1$ върху $\alpha$ и работата на $\M_\omega$ върху $\omega$, докато намери стъпка $s$, такава че:    
    \item 
      ако симулацията на $\M_1$ завършва за $s$ на брой стъпки като приема думата $\alpha$, то $\M'$ завършва като приема думата $\alpha$;
    \item
      ако симулацията на $\M_\omega$ завършва за $s$ на брой стъпки като приема думата $\omega$, 
      то $\M'$ продължава като симулира работата $\M_2$ върху $\alpha$.
    \item
      Ако симулацията на $\M_2$ завърши като приема думата $\alpha$, то $\M'$ завършва като приема думата $\alpha$.
    \end{enumerate}
  \end{itemize}
  Съобразете сами, че получаваме следното:
  \begin{align*}
    \L(\M') = 
    \begin{cases}
      L_2, & \text{ако } \omega \in \L(\M_\omega)\\
      L_1, & \text{ако } \omega \not\in \L(\M_\omega).
    \end{cases}
  \end{align*}
  Понеже $L_2 \not\in \Ss$, а $L_1 \in \Ss$, можем да заключим, че за произволна дума $\omega \in \{0,1\}^\star$ са изпълнени импликациите:
  \begin{align*}
    \omega \in \Ldiag & \implies \L(\M_{f(\omega)}) = L_1 \implies f(\omega) \in \Code(\Ss)\\
    \omega \in \Lcode\setminus\Ldiag & \implies \L(\M_{f(\omega)}) = L_2 \implies f(\omega) \not\in \Code(\Ss)\\
    \omega \not\in \Lcode & \implies f(\omega) = \omega \implies f(\omega)  \not\in \Code(\Ss),
  \end{align*}
  които можем да обединим в следната еквивалентност:
  \[\omega \in \Ldiag \iff f(\omega) \in \Code(\Ss),\]
  Това означава, че ефективно можем да сведем въпрос за принадлежност в $\Ldiag$
  към въпрос за принадлежност в $\Code(\Ss)$.
  Следователно, ако $\Code(\Ss)$ е полуразрешим език, то $\Ldiag$ е полуразрешим език, което е противоречие.  
\end{hint}

\begin{corollary}
  Директно от \Lemma{rice-shapiro:extension} следва, че за всяко от следните свойства $\Ss$ на полуразрешимите езици, 
  $\Code(\Ss)$ {\bf не} е полуразрешим език, където:
  \begin{itemize}
  \item
    $\Ss$ е свойството крайност, т.е. езикът
    \[\Code(\Ss) = \{\omega \mid \text{$\omega$ е код на машина на Тюринг и }|\L(\M_\omega)| < \infty\}\]
    не е полуразрешим;
  \item
    $\Ss$ е свойството празнота, т.е. езикът
    \[\Code(\Ss) = \{\omega \mid \text{$\omega$ е код на машина на Тюринг и }\L(\M_\omega) = \emptyset\}\]
    не е полуразрешим;
  \item
    $\Ss$ е свойството разрешимост, т.е. езикът
    \[\Code(\Ss) = \{\omega \mid \text{$\omega$ е код на машина на Тюринг и }\L(\M_\omega) \text{ е разрешим} \}\]
    не е полуразрешим;
  \item
    $\Ss$ е свойството безконтекстност, т.е. езикът
    \[\Code(\Ss) = \{\omega \mid \text{$\omega$ е код на машина на Тюринг и }\L(\M_\omega) \text{ е безконтекстен} \}\]
    не е полуразрешим;
  \item
    $\Ss$ е свойството регулярност, т.е. езикът
    \[\Code(\Ss) = \{\omega \mid \text{$\omega$ е код на машина на Тюринг и }\L(\M_\omega) \text{ е регулярен} \}\]
    не е полуразрешим.
  \end{itemize}
\end{corollary}


\begin{framed}
  \begin{theorem}[Райс-Шапиро]
    \label{th:rice-shapiro}
    \index{Райс}
    \index{Шапиро}
    Нека $\Code(\Ss)$ е полуразрешим език. Тогава е изпълнена еквивалентността:
    \[L \in \Ss \iff (\exists L_0 \subseteq \Sigma^\star )[L_0\text{ е краен и }L_0 \subseteq L\ \&\ L_0 \in \Ss].\]
  \end{theorem}
\end{framed}
\begin{proof}
  \Lemma{rice-shapiro:finite} може да се формулира така:
  ако $\Code(\Ss)$ е полуразрешим, то всеки език $L \in \Ss$ има краен подезик $L_0 \subseteq L$, за който $L_0 \in \Ss$. Това ни дава посоката $(\Rightarrow)$ на \Theorem{rice-shapiro}.

  \Lemma{rice-shapiro:extension} може да се формулира така:
  ако $\Code(\Ss)$ е полуразрешим, то за всеки два езика $L_1 \subseteq L_2$, ако $L_1 \in \Ss$,
  то $L_2 \in \Ss$. Това ни дава посоката $(\Leftarrow)$ на \Theorem{rice-shapiro}.
\end{proof}


% % \section{Проблеми за безконтекстни езици}

% % \begin{lemma}
% %   Нека е дадена $\M = \TM$.
% %   Тогава езикът 
% %   \[L = \{\alpha\sharp\beta^R \mid \alpha,\beta \in \Gamma^\star Q \Gamma^\star\ \&\  \alpha \vdash_\M \beta\}\]
% %   е безконтекстен.
% % \end{lemma}
% % \begin{proof}
% %   Ще покажем, че съществува стеков автомат $P$, за който $\L_S(P) = L$.
% %   Четем буквата $X$. Тогава:
% %   \begin{itemize}
% %   \item 
% %     ако $\delta_\M(q,X) =(p,Y,R)$, то слагаме $Yp$ на върха на стека;
% %   \item
% %     ако $\delta_\M(q,X) =(p,Y,L)$, то ако $Z$ е върха на стека, заменяме $Z$ с $pZY$;
% %   \end{itemize}
% % \end{proof}



% % \begin{thm}
% %   Неразрешим е проблемът за проверка дали при дадени две произволни безконтекстни граматики $G_1$ и $G_2$,
% %   $\L(G_1) \cap \L(G_2) = \emptyset$.  
% % \end{thm}

% % \begin{thm}
% %   Неразрешим е проблемът за проверка дали при дадена произволна безконтекстна граматика $G$,
% %   $\L(G) = \Sigma^\star$.  
% % \end{thm}


% % \section{Въпроси}

% % Вярно ли е, че следният проблем е {\em разрешим}:
% % \begin{itemize}
% % \item
% %   за произволна безконтекстна граматика $G$, проверява дали $\L(G) = \emptyset$?
% % \item
% %   за произволна безконтекстна граматика $G$, проверява дали $\L(G) = \Sigma^\star$?
% % \item
% %   за произволни безконтекстни граматики $G_1$ и $G_2$, проверява дали $\L(G_1) \cap \L(G_2) = \emptyset$?
% % \item
% %   за произволни безконтекстни граматики $G_1$ и $G_2$, проверява дали $\L(G_1) \cap \L(G_2) = \Sigma^\star$?
% % \item
% %   за произволни безконтекстни граматики $G_1$ и $G_2$, проверява дали $\L(G_1) = \L(G_2)$?
% % \item
% %   за произволни безконтекстни граматики $G_1$ и $G_2$, проверява дали $\L(G_1) \subseteq \L(G_2)$?
% % \item
% %   за произволна безконтекстна граматика $G$ и произволен регулярен израз $r$,
% %   проверява дали $\L(G) = \L(r)$?
% % \item
% %   за произволна безконтекстна граматика $G$ и произволен регулярен израз $r$,
% %   проверява дали $\L(G) \subseteq \L(r)$?
% % \item
% %   за произволна безконтекстна граматика $G$ и произволен регулярен израз $r$,
% %   проверява дали $\L(r) \subseteq \L(G)$?
% % \item
% %   за произволни безконтекстни граматики $G_1$ и $G_2$, проверява дали $\L(G_1) \subseteq \L(G_2)$ 
% %   е безконтекстен език ?
% % \item
% %   за произволна безконтекстна граматика $G$, проверява дали $\Sigma^\star \setminus \L(G)$
% %   е безконтекстен език ?
% % \item
% %   за произволна безконтекстна граматика $G$, проверява дали $\L(G)$ е регулярен език?
% % \end{itemize}


%%% Local Variables:
%%% mode: latex
%%% TeX-master: "../eai"
%%% End:
